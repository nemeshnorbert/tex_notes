% chktex-file 19
% chktex-file 35
\documentclass[12pt]{article}
\usepackage[utf8]{inputenc}
\usepackage[T1,T2A]{fontenc}
\usepackage[left=2cm,right=2cm,top=2cm,bottom=2cm,bindingoffset=0cm]{geometry}
\usepackage{mathrsfs}
\usepackage{amssymb}
\usepackage{amsmath}
\usepackage{amsthm}
\usepackage{enumerate}
\usepackage{longtable}
\usepackage[russian]{babel}
\usepackage[matrix,arrow,curve]{xy}
\usepackage[colorlinks=true, urlcolor=blue, linkcolor=blue, citecolor=blue,
    pdfborder={0 0 0}]{hyperref}
\usepackage{enumitem}

%-------------------------------------------------------------------------------
\newtheorem{theorem}{Теорема}[section]
\newtheorem{lemma}[theorem]{Лемма}
\newtheorem{proposition}[theorem]{Предложение}
\newtheorem{remark}[theorem]{Замечание}
\newtheorem{corollary}[theorem]{Следствие}
\newtheorem{definition}[theorem]{Определение}
\newtheorem{example}[theorem]{Пример}

\newcommand{\projtens}{\mathbin{\widehat{\otimes}}}
\newcommand{\convol}{\ast}
\newcommand{\projmodtens}[1]{\mathbin{\widehat{\otimes}}_{#1}}
\newcommand{\isom}{\mathop{\mathbin{\cong}}}
%-------------------------------------------------------------------------------


\begin{document}

\begin{center}
    \Large \textbf{Относительная проективность модулей $L_p$}\\[0.5cm]
    \small {Н. Т. Немеш}
\end{center}

\thispagestyle{empty}

\medskip
\textbf{Аннотация:} В работе даны критерии относительной проективности
$L_p$-пространств рассмотренных как левые банаховы модули над алгеброй
ограниченных измеримых функций ($1\leq p\leq+\infty$) и алгеброй непрерывных
исчезающих в бесконечности функций ($1\leq p <+\infty$). Главный результат
статьи: для локально компактного хаусдорфова пространства $S$ и локально
конечной внутренне компактно регулярной борелевской меры $\mu$ относительная
проективность $C_0(S)$-модуля $L_\infty(S,\mu)$ влечет внутренне открытую
регулярность меры и псевдокомпактность ее носителя.

\medskip
\textbf{Ключевые слова:} проективный модуль, $L_p$-пространство, нормальная
мера, псевдокомпактное пространство.

\bigskip

%-------------------------------------------------------------------------------
% Introduction
%-------------------------------------------------------------------------------

\section{Введение}\label{SectionIntroduction}

Основной вопрос банаховой гомологии звучит следующим образом: какова
гомологическая размерность данной банаховой алгебры $A$? Для этого необходимо
уметь отвечать на другой вопрос: является ли данный банахов $A$-модуль
проективным? Для многих модулей функционального анализа ответы известны. При
этом все еще есть примеры классических модулей анализа, для которых данный
вопрос не решен, например $L_p$-пространства. Мы будем рассматривать
пространства Лебега как левые банаховы модули над алгеброй исчезающих на
бесконечности непрерывных функций, определенных на локально компактном
хаусдорфовом пространстве, и как модули над алгеброй ограниченных измеримых
функций. Для этих пространств мы дадим необходимые и достаточные условия их
относительной проективности. Особый интерес представляет критерий относительной
проективности модуля $L_\infty$. Дело в том, что один из основных результатов
банаховой гомологии --- теорема о глобальной размерности
[\cite{HelHomolBanTopAlg}, предложение V.2.21] основывается на том факте, что
гомологическая размерность модуля ограниченных последовательностей на алгеброй
сходящихся последовательностей равна 2. В частности этот модуль не проективен.
Как мы увидим, это поведение типично для большинства модулей $L_\infty$. Перед
тем, как перейти к основной части статьи, нам понадобится несколько определений.

Пусть $M$ --- подмножество множества $N$, тогда $\chi_M$ будет обозначать
индикаторную функцию множества $M$. Для произвольной функции $f:N\to L$ через
$f|_M$ мы будем обозначать ее ограничение на $M$. Символ $1_M$ будет обозначать
тождественное отображение на $M$.

Пусть $S$ --- произвольное топологическое пространство и $M$ --- его
подмножество. Тогда через $\operatorname{cl}_S(M)$ и $\operatorname{int}_S(M)$
мы будем обозначать замыкание и внутренность $M$ в $S$.

Все банаховы пространства в этой статье рассматриваются над полем комплексных
чисел. Для заданных банаховых пространств $X$ и $Y$ через $X\oplus_1 Y$ мы будем
обозначать их $\ell_1$-сумму, а через $X\projtens Y$ их проективное тензорное
произведение. Будем говорить, что банахово пространство $X$ дополняемо в $Y$,
если $X$ --- подпространство в $Y$, и существует ограниченный линейный оператор
$P:Y\to Y$ такой что $P|_X=1_X$ и $\operatorname{Im}(P)=X$. Для $1\leq
    p\leq+\infty$ и заданного пространства с мерой $(X,\mu)$ через $L_p(X,\mu)$ мы
будем обозначать банахово пространство классов эквивалентности $p$-интегрируемых
(или существенно ограниченных, если $p=+\infty$) функций на $X$. Элементы
$L_p(X,\mu)$ будут обозначаться через $[f]$. Отметим, что все $L_p$-пространства
обладают свойством аппроксимации.

Для заданной банаховой алгебры $A$ через $A_+:=A\oplus_1 \mathbb{C}$ мы будем
обозначать ее стандартную унитизацию. Мы будем рассматривать только левые
банаховы модули с сжимающим билинейным оператором внешнего умножения
$\cdot:A\times X\to X$. Если $A$ --- банахова алгебра с единицей $e$, то банахов
$A$-модуль $X$ называется унитальным, если $e\cdot x=x$ для всех $x\in X$. Для
заданного банахова $A$-модуля $X$ его существенной частью $X_{ess}$ называется
замкнутая линейная оболочка множества $A\cdot X$. Мы будем называть модуль $X$
существенным, если $X=X_{ess}$. Очевидно, любой унитальный банахов модуль
существенный. Пусть $X$ и $Y$ --- два банаховых $A$-модуля, тогда отображение
$\phi:X\to Y$ будем называть $A$-морфизмом, если оно является непрерывным
морфизмом $A$-модулей. Банаховы $A$-модули и $A$-морфизмы образуют категорию,
которую мы будем обозначать через $A-\mathbf{mod}$.

Категория $A-\mathbf{mod}$ имеет свое понятие проективности. Произвольный
$A$-морфизм $\xi:X\to Y$ будем называть допустимым если существует правый
обратный ограниченный линейный оператор $\eta:Y\to X$, т.е., если $\xi\eta=1_Y$.
Банахов $A$-модуль $P$ будем называть относительно проективным, если для любого
допустимого $A$-морфизма $\xi:X\to Y$ и любого $A$-морфизма $\phi:P\to Y$
существует $A$-морфизм $\psi:P\to X$, делающий диаграмму
$$
    \xymatrix{
    & {X} \ar[d]^{\xi}\\  % chktex 3
    {P} \ar@{-->}[ur]^{\psi} \ar[r]^{\phi} &{Y}}  % chktex 3
$$
коммутативной. Вместо проверки по определению можно показать, что банахов
$A$-модуль $P$ относительно проективен, предъявив $A$-морфизм $\sigma:P\to
    A_+\projtens P$, являющийся правым обратным каноническому $A$-морфизму
$\pi_P^+:A_+\projtens P\to P:(a\oplus_1 z)\projtens x\mapsto a\cdot x+z x$
[\cite{HelHomolBanTopAlg}, предложение IV.1.1]. Если банахов $A$-модуль $P$
существенный, то он проективен тогда и только тогда, когда $A$-морфизм
$\pi_P:A\projtens P\to P: a\projtens x\mapsto a\cdot x$ обладает правым обратным
$A$-морфизмом [\cite{HelHomolBanTopAlg}, предложение IV.1.2].

%-------------------------------------------------------------------------------
%	Necessary conditions
%-------------------------------------------------------------------------------

\section{Необходимые условия относительной
  проективности}\label{NecessaryConditions}

В этом параграфе мы покажем, что для относительно проективного $A$-модуля $X$
его существенная часть дополняема и $A$-значные $A$-морфизмы разделяют точки
существенной части. Эти необходимые условия будут играть ключевую роль в статье.

\begin{proposition}\label{MorphDecomp} Пусть $X$ --- банахов $A$-модуль и $E$
    --- банахово пространство. Пусть $j_E:A_+\projtens E\to (A\projtens
        E)\oplus_1 E$ обозначает естественный изоморфизм. Тогда для любого
    $A$-морфизма $\sigma:X\to A_+\projtens E$ существуют ограниченные линейные
    операторы $\sigma_1:X\to A\projtens E$, $\sigma_2:X\to E$ такие, что
    \begin{enumerate}[label = (\roman*)]
        \item $j_E(\sigma(x))=\sigma_1(x)\oplus_1 \sigma_2(x)$ для всех $x\in
                  X$;

        \item $\sigma_1(a\cdot x)=a\cdot \sigma_1(x)+a\projtens \sigma_2(x)$ для
              всех $x\in X$ и $a\in A$;

        \item $\sigma_2(a\cdot x)=0$ для всех $x\in X$ и $a\in A$.

    \end{enumerate}
    \noindent
    Как следствие, $\sigma_1|_{X_{ess}}$ --- $A$-морфизм и
    $\sigma_2|_{X_{ess}}=0$.

\end{proposition}
\begin{proof} Рассмотрим ограниченные линейные операторы $q_1:A_+\projtens X\to
        A\projtens X: (a\oplus_1 z)\projtens x\mapsto a\projtens x$ и
    $q_2:A_+\projtens X\to X: (a\oplus_1 z)\projtens x\mapsto z x$.
    Определим отображения $\sigma_1=q_1\sigma$, $\sigma_2=q_2\sigma$.
    Очевидно, что $j_E=q_1\oplus_1 q_2$, поэтому
    $j_E(\sigma(x))=\sigma_1(x)\oplus_1 \sigma_2(x)$ для всех $x\in X$.
    Заметим, что $a\cdot u=a\cdot q_1(u)+a\projtens q_2(u)$ для всех $a\in
        A$ и $u\in A_+\projtens X$. Поскольку $\sigma$ --- $A$-морфизм, легко
    проверить, что $\sigma_1(a\cdot x)=a\cdot \sigma_1(x)+a\projtens
        \sigma_2(x)$ и $\sigma_2(a\cdot x)=0$ для всех $a\in A$, $x\in X$.
\end{proof}

\begin{proposition}\label{ProjModEssPartCompl} Пусть $X$ --- относительно
    проективный банахов $A$-модуль. Тогда $X_{ess}$ дополняемо в $X$ как
    банахово пространство.
\end{proposition}
\begin{proof} Поскольку $X$ относительно проективен, существует $A$-морфизм
    $\sigma:X\to A_+\projtens X$ такой, что $\pi_X^+\sigma=1_X$. Пусть
    $\sigma_1$ и $\sigma_2$ --- ограниченные линейные операторы из
    предложения~\ref{MorphDecomp}. Теперь рассмотрим $A$-морфизм
    $\pi_X:A\projtens X\to X:a\projtens x\mapsto a\cdot x$, тогда для всех $x\in
        X$ выполнено $x=\pi_X^+(\sigma(x))=\pi_X(\sigma_1(x)) + \sigma_2(x)$.
    Рассмотрим ограниченный линейный оператор $\eta = \pi_X\sigma_1$. Так как
    $\sigma_2|_{X_{ess}}=0$, то $\eta|_{X_{ess}}=1_X$. Более того,
    $\operatorname{Im}(\eta)\subset\operatorname{Im}(\pi_X)=X_{ess}$,
    следовательно $\eta$ --- ограниченный линейный проектор $X$ на $X_{ess}$.
\end{proof}

Следующее предложение --- простое обобщение [\cite{SelivBiprojBanAlg}, лемма
1.4].

\begin{proposition}\label{RelProjNecesCond} Пусть $A$ --- банахова алгебра и $X$
    --- относительно проективный банахов $A$-модуль. Допустим, что $A$ или $X$
    обладает свойством аппроксимации. Тогда
    \begin{enumerate}[label = (\roman*)]
        \item для любого ненулевого $x\in X$ существует $A$-морфизм $\phi:X\to
                  A_+$ такой, что $\phi(x)\neq 0$;

        \item для любого ненулевого $x\in X_{ess}$ существует $A$-морфизм
              $\psi:X_{ess}\to A$ такой, что $\psi(x)\neq 0$.
    \end{enumerate}
\end{proposition}
\begin{proof} Пусть $i_E:E\projtens \mathbb{C}\to E$ --- естественный
    изоморфизм.

    $(i)$ Зафиксируем ненулевой $x\in X$. Так как $X$ относительно проективен,
    то существует $A$-морфизм $\sigma:X\to A_+\projtens X$ такой что
    $\pi_X^+\sigma=1_X$. Рассмотрим $u:=\sigma(x)\in A_+\projtens X$. Поскольку
    $\pi_X^+(u)=x\neq 0$, то $u\neq 0$. Так как $A$ или $X$ обладает свойством
    аппроксимации, то существует $f\in A_+^*$ и $g\in X^*$ такие, что
    $(f\projtens g)(u)\neq 0$ [\cite{GrothProdTenTopNucl}, следствие I.5.1, стр.
            168]. Рассмотрим $a:=((1_{A_+}\projtens g)(u))\in A_+\projtens\mathbb{C}$ и
    $F:=(f\projtens 1_{\mathbb{C}})\in A_+^*\projtens\mathbb{C}$. Так как
    $F(a)=(f\projtens g)(u)\neq 0$, то $a\neq 0$. Теперь легко проверить, что
    линейный оператор $\xi:=(1_{A_+}\projtens g)\sigma$ является $A$-морфизмом.
    Очевидно, что $\xi(x)=a\neq 0$. Остается положить $\phi=i_{A_+}\xi$.

    $(ii)$ Зафиксируем ненулевой $x\in X_{ess}$. Пусть $\xi$ --- морфизм
    построенный в пункте $(i)$. Рассмотрим морфизмы $\xi_1$ и $\xi_2$ из
    предложения~\ref{MorphDecomp}. Так как
    $\xi(x)=j_{\mathbb{C}}(\xi_1(x)\oplus_1\xi_2(x))\neq 0$ и $\xi_2(x)=0$, то
    $\xi_1(x)\neq 0$. Из того же предложения известно, что $\xi_1|_{X_{ess}}$
    --- $A$-морфизм, поэтому остается положить $\psi=i_A \xi_1|_{X_{ess}}$.

\end{proof}

-------------------------------------------------------------------------------
Preliminaries on general measure theory
-------------------------------------------------------------------------------

\section{Предварительные сведения по общей теории
  меры}\label{SectionPreliminariesOnGeneralMeasureTheory}

Всестороннее изучение общих пространств с мерой можно найти
в~\cite{FremMeasTh2}. Мы будем использовать определения из этой монографии.

Пусть $X$ --- произвольное множество. Под мерой мы будем понимать счетно
аддитивную функцию множеств со значениями в $[0,+\infty]$, определенную на
$\sigma$-алгебре $\Sigma$ измеримых подмножеств множества $X$. Если $F$ ---
измеримое множество, то корректно определены меры
$\mu^F:\Sigma\to[0,+\infty]:E\mapsto \mu(E\cap F)$ и
$\mu_F:\Sigma_F\to[0,+\infty]: E\mapsto \mu(E)$, где
$\Sigma_F=\{E\in\Sigma:E\subset F\}$. Измеримое множество $E$ называется атомом,
если $\mu(A)>0$ и для каждого измеримого подмножества $B\subset A$ верно или
$\mu(B)=0$ или $\mu(A\setminus B)=0$. Мера $\mu$ называется чисто атомической,
если каждое измеримое подмножество положительной меры содержит атом. Мера $\mu$
называется полуконечной, если для любого измеримого множества $A$ бесконечной
меры существует измеримое подмножество в $A$ конечной положительной меры.
Семейство $\mathcal{D}$ измеримых множеств конечной меры называется разложением
$X$, если для любого измеримого множества $E$ верно
$\mu(E)=\sum_{D\in\mathcal{D}}\mu(E\cap D)$ и множество $F$ измеримо если $F\cap
    D$ измеримо для всех $D\in\mathcal{D}$. Наконец, мера $\mu$ называется
разложимой, если она полуконечна и существует разложение $X$. На самом деле
пространство с мерой разложимо тогда и только тогда, когда оно является
дизъюнктным объединением пространств конечной меры [\cite{FremMeasTh2},
упражнение 214X (i)]. Большинство мер встречающихся в функциональном анализе
разложимы.

\begin{definition}\label{AtomCore} Пусть $A$ --- атом в пространстве с мерой
    $(X,\mu)$, тогда измеримое множество $C\subset A$ называется ядром $A$, если
    $C$ --- атом и единственные измеримые подмножества в $C$ --- это
    $\varnothing$ и $C$. Атом $A$ называется твердым, если у него есть ядро.
    Очевидно, если ядро существует, то оно единственно, и в этом случае мы будем
    обозначать ядро через $A^\bullet$.
\end{definition}

\begin{proposition}\label{GenniunelyAtomicMeasCharac} Пусть $(X,\mu)$ ---
    непустое пространство с конечной мерой такое, что единственное множество
    меры $0$ в $X$ --- это пустое множество. Тогда $(X,\mu)$ --- чисто
    атомическое пространство и каждый атом твердый.
\end{proposition}
\begin{proof} Пусть $E$ --- измеримое подмножество положительной меры. Пусть
    $x\in E$, тогда рассмотрим величину $c:=\inf \{\mu(F):x\in F\in \Sigma,\;
        F\subset E\}$. Для любого $n\in\mathbb{N}$ существует $E_n\in\Sigma$ такое,
    что $x\in E_n\subset E$ и $\mu(E_n)<c+2^{-n}$. Определим $A=\bigcap
        \{E_n:n\in\mathbb{N}\}\subset E$, тогда $x\in A\in\Sigma$ и $\mu(A)=c$. По
    построению $A$ непусто, поэтому $\mu(A)>0$. Пусть $B$ --- измеримое
    подмножество в $A$. Если $x\in A\setminus B$, то $c\leq\mu(A\setminus
        B)\leq\mu(A)=c$, т.е. $\mu(B)=0$. Аналогично, если $x\in B$ мы получаем
    $\mu(A\setminus B)=0$. Таким образом, $A\subset E$ --- атом. Поскольку $E$
    произвольно, $(X,\mu)$ чисто атомическое пространство.

    Теперь пусть $A$ --- атом в $(X,\mu)$. Если $B\in\Sigma$ и $B\subset A$, то
    или $\mu(B)$ или $\mu(A\setminus B)=0$. Из предположения на $(X,\mu)$ мы
    получаем, что или $B$ или $A\setminus B$ пусто. Следовательно $A^\bullet=A$.
\end{proof}

%-------------------------------------------------------------------------------
%	Projectivity of $B(\Sigma)$-modules L_p(X,\mu)
%-------------------------------------------------------------------------------

\section{Относительная проективность
  \texorpdfstring{$B(\Sigma)$}{BSigma}-модулей
  \texorpdfstring{$L_p(X,\mu)$}{LpXmu}
 }\label{SectionRelativeProjectivityOfBSigmaModulesLpXmu}

Пусть $(X,\mu)$ --- пространство с мерой. Через $B(\Sigma)$ мы будем обозначать
алгебру измеримых ограниченных функций с $\sup$ нормой. В этом параграфе мы
дадим критерий относительно проективности левых $B(\Sigma)$-модулей
$L_p(X,\mu)$. Говоря неформально, все такие модули выглядят как
$\ell_\infty(\Lambda)$-модули $\ell_p(\Lambda)$ для некоторого индексного
множества $\Lambda$.

\begin{proposition}\label{BSigmaModLpRetrProj} Пусть $(X,\mu)$ --- пространство
    с мерой. Пусть $1\leq p\leq +\infty$ и $L_p(X,\mu)$ --- относительно
    проективный $B(\Sigma)$-модуль. Тогда для любого измеримого множества
    $B\subset X$ банахов $B(\Sigma)$-модуль $L_p(X,\mu^B)$ относительно
    проективен.
\end{proposition}
\begin{proof}
    Для $B(\Sigma)$-морфизмов
    $$
        \pi:L_p(X,\mu)\to L_p(X,\mu^B):[f]\mapsto [f]\chi_B,
    $$
    $$
        \sigma:L_p(X,\mu^B)\to L_p(X,\mu):[f]\mapsto [f].
    $$
    легко проверить,что выполнено $\pi\sigma=1_{L_p(X,\mu^B)}$. Другими словами,
    $L_p(X,\mu^B)$ --- ретракт $L_p(X,\mu)$ в $B(\Sigma)-\mathbf{mod}$. Теперь
    результат следует из [\cite{HelBanLocConvAlg}, предложение VII.1.6].
\end{proof}

\begin{proposition}\label{LpBSigmaModNecessCond} Пусть $(X,\mu)$ --- разложимое
    пространство с мерой и $L_p(X,\mu)$ --- относительно проективный
    $B(\Sigma)$-модуль. Тогда $(X,\mu)$ является дизъюнктным объединением
    твердых атомов конечной меры.
\end{proposition}
\begin{proof} Пусть $\mathcal{D}$ --- разложение $X$ на измеримые подмножества
    конечной меры. Зафиксируем $D\in\mathcal{D}$ и введем обозначение
    $\nu:=\mu^D$. По предложению~\ref{BSigmaModLpRetrProj} банахов
    $B(\Sigma)$-модуль $L_p(X,\nu)$ относительно проективен. Рассмотрим
    произвольное множество $E\in\Sigma$ положительной меры $\nu$. Так как мера
    $\nu$ конечна, то конечно и $\nu(E)$. Тогда $[f]:=[\chi_E]$ --- корректно
    определенный ненулевой элемент в $L_p(X,\nu)$. Так как $B(\Sigma)$ ---
    унитальная алгебра, то модуль $L_p(X,\nu)$ существенный. Теперь из
    предложения~\ref{RelProjNecesCond} мы получаем $B(\Sigma)$-морфизм
    $\psi:L_p(X,\nu)\to B(\Sigma)$ такой, что $\psi([f])\neq 0$. Следовательно,
    множество $F:=a^{-1}(\mathbb{C}\setminus \{0\})\in\Sigma$ непусто. Отметим,
    что $[f]=[f]\chi_E$, поэтому $a=\psi([f]\chi_E)=\psi([f])\chi_E=a\chi_E$.
    Значит, $a|_{X\setminus E}=0$ и $F\subset E$. Рассмотрим произвольное
    измеримое множество $A\subset F$ с нулевой $\nu$ мерой. Тогда $[\chi_A]$ ---
    нулевой элемент в  $L_p(X,\nu)$ и $[\chi_A]=[\chi_E]\chi_A$. Следовательно,
    $a\chi_A=\psi([\chi_E])\chi_A=\psi([\chi_E]\chi_A)=\psi([\chi_A])=0$. Так
    как $A\subset F$ и $a$ не равно нулю ни в одной точке множества $F$, то
    $A=\varnothing$. Поскольку $F\neq \varnothing$, то из
    предложения~\ref{GenniunelyAtomicMeasCharac} мы получаем, что пространство с
    мерой $(F,\nu_F)$ имеет твердый атом. Таким образом, мы показали, что любое
    измеримое множество $E$ положительной $\nu$ меры содержит твердый атом.
    Тогда из стандартного приема с леммой Цорна мы получаем, что $(X,\mu^D)$ ---
    дизъюнктное объединение твердых атомов. Такой же вывод верен и для
    $(X,\mu_D)$. Так как мера $\mu_D$ конечна, то конечен каждый ее атом.
    Поскольку $D$ произвольно, то вывод теоремы следует из [\cite{FremMeasTh2},
    упражнение 214X (i)].
\end{proof}

Пусть $(X,\mu)$ --- пространство с мерой и $A$ --- измеримое подмножество
конечной положительной меры. Тогда корректно определен ограниченный линейный
функционал $m_A:B(\Sigma)\to\mathbb{C}:a\mapsto {\mu(A)}^{-1}\int_A f(x)d\mu(x)$
нормы 1.

\begin{proposition}\label{HardAtomicMeasProp} Пусть $(X,\mu)$ --- дизъюнктное
    объединение семейства $\mathcal{A}$ твердых атомов конечной меры. Тогда
    \begin{enumerate}[label = (\roman*)]
        \item множество $X^\bullet:=\bigcup \{A^\bullet:A\in\mathcal{A}\}$
              измеримо и $\mu(X\setminus X^\bullet)=0$;

        \item для любого атома $A\in\mathcal{A}$ и любых функций $a,b\in
                  B(\Sigma)$ выполнено $a|_{A^\bullet}=m_{A^\bullet}(a)$ и
              $m_{A^\bullet}(ab)=m_{A^\bullet}(a)m_{A^\bullet}(b)$;

        \item для любой функции $a\in B(\Sigma)$ существует функция $b\in
                  B(\Sigma)$ такая, что $b|_{X^\bullet}=0$ и выполнено
              поточечное равенство $a=\sum_{A\in\mathcal{A}}
                  m_{A^\bullet}(a)\chi_{A^\bullet} + b$;

        \item для любой функции $[f]\in L_p(X,\mu)$ верно
              $[f]=[\sum_{A\in\mathcal{A}} m_{A^\bullet}(f)\chi_{A^\bullet}]$.
    \end{enumerate}
\end{proposition}
\begin{proof} $(i)$ Так как $\mathcal{A}$ --- разложение $X$, то $(X,\mu)$ ---
    разложимое пространство с мерой и $X^\bullet$ измеримо. Заметим, что
    дизъюнктные множества $A\setminus A^\bullet$ для $A\in\mathcal{A}$ имеют
    нулевую меру, а значит и их объединение $X\setminus X^\bullet$ имеет нулевую
    меру.

    $(iii)$ Зафиксируем $a\in B(\Sigma)$ и $A\in \mathcal{A}$. Так как
    $A^\bullet$ содержит только два измеримых подмножества, то $a$ --- константа
    на $A^\bullet$. Значит, $a|_{A^\bullet}=m_{A^\bullet}(a)$. Как следствие,
    для измеримой функции
    $b=a-\sum_{A\in\mathcal{A}}m_{A^\bullet}(a)\chi_{A^\bullet}$ мы имеем
    $b|_{X^\bullet}=0$.

    $(iv)$ Результат немедленно следует из пункта $(iii)$.
\end{proof}

\begin{proposition}\label{LpBSigmaModSuffCond} Пусть $1\leq p\leq +\infty$ и
    $(X,\mu)$ --- дизъюнктное объединение семейства твердых атомов конечной
    меры. Тогда $B(\Sigma)$-модуль $L_p(X,\mu)$ относительно проективен.
\end{proposition}
\begin{proof} Пусть $\mathcal{A}$ обозначает множество твердых атомов в $X$.

    Рассмотрим случай $p=+\infty$. Определим ограниченный линейный оператор
    $$
        \rho:L_\infty(X,\mu)\to B(\Sigma):
        [f]\mapsto\sum_{A\in\mathcal{A}}m_{A^\bullet}(f)\chi_{A^\bullet}.
    $$
    Из пункта $(ii)$ предложения~\ref{HardAtomicMeasProp} следует, что $\rho$
    является $B(\Sigma)$-морфизмом. Следовательно, $\sigma=\rho\projtens
        1_{L_\infty(X,\mu)}$ тоже является $B(\Sigma)$-морфизмом. Из пункта $(iv)$
    предложения~\ref{HardAtomicMeasProp} мы получаем, что
    $\pi_{L_\infty(X,\mu)}\sigma=1_{L_\infty(X,\mu)}$. Так как
    $B(\Sigma)$-модуль $L_\infty(X,\mu)$ унитальный, то из
    [\cite{HelHomolBanTopAlg}, предложение IV.1.2] следует его относительная
    проективность.

    Рассмотрим случай $1\leq p<+\infty$. Пусть $[f]\in L_p(X,\mu)$, тогда из
    пункта $(iv)$ предложения~\ref{HardAtomicMeasProp} мы имеем
    $[f]=[\sum_{A\in\mathcal{A}}m_{A^\bullet}(f)\chi_{A^\bullet}]$. Более того,
    поскольку $p<+\infty$ выполнено
    $[f]=\sum_{A\in\mathcal{A}}m_{A^\bullet}(f)[\chi_{A^\bullet}]$ в
    $L_p(X,\mu)$. Заметим, что последняя сумма содержит не более чем счетное
    количество ненулевых слагаемых. Через $\mathcal{A}_f$ мы обозначим индексы,
    для которых эти слагаемые ненулевые. Рассмотрим произвольное конечное
    подмножество $\mathcal{F}=\{A_1,\ldots,A_n\}\subset\mathcal{A}_f$, и введем
    обозначения $x_k=\chi_{A_k^\bullet}$,
    $y_k=m_{A_k^\bullet}(f)[\chi_{A_k^\bullet}]$ для $k\in \{1,\ldots,n\}$.
    Пусть $\omega\in\mathbb{C}$ --- любой корень $n$-ой степени из $1$. Так как
    $\mathcal{F}$ --- дизъюнктное семейство, то $\Vert\sum_{k=1}^n \omega^k
        x_k\Vert_{B(\Sigma)}\leq 1$ и $\Vert \sum_{k=1}^n\omega^k
        y_k\Vert_{L_p(X,\mu)}\leq\Vert [f]\Vert_{L_p(X,\mu)}$. Следовательно, из
    [\cite{HelHomolBanTopAlg}, предложение II.2.44] получаем, что для любой
    функции $[f]\in L_p(X,\mu)$ корректно определен элемент
    $\sigma_f=\sum_{A\in\mathcal{A}_f} x_k\projtens y_k=\sum_{A\in\mathcal{A}}
        x_k\projtens y_k\in B(\Sigma)\projtens L_p(X,\mu)$ нормы не более $\Vert
        f\Vert_{L_p(X,\mu)}$. Используя пункт $(ii)$
    предложения~\ref{HardAtomicMeasProp}, легко проверить, что отображение
    $$
        \sigma: L_p(X,\mu)\to B(\Sigma)\projtens L_p(X,\mu):
        [f]\mapsto \sum_{A\in\mathcal{A}}
        m_{A^\bullet}(f)\chi_{A^\bullet}\projtens[\chi_{A^\bullet}]
    $$
    является корректно определенным $B(\Sigma)$-морфизмом нормы не более $1$. Из
    пункта $(iv)$ предложения~\ref{HardAtomicMeasProp} мы получаем, что
    $\pi_{L_p(X,\mu)}\sigma=1_{L_p(X,\mu)}$. Поскольку $B(\Sigma)$-модуль
    $L_p(X,\mu)$ унитальный, то из [\cite{HelHomolBanTopAlg}, предложение
    IV.1.2] следует его относительная проективность.
\end{proof}

\begin{theorem}\label{LpBSigmaModCrit} Пусть $(X,\mu)$ --- разложимое
    пространство с мерой и $1\leq p\leq +\infty$. Тогда следующие условия
    эквивалентны:
    \begin{enumerate}[label = (\roman*)]
        \item $L_p(X,\mu)$ --- относительно проективный $B(\Sigma)$-модуль;

        \item $(X,\mu)$ --- дизъюнктное объединение твердых атомов конечной
              меры.
    \end{enumerate}
\end{theorem}
\begin{proof} Результат следует из предложений~\ref{LpBSigmaModNecessCond}
    и~\ref{LpBSigmaModSuffCond}.
\end{proof}

%-------------------------------------------------------------------------------
%   Preliminaries on topological measure theory
%-------------------------------------------------------------------------------

\section{Предварительные сведения по топологической теории
  меры}\label{SectionPreliminariesOnTopologicalMeasureTheory}

Подробное обсуждение мер на топологических пространствах можно найти
в~\cite{FremMeasTh4.1}. С этого момента мы рассматриваем меры $\mu$,
определенные на $\sigma$-алгебре $Bor(S)$ борелевских множеств топологического
пространства $S$. Через $\operatorname{supp}(\mu)$ мы будем обозначать носитель
$\mu$. Мера $\mu$ называется
\begin{enumerate}[label = (\roman*)]
    \item строго положительной, если $\operatorname{supp}(\mu)=S$;

    \item с полным носителем, если $\mu(S\setminus\operatorname{supp}(\mu))=0$;

    \item локально конечной, если каждая точка в $S$ имеет окрестность конечной
          меры;

    \item внутренне компактно регулярной, если $\mu(E)=\sup \{\mu(K): K\subset
              E, K\mbox{ компактно}\}$ для всех $E\in Bor(S)$;

    \item внешне открыто регулярной, если $\mu(E)=\inf \{\mu(U): E\subset U,
              U\mbox{ открыто}\}$ для всех $E\in Bor(S)$;

    \item внутренне открыто регулярной, если $\mu(E)=\sup \{\mu(U): U\subset E,
              U\mbox{ открыто}\}$ для всех $E\in Bor(S)$;

    \item остаточной, если $\mu(E)=0$ для всех борелевских нигде не плотных
          множеств $E$;

    \item нормальной, если она остаточная и с полным носителем.
\end{enumerate}

Если мера $\mu$ локально конечна, то все компактные множества имеют конечную
меру [\cite{FremMeasTh4.1}, предложение 411G (a)]. Любая конечная внутренне
компактно регулярная мера внешне открыто регулярна [\cite{FremMeasTh4.1},
предложение 411X (a)]. Очевидно, что мера $\mu^B$ внутренне компактно регулярна
для любого $B\in Bor(S)$, когда мера $\mu$ внутренне компактно регулярна.

\begin{proposition}\label{InnerOpenRegMeasCharac} Пусть $S$ --- локально
    компактное хаусдорфово пространство и $\mu$ --- борелевская мера на $S$.
    Тогда

    \begin{enumerate}[label = (\roman*)]
        \item мера $\mu$ внутренне открыто регулярна тогда и только тогда, когда
              $\mu(E)=\mu(\operatorname{int}_S(E))$ для всех $E\in Bor(S)$;

        \item если мера $\mu$ конечна и внутренне открыто регулярна, то $\mu$
              --- остаточная мера и
              $\mu(E)=\mu(\operatorname{int}_S(E))=\mu(\operatorname{cl}_S(E))$
              для всех $E\in Bor(S)$;

        \item если мера $\mu$ конечна, внутренне компактно регулярна и внутренне
              открыто регулярна, то $\mu$ --- нормальная мера.
    \end{enumerate}
\end{proposition}
\begin{proof} $(i)$ Достаточно заметить, что супремум в определении внутренне
    открыто регулярной меры достигается на максимальном открытом подмножестве
    $E$, то есть на $\operatorname{int}_S(E)$.

    $(ii)$ Первое равенство было доказано в предыдущем пункте. Так как $\mu$ ---
    конечная мера, то для всех $E\in Bor(S)$ выполнено
    $\mu(E)=\mu(S)-\mu(\operatorname{int}_S(S\setminus E))=\mu(S)-\mu(S\setminus
        \operatorname{cl}_S(E))=\mu(\operatorname{cl}_S(E))$. Теперь рассмотрим
    нигде не плотное борелевское множество $E\subset S$, тогда
    $\mu(E)=\mu(\operatorname{cl}_S(E))
        =\mu(\operatorname{cl}_S(\operatorname{int}_S(E)))=\mu(\varnothing)=0$. Так
    как $E$ произвольно, то $\mu$ --- остаточная мера.

    $(iii)$ Всякая внутренне компактно регулярная мера имеет полный носитель
    [\cite{FremMeasTh4.1}, предложение 411C, 411N (d)]. Все остальное следует из
    пунктов $(i)$ и $(ii)$.
\end{proof}

\begin{proposition}\label{NormalMeasCharac} Пусть $S$ --- локально компактное
    хаусдорфово пространство и $\mu$ --- борелевская мера на $S$. Допустим, что
    любое компактное множество $K\subset S$ положительной меры содержит открытое
    множество $U\subset K$ положительной меры. Тогда
    \begin{enumerate}[label = (\roman*)]
        \item $\mu(K)=\mu(\operatorname{int}_S(K))$ для любого компактного
              множества $K\subset S$;

        \item если мера $\mu$ внутренне компактно регулярная, то $\mu$ внутренне
              открыто регулярна.
    \end{enumerate}

\end{proposition}
\begin{proof} Обозначим $K'=K\setminus \operatorname{int}_S(K)$. Это замкнутое
    подмножество компакта $K$, следовательно, $K'$ --- компакт. Допустим, что
    $\mu(K')>0$, тогда существует открытое множество $U\subset K'$ положительной
    меры. Как следствие, $U\subset K$ --- непустое открытое множество не
    пересекающееся с $\operatorname{int}_S(K)$. Противоречие, значит $\mu(K')=0$
    и $\mu(K)=\mu(\operatorname{int}_S(K))$.

    $(ii)$ Зафиксируем $c<\mu(B)$. Так как мера $\mu$ внутренне компактно
    регулярна, то существует компактное множество $K\subset B$ такое, что
    $c<\mu(K)$. Из предыдущего пункта мы получаем
    $c<\mu(K)=\mu(\operatorname{int}_S(K))\leq\mu(\operatorname{int}_S(B))$. Так
    как $c<\mu(B)$ произвольно, то мы заключаем
    $\mu(B)\leq\mu(\operatorname{int}_S(B))$. Обратное неравенство очевидно.
\end{proof}

\begin{proposition}\label{MeasAtomCharac} Пусть $S$ --- локально компактное
    хаусдорфово пространство и $\mu$ --- локально конечная внутренне компактно
    регулярная борелевская мера на $S$. Пусть $A$ --- атом меры $\mu$. Тогда
    \begin{enumerate}[label = (\roman*)]
        \item $\mu(A)$ конечно;
        \item существует точка $s\in A$ такая, что $\mu(A)=\mu(\{s\})$.
    \end{enumerate}
\end{proposition}
\begin{proof} $(i)$ Так как мера $\mu$ внутренне компактно регулярна, то
    существует компактное множество $K\subset A$ положительной меры. Так как
    мера $\mu$ локально конечна, то $\mu(K)$ конечно. Поскольку $A$ --- атом и
    $\mu(K)>0$, мы получаем $\mu(A)=\mu(K)<+\infty$.

    $(ii)$ По предыдущему пункту $0<\mu(A)<+\infty$. Пусть $\mathcal{K}$
    обозначает компактные подмножества $A$ той же самой меры что и $A$. Так как
    мера $\mu$ внутренне компактно регулярна, то существует компактное множество
    $K\subset A$ положительной меры. Так как $A$ --- атом, то $\mu(K)=\mu(A)$,
    значит $K\in\mathcal{K}$. Таким образом $\mathcal{K}$ непусто. Теперь
    рассмотрим два произвольных множества $K',K''\in\mathcal{K}$. Очевидно,
    $C:=K'\cap K''$ --- компактное подмножество в $A$. Допустим, что $\mu(C)=0$,
    и рассмотрим $L'=K'\setminus C$, $L''=K''\setminus C$. Это два дизъюнктных
    подмножества $A$, таких, что $\mu(L')=\mu(L'')=\mu(A)$, поэтому $\mu(A)\geq
        \mu(L'\cup L'')=2\mu(A)$. Противоречие, значит $\mu(C)>0$ и, как следствие,
    $C\in\mathcal{K}$. Поскольку $K', K''\in \mathcal{K}$ произвольны мы
    показали, что $\mathcal{K}$ --- семейство компактных множеств со свойством
    конечного пересечения. Следовательно, $K^*=\bigcap\mathcal{K}$ непусто.
    Очевидно, что $K^*$ компактно как пересечение компактных множеств. Допустим,
    $K^*$ содержит две различные точки $s'$ и $s''$. Рассмотрим одноточечные
    множества $C'=\{s'\}$ и $C''=\{s''\}$. Допустим, $\mu(C')>0$, тогда
    $\mu(C')=\mu(A)$ и $C'\in\mathcal{K}$, так как $A$ --- атом. Это
    противоречит минимальности $K^*$ так как $C'$ --- собственное подмножество
    $K^*$, поэтому $\mu(C')=0$. Аналогично, $\mu(C'')=0$. Рассмотрим
    $L=K^*\setminus (C'\cup C'')$, тогда $\mu(L)=\mu(K^*)=\mu(A)>0$. Так как
    мера $\mu$ внутренне компактно регулярна, то существует компактное множество
    $\hat{K}\subset L\subset A$ положительной меры, значит
    $\mu(\hat{K})=\mu(A)$. По построению $\hat{K}\in\mathcal{K}$ --- собственное
    подмножество $\mathcal{K}$. Это противоречит минимальности $K^*$, значит
    $K^*$ непустое множество без двух различных точек, а значит одноточечное.
    Таким образом, $\mu(A)=\mu(K^*)=\mu(\{s\})$ для некоторого $s\in A$.
\end{proof}

%-------------------------------------------------------------------------------
%	Projectivity of $C_0(S)$-modules L_p(S,\mu)
%-------------------------------------------------------------------------------

\section{Относительная проективность \texorpdfstring{$C_0(S)$}{C0S}-модулей
  \texorpdfstring{$L_p(S,\mu)$}{LpSmu}
 }\label{SectionRelativeProjectivityOfC0SModulesLpSmu}

Результаты этого параграфа в некотором смысле аналогичны тем, что получены для
модулей над алгеброй ограниченных измеримых функций, но случай $p=+\infty$,
похоже, не имеет простого критерия.

\begin{proposition}\label{LpEssC0ModCharac} Пусть $S$ --- локально компактное
    хаусдорфово пространство, $\mu$ --- локально конечная борелевская мера на
    $S$ и $1\leq p\leq+\infty$. Тогда
    \begin{enumerate}[label = (\roman*)]
        \item $[f]\in {L_p(S,\mu)}_{ess}$ тогда и только тогда, когда для любого
              $\varepsilon >0$ существует компактное множество $K\subset S$
              такое, что $\Vert [f\chi_{S\setminus K}]\Vert_{L_p(S,\mu)}<
                  \varepsilon$;

        \item если $p<+\infty$ и мера $\mu$ внутренне компактно регулярна, то
              ${L_p(S,\mu)}_{ess}=L_p(S,\mu)$.
    \end{enumerate}

    В частности, для любого компактного множества $K\subset S$ и $[f]\in
        L_p(S,\mu)$ верно $[f]\chi_K\in {L_p(S,\mu)}_{ess}$.
\end{proposition}
\begin{proof} Стандартное рассмотрение плотных подпространств.
\end{proof}


\begin{proposition}\label{MorphLpEssC0Prop} Пусть $S$ --- локально компактное
    хаусдорфово пространство, $\mu$ --- локально конечная борелевская мера на
    $S$. Допустим, задан $C_0(S)$-морфизм $\psi:{L_p(S,\mu)}_{ess}\to C_0(S)$
    где $1\leq p\leq+\infty$, функция $[f]\in L_p(S,\mu)$ и компактное множество
    $K\subset S$. Тогда

    $(i)$ если $[f]=[f]\chi_K$, то $\psi(f)|_{S\setminus K}=0$;

    $(ii)$ если $[f]=[f]\chi_K$ и $\psi(f)\neq 0$, то существует открытое
    множество $U\subset K$ положительной меры.
\end{proposition}
\begin{proof} $(i)$ Из пункта $(i)$ предложения~\ref{LpEssC0ModCharac} мы имеем
    $[f]=[f]\chi_K\in {L_p(S,\mu)}_{ess}$, поэтому можно говорить о функции
    $a=\psi(f)\in C_0(S)$. Пусть $V$ --- открытое множество, содержащее $K$,
    тогда существует непрерывная функция $b\in C_0(S)$ такая, что $b|_K=1$ и
    $b|_{S\setminus V}=0$ [\cite{DalesBanSpContFunDualSp}, теорема 1.4.25]. По
    построению $\chi_K=b\chi_K$, поэтому
    $a=\psi([f])=\psi([f]\chi_K)=\psi(b[f]\chi_K)=b\psi([f]\chi_K)=b\psi([f])=ba$.
    Поскольку $b|_{S\setminus V}=0$, то мы получаем $a|_{S\setminus V}=0$. Так
    как пространство $S$ хаусдорфово и $V$ --- произвольное открытое множество,
    содержащее $K$, то $a|_{S\setminus K}=0$.

    $(ii)$ Используя обозначения предыдущего пункта, мы имеем $a\neq 0$ и
    $a|_{S\setminus K}=0$. Рассмотрим неотрицательную непрерывную функцию
    $c=|a|$, тогда $c\neq 0$ и $c|_{S\setminus K}=0$. Так как $c\neq 0$, то
    открытое множество $U=c^{-1}((0, +\infty))$ непусто. Более того, $U\subset
        K$ так как $c|_{S\setminus K}=0$. Теперь рассмотрим произвольную точку $s\in
        U$. По построению $a(s)\neq 0$. Поскольку $\{s\}$ компактно, то существует
    непрерывная функция $e\in C_0(S)$ такая, что $e(s)=1$ и $e|_{S\setminus
        U}=0$ [\cite{DalesBanSpContFunDualSp}, теорема 1.4.25]. Рассмотрим функцию
    $[g]=e[f]\in {L_p(S,\mu)}_{ess}$, тогда $\psi([g])\neq 0$, так как
    $\psi([g])(s)=\psi(e[f])(s)=(e\psi([f]))(s)=e(s)\psi([f])(s)=a(s)\neq 0$.
    Поскольку $\psi([g])\neq 0$, то мы имеем $[g]\neq 0$ в ${L_p(S,\mu)}_{ess}$.
    Следовательно, $\mu(U)>0$, так как по построению $[g]\chi_{S\setminus U}=0$.
\end{proof}

\begin{proposition}\label{LpC0ModNecessCond}  Пусть $S$ --- локально компактное
    хаусдорфово пространство, $\mu$ --- локально конечная внутренне компактно
    регулярная борелевская мера на $S$. Пусть $1\leq p\leq +\infty$ и
    $L_p(S,\mu)$ --- относительно проективный $C_0(S)$-модуль. Тогда
    \begin{enumerate}[label = (\roman*)]
        \item мера $\mu$ внутренне открыто регулярна;

        \item любой атом меры $\mu$ является изолированной точкой в $S$;

        \item если $p<+\infty$ и мера $\mu$ внешне открыто регулярна, то мера
              $\mu$ чисто атомическая.
    \end{enumerate}
\end{proposition}
\begin{proof} $(i)$ Пусть $K\subset S$ --- компактное множество положительной
    меры. Из пункта $(i)$ предложения~\ref{LpEssC0ModCharac} следует, что
    функция $[f]:=[\chi_K]$ не равна нулю в ${L_p(S,\mu)}_{ess}$. Так как
    $C_0(S)$-модуль $L_p(S,\mu)$ относительно проективен, то по пункту $(ii)$
    предложения~\ref{RelProjNecesCond} существует $C_0(S)$-морфизм
    $\psi:L_p(S,\mu)\to C_0(S)$ такой, что $\psi([f])\neq 0$. Теперь из пункта
    $(ii)$ предложения~\ref{MorphLpEssC0Prop} мы получаем, что существует
    открытое множество $U\subset K$ положительной меры. Поскольку $K\subset S$
    произвольно, мы можем применить пункт $(ii)$
    предложения~\ref{NormalMeasCharac}. Тогда
    $\mu(B)=\mu(\operatorname{int}_S(B))$ для любого борелевского множества
    $B\subset S$. Осталось применить предложение~\ref{InnerOpenRegMeasCharac}.

    $(ii)$ Пусть $A$ --- атом меры $\mu$. Из пункта $(ii)$
    предложения~\ref{MeasAtomCharac} следует существование точки $s\in A$ такой,
    что $\mu(\{s\})=\mu(A)>0$. Из пункта $(i)$ следует, что
    $\mu(\operatorname{int}_S(\{s\}))=\mu(\{s\})>0$. Следовательно, $\{s\}$ ---
    открытое множество, т.е. $s$ --- изолированная точка.

    $(iii)$ Пусть $S_a^\mu$ --- множество одноточечных атомов меры $\mu$ и
    $S_c^\mu=S\setminus S_a^\mu$. Из пункта $(ii)$ мы знаем, что все атомы ---
    суть изолированные точки, поэтому $S_c^\mu$ замкнутое, а значит и
    борелевское множество. Рассмотрим произвольное компактное подмножество
    $K\subset S_c^\mu$. Допустим, что $\mu(K)>0$, тогда из пункта $(i)$
    предложения~\ref{LpEssC0ModCharac} функция $[f]:=[\chi_K]$ не равна нулю в
    ${L_p(S,\mu)}_{ess}$. Так как $C_0(S)$-модуль $L_p(S, \mu)$ относительно
    проективен, то из пункта $(ii)$ предложения~\ref{RelProjNecesCond} и
    предложения~\ref{MorphLpEssC0Prop} мы получаем $C_0(S)$-морфизм
    $\psi:{L_p(S,\mu)}_{ess}\to C_0(S)$ такой, что $\psi([f])\neq 0$ и
    $\psi([f])|_{S\setminus K}=0$. Обозначим $a:=\psi([f])\neq 0$. Так как
    $a|_{S\setminus K}=0$, то существует точка $s\in K$ такая, что $a(s)\neq 0$.
    Зафиксируем $\varepsilon > 0$. Заметим, что $s$ не является атомом, так как
    $s\in K\subset S_c^\mu$, значит, из внешней открытой регулярности меры $\mu$
    мы получаем открытое множество $W\subset S$ такое, что $s\in W$ и
    $\mu(W)<\varepsilon$. Так как $\{s\}$ компактно, то существует непрерывная
    функция $b\in C_0(S)$ такая, что $b(s)=1$, $0\leq b\leq 1$ и $b|_{S\setminus
        W}=0$ [\cite{DalesBanSpContFunDualSp}, теорема 1.4.25]. Так как $p<+\infty$,
    то мы получаем $\Vert b[f]\Vert_{L_p(S,\mu)} \leq {\mu(W\cap
            K)}^{1/p}<\varepsilon^{1/p}$. Наконец,
    $$
        |a(s)|=|a(s)b(s)|=|(ba)(s)|=|(b\psi([f]))(s)|
        =|\psi(b[f])(s)|\leq\Vert \psi (b[f])\Vert_{C_0(S)}\leq
    $$
    $$
        \leq\Vert\psi\Vert\Vert b[f]\Vert_{L_p(S,\mu)}
        \leq\Vert\psi\Vert\varepsilon^{1/p}.
    $$
    Поскольку $\varepsilon>0$ произвольно $|a(s)|=0$, но $a(s)\neq 0$ по выбору
    $s$. Противоречие, значит, $\mu(K)=0$. Так как $K\subset S_c^\mu$
    произвольно, то из внутренней компактной регулярности $\mu$ следует
    $\mu(S_c^\mu)=0$. Другим словами, мера $\mu$ чисто атомическая.
\end{proof}

\begin{proposition}\label{C0ModLpRetrProj} Пусть $S$ --- локально компактное
    хаусдорфово пространство, $\mu$ --- борелевская мера на $S$. Пусть $1\leq
        p\leq +\infty$ и $L_p(S,\mu)$ --- относительно проективный $C_0(S)$-модуль.
    Тогда для любого борелевского множества $B\subset S$ банахов $C_0(S)$-модуль
    $L_p(S,\mu^B)$ относительно проективен.
\end{proposition}
\begin{proof} Доказательство такое же как и в
    предложении~\ref{BSigmaModLpRetrProj}.
\end{proof}

\begin{theorem}\label{ReflLpC0ModCrit} Пусть $S$ --- локально компактное
    хаусдорфово пространство, $\mu$ --- разложимая внутренне компактно
    регулярная борелевская мера на $S$. Пусть $1\leq p< +\infty$. Тогда
    следующие условия эквивалентны:
    \begin{enumerate}[label = (\roman*)]
        \item $L_p(S,\mu)$ --- относительно проективный $C_0(S)$-модуль;

        \item мера $\mu$ чисто атомическая, и все атомы являются изолированными
              точками.
    \end{enumerate}
\end{theorem}
\begin{proof} $(i)\implies (ii)$ Пусть $\mathcal{D}$ разложение $S$ на
    борелевские множества конечной меры. Зафиксируем $D\in\mathcal{D}$ и
    рассмотрим $C_0(S)$-модуль $L_p(S,\mu^D)$. Так как множество $D$ имеет
    конечную меру, то $\mu^D$ --- конечная, внутренне компактно регулярная и
    внешне открыто регулярная мера. По предложению~\ref{C0ModLpRetrProj} банахов
    $C_0(S)$-модуль $L_p(S,\mu^D)$ относительно проективен. Из пункта $(iii)$
    предложения~\ref{LpC0ModNecessCond} мы получаем, что $\mu^D$ (и тем более
    $\mu_D$) --- чисто атомическая мера, и все ее атомы являются изолированными
    точками. Так как $D\in\mathcal{D}$ произвольно, то по предложению
    [\cite{FremMeasTh2}, упражнение 214X (i)] мера $\mu$ чисто атомическая мера,
    и все ее атомы являются изолированными точками.

    $(ii)\implies (i)$ Пусть $S_a^\mu$ обозначает множество одноточечных атомов
    меры $\mu$. Так как все точки в $S_a^\mu$ изолированы, то пространство
    $S_a^\mu$ дискретно и $C_0(S_a^\mu)$ --- бипроективная алгебра
    [\cite{HelHomolBanTopAlg}, теорема 4.5.26]. Так как $p<+\infty$, мера $\mu$
    атомическая мера и все ее атомы являются изолированными точками, то
    $C_0(S_a^\mu)$-модуль $L_p(S,\mu)$ существенный. Учитывая все вышесказанное,
    из [\cite{HelBanLocConvAlg}, предложение VII.1.60(II)] мы получаем, что
    $C_0(S_a^\mu)$-модуль $L_p(S,\mu)$ относительно проективен. Так как
    $S_a^\mu$ --- открытое подмножество $S$, то $C_0(S_a^\mu)$ является
    двусторонним замкнутым идеалом  $C_0(S)$. Теперь из
    [\cite{RamsHomPropSemgroupAlg}, предложение 2.3.3(i)] следует, что
    $C_0(S)$-модуль $L_p(S,\mu)$ относительно проективен.
\end{proof}

Случай $C_0(S)$-модуля $L_\infty(S,\mu)$ намного сложнее. Мы дадим два
необходимых, но достаточно ограничительных условия относительно проективности.

\begin{definition}\label{WideFamilyDef} Пусть $S$ --- локально компактное
    хаусдорфово пространство, $\mu$ --- борелевская мера на $S$. Семейство
    $\mathcal{F}$ борелевских подмножеств $S$ называется широким, если
    \begin{enumerate}[label = (\roman*)]
        \item каждый элемент $F$ имеет конечную положительную меру и содержится
              в некотором компактном подмножестве;

        \item каждое компактное подмножество $S$ пересекает лишь конечное число
              множеств из $\mathcal{F}$;

        \item любые два различных множества в $\mathcal{F}$ не пересекаются.
    \end{enumerate}
\end{definition}

\begin{proposition}\label{LInfEssNotCompl} Пусть $S$ --- локально компактное
    хаусдорфово пространство, $\mu$ --- борелевская мера на $S$. Если $S$
    содержит широкое семейство $\mathcal{F}$, то существенная часть
    $C_0(S)$-модуля $L_\infty(S,\mu)$ не дополняема в $L_\infty(S,\mu)$.
\end{proposition}
\begin{proof} Допустим, что ${L_\infty(S,\mu)}_{ess}$ дополняемо в
    $L_\infty(S,\mu)$, тогда существует ограниченный линейный оператор
    $P:L_\infty(S,\mu)\to {L_\infty(S,\mu)}_{ess}$ такой, что $P([f])=[f]$ для
    всех $[f]\in {L_\infty(S,\mu)}_{ess}$. Теперь для данного широкого семейства
    $\mathcal{F}={(F_\lambda)}_{\lambda\in\Lambda}$ мы определим ограниченный
    линейный оператор
    $$
        I:\ell_\infty(\Lambda)\to L_\infty(S,\mu):
        x\mapsto\biggl[\sum_{\lambda\in\Lambda}x_\lambda \chi_{F_\lambda}\biggr]
    $$
    который корректно определен так как семейство $\mathcal{F}$ дизъюнктное.
    Рассмотрим $x\in c_0(\Lambda)$. Зафиксируем $\varepsilon > 0$, тогда
    существует конечное подмножество $\Lambda_0\subset\Lambda$ такое, что
    $|x_\lambda|<\varepsilon$ для всех $\lambda\in\Lambda\setminus\Lambda_0$.
    Пусть $K_\lambda$ обозначает компактное множество содержащее $F_\lambda$ для
    $\lambda\in\Lambda$. Тогда $K_0=\bigcup_{\lambda\in\Lambda_0}K_\lambda$ ---
    компакт. Если $s\in S\setminus K$, то $\chi_{F_\lambda}(s)=0$ для всех
    $\lambda\in\Lambda\setminus\Lambda_0$. Следовательно, $\Vert
        I(x)\chi_{S\setminus K}\Vert_{L_\infty(S,\mu)}
        =\Vert[\sum_{\lambda\in\Lambda\setminus\Lambda_0}x_\lambda\chi_{F_\lambda}]
        \Vert_{L_\infty(S,\mu)}
        =\sup_{\lambda\in\Lambda\setminus\Lambda_0}|x_\lambda|<\varepsilon$. Теперь
    из пункта $(i)$ предложения~\ref{LpEssC0ModCharac} мы получаем, что $I(x)\in
        {L_\infty(S,\mu)}_{ess}$. Далее мы определим ограниченный линейный оператор
    $$
        R:L_\infty(S,\mu)\to c_0(\Lambda):
        [f]\mapsto \biggl(
        \lambda\mapsto{\mu(F_\lambda)}^{-1}\int_{F_\lambda} f(s)d\mu(s)
        \biggr).
    $$
    Единственная вещь, которая требует пояснения --- это тот факт, что образ $R$
    содержится в $c_0(\Lambda)$. Зафиксируем $[f]\in {L_\infty(S,\mu)}_{ess}$.
    Пусть $\varepsilon>0$. Из пункта $(i)$ предложения~\ref{LpEssC0ModCharac}
    следует, что существует компакт $K\subset S$ такой, что $\Vert
        [f]\chi_{K}\Vert_{L_\infty(S,\mu)}<\varepsilon$. Рассмотрим множество
    $\Lambda_K=\{\lambda\in\Lambda: K\cap F_\lambda\neq\varnothing \}$. По
    определению семейства $\mathcal{F}$ множество $\Lambda_K$ конечно. Для
    любого $\lambda\in\Lambda\setminus \Lambda_K$ выполнено $F_\lambda\cap
        K=\varnothing$, поэтому $|{R(f)}_\lambda|\leq\varepsilon$. Так как
    $\varepsilon>0$ произвольно $R([f])\in c_0(\Lambda)$. Теперь рассмотрим
    ограниченный линейный оператор $Q=RPI$. Напомним, что $P([f])=[f]$ для всех
    $[f]\in {L_\infty(S,\mu)}_{ess}$. Тогда легко проверить, что для всех $x\in
        c_0(\Lambda)$ и $\lambda\in\Lambda$ верно ${Q(x)}_\lambda=x_\lambda$. Таким
    образом, $Q:\ell_\infty(\Lambda)\to c_0(\Lambda)$ --- ограниченный линейный
    оператор такой, что $Q(x)=x$ для всех $x\in c_0(\Lambda)$. Так как $\Lambda$
    бесконечно мы получаем противоречие с теоремой
    Филлипса~\cite{PhilOnLinTran}. Следовательно, ${L_\infty(S,\mu)}_{ess}$
    недополняемо в $L_\infty(S,\mu)$.
\end{proof}

Теперь нам нужно напомнить некоторые понятия из общей топологии. Семейство
$\mathcal{F}$ подмножеств в топологическом пространстве $S$ называется локально
конечным, если каждая точка $S$ имеет открытую окрестность, которая пересекается
лишь с конечным числом множеств из $\mathcal{F}$. Топологическое пространство
$S$ называется псевдокомпактным, если каждое локально конечное дизъюнктное
семейство непустых открытых мнoжеств конечно.

\begin{proposition}\label{LInfRelProjSuppCond} Пусть $S$ --- локально компактное
    хаусдорфово пространство, $\mu$ --- локально конечная борелевская мера на
    $S$. Если $C_0(S)$-модуль $L_\infty(S,\mu)$ относительно проективен, то
    носитель меры $\operatorname{supp}(\mu)$ псевдокомпактен.
\end{proposition}
\begin{proof} Обозначим $M:=\operatorname{supp}(\mu)$. Допустим, что $M$ не
    псевдокомпактно, тогда существует бесконечное дизъюнктное локально конечное
    семейство $\mathcal{U}$ непустых открытых множеств в $M$. Так как $S$
    локально компактно, то для каждого $U\in\mathcal{U}$ мы можем выбрать
    непустое открытое множество $V_U$ и компакт $K_U$ такие, что $V_U\subset
        K_U\subset  U$. Так как мера $\mu$ локально конечна, то мы можем выбрать $V$
    так, чтобы $\mu(V)$ было конечно. Более того, $\mu(V)>0$ так как $V$
    открытое подмножество $M$. Очевидно, что семейство
    $\mathcal{V}=\{V_U:U\in\mathcal{U}\}$ бесконечно, дизъюнктно и локально
    конечно. Таким образом, для любого $s\in S$ существует открытое множество
    $W_s$ такое, что $s\in W_s$ и множество $\{V\in\mathcal{V}: V\cap
        W_s\neq\varnothing \}$ конечно.

    По построению $\mathcal{V}$ --- дизъюнктное семейство открытых множеств
    положительной меры, и каждое множество семейства содержится в своем
    компакте. Пусть $K\subset S$ --- произвольный компакт. Тогда $\{W_s:s\in
        K\}$ --- открытое покрытие $K$. Поскольку $K$ компактно, то существует
    конечное множество $S_0$ такое, что $\{W_s:s\in S_0\}$ --- покрытие $K$. Так
    как каждое множество $W_s$ пересекает лишь конечное число множеств из
    $\mathcal{V}$, то этим же свойством обладает $\bigcup_{s\in S_0}W_s$ и тем
    более $K$. Таким образом, $\mathcal{V}$ --- широкое семейство. По
    предложению~\ref{LInfEssNotCompl} существенная часть $C_0(S)$-модуля
    $L_\infty(S,\mu)$ не дополняема в $L_\infty(S,\mu)$. Теперь из
    предложения~\ref{ProjModEssPartCompl} следует, что $L_\infty(S,\mu)$ не
    является относительно проективным $C_0(S)$-модулем. Противоречие, значит,
    пространство $M$ псевдокомпактно.
\end{proof}

\begin{theorem}\label{LInfReProjNecessCond} Пусть $S$ --- локально компактное
    хаусдорфово пространство, $\mu$ --- локально конечная внутренне компактно
    регулярная борелевская мера на $S$. Если $C_0(S)$-модуль $L_\infty(S,\mu)$
    относительно проективен, то мера $\mu$ внутренне открыто регулярна и ее
    носитель $\operatorname{supp}(\mu)$ псевдокомпактен.

\end{theorem}
\begin{proof} Результат следует из предложений~\ref{LpC0ModNecessCond}
    и~\ref{LInfRelProjSuppCond}.
\end{proof}

Хотя последняя теорема и не является критерием, следует сказать несколько слов о
том, как этот гипотетический критерий мог бы выглядеть. Последняя теорема
накладывает ограничения на топологию пространства $S$, но эта теорема не может
описать ее полностью. Действительно, рассмотрим произвольное локально компактное
хаусдорфово пространство $S$, в котором есть хотя бы одна изолированная точка
$\{s\}$. Пусть $\mu$ --- мера сосредоточенная в точке $\{s\}$. Легко проверить,
что получающийся $C_0(S)$-модуль $L_\infty(S,\mu)$ относительно проективен.
Таким образом, нам следует ограничиться рассмотрением строго положительных мер.

Если мера $\mu$ строго положительна, то в предположениях
предложения~\ref{LInfReProjNecessCond} пространство $S$ псевдокомпактно.
Напомним, что каждая непрерывная функция на псевдокомпактном пространстве
ограничена [\cite{HrusPseudCompTopSp}, теорема 1.1.3(3)]. Теперь заметим, что
любая конечная внутренне открыто регулярная мера является остаточной, тогда по
результату [\cite{ZindResMeasLocCompSp}, следствие 2.7] каждая измеримая функция
на $S$ непрерывна на открытом плотном множестве. Эти факты позволяют
предположить, что $S$ обладает своеобразной топологией. Действительно, если
пространство $S$ не имеет изолированных точек и обладает ненулевой конечной
нормальной мерой, то $S$ не может быть сепарабельным локально компактным
хаусдорфовым [\cite{DalesBanSpContFunDualSp}, предложение 4.7.20], локально
связным локально компактным хаусдорфовым [\cite{DalesBanSpContFunDualSp},
предложение 4.7.23], связным локально компактным хаусдорфовым $F$-пространством
[\cite{DalesBanSpContFunDualSp}, предложение 4.7.24], сепарабельным метризуемым
[\cite{FlachNormMeasTopSp}, пример 1].

Из предыдущего обсуждения заманчиво предположить, что пространство $C_0(S)$
``похоже'' на  $L_\infty(S,\mu)$, когда мера $\mu$ строго положительна и модуль
$L_\infty(S,\mu)$ относительно проективен. В этом направлении есть следующий
результат.

\begin{proposition}\label{LInfReProjSuffCond} Пусть $S$ --- гиперстоуново
    пространство и $\mu$ --- конечная строго положительная нормальная внутренне
    компактно регулярная борелевская мера на $S$. Тогда $C_0(S)$-модуль
    $L_\infty(S,\mu)$ относительно проективен.
\end{proposition}
\begin{proof} Из [\cite{DalesBanSpContFunDualSp}, следствие 4.7.6] следует, что
    пространства $L_\infty(S,\mu)$ и $C_0(S)$ изоморфны как $C^*$-алгебры. В
    частности, $L_\infty(S,\mu)$ изоморфно $C_0(S)$ как $C_0(S)$-модуль. Так как
    $S$ компактно, то $C_0(S)$ --- унитальная алгебра и поэтому она относительно
    проективна как $C_0(S)$-модуль [\cite{HelBanLocConvAlg}, пример VII.1.1].
\end{proof}

Для строго положительных мер последнее предложение является единственным
известным примером относительно проективного $C_0(S)$-модуля $L_\infty(S,\mu)$.

\section{Финансирование}\label{SectionFunding} Работа поддержана Российским
Фондом Фундаментальных Исследований (грант No. 19--01--00447).

\begin{thebibliography}{999}
    %
    \bibitem{HelHomolBanTopAlg}
    \textit{A. Ya. Helemskii} The homology of Banach and topological algebras,
    Springer, 41 (1989)
    %
    %
    \bibitem{SelivBiprojBanAlg}
    \textit{Yu. V. Selivanov} Biprojective Banach algebras, Math. USSR-Izv.,
    15:2 (1980), 387–399
    %
    %
    \bibitem{GrothProdTenTopNucl}
    \textit{A. Grothendieck} Produits tensoriels topologiques et espaces
    nucl{\'e}aires, Mem. Amer. Math. Soc., №16 (1955)
    %
    %
    \bibitem{FremMeasTh2}, \textit{D. H. Fremlin} Measure Theory, Vol. 2,
    {2003}, Torres Fremlin
    %
    %
    \bibitem{FremMeasTh4.1}, \textit{D. H. Fremlin} Measure Theory, Vol. 4(1),
    {2003}, Torres Fremlin
    %
    %
    \bibitem{DalesBanSpContFunDualSp}
    \textit{H. G. Dales, F. K. Dashiel Jr., A.T.-M. Lau, D. Strauss} Banach
    spaces of continuous functions as dual spaces, Berlin, Springer (2016)
    %
    %
    \bibitem{HelBanLocConvAlg}
    \textit{A.Ya. Helemskii}, Banach and locally convex algebras. Oxford
    University Press, (1993)
    %
    %
    \bibitem{RamsHomPropSemgroupAlg}
    \textit{P. Ramsden} Homological properties of semigroup algebras, The
    University of Leeds, PhD thesis (2009)
    %
    %
    \bibitem{PhilOnLinTran}
    \textit{R. S. Phillips} On linear transformations, Trans. Amer. Math. Soc.
    48 (1940), 516--541
    %
    %
    \bibitem{HrusPseudCompTopSp}
    \textit{M. Hrusak, A. Tamariz-Mascarua, M. Tkachenko} Pseudocompact
    topological spaces. Springer (2018)
    %
    %
    \bibitem{ZindResMeasLocCompSp}
    \textit{O. Zindulka} Residual measures in locally compact spaces. Topology
    and its Applications 108, no. 3 (2000), 253--265.
    %
    %
    \bibitem{FlachNormMeasTopSp}
    \textit{J. Flachsmeyer} Normal and category measures on topological spaces.
    General Topology and its Relations to Modern Analysis and Algebra (1972),
    109--116.
    %
\end{thebibliography}

Норберт Немеш, Факультет механики и математики, Московский Государственный
Университет, Москва 119991 Россия

\textit{E-mail address:} nemeshnorbert@yandex.ru

\end{document}