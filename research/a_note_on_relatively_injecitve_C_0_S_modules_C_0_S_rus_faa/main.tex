% chktex-file 19 chktex-file 35
\documentclass[12pt]{article}
\usepackage[
    left=2cm,right=2cm,top=2cm,bottom=2cm,bindingoffset=0cm]{geometry}
\usepackage{amssymb}
\usepackage{amsmath}
\usepackage{amsthm}
\usepackage{enumerate}
\usepackage[T1,T2A]{fontenc}
\usepackage[utf8]{inputenc}
\usepackage[russian]{babel}
\usepackage[matrix,arrow,curve]{xy}
\usepackage[
    colorlinks=true, urlcolor=blue, linkcolor=blue, citecolor=blue,
    pdfborder={0 0 0}]{hyperref}

%-------------------------------------------------------------------------------
\newtheorem{theorem}{Теорема}[section]
\newtheorem{lemma}[theorem]{Лемма}
\newtheorem{proposition}[theorem]{Предложение}
\newtheorem{remark}[theorem]{Замечание}
\newtheorem{corollary}[theorem]{Следствие}
\newtheorem{definition}[theorem]{Определение}
\newtheorem{example}[theorem]{Пример}

\newcommand{\projtens}{\mathbin{\widehat{\otimes}}}
\newcommand{\convol}{\ast}
\newcommand{\projmodtens}[1]{\mathbin{\widehat{\otimes}}_{#1}}
\newcommand{\isom}[1]{\mathop{\mathbin{\cong}}\limits_{#1}}
%-------------------------------------------------------------------------------

\begin{document}

\begin{center}
    \Large \textbf{Об относительной инъективности $C_0(S)$-модулей
        $C_0(S)$}\\[0.5cm]
    \small {Н. Т. Немеш}\\[0.5cm]
\end{center}

\thispagestyle{empty}

\textbf{Аннотация:} В этой статье мы обсудим некоторые необходимые и некоторые
достаточные условия относительной инъективности $C_0(S)$-модулей $C_0(S)$, где
$S$ --- локально компактное хаусдорфово пространство. Также мы докажем версию
теоремы Собчика для банаховых модулей. Основной результат статьи: если
$C_0(S)$-модуль $C_0(S)$ относительно инъективен, то для любой предельной точки
$s\in S$ выполнено $S=\beta(S\setminus \{s\})$.

\textbf{Ключевые слова:} инъективный банахов модуль, $C_0(S)$-пространство,
почти компактное пространство.

\medskip
\textbf{Abstract:} In this note we discuss some necessary and some sufficient
conditions for relative injectivity of the $C_0(S)$-module $C_0(S)$, where $S$
is a locally compact Hausdorff space. We also give a Banach module version of
Sobczyk's theorem. The main result of the paper is as follows: if
$C_0(S)$-module $C_0(S)$ is relatively injective then  the equality
$S=\beta(S\setminus \{s\})$ holds for any limit point $s\in S$.
\medskip

\textbf{Keywords:} injective Banach module, $C_0(S)$-space, almost compact
space.

\bigskip

%-------------------------------------------------------------------------------
%	Introduction
%-------------------------------------------------------------------------------

\section{Введение}\label{SectionIntroduction}

Задачи продолжения отображений играли важную роль в функциональном анализе с
самого его зарождения. Первый пример успешно решенной задачи подобного рода это
теорема
Хана-Банаха~\cite{HahnLinSystInLinSp,BanachOnLinFuncI,BanachOnLinFuncII}. В
современной терминологии эта теорема утверждает, что поле комплексных чисел
является инъективным объектом в категории банаховых пространств. Все известные
инъективные банаховы пространства изоморфны пространству непрерывных функций на
некотором компактном пространстве~\cite{BlascIvorConstrInjSpCK}. Этот факт дает
мотивировку нашему изучению инъективности пространств непрерывных функций, но в
этот раз мы рассматриваем их как банаховы модули.

%-------------------------------------------------------------------------------
%	Preliminaries
%-------------------------------------------------------------------------------

\section{Предварительные сведения}\label{SectionPreliminaries}

Прежде чем мы перейдем к основной теме статьи нам нужно напомнить несколько
определений и договориться об обозначениях.

Пусть $M$ ---  подмножество множества $N$, тогда $\chi_M$ обозначает
индикаторную функцию $M$. Если $f:N\to L$ --- произвольная функция, то $f|_M$
обозначает ее ограничение на $M$.

Пусть $S$ ---  хаусдорфово топологическое пространство. Пространство $S$
называется \textit{экстремально несвязным}, если замыкание любого открытого
множества в $S$ открыто; \textit{стоуновым}, если оно компактно и экстремально
несвязно; \textit{псевдокомпактным}, если любая непрерывная функция на $S$
ограничена. Если $S$ --- хаусдорфово некомпактное локально компактное
пространство, то через $\alpha S$ мы будем обозначать \textit{александровскую
    компактификацию} $S$, а через $\beta S$ мы будем обозначать
\textit{стоун-чеховскую компактификацию} $S$. \textit{Стоун-чеховский нарост}
$\beta S\setminus S$ мы будем обозначать через $S^*$. Хаусдорфово некомпактное
топологическое пространство $S$ называется \textit{почти компактным}, если
$\alpha S=\beta S$. Типичный пример почти компактного пространства это $[0,
    \omega_1)$,  % chktex 9
где $\omega_1$ --- первый несчетный ординал~\cite[глава 1.3]{HrusPsdCompTopSp}.
Больше об экстремально несвязных, псевдокомпактных и почти компактных
пространствах можно найти в~\cite[раздел 6.2]{EngkingGenTop},~\cite[раздел
    3.10]{EngkingGenTop} и~\cite[глава 1.3]{HrusPsdCompTopSp} соответственно.

Для заданного некомпактного локально компактного хаусдорфова пространства $S$
можно рассмотреть базу фильтра $\mathcal{B}_S$, состоящую из дополнений
компактных подмножеств $S$. Фильтр $\mathcal{F}_S$, порожденный $\mathcal{B}_S$,
называется \textit{фильтром Фреше} на $S$. Теперь мы можем определить несколько
функциональных пространств на $S$.  Через $C(S)$ мы обозначим пространство
непрерывных функций на $S$. Это пространство нормируемо, если $S$ компактно.
Через $C_b(S)$ мы обозначим банахово пространство непрерывных ограниченных
функций на $S$. Символ $C_l(S)$ будет обозначать пространство непрерывных
функций сходящихся по фильтру $\mathcal{F}_S$ к конечному пределу. Через
$C_0(S)$ мы обозначим замкнутое подпространство $C_l(S)$, состоящее из функций,
сходящихся по $\mathcal{F}_S$ к нулю (эти функции также называются исчезающими
на бесконечности). Если $S$ компактно, то все вышеперечисленные функциональные
пространства совпадают с $C(S)$.

Пусть $A$ --- банахова алгебра. Мы будем рассматривать только правые банаховы
модули над $A$ с сжимающим билинейным оператором внешнего умножения
$\cdot:X\times A\to X$. Пусть $X$ и $Y$ --- два правых банаховых $A$-модуля,
тогда отображение $\phi:X\to Y$ называется \textit{$A$-морфизмом}, если оно
непрерывно и является морфизмом $A$-модулей. Правые банаховы $A$-модули и
$A$-морфизмы образуют категорию, которую мы будем обозначать $\mathbf{mod}-A$.

Понятие инъективности в $\mathbf{mod}-A$ может быть определено несколькими
способами. Пусть $\xi:X\to Y$ --- $A$-морфизм. Он называется
\textit{относительно допустимым}, если $\eta\circ \xi=1_X$ для некоторого
ограниченного линейного оператора $\eta:Y\to X$; \textit{топологически
    допустимым}, если $\xi$ является линейным гомеоморфизмом на свой образ;
\textit{метрически допустимым}, если $\xi$ изометричен. Банахов $A$-модуль $J$
называется \textit{относительно инъективным} (соотв. \textit{топологически
    инъективным}, соотв. \textit{метрически инъективным}) если для любого
относительно (соотв. топологически, соотв. метрически) допустимого $A$-морфизма
$\xi:X\to Y$ и любого $A$-морфизма $\phi:X\to J$ существует непрерывный (соотв.
непрерывный, соотв. непрерывный с такой же нормой, как  $\phi$) морфизм
$A$-модулей $\psi:Y\to J$, делающий диаграмму
$$
    \xymatrix{
    & {Y} \\
    {J} \ar@{<--}[ur]^{\psi} &{X} \ar[u]^{\xi} \ar[l]^{\phi}}  % chktex 3
$$
коммутативной.

Если $A=\{0\}$, то категория $\mathbf{mod}-A$ превращается в обычную категорию
банаховых пространств. В этом случае все банаховы пространства относительно
инъективны. Что касается топологически и метрически инъективных банаховых
пространств, то в стандартной литературе они называются
\textit{$\mathcal{P}_\lambda$-пространствами} и
\textit{$\mathcal{P}_1$-пространствами} соответственно.
До сих пор не найдено простого описания
$\mathcal{P}_\lambda$-пространств~\cite[стр. vi]{AvilSepInjBanSp},
но для $\mathcal{P}_1$-пространств вопрос закрыт. Всякое
$\mathcal{P}_1$-пространство изометрически изоморфно $C(S)$-пространству, где
$S$ --- стоуново пространство~\cite{HasumExtPropCompBanSp}.

%-------------------------------------------------------------------------------
%   Necessary and sufficient conditions of injectivity
%-------------------------------------------------------------------------------

\section{Необходимые и достаточные условия
  инъективности}\label{SecionNecessaryAndSufficientConditionsForInjectivity}

В этом параграфе мы обсудим необходимые и достаточные условия относительной
инъективности $C_0(S)$-модулей $C_0(S)$, где $S$ --- локально компактное
хаусдорфово пространство. Мы начнем с очень ограничительного достаточного
условия.

\begin{proposition}\label{SStonImplRelInjCSModCS} Пусть $S$ --- стоуново
    пространство. Тогда $C(S)$-моудль $C(S)$ относительно инъективен.
\end{proposition}
\begin{proof} Обозначим $A=C(S)$. Поскольку $S$ --- стоуново пространство, то
    $A$ является $AW^*$-алгеброй~\cite[глава 1, параграф 7]{BerbBaerStRng}.
    В~\cite[теорема 2]{TakHanBanThAndJordDecomOfModMap} было показано, что любая
    $AW^*$-алгебра метрически инъективна, как бимодуль над собой. Внимательное
    изучение доказательства этой теоремы показывает, что те же самые рассуждения
    верны и для правого $A$-модуля $A$. Осталось заметить, что всякий метрически
    инъективный модуль относительно инъективен.
\end{proof}

Чтобы дать достаточно обременительное необходимое условие относительной
инъективности $C(S)$-модуля $C(S)$, мы начнем со специального случая.

\begin{proposition}\label{RelInjCaSModCaSImplSAlmComp} Пусть $S$ ---
    некомпактное локально компактное хаусдорфово пространство. Предположим, что
    $C(\alpha S)$-модуль $C(\alpha S)$ относительно инъективен. Тогда $S$ ---
    почти компактное пространство.
\end{proposition}
\begin{proof} Очевидно, банаховы алгебры $C(\alpha S)$ и $C_l(S)$ изометрически
    изоморфны, поэтому $C_l(S)$-модуль $C_l(S)$ относительно инъективен.
    Заметим, что $C_0(S)$ это двусторонний идеал в $C_l(S)$, состоящий из
    функций, исчезающих на бесконечности. Этот идеал дополняем посредством
    проекции $P:C_l(S)\to C_0(S):x\mapsto x-(\lim_{\mathcal{F}_S}x(s))\chi_{S}$.
    Рассмотрим изометрическое вложение $\xi:C_0(S)\to C_l(S):x\mapsto x$,
    которое является $C_l(S)$-морфизмом. Так как $P\circ\xi=1_{C_0(S)}$, то
    $\xi$ относительно допустим.

    Зафиксируем $f \in C_b(S)$ и рассмотрим $C_l(S)$-морфизм $\phi:C_0(S)\to
        C_l(S):x\mapsto f \cdot x$. Поскольку $C_l(S)$  --- относительно
    инъективный $C_l(S)$-модуль, то существует $C_l(S)$-морфизм
    $\psi:C_l(S)\to C_l(S)$ такой, что $\phi=\psi\circ\xi$. В частности, для
    всех $x\in C_0(S)$ мы имеем $f\cdot
        x=\phi(x)=\psi(\xi(x))=\psi(x)=\psi(x\cdot \chi_{S})=x\cdot
        \psi(\chi_{S})$. Рассмотрим произвольную точку $s\in S$. Так как
    пространство $S$ хаусдорфово и локально компактно, то существует
    непрерывная функция $e\in C_0(S)$ такая, что $e(s)=1$~\cite[следствие
        3.3.3]{EngkingGenTop}. Значит, $f(s)=f(s)e(s)=(f\cdot
        e)(s)=(e\cdot\psi(\chi_{S}))(s)=e(s)\psi(\chi_{S})(s)=\psi(\chi_{S})(s)$.
    Поскольку $s\in S$ произвольно $f=\psi(\chi_{S})$. По построению
    $\psi(\chi_S)\in C_l(S)$, поэтому $f\in C_l(S)$. Так как функция $f\in
        C_b(S)$ была выбрана произвольно, то $C_b(S)\subset C_l(S)$. Это
    возможно только, если $C_b(S)=C_l(S)$.

    Вспомним, что в категории банаховых пространств $C_b(S)$ изометрически
    изоморфно $C(\beta S)$ и $C_l(S)$ изометрически изоморфно $C(\alpha S)$.
    Отсюда мы заключаем, что банаховы пространства $C(\beta S)$ и $C(\alpha S)$
    изометрически изоморфны. По теореме Банаха-Стоуна~\cite[теорема
        83]{StoneBanStAppBoolRngToTop} пространства $\alpha S$ и $\beta S$
    гомеоморфны. Следовательно, $S$ почти компактно.
\end{proof}

Настало время определить еще одно понятие компактности.

\begin{definition} Компактное хаусдорфово пространство $S$ называется равномерно
    компактным, если для каждой предельной точки $s\in S$ пространство
    $S\setminus \{s\}$ почти компактно.
\end{definition}

Другими словами компактное хаусдорфово пространство $S$ равномерно компактно
если для каждой предельной точки $s\in S$ выполнено $S=\beta(S\setminus \{s\})$.

\begin{proposition}\label{StoneSpUnifComp} Стоуновы пространства равномерно
    компактны.
\end{proposition}
\begin{proof} Пусть $S$ --- стоуново пространство и $s\in S$ его предельная
    точка. Обозначим $S_\circ=S\setminus \{s\}$. Так как пространство $S$
    компактно и $S_\circ$ его открытое подмножество, то $S_\circ$ локально
    компактно. Поскольку $s$ --- предельная точка в $S$, то пространство
    $S_\circ$ некомпактно и $\alpha S_\circ=S$. Рассмотрим два функционально
    отделимых множества $A,B\subset S_\circ$. По определению это значит, что
    существует непрерывная функция $f:S_\circ\to\mathbb[0,1]$ такая, что
    $f|_A=\{0\}$ и $f|_B=\{1\}$. Рассмотрим непересекающиеся множества
    $U=f^{-1}([0,1/3))\subset S_\circ$ % chktex 9
    и $V=f^{-1}((1/2,1])\subset S_\circ$. % chktex 9
    Очевидно, $U$ и $V$ открыты в $S$. Так как $S$ экстремально несвязно, то $U$
    и $V$ имеют непересекающиеся замыкания в $S$. Так как $A\subset U$,
    $B\subset V$,  то множества $A$ и $B$ так же имеют непересекающиеся
    замыкания в $S$. По теореме~\cite[теорема 3.6.2]{EngkingGenTop} мы получаем,
    что $\beta S_\circ=\alpha S_\circ=S$.
\end{proof}

\begin{corollary}\label{MesSpUnifCompIffFin} Метризуемое пространство равномерно
    компактно тогда и только тогда, когда оно конечно.
\end{corollary}
\begin{proof} Пусть $S$ --- метризуемое пространство с топологией индуцированной
    метрикой $d$. Предположим, что $S$ равномерно компактно. Допустим, что $S$
    имеет предельную точку $s\in S$. Тогда пространство $S_\circ= S\setminus
        \{s\}$ почти компактно и как следствие псевдокомпактно~\cite[предложение
        1.3.10]{HrusPsdCompTopSp}. Рассмотрим непрерывную функцию
    $f:S_\circ\to\mathbb{R}:x\mapsto {d(x,s)}^{-1}$. Эта функция неограниченна
    потому, что $s\in S$ передельная точка. Таким образом, $S_\circ$ не
    псевдокомпактно. Противоречие, значит $S$ --- компактное метризуемое
    пространство без предельных точек, т.е. $S$ конечно.

    Обратно, если пространство $S$ конечно, то оно по тривиальным причинам
    равномерно компактно.
\end{proof}

Следующий пример принадлежит К.П. Харту.

\begin{proposition}\label{UcountProdCompSpIsUnifComp} Тихоновское произведение
    несчетного семейства компактных хаусдорфовых пространств, состоящих из более
    чем одной точки, равномерно компактно.
\end{proposition}
\begin{proof} Пусть $\mathcal{S}={(S_\lambda)}_{\lambda\in\Lambda}$ ---
    семейство нетривиальных компактных хаусдорфовых пространств. По теореме
    Тихонова их произведение $S$ компактно~\cite[теорема 3.2.4]{EngkingGenTop}.
    Пусть $s\in S$ --- предельная точка в $S$. Поскольку пространства
    $S_\lambda$ состоят из более чем одной точки для всех $\lambda\in\Lambda$,
    то существует точка $s'\in S$ такая, что $s_\lambda\neq s'_\lambda$ для всех
    $\lambda\in\Lambda$. Пусть $\Sigma(s')$ --- $\Sigma$-произведение
    $\mathcal{S}$ в точке $s'$, то есть $\Sigma(s')$ состоит из всех точек $S$,
    отличающихся от $s'$ в не более чем счетном числе координат. Из
    упражнения~\cite[упражнение 3.12.23(c)]{EngkingGenTop} следует, что
    $S=\beta(\Sigma(s'))$. Так как 
    $\Sigma(s')\subset S\setminus \{s\}\subset S=\beta(\Sigma(s'))$, 
    то по следствию~\cite[следствие 3.6.9]{EngkingGenTop}
    получаем $\beta(S\setminus \{s\})=\beta(\Sigma(s'))=S$. Так как $s\in S$
    произвольная предельная точка, то $S$ равномерно компактно.
\end{proof}

В некоторых случаях свойство равномерной компактности зависит от выбранных
аксиом теории множеств. Обозначим через \textsc{ZFC} стандартную теорию множеств
Цермело-Френкеля с добавленной аксиомой выбора. Через \textsc{CH}$_n$ мы
обозначим аксиому о том, что мощность континуума равняется $n$-му несчетному
кардиналу. Наконец, \textsc{MA} будет обозначать аксиому Мартина
(см.~\cite{KunSetThIndepPrf}). С одной стороны, $\mathbb{N}^*$ не равномерно
компактно в \textsc{ZFC + CH$_1$}~\cite{FinGillExtContFuncbN}. С другой стороны,
утверждение, что $\mathbb{N}^*$ равномерно компактно совместно с \textsc{ZFC +
    MA + CH$_2$}~\cite{DouKunMillCStarEmbdDenPropSbspStoneRemN}.

Мы готовы сформулировать основной результат статьи.

\begin{theorem}\label{RelInjCSModCSImplUnifCompS} Пусть $S$ --- локально
    компактное хаусдорфово пространство. Если $C_0(S)$-модуль $C_0(S)$
    относительно инъективен, то $S$ равномерно компактно.
\end{theorem}
\begin{proof} Так как $C_0(S)$-модуль $C_0(S)$ относительно инъективен, то
    из~\cite[следствие 2.2.8 (i)]{RamsHomPropSemgroupAlg} мы знаем, что $C_0(S)$
    обладает левой единицей. Таким образом, $\chi_S\in C_0(S)$, значит $S$
    компактно. Пусть $s$ --- предельная точка в $S$ и
    $S_\circ=S\setminus \{s\}$. Так как $\alpha S_\circ=S$, мы видим,
    что $C(\alpha S_\circ)$-модуль $C(\alpha S_\circ)$ относительно инъективен.
    По предложению~\ref{RelInjCaSModCaSImplSAlmComp} пространство $S_\circ$
    почти компактно. Так как предельная точка $s\in S$ выбрана произвольно, то
    пространство $S$ равномерно компактно.
\end{proof}

\begin{corollary}\label{CSModCSRelInjImplSHasNoConvSeq} Пусть $S$ --- компактное
    метризуемое пространство. Если $C(S)$-модуль $C(S)$ относительно инъективен,
    то $S$ конечно.
\end{corollary}
\begin{proof}
    Результат непосредственно следует из
    теоремы~\ref{RelInjCSModCSImplUnifCompS} и
    следствия~\ref{MesSpUnifCompIffFin}.
\end{proof}

На данный момент все известные примеры локально компактных пространств $S$, для
которых $C_0(S)$-модуль $C_0(S)$ относительно инъективен, являются экстремально
несвязными. Было бы интересно построить примеры, не являющиеся экстремально
несвязными, если таковые вообще есть. Первый кандидат --- это пространство
$\mathbb{N}^*$. Оно не является экстремально несвязным~\cite[пример
    6.2.31]{EngkingGenTop}, но оно равномерно компактно при некоторых
теоретико-множественных предположениях. Однако, следует помнить, что
пространство $C(\mathbb{N}^*)$ не является инъективным банаховым
пространством~\cite[следствие 2]{AmirProjContFuncSp}. Благодаря
предложению~\ref{UcountProdCompSpIsUnifComp}, есть еще один возможный кандидат
--- это несчетная степень дискретного пространства $\{0, 1\}$.

%-------------------------------------------------------------------------------
%   A module version of Sobczyk theorem
%-------------------------------------------------------------------------------

\section{Теорема Собчика для банаховых модулей}\label{SectionExamples}

В классической теории все бесконечномерные инъективные банаховы пространства
несепарабельны, поскольку содержат копию
$\ell_\infty(\mathbb{N})$~\cite[следствие 1.1.4]{RosOnRelDisjFamOfMeas}. Собчик
показал, что пространство $c_0$ инъективно, но в категории сепарабельных
банаховых пространств~\cite[теорема 5]{SobProjmOnc0}. Позже Зиппин
доказал~\cite{ZipSepExtProbm}, что все пространства инъективные в категории
сепарабельных банаховых пространств изоморфны $c_0$.  Мы докажем, что
$\ell_\infty(\Lambda)$-модуль $c_0(\Lambda)$ относительно инъективен для любого
множества $\Lambda$. Заметим, что по теореме~\ref{RelInjCSModCSImplUnifCompS}
банахов $c_0(\Lambda)$-модуль $c_0(\Lambda)$ не является относительно
инъективным для бесконечного $\Lambda$.

Теперь нам нужно напомнить несколько понятий из теории банаховых пространств.
Ограниченный линейный оператор $T$ называется \textit{слабо компактным}, если он
переводит ограниченные множества в относительно слабо компактные. Ограниченный
линейный оператор $T$ называется \textit{вполне непрерывным}, если он переводит
слабо сходящиеся последовательности в последовательности, сходящиеся по норме.

Банахово пространство $E$ называется \textit{пространством Гротендика}, если
каждая слабо$^*$ сходящаяся последовательность в $E^*$ сходится слабо. Очевидно
все рефлексивные пространства суть пространства Гротендика. Банахово
пространство $E$ называется \textit{слабо компактно порожденным}, если
существует слабо компактное множество $K\subset E$, линейная оболочка которого
плотна в $E$. Типичные примеры слабо компактно порожденных пространств --- это
рефлексивные и сепарабельные пространства~\cite[параграф 13.1]{FabHabBanSpTh}.
Наконец, мы будем говорить, что банахово пространство $E$ имеет \textit{свойство
    Данфорда-Петтиса}, если для любой слабо сходящейся к нулю последовательности
${(f_n)}_{n\in\mathbb{N}}\subset E^*$ и любой слабо сходящейся к нулю
последовательности ${(x_n)}_{n\in\mathbb{N}}\subset E$ выполнено
$\lim_{n\to\infty} f_n(x_n)=0$. Для любого компактного пространства $S$ банахово
пространство $C(S)$ имеет свойство Данфорда-Петтиса~\cite{DunfPetLinOpSumFunc}.

\begin{proposition}\label{OpLInfc0CompContWeakComp} Любой ограниченный линейный
    оператор $T:\ell_\infty(\Lambda)\to c_0(\Lambda)$ слабо компактен и вполне
    непрерывен.
\end{proposition}
\begin{proof} Заметим, что пространство $\ell_\infty(\Lambda)$ изометрически
    изоморфно $C(\beta\Lambda)$. Так как $\beta\Lambda$ является стоуновым
    пространством, то $\ell_\infty(\Lambda)$ --- пространство
    Гротендика~\cite[теорема 9, стр. 168]{GrothApplFabilCompCK}. Пространство
    $c_0(\Lambda)$ слабо компактно порождено~\cite[параграф 13.1 пример
        (iii)]{FabHabBanSpTh}. Тогда оператор $T$ слабо компактен~\cite[пример
        13.33]{FabHabBanSpTh}. Снова, поскольку $\ell_\infty(\Lambda)$ есть
    $C(S)$-пространство для $S=\beta\Lambda$, то $\ell_\infty(\Lambda)$ обладает
    свойством Данфорда-Петтиса~\cite[теорема 13.43]{FabHabBanSpTh}. Таким
    образом, всякий слабо компактный оператор с областью определения
    $\ell_\infty(\Lambda)$ вполне непрерывен~\cite[предложение
        13.42]{FabHabBanSpTh}.
\end{proof}

\begin{proposition}\label{FrechFiltConvCharac} Пусть $\Lambda$ --- бесконечное
    множество, $x:\Lambda\to\mathbb{C}$ --- функция такая, что
    $\lim_{n\to\infty} x(\lambda_n)=0$ для всех последовательностей
    ${(\lambda_n)}_{n\in\mathbb{N}}$ различных элементов в $\Lambda$. Тогда
    $\lim_{\mathcal{F}_{\Lambda}}x(\lambda)=0$.
\end{proposition}
\begin{proof} Допустим, это не так. Тогда найдется $\epsilon > 0$ такое, что для
    любого $L\in\mathcal{F}_{\Lambda}$ существует $\lambda\in L$ со свойством
    $|x(\lambda)|>\epsilon$. По индукции мы можем построить последовательность
    ${(\lambda_k)}_{k\in\mathbb{N}}$ различных элементов в $\Lambda$ такую, что
    $|x(\lambda_n)|\geq \epsilon$. Следовательно, $\lim_{k\to\infty}
        x(\lambda_k)\neq 0$. Противоречие.
\end{proof}

\begin{proposition}\label{OpLInfc0DiagConv0} Пусть $\Lambda$ --- бесконечное
    множество. Тогда для любого ограниченного линейного оператора
    $T:\ell_\infty(\Lambda)\to c_0(\Lambda)$ выполнено
    $\lim_{\mathcal{F}_{\Lambda}}T(\chi_{ \{\lambda \}})(\lambda)=0$.
\end{proposition}
\begin{proof} Рассмотрим произвольную последовательность
    ${(\lambda_n)}_{n\in\mathbb{N}}$ различных элементов $\Lambda$. Тогда
    ${(\chi_{\{\lambda_n\}})}_{n\in\mathbb{N}}$ слабо сходится к 0 в
    $c_0(\Lambda)$, и тем более в $\ell_\infty(\Lambda)$. По
    предложению~\ref{OpLInfc0CompContWeakComp} оператор $T$ вполне непрерывен,
    поэтому $T(\chi_{\{\lambda_n\}})$ сходится к 0 по норме. В частности,
    $\lim_{n\to\infty} T(\chi_{\{\lambda_n\}})(\lambda_n)=0$. Теперь из
    предложения~\ref{FrechFiltConvCharac} мы получаем нужное равенство.
\end{proof}

\begin{theorem}\label{RelInjLInfModc0} Для любого множества $\Lambda$ правый
    $\ell_\infty(\Lambda)$-модуль $c_0(\Lambda)$ относительно инъективен.
\end{theorem}
\begin{proof} Допустим, что множество $\Lambda$ бесконечно. Из~\cite[предложение
        IV.1.39]{HelHomolBanTopAlg} следует, что достаточно предъявить морфизм
    правых $\ell_\infty(\Lambda)$-модулей правый обратный к
    $\rho:c_0(\Lambda)\to\mathcal{B}(\ell_\infty(\Lambda),
        c_0(\Lambda)):x\mapsto(a\mapsto x\cdot a)$. Он действительно существует.
    Рассмотрим линейный оператор $\tau:\mathcal{B}(\ell_\infty(\Lambda),
        c_0(\Lambda))\to c_0(\Lambda): T\mapsto(\lambda\to T(\chi_{ \{\lambda
            \}})(\lambda))$. По предложению~\ref{OpLInfc0DiagConv0} он корректно
    определен. Легко показать, что $\tau$ --- сжимающий морфизм правых
    $\ell_\infty(\Lambda)$-модулей.

    Если множество $\Lambda$ конечно, то оно является стоуновым пространством.
    Тогда $c_0(\Lambda)=\ell_\infty(\Lambda)=C(\Lambda)$ и по
    предложению~\ref{SStonImplRelInjCSModCS} банахов
    $\ell_\infty(\Lambda)$-модуль $c_0(\Lambda)$ относительно инъективен.
\end{proof}

%-------------------------------------------------------------------------------
%   Funding
%-------------------------------------------------------------------------------

\section{Финансирование}\label{SectionFunding} Работа выполнена при поддержке
Российского фонда фундаментальных исследований (грант №19--01--00447).

%-------------------------------------------------------------------------------
%   Bibliography
%-------------------------------------------------------------------------------

\begin{thebibliography}{999}
    %
    \bibitem{HahnLinSystInLinSp}
    \textit{Hahn, H.}, {\"U}ber lineare Gleichungssysteme in linearen
    R{\"a}umen., J. Reine Angew. Math., 157, 214--229, 1927.
    %
    %
    \bibitem{BanachOnLinFuncI}
    \textit{Banach, S.}, Sur les fonctionnelles lin{\'e}aires, Stud. Math.,
    1(1), 211--216, 1929.
    %
    %
    \bibitem{BanachOnLinFuncII}
    \textit{Banach, S.}, Sur les fonctionnelles lin{\'e}aires II, Stud. Math.,
    1(1), 223--239, 1929.
    %
    %
    \bibitem{BlascIvorConstrInjSpCK}
    \textit{Blasco, J.L. and Ivorra, C.}, On constructing injective spaces of
    type $C(K)$, Indagationes Mathematicae, 9(12), 161--172, 1998.
    %
    %
    \bibitem{EngkingGenTop}
    \textit{Энгелькинг, Р.}, Общая топология, 1986.
    %
    %
    \bibitem{HrusPsdCompTopSp}
    \textit{Hru{\v{s}}{\'a}k, M. and Tamariz-Mascar{\'u}a, A. and Tkachenko,
    M.}, Pseudocompact topological spaces, Springer, 2018.
    %
    %
    %
    \bibitem{HelHomolBanTopAlg}
    \textit{Helemskii, A. Ya.}, The homology of Banach and topological algebras,
    Springer, 1989.
    %
    %
    \bibitem{AvilSepInjBanSp}
    \textit{
        Avil{\'e}s, A. and S{\'a}nchez, F. C. and Castillo, J. MF.\@ and
        Gonz{\'a}lez, M. and Moreno, Y.},
    Separably injective Banach spaces, 17--65, 2016.
    %
    %
    \bibitem{HasumExtPropCompBanSp}
    \textit{Hasumi, M.}, The extension property of complex Banach spaces, Tohoku
    Mathematical Journal, Second Series, 10(2), 135--142, 1958.
    %
    %
    \bibitem{BerbBaerStRng}
    \textit{Berberian, S. K.}
    Baer $^*$-rings, 195, 2010.
    %
    %
    \bibitem{TakHanBanThAndJordDecomOfModMap}
    \textit{Takesaki, M.}, On the Hahn-Banach type theorem and the Jordan
    decomposition of module linear mapping over some operator algebras, Kodai
    Mathematical Seminar Reports, 12, 1--10, 1960.
    %
    %
    \bibitem{StoneBanStAppBoolRngToTop}
    \textit{Stone, M. H.}, Applications of the theory of Boolean rings to
    general topology, Transactions of the American Mathematical Society, 41(3),
    375--481, 1937.
    %
    %
    \bibitem{KunSetThIndepPrf}
    \textit{Kunen, K.}, Set theory: an introduction to independence proofs,
    Elsevier, 2014.
    %
    %
    \bibitem{FinGillExtContFuncbN}
    \textit{Fine, N. J. and Gillman, L.}
    Extension of continuous functions in $\beta {\mathbf{N}}$ Bull. Amer. Math.
    Soc., 66, 376--381, 1960.
    %
    %
    \bibitem{DouKunMillCStarEmbdDenPropSbspStoneRemN}
    \textit{van Douwen, E. and  Kunen,  K. and van Mill, J.}
    There can be $C^*$-embedded dense proper subspaces in $\beta\omega-\omega$,
    Proc. Amer. Math. Soc., 105(2), 462–-470, 1989.
    %
    %
    \bibitem{RamsHomPropSemgroupAlg}
    \textit{Ramsden, P.}, Homological properties of semigroup algebras, The
    University of Leeds, 2009.
    %
    %
    \bibitem{AmirProjContFuncSp}
    \textit{Amir, D.}, Projections onto continuous function spaces, Proceedings
    of the American Mathematical Society, 15(3), 396--402, 1964.
    %
    %
    \bibitem{RosOnRelDisjFamOfMeas}
    \textit{Rosenthal, H.}, On relatively disjoint families of measures, with
    some applications to Banach space theory Stud. Math., 37(1), 13--36, 1970.
    %
    %
    \bibitem{SobProjmOnc0}
    \textit{Sobczyk, A.}, Projection of the space ($m$) on its subspace ($c_0$),
    Bulletin of the American Mathematical Society, 47(12), 938--947, 1941.
    %
    %
    \bibitem{ZipSepExtProbm}
    \textit{Zippin, M.}, The separable extension problem, Israel Journal of
    Mathematics, 26(3--4), 372--387, 1977.
    %
    %
    \bibitem{FabHabBanSpTh}
    \textit{Fabian, M. and Habala, P. and Hajek, P. and Montesinos, V.
        and Zizler, V.}
    Banach space theory. The basis for linear and non-linear analysis. Springer,
    2011.
    %
    %
    \bibitem{GrothApplFabilCompCK}
    \textit{{Grothendieck, A.}}, Sur les applications lin{\'e}aires faiblement
    compactes d'espaces du type $C(K)$, Canadian Journal of Mathematics, 5,
    129--173, 1953.
    %
    %
    \bibitem{DunfPetLinOpSumFunc}
    \textit{Dunford, N. and Pettis, B. J.}, Linear operations on summable
    functions, Transactions of the American Mathematical Society, 47(3),
    323--392, 1940.
    %
\end{thebibliography}

Норберт Немеш, Факультет механики и математики, Московский государственный
университет им. М. В. Ломоносова, Москва 119991 Россия

\textit{E-mail:} nemeshnorbert@yandex.ru


\end{document}  % chktex 17