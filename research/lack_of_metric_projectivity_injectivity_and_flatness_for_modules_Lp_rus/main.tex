% chktex-file 35
\documentclass[12pt]{article}
\usepackage[left=2cm,right=2cm,top=2cm,bottom=2cm,bindingoffset=0cm]{geometry}
\usepackage{amssymb}
\usepackage{amsmath}
\usepackage{amsthm}
\usepackage{mathrsfs}
\usepackage{enumerate}
\usepackage[T1,T2A]{fontenc}
\usepackage[utf8]{inputenc}
\usepackage[russian]{babel}
\usepackage[matrix,arrow,curve]{xy}
\usepackage[colorlinks=true, urlcolor=blue, linkcolor=blue, citecolor=blue,
    pdfborder={0 0 0}]{hyperref}
\usepackage{enumitem}

%------------------------------------------------------------------------------
\newtheorem{theorem}{Теорема}[section]
\newtheorem{lemma}[theorem]{Лемма}
\newtheorem{proposition}[theorem]{Предложение}
\newtheorem{remark}[theorem]{Замечание}
\newtheorem{corollary}[theorem]{Следствие}
\newtheorem{definition}[theorem]{Определение}
\newtheorem{example}[theorem]{Пример}

\newcommand{\projtens}{\mathbin{\widehat{\otimes}}}
\newcommand{\convol}{\ast}
\newcommand{\projmodtens}[1]{\mathbin{\widehat{\otimes}}_{#1}}
\newcommand{\isom}[1]{\mathop{\mathbin{\cong}}\limits_{#1}}
%------------------------------------------------------------------------------

\begin{document}

\begin{center}
    \Large \textbf{Отсутствие метрической проективности, инъективности и 
    плоскости для модулей $L_p$}\\[0.5cm]
    \small {Н. Т. Немеш}\\[0.5cm]
\end{center}

\thispagestyle{empty}

\medskip
\textbf{Аннотация:} В данной статье мы докажем, что для хаусдорфова 
локально компактного пространства $S$ и разложимой борелевской меры $\mu$ 
метрическая проективность, инъективность или 
плоскость $C_0(S)$-модуля $L_p(S,\mu)$ влечет чистую атомарность меры $\mu$, 
причем количество атомов не превосходит $1$.
\medskip

\textbf{Ключевые слова:} метрическая проективность, метрическая инъективность, 
метрическая плоскость, $L_p$-пространство.

\bigskip

%-------------------------------------------------------------------------------
%  Introduction
%-------------------------------------------------------------------------------

\section{Введение}\label{SctnIntro}

Эта статья завершает исследование автора гомологических свойств модулей $L_p$. 
В работе \cite{NemRelProjModLp} было показано, что модули $L_p$ относительно 
проективны для небольшого класса пространств с мерой, а именно для чисто 
атомарных пространств с мерой, причем атомы являются изолированными точками. Цель 
данной статьи --- решить ту же задачу для метрической проективности, 
инъективности и плоскости. Ожидалось, что для метрической теории класс 
пространств с мерой должен быть еще меньше. Как показано в этой статье, этот 
класс включает в себя только чисто атомарные пространства с мерой с не более 
чем одним атомом. Можно сказать, что модули $L_p$ почти 
никогда не являются метрически проективными, инъективными или плоскими.

Прежде чем перейти к содержанию статьи, мы дадим несколько определений. 
Для любого натурального числа $n\in\mathbb{N}$ мы обозначаем множество первых 
$n$ натуральных чисел как $\mathbb{N}_n$. Пусть $M$ --- подмножество множества 
$N$, тогда $\chi_M$ обозначает индикаторную функцию $M$. Символ $1_N$ обозначает 
тождественное отображение на $N$. Если $k,l\in N$, то $\delta_{k}^{l}$ обозначает 
их символ Кронекера.

Все банаховы пространства, обсуждаемые в этой статье, рассматриваются над полем
комплексных чисел. Мы будем активно использовать следующую конструкцию из 
теории банаховых пространств: для данного семейства банаховых 
пространств $\{E_\lambda: \lambda\in\Lambda\}$, 
через $\bigoplus_p\{E_\lambda: \lambda\in\Lambda\}$ мы будем обозначать 
их $\ell_p$-сумму (см. [\cite{HelLectAndExOnFuncAn}, предложение 1.1.7]). 
Аналогично, для семейства линейных операторов $T_\lambda:E_\lambda\to F_\lambda$, 
где $\lambda\in\Lambda$, их $\ell_p$-сумма обозначается 
как $\bigoplus_p\{T_\lambda:\lambda\in\Lambda\}$ 
(см. [\cite{HelLectAndExOnFuncAn}, предложение 1.1.7]). Через $E\projtens F$ мы 
будем обозначать проективное тензорное произведение 
банаховых пространств $E$ и $F$ [\cite{HelLectAndExOnFuncAn}, теорема 2.7.4].

Пусть $T:E\to F$ --- ограниченный линейный оператор между банаховыми 
пространствами. Далее мы дадим количественную версию определения 
вложения с замкнутым образом: если существует константа $c>0$ такая, 
что $c\Vert T(x)\Vert\geq \Vert x\Vert$ для всех $x\in E$, тогда оператор $T$ 
называется $c$-топологически инъективным. Аналогично, 
количественное определение открытого отображения звучит так: если существует 
константа $c>0$ такая, что для любого $y\in F$ мы можем найти $x\in E$ такой, 
что $T(x)=y$ и $c\Vert y\Vert\geq \Vert x\Vert$, то отображение $T$ 
называется $c$-топологически сюръективным. Наконец, линейный или билинейный
оператор называется сжимающим, если его норма не превышает $1$.

Пусть $A$ --- банахова алгебра. Мы будем работать как с левыми, так и с 
правыми банаховыми $A$-модулями, предполагая, что у всех модулей сжимающий 
билинейный оператор внешнего умножения. Пусть 
$X$ и $Y$ --- два банаховых $A$-модуля, тогда отображение $\phi:X\to Y$ называется 
$A$-морфизмом, если оно является непрерывным $A$-модульным отображением. Все 
левые банаховы $A$-модули и их $A$-морфизмы образуют категорию, обозначаемую 
как $A-\mathbf{mod}$. Аналогично, можно определить категорию $\mathbf{mod}-A$, 
состоящую из правых $A$-модулей. Наконец, $X\projmodtens{A} Y$ обозначает 
проективное тензорное произведение левого $A$-модуля $X$ и 
правого $A$-модуля $Y$ (см. [\cite{HelBanLocConvAlg}, определение VI.3.18]).

Теперь мы переходим к определениям метрической проективности, инъективности 
и плоскости. Первая работа по этой теме была опубликована в 1978 году 
Гравеном \cite{GravInjProjBanMod}. Позже эквивалентные определения были даны 
Уайтом \cite{WhiteInjmoduAlg} и Хелемски \cite{HelMetrFrQMod,HelMetrFlatNorMod}.

Левый банахов $A$-модуль $P$ называется метрически проективным, если для 
любого $c$-топологически сюръективного $A$-морфизма $\xi:X\to Y$ и любого 
$A$-морфизма $\phi:P\to Y$, существует $A$-морфизм $\psi:P\to X$ такой, что 
$\Vert\psi\Vert\leq c$ и диаграмма
$$
    \xymatrix{
    & {X} \ar[d]^{\xi}\\
    {P} \ar@{-->}[ur]^{\psi} \ar[r]^{\phi} &{Y}}
$$
коммутативна. Исходное определение было несколько 
иным [\cite{GravInjProjBanMod}, определение 2.4], но оно все еще 
эквивалентно приведенному выше. Простейшим примером метрически проективного 
$A$-модуля является сама алгебра $A$, при условии, что она унитальна 
[\cite{GravInjProjBanMod}, теорема 2.5].

Правый банахов $A$-модуль $J$ называется метрически инъективным, если для 
любого $c$-топологически инъективного $A$-морфизма $\xi:Y\to X$ и 
любого $A$-морфизма $\phi:Y\to J$, существует $A$-морфизм $\psi:X\to J$ такой, 
что $\Vert\psi\Vert\leq c$ и диаграмма
\[
    \xymatrix{
    & {X} \ar@{-->}[dl]_{\psi} \\
    {J} &{Y} \ar[l]_{\phi} \ar[u]_{\xi}}
\]
коммутативна. Наше определение эквивалентно исходному 
[\cite{GravInjProjBanMod}, определение 3.1]. Следует напомнить, 
что $P^*$ является метрически инъективным $A$-модулем, когда $P$ метрически 
проективен [\cite{GravInjProjBanMod}, theorem 3.2]. Следовательно, если алгебра 
$A$ унитальна, то правый банахов $A$-модуль $A^*$ метрически инъективен.

Левый $A$-модуль $F$ называется метрически плоским, если для каждого 
$c$-топологически инъективного $A$-морфизма $\xi:X\to Y$ правых $A$-модулей, 
оператор $\xi\projmodtens{A} 1_F:X\projmodtens{A} F\to Y\projmodtens{A} F$ 
является $c$-топологически инъективным. Это определение было неявно дано в 
теореме [\cite{GravInjProjBanMod}, теорема 3.10]. Из этой теоремы и вышеприведенных 
замечаний мы заключаем, что любой метрически проективный модуль также 
метрически плоский. В частности, унитальная банахова алгебра $A$ является 
метрически плоским $A$-модулем.

%-------------------------------------------------------------------------------
%	Metric injectivity of linftyn modules lpn
%-------------------------------------------------------------------------------

\section{Метрическая инъективность конечномерного \\
\texorpdfstring{$\ell_\infty(\Lambda)$}{linftyLmbd}-модуля 
\texorpdfstring{$\ell_p(\Lambda)$}{lpLmbd}}
\label{MetrInjlinftynlpn}

Пусть $\Lambda$ --- произвольное индексное множество 
и $1\leq p\leq +\infty$, тогда через $\ell_p(\Lambda)$ мы будем обозначать 
стандартное пространство $\ell_p$. Его норма обозначается $\Vert\cdot\Vert_p$, а 
естественный базис $(e_\lambda)_{\lambda\in\Lambda}$. 
Для $1\leq p<+\infty$ мы часто будем использовать отождествление
$\ell_p(\Lambda)^*=\ell_{p^*}(\Lambda)$, где $p^*=p/(p-1)$. 
По соглашению, $1/0=+\infty$, поэтому $1^*=+\infty$. 
Пространство $\ell_p(\Lambda)$ можно рассматривать как левый, так и правый 
банахов модуль над банаховой алгеброй $\ell_\infty(\Lambda)$. В этом разделе 
мы покажем, что для конечного $\Lambda$ 
правый $\ell_\infty(\Lambda)$-модуль $\ell_p(\Lambda)$ метрически инъективен 
только если $\Lambda$ содержит не более одного элемента.

\begin{definition}\label{StdEmbd} 
    Пусть $\Lambda$ --- произвольное множество, $1<p<+\infty$ 
    и $\mathcal{F}$ --- ограниченное подмножество $\ell_{p^*}(\Lambda)$. 
    Определим линейный оператор
    \[
        \xi_{\mathcal{F}}: 
        \ell_p(\Lambda)\to\bigoplus_\infty\{\ell_1(\Lambda):f\in\mathcal{F}\},\,
        x \mapsto \bigoplus_\infty\{ x\cdot f: f\in\mathcal{F}\}.
    \]
\end{definition}

Очевидно, $\xi_{\mathcal{F}}$ --- это $\ell_\infty(\Lambda)$-морфизм с нормой 
$\sup\{\Vert f\Vert_{p^*}: f\in\mathcal{F}\}$.

\begin{definition}\label{StdEmbdCoercv}
    Пусть $\Lambda$ --- произвольное множество, $1<p<+\infty$ 
    и $\mathcal{F}$ --- ограниченное подмножество $\ell_{p^*}(\Lambda)$. 
    Определим константу обратимости для оператора $\xi_{\mathcal{F}}$ как
    \[
        \gamma_{\mathcal{F}}=\sup\{
            \Vert x\Vert_p: 
            x\in\ell_p^n,\,\, \Vert \xi_{\mathcal{F}}(x)\Vert\leq 1
        \}.
    \]
\end{definition}

Заметим, что $\xi_{\mathcal{F}}$ $\gamma_{\mathcal{F}}$-топологически инъективен 
только если $\gamma_{\mathcal{F}}$ конечно.

\begin{proposition}\label{LinfnMorphlpntolqnCharac}
    Пусть $\Lambda$ --- произвольное множество, $1\leq p,q\leq +\infty$, 
    и задан $\ell_\infty(\Lambda)$-морфизм правых модулей 
    $\phi:\ell_p(\Lambda)\to \ell_{q}(\Lambda)$. Тогда существует 
    вектор $\eta\in\ell_\infty(\Lambda)$ такой, что $\phi(x)=\eta\cdot x$ 
    для всех $x\in \ell_p(\Lambda)$.
\end{proposition}
\begin{proof}
    Обозначим $\eta_\lambda=\phi(e_\lambda)_\lambda$, где $\lambda\in\Lambda$. 
    Для любого $x\in\ell_p(\Lambda)$ и $\lambda\in\Lambda$, мы имеем
    \[
        \phi(x)_\lambda
        =(\phi(x)\cdot e_\lambda)_\lambda
        =\phi(x\cdot e_\lambda)_\lambda
        =\phi(x_\lambda e_\lambda)_\lambda
        =x_\lambda\phi(e_\lambda)_\lambda
        =x_\lambda\eta_\lambda
        =(\eta\cdot x)_\lambda.
    \]
    Следовательно, $\phi(x)=\eta\cdot x$. По 
    построению $\Vert\eta\Vert_\infty\leq\Vert\phi\Vert$, 
    так что $\eta\in\ell_\infty(\Lambda)$.
\end{proof}

\begin{proposition}\label{ExtMorphSuml1ntlpnCharac}
    Пусть $\Lambda$ - множество, $1<p<+\infty$ 
    и $\mathcal{F}\subset \ell_{p^*}(\Lambda)$ конечное множество. Тогда для любого 
    морфизма правых $\ell_\infty(\Lambda)$-модулей
    $\psi:\bigoplus_\infty\{\ell_1(\Lambda):f\in\mathcal{F}\}\to\ell_p(\Lambda)$ 
    существует семейство векторов $\eta\in\ell_\infty(\Lambda)^\mathcal{F}$ 
    таких, что
    \[
        \psi(t)=\sum_{f\in\mathcal{F}} \eta_f \cdot t_f
    \]
    для всех $t\in \bigoplus_\infty\{ \ell_1(\Lambda):f\in\mathcal{F}\}$.
\end{proposition}
\begin{proof}
    Для каждого $f\in\mathcal{F}$ мы определяем естественное вложение
    $
        \operatorname{in}_f:
        \ell_1(\Lambda)\to\bigoplus_\infty\{\ell_1(\Lambda):f\in\mathcal{F}\},
    $
    которое является морфизмом правых $\ell_\infty(\Lambda)$-модулей. Далее мы 
    определим $\ell_\infty(\Lambda)$-морфизм 
    $\psi_f=\psi\circ \operatorname{in}_f$. 
    По предложению \ref{LinfnMorphlpntolqnCharac} существует 
    вектор $\eta_f\in\ell_\infty(\Lambda)$ такой, что $\psi_f(x)=\eta_f\cdot x$ 
    для всех $x\in\ell_1(\Lambda)$. Поскольку $\mathcal{F}$ конечно, то для 
    всех $t\in \bigoplus_\infty\{ \ell_1(\Lambda) : f\in \mathcal{F}\}$ 
    выполнено
    \[
        \psi(t)
        =\psi\left(\bigoplus_\infty\{ t_f : f\in\mathcal{F}\}\right)
        =\psi\left(\sum_{f\in\mathcal{F}} \operatorname{in}_f(t_f)\right)
        =\sum_{f\in\mathcal{F}}\psi_f(t_f)
        =\sum_{f\in\mathcal{F}} \eta_f\cdot t_f.
    \]
\end{proof}

\begin{definition}\label{ParamExtMorph}
    Пусть $\Lambda$ --- произвольное множество, $1<p<+\infty$ 
    и $\mathcal{F}\subset \ell_{p^*}(\Lambda)$ --- 
    конечное множество. Для заданного семейства 
    $\eta\in \ell_\infty(\Lambda)^\mathcal{F}$ мы определяем линейный оператор
    \[
        \psi_{\eta}:
        \bigoplus_\infty\{\ell_1(\Lambda):f\in\mathcal{F}\}\to\ell_p(\Lambda),\,
        t\mapsto\sum_{f\in\mathcal{F}} \eta_f\cdot t_f.
    \]
\end{definition}

\begin{definition}\label{ExtMorphs}
    Пусть $\Lambda$ --- произвольное множество, $1<p<+\infty$ 
    и $\mathcal{F}\subset\ell_{p^*}(\Lambda)$ --- 
    конечное множество, тогда положим по определению
    \[
        \mathcal{N}_{\mathcal{F}}=\left\{
            \eta\in \ell_\infty(\Lambda)^{\mathcal{F}} : 
            \sum_{f\in\mathcal{F}} \eta_{f,\lambda}f_\lambda=1,\,\,
            \lambda\in\Lambda
        \right\}.
    \]
\end{definition}

\begin{proposition}\label{StdEmbdLeftInvCharac}
    Пусть $\Lambda$ --- произвольное множество, $1<p<+\infty$ 
    и $\mathcal{F}\subset\ell_{p^*}(\Lambda)$ --- 
    конечное множество. В этом случае $\psi_\eta$ является левым 
    обратным $\ell_\infty(\Lambda)$-морфизмом для $\xi_{\mathcal{F}}$ тогда и 
    только тогда, когда $\eta\in\mathcal{N}_{\mathcal{F}}$.
\end{proposition}
\begin{proof} 
    Предположим, что $\psi_{\eta}$ --- левый обратный морфизм 
    для $\xi_{\mathcal{F}}$, тогда для любого $\lambda\in\Lambda$ выполнено
    \[
        1=(e_\lambda)_\lambda
        =\psi_{\eta}(\xi_{\mathcal{F}}(e_\lambda))_\lambda
        =\psi_{\eta}\left(\bigoplus_\infty\{
            e_\lambda\cdot f: f\in\mathcal{F}
        \}\right)_\lambda
        =\left(
            \sum_{f\in\mathcal{F}} \eta_f\cdot e_\lambda\cdot f
        \right)_\lambda
        =\sum_{f\in\mathcal{F}} \eta_{f,\lambda}f_\lambda.
    \]
    Таким образом, $\eta\in\mathcal{N}_{\mathcal{F}}$.
    Обратно, пусть $\eta\in\mathcal{N}_{\mathcal{F}}$, тогда для 
    любого $x\in\ell_p(\Lambda)$ выполнено
    \[
        \psi_\eta(\xi_{\mathcal{F}}(x))
        =\sum_{f\in\mathcal{F}}\eta_f\cdot x\cdot f 
        =\sum_{f\in\mathcal{F}}\sum_{\lambda\in\Lambda} 
            (\eta_f\cdot x\cdot f)_\lambda e_\lambda 
        =\sum_{\lambda\in\Lambda} 
            \left(\sum_{f\in\mathcal{F}}\eta_{f,\lambda}f_\lambda\right) 
            x_\lambda e_\lambda 
        =\sum_{\lambda\in\Lambda} x_\lambda e_\lambda 
        =x. 
    \]
    Следовательно, морфизм $\psi_\eta$ является левым обратным 
    для $\xi_{\mathcal{F}}$.
\end{proof}

\begin{proposition}\label{ExtMorphNorm}
    Пусть $\Lambda$ --- конечное множество, $1<p<+\infty$ 
    и $\mathcal{F}\subset\ell_{p^*}(\Lambda)$ --- конечное множество. 
    Предположим, что $\eta\in\ell_\infty(\Lambda)^\mathcal{F}$, тогда 
    \[
        \Vert \psi_{\eta}\Vert
        =\max\left\{
            \left(\sum_{\lambda\in\Lambda}
                \left|
                    \sum_{f\in\mathcal{F}} |\eta_{f,\lambda}| 
                    \delta_{\lambda}^{d(f)}
                \right|^p
            \right)^{1/p} : 
            d\in\Lambda^\mathcal{F}
        \right\}.
    \]
\end{proposition}
\begin{proof}
    По определению операторной нормы,
    \[
    \begin{aligned}
        \Vert\psi_{\eta}\Vert
        &=\sup\left\{
            \Vert\psi_{\eta}(t)\Vert_p:
            t\in \bigoplus_\infty\{\ell_1(\Lambda):f\in\mathcal{F}\},\,\,
            \Vert t\Vert\leq 1
        \right\} \\
        &=\sup\left\{
            \left \Vert\sum_{f\in\mathcal{F}}\eta_f\cdot t_f\right \Vert_p:
            t_f\in\ell_1(\Lambda),\, f\in\mathcal{F},\,\,
            \max\{\Vert t_f\Vert:f\in\mathcal{F}\}\leq 1
        \right\} \\
        &=\sup\left\{
            \left(\sum_{\lambda\in\Lambda}
                \left|
                    \sum_{f\in\mathcal{F}}\eta_{f,\lambda} t_{f,\lambda}
                \right|^p
            \right)^{1/p}:
            \sum_{\lambda\in\Lambda} |t_{f,\lambda}|\leq 1,\,\, 
            t_{f,\lambda}\in\mathbb{C},\,\, f\in\mathcal{F}, \lambda\in\Lambda
        \right\}.
    \end{aligned}
    \]
    Для каждого $\lambda\in\Lambda$ и $f\in\mathcal{F}$ мы обозначаем 
    $r_{f,\lambda}=|t_{f,\lambda}|$ 
    и $\alpha_{f,\lambda}=\operatorname{arg}(t_{f,\lambda})$. 
    Тогда $t_{f,\lambda}=r_{f,\lambda} e^{i \alpha_{f,\lambda}}$. Таким образом,
    \[
    \begin{aligned}
        \Vert \psi_{\eta}\Vert
        &=\sup\left\{
            \left(\sum_{\lambda\in\Lambda}
                \left|
                    \sum_{f\in\mathcal{F}}
                        \eta_{f,\lambda} r_{f,\lambda} e^{i \alpha_{f,\lambda}}
                \right|^p
            \right)^{1/p}:
            \sum_{\lambda\in\Lambda} r_{f,\lambda}\leq 1,\,\, 
            r_{f,\lambda}\in\mathbb{R}_+,\,\, 
            \alpha_{f,\lambda}\in\mathbb{R},\,\, 
            f\in\mathcal{F},\, \lambda\in\Lambda
        \right\}.
    \end{aligned}
    \]

    Для каждого $\lambda\in\Lambda$ рассмотрим 
    вектора  $a_\lambda=(\eta_{f,\lambda} r_{f,\lambda})_{f\in\mathcal{F}}$ 
    и $b_\lambda=(e^{-i\alpha_{f,\lambda}})_{f\in\mathcal{F}}$  
    в $\ell_2(\mathcal{F})$. По неравенству Коши-Буняковского скалярное 
    произведение $a_\lambda$ и $b_\lambda$ достигает максимального по модулю 
    значения только если $a_\lambda=k b_\lambda$ для некоторого $k\in\mathbb{C}$. 
    Это возможно только если 
    $
        \operatorname{arg}(\eta_{f,\lambda} r_{f,\lambda})
        =
        -\alpha_{f,\lambda}+\operatorname{arg}(k)
    $
    для всех $f\in\mathcal{F}$. Как следствие, для всех $\lambda\in\Lambda$ 
    максимум выражения 
    $
        \left|
            \sum_{f\in\mathcal{F}}
                \eta_{f,\lambda} r_{f,\lambda} e^{i \alpha_{f,\lambda}}
        \right|
    $
    достигается если 
    $
        \alpha_{f,\lambda}
        =
        \operatorname{arg}(k)-\operatorname{arg}(\eta_{f,\lambda})
    $ для всех $f\in\mathcal{F}$. В этом случае
    \[
    \begin{aligned}
        \Vert \psi_{\eta}\Vert
        &=\sup\left\{
            \left(\sum_{\lambda\in\Lambda}
                \left|
                    \sum_{f\in\mathcal{F}}|\eta_{f,\lambda}| r_{f,\lambda}
                \right|^p
            \right)^{1/p}:
            \sum_{\lambda\in\Lambda} r_{f,\lambda}\leq 1,\,\, 
            r_{f,\lambda}\in\mathbb{R}_+,\,\, 
            f\in\mathcal{F},\, \lambda\in\Lambda
        \right\}.
    \end{aligned}
    \]
    Рассмотрим линейные операторы
    $
        \tau_f:\mathbb{R}^\Lambda\to\ell_p(\Lambda): r\mapsto \eta_f\cdot r
    $
    для $f\in\mathcal{F}$. Тогда,
    \[
    \begin{aligned}
        \Vert\psi_{\eta}\Vert
        &=\sup\left\{
            \left \Vert\sum_{f\in\mathcal{F}}\tau_f(r_f)\right \Vert_p:
            \sum_{\lambda\in\Lambda} r_{f,\lambda}\leq 1,\,\, 
            r_{f,\lambda}\in\mathbb{R}_+,\,\, 
            f\in\mathcal{F},\, \lambda\in\Lambda
        \right\}.
    \end{aligned}
    \]
    Поскольку линейные операторы $(\tau_f)_{f\in\mathcal{F}}$ принимают 
    значения в $\ell_p(\Lambda)$, чья норма строго выпукла, то функция
    $
        F:(\mathbb{R}^\Lambda)^\mathcal{F}\to\mathbb{R}_+,\, 
        r\mapsto \left \Vert\sum_{f\in\mathcal{F}} \tau_f(r_f)\right \Vert_p
    $
    строго выпукла. Поскольку множество
    $
        C=\left\{ 
            r\in(\mathbb{R}^\Lambda)^\mathcal{F} :
            \sum_{\lambda\in\Lambda} r_{f,\lambda}\leq 1,\,\, 
            r_f\in\mathbb{R}^\Lambda_+,\,\, 
            f\in\mathcal{F},\, \lambda\in\Lambda
        \right\}
    $
    является выпуклым многогранником в конечномерном пространстве, 
    то $F$ достигает максимума на $\operatorname{ext}(C)$ --- множестве 
    экстремальных точек $C$. Таким образом,
    \[
        \Vert\psi_\eta\Vert=\max\{F(r) : r\in \operatorname{ext}(C)\},
    \]
    Очевидно, что $r\in \operatorname{ext}(C)$ тогда и только тогда, когда $r=0$ или для некоторой 
    функции $d:\mathcal{F}\to\Lambda$ и для всех $\lambda\in\Lambda$, $f\in\mathcal{F}$ 
    выполняется $r_{f,\lambda}=\delta_{\lambda}^{d(f)}$. Таким образом,
    \[
    \begin{aligned}
        \Vert\psi_{\eta}\Vert
        &=\max\left\{
            \left(\sum_{\lambda\in\Lambda}
                \left|
                    \sum_{f\in\mathcal{F}}
                        |\eta_{f,\lambda}| \delta_{\lambda}^{d(f)}
                \right|^p
            \right)^{1/p}:
            d\in\Lambda^\mathcal{F}
        \right\}.
    \end{aligned}
    \]
\end{proof}

\begin{definition}\label{ExtMorphsNormInf}
    Пусть $\Lambda$ --- произвольное множество, $1<p<+\infty$ 
    и $\mathcal{F}\subset\ell_{p^*}(\Lambda)$ --- конечное множество. 
    Тогда положим по определению
    \[
        \nu_{\mathcal{F}}=\inf\{
            \Vert\psi_{\eta}\Vert : \eta\in\mathcal{N}_{\mathcal{F}}
        \}.
    \]
\end{definition}

\begin{definition}\label{SpclFuncFam}
    Пусть $\Lambda$ --- конечное множество и $\kappa\in\mathbb{R}$. 
    Обозначим $f_\lambda=e_\lambda$ 
    и $f_{\star}=\kappa\sum_{\lambda\in\Lambda} e_\lambda$. 
    Теперь положим по определению
    \[
        \mathcal{F}_{\kappa}(\Lambda)
        =\{f_\lambda: \lambda\in\Lambda\}
        \cup
        \{f_\star\}.
    \]
\end{definition}

\begin{remark}\label{StdEmbdSpcl}
    Пусть $\Lambda$ --- конечное множество, $1<p<+\infty$ 
    и $\kappa\in\mathbb{R}$. Тогда мы можем 
    рассматривать $\mathcal{F}_{\kappa}(\Lambda)$ как ограниченное подмножество 
    в $\ell_{p^*}(\Lambda)$. В этом случае для любого $x\in\ell_p(\Lambda)$ 
    имеем 
    \[
        \xi_{\mathcal{F}_{\kappa}(\Lambda)}(x)
        =
        \left(\bigoplus_\infty\{x\cdot e_\lambda:\lambda\in\Lambda\}\right)
            \bigoplus_{\infty} 
        (\kappa x),
        \qquad
        \Vert\xi_{\mathcal{F}_{\kappa}(\Lambda)}(x)\Vert
        =
        \max\{\Vert x\Vert_\infty, \Vert\kappa x\Vert_1\}.
    \]
\end{remark}

\begin{proposition}\label{StdEmbdSpclCoerciv}
    Предположим, что $\Lambda$ --- конечное множество с $n>1$ 
    элементами, $1<p<+\infty$ и $n^{-1}<\kappa<(n-1)^{-1}$. Тогда
    \[
        \gamma_{\mathcal{F}_{\kappa}(\Lambda)}
        =(n-1+(\kappa^{-1}-(n-1))^p)^{1/p}.
    \]
\end{proposition}
\begin{proof}
    По определению константы обратимости,
    \[
    \begin{aligned}
        \gamma_{\mathcal{F}_{\kappa}(\Lambda)}
        &=\sup\{
            \Vert x\Vert_p : 
            x\in\ell_p(\Lambda),\, 
            \Vert \xi_{\mathcal{F}_{\kappa}(\Lambda)}(x)\Vert\leq 1
        \} \\
        &=\sup\left\{
            \left( \sum_{\lambda\in\Lambda} |x_\lambda|^p\right)^{1/p} : 
            x\in\mathbb{C}^\Lambda,\, 
            \max\left\{
                \max\{|x_\lambda|:\lambda\in\Lambda\},
                \kappa\sum_{\lambda\in\Lambda} |x_\lambda|
            \right\}\leq 1
        \right\} \\
        &=\sup\left\{
            \left( \sum_{\lambda\in\Lambda} |t_\lambda|^p\right)^{1/p} : 
            t\in\mathbb{R}^\Lambda,\, 
            \sum_{\lambda\in\Lambda} |t_\lambda|\leq \kappa^{-1},\,
            |t_\lambda|\leq 1,\,\lambda\in\Lambda\,
        \right\}. \\
    \end{aligned}
    \]
    Рассмотрим функцию
    $
        F:
        \mathbb{R}^\Lambda\to\mathbb{R},\, 
        t\mapsto \left(\sum_{\lambda\in\Lambda}|t_\lambda|^p\right)^{1/p}
    $
    и выпуклый многогранник
    $
        C=\left\{ 
            t\in\mathbb{R}^\Lambda : 
            \sum_{\lambda\in\Lambda} |t_\lambda|\leq \kappa^{-1},\,
            |t_\lambda|\leq 1,\,\lambda\in\Lambda\,
        \right\}
    $
    в конечномерном пространстве. Поскольку функция $F$ строго выпукла, то $F$ 
    достигает своего максимума на $\operatorname{ext}(C)$ --- множестве 
    экстремальных точек $C$. Следовательно,
    \[
        \gamma_{\mathcal{F}_{\kappa}(\Lambda)}=\max\{
            F(t):t\in\operatorname{ext}(C)
        \}.
    \]
    Геометрически $C$ --- это $n$-мерный куб $[-1,1]^n$ чьи вершины были 
    отрезаны гиперплоскостями вида $\pm t_1\pm t_2\pm\ldots\pm t_n=\kappa^{-1}$.
    Очевидно, что любая точка $t\in \operatorname{ext}(C)$ имеет все координаты, 
    кроме одной, равные $1$ или $-1$. Поэтому,
    \[
        \operatorname{ext}(C)=\left\{ 
            t\in\mathbb{R}^\Lambda : 
            \exists \lambda'\in\Lambda\quad |t_{\lambda'}|=\kappa^{-1}-(n-1)\,
            \wedge\, 
            \forall \lambda\in\Lambda\setminus\{\lambda'\}\quad |t_\lambda|=1
        \right\}.
    \]
    Как следствие,
    $
        \gamma_{\mathcal{F}_{\kappa}(\Lambda)}=(n-1+(\kappa^{-1}-(n-1))^p)^{1/p}.
    $
\end{proof}

\begin{proposition}\label{ExtMorphsNormLwrBnd}
    Предположим, что $\Lambda$ --- конечное множество с $n > 1$ 
    элементами, $1 < p < +\infty$ и $n^{-1} < \kappa < (n-1)^{-1}$. Тогда
    \[
        \nu_{\mathcal{F}_{\kappa}(\Lambda)}
        \geq 
        \kappa^{-1}
        \left(
            \left(\frac{n-1}{\kappa^{-1}-1}\right)^{\frac{p}{p-1}}+n-1
        \right)^{1/p}
        \left(
            \left(\frac{n-1}{\kappa^{-1}-1}\right)^{\frac{1}{p-1}}-1+\kappa^{-1}
        \right)^{-1}.
    \]
\end{proposition}
\begin{proof}
    По построению
    $
        \mathcal{N}_{\mathcal{F}_{\kappa}(\Lambda)}=\{
            \eta\in\ell_\infty(\Lambda)^{\mathcal{F}_{\kappa}(\Lambda)}:
            \eta_{f_\lambda,\lambda}+\kappa \eta_{f_\star, \lambda}=1,\, 
            \lambda\in\Lambda
        \}.
    $
    Для каждого $\lambda\in\Lambda$ рассмотрим функцию
    \[
        d_\lambda:\mathcal{F}_\kappa(\Lambda)\to\Lambda,\,
        f_i\mapsto
        \begin{cases}
            i\quad &\text{если } i\neq \star,\\
            \lambda\quad &\text{если } i=\star.
        \end{cases}
    \]
    Тогда из предложения \ref{ExtMorphNorm}, для 
    любой $\eta\in\ell_\infty(\Lambda)^{\mathcal{F}_{\kappa}(\Lambda)}$ мы 
    получим
    \[
    \begin{aligned}
        \Vert\psi_{\eta}\Vert
        &\geq\max\left\{
            \left(\sum_{\lambda\in\Lambda}
                \left|
                    \sum_{f\in\mathcal{F}_{\kappa}(\Lambda)} 
                        |\eta_{f,\lambda}|\delta_{\lambda}^{d_{\lambda'}(f)}
                \right|^p
            \right)^{1/p}:
            \lambda'\in\Lambda
        \right\} \\
        &=\max\left\{
            \left(
                (|\eta_{f_{\lambda'},\lambda'}|+|\eta_{f_\star,\lambda'}|)^p
                +
                \sum_{\lambda\in\Lambda,\lambda\neq \lambda'} 
                    |\eta_{f_\lambda,\lambda}|^p
            \right)^{1/p}:
            \lambda'\in\Lambda
        \right\}. \\
    \end{aligned}
    \]
    Для любого $\eta\in\mathcal{N}_{\mathcal{F}_{\kappa}(\Lambda)}$ 
    и $\lambda\in\Lambda$ 
    выполнено $\eta_{f_\lambda,\lambda}+\kappa \eta_{f_\star, \lambda}=1$. 
    Следовательно, по обратному неравенству треугольника
    \[
    \begin{aligned}
        \Vert \psi_{\eta}\Vert
        &\geq\max\left\{
            \left(
                (
                    |\eta_{f_{\lambda'},\lambda'}|+
                    |\kappa^{-1}(1-\eta_{f_{\lambda'},\lambda'})|
                )^p
                +
                \sum_{\lambda\in\Lambda,\lambda\neq \lambda'}
                    |\eta_{f_\lambda,\lambda}|^p
            \right)^{1/p}:
            \lambda'\in\Lambda
        \right\} \\
        &\geq\max\left\{
            \left(
                (
                    |\eta_{f_{\lambda'},\lambda'}|+
                    \kappa^{-1}|1-|\eta_{f_{\lambda'},\lambda'}||
                )^p
                +
                \sum_{\lambda\in\Lambda,\lambda\neq \lambda'} 
                    |\eta_{f_\lambda,\lambda}|^p
            \right)^{1/p}:
            \lambda'\in\Lambda
        \right\} \\
    \end{aligned}
    \]
    для любой $\eta\in\mathcal{N}_{\mathcal{F}_{\kappa}(\Lambda)}$. Обозначим
    \[
    \begin{aligned}
        \alpha_i&:\mathbb{R}^n\to\mathbb{R},\,
        t\mapsto \left(
            (|t_i|+\kappa^{-1}|1-|t_i||)^p+\sum_{k=1,k\neq i}^n |t_k|^p
        \right)^{1/p} \quad\text{for }\quad i\in\mathbb{N}_n \\
        \alpha&:\mathbb{R}^n\to\mathbb{R},\,
        t\mapsto\max\{\alpha_i(t):i\in\mathbb{N}_n\}.
    \end{aligned}
    \]
    Тогда для любого нумерации 
    элементов $\Lambda=\{\lambda_1,\ldots,\lambda_n\}$ получаем
    \[
        \nu_{\mathcal{F}_{\kappa}(\Lambda)}
        \geq\inf\{
            \Vert \psi_{\eta}\Vert : 
            \eta\in\mathcal{N}_{\mathcal{F}_{\kappa}(\Lambda)}
        \}
        =\inf\{
            \alpha(
                |\eta_{f_{\lambda_1},\lambda_1}|,
                \ldots,
                |\eta_{f_{\lambda_n},\lambda_n}|
            ) : 
            \eta\in\mathcal{N}_{\mathcal{F}_{\kappa}(\Lambda)}
        \}
        =\inf\{\alpha(t) : t\in\mathbb{R}_+^n\}.
    \]
    Рассмотрим функции
    \[
        F_1:\mathbb{R}_+\to\mathbb{R},\, 
            t\mapsto t, \quad
        F_2:\mathbb{R}_+\to\mathbb{R},\, 
            t\mapsto t+\kappa^{-1}|1-t|, \quad
        F_3:\mathbb{R}^n\to\mathbb{R},\, t\mapsto 
            \left(\sum_{k=1}^n|t_k|^p\right)^{1/p}.
    \]
    Очевидно, что для каждого $i\in\mathbb{N}_n$ функция $\alpha_i$ является 
    композицией функций $F_1$, $F_2$ и $F_3$. Поскольку $F_1$ 
    и $F_2$ --- выпуклые функции, а $F_3$ --- строго выпуклая, то все 
    функции $(\alpha_i)_{i\in\mathbb{N}_n}$ строго выпуклы 
    на $\mathbb{R}_+^n$. Следовательно, будет строго выпуклым и их 
    максимум $\alpha$. Заметим, что функция $\alpha$ непрерывна, строго выпукла, 
    и $\lim_{\Vert t\Vert\to+\infty}\alpha(t)=+\infty$. Следовательно, $\alpha$ 
    имеет единственный глобальный минимум в точке $t_0\in\mathbb{R}_+^n$. Обратим 
    внимание, что $\alpha$ инвариантна относительно перестановки аргументов. 
    Тогда из единственности глобального минимума мы можем заключить, что все 
    координаты точки $t_0$ равны между собой. Как следствие,
    \[
    \begin{aligned}
        \nu_{\mathcal{F}_{\kappa}(\Lambda)}
        &\geq\inf\{\alpha(t) : t\in\mathbb{R}_+^n\} \\
        &=\inf\{\alpha(s,\ldots,s) : s\in\mathbb{R}_+\} \\
        &=\inf\{((s+\kappa^{-1}|1-s|)^p+(n-1)s^p)^{1/p} : s\in\mathbb{R}_+\}.
    \end{aligned}
    \]
    Рассмотрим функцию
    $
        F:\mathbb{R}_+\to\mathbb{R},\,
        s\mapsto ((n-1)s^p+(s+\kappa^{-1}|1-s|)^p)^{1/p}.
    $
    Очевидно, что
    \[
        F(s)=
        \begin{cases}
            ((s+\kappa^{-1}(1-s))^p+(n-1)s^p)^{1/p}
            \quad&\text{если }\quad 0\leq s\leq 1, \\
            \left(
                (\kappa^{-1}+1)^p
                \left(s-\frac{\kappa^{-1}}{\kappa^{-1}+1}\right)^p+
                (n-1)s^p
            \right)^{1/p}
            \quad&\text{если }\quad s>1.
        \end{cases}
    \]
    Поскольку функция $F$ непрерывна и очевидно возрастает на $(1,+\infty)$, 
    то $F$ достигает своего минимума на $[0, 1]$. Найдем стационарные точки 
    $F$ на $[0, 1]$. Для $s\in[0,1]$ имеем
    \[
        F'(s)=
        ((s+\kappa^{-1}(1-s))^p+(n-1)s^p)^{1/p-1}
        ((\kappa^{-1}+(1-\kappa^{-1})s)^{p-1}(1-\kappa^{-1})+(n-1)s^{p-1}).
    \]
    Стационарная точка может быть найдена из уравнения $F'(s)=0$. 
    Решение этого уравнения имеет вид
    \[
        s_0
        =
        \kappa^{-1}
        \left(
            \left(
                \frac{n-1}{\kappa^{-1}-1}
            \right)^{\frac{1}{p-1}}
            -1+\kappa^{-1}
        \right)^{-1}.\\
    \]
    По предположению $n^{-1}<\kappa$, поэтому $\frac{n-1}{\kappa^{-1}-1}<1$ и,
    следовательно, $0<s_0<1$. Поскольку $F$ выпукла, то $s_0$ является точкой 
    минимума на $[0,1]$. Минимум равен
    \[
        F(s_0)
        =\kappa^{-1}
        \left(
            \left(\frac{n-1}{\kappa^{-1}-1}
            \right)^{\frac{p}{p-1}}
            +n-1
        \right)^{1/p}
        \left(
            \left(\frac{n-1}{\kappa^{-1}-1}
            \right)^{\frac{1}{p-1}}
            -1+\kappa^{-1}
        \right)^{-1}.
    \]
    Это дает требуемую нижнюю оценку 
    для $\nu_{\mathcal{F}_{\kappa}(\Lambda)}$.
\end{proof}

\begin{proposition}\label{SpclLyapIneq}
    Пусть $n\in\mathbb{N}$, $r>1$, и $x\in\mathbb{C}^n$. Тогда
    \[
        \Vert x\Vert_r\leq \Vert x\Vert_1^{1/r}\Vert x\Vert_\infty^{1-1/r}.
    \]
    Равенство достигается тогда и только тогда, когда все ненулевые компоненты
    вектора $x$ имеют одинаковые абсолютные значения.
\end{proposition}
\begin{proof}
    Для любого $r>1$ и любого $x\in\mathbb{C}^n$, мы имеем
    \[
        \Vert x\Vert_r
        =\left(\sum_{k=1}^n |x_k|^r \right)^{1/r}
        =\left(\sum_{k=1}^n |x_k| |x_k|^{r-1} \right)^{1/r}
        \leq\left(\sum_{k=1}^n |x_k| \right)^{1/r} 
        \left(\max\limits_{k\in\mathbb{N}_n}|x_k|^{r-1}\right)^{1/r}
        =\Vert x\Vert_1^{1/r}\Vert x\Vert_\infty^{1-1/r}.
    \]
\end{proof}

\begin{proposition}\label{CompStdEmbdCoercvAndExtMorphsNormInf}
    Пусть $\Lambda$ - конечное множество с $n>1$ элементами, $1<p<+\infty$ 
    и $n^{-1}<\kappa<(n-1)^{-1}$. Тогда
    $
    \nu_{\mathcal{F}_{\kappa}(\Lambda)}
    >
    \gamma_{\mathcal{F}_{\kappa}(\Lambda)}.
    $
\end{proposition}
\begin{proof}
    Используя результаты предложений \ref{StdEmbdSpclCoerciv} 
    и \ref{ExtMorphsNormLwrBnd}, достаточно показать, что
    \[
        \kappa^{-1}
        \left(
            \left(\frac{n-1}{\kappa^{-1}-1}
            \right)^{\frac{p}{p-1}}
            +n-1
        \right)^{1/p}
        \left(
            \left(\frac{n-1}{\kappa^{-1}-1}
            \right)^{\frac{1}{p-1}}
            -1+\kappa^{-1}
        \right)^{-1}
        >
        (n-1+(\kappa^{-1}-(n-1))^p)^{1/p}.
    \]
    Сделаем замену $m=n-1\in\mathbb{N}$ и $\rho=\kappa^{-1}\in(m, m+1)$. 
    Тогда последнее неравенство эквивалентно
    \[
        \rho
        \left(
        \left(\frac{m}{\rho-1}
            \right)^{\frac{p}{p-1}}
            +m
        \right)^{1/p}
        \left(
            \left(\frac{m}{\rho-1}
            \right)^{\frac{1}{p-1}}
            -1+\rho
        \right)^{-1}
        >
        (m+(\rho-m)^p)^{1/p}.
    \] 
    После упрощения мы получаем 
    \[
        \frac{m\rho}{\rho-1}
        >
        (m+(\rho-m)^p)^{1/p}
        \left(
            m+\left(\frac{m}{\rho-1}\right)^{p^*}
        \right)^{1/p^*}.
    \]
    Чтобы доказать это неравенство, мы применяем предложение \ref{SpclLyapIneq} к 
    вектору $x=(1,\ldots,1,\rho-m)^T\in\mathbb{C}^{m+1}$ с $r=p$ и к вектору 
    $x=(1,\ldots,1,\frac{m}{\rho-1})\in\mathbb{C}^{m+1}$ с $r=p^*$. 
    Поскольку $m<\rho<m+1$, то компоненты этих векторов не все попарно равны,
    поэтому неравенства строгие:
    \[
    \begin{aligned}
        (m+(\rho-m)^p)^{1/p}
        &<
        (m+(\rho-m))^{1/p} 1^{1-1/p},\\
        \left(m+\left(\frac{m}{\rho-1}\right)^{p^*}\right)^{1/p^*}
        &<
        \left(m+\frac{m}{\rho-1}\right)^{1/p^*}
        \left(\frac{m}{\rho-1}\right)^{1-1/p^*}.
    \end{aligned}
    \]
    Перемножив эти неравенства, мы получаем желаемый результат.
\end{proof}

\begin{proposition}\label{RetrPrblmNoSln}
    Пусть $\Lambda$ --- конечное множество с $n>1$ элементами, $1<p<+\infty$ 
    и $n^{-1}<\kappa<(n-1)^{-1}$. Тогда для 
    любого $\ell_\infty(\Lambda)$-морфизма $\psi$, который является левым 
    обратным к $\xi_{\mathcal{F}_{\kappa}(\Lambda)}$, 
    выполнено $\Vert \psi\Vert>\gamma_{\mathcal{F}_{\kappa}(\Lambda)}$.  
\end{proposition}
\begin{proof}
    Из утверждения \ref{ExtMorphSuml1ntlpnCharac} мы знаем, что существует 
    семейство 
    векторов $\eta\in\ell_\infty(\Lambda)^{\mathcal{F}_{\kappa}(\Lambda)}$ 
    таких, что  $\psi=\psi_{\eta}$. Из определения \ref{ExtMorphsNormInf} 
    следует, 
    что $\Vert \psi_{\eta}\Vert\geq \nu_{\mathcal{F}_{\kappa}(\Lambda)}$. 
    Теперь, из утверждения \ref{CompStdEmbdCoercvAndExtMorphsNormInf} получаем 
    $\Vert\psi\Vert\geq\nu_{\mathcal{F}_{\kappa}(\Lambda)} > 
    \gamma_{\mathcal{F}_{\kappa}(\Lambda)}$.
\end{proof}

\begin{proposition}\label{linftnModlpnIsntMetInjCharac}
    Пусть $\Lambda$ --- конечное множество. Тогда 
    правый $\ell_\infty(\Lambda)$-модуль $\ell_p(\Lambda)$ метрически 
    инъективен тогда и только тогда, когда $\Lambda$ содержит не 
    более $1$ элемента.
\end{proposition}
\begin{proof}
    Предположим, что правый $\ell_\infty(\Lambda)$-модуль $\ell_p(\Lambda)$ 
    метрически инъективен и $\Lambda$ содержит $n$ элементов. Предположим, 
    что $n>1$. Выберем любое вещественное число $\kappa\in(n^{-1},(n-1)^{-1})$. 
    По утверждению \ref{StdEmbdSpclCoerciv}, 
    константа $\gamma_{\mathcal{F}_{\kappa}(\Lambda)}$ конечна, 
    следовательно, оператор $\xi_{\mathcal{F}_{\kappa}(\Lambda)}$
    будет $\gamma_{\mathcal{F}_{\kappa}(\Lambda)}$-топологически инъективным. 
    Из метрической инъективности $\ell_p(\Lambda)$ следует, что существует 
    $\ell_\infty(\Lambda)$-морфизм $\psi$, который является левым обратным к 
    $\xi_{\mathcal{F}_{\kappa}(\Lambda)}$, 
    причем $\Vert\psi\Vert\leq\gamma_{\mathcal{F}_{\kappa}(\Lambda)}$. Это 
    противоречит утверждению \ref{linftnModlpnIsntMetInjCharac}, 
    поэтому $n\leq 1$.

    Теперь предположим, что $\Lambda$ содержит не более 1 элемента. 
    Если $\Lambda$ пусто, то $\ell_p(\Lambda)=\{0\}$. Нулевой модуль всегда 
    инъективен. Если $\Lambda$ содержит один элемент, то $\ell_p(\Lambda)$ 
    изометрически изоморфен $\ell_\infty(\Lambda)^*$ 
    как $\ell_\infty(\Lambda)$-модуль. Сопряженное пространство унитальной 
    алгебры всегда метрически инъективно.
\end{proof}

%-------------------------------------------------------------------------------
%   Предварительные сведения теории меры
%-------------------------------------------------------------------------------

\section{Предварительные сведения по теории меры}
\label{MeasThPrelim}

В этом разделе мы подготовим почву для основной теоремы. Хотя она формулируется 
для борелевских мер на локально компактных пространствах, мы будем доказывать 
все утверждения этого раздела для общих измеримых пространств. Детальное 
изучение общих измеримых пространств можно найти в~\cite{FremMeasTh2}.

Пусть $\Omega$ --- некоторое множество. Под мерой мы понимаем счетно-аддитивную 
функцию множеств со значениями в $[0,+\infty]$, определенную 
на $\sigma$-алгебре $\Sigma$ измеримых подмножеств множества $\Omega$. 
Пара $(\Omega,\mu)$ называется измеримым пространством. 
Измеримое множество $A$ называется атомом, если $\mu(A)>0$ и для любого 
измеримого подмножества $B\subset A$ либо $\mu(B)=0$, 
либо $\mu(A\setminus B)=0$. Мера $\mu$ называется чисто атомарной, если каждое 
измеримое множество с положительной мерой имеет атом. Мера $\mu$ называется 
полуограниченной, если для любого измеримого множества $E$ бесконечной меры 
существует измеримое подмножество $E$ с конечной положительной мерой. 
Семейство $\mathcal{D}$ измеримых подмножеств с конечной мерой называется 
разложением $\Omega$, если для любого измеримого множества $E$ 
выполнено $\mu(E)=\sum_{D\in\mathcal{D}}\mu(E\cap D)$ и множество $F$ измеримо, 
когда $F\cap D$ измеримо для всех $D\in\mathcal{D}$. Наконец, мера $\mu$ 
называется разложимой, если она полуограничена и имеет разложение $\Omega$. 
Большинство мер, встречающихся в функциональном анализе, являются разложимыми.

Мы определим несколько банаховых пространств, построенных по пространствам с 
мерой. Пусть $(\Omega,\mu)$ --- пространство с мерой. Через $B(\Sigma)$ мы 
обозначаем алгебру ограниченных измеримых функций с нормой $\sup$. 
Для $1\leq p\leq +\infty$ через $L_p(\Omega,\mu)$ мы обозначаем банахово 
пространство классов эквивалентности $p$-интегрируемых (или существенно 
ограниченных, если $p=+\infty$) функций на $\Omega$. Элементы $L_p(\Omega,\mu)$ 
обозначаются $[f]$.

\begin{definition}\label{GnrlzdMean}
    Пусть $(\Omega,\mu)$ --- пространство с мерой, $E$ --- измеримое множество 
    конечной положительной меры и $f:\Omega\to\mathbb{C}$ --- измеримая 
    функция. Для любого $r\in\mathbb{R}$ мы определим линейное отображение
    \[
        m_{E,r}(f)=\mu(E)^{\frac{1}{r}-1}\int_E f(\omega)d\mu(\omega).
    \]
    Отметим, что $m_{E,r}(f)=\mu(E)^{1/r}m_{E,\infty}(f)$ и $m_{E,\infty}(f)$ 
    есть ничто иное как среднее значение $f$ на $E$.
\end{definition}

\begin{proposition}\label{GnrlzdMeanProp}
    Пусть $E$ --- подмножество конечной меры пространства с мерой $(\Omega,\mu)$ 
    и $r\in\mathbb{R}$. Тогда для любых измеримых 
    функций $f:\Omega\to\mathbb{C}$, $g:\Omega\to\mathbb{C}$ выполнено:
    \begin{enumerate}[label = (\roman*)]
        \item $m_{E,r}(\chi_E)=\mu(E)^{1/r}$;
        \item $m_{E,r}(f)=m_{E,r}(f\chi_E)$;
        \item Если $E$ --- атом, то $m_{E,\infty}(f)$ --- конечное число;
        \item Если $E$ --- атом, то $f=m_{E,\infty}(f)$ почти всюду на $E$;
        \item Если $E$ --- атом, 
        то $m_{E,\infty}(f)m_{E,\infty}(g)=m_{E,\infty}(f\cdot g)$;
        \item Если $E$ --- атом, 
        то $m_{E,r}(f)m_{E,s}(g)=m_{E,\frac{rs}{r+s}}(f\cdot g)$.
    \end{enumerate}
\end{proposition}
\begin{proof}
    Пункты $(i)$ и $(ii)$ очевидны.

    $(iii)$ Без потери общности предположим, что функция $f$ принимает только 
    вещественные значения. Для каждого $n\in\mathbb{N}$ рассмотрим измеримое 
    множество $A_n=\{\omega\in\Omega:f(\omega)\leq n\}$. Очевидно, 
    $(A_n)_{n\in\mathbb{N}}$ --- это неубывающая последовательность
    и $E=\bigcup_{n\in\mathbb{N}}A_n$, 
    поэтому $\mu(E)=\sup\{\mu(A_n):n\in\mathbb{N}\}$. С другой стороны, для 
    каждого $n\in\mathbb{N}$ множество $A_n$ --- измеримое подмножество 
    атома $E$, поэтому либо $\mu(A_n)=0$, либо $\mu(A_n)=\mu(E)$. 
    Как следствие, $\mu(A_N)=\mu(E)$ для некоторого $N\in\mathbb{N}$. 
    Другими словами, $f\leq N$ почти всюду на $E$.
    Значит, $m_{E,\infty}(f)\leq N<+\infty$. Аналогично можно показать, 
    что $m_{E,\infty}(f)>-\infty$.

    $(iv)$ Без потери общности предположим, что функция $f$ принимает только 
    вещественные значения. Обозначим $k=m_{E,\infty}(f)$. Из пункта $(iii)$ мы 
    знаем, что $k$ конечно. Рассмотрим 
    множество $A_+=\{\omega\in E: f(\omega)>k\}$. Поскольку $A_+$ является 
    измеримым подмножеством атома $E$ конечной меры, либо $\mu(A_+)=0$, 
    либо $\mu(A_+)=\mu(E)>0$. В последнем случае получаем
    \[
        \int_E f(\omega)d\mu(\omega)
        =\int_{A_+}f(\omega)d\mu(\omega)
        >k\mu(A_+)
        =k\mu(E)
        =\int_E f(\omega)d\mu(\omega).
    \]
    Противоречие, так что $\mu(A_+)=0$. Аналогично можно показать, что 
    множество $A_-=\{\omega\in E:f(\omega)<k\}$ также имеет меру ноль. 
    Таким образом, $f=k$ почти всюду на $E$.

    Пункт $(vi)$ непосредственно следует из $(v)$, который, в свою очередь, 
    является простым следствием $(iv)$.
\end{proof}

\begin{proposition}\label{LpOnPurAtomMeasSpRepr}
    Пусть $(\Omega,\mu)$ --- чисто атомарное пространство с мерой. 
    Пусть $\mathcal{A}$ --- разложение $\Omega$ на атомы конечной меры. Тогда 
    для любого $1\leq p<+\infty$ линейные отображения
    \[
        I_p:
        L_p(\Omega,\mu)\to \ell_p(\mathcal{A}),\,
        [f]\mapsto\sum_{A\in\mathcal{A}} m_{A,p}(f) e_A,
        \quad
        J_p:
        \ell_p(\mathcal{A})\to L_p(\Omega,\mu),\,
        x\mapsto\sum_{A\in\mathcal{A}} x_A m_{A,-p}(\chi_A) [\chi_A]\\
    \]
    являются взаимно обратными изометрическими изоморфизмами.
\end{proposition}
\begin{proof} 
    Очевидно, $I_p$ и $J_p$ --- линейные операторы. Поскольку $\mathcal{A}$ --- 
    разложение $\Omega$ на атомы, то
    $
        [f]=\sum_{A\in\mathcal{A}} m_{A,\infty}(f)[\chi_A]
    $
    для любого $[f]\in L_p(\Omega,\mu)$. Используя 
    утверждение \ref{GnrlzdMeanProp} для 
    каждого $[f]\in L_p(\Omega,\mu)$ и $A\in\mathcal{A}$ получаем
    \[
        \int_A |f(\omega)|^pd\mu(\omega)
        =\int_A\left|m_{A,\infty}(f)\right|^pd\mu(\omega)
        =\mu(A)\left|m_{A,\infty}(f)\right|^p
        =\left|m_{A,p}(\chi_A) m_{A,\infty}(f)\right|^p
        =|m_{A,p}(f)|^p.
    \]
    Таким образом, для любого $[f]\in L_p(\Omega,\mu)$ мы имеем
    \[
    \begin{aligned}
        \Vert I_p([f])\Vert
        =\left( \sum_{A\in\mathcal{A}} |m_{A,p}(f)|^p\right)^{1/p} 
        =\left( 
            \sum_{A\in\mathcal{A}} \int_A |f(\omega)|^pd\mu(\omega)
        \right)^{1/p} 
        =\left( \int_{\Omega} |f(\omega)|^pd\mu(\omega)\right)^{1/p} 
        =\Vert [f]\Vert,
    \end{aligned}
    \]
    значит $I_p$ --- изометрия. Заметим, что для 
    любого $x\in\ell_p(\mathcal{A})$ и $A\in\mathcal{A}$ выполнено
    $
        m_{A,p}(J_p(x))
        =m_{A,p}(J_p(x)\chi_A)
        =m_{A,p}(x_A m_{A,-p}(\chi_A)[\chi_A])
        =x_A.
    $
    Следовательно, для любого $x\in\ell_p(\mathcal{A})$, у нас есть равенство
    \[
        I_p(J_p(x))
        =\sum_{A\in\mathcal{A}}m_{A,p}(J_p(x))e_A
        =\sum_{A\in\mathcal{A}}x_A e_A
        =x.
    \]
    Другими словами, $I_p\circ J_p=1_{\ell_p(\mathcal{A})}$. 
    Заметим, что
    $
        I_p\circ(1_{L_p(\Omega,\mu)}-J_p\circ I_p)
        =I_p-I_p\circ J_p\circ I_p
        =I_p-I_p
        =0
    $. 
    Поскольку, оператор $I_p$ изометричен, а значит инъективен, 
    то $1_{L_p(\Omega,\mu)}-J_p\circ I_p=0$, 
    т.е. $J_p\circ I_p=1_{L_p(\Omega,\mu)}$. Таким образом, $J_p$ и $I_p$ ---
    взаимно обратные операторы. Тогда $J_p$ --- изометрия как оператор обратный
    к изометрии $I_p$.
\end{proof}

\begin{proposition}\label{SwtchMorphBtwnAtomMeasSp}
    Пусть $(\Omega,\mu)$ --- чисто атомарное пространство с мерой, 
    и $\mathcal{A}$ --- разложение $\Omega$ на атомы конечной меры. 
    Предположим, что $1\leq p,q<+\infty$, тогда
    \begin{enumerate}[label = (\roman*)]
        \item Если $\Phi:L_p(\Omega,\mu)\to L_q(\Omega,\mu)$ --- 
        $B(\Sigma)$-морфизм, то отображение $I_q\circ \Phi\circ J_p$ 
        является $\ell_\infty(\mathcal{A})$-морфизмом с той же нормой;

        \item Если $\phi:\ell_p(\mathcal{A})\to \ell_q(\mathcal{A})$ --- 
        $\ell_\infty(\mathcal{A})$-морфизм, то 
        отображение $J_q\circ \phi\circ I_p$ является 
        $B(\Sigma)$-морфизмом с той же нормой;
    \end{enumerate} 
\end{proposition}
\begin{proof} 
    $(i)$ Обозначим $\phi=I_q\circ \Phi\circ J_p$, тогда по 
    утверждению \ref{GnrlzdMeanProp} для любого атома $A\in\mathcal{A}$ имеем
    $
        \phi(e_A)
        =I_q(\Phi(J_p(e_A)))
        =I_q(\Phi(m_{A,-p}(\chi_A)[\chi_A]))
        =m_{A,-p}(\chi_A)I_q(\Phi([\chi_A])).
    $
    Поскольку $A$ --- атом, то
    $
        \Phi([\chi_A])
        =\Phi([\chi_A]\cdot\chi_A)
        =\Phi([\chi_A])\cdot\chi_A
        =m_{A,\infty}(\Phi([\chi_A]))[\chi_A].
    $
    Следовательно,
    \[
    \begin{aligned}
        \phi(e_A)
        &=m_{A,-p}(\chi_A)I_q(m_{A,\infty}(\Phi([\chi_A]))[\chi_A]) 
        =m_{A,\infty}(\Phi([\chi_A]))m_{A,-p}(\chi_A)I_q([\chi_A]) \\
        &=m_{A,\infty}(\Phi([\chi_A]))m_{A,-p}(\chi_A)m_{A,q}(\chi_A)e_A 
        =\mu(A)^{1/q-1/p}m_{A,\infty}(\Phi([\chi_A]))e_A. \\
    \end{aligned}
    \]
    Теперь для любого $x\in\ell_p(\mathcal{A})$ и $a\in\ell_\infty(A)$, имеем
    \[
    \begin{aligned}
        \phi(x\cdot a)
        =\sum_{A\in\mathcal{A}} x_A a_A \phi(e_A) 
        =\sum_{A\in\mathcal{A}} 
            x_A a_A \mu(A)^{1/q-1/p}m_{A,\infty}(\Phi([\chi_A]))e_A 
        =\sum_{A\in\mathcal{A}} (x_A \phi(e_A))\cdot a 
        =\phi(x)\cdot a. \\
    \end{aligned}
    \]
    Следовательно, $\phi$ является $\ell_\infty(\mathcal{A})$-морфизмом. 
    Согласно утверждению \ref{LpOnPurAtomMeasSpRepr}, отображения $I_q$ 
    и $J_p$ --- изометрические изоморфизмамы; следовательно, операторы $\phi$ 
    и $\Phi$ имеют одинаковую норму.

    $(ii)$ Обозначим $\Phi=J_q\circ \phi\circ I_p$, тогда по 
    утверждению \ref{GnrlzdMeanProp} для любого атома $A\in\mathcal{A}$ имеем
    $
        \Phi([\chi_A])
        =J_q(\phi(I_p([\chi_A])))
        =J_q(\phi(m_{A,p}(\chi_A)e_A))
        =m_{A,p}(\chi_A)J_q(\phi(e_A)).
    $
    Более того,
    $
        \phi(e_A)
        =\phi(e_A\cdot e_A)
        =\phi(e_A)\cdot e_A
        =\phi(e_A)_A e_A,
    $
    следовательно,
    \[
    \begin{aligned}
        \Phi([\chi_A])
        &=m_{A,p}(\chi_A)J_q(\phi(e_A)_A e_A) 
        =\phi(e_A)_A m_{A,p}(\chi_A)J_q( e_A) \\
        &=\phi(e_A)_A m_{A,p}(\chi_A)m_{A,-q}(\chi_A) [\chi_A] 
        =\mu(A)^{1/p-1/q}\phi(e_A)_A [\chi_A]. \\
    \end{aligned}
    \]
    Теперь для любого $[f]\in L_p(\Omega, \mu)$ и $a\in B(\Sigma)$, имеем
    \[
    \begin{aligned}
        \Phi([f]\cdot a)
        &=\Phi\left(
            \sum_{A\in\mathcal{A}} m_{A,\infty}(f\cdot a)[\chi_A]
        \right) 
        =\sum_{A\in\mathcal{A}} m_{A,\infty}(f\cdot a) \Phi([\chi_A]) \\
        &=\sum_{A\in\mathcal{A}} 
            m_{A,\infty}(f) m_{A,\infty}(a) \mu(A)^{1/p-1/q} \phi(e_A)_A [\chi_A] 
        =\sum_{A\in\mathcal{A}} 
            (m_{A,\infty}(f) \mu(A)^{1/p-1/q} \phi(e_A)_A [\chi_A])\cdot a \\
        &=\sum_{A\in\mathcal{A}} 
            m_{A,\infty}(f) \Phi([\chi_A])\cdot a 
        =\Phi\left(\sum_{A\in\mathcal{A}} 
            m_{A,\infty}(f) [\chi_A]\right)\cdot a 
        =\Phi([f])\cdot a. \\
    \end{aligned}
    \]
    Таким образом, $\Phi$ является $B(\Sigma)$-морфизмом. Согласно 
    утверждению \ref{LpOnPurAtomMeasSpRepr}, отображения $J_q$ и $I_p$ --- 
    изометрические изоморфизмамы; следовательно, операторы $\Phi$ и $\phi$ имеют 
    одинаковую норму.
\end{proof}

\begin{proposition}\label{LpFinDimCharac}
    Пусть $(\Omega,\mu)$ --- разложимое пространство с мерой 
    и $1\leq p\leq+\infty$. Если $L_p(\Omega,\mu)$ является конечномерным 
    пространством, то пространство $(\Omega,\mu)$ является чисто атомарным с 
    конечным числом атомов конечной меры.
\end{proposition} 
\begin{proof}
    Предположим, что пространство $(\Omega,\mu)$ не является чисто атомарным, 
    тогда существует безатомное измеримое множество $E\subset \Omega$. 
    Из~[\cite{FremMeasTh2}, утверждение 215D] мы можем считать, что $E$ имеет 
    положительную конечную меру. Из~[\cite{FremMeasTh2}, упражнение 215X(e)] мы 
    заключаем, что существует счетное семейство $\mathcal{E}$ попарно 
    непересекающихся множеств положительной конечной меры. В этом 
    случае $([\chi_{E}])_{E\in\mathcal{E}}$ является счетным линейно независимым 
    множеством в $L_p(\Omega,\mu)$. Следовательно, пространство 
    $L_p(\Omega,\mu)$  бесконечномерно. Противоречие, поэтому $(\Omega,\mu)$ 
    является чисто атомарным. Пусть $\mathcal{A}$ --- семейство атомов, 
    объединение которых составляет $\Omega$. Поскольку пространство $\Omega$ 
    разложимо, то все эти атомы имеют конечную меру. 
    Следовательно, $([\chi_{A}])_{A\in\mathcal{A}}$ 
    является линейно независимым множеством. Поскольку 
    пространство $L_p(\Omega,\mu)$ конечномерно, 
    то $\mathcal{A}$ --- конечное множество.
\end{proof}

%-------------------------------------------------------------------------------
% Метрическая проективность, инъективность и плоскость модулей C_0(S) L_p(S,\mu)
%-------------------------------------------------------------------------------

\section{Метрическая проективность, инъективность и плоскость 
\texorpdfstring{$C_0(S)$}{C0(S)}-модулей 
\texorpdfstring{$L_p(S,\mu)$}{LpSmu}}
\label{MetrProInjFltOfC0SModLp}

Пусть $S$ --- хаусдорфово локально компактное пространство. 
Под $\operatorname{Bor}(S)$ мы понимаем $\sigma$-алгебру, порожденную открытыми 
подмножествами $S$. В этом разделе мы будем рассматривать только разложимые 
борелевские меры. Пусть $\mu$ --- такая мера на $S$. Мы покажем, что 
для $1<p<+\infty$ банаховы $C_0(S)$-модули $L_p(S,\mu)$ практически никогда не 
являются метрически проективными, инъективными или плоскими. 

\begin{proposition}\label{C0SMorphBtwnReflxSpIsBMorph}
    Пусть $S$ --- хаусдорфово локально компактное пространство. Тогда 
    \begin{enumerate}[label = (\roman*)]
        \item любой рефлексивный $C_0(S)$-модуль (левый или правый) может быть 
        снабжен структурой $B(\operatorname{Bor}(S))$-модуля, согласованной 
        с умножением на элементы алгебры $C_0(S)$;
        \item любой $C_0(S)$-морфизм рефлексивных модулей 
        является $B(\operatorname{Bor}(S))$-морфизмом;
    \end{enumerate}     
\end{proposition}
\begin{proof} 
    $(i)$ Предположим, что $Z$ --- $C_0(S)$-модуль, тогда посредством умножения 
    Аренса $Z^{**}$ является $C_0(S)^{**}$-модулем 
    [\cite{DalBanAlgAutCont}, предложение 2.6.15(iii)]. Если модуль $Z$ 
    рефлексивен, то естественное вложение $\iota_Z:Z\to Z^{**}$ является 
    изометрическим изоморфизмом. Напомним, что $B(\operatorname{Bor}(S))$ 
    является подалгеброй $C_0(S)^{**}$ 
    [\cite{DalBanAlgAutCont}, предложение 4.2.30], следовательно, мы можем 
    наделить $Z$ структурой $B(\operatorname{Bor}(S))$-модуля по 
    формуле $z\cdot b=\iota_Z^{-1}(\iota_Z(z)\cdot b)$ для $z\in Z$ 
    и $b\in B(\operatorname{Bor}(S))$.
    
    $(ii)$ Пусть $\phi:X\to Y$ --- морфизм рефлексивных $C_0(S)$-модулей. 
    Тогда $\phi^{**}$ является $C_0(S)^{**}$-морфизмом 
    [\cite{DalBanAlgAutCont}, предложение A.3.53]. Как мы отметили 
    выше, $X$ и $Y$ являются $B(\operatorname{Bor}(S))$-модулями 
    и $\iota_X$, $\iota_Y$ являются изометрическими изоморфизмами. 
    Поскольку $\phi=\iota_Y^{-1}\circ \phi^{**}\circ \iota_X$, то $\phi$ 
    является $B(\operatorname{Bor}(S))$-морфизмом.
\end{proof}

\begin{remark}\label{C0SModIsBBorSMod}
    Пусть $S$ --- хаусдорфово локально компактное пространство и $\mu$ --- 
    конечная регулярная борелевская мера на $S$. Для любого $1<p<+\infty$ мы 
    можем рассмотреть рефлексивный $C_0(S)$-модуль $L_p(S,\mu)$ и используя 
    предложение \ref{C0SMorphBtwnReflxSpIsBMorph} сделать 
    его $B(\operatorname{Bor}(S))$-модулем. 
    Согласно [\cite{HelTensProdAndMultModLp}, предложение 2.2] новое внешнее 
    умножение будет совпадать с поточечным.
\end{remark}

\begin{proposition}\label{MetInjC0SModLpSmuOnFinAtmMeasSpCharac}
    Пусть $S$ --- хаусдорфово локально компактное пространство, а $\mu$ --- 
    чисто атомарная борелевская мера на $S$ с конечным числом атомов конечной 
    меры. Предположим, что $1<p<+\infty$ и $C_0(S)$-модуль $L_p(S,\mu)$ 
    метрически инъективен. Тогда $\mu$ имеет не более чем один атом.
\end{proposition}
\begin{proof}
    Пусть $\mathcal{A}$ --- разложение $S$ на $n$ атомов конечной меры. 
    Предположим, что $n>1$. Выберем любое $\kappa\in(n^{-1}, (n-1)^{-1})$. 
    Теперь положим $\mathcal{F}=\mathcal{F}_{\kappa}(\mathcal{A})$. Для 
    каждого $f\in \mathcal{F}$ определим $\ell_\infty(\mathcal{A})$-морфизм 
    $
        m_f:
        \ell_p(\mathcal{A})\to\ell_1(\mathcal{A}),\,
        x\mapsto x\cdot f.
    $
    Мы также будем использовать естественное вложение 
    $
        \operatorname{in}_f:
        L_1(S,\mu)\to\bigoplus_\infty\{L_1(S,\mu):f'\in\mathcal{F}\},
    $
    и естественную проекцию
    $
        \operatorname{pr}_f:
        \bigoplus_\infty\{
            \ell_1(\mathcal{A}):f'\in\mathcal{F}
        \}\to\ell_1(\mathcal{A}),
    $
    которые являются $B(\operatorname{Bor}(S))$-морфизмом 
    и $\ell_\infty(\mathcal{A})$-морфизмом соответственно. Теперь 
    рассмотрим $B(\operatorname{Bor}(S))$-морфизмы 
    $
        I_p^\infty=\bigoplus_\infty\{I_p:f\in\mathcal{F}\},
    $
    и
    $
        J_p^\infty=\bigoplus_\infty\{J_p:f\in\mathcal{F}\}.
    $
    Это изометрические изоморфизмы и легко проверить, что 
    $
        I_p^\infty \circ \operatorname{in}_f=\operatorname{in}_f\circ I_p,
    $
    и
    $
        \operatorname{pr}_f\circ I_p^\infty=I_p\circ \operatorname{pr}_f.
    $
    По утверждению \ref{SwtchMorphBtwnAtomMeasSp}, 
    отображение $\Xi_f=J_1\circ m_f\circ I_p$ 
    является $B(\operatorname{Bor}(S))$-морфизмом. Следовательно, оператор 
    $
        \Xi_{\mathcal{F}}=\sum_{f\in\mathcal{F}}\operatorname{in}_f\circ \Xi_f
    $
    тоже является $B(\operatorname{Bor}(S))$-морфизмом и, как 
    следствие, $C_0(S)$-морфизмом. Заметим, что
    $
        I_1^\infty\circ\Xi_\mathcal{F}\circ J_p
        = \sum_{f\in\mathcal{F}} 
            \operatorname{in}_f\circ m_f 
        =\xi_{\mathcal{F}}.
    $
    По утверждению \ref{StdEmbdSpclCoerciv}, константа 
    обратимости $\gamma_{\mathcal{F}}$ конечна и положительна, 
    поэтому оператор $\xi_{\mathcal{F}}$ $\gamma_{\mathcal{F}}$-топологически 
    инъективен. Поскольку $I_1$ и $J_p$ --- изометрические изоморфизмы, 
    оператор $\Xi_{\mathcal{F}}$ также $\gamma_{\mathcal{F}}$-топологически 
    инъективен. По предположению, $C_0(S)$-модуль $L_p(S,\mu)$ 
    метрически инъективен, значит существует $C_0(S)$-морфизм 
    $
        \Psi:
        \bigoplus_\infty\{ L_1(S,\mu):f\in\mathcal{F}\}\to L_p(S,\mu)
    $
    такой, что $\Psi\circ \Xi_{\mathcal{F}}=1_{L_p(S,\mu)}$ 
    и $\Vert \Psi\Vert\leq \gamma_{\mathcal{F}}$.

    Снова, для каждого $f\in\mathcal{F}$ мы определим 
    $C_0(S)$-морфизм $\Psi_f=\Psi\circ \operatorname{in}_f$. 
    Поскольку множество $\mathcal{A}$ конечно, то пространства $L_p(S,\mu)$ 
    и $L_1(S,\mu)$ конечномерны и, следовательно, рефлексивны. 
    Отсюда, согласно утверждению \ref{C0SMorphBtwnReflxSpIsBMorph} и 
    замечанию \ref{C0SModIsBBorSMod}, для любого $f\in\mathcal{F}$, 
    оператор $\Psi_f$ является $B(\operatorname{Bor}(S))$-морфизмом. 
    Заметим, что
    $
        \Psi=\sum_{f\in\mathcal{F}} \Psi_f\circ\operatorname{pr}_f.
    $
    Рассмотрим ограниченный линейный оператор
    $
        \psi
        =I_p\circ\Psi\circ J_1^\infty
        =\sum_{f\in\mathcal{F}} 
            I_p\circ\Psi_f\circ J_1\circ\operatorname{pr}_f.
    $
    Согласно утверждению \ref{SwtchMorphBtwnAtomMeasSp}, для 
    каждого $f\in \mathcal{F}$, отображение $I_p\circ \Psi_f\circ J_1$ является 
    $\ell_\infty(\mathcal{A})$-морфизмом. Значит, $\psi$ также является 
    $\ell_\infty(\mathcal{A})$-морфизмом. Так как $I_p$ 
    и $J_1^{\infty}$ --- изометрические изоморфизмы, 
    то $\Vert\psi\Vert=\Vert\Psi\Vert$. Кроме того,
    $
        \psi\circ\xi_\mathcal{F}
        = I_p\circ\Psi\circ J_1^{\infty}\circ 
            I_1^{\infty}\circ \Xi_{\mathcal{F}}\circ J_p
        = I_p\circ\Psi\circ \Xi_{\mathcal{F}}\circ J_p
        = I_p\circ J_p
        = 1_{\ell_p(\mathcal{A})}.
    $
    Таким образом, мы построили $\ell_\infty(\mathcal{A})$-морфизм $\psi$
    такой, что $\psi\circ\xi_{\mathcal{F}}=1_{\ell_p(\mathcal{A})}$ 
    и $\Vert \psi\Vert\leq\gamma_{\mathcal{F}}$. Поскольку мы предположили, 
    что $n>1$, мы приходим к противоречию с утверждением \ref{RetrPrblmNoSln}. 
    Следовательно, $n\leq 1$, то есть $\mathcal{A}$ содержит не более чем один 
    атом.
\end{proof}

В следующем предложении мы упомянем так 
называемые $\mathscr{L}_\infty^g$-пространства. У них достаточно длинное 
определение [\cite{DefFloTensNorOpId}, определение 3.13] и мы не будем его 
здесь давать. Достаточно будет сказать, что это банаховы пространства у которых 
конечномерные подпространства ``очень похожи'' на 
конечномерные $\ell_\infty$-пространства. 

\begin{proposition}\label{MetInjC0SModLpSmuCharac}
    Пусть $S$ --- хаусдорфово локально компактное пространство, а $\mu$ --- 
    разложимая борелевская мера на $S$. Предположим, что $1 < p < +\infty$ 
    и $C_0(S)$-модуль $L_p(S,\mu)$ метрически инъективен, тогда $\mu$ --- чисто 
    атомарная мера с не более чем одним атомом.
\end{proposition}
\begin{proof} 
    Пусть $K$ --- александровская компактификация пространства $S$. 
    Тогда пространство $C_0(S)$ дополняемо в $C(K)$. 
    Из [\cite{DefFloTensNorOpId}, лемма 4.4] мы знаем, что пространство $C(K)$ 
    является $\mathscr{L}_\infty^g$-пространством. Значит этому классу 
    принадлежит и $C_0(S)$, как дополняемое подпространство $C(K)$ 
    [\cite{DefFloTensNorOpId}, следствие 23.1.2(1)]. 
    Таким образом, $L_p(S,\mu)$ является рефлексивным 
    [\cite{FremMeasTh2}, теорема 244K] метрически инъективным модулем над 
    алгеброй $C_0(S)$, которая является $\mathscr{L}_\infty^g$-пространством. 
    Из [\cite{NemGeomProjInjFlatBanMod}, следствие 3.14] следует, что этот 
    модуль должен быть конечномерным. Из предложения \ref{LpFinDimCharac} мы 
    заключаем, что $\mu$ является чисто атомарной мерой с конечным числом атомов 
    конечной меры. Наконец, 
    предложение \ref{MetInjC0SModLpSmuOnFinAtmMeasSpCharac} гарантирует, что 
    у $\mu$ не более одного атома.
\end{proof}

\begin{theorem}\label{MetInjPlotjFlatC0SModLpSmuCharac}
    Пусть $S$ --- хаусдорфово локально компактное пространство, а $\mu$ --- 
    разложимая борелевская мера на $S$. Предположим, что $1 < p < +\infty$ 
    и $C_0(S)$-модуль $L_p(S,\mu)$ метрически проективный, инъективный или 
    плоский. Тогда $\mu$ --- чисто атомарная мера с не более чем одним атомом.
\end{theorem}
\begin{proof} 
    Если $L_p(S,\mu)$ является метрически инъективным $C_0(S)$-модулем, то 
    результат следует из 
    предложения \ref{MetInjC0SModLpSmuCharac}. 
    
    Предположим, что $L_p(S,\mu)$ является метрически плоским $C_0(S)$-модулем. 
    Тогда из [\cite{NemGeomProjInjFlatBanMod}, предложение 2.21] следует, что
    сопряженный $C_0(S)$-модуль $L_p(S,\mu)^*$ является метрически инъективным. 
    Заметим, что $C_0(S)$-модули $L_p(S,\mu)^*$ и $L_{p^*}(S,\mu)$ изометрически 
    изоморфны, причем $1 < p^* < +\infty$ т.к. $1 < p < +\infty$. Тогда, 
    результат следует из предыдущего абзаца.
    
    Если $L_p(S,\mu)$ является метрически проективным $C_0(S)$-модулем, то 
    по [\cite{NemGeomProjInjFlatBanMod}, предложение 2.26], он также метрически 
    плоский. Следовательно, результат следует из предыдущего абзаца.
\end{proof}

\begin{thebibliography}{999}
    %
    \bibitem{NemRelProjModLp}\textit{N. T. Nemesh}, Relative projectivity of 
    modules $\it{L_p}$, Math. Notes, Volume 111, p 103--114, 2022.
    %
    %
    \bibitem{HelLectAndExOnFuncAn}\textit{A. Ya. Helemskii}, Lectures and 
    exercises on functional analysis, Vol. 233, 
    American Mathematical Society Providence, RI, 2006.
    %
    %
    \bibitem{HelBanLocConvAlg}\textit{A. Ya. Helemskii}, Banach and locally 
    convex algebras, Oxford University Press, 1993.
    %
    %
    \bibitem{GravInjProjBanMod}\textit{A. W. M. Graven}, Injective and 
    projective Banach modules, Indag. Math. (Proceedings), Vol. 82, p. 253--272,
    Elsevier, 1979.
    %
    \bibitem{WhiteInjmoduAlg}\textit{M. C. White}, Injective modules for uniform 
    algebras, Proc. Lond. Math. Soc., Vol. 3(1), p. 155--184, Oxford University 
    Press, 1996.
    %
    \bibitem{HelMetrFrQMod}\textit{A. Ya. Helemskii}, Metric freeness and 
    projectivity for classical and quantum normed modules, Sb. Mat., 
    Vol. 204(7), p. 1056--1083, IOP Publishing, 2013.
    %
    %
    \bibitem{HelMetrFlatNorMod}\textit{A. Ya. Helemskii}, Metric version of 
    flatness and Hahn-Banach type theorems for normed modules over sequence 
    algebras, Stud. Math., Vol. 206(2), p. 135--160, Institute of Mathematics, 
    2011.
    %
    %
    \bibitem{FremMeasTh2}\textit{D. H. Fremlin}, Measure Theory, Vol. 2,
    Torres Fremlin, 2003.
    %
    %
    \bibitem{DalBanAlgAutCont}\textit{H. G. Dales}, Banach algebras and 
    automatic continuity, Clarendon Press, 2000.
    %
    %
    \bibitem{HelTensProdAndMultModLp}\textit{Хелемский, А.Я.}, Тензорные 
    произведения и мультипликаторы модулей $\it{L_p}$ на локально компактных 
    пространствах с мерой, Матем. заметки, Vol. 96(3), p. 450--469, 2014.
    %
    %
    \bibitem{DefFloTensNorOpId}\textit{A. Defant, K. Floret}, Tensor norms and 
    operator ideals, Vol. 176, Elsevier, 1992.
    %
    %
    \bibitem{NemGeomProjInjFlatBanMod}\textit{N. T. Nemesh}, The Geometry of 
    projective, injective and flat Banach modules, J. Math. Sci. (New York), 
    Volume 237, Issue 3, Pages 445–459, 2019.
    %
\end{thebibliography}

Норберт Немеш, Факультет Механики и Математики, Московский государственный 
университет, Москва 119991, Россия

\textit{Адрес электронной почты:} nemeshnorbert@yandex.ru


\end{document}
