% chktex-file 35 chktex-file 19

\documentclass[12pt]{article}
\usepackage[left=2cm,right=2cm, top=2cm,bottom=2cm,bindingoffset=0cm]{geometry}
\usepackage{amssymb,amsmath}
\usepackage[utf8]{inputenc}
\usepackage[matrix,arrow,curve]{xy}
\usepackage[final]{graphicx}
\usepackage{mathrsfs}
\usepackage[colorlinks=true, urlcolor=blue, linkcolor=blue, citecolor=blue,
    pdfborder={0 0 0}]{hyperref}
\usepackage{enumitem}

\newtheorem{theorem}{Theorem}[subsection]
\newtheorem{lemma}[theorem]{Lemma}
\newtheorem{proposition}[theorem]{Proposition}
\newtheorem{remark}[theorem]{Remark}
\newtheorem{corollary}[theorem]{Corollary}
\newtheorem{definition}[theorem]{Definition}
\newtheorem{example}[theorem]{Example}

\newenvironment{proof}{\par $\triangleleft$}{$\triangleright$}

\pagestyle{plain}

\begin{document}

\begin{center}

    \Large \textbf{Homological triviality of the category of modules
        $L_p$}\\[0.5cm]
    \small {Norbert Nemesh}\\[0.5cm]

\end{center}
\thispagestyle{empty}

\begin{abstract}
    We give complete characterisation of topologically injective (bounded
    below), topologically surjective (open mapping), isometric and coisometric
    (quotient mapping) multiplication operators between $L_p$ spaces defined on
    different $\sigma$-finite measure spaces. We prove that all such operators
    invertible from the left or from the right. As the consequence we prove that
    all objects of the category of $L_p$ spaces considered as left Banach
    modules over algebra of bounded measurable functions are metrically,
    extremelly and relatively projective, injective and flat.
\end{abstract}

%-------------------------------------------------------------------------------
% Preliminaries
%-------------------------------------------------------------------------------

\section{Preliminaries}

%-------------------------------------------------------------------------------
% Measure theoretic facts
%-------------------------------------------------------------------------------

\subsection{Measure theoretic facts}

Let $(\Omega,\Sigma,\mu)$ be a measure space with $\sigma$-additive real valued
measure. We say that $\Omega'\in\Sigma$ is an atom if $\mu(\Omega')>0$ and for
every $E\in\Sigma$ such that $E\subset\Omega'$ either $\mu(E)=0$ or
$\mu(\Omega'\setminus E)=0$. By $A(\Omega,\mu)$ we denote the set of atoms of
$(\Omega,\Sigma,\mu)$. Now we present several standard facts from measure
theory.

\begin{lemma}\label{AtomDescInSigmFinMeasSp} If $(\Omega,\Sigma,\mu)$ is a
    $\sigma$-finite measure space then all its atoms are of the finite measure.
\end{lemma}
\begin{proof} Since $(\Omega,\Sigma,\mu)$ is $\sigma$-finite we have
    representaion $\Omega=\bigcup_{n\in\mathbb{N}} F_n$ as disjoint union of
    measurable sets of finite measure. Assume we have $\Omega'\in A(\Omega,\mu)$
    of infinite measure. For $n\in\mathbb{N}$ define
    $\widetilde{F}_n=F_n\cap \Omega'\in\Sigma$. Fix $n\in\mathbb{N}$, then
    $\widetilde{F}_n\subset\Omega'$
    and either $\mu(\widetilde{F}_n)=0$ or $\mu(\widetilde{F}_n)=+\infty$. Since
    $\mu(\widetilde{F}_n)\leq\mu(F_n)<+\infty$ we get $\mu(\widetilde{F}_n)=0$.
    As $n\in\mathbb{N}$ is arbitrary we get
    $\mu(\Omega')=\mu(\bigcup_{n\in\mathbb{N}}\widetilde{F}_n)
        =\sum_{n\in\mathbb{N}}\mu(\widetilde{F}_n)=0$.
    Contradiction, so $\mu(\Omega')<+\infty$.
\end{proof}

\begin{lemma}\label{PureAtomSpDecomp} Let $(\Omega,\Sigma,\mu)$ be a purely
    atomic measure space, then there exist pairwise disjoint family of atoms $
        \{\Omega_\lambda:\lambda\in\Lambda \}$ such that
    $\Omega=\bigcup_{\lambda\in\Lambda}\Omega_\lambda$. If $(\Omega,\Sigma,\mu)$
    is $\sigma$-finite, then the family $ \{\Omega_\lambda:\lambda\in\Lambda \}$
    is at most countable.
\end{lemma}
\begin{proof}
    Let
    $\mathcal{F}= \{F\subset A(\Omega,\mu):
        \Omega',\Omega''\in F\implies \Omega'\cap\Omega''=\varnothing \}$.
    For $F',F''\in\mathcal{F}$ we take by
    definition $F'\leq F''$ if $F'\subset F''$. In this case
    $(\mathcal{F},\leq)$ is a partially ordered set in which every totally
    ordered set have upper bound. By Zorn's lemma we have maximal element
    $\widetilde{F}= \{\widetilde{\Omega}_\lambda:\lambda\in\Lambda \}$. Define
    $\Omega_0
        =\Omega\setminus(\bigcup_{\lambda\in\Lambda}\widetilde{\Omega}_\lambda)$.
    If $\mu(\Omega_0)>0$ then since $\Omega$ is purely atomic there exist
    $\Omega_1\in A(\Omega,\mu)$ such that $\Omega_1\subset\Omega_0$. Consider
    $F=\widetilde{F}\cup  \{\Omega_1 \}\in\mathcal{F}$, then $\widetilde{F}\leq
        F$ and $\widetilde{F}\neq F$. This contradicts maximality of
    $\widetilde{F}$, hence $\mu(\Omega_0)=0$. Now take any
    $\lambda_0\in\Lambda$, then define
    $$
        \Omega_\lambda=
        \begin{cases}
            \widetilde{\Omega}_{\lambda_0}\cup\Omega_0\quad &
            \text{if}\quad\lambda=\lambda_0                   \\
            \widetilde{\Omega}_\lambda            \quad     &
            \text{if}\quad\lambda\neq\lambda_0
        \end{cases}
    $$
    Clearly $\widetilde{\Omega}_{\lambda_0}\cup\Omega_0$ is an atom disjoint
    from atoms $\widetilde{\Omega}_\lambda$ for $\lambda\neq\lambda_0$. Hence $
        \{\Omega_\lambda:\lambda\in\Lambda \}$ is the desired family.

    If $(\Omega,\Sigma,\mu)$ is $\sigma$-finite we have representation
    $\Omega=\bigcup_{n\in\mathbb{N}}E_n$ as disjoint union of measurable sets of
    finite measure. Define $\Omega_{\lambda, n}=\Omega_\lambda\cap E_n$, then
    for each $n\in\mathbb{N}$ the we have
    $E_n=\bigcup_{\lambda\in\Lambda}\Omega_{\lambda,n}$ and
    $\Omega_{\lambda',n}\cap\Omega_{\lambda'',n}=\varnothing$ for
    $\lambda'\neq\lambda''$. Since $\mu(E_n)<+\infty$, then the family $
        \{\lambda\in\Lambda:\mu(\Omega_{\lambda,n})>k^{-1} \}$ is finite for
    every
    $k\in\mathbb{N}$. Thus the family
    $\Lambda_n= \{\lambda\in\Lambda:\mu(\Omega_{\lambda,n})>0 \}$ is at
    most countable. Since for all $\lambda\in\Lambda$ we have a representaion
    $\Omega_\lambda=\bigcup_{n\in\mathbb{N}}\Omega_{\lambda,n}$ where
    $\mu(\Omega_\lambda)>0$ and
    $\Omega_{\lambda,n}\cap\Omega_{\lambda,m}=\varnothing$, then
    $\mu(\Omega_{\lambda,n})>0$ for some $n\in\mathbb{N}$. In other words
    $\Lambda=\bigcup_{n\in\mathbb{N}}\Lambda_n$, so $\Lambda$ is at most
    countable as union at most countable sets $\Lambda_n$.
\end{proof}

\begin{theorem}[\cite{RoyJAtNonAtMeas}, 2.1]\label{MeasSpDecomp} Let
    $(\Omega,\Sigma,\mu)$ be a $\sigma$-finite measure space, then there exist
    purely atomic measure $\mu_1:\Sigma\to[0,+\infty]$ and non atomic measure
    $\mu_2:\Sigma\to[0,+\infty]$ such that $\mu=\mu_1+\mu_2$ and
    $\mu_1\perp\mu_2$.
\end{theorem}

\begin{theorem}[\cite{SierpConFamlFunc}]\label{ContOfNonAtmMeas} Let
    $(\Omega,\Sigma,\mu)$ be nonatomic measure space. If $E\in\Sigma$  with
    $\mu(E)>0$, then for all $t\in[0,\mu(E)]$ there exist $F\in\Sigma$ such that
    $F\subset E$ and $\mu(F)=t$
\end{theorem}

\begin{theorem}[\cite{RoyJLebDecompTh}, 2.1]\label{LebMeasDecomp} Let
    $(\Omega,\Sigma,\mu)$, $(\Omega,\Sigma,\nu)$ be $\sigma$-finite measure
    spaces, then there exist a measurable function $\rho_{\nu,\mu}$ a
    $\sigma$-finite measure $\nu_s:\Sigma\to[0,+\infty]$ and a set
    $\Omega_s\in\Sigma$ such that
    \begin{enumerate}[label = (\roman*)]
        \item $\nu=\rho_{\nu,\mu}\cdot\mu+\nu_s$

        \item $\mu\perp\nu_s$ i.e.  $\mu(\Omega_s)=\nu_s(\Omega_c)=0$, where
              $\Omega_c=\Omega\setminus\Omega_s$
    \end{enumerate}
\end{theorem}

%-------------------------------------------------------------------------------
% Decomposition of Lp spaces
%-------------------------------------------------------------------------------

\subsection{Decompostion of \texorpdfstring{$L_p$}{Lp} spaces}

All linear spaces in this article are considered over field $\mathbb{C}$. By
$L_0(\Omega,\mu)$ we denote the linear space of measurable functions on
$\Omega$. If $p=\infty$ then we take by definition that $1/p=0$. All equalities
and inequalities about $L_p$ functions are understood up to sets of measure
zero.

\begin{proposition}\label{LpSpDecomp} Let $(\Omega,\Sigma,\mu)$ be a measure
    space and $p\in[1,+\infty]$. Assume we have represetation
    $\Omega=\bigcup_{\lambda\in\Lambda}\Omega_\lambda$ where
    $\Omega_{\lambda'}\cap\Omega_{\lambda''}=\varnothing$ for
    $\lambda'\neq\lambda''$. Then the map
    $$
        I_p:L_p(\Omega,\mu)\to \bigoplus\limits_p \left \{
        L_p(\Omega_\lambda,\mu|_{\Omega_\lambda}):\lambda\in\Lambda
        \right \}, f\mapsto (\lambda\mapsto f|_{\Omega_\lambda})
    $$
    is an isometric isomorphism.
\end{proposition}
\begin{proof}
    If $f\in L_p(\Omega,\mu)$, then, of course, $f|_{\Omega_\lambda}\in
        L_p(\Omega_\lambda,\mu|_{\Omega_\lambda})$ for $\lambda\in\Lambda$.
    So $I_p$ is well defined and, obviously, it is linear. Now it is remains
    to prove that $I_p$ is surjective and isometric. Let $f_\lambda\in
        L_p(\Omega_\lambda,\mu|_{\Omega_\lambda})$ for $\lambda\in\Lambda$ then
    define $f(\omega)=f_\lambda(\omega)$ if $\omega\in \Omega_\lambda$. Clearly
    ${I_p(f)}_\lambda=f_\lambda$ so $I_p$ is surjective. Then for
    $p\geq 1$ we have
    $$
        \Vert I_p(f)\Vert_{\bigoplus\limits_p \left \{
        L_p(\Omega_\lambda,\mu|_{\Omega_\lambda}):\lambda\in\Lambda
        \right \}}
        ={\left(\sum\limits_{\lambda\in\Lambda}
        \int_{\Omega_\lambda}{|f|}_{\Omega_\lambda}(\omega)|^p
        d\mu|_{\Omega_\lambda}(\omega)
        \right)}^{1/p}
    $$
    $$
        ={\left(\int_{\Omega}{|f(\omega)|}^p d\mu(\omega)\right)}^{1/p}
        =\Vert f\Vert_{L_p(\Omega,\mu)}
    $$
    Similarly for $p=\infty$ we have
    $$
        \Vert I_\infty(f)\Vert_{\bigoplus\limits_p \left \{
        L_p(\Omega_\lambda,\mu|_{\Omega_\lambda}):\lambda\in\Lambda
        \right \}}
        =\sup\limits_{\lambda\in\Lambda}\mathop{\operatorname{essup}}_{
            \omega\in\Omega_\lambda}|f|_{\Omega_\lambda}(\omega)|
        =\mathop{\operatorname{essup}}_{\omega\in\Omega}|f(\omega)|
        =\Vert f\Vert_{L_\infty(\Omega,\mu)}
    $$
    In both cases $I_p$ is isometric.
\end{proof}

\begin{lemma}\label{FuncDescOnAtom} Let $(\Omega,\Sigma,\mu)$ be a
    $\sigma$-finite measure space with atom $\Omega'$. Then
    \begin{enumerate}[label = (\roman*)]
        \item If $p\in[1,+\infty]$ and $f\in L_p(\Omega',\mu|_{\Omega'})$, then
              $$
                  f(\omega)
                  ={\mu(\Omega')}^{-1}\int_{\Omega'} f(\omega')d\mu(\omega')
              $$
              for $\omega\in\Omega'$.

        \item If $p\in[1,+\infty]$ the map
              $$
                  J_p:L_p(\Omega',\mu|_{\Omega'})\to \ell_p( \{1 \}),
                  f\mapsto\left(1\mapsto {\mu(\Omega')}^{1/p-1}\int_{\Omega'}
                  f(\omega')d\mu(\omega')\right)
              $$
              is an isometric isomorphism.
    \end{enumerate}

\end{lemma}
\begin{proof}
    Since $(\Omega,\Sigma,\mu)$ is $\sigma$-finite, by
    lemma~\ref{AtomDescInSigmFinMeasSp} we have $\mu(\Omega')<+\infty$.
    $(i)$ Since $\mu(\Omega')<+\infty$, then $f\in
        L_p(\Omega',\mu|_{\Omega'})\subset L_1(\Omega',\mu|_{\Omega'})$. For the
    beginning assume that $f$ is a real valued function. Denote
    $k={\mu(\Omega')}^{-1}\int_{\Omega'} f(\omega')d\mu(\omega')$, then consider
    set $S_-=f^{-1}((k,+\infty])$. Since $\Omega'$ is an atom then  % chktex 9
    $\mu(S_-)=\mu(\Omega')$ or $\mu(S_-)=0$. In the first case we get
    $$
        \int_{\Omega'} f(\omega')\mu(\omega')
        =\int_{S_-} f(\omega')\mu(\omega')
        <\int_{S_-} c\mu(\omega')
        =k\mu(S_-)
        =k\mu(\Omega')
        =\int_{\Omega'} f(\omega')\mu(\omega')
    $$
    Contradiction, hence $\mu(S_-)=0$. Similarly we get that $\mu(S_+)=0$ for
    $S_+=f^{-1}([-\infty,k))$. Hence $f(\omega)=k$ for $\mu$-almost  % chktex 9
    all $\omega\in\Omega'$. If $f$ is complex valued we apply previous result to
    $\operatorname{Re}(f),\operatorname{Im}(f)\in L_1(\Omega',\mu|_{\Omega'})$
    and get that
    $$
        f(\omega)
        =\operatorname{Re}(f)(\omega)+i\operatorname{Im}(f)(\omega)
    $$
    $$
        ={\mu(\Omega')}^{-1}\int_{\Omega'}
        \operatorname{Re}(f)(\omega')d\mu(\omega')
        +i{\mu(\Omega')}^{-1}\int_{\Omega'}
        \operatorname{Im}(f)(\omega')d\mu(\omega')
        ={\mu(\Omega')}^{-1}\int_{\Omega'} f(\omega')d\mu(\omega')
    $$
    for $\mu$-almost all $\omega\in\Omega$

    $(ii)$ Obviously $J_p$ is linear. Take any $z\in\mathbb{C}$ and consider
    function $f=z{\mu(\Omega_1)}^{-1/p}\chi_{\Omega'}$, then $J_p(f)(1)=z$. Thus
    $J_p$ is surjective. Now for $p\in[1,+\infty]$ and all $f\in
        L_p(\Omega',\mu|_{\Omega'})$ we have
    $$
        \Vert J_p(f)\Vert_p
        =\left|{\mu(\Omega')}^{1/p-1}\int_{\Omega'}f(\omega')d\mu(\omega')\right|
        =\left|{\mu(\Omega')}^{1/p-1}k\mu(\Omega')\right|
        ={\mu(\Omega')}^{1/p}|k|
        =\Vert f\Vert_{L_p(\Omega',\mu|_{\Omega'})}
    $$
    Thus, the map $J_p$ is a surjective isometry, hence isometric isomorphism.
\end{proof}

\begin{proposition}\label{DescOfLpSpOnPureAtomMeasSp} Let $(\Omega,\Sigma,\mu)$
    be a $\sigma$-finite purely atomic measure space, then for $p\in[1,+\infty]$,
    the map
    $$
        \widetilde{I}_p:L_p(\Omega,\mu)\to \ell_p(\Lambda):
        f\mapsto\left (\lambda\mapsto J_p(f|_{\Omega_\lambda})(1)\right)
    $$
    is an isometric isomorphism. Here
    $ \{\Omega_\lambda:\lambda\in\Lambda \}\subset A(\Omega,\mu)$
    is at most countable family of pairwise disjoint
    atoms such that $\Omega=\bigcup_{\lambda\in\Lambda}\Omega_\lambda$.
\end{proposition}
\begin{proof}
    By lemma~\ref{PureAtomSpDecomp} we have a family $
        \{\Omega_\lambda:\lambda\in\Lambda \}\subset A(\Omega,\mu)$ of pairwise
    disjoint atoms whose union is $\Omega$. By proposition~\ref{LpSpDecomp} we
    have an isometric isomorphism
    $$
        L_p(\Omega,\mu)
        \cong_1 \bigoplus\limits_p
        \left \{
        L_p(\Omega_\lambda,\mu|_{\Omega_\lambda}):\lambda\in\Lambda
        \right \}
    $$
    via the map $I_p(f)=\oplus_p  \{f|_{\Omega_\lambda}:\lambda\in\Lambda \}$.
    By lemma~\ref{FuncDescOnAtom} we know that
    $L_p(\Omega_\lambda,\mu|_{\Omega_\lambda})\cong_1\ell_p( \{1 \})$ via the
    map $J_p$. So we get
    $$
        L_p(\Omega,\mu)
        \cong_1 \bigoplus\limits_p \left \{
        L_p(\Omega_\lambda,\mu|_{\Omega_\lambda}):\lambda\in\Lambda
        \right \}
        \cong_1 \bigoplus\limits_p \left \{
        \ell_p( \{1 \}):\lambda\in\Lambda
        \right \}
        =\ell_p(\Lambda)
    $$
    via the map $\widetilde{I}_p$. Since $(\Omega,\Sigma,\mu)$ is
    $\sigma$-finite by lemma~\ref{PureAtomSpDecomp} we get that $\Lambda$ is at
    most countable.
\end{proof}

\begin{proposition}\label{DescOfLpSpOnMeasSp} Let $p\in[1,+\infty]$. Let
    $(\Omega,\Sigma,\mu)$ be $\sigma$-finite measure space, then
    there exist at most countable family of atoms
    $ \{\Omega_\lambda:\lambda\in\Lambda \}$ and a set
    $\Omega_{na}=\Omega\setminus\left(
        \bigcup_{\lambda\in\Lambda}\Omega_\lambda
        \right)=\Omega\setminus\Omega_{a}$
    such that the map
    $$
        \hat{I}_p:L_p(\Omega,\mu)\to
        L_p(\Omega_{na},\mu|_{\Omega_{na}})
        \bigoplus_{p}L_p(\Omega_{a},\mu|_{\Omega_{a}}),
        f\mapsto (f|_{\Omega_{na}},f|_{\Omega_{a}})
    $$
    is isometric isomorphism and $\mu|_{\Omega_{na}}$ is nonatomic.
\end{proposition}
\begin{proof} By theorem~\ref{MeasSpDecomp} we have mutually singular purely
    atomic measure $\mu_1$ and nonatomic measure $\mu_2$ whose sum is $\mu$.
    Since they are singular there exist $\Omega_a\in\Sigma$ such that
    $\mu_2(\Omega_a)=\mu_1(\Omega\setminus\Omega_a)=0$. Thus
    $\mu|_{\Omega_a}=\mu_1$ is purely atomic and
    $\mu|_{\Omega_{na}}=\mu_2$ is nonatomic. Here
    $\Omega_{na}=\Omega\setminus\Omega_a$. By proposition~\ref{LpSpDecomp}
    $$
        \hat{I}_p:L_p(\Omega,\mu)\to
        L_p(\Omega_{na},\mu|_{\Omega_{na}})
        \bigoplus_{p}L_p(\Omega_{a},\mu|_{\Omega_{a}}),
        f\mapsto (f|_{\Omega_{na}},f|_{\Omega_{a}})
    $$
    is an isometric isomorphism.
\end{proof}

\begin{proposition}\label{ChngOfDenst} Let $(\Omega,\Sigma,\mu)$ be a
    $\sigma$-finite measure space. Let $\rho\in L_0(\Omega,\mu)$ be a  positive
    function. Then
    $$
        \bar{I}_p:L_p(\Omega,\mu)\to
        L_p(\Omega,\rho\cdot \mu), f\mapsto\rho^{-1/p}\cdot f
    $$
    is an isometric isomorphism for all $p\in[1,+\infty]$.
\end{proposition}
\begin{proof} Obviously $\bar{I}_p$ is linear. Let $p\geq 1$,
    then for
    all $f\in L_p(\Omega,\mu)$ we have
    $$
        \Vert \bar{I}_p(f)\Vert_{L_p(\Omega,\rho\cdot\mu)}
        ={\left(
        \int_{\Omega}
            {|\rho^{-1/p}(\omega)f(\omega)|}^p\rho(\omega)d\mu(\omega)
        \right)}^{1/p}
            ={\left(\int_{\Omega}{|f(\omega)|}^p d\mu(\omega) \right)}^{1/p}
        =\Vert f\Vert_{L_p(\Omega,\mu)}
    $$
    so $\bar{I}_p$ is an isometry. Now for arbitrary
    $f\in L_p(\Omega,\rho\cdot\mu)$ consider $h=\rho^{1/p}\cdot f$, then
    $$
        \Vert h\Vert_{L_p(\Omega,\mu)}
        ={\left(
        \int_{\Omega}
            {|\rho^{1/p}(\omega)f(\omega)|}^p d\mu(\omega)
        \right)}^{1/p}
            ={\left(
                \int_{\Omega}{|f(\omega)|}^p\rho(\omega)d\mu(\omega)
                \right)}^{1/p}
        =\Vert f\Vert_{L_p(\Omega,\rho\cdot\mu)}
    $$
    So $h\in L_p(\Omega,\mu)$ and obviously $\bar{I}_p(h)=f$. Thus $\bar{I}_p$
    is a surjective isometry, i.e.\ isometric isomorphism. Now let $p=+\infty$,
    then $\bar{I}_\infty$ is an identity map. Hence it is enough to show it is
    isometric. Let $E\in\Sigma$ with $\mu(E)=0$, then
    $(\rho\cdot\mu)(E)=\int_E\rho(\omega)d\mu(\omega)=0$. On the other hand if
    $(\rho\cdot\mu)(E)=\int_E\rho(\omega)d\mu(\omega)=0$ then from positivity of
    $\rho$ it follows that $\mu(E)=0$. So for all $f\in L_\infty(\Omega,\mu)$ we
    have
    $$
        \Vert\bar{I}_\infty(f)\Vert_{L_\infty(\Omega,\rho\cdot\mu)}
        =\inf \{C>0:(\rho\cdot\mu)(|f|^{-1}((C,+\infty)))=0 \}
    $$
    $$
        =\inf \{C>0:\mu(|f|^{-1}((C,+\infty)))=0 \}
        =\Vert f\Vert_{L_\infty(\Omega,\mu)}
    $$
    Thus, $\bar{I}_\infty$ is an isometry.
\end{proof}

\begin{proposition}\label{FinDimlpEquivNorms} Let $\Lambda$ be a finite set and
    $p,q\in[1,+\infty]$, then there exist $C_{p,q}$ such that
    $\Vert x\Vert_{\ell_p(\Lambda)}\leq C_{p,q}\Vert x\Vert_{\ell_q(\Lambda)}$
    for all $x\in\mathbb{C}^\Lambda$.
\end{proposition}
\begin{proof} Since $(\mathbb{C}^\Lambda,\Vert\cdot\Vert_{\ell_r(\Lambda)})$ is
    a normed space of finite dimension equal to $\operatorname{Card}(\Lambda)$
    for every $r\in[1,+\infty]$, then all norms
    $\left \{
        \Vert\cdot\Vert_{\ell_r(\Lambda)}:r\in[1,+\infty]
        \right \}$ on
    $\mathbb{C}^\Lambda$ are equivalent. So we get the desired inequality.
\end{proof}

%-------------------------------------------------------------------------------
% Mutiplication operators
%-------------------------------------------------------------------------------

\subsection{Mutiplication operators}

Let $(\Omega,\Sigma,\mu)$ and $(\Omega,\Sigma,\nu)$ be two measure spaces with
the same $\sigma$-algebra of measurable sets. For a given $g\in L_0(\Omega,\mu)$
and $p,q\in[1,+\infty]$ we define the multiplication operator
$$
    M_g:L_p(\Omega,\mu)\to L_q(\Omega,\nu), f\mapsto g\cdot f
$$
Of course certain restrictions on $g$, $\mu$ and $\nu$ are required for $M_g$ to
be well defined. For a given $E\in\Sigma$ by $M_g^E$ we will denote the linear
operator
$$
    M_g^E:L_p(E,\mu|_E)\to L_q(E,\nu|_E),f\mapsto g|_E\cdot f
$$
It is well defined because $f|_{\Omega\setminus E}=0$ implies
$M_g(f)|_{\Omega\setminus E}=0$.

\begin{proposition}\label{MultpOpSurjInjDesc} Let $(\Omega,\Sigma,\mu)$ be a
    measure space and $g\in L_0(\Omega,\mu)$. Denote $Z_g=g^{-1}( \{0 \})$,
    then for the operator $M_g:L_p(\Omega,\mu)\to L_q(\Omega,\mu)$ we have

    $(i)$ $\operatorname{Ker}(M_g)= \{f\in L_p(\Omega,\mu):f|_{\Omega\setminus
        {Z_g}}=0 \}$, so $M_g$ is injective if and only if $\mu(Z_g)=0$

    $(ii)$ $\operatorname{Im}(M_g)\subset \{h\in L_q(\Omega,\mu): h|_{Z_g}=0
        \}$,so if $M_g$ is surjective then $\mu(Z_g)=0$.

\end{proposition}
\begin{proof}
    $(i)$ We have the following chain of equivalences
    $$
        f\in\operatorname{Ker}(M_g)
        \Longleftrightarrow g\cdot f=0
        \Longleftrightarrow f|_{\Omega\setminus Z_g}=0
    $$
    And we get the desired equality.

    $(ii)$ Since $g|_{Z_g}=0$ then $M_g(f)|_{Z_g}=(g\cdot f)|_{Z_g}=0$ for all
    $f\in L_p(\Omega,\mu)$, thus we get the inclusion. If $M_g$ is surjective
    then, clearly,  $\mu(Z_g)=0$.
\end{proof}
\newline

We want to classify multiplication operators according to following definitions

\begin{definition}\label{DefNorOpType} Let $ T:E\to F$ be bounded linear
    operator between normed spaces $E$ and $F$, then $ T$ is called
    \newline
    $(i)$ \textit{$c$-topologically injective}, if there exist $c > 0$ such that
    for all $x \in E$ holds $\Vert x\Vert_E\leq c\Vert  T(x)\Vert_F$.
    \newline
    $(ii)$ \textit{(strictly) $c$-topologically surjective}, if for all $c'>c$
    and  $y\in F$ there exist $x \in E$ such that $ T(x) = y$ and $\Vert x
        \Vert_E < c' \Vert y \Vert_F$ ($\Vert x \Vert_E \leq c \Vert y \Vert_F$).
    \newline
    $(iii)$ (strictly) coisometric, if it is contractive and (strictly)
    $1$-topologically surjective.
\end{definition}

\begin{remark} If the constant $c$ is out of interest then we will simply say
    that operator is topologically injective or topologically surjective. In
    this case also there is no difference between topologically surjective and
    strictly topologically surjective operators.
\end{remark}

For a given $E\subset \Omega$ and $f\in L_0(E,\mu|_{E})$ by $\widetilde{f}$ we
will denote the function such that $\widetilde{f}(\omega)=f(\omega)$ if
$\omega\in E$ and $\widetilde{f}(\omega)=0$ otherwise.

\begin{proposition}\label{MultOpDecompDecomp} Let $(\Omega,\Sigma,\mu)$,
    $(\Omega,\Sigma,\nu)$ be measure spaces and $p,q\in[1,+\infty]$. Assume we
    have a represetation $\Omega=\bigcup_{\lambda\in\Lambda}\Omega_\lambda$ of
    $\Omega$ as finite disjoint union of measurable sets. Then

    $(i)$ operator $M_g$ be $c$-topologically injective for some $c>0$ if and
    only if operators $M_g^{\Omega_\lambda}$ are $c'$-topologically injective
    for all $\lambda\in\Lambda$ and some $c'>0$

    $(ii)$ operator $M_g$ is $c$-topologically surjective for some $c>0$ if and
    only if operators $M_g^{\Omega_\lambda}$ are $c'$-topologically surjective
    for all $\lambda\in\Lambda$ and some $c'>0$

    $(iii)$ if operator $M_g$ is isometric then so are $M_g^{\Omega_\lambda}$
    for all $\lambda\in\Lambda$

    $(iv)$ if operator $M_g$ is coisometric then so are $M_g^{\Omega_\lambda}$
    for all $\lambda\in\Lambda$

\end{proposition}
\begin{proof}
    $(i)$ Let $M_g$ is $c$-topologically injective. Fix $\lambda\in\Lambda$ and
    $f\in L_p(\Omega_\lambda,\mu|_{\Omega_\lambda})$, then
    $$
        \Vert
        M_g^{\Omega_\lambda}(f)
        \Vert_{L_q(\Omega_\lambda,\nu|_{\Omega_\lambda})}
        =\Vert g\cdot \widetilde{f}\Vert_{L_q(\Omega,\nu)}
        \geq c^{-1}\Vert\widetilde{f}\Vert_{L_p(\Omega,\mu)}
        =c^{-1}\Vert f\Vert_{L_p(\Omega_\lambda,\mu|_{\Omega_\lambda})}
    $$
    So $M_g^{\Omega_\lambda}$ is $c$-topologically injective for all
    $\lambda\in\Lambda$.

    Conversely, assume operators $ \{M_g^{\Omega_\Lambda}:\lambda\in\Lambda \}$
    are $c'$-topologically injective. Let $f\in L_p(\Omega,\mu)$. Using
    propositions~\ref{LpSpDecomp},~\ref{FinDimlpEquivNorms} we get
    $$
        \Vert M_g(f)\Vert_{L_q(\Omega,\nu)}
        =\left\Vert\left(
        \Vert M_g^{\Omega_\lambda}
        (f|_{\Omega_\lambda})\Vert_{
        L_q(\Omega_\lambda,\nu|_{\Omega_\lambda})}:
        \lambda\in\Lambda\right)
        \right\Vert_{\ell_q(\Lambda)}
    $$
    $$
        \geq {(c')}^{-1}\left\Vert\left(
        \Vert f|_{\Omega_\lambda}\Vert_{
        L_p(\Omega_\lambda,\mu|_{\Omega_\lambda})}
        :\lambda\in\Lambda\right)
        \right\Vert_{\ell_q(\Lambda)}
    $$
    $$
        \geq {(c')}^{-1} C_{p,q}^{-1}\left\Vert\left(
        \Vert f|_{\Omega_\lambda}\Vert_{
        L_p(\Omega_\lambda,\mu|_{\Omega_\lambda})}:
        \lambda\in\Lambda\right)
        \right\Vert_{\ell_p(\Lambda)}
        ={(c')}^{-1}C_{p,q}^{-1}\Vert f\Vert_{L_p(\Omega,\mu)}
    $$
    Since $f$ is arbitrary $M_g$ is $c$-topologically injective for
    $c=c'C_{p,q}>0$.

    $(ii)$ Let $M_g$ is $c$-topologically surjective. Fix $\lambda\in\Lambda$
    and $h\in L_q(\Omega_\lambda,\nu|_{\Omega_\lambda})$. Then there exist
    $\widetilde{f}\in L_p(\Omega,\mu)$ such that
    $M_g(\widetilde{f})=\widetilde{h}$ and $\Vert
        \widetilde{f}\Vert_{L_p(\Omega,\mu)}\leq c\Vert
        \widetilde{h}\Vert_{L_q(\Omega,\nu)}$. Consider
    $f=\widetilde{f}|_{\Omega_\lambda}$, then
    $M_g^{\Omega_\lambda}(f)=\widetilde{h}|_{\Omega_\lambda}=h$ and $\Vert
        f\Vert_{L_p(\Omega_\lambda,\mu|_{\Omega_\lambda})}\leq \Vert
        \widetilde{f}\Vert_{L_p(\Omega,\mu)}\leq
        c\Vert\widetilde{h}\Vert_{L_q(\Omega,\nu)}=c\Vert
        h\Vert_{L_q(\Omega_\lambda,\nu|_{\Omega_\lambda})}$.
    Since $h$  is arbitrary then $M_g^{\Omega_\lambda}$ is $c$-topologically
    surjective for all $\lambda\in\Lambda$.

    Conversely, assume operators $ \{M_g^{\Omega_\Lambda}:\lambda\in\Lambda \}$
    are $c'$-topologically surjective. Let $h\in L_q(\Omega,\nu)$. From
    assumption for each $\lambda\in\Lambda$ we have $f_\lambda\in
        L_p(\Omega_\lambda,\mu|_{\Omega_\lambda})$ such that
    $M_g^{\Omega_\lambda}(f_\lambda)=h|_{\Omega_\lambda}$ and $\Vert
        f_\lambda\Vert_{L_p(\Omega_\lambda,\mu|_{\Omega_\lambda})}\leq c'\Vert
        h|_{\Omega_\lambda}\Vert_{L_q(\Omega_\lambda,\nu|_{\Omega_\lambda})}$.
    Define $f\in L_0(\Omega,\mu)$ such that $f(\omega)=f_\lambda(\omega)$ if
    $\omega\in\Omega_\lambda$.  Using
    propositions~\ref{LpSpDecomp},~\ref{FinDimlpEquivNorms} we get
    $$
        \Vert f\Vert_{L_p(\Omega,\mu)}
        =\left\Vert
        \left(
        \Vert f_\lambda\Vert_{
        L_p(\Omega_\lambda,\mu|_{\Omega_\lambda})
        }:\lambda\in\Lambda
        \right)\right\Vert_{\ell_p(\Lambda)}
        \leq c'\left\Vert\left(\Vert h|_{\Omega_\lambda}\Vert_{
        L_q(\Omega_\lambda,\nu|_{\Omega_\lambda})}
        :\lambda\in\Lambda\right)\right\Vert_{\ell_p(\Lambda)}
    $$
    $$
        \leq c'C_{p,q}\left\Vert\left(
        \Vert h|_{\Omega_\lambda}\Vert_{
        L_q(\Omega_\lambda,\nu|_{\Omega_\lambda})
        }:\lambda\in\Lambda
        \right)\right\Vert_{\ell_q(\Lambda)}
        =c'C_{p,q}\Vert h\Vert_{L_q(\Omega,\nu)}
    $$
    Obviously, $M_g(f)=h$. Since $h$ is arbitrary we get that $M_g$ is
    $c$-topologiclly surjective for $c=c'C_{p,q}>0$.

    $(iii)$ Fix $\lambda\in\Lambda$ and $f\in
        L_p(\Omega_\lambda,\mu|_{\Omega_\lambda})$, then
    $$
        \Vert M_g^{\Omega_\lambda}(f)\Vert_{
        L_q(\Omega_\lambda,\nu|_{\Omega_\lambda})}
        =\Vert g\cdot \widetilde{f}\Vert_{L_q(\Omega,\nu)}
        =\Vert\widetilde{f}\Vert_{L_p(\Omega,\mu)}
        =\Vert f\Vert_{L_p(\Omega_\lambda,\mu|_{\Omega_\lambda})}
    $$
    So $M_g^{\Omega_\lambda}$ is isometric for all $\lambda\in\Lambda$

    $(iv)$ Fix $\lambda\in\Lambda$. Since $M_g$ is coisometric it is
    $1$-topologically surjective and contractive. So from paragraph $(ii)$ we
    see that $M_g^{\Omega_\lambda}$ is $1$-topologically surjective. Let $f\in
        L_p(\Omega_\lambda,\mu|_{\Omega_\lambda})$. Since $M_g$ is contractive we
    get
    $$
        \Vert M_g^{\Omega_\lambda}(f)\Vert_{
        L_q(\Omega_\lambda,\nu|_{\Omega_\lambda})}
        =\Vert M_g(\widetilde{f})\chi_{\Omega_\lambda}\Vert_{L_q(\Omega,\nu)}
        =\Vert M_g(\widetilde{f}\chi_{\Omega_\lambda})\Vert_{L_q(\Omega,\nu)}
        \leq \Vert\widetilde{f}\chi_{\Omega_\lambda}\Vert_{L_p(\Omega,\mu)}
        =\Vert f\Vert_{L_p(\Omega_{\lambda},\mu|_{\Omega_\lambda})}
    $$
    Since $M_g^{\Omega_\lambda}$ is contractive and $1$-topologically injective
    it is coisometric.
\end{proof}


\begin{proposition}\label{MultOpCharacBtwnTwoSingMeasSp} Let
    $(\Omega,\Sigma,\mu)$ and $(\Omega,\Sigma,\nu)$ be two $\sigma$-finite
    measure spaces. Let $p,q\in[1,+\infty]$ and $g\in L_0(\Omega,\mu)$. If
    $\mu\perp\nu$ then $M_g:L_p(\Omega,\mu)\to L_q(\Omega,\nu_s)$ is zero
    operator.
\end{proposition}
\begin{proof} Since $\mu\perp\nu$, then there exist $\Omega_s\in\Sigma$ such
    that $\mu(\Omega_s)=\nu(\Omega_c)=0$, where
    $\Omega_c=\Omega\setminus\Omega_a$. Since $\mu(\Omega_s)=0$, then
    $\chi_{\Omega_c}=\chi_{\Omega}$ in $L_p(\Omega,\mu)$ and $\chi_{\Omega_c}=0$
    in $L_q(\Omega,\nu)$. Now for all $f\in L_p(\Omega,\mu)$ we have
    $M_g(f)=M_g(f\cdot \chi_{\Omega})=M_g(f\cdot \chi_{\Omega_c})=g\cdot
        f\cdot\chi_{\Omega_c}=0$. Since $f$ is arbitrary $M_g=0$.
\end{proof}



%-------------------------------------------------------------------------------
% Properties of mutiplication operators
%-------------------------------------------------------------------------------

\section{Properties of multiplication operators}

%-------------------------------------------------------------------------------
% Topologically injective and isometric operators
%-------------------------------------------------------------------------------

\subsection{Topologically injective and isometric operators}

This is the main section for subsequent study of multiplication operators. In
the end we will show that topologically injective multiplication operators are
isomorphisms or invertible from the left.

\begin{proposition}\label{MulpOpPropIfPeqqualsQ} Let $(\Omega,\Sigma,\mu)$ be a
    measure space and $g\in L_0(\Omega,\mu)$. Let $p=q$, then

    $(i)$ $M_g\in\mathcal{B}(L_p(\Omega,\mu))$ if and only if $g\in
        L_\infty(\Omega,\mu)$.

    $(ii)$ $M_g$ is an isomorphism if and only if $C\geq |g|\geq c$ for some
    $C,c>0$.
\end{proposition}
\begin{proof}
    $(i)$ Assume $M_g\in\mathcal{B}(L_p(\Omega,\mu))$. Assume there exist
    $E\in\Sigma$ with $\mu(E)>0$ such that $|g|_E|>\Vert M_g\Vert$, then
    $$
        \Vert M_g(\chi_E)\Vert_{L_p(\Omega,\mu)}
        =\Vert g\cdot\chi_E\Vert_{L_p(\Omega,\mu)}
        >\Vert M_g\Vert\Vert\chi_E\Vert_{L_p(\Omega,\mu)}
    $$
    Contradiction, hence for all $E\in\Sigma$ with $\mu(E)>0$ we have
    $|g|_E|\leq \Vert M_g\Vert$ i.e.  $|g|\leq \Vert M_g\Vert$. Thus $g\in
        L_\infty(\Omega,\mu)$

    Conversely, let $g\in L_\infty(\Omega,\mu)$ then $|g|\leq C$ for some $C>0$.
    Now for any $p\in[1,+\infty]$ we have and all $f\in L_p(\Omega,\mu)$ we have
    $$
        \Vert M_g(f)\Vert_{L_p(\Omega,\mu)}
        =\Vert g\cdot f\Vert_{L_p(\Omega,\mu)}
        \leq C\Vert f\Vert_{L_p(\Omega,\mu)}
    $$
    Hence $M_g\in\mathcal{B}(L_p(\Omega,\mu))$

    $(ii)$ Note that $M_g^{-1}=M_{1/g}$ as linear maps provided $g$ is
    invertible. Now $M_g$ is an isomorphism if and only if $M_g$ and $M_g^{-1}$
    are bounded operators. From previous paragraph and equality
    $M_g^{-1}=M_{1/g}$ we see that it is equivalent to boundedness of $g$ and
    $1/g$. This is equivalent to $C\geq|g|\geq c$ for some $C,c>0$.
\end{proof}


\begin{proposition}\label{EquivMultOp} Let $(\Omega,\Sigma,\mu)$ be a
    $\sigma$-finite purely atomic measure space. Let $1\leq p,q\leq +\infty$ and
    $g\in L_0(\Omega,\mu)$, then the operator
    $\widetilde{M}_{\widetilde{g}}:=\widetilde{I}_q
        M_g\widetilde{I}_p^{-1}\in\mathcal{B}(\ell_p(\Lambda),\ell_q(\Lambda))$ is a
    multiplication operator by the function
    $\widetilde{g}:\Lambda\to\mathbb{C},\lambda\mapsto
        {\mu(\Omega_\lambda)}^{1/q-1/p-1}\int_{\Omega_\lambda}f(\omega)d\mu(\omega)$
    where $ \{\Omega_\lambda:\lambda\in\Lambda \}$ is at most countable
    decomposition of $\Omega$ into pairwise disjoint atoms guranteed by
    proposition~\ref{DescOfLpSpOnPureAtomMeasSp}.
\end{proposition}
\begin{proof} Let $p,q\in[1,+\infty]$. For any $x\in\ell_p(\Lambda)$ we have
    $$
        \widetilde{M}_{\widetilde{g}}(x)(\lambda)
        =(\widetilde{I}_q((M_g\widetilde{I}_p^{-1})(x))(\lambda))
        =J_q(M_g(\widetilde{I}_p^{-1}(x))|_{\Omega_\lambda})(1)
    $$
    $$
        =J_q((g\cdot\widetilde{I}_p^{-1}(x))|_{\Omega_\lambda})(1)
        ={\mu(\Omega_\lambda)}^{1/q-1}
        \int_{\Omega_\lambda}(g|_{\Omega_\lambda}\cdot
        \widetilde{I}_p^{-1}(x)|_{\Omega_\lambda})(\omega)d\mu(\omega)
    $$
    $$
        ={\mu(\Omega_\lambda)}^{1/q-1}
        \int_{\Omega_\lambda}(g\cdot {\mu(\Omega)}^{-1/p}x(\lambda)
        \chi_{\Omega_{\lambda}})(\omega)d\mu(\omega)
        =x(\lambda){\mu(\Omega_\lambda)}^{1/q-1/p-1}
        \int_{\Omega_\lambda} g(\omega)d\mu(\omega)
    $$
    Thus $\widetilde{M}_{\widetilde{g}}$ is a multiplication operator where
    $\widetilde{g}(\lambda)=
        {\mu(\Omega_\lambda)}^{1/q-1/p-1}
        \int_{\Omega_\lambda}g(\omega)d\mu(\omega)$
\end{proof}
\newline

Since $\widetilde{I}_p$ and $\widetilde{I}_q$ are isometric isomorphisms then
$M_g$ is topologically injective if and only if $\widetilde{M}_{\widetilde{g}}$
is topologically injective.

\begin{proposition}\label{TopInjMultOpCharacOnPureAtomMeasSp} Let
    $(\Omega,\Sigma,\mu)$ be $\sigma$-finite purely atomic measure space,
    $p,q\in[1,+\infty]$ and $g\in L_0(\Omega,\mu)$ then the following are
    equivalent
    \begin{enumerate}[label = (\roman*)]
        \item $M_g\in\mathcal{B}(L_p(\Omega,\mu),L_q(\Omega,\mu))$ is
              topologically injective;

        \item $|g|\geq c$ for some $c>0$ and if $p\neq q$ the space
              $(\Omega,\Sigma,\mu)$ have finitely many atoms.
    \end{enumerate}
\end{proposition}
\begin{proof}
    $(i)\implies (ii)$ Assume $M_g$ is topologically injective, then so is
    $\widetilde{M}_{\widetilde{g}}$, i.e.
    $\Vert\widetilde{M}_{\widetilde{g}}(x)\Vert_{\ell_q(\Lambda)}\geq c'\Vert
        x\Vert_{\ell_p(\Lambda)}$ for all $x\in\ell_p(\Lambda)$ and some $c'>0$. Now
    we use at most countable decomposition $ \{\Omega_\lambda:\lambda\in\Lambda
        \}$ of $\Omega$ into pairwise disjoint atoms of $\Omega$ given by
    proposition~\ref{DescOfLpSpOnPureAtomMeasSp}. Now we will consider two big
    cases.

    $(1)$ Let $p\neq q$. Assume $\Lambda$ is countable.

    $(1.1)$ Consider subcase $p,q<+\infty$. If $\Lambda$ is countable, then we
    get a  contradiction, because by Pitt's theorem (see proposition 2.1.6
    in~\cite{KalAlbTopicsBanSpTh}) there is no embeddings between
    $\ell_p(\Lambda)$ and $\ell_q(\Lambda)$ spaces for countable $\Lambda$ and
    $1\leq p,q<+\infty$, $p\neq q$.

    $(1.2)$ Consider subcase $q=+\infty$. Take any finite family
    $F\subset\Lambda$, then
    $$
        \sup_{\lambda\in\Lambda}|\widetilde{g}(\lambda)|
        \geq\max_{\lambda\in F}|\widetilde{g}(\lambda)|
        =\left\Vert\widetilde{M}_{\widetilde{g}}\left(
        \sum_{\lambda\in F}e_\lambda
        \right)\right\Vert_{\ell_\infty(\Lambda)}
        \geq c'\left\Vert
        \sum_{\lambda\in F}e_\lambda
        \right\Vert_{\ell_p(\Lambda)}
        =c'\operatorname{Card}(F)
    $$
    Since $\Lambda$ is countable
    $\sup_{\lambda\in\Lambda}|\widetilde{g}(\lambda)|\geq
        c'\sup_{F\subset\Lambda}\operatorname{Card}(F)=+\infty$. On the other hand,
    since $\widetilde{M}_{\widetilde{g}}$ is bounded we have
    $$
        \sup_{\lambda\in\Lambda}|\widetilde{g}(\lambda)|
        =\sup_{\lambda\in\Lambda}\Vert
        \widetilde{M}_{\widetilde{g}}(e_\lambda)\Vert_{\ell_\infty(\Lambda)}
        \leq\Vert\widetilde{M}_{\widetilde{g}}\Vert\Vert
        e_\lambda\Vert_{\ell_p(\Lambda)}
        =\Vert\widetilde{M}_{\widetilde{g}}\Vert<+\infty
    $$
    Contradiction.

    $(1.3)$ Consider subcase $p=+\infty$. Since $\Lambda$ is countable then
    $\ell_\infty(\Lambda)$ is non separable and $\ell_q(\Lambda)$ is separable.
    As $\widetilde{M}_{\widetilde{g}}$ is topologically injective,
    then $\operatorname{Im}(\widetilde{M}_{\widetilde{g}})$ is non separable
    subspace of $\ell_q(\Lambda)$. Contradiction.

    In all subcases we got contradiction, hence $\Lambda$ is finite i.e.
    $(\Omega,\Sigma,\mu)$ have finitely many atoms. From
    lemma~\ref{FuncDescOnAtom} we see that $g$ is completely determined by its
    values $k_\lambda\in\mathbb{C}$ on atoms
    $ \{\Omega_\lambda:\lambda\in\Lambda \}$. By
    proposition~\ref{MultpOpSurjInjDesc} the function $g$ is zero only on
    sets of measre zero, so $k_\lambda\neq 0$ for all $\lambda\in\Lambda$.
    Since $\Lambda$ is finite we conclude $|g|\geq c$
    for $c=\min_{\lambda\in\Lambda}|k_\lambda|$.


    $(2)$ Let $p=q$. Fix $\lambda\in\Lambda$, then
    $$
        |\widetilde{g}(\lambda)| =\Vert \widetilde{g}\cdot
        e_\lambda\Vert_{\ell_q(\Lambda)} =\Vert
        \widetilde{M}_{\widetilde{g}}(e_\lambda)\Vert_{\ell_q(\Lambda)} \geq
        c'\Vert e_\lambda\Vert_{\ell_p(\Lambda)} =c'
    $$
    From lemma~\ref{FuncDescOnAtom} for $\mu$-almost all
    $\omega\in\Omega_\lambda$ we have
    $$
        |g(\omega)|
        =\left|{\mu(\Omega_\lambda)}^{-1}\int_{\Omega_\lambda}
        g(\omega)d\mu(\omega)\right|
        =\left|{\mu(\Omega_\lambda)}^{-1}{\mu(\Omega_\lambda)}^{1+1/p-1/p}
        \widetilde{g}(\lambda)\right|
        =|\widetilde{g}(\lambda)|\geq c'
    $$
    Since $\lambda\in\Lambda$ is arbitrary and
    $\Omega=\bigcup_{\lambda\in\Lambda}\Omega_\lambda$, then $|g|\geq c'$.

    $(ii)\implies (i)$ Assume $|g|\geq c$ for $c>0$. Then from
    proposition~\ref{EquivMultOp} we see that $|\widetilde{g}|\geq c$.

    $(1)$ Let $p\neq q$. Then we additionally assume that $(\Omega,\Sigma,\mu)$
    have finitely many atoms. Now from
    proposition~\ref{DescOfLpSpOnPureAtomMeasSp} we get that $L_p(\Omega,\mu)$
    is finite dimensional. From assumption on $g$ we see it is have no zero
    values. Hence operator $M_g$ is topologically injective.

    $(2)$ Let $p=q$, then for all $x\in\ell_p(\Lambda)$ we have
    $$
        \Vert \widetilde{M}_{\widetilde{g}}\Vert_{\ell_p(\Lambda)}=\Vert g\cdot
        x\Vert_{\ell_p(\Lambda)}\geq c\Vert x\Vert_{\ell_p(\Lambda)}
    $$
    so $\widetilde{M}_{\widetilde{g}}$ is topologically injective and
    so does $M_g$.
\end{proof}

\begin{proposition}\label{TopInjMultOpCharacOnNonAtomMeasSp} Let
    $(\Omega,\Sigma,\mu)$ be a nonatomic measure space, $p,q\in[1,+\infty]$ and
    $g\in L_0(\Omega,\mu)$ then the following are equivalent
    \begin{enumerate}[label = (\roman*)]
        \item $M_g\in\mathcal{B}(L_p(\Omega,\mu),L_q(\Omega,\mu))$ is
              topologically injective

        \item $|g|\geq c$ for some $c>0$ and $p=q$.
    \end{enumerate}
\end{proposition}
\begin{proof}
    $(i)\implies (ii)$ Assume $M_g$ is topologically injective i.e.
    $\Vert M_g(f)\Vert_{L_q(\Omega,\mu)}\geq c\Vert f\Vert_{L_p(\Omega,\mu)}$
    for some $c>0$ and all $f\in L_p(\Omega,\mu)$. We will consider three cases.

    $(1)$  Let $p>q$. There exist $C>0$ and $E\in\Sigma$ with $\mu(E)>0$ such
    that $|g|_E|\leq C$, otherwise $M_g$ is not well defined. By
    theorem~\ref{ContOfNonAtmMeas} we have a sequence
    $ \{E_n:n\in\mathbb{N} \}\subset\Sigma$ of subsets of $E$ such that
    $\mu(E_n)=2^{-n}$. Then since $p>q$ we get
    $$
        c \leq\frac{\Vert M_g(\chi_{E_n})\Vert_{L_q(\Omega,\mu)}}{\Vert
            \chi_{E_n}\Vert_{L_p(\Omega,\mu)}}
        \leq\frac{C\chi_{E_n}\Vert_{L_q(\Omega,\mu)}}{\Vert
            \chi_{E_n}\Vert_{L_p(\Omega,\mu)}} \leq C{\mu(E_n)}^{1/q-1/p}
    $$
    $$
        c \leq\inf_{n\in\mathbb{N}}C{\mu(E_n)}^{1/q-1/p} =C\inf_{n\in\mathbb{N}}
        2^{n(1/p-1/q)}=0
    $$
    Contradiction, so in this case $M_g$ can not be topologically injective.

    $(2)$ Let $p=q$. Fix $c'<c$. Assume there exist $E\in\Sigma$ with
    $\mu(E)>0$ and $|g|_{E}|<c'$, then
    $$
        \Vert M_g(\chi_{E})\Vert_{L_p(\Omega,\mu)} =\Vert g
        \cdot\chi_{E}\Vert_{L_p(\Omega,\mu)} \leq c' \Vert
        \chi_{E}\Vert_{L_p(\Omega,\mu)} <c\Vert \chi_{E}\Vert_{L_p(\Omega,\mu)}
    $$
    Contradiction. Since $c'<c$ is arbitrary we conclude $|g|_E|\geq c$ for any
    $E\in\Sigma$ with $\mu(E)>0$. Thus $|g|\geq c$.

    $(3)$ Let $p<q$. Assume we have some $c'>0$ and $E\in\Sigma$ such that
    $\mu(E)>0$, $|g|_E|>c'$. By theorem~\ref{ContOfNonAtmMeas} we have a
    sequence  $ \{E_n:n\in\mathbb{N} \}\subset\Sigma$ of subsets of $E$ such
    that $\mu(E_n)=2^{-n}$. Then from inequality $p<q$ we get
    $$
        \Vert M_g\Vert \geq\frac{\Vert
            M_g(\chi_{E_n})\Vert_{L_q(\Omega,\mu)}}{\Vert
            \chi_{E_n}\Vert_{L_p(\Omega,\mu)}}
        \geq\frac{c'\Vert\chi_{E_n}\Vert_{L_q(\Omega,\mu)}}{\Vert
            \chi_{E_n}\Vert_{L_p(\Omega,\mu)}} \geq c'{\mu(E_n)}^{1/q-1/p}
    $$
    $$
        \Vert M_g\Vert \geq\sup_{n\in\mathbb{N}}c'{\mu(E_n)}^{1/q-1/p} \geq
        c'\sup_{n\in\mathbb{N}}2^{n(1/p-1/q)} =+\infty
    $$
    Contradiction, hence $g=0$. In this case by
    proposition~\ref{MultpOpSurjInjDesc} operator $M_g$ is not topologically
    injective.

    $(ii)\implies (i)$ Conversely, assume $|g|\geq c$ for $c>0$ and $p=q$.
    Then for all $f\in L_p(\Omega,\mu)$ we have
    $$
        \Vert M_g(f)\Vert_{L_p(\Omega,\mu)} =\Vert g\cdot
        f\Vert_{L_p(\Omega,\mu)} \geq c\Vert f\Vert_{L_p(\Omega,\mu)}
    $$
    So $M_g$ is topologically injective.
\end{proof}

\begin{theorem}\label{TopInjMultOpCharacOnMeasSp} Let $(\Omega,\Sigma,\mu)$ be
    a $\sigma$-finite measure space, $p,q\in[1,+\infty]$ and
    $g\in L_0(\Omega,\Sigma,\mu)$, then the following are equivalent
    \begin{enumerate}[label = (\roman*)]
        \item $M_g\in\mathcal{B}(L_p(\Omega,\mu),L_q(\Omega,\mu))$ is
              toplogically injective

        \item $M_g$ is an isomorphism

        \item $|g|\geq c$ for some $c>0$, if $p\neq q$ the space
              $(\Omega,\Sigma,\mu)$ consist of finitely many atoms
    \end{enumerate}
\end{theorem}
\begin{proof} $(i)\implies (iii)$
    By proposition~\ref{DescOfLpSpOnMeasSp} we have decomposition
    $\Omega=\Omega_{a}\cup\Omega_{na}$, where
    $(\Omega_{na},\Sigma|_{\Omega_{na}},\mu|_{\Omega_{na}})$ is a nonatomic
    measure space and $(\Omega_{a},\Sigma|_{\Omega_{a}},\mu|_{\Omega_{a}})$ is
    a purely atomic measure space. By proposition~\ref{MultOpDecompDecomp}
    operator $M_g$ is topologically injective if and only if so does
    $M_g^{\Omega_{a}}$ and $M_g^{\Omega_{na}}$.
    Propositions~\ref{TopInjMultOpCharacOnPureAtomMeasSp}
    and~\ref{TopInjMultOpCharacOnNonAtomMeasSp} give necessary and sufficient
    conditions for $M_g^{\Omega_{a}}$ and $M_g^{\Omega_{na}}$ to be
    topologically injective. So we get the result.

    $(i)\implies (ii)$ Assume $M_g$ is topologically injective. If $p=q$
    from considerations above it follows that $|g|\geq c$ for some $c>0$.
    Since $M_g$ is bounded from proposition~\ref{MulpOpPropIfPeqqualsQ} we also
    have $C\geq |g|$ for some $C>0$. Now from the same proposition we conclude
    that $M_g$ is an isomorphism because $C\geq|g|\geq c$. Assume $p\neq q$,
    then from previous paragraph the space $(\Omega,\Sigma,\mu)$ consist of
    finite amount of atoms and $g$ is injective. Hence from
    proposition~\ref{DescOfLpSpOnPureAtomMeasSp} we get
    $\operatorname{dim}(L_p(\Omega,\Sigma,\mu))
        =\operatorname{dim}(\ell_p(\Lambda))
        =\operatorname{Card}(\Lambda)<+\infty$. Similarly,
    $\operatorname{dim}(L_q(\Omega,\Sigma,\mu))
        =\operatorname{Card}(\Lambda)<+\infty$ Since $g$ is injective by
    proposition~\ref{MultpOpSurjInjDesc} operator $M_g$ is injective.
    Thus $M_g$ is an injective operator between finite dimensional spaces of
    equal dimension. Hence it is an isomorphism.

    $(ii)\implies (i)$ Conversely, if $M_g$ is an isomorphism, clearly, it is
    topologically injective.
\end{proof}

\begin{proposition}\label{TopInjMultOpCharacBtwnTwoContMeasSp}
    Let $(\Omega,\Sigma,\mu)$ be a $\sigma$-finite measure space,
    $p,q\in[1,+\infty]$ and $g,\rho\in L_0(\Omega,\mu)$ and $\rho$ is non
    negative then the following are equivalent
    \begin{enumerate}[label = (\roman*)]
        \item $M_g\in\mathcal{B}(L_p(\Omega,\mu),L_q(\Omega,\rho\cdot\mu))$ is
              topologically injective

        \item $M_g$ is an isomorphism

        \item $\rho$ is  positive, $|g\cdot \rho^{1/q}|\geq c$ for some
              $c>0$, if $p\neq q$ the space $(\Omega,\Sigma,\mu)$ consist
              of finitely many atoms.
    \end{enumerate}
\end{proposition}
\begin{proof} $(i)\implies (iii)$ Consider set $E=\rho^{-1}( \{0 \})$.
    Assume $\mu(E)>0$ then $\chi_E\neq 0$ in $L_p(\Omega,\mu)$. On the other
    hand $(\rho\cdot\mu)(E)=\int_E\rho(\omega)d\mu(\omega)=0$, so $\chi_E=0$ in
    $L_q(\Omega,\rho\cdot\mu)$ and $M_g(\chi_E)=g\cdot\chi_E=0$ in
    $L_q(\Omega,\rho\cdot\mu)$. Thus we see that $M_g$ is not injective and as
    the consequence it is not topologically injective. Contradiction, so
    $\mu(E)=0$ and $\rho$ is  positive. Hence by proposition~\ref{ChngOfDenst}
    we have an isometric isomorphism
    $\bar{I}_q:L_q(\Omega,\mu)\to L_q(\Omega,\rho\cdot\mu),
        f\mapsto \rho^{-1/q}\cdot f$. Obviously
    $M_{g\cdot\rho^{1/q}}
        =\bar{I}_q^{-1} M_g\in\mathcal{B}(L_p(\Omega,\mu),L_q(\Omega,\mu))$. Since
    $\bar{I}_q$ is an isometric isomorphism and $M_g$ is topologically
    injective, then $M_{g\cdot \rho^{1/q}}$ is topologically injective. From
    theorem~\ref{TopInjMultOpCharacOnMeasSp} we get that
    $|g\cdot\rho^{1/q}|\geq c$ for some $c>0$ and if $p\neq q$ the space
    is $(\Omega,\Sigma,\mu)$ consist of finite amount of atoms.

    $(iii)\implies (i)$ By theorem~\ref{TopInjMultOpCharacOnMeasSp} operator
    $M_{g\cdot\rho^{1/q}}$ is topologically injective. Since $\rho$ is positive
    by proposition~\ref{ChngOfDenst} we have an isometric isomorohism
    $\bar{I}_q$. Then from equality $M_g=\bar{I}_q M_{g\cdot\rho^{1/q}}$ it
    follows that $M_g$ is also topologically injective.

    $(i)\implies (ii)$ As we proved above this implies that
    $M_{g\cdot\rho^{1/q}}$ is topologically injective and $\bar{I}_q$ is an
    isometric isomorphism. By theorem~\ref{TopInjMultOpCharacOnMeasSp}
    $M_{g\cdot\rho^{1/q}}$ is an isomorphism. Since
    $M_g=\bar{I}_q M_{g\cdot\rho^{1/q}}$ and $\bar{I}_q$ is an isometric
    isomorphism, then $M_g$ is also an isomorphism.

    $(ii)\implies (i)$ If $M_g$ is an isomorphism, then, obviously, it is
    topologically injective.
\end{proof}

\begin{theorem}\label{TopInjMultOpCharacBtwnTwoMeasSp}
    Let $(\Omega,\Sigma,\mu)$, $(\Omega,\Sigma,\nu)$ be two $\sigma$-finite
    measure spaces, $p,q\in[1,+\infty]$ and $g\in L_0(\Omega,\mu)$, then the
    following are eqivalent
    \begin{enumerate}[label = (\roman*)]
        \item $M_g\in\mathcal{B}(L_p(\Omega,\mu), L_q(\Omega,\nu))$ is
              topologically injective

        \item $M_g^{\Omega_c}$ is an isomorphism

        \item $\rho_{\nu,\mu}$ is positive,
              $|g\cdot\rho_{\nu,\mu}^{1/q}|_{\Omega_c}|\geq c$ for some $c>0$,
              if $p\neq q$ the space $(\Omega,\Sigma,\mu)$ consist of
              finitely many atoms.
    \end{enumerate}
\end{theorem}
\begin{proof}
    By proposition~\ref{MultOpDecompDecomp} operator $M_g$ is topologically
    injective if and only if operators
    $M_g^{\Omega_c}:L_p(\Omega_c,\mu|_{\Omega_c})\to
        L_q(\Omega_c,\rho_{\nu,\mu}\cdot\mu|_{\Omega_c})$ and
    $M_g^{\Omega_s}:L_p(\Omega_s,\mu|_{\Omega_s})\to
        L_q(\Omega_s,\nu_s|_{\Omega_s})$ are topologically injective. By
    proposition~\ref{MultOpCharacBtwnTwoSingMeasSp} operator $M_g^{\Omega_s}$ is
    zero. Since $\mu(\Omega_s)=0$, then $L_p(\Omega_s,\mu|_{\Omega_s})= \{0 \}$.
    From these two facts we conclude that $M_g^{\Omega_s}$ is topologically
    injective. Thus topological injectivity of $M_g$ is equivalent to
    topological injectivity of  $M_g^{\Omega_c}$. It is remains to apply
    proposition~\ref{TopInjMultOpCharacBtwnTwoContMeasSp}.
\end{proof}

\begin{theorem}\label{TopInjMultOpDescBtwnTwoMeasSp}
    Let $(\Omega,\Sigma,\mu)$, $(\Omega,\Sigma,\nu)$ be two $\sigma$-finite
    measure spaces, $p,q\in[1,+\infty]$ and $g\in L_0(\Omega,\mu)$, then
    the following are equivalent
    \begin{enumerate}[label = (\roman*)]
        \item $M_g\in\mathcal{B}(L_p(\Omega,\mu),L_q(\Omega,\nu))$ is
              topologically injective

        \item $M_{\chi_{\Omega_c}/g}\in
                  \mathcal{B}(L_q(\Omega,\nu), L_p(\Omega,\mu))$ its left inverse
              topologically surjective operator
    \end{enumerate}
\end{theorem}
\begin{proof}
    $(i)\implies (ii)$ By proposition~\ref{MultOpDecompDecomp}
    $M_g^{\Omega_c}$ is topologically injective. By
    proposition~\ref{TopInjMultOpCharacBtwnTwoContMeasSp} operator
    $M_g^{\Omega_c}$ is invertible and
    ${(M_g^{\Omega_c})}^{-1}=M_{1/g}^{\Omega_c}$. Then for
    all $h\in L_q(\Omega,\nu)$ we have
    $$
        \Vert M_{\chi_{\Omega_c}/g}(h)\Vert_{L_p(\Omega,\mu)}= \Vert
        M_{1/g}(h)\chi_{\Omega_c}\Vert_{L_p(\Omega,\mu)}= \Vert
        M_{1/g}^{\Omega_c}(h|_{\Omega_c})\Vert_{L_p(\Omega_c,\mu|_{\Omega_c})}
    $$
    $$
        \leq\Vert M_{1/g}^{\Omega_c}\Vert\Vert
        h|_{\Omega_c}\Vert_{L_q(\Omega_c,\nu|_{\Omega_c})} \leq\Vert
        M_{1/g}^{\Omega_c}\Vert\Vert h\Vert_{L_q(\Omega,\nu)}
    $$
    So $M_{\chi_{\Omega_c}/g}$ is bounded. Now note that for
    all $f\in L_p(\Omega,\mu)$ we have
    $$
        M_{\chi_{\Omega_c}/g}(M_g(f)) =M_{\chi_{\Omega_c}/g}(g\cdot f)
        =(\chi_{\Omega_c}/g)\cdot g\cdot f =f\cdot\chi_{\Omega_c}
    $$
    Since $\mu(\Omega\setminus\Omega_c)=0$, then
    $\chi_{\Omega_c}=\chi_{\Omega}$, so
    $M_{\chi_{\Omega_c}/g}(M_g(f))=f\cdot\chi_{\Omega_c}=f\cdot\chi_{\Omega}=f$.
    This means that $M_g$ have left inverse multiplication operator. Take
    any $f\in L_p(\Omega,\mu)$, then for $h=M_g(f)$ we have
    $M_{\chi_{\Omega_c}/g}(h)=f$ and
    $\Vert h\Vert_{L_q(\Omega,\nu)}
        \leq\Vert M_g\Vert\Vert f\Vert_{L_p(\Omega,\mu)}$. Since $h$ is
    arbitrary $M_{\chi_{\Omega_c}/g}$ is topologically surjective.

    Conversely if $M_g$ have left inverse $M_{\chi_{\Omega_c}/g}$
    then for all $f\in L_p(\Omega,\mu)$ we have
    $$
        \Vert M_g(f)\Vert_{L_q(\Omega,\nu)} \geq\Vert
        M_{\chi_{\Omega_c}/g}\Vert^{-1}\Vert
        M_{\chi_{\Omega_c}/g}(M_g(f))\vert_{L_p(\Omega,\mu)} \geq\Vert
        M_{\chi_{\Omega_c}/g}\Vert^{-1}\Vert f\Vert_{L_p(\Omega,\mu)}
    $$
    So $M_g$ is topologically injective.
\end{proof}


\begin{proposition}\label{IsomMultOpCharacOnMeasSp}
    Let $(\Omega,\Sigma,\mu)$ be a $\sigma$-finite measure space,
    $p,q\in[1,+\infty]$ and $g\in L_0(\Omega,\mu)$, then the following
    are equivalent
    \begin{enumerate}[label = (\roman*)]
        \item $M_g\in\mathcal{B}(L_p(\Omega,\mu),L_q(\Omega,\mu))$ is
              an isometry

        \item $|g|={\mu(\Omega)}^{1/p-1/q}$, if $p\neq q$,
              then $(\Omega,\Sigma,\mu)$ consist of single atom.
    \end{enumerate}
\end{proposition}
\begin{proof} $(i)\implies (ii)$ Let $p=q$. Assume there exist
    $E\in\Sigma$ with $\mu(E)>0$ such that $|g|_E|<1$, then
    $$
        \Vert M_g(\chi_E)\Vert_{L_p(\Omega,\mu)} =\Vert
        g\cdot\chi_E\Vert_{L_p(\Omega,\mu)} <\Vert\chi_E\Vert_{L_p(\Omega,\mu)}
        =\Vert M_g(\chi_E)\Vert_{L_p(\Omega,\mu)}
    $$
    Contradiction, hence for all $E\in\Sigma$ with $\mu(E)>0$ we
    have $|g|_E|\geq 1$ i.e.  $|g|\geq 1$. Assume there exist
    $E\in\Sigma$ with $\mu(E)>0$ such that $|g|_E|>1$, then
    $$
        \Vert M_g(\chi_E)\Vert_{L_p(\Omega,\mu)} =\Vert
        g\cdot\chi_E\Vert_{L_p(\Omega,\mu)} >\Vert\chi_E\Vert_{L_p(\Omega,\mu)}
        =\Vert M_g(\chi_E)\Vert_{L_p(\Omega,\mu)}
    $$
    Contradiction, hence for all $E\in\Sigma$ with $\mu(E)>0$ we have
    $|g|_E|\leq 1$ i.e.  $|g|\leq 1$. From both inequalities we get
    $|g|=1={\mu(\Omega)}^{1/p-1/q}$. Let $p\neq q$, then since $M_g$ is an
    isometry it is topologically injective. By
    theorem~\ref{TopInjMultOpCharacOnMeasSp} the space $(\Omega,\Sigma,\mu)$
    consist of finitely many atoms. Assume there is at least two disjoint
    atoms, say $\Omega_1$ and $\Omega_2$. By lemma~\ref{AtomDescInSigmFinMeasSp}
    they are of finite measure, so we can consider respective normalized
    functions
    $h_k={\Vert\chi_{\Omega_k}\Vert_{L_p(\Omega,\mu)}}^{-1}\chi_{\Omega_k}$
    where $k\in \{1,2 \}$. Since they these atoms are disjoint $h_1h_2=0$ and
    as the result $M_g(h_1)M_g(h_2)=0$. Note that for any $r\in[1,+\infty]$ and
    all $f_1,f_2\in L_r(\Omega,\mu)$ such that $f_1f_2=0$ we have
    $$
        \Vert f_1+f_2\Vert_{L_r(\Omega,\mu)} =\left\Vert\left(\Vert
        f_\lambda\Vert_{L_r(\Omega,\mu)}:\lambda\in \{1,2
        \}\right)\right\Vert_{\ell_r( \{1,2 \})}
    $$
    Hence
    $$
        \Vert M_g(h_1+h_2)\Vert_{L_q(\Omega,\mu)} =\Vert
        h_1+h_2\Vert_{L_p(\Omega,\mu)} =\left\Vert\left( 1 :\lambda\in \{1,2
        \}\right)\right\Vert_{\ell_p( \{1,2 \})} =2^{1/p}
    $$
    But on the other hand
    $$
        \Vert M_g(h_1+h_2)\Vert_{L_q(\Omega,\mu)} =\Vert
        M_g(h_1)+M_g(h_2)\Vert_{L_q(\Omega,\mu)} =\left\Vert\left(\Vert
        M_g(h_\lambda)\Vert_{L_q(\Omega,\mu)}:\lambda\in \{1,2
        \}\right)\right\Vert_{\ell_q( \{1,2 \})}
    $$
    $$
        =\left\Vert\left(\Vert h_\lambda\Vert_{L_p(\Omega,\mu)}:\lambda\in \{1,2
        \}\right)\right\Vert_{\ell_q( \{1,2 \})} =\left\Vert\left(1:\lambda\in
        \{1,2 \}\right)\right\Vert_{\ell_q( \{1,2 \})} =2^{1/q}
    $$
    Thus $2^{1/p}=2^{1/q}$. Contradiction, so $(\Omega,\Sigma,\mu)$ consist of
    single atom. In this case from lemma~\ref{FuncDescOnAtom} it follows that
    for all $f\in L_p(\Omega,\mu)$ we have
    $$
        \Vert M_g(f)\Vert_{L_q(\Omega,\mu)} =\Vert J_q(M_g(f))\Vert_{\ell_q( \{1
            \})} =\Vert J_q(g\cdot f)\Vert_{\ell_q( \{1 \})}
        ={\mu(\Omega)}^{1/q-1}\left|\int_\Omega g(\omega)
        f(\omega)d\mu(\omega)\right|
    $$
    $$
        \Vert f\Vert_{L_p(\Omega,\mu)} =\Vert J_p(f)\Vert_{\ell_p( \{1 \})}
        ={\mu(\Omega)}^{1/p-1}\left|\int_\Omega f(\omega)d\mu(\omega)\right|
    $$
    By $c$ we denote the constant value of $g$, then
    $$
        \Vert M_g(f)\Vert_{L_q(\Omega,\mu)}
        ={\mu(\Omega)}^{1/q-1}\left|\int_\Omega g(\omega)
        f(\omega)d\mu(\omega)\right|
        ={\mu(\Omega)}^{1/q-1}|c|\left|\int_\Omega
        f(\omega)d\mu(\omega)\right|
    $$
    From this equality we conclude that in this case $M_g$ is an isometry if
    $$
        |g|=|c|={\mu(\Omega)}^{1/p-1/q}
    $$

    $(ii)\implies (i)$. Let $p=q$, then $|g|=1$. So for
    all $f\in L_p(\Omega,\mu)$ we have
    $$
        \Vert M_g(f)\Vert_{L_p(\Omega,\mu)} =\Vert g\cdot
        f\Vert_{L_p(\Omega,\mu)} =\Vert |g|\cdot f\Vert_{L_p(\Omega,\mu)} =\Vert
        f\Vert_{L_p(\Omega,\mu)}
    $$
    hence $M_g$ is an isometry. Let $p\neq q$, then $(\Omega,\Sigma,\mu)$
    consist of single atom and we conclude
    $$
        \Vert M_g(f)\Vert_{L_q(\Omega,\mu)}
        ={\mu(\Omega)}^{1/q-1}\left|\int_\Omega g(\omega)
        f(\omega)d\mu(\omega)\right|={\mu(\Omega)}^{1/q-1}|c|\left|\int_\Omega
        f(\omega)d\mu(\omega)\right|
    $$
    $$
        ={\mu(\Omega)}^{1/p-1}\left|\int_\Omega f(\omega)d\mu(\omega)\right|
        =\Vert f\Vert_{L_p(\Omega,\mu)}
    $$
    hence $M_g$ is isometric.
\end{proof}

\begin{proposition}\label{IsomMultOpCharacBtwnTwoContMeasSp}
    Let $(\Omega,\Sigma,\mu)$ be a $\sigma$-finite measure space,
    $p,q\in[1,+\infty]$ and $g,\rho\in L_0(\Omega,\mu)$, and $\rho$ is non
    negative, then the following are equivalent
    \begin{enumerate}[label = (\roman*)]
        \item $M_g\in\mathcal{B}(L_p(\Omega,\mu), L_q(\Omega,\rho\cdot\mu))$
              is isometric

        \item $M_g$ is isometric isomorphism

        \item $\rho$ is positive, $|g\cdot \rho^{1/q}|={\mu(\Omega)}^{1/p-1/q}$
              and if $p\neq q$ the sapce $(\Omega,\Sigma,\mu)$ consist of single atom.
    \end{enumerate}
\end{proposition}
\begin{proof} $(i)\implies (iii)$ Since $M_g$ is isometric, it is topologically
    injective and by theorem~\ref{TopInjMultOpCharacBtwnTwoMeasSp} we see that
    $\rho$ is positive. Hence by proposition~\ref{ChngOfDenst} we have an
    isometric isomorphism
    $\bar{I}_q:L_q(\Omega,\mu)\to L_q(\Omega,\rho\cdot\mu),
        f\mapsto \rho^{-1/q}\cdot f$. Obviously
    $M_{g\cdot\rho^{1/q}}
        =\bar{I}_q^{-1} M_g\in\mathcal{B}(L_p(\Omega,\mu),L_q(\Omega,\mu))$. Since
    $\bar{I}_q$ is an isometric isomorphism and $M_g$ is isometric, then
    $M_{g\cdot \rho^{1/q}}$ is isometric too. It is remains to apply
    theorem~\ref{IsomMultOpCharacOnMeasSp}.

    $(iii)\implies (i)$ By theorem~\ref{IsomMultOpCharacOnMeasSp} operator
    $M_{g\cdot\rho^{1/q}}$ is isometric. Since $\rho$ is positive by
    proposition~\ref{ChngOfDenst} we have an isometric isomorohism $\bar{I}_q$.
    Then from equality $M_g=\bar{I}_q M_{g\cdot\rho^{1/q}}$ it follows
    that $M_g$ is also isometric.

    $(i)\implies (ii)$ Since $M_g$ is isometric, it is topologically
    injective and by proposition~\ref{TopInjMultOpCharacBtwnTwoContMeasSp} it
    is an isomorphism, which is isometric by assumption.

    $(ii)\implies (i)$ Since $M_g$ is an isometric isomorphism, trivially,
    it is isometric.
\end{proof}

\begin{theorem}\label{IsomMultOpCharacBtwnTwoMeasSp}
    Let $(\Omega,\Sigma,\mu)$, $(\Omega,\Sigma,\nu)$ be two $\sigma$-finite
    measure spaces, $p,q\in[1,+\infty]$ and $g\in L_0(\Omega,\mu)$, then the
    following are equivalent
    \begin{enumerate}[label = (\roman*)]
        \item $M_g$ is isometric

        \item $M_g^{\Omega_c}$ is isometric

        \item $\rho_{\nu,\mu}$ is positive,
              $|g\cdot \rho_{\nu,\mu}^{1/q}|_{\Omega_c}|={\mu(\Omega_c)}^{1/p-1/q}$
              and if  $p\neq q$ the space $(\Omega,\Sigma,\mu)$ consist
              of single atom.
    \end{enumerate}
\end{theorem}
\begin{proof}
    $(i)\implies (ii)\implies (iii)$ Since $M_g$ is isometric, by
    proposition~\ref{MultOpDecompDecomp} operator $M_g^{\Omega_c}$, is
    isometric. It is remains to apply
    proposition~\ref{IsomMultOpCharacBtwnTwoContMeasSp}.

    $(iii)\implies (i)$ By proposition~\ref{IsomMultOpCharacBtwnTwoContMeasSp}
    operator $M_g^{\Omega_c}$ is isometric. Now take arbitrary
    $f\in L_p(\Omega,\mu)$. Since $\mu(\Omega\setminus\Omega_c)=0$,
    then $\chi_{\Omega_c}=\chi_{\Omega}$ in $L_p(\Omega,\mu)$. As the
    result $f=f\chi_{\Omega}=f\chi_{\Omega_c}=f\chi_{\Omega_c}\chi_{\Omega_c}$
    in $L_p(\Omega,\mu)$ and $M_g(f)=M_g(f\chi_{\Omega_c})\chi_{\Omega_c}$.
    Thus using that $M_g^{\Omega_c}$ is isometric we get
    $$
        \Vert M_g(f)\Vert_{L_q(\Omega,\nu)} =\Vert
        M_g(f\chi_{\Omega_c})\chi_{\Omega_c}\Vert_{L_q(\Omega,\nu)} =\Vert
        M_g(f\chi_{\Omega_c})\Vert_{L_q(\Omega_c,\nu|_{\Omega_c})}
    $$
    $$
        =\Vert
        M_g^{\Omega_c}(f|_{\Omega_c})\Vert_{L_q(\Omega_c,\nu|_{\Omega_c})}
        =\Vert f|_{\Omega_c}\Vert_{L_p(\Omega_c,\mu|_{\Omega_c})}
    $$
    Since $\mu(\Omega\setminus\Omega_c)=0$ we have
    $\Vert f|_{\Omega_c}\Vert_{L_p(\Omega_c,\mu|_{\Omega_c})}
        =\Vert f\Vert_{L_p(\Omega,\mu)}$ so $\Vert M_g(f)\Vert_{L_q(\Omega,\nu)}
        =\Vert f\Vert_{L_p(\Omega,\mu)}$ i.e.  $M_g$ is isometric.
\end{proof}

\begin{theorem}\label{IsomMultOpDescBtwnTwoMeasSp}
    Let $(\Omega,\Sigma,\mu)$, $(\Omega,\Sigma,\nu)$ be two $\sigma$-finite
    measure spaces, $p,q\in[1,+\infty]$ and $g\in L_0(\Omega,\mu)$, then the
    following are equivalent
    \begin{enumerate}[label = (\roman*)]
        \item $M_g\in\mathcal{B}(L_p(\Omega,\mu),L_q(\Omega,\nu))$ is isometric

        \item $M_{\chi_{\Omega_c}/g}\in
                  \mathcal{B}(L_q(\Omega,\nu), L_p(\Omega,\mu))$ its left inverse
              strictly coisometric operator.
    \end{enumerate}
\end{theorem}
\begin{proof}
    $(i)\implies (ii)$ By proposition~\ref{MultOpDecompDecomp} operator
    $M_g^{\Omega_c}$ is isometric and by
    proposition~\ref{IsomMultOpCharacBtwnTwoContMeasSp} it is invertible with
    ${(M_g^{\Omega_c})}^{-1}=M_{1/g}^{\Omega_c}$. Since $M_g^{\Omega_c}$ is
    isometric then so does its inverse. Then for all $h\in L_q(\Omega,\nu)$
    we have
    $$
        \Vert M_{\chi_{\Omega_c}/g}(h)\Vert_{L_p(\Omega,\mu)}= \Vert
        M_{1/g}(h|_{\Omega_c})\Vert_{L_p(\Omega_c,\mu|_{\Omega_c})}= \Vert
        M_{1/g}^{\Omega_c}(h|_{\Omega_c})\Vert_{L_p(\Omega_c,\mu|_{\Omega_c})}
    $$
    $$
        =\Vert h|_{\Omega_c}\Vert_{L_q(\Omega_c,\nu|_{\Omega_c})} \leq \Vert h
        \Vert_{L_q(\Omega,\nu)}
    $$
    So $M_{\chi_{\Omega_c}/g}$ is contractive. Now note that for
    all $f\in L_p(\Omega,\mu)$ we have
    $$
        M_{\chi_{\Omega_c}/g}(M_g(f)) =M_{\chi_{\Omega_c}/g}(g\cdot f)
        =(\chi_{\Omega_c}/g)\cdot g\cdot f =f\cdot\chi_{\Omega_c}
    $$
    Since $\mu(\Omega\setminus\Omega_c)=0$, then
    $\chi_{\Omega_c}=\chi_{\Omega}$, so
    $M_{\chi_{\Omega_c}/g}(M_g(f))=f\cdot\chi_{\Omega_c}=f\cdot\chi_{\Omega}=f$.
    This means that $M_g$ have left inverse multiplication operator. Take any
    $f\in L_p(\Omega,\mu)$, then for $h=M_g(f)$ we have
    $M_{\chi_{\Omega_c}/g}(h)=f$ and
    $\Vert h\Vert_{L_q(\Omega,\nu)}\leq\Vert f\Vert_{L_p(\Omega,\mu)}$ i.e.
    $M_{\chi_{\Omega_c}}/g$ is strictly $1$-topologigically surjective. Since
    $M_{\chi_{\Omega_c}/g}$ is also contractive, it is strictly coisometric.

    $(ii)\implies (i)$ Take any $f\in L_p(\Omega,\mu)$, then there exist
    $h\in L_q(\Omega,\nu)$ such that $M_{\chi_{\Omega_c}/g}(h)=f$ and
    $\Vert h\Vert_{L_q(\Omega,\nu)}\leq \Vert f\Vert_{L_p(\Omega,\mu)}$. Hence
    $$
        \Vert M_g(f)\Vert_{L_q(\Omega,\nu)} =\Vert
        M_g(M_{\chi_{\Omega_c}/g}(h))\Vert_{L_q(\Omega,\nu)} =\Vert
        \chi_{\Omega_c}h\Vert_{L_q(\Omega,\nu)} \leq\Vert
        h\Vert_{L_q(\Omega,\nu|)} \leq\Vert f\Vert_{L_p(\Omega,\mu)}
    $$
    Since $M_{\chi_{\Omega_c}/g}$ is contractive and left inverse to $M_g$ then
    $$
        \Vert f\Vert_{L_p(\Omega,\mu)} =\Vert
        M_{\chi_{\Omega_c}/g}(M_g(f))\Vert_{L_p(\Omega,\mu)} \leq\Vert
        M_g(f)\Vert_{L_q(\Omega,\nu)}
    $$
    so $\Vert M_g(f)\Vert_{L_q(\Omega,\nu)}=\Vert f\Vert_{L_p(\Omega,\mu)}$.
    Since $f$ is arbitrary $M_g$ is isometric.
\end{proof}

%-------------------------------------------------------------------------------
% Topologically surjective and coisometric operators
%-------------------------------------------------------------------------------

\subsection{Topologically surjective and coisometric operators}

Description of topologically surjective operators is slightly easier to obtain.
We will show that all such operators are isomorphisms or invertible from the
right. Most of the proofs goes along the lines of previous sections.

\begin{theorem}\label{TopSurMultOpCharacOnMeasSp} Let $(\Omega,\Sigma,\mu)$ be
    a $\sigma$-finite measure space, $p,q\in[1,+\infty]$ and
    $g\in L_0(\Omega,\mu)$, then the following are equivalent
    \begin{enumerate}[label = (\roman*)]
        \item $M_g\in\mathcal{B}(L_p(\Omega,\mu),L_q(\Omega,\mu))$ is
              topologically surjective

        \item $M_g$ is an isomorphism

        \item $|g|\geq c$ for some $c>0$, if $p\neq q$ the space
              $(\Omega,\Sigma,\mu)$ consist of finitely many atoms.
    \end{enumerate}
\end{theorem}
\begin{proof} $(i)\implies (iii)$ Since $M_g$ be topologically surjective,
    then it is surjective and by proposition~\ref{MultpOpSurjInjDesc} it is
    also injective. Thus $M_g$ is bijective. Since $L_p$ spaces are complete,
    from open mapping theorem we see that $M_g$ is an isomorphism.

    $(ii)\implies (i)$ If $M_g$ is an isomorphism, obviously, it is
    topologically surjective.

    $(i)\implies (iii)$ Follows from theorem~\ref{TopInjMultOpCharacOnMeasSp}
\end{proof}

\begin{proposition}\label{TopSurMultOpCharacBtwnTwoContMeasSp}
    Let $(\Omega,\Sigma,\nu)$ be a $\sigma$-finite measure space,
    $p,q\in[1,+\infty]$ and $g,\rho\in L_0(\Omega,\rho\cdot\nu)$ and $\rho$ is
    non negative, then the following are equivalent
    \begin{enumerate}[label = (\roman*)]
        \item $M_g\in\mathcal{B}(L_p(\Omega,\rho\cdot\nu),L_q(\Omega,\nu))$ is
              topologically surjective

        \item $M_g$ is an isomorphism

        \item $\rho$ is positive, $|g\cdot \rho^{-1/p}|\geq c$ for some
              $c>0$, if $p\neq q$ the space $(\Omega,\Sigma,\mu)$ consist of
              finitely many atoms.
    \end{enumerate}
\end{proposition}
\begin{proof} $(i)\implies (iii)$ Consider set $E=\rho^{-1}( \{0 \})$.
    Assume $\nu(E)>0$ then $\chi_E\neq 0$ in $L_p(\Omega,\nu)$. On the other
    hand $(\rho\cdot\nu)(E)=\int_E\rho(\omega)d\nu(\omega)=0$, so $\chi_E=0$
    in $L_q(\Omega,\rho\cdot\mu)$. Then for all $f\in L_p(\Omega,\rho\cdot\nu)$
    holds $M_g(f)\chi_E=M_g(f\cdot\chi_E)=M_g(0)=0$ in $L_q(\Omega,\nu)$.
    The last equality means that
    $\operatorname{Im}(M_g)\subset \{h\in L_q(\Omega,\mu): h|_E=0 \}$. Since
    $\nu(E)\neq 0$ we see that $M_g$ is not surjective and as the consequence
    it is not topologically surjective. Contradiction, so $\nu(E)=0$ and
    $\rho$ is positive. Hence by proposition~\ref{ChngOfDenst} we have an
    isometric isomorphism
    $\bar{I}_p:L_p(\Omega,\nu)\to
        L_p(\Omega,\rho\cdot\nu),f\mapsto \rho^{-1/p}\cdot f$. Obviously
    $M_{g\cdot\rho^{-1/p}}
        =M_g \bar{I}_p\in\mathcal{B}(L_p(\Omega,\nu),L_q(\Omega,\nu))$. Since
    $\bar{I}_p$ is an isometric isomorphism and $M_g$ is topologically
    surjective, then $M_{g\cdot \rho^{-1/p}}$ is topologically surjective.
    It is remains to apply theorem~\ref{TopSurMultOpCharacOnMeasSp}.

    $(iii)\implies (i)$ By theorem~\ref{TopSurMultOpCharacOnMeasSp} operator
    $M_{g\cdot\rho^{-1/p}}$ is topologically surjective. Since $\rho$ is
    positive by proposition~\ref{ChngOfDenst} we have an isometric
    isomorohism $\bar{I}_p$. Then from equality
    $M_g= M_{g\cdot\rho^{-1/p}}\bar{I}_p^{-1}$ it follows that $M_g$ is also
    topologically surjective.

    $(i)\implies (ii)$ As we proved above this operator
    $M_{g\cdot\rho^{1/q}}$ is topologically injective and $\bar{I}_q$ is an
    isometric isomorphism. By theorem~\ref{TopSurMultOpCharacOnMeasSp}
    $M_{g\cdot\rho^{1/q}}$ is an isomorphism. Since
    $M_g=\bar{I}_q M_{g\cdot\rho^{1/q}}$ we see that $M_g$ is also an
    isomorphism, as composition of such.

    $(ii)\implies (i)$. If $M_g$ is an isomorphism, obviously, it is
    topologically surjective.
\end{proof}

\begin{theorem}\label{TopSurMultOpCharacBtwnTwoMeasSp} Let
    $(\Omega,\Sigma,\mu)$, $(\Omega,\Sigma,\nu)$ be two $\sigma$-finite measure
    spaces, $p,q\in[1,+\infty]$ and $g\in L_0(\Omega,\mu)$, then the following
    are equivalent
    \begin{enumerate}[label = (\roman*)]
        \item $M_g\in\mathcal{B}(L_p(\Omega,\mu), L_q(\Omega,\nu))$ is
              topologically surjective

        \item $M_g^{\Omega_c}$ is topologically surjective

        \item $\rho_{\mu,\nu}$ is positive,
              $|g\cdot\rho_{\mu,\nu}^{-1/p}|_{\Omega_c}|\geq c$ for some
              $c>0$, if $p\neq q$ the space $(\Omega,\Sigma,\mu)$ consist
              of finitely many atoms.
    \end{enumerate}
\end{theorem}
\begin{proof}
    By proposition~\ref{MultOpDecompDecomp} operator $M_g$ is topologically
    surjective if and only if operators
    $M_g^{\Omega_c}:L_p(\Omega_c,\rho_{\mu,\nu}\cdot\nu|_{\Omega_c})\to
        L_q(\Omega_c,\nu|_{\Omega_c})$ and $M_g^{\Omega_s}:
        L_p(\Omega_s,\mu_s|_{\Omega_s})\to L_q(\Omega_s,\nu|_{\Omega_s})$ are
    topologically surjective. By proposition~\ref{MultOpCharacBtwnTwoSingMeasSp}
    operator $M_g^{\Omega_s}$ is zero. Since $\nu(\Omega_s)=0$, then
    $L_p(\Omega_s,\nu|_{\Omega_s})= \{0 \}$. From these two facts we conclude
    that $M_g^{\Omega_s}$ is topologically surjective. Thus topological
    surjectivity of $M_g$ is equivalent to topological injectivity of
    $M_g^{\Omega_c}$. It is remains to apply
    proposition~\ref{TopSurMultOpCharacBtwnTwoContMeasSp}.
\end{proof}

\begin{theorem}\label{TopSurMultOpDescBtwnTwoMeasSp}
    Let $(\Omega,\Sigma,\mu)$, $(\Omega,\Sigma,\nu)$ be two $\sigma$-finite
    measure spaces, $p,q\in[1,+\infty]$ and $g\in L_0(\Omega,\mu)$, then the
    following are equivalent
    \begin{enumerate}[label = (\roman*)]
        \item $M_g\in\mathcal{B}(L_p(\Omega,\mu),L_q(\Omega,\nu))$ is
              topologically surjective

        \item $M_{\chi_{\Omega_c}/g}
                  \in\mathcal{B}(L_q(\Omega,\nu), L_p(\Omega,\mu))$ its right inverse
              topologically injective operator.
    \end{enumerate}
\end{theorem}
\begin{proof}
    $(i)\implies (ii)$ By proposition~\ref{MultOpDecompDecomp} operator
    $M_g^{\Omega_c}$ is topologically surjective. By
    proposition~\ref{TopSurMultOpCharacBtwnTwoContMeasSp} it is invertable
    and ${(M_g^{\Omega_c})}^{-1}=M_{1/g}^{\Omega_c}$. Then for
    all $h\in L_q(\Omega,\nu)$ we have
    $$
        \Vert M_{\chi_{\Omega_c}/g}(h)\Vert_{L_p(\Omega,\mu)}= \Vert
        M_{1/g}(h|_{\Omega_c})\Vert_{L_p(\Omega_c,\mu|_{\Omega_c})}= \Vert
        M_{1/g}^{\Omega_c}(h|_{\Omega_c})\Vert_{L_p(\Omega_c,\mu|_{\Omega_c})}
    $$
    $$
        \leq\Vert M_{1/g}^{\Omega_c}\Vert\Vert
        h|_{\Omega_c}\Vert_{L_q(\Omega_c,\nu|_{\Omega_c})} \leq\Vert
        M_{1/g}^{\Omega_c}\Vert\Vert h\Vert_{L_q(\Omega,\nu)}
    $$
    So $M_{\chi_{\Omega_c}/g}$ is bounded. Now note that for
    all $h\in L_q(\Omega,\nu)$ we have
    $$
        M_g(M_{\chi_{\Omega_c}/g}(h)) =M_g(\chi_{\Omega_c}/g\cdot h)
        =g\cdot(\chi_{\Omega_c}/g)\cdot  h =h\cdot\chi_{\Omega_c}
    $$
    Since $\nu(\Omega\setminus\Omega_c)=0$, then
    $\chi_{\Omega_c}=\chi_{\Omega}$, so
    $M_g(M_{\chi_{\Omega_c}/g}(h))=h\cdot\chi_{\Omega_c}=h\cdot\chi_{\Omega}=h$.
    This means that $M_g$ have right inverse multiplication operator.
    Take any $h\in L_q(\Omega,\nu)$, then
    $$
        \Vert M_{\chi_{\Omega_c}/g}(h)\Vert_{L_p(\Omega,\mu)} \geq\Vert
        M_g\Vert\Vert M_g(M_{\chi_{\Omega_c}/g}(h))\Vert_{L_q(\Omega,\nu)}
        \geq\Vert M_g\Vert\Vert h\Vert_{L_q(\Omega,\nu)}
    $$
    Since $h$ is arbitrary $M_{\chi_{\Omega_c}/g}$ is topologically injective.

    $(ii)\implies (i)$ Take arbitrary $h\in L_q(\Omega,\nu)$ and consider
    $f=M_{\chi_{\Omega_c}/g}(h)$. Then $M_g(f)=M_g(M_{\chi_{\Omega_c}/g}(h))=h$
    and $\Vert f\Vert_{L_p(\Omega,\mu)}
        \leq\Vert M_{\chi_{\Omega_c}/g}\Vert\Vert h\Vert_{L_q(\Omega,\nu)}$. Since
    $h$ is arbitrary $M_g$ is topologically surjective.
\end{proof}


\begin{theorem}\label{CoisomMultOpCharacOnMeasSp}
    Let $(\Omega,\Sigma,\mu)$ be a $\sigma$-finite measure space,
    $p,q\in[1,+\infty]$ and $g\in L_0(\Omega,\mu)$, then the
    following are equivalent
    \begin{enumerate}[label = (\roman*)]
        \item $M_g\in\mathcal{B}(L_p(\Omega,\mu),L_q(\Omega,\mu))$
              is coisometric

        \item $M_g$ is an isometric isomorphism

        \item $|g|={\mu(\Omega)}^{1/q-1/p}$, if $p\neq q$ the space
              $(\Omega,\Sigma,\mu)$ consist of single atom.
    \end{enumerate}
\end{theorem}
\begin{proof} Since $M_g$ is coisometric it is topologically injective so from
    theorem~\ref{TopSurMultOpCharacOnMeasSp} we get that $M_g$ is in fact
    isomorphism. As the consequence it is injective, but injective coisometric
    operator is an isometric isomorphisms. It is remains to note that every
    isometric isomorphism is a strict coisometry. Thus we conclude that $M_g$
    is coisometric if and only if it is strictly coisometric if and only if it
    is isometric isomorphism. Now we apply
    theorem~\ref{IsomMultOpCharacOnMeasSp}.
\end{proof}

\begin{proposition}\label{CoisomMultOpCharacBtwnTwoContMeasSp}
    Let $(\Omega,\Sigma,\nu)$ be a $\sigma$-finite measure space,
    $p,q\in[1,+\infty]$ and $g,\rho\in L_0(\Omega,\rho\cdot\nu)$ and $\rho$ is
    non negative, then the following are eqivalent
    \begin{enumerate}[label = (\roman*)]
        \item $M_g\in\mathcal{B}(L_p(\Omega,\rho\cdot\nu),L_q(\Omega,\nu))$
              is coisometric

        \item $M_g$ is an isometric isomorphism

        \item $\rho$ is positive,
              $|g\cdot \rho^{-1/p}|={\mu(\Omega)}^{1/p-1/q}$, if $p\neq q$ the space
              $(\Omega,\Sigma,\mu)$ consist single atom.
    \end{enumerate}
\end{proposition}
\begin{proof} $(i)\implies (ii)$ Assume $M_g$ is coisometric, then it is
    topologically surjective. By
    theorem~\ref{TopSurMultOpCharacBtwnTwoContMeasSp} $M_g$ is an isomorphism,
    hence bijective. It is remains to note that bijective coisometry is an
    isometric isomorphism.

    $(ii)\implies (i)$ If $M_g$ is an isometric isomorphism, of course, it is
    coisometry and even more a strict coisometry.

    $(i)\implies (iii)$ Follows from
    proposition~\ref{IsomMultOpCharacBtwnTwoContMeasSp}.
\end{proof}

\begin{theorem}\label{CoisomMultOpCharacBtwnTwoMeasSp}
    Let $(\Omega,\Sigma,\mu)$, $(\Omega,\Sigma,\nu)$ be two $\sigma$-finite
    measure spaces, $p,q\in[1,+\infty]$ and $g\in L_0(\Omega,\mu)$, then the
    following are equivalent
    \begin{enumerate}[label = (\roman*)]
        \item $M_g\in\mathcal{B}(L_p(\Omega,\mu), L_q(\Omega,\nu))$ is
              coisometric

        \item $M_g^{\Omega_c}$ is an isometric isomorphism

        \item $\rho_{\mu,\nu}$ is positive,
              $|g\cdot\rho_{\mu,\nu}^{-1/p}|_{\Omega_c}|={\mu(\Omega_c)}^{1/p-1/q}$,
              if $p\neq q$ the space $(\Omega,\Sigma,\mu)$ consist of single atom.
    \end{enumerate}
\end{theorem}
\begin{proof} $(i)\implies (ii)$ Since $M_g$ is coisimetric, then from
    proposition~\ref{MultOpDecompDecomp} we know that $M_g^{\Omega_c}$ is also
    coisometric. From proposition~\ref{CoisomMultOpCharacBtwnTwoContMeasSp} we
    get that $M_g^{\Omega_c}$ is an isometric isomorphism.

    $(ii)\implies (i)$ Take arbitrary $h\in L_q(\Omega,\nu)$, then there
    exist $f\in L_p(\Omega_c,\mu|_{\Omega_c})$ such that
    $M_g^{\Omega_c}(f)=h|_{\Omega_c}$. By
    proposition~\ref{MultOpCharacBtwnTwoSingMeasSp} operator
    $M_g^{\Omega_s}=0$, so
    $$
        M_g(\widetilde{f})
        =\widetilde{M_g^{\Omega_c}(\widetilde{f}|_{\Omega_c})}
        +\widetilde{M_g^{\Omega_s}(\widetilde{f}|_{\Omega_s})}
        =\widetilde{h|_{\Omega_c}}
    $$
    Since $\nu(\Omega_s)=0$, then
    $\Vert h-\widetilde{h|_{\Omega_c}}\Vert_{L_q(\Omega,\nu)}
        =\Vert h\chi_{\Omega_s}\Vert_{L_q(\Omega,\nu)}=0$ and we conclude
    $h=\widetilde{h|_{\Omega_c}}$. So we found
    $\widetilde{f}\in L_p(\Omega,\mu)$ such that
    $M_g(\widetilde{f})=h$ and
    $\Vert \widetilde{f}\Vert_{L_p(\Omega,\mu)}
        =\Vert f\Vert_{L_p(\Omega_c,\mu|_{\Omega_c})}
        =\Vert h|_{\Omega_c}\Vert_{L_q(\Omega_c,\nu|_{\Omega_c})}
        \leq\Vert h\Vert_{L_q(\Omega,\nu)}$. Since $h$ is arbitrary then
    $M_g$ is $1$-topologically surjective. For all
    $f\in L_p(\Omega,\mu)$ we have
    $$
        \Vert M_g(f)\Vert_{L_q(\Omega,\nu)}
        =\Vert\widetilde{M_g^{\Omega_c}(f|_{\Omega_c})}
        +\widetilde{M_g^{\Omega_s}(f|_{\Omega_s})}\Vert_{L_q(\Omega,\nu)}
        =\Vert\widetilde{M_g^{\Omega_c}(f|_{\Omega_c})}\Vert_{L_q(\Omega,\nu)}
    $$
    $$
        =\Vert
        M_g^{\Omega_c}(f|_{\Omega_c})\Vert_{L_q(\Omega_c,\nu|_{\Omega_c})}
        =\Vert f|_{\Omega_c}\Vert_{L_p(\Omega_c,\mu|_{\Omega_c})} \leq\Vert f
        \Vert_{L_p(\Omega,\mu)}
    $$
    Since $f$ is arbitrary, then $M_g$ is contractive, but it is also
    $1$-topologically injective. Thus $M_g$ is coisometric.

    $(i)\implies (iii)$ Follows from
    proposition~\ref{CoisomMultOpCharacBtwnTwoContMeasSp}
\end{proof}

\begin{theorem}\label{CoisomMultOpDescBtwnTwoMeasSp}
    Let $(\Omega,\Sigma,\mu)$, $(\Omega,\Sigma,\nu)$ be two $\sigma$-finite
    measure spaces, $p,q\in[1,+\infty]$ and $g\in L_0(\Omega,\mu)$, then the
    following are equivalent
    \begin{enumerate}[label = (\roman*)]
        \item $M_g\in\mathcal{B}(L_p(\Omega,\mu),L_q(\Omega,\nu))$ is
              coisometric

        \item $M_{\chi_{\Omega_c}/g}
                  \in\mathcal{B}(L_q(\Omega,\nu), L_p(\Omega,\mu))$ its right inverse
              isometric operator.
    \end{enumerate}
\end{theorem}
\begin{proof}
    $(i)\implies (ii)$ By proposition~\ref{MultOpDecompDecomp} operator
    $M_g^{\Omega_c}$ is coisometric and by
    proposition~\ref{CoisomMultOpCharacBtwnTwoContMeasSp} it is isometric,
    invertable and ${(M_g^{\Omega_c})}^{-1}=M_{1/g}^{\Omega_c}$. Then for
    all $h\in L_q(\Omega,\nu)$ we have
    $$
        \Vert M_{\chi_{\Omega_c}/g}(h)\Vert_{L_p(\Omega,\mu)}= \Vert
        M_{1/g}(h)\chi_{\Omega_c}\Vert_{L_p(\Omega,\mu)}= \Vert
        M_{1/g}(h|_{\Omega_c})\Vert_{L_p(\Omega_c,\mu|_{\Omega_c})}= \Vert
        M_{1/g}^{\Omega_c}(h|_{\Omega_c})\Vert_{L_p(\Omega_c,\mu|_{\Omega_c})}
    $$
    $$
        =\Vert h|_{\Omega_c}\Vert_{L_q(\Omega_c,\nu|_{\Omega_c})} \leq\Vert
        h\Vert_{L_q(\Omega,\nu)}
    $$
    So $M_{\chi_{\Omega_c}/g}$ is contractive. Now note that for all
    $h\in L_q(\Omega,\nu)$ we have
    $$
        M_g(M_{\chi_{\Omega_c}/g}(h)) =M_g(\chi_{\Omega_c}/g\cdot h)
        =g\cdot(\chi_{\Omega_c}/g)\cdot  h =h\cdot\chi_{\Omega_c}
    $$
    Since $\nu(\Omega\setminus\Omega_c)=0$, then
    $\chi_{\Omega_c}=\chi_{\Omega}$, so
    $M_g(M_{\chi_{\Omega_c}/g}(h))=h\cdot\chi_{\Omega_c}=h\cdot\chi_{\Omega}=h$.
    This means that $M_g$ have right inverse multiplication operator. Take any
    $h\in L_q(\Omega,\nu)$, then
    $$
        \Vert M_{\chi_{\Omega_c}/g}(h)\Vert_{L_p(\Omega,\mu)} \geq\Vert
        M_g\Vert\Vert M_g(M_{\chi_{\Omega_c}/g}(h))\Vert_{L_q(\Omega,\nu)}
        \geq\Vert h\Vert_{L_q(\Omega,\nu)}
    $$
    Since $h$ is arbitrary $M_{\chi_{\Omega_c}/g}$ is $1$-topologically
    injective, but it is contractive. Thus $M_g$ is isometric.

    $(ii)\implies (i)$ Take arbitrary $h\in L_q(\Omega,\nu)$ and
    consider $f=M_{\chi_{\Omega_c}/g}(h)$. Then
    $M_g(f)=M_g(M_{\chi_{\Omega_c}/g}(h))=h$ and
    $\Vert f\Vert_{L_p(\Omega,\mu)}\leq\Vert h\Vert_{L_q(\Omega,\nu)}$. Since
    $h$ is arbitrary $M_g$ is strictly $1$-topologically surjective. Let
    $f\in L_p(\Omega,\mu)$. By assumption $M_{\chi_{\Omega_c}/g}$ so
    $$
        \Vert M_g(f)\Vert_{L_q(\Omega,\nu)} =\Vert
        M_{\chi_{\Omega_c}/g}(M_g(f))\Vert_{L_p(\Omega,\mu)} =\Vert
        f\chi_{\Omega_c}\Vert_{L_p(\Omega,\mu)} \leq\Vert
        f\Vert_{L_p(\Omega,\mu)}
    $$
    Since $f$ is arbitrary $M_g$ is contractive, but it is also strictly
    $1$-topologically surjective, hence strictly coisometric.
\end{proof}
\newline

%-------------------------------------------------------------------------------
% Projective, injective and flat B(Omega)-modules in the category of L_p spaces
%-------------------------------------------------------------------------------

\section{
  Projective, injective and flat
  \texorpdfstring{$B(\Omega)$}{B (Omega)}-modules in
  the category of \texorpdfstring{$L_p$}{Lp} spaces
 }

%-------------------------------------------------------------------------------
% Morphisms of B(Omega)-modules Lp
%-------------------------------------------------------------------------------

\subsection{
    Morphisms of \texorpdfstring{$B(\Omega)$}{B (Omega)}-modules
    \texorpdfstring{$L_p$}{Lp}
}

By $B(\Omega)$ we will denote Banach algebra of bounded measurable functions
on measurable space $(\Omega,\Sigma)$ with $\sup$-norm. Obviously for
any $b\in B(\Omega)$ and any $f\in L_p(\Omega,\mu)$ we have
$$
    \Vert b\cdot f\Vert_{L_p(\Omega,\mu)} \leq\Vert b\Vert_{B(\Omega)}\Vert
    f\Vert_{L_p(\Omega,\mu)}
$$
Hence every $L_p$ space is a left/right/two sided Banach $B(\Omega)$-module.
Since for the same $f$ and $b$ we have $b\cdot f=f\cdot b$, and
algebra $B(\Omega)$ is commutative we can restrict our considerations
to the left modules.

By $M(\Omega)$ we will denote Banach space of finite complex valued
$\sigma$-additive measures on $\Omega$. By $\mathscr{L}$ we denote full
subcategory of left Banach $B(\Omega)$ modules consisting of $L_p(\Omega,\mu)$
spaces for some $\mu\in M(\Omega)$. By $\mathscr{L}_1$ we will denote the
category with the same objects but with contractive morphisms only.
By $\mathscr{L}^{\operatorname{op}}$ we will denote the category of the right
$B(\Omega)$ modules of the form $L_p(\Omega,\mu)$.
In~\cite{HelTensProdAndMultModLp} Helemskii gave a complete characterisation of
morphisms of $\mathscr{L}$, but only for for locally compact $\Omega$, with
Borel $\sigma$-algebra. Careful inspection of his proof shows that this
characterization valid for all $\sigma$-finite measure spaces.

Let $p,q\in[1,+\infty]$ and $\mu,\nu\in M(\Omega)$. Denote
$\Omega_+= \{\omega\in\Omega_c:\rho_{\nu,\mu}(\omega)>0 \}$. Recall that
by $\Omega_a$ we denote atomic part of measure space $(\Omega,\Sigma,\mu)$
provided by proposition~\ref{DescOfLpSpOnMeasSp}. Of course, we may assume
that $\Omega_a\subset\Omega_c$. Introduce the notation
$$
    L_{p,q,\mu,\nu}(\Omega)=
    \begin{cases}
        \{g\in L_0(\Omega,\mu):g\in
        L_{pq/(p-q)}(\Omega,\rho_{\nu,\mu}^{p/(p-q)}\cdot\mu),\quad
        g|_{\Omega\setminus\Omega_+}=0 \}                              &
        \text{if}\quad p>q                                               \\
        \{g\in L_0(\Omega,\mu):g\rho_{\nu,\mu}^{1/p}\in
        L_{\infty}(\Omega,\mu),\quad g|_{\Omega\setminus\Omega_+}=0 \} &
        \text{if}\quad p=q
        \\
        \{g\in L_0(\Omega,\mu):g\rho_{\nu,\mu}^{1/p}\mu^{pq/(p-q)}\in
        L_{\infty}(\Omega,\mu),\quad g|_{\Omega\setminus\Omega_a}=0 \} &
        \text{if}\quad p<q
        \\
    \end{cases}
$$
$$
    \Vert g\Vert_{L_{p,q,\mu,\nu}(\Omega)}=
    \begin{cases}
        \Vert g\Vert_{L_{pq/(p-q)}(\Omega,\rho^{p/(p-q)}\cdot\mu)} &
        \text{if}\quad p>q
        \\
        \Vert g\rho_{\nu,\mu}^{1/p}\Vert_{L_{\infty}(\Omega,\mu)}  &
        \text{if}\quad p=q
        \\
        \Vert g\rho_{\nu,\mu}^{1/p}\mu^{pq/(p-q)}\Vert_{L_{\infty}(\Omega,\mu)}
                                                                   &
        \text{if}\quad
        p<q                                                          \\
    \end{cases}
$$
\begin{theorem}[\cite{HelTensProdAndMultModLp}, 4.1]\label{LpModMorphCharac}
    Let $p,q\in[1,+\infty]$ and $\mu,\nu\in M(\Omega)$,then there exist
    isometric isomorphism
    $$
        \mathcal{I}_{p,q,\mu,\nu}:
        L_{p,q,\mu,\nu}(\Omega)
        \to\operatorname{Hom}_{\mathscr{L}}(L_p(\Omega,\mu),L_q(\Omega,\nu)):
        g\mapsto M_g
    $$
\end{theorem}

Simply speaking all morphisms in $\mathscr{L}$ are multiplication operators.
Now we need definitions for different types of ``good'' morphisms from the
point of view of Banach homology theory to describe variants of projectivity,
injectivity and flatness.

\begin{definition}\label{AdmEpiMorph} Let $\mathscr{C}$ be a category of left
    Banach modules over algebra $A$. Let $X,Y\in\operatorname{Ob}(\mathscr{C})$,
    then we say that a morphism
    $\varphi\in\operatorname{Hom}_{\mathscr{C}}(X,Y)$ is a
    relatively/metrically/extremelly admissible epimorphism if it is
    topologically surjective/strictly coisometric/coisometric.
\end{definition}

\begin{definition}\label{AdmMonoMorph} Let $\mathscr{C}$ be a category of
    left Banach modules over algebra $A$. Let
    $X,Y\in\operatorname{Ob}(\mathscr{C})$, then we say that
    morphism $\varphi\in\operatorname{Hom}_{\mathscr{C}}(X,Y)$ is a
    relatively/metrically/extremelly admissible monomorphism if it is
    topologically injective/isometric/isometric.
\end{definition}


All these notions are due to Helemskii (\cite{HelMetFrPoj}). Now results of
previous section may be reformulated as follows:
\begin{enumerate}[label = (\roman*)]
    \item  All relatively/metrically/extremely admissible epimorphisms in
          $\mathscr{L}$ are retractions and vice versa.

    \item All relatively/metrically/extremely admissible monomorphisms in
          $\mathscr{L}$ are coretractions and vice versa.
\end{enumerate}


%-------------------------------------------------------------------------------
% Injective of modules Lp
%-------------------------------------------------------------------------------

\subsection{Injective modules \texorpdfstring{$L_p$}{Lp}}

\begin{definition} Let $\mathscr{C}$ be a category of left Banach modules over
    algebra $A$. We say that $I\in\operatorname{Ob}(\mathscr{C})$ is
    relatively/metrically/extremely injecive if the functor
    $\operatorname{Hom}_{\mathscr{C}}(-,I)$ from
    $\mathscr{C}$/$\mathscr{C}_1$/$\mathscr{C}_1$ to $\mathscr{B}an$ maps
    relatively/metricaly/extremely admissible monomorphisms to
    surjective/strictly coisometric/coisometric operators.
\end{definition}

\begin{theorem} Every $B(\Omega)$ module $L_p$ is
    relatively/metrically/extremely injective in $\mathscr{L}$.
\end{theorem}
\begin{proof}
    Relative injectivity. Let $I,X,Y\in\operatorname{Ob}(\mathscr{L})$.
    Take arbitrary $\varphi\in\operatorname{Hom}_{\mathscr{L}}(X,I)$ and
    relatively admissible monomorphism
    $i\in\operatorname{Hom}_{\mathscr{L}}(X, Y)$. By
    theorem~\ref{TopInjMultOpDescBtwnTwoMeasSp} we have  topologically
    surjective $\pi\in\operatorname{Hom}_{\mathscr{L}}(Y, X)$ such
    that $\pi i=1_{X}$. Then for $\psi=\varphi\pi$ we have
    $\operatorname{Hom}_{\mathscr{L}}(i, I)(\psi)=\varphi$. Hence
    $\operatorname{Hom}_{\mathscr{L}}(i, I)$ is surjective. This means that
    $I$ is relatively injective.

    Metric/extreme injectivity. Let $I,X,Y\in\operatorname{Ob}(\mathscr{L}_1)$
    Take arbitrary $\varphi\in\operatorname{Hom}_{\mathscr{L}_1}(X, I)$ and
    metrically/extremely admissible monomorphism
    $i\in\operatorname{Hom}_{\mathscr{L}_1}(X, Y)$. By
    theorem~\ref{IsomMultOpDescBtwnTwoMeasSp} we have coisometric
    $\pi\in\operatorname{Hom}_{\mathscr{L}_1}(Y, X)$ such that
    $\pi i=1_{X}$. Then for $\psi=\varphi\pi$ we have
    $\operatorname{Hom}_{\mathscr{L}_1}(i, I)(\psi)=\varphi$ and what is
    more $\Vert\psi\Vert=\Vert\varphi\Vert$ because
    $\Vert\psi\Vert\leq\Vert\varphi\Vert\Vert\pi\Vert=\Vert\varphi\Vert$ and
    $\Vert\varphi\Vert\leq\Vert\psi\Vert\Vert i\Vert=\Vert\psi\Vert$. Hence
    $\operatorname{Hom}_{\mathscr{L}_1}(i,I)$ is strictly coisometric and a
    fortiori coisometric. This means that $I$ is metrically/extremely injective.
\end{proof}

%-------------------------------------------------------------------------------
% Projective of modules Lp
%-------------------------------------------------------------------------------

\subsection{Projective modules \texorpdfstring{$L_p$}{Lp}}

\begin{definition} Let $\mathscr{C}$ be a category of left Banach modules over
    algebra $A$. We say that $P\in\operatorname{Ob}(\mathscr{C})$ is
    relatively/metrically/extremely projective if the functor
    $\operatorname{Hom}_{\mathscr{C}}(P,-)$ from
    $\mathscr{C}$/$\mathscr{C}_1$/$\mathscr{C}_1$ to $\mathscr{B}an$ maps
    relatively/metricaly/extremely admissible epimorphisms to
    surjective/strictly coisometric/coisometric operators.
\end{definition}

\begin{theorem} Every $B(\Omega)$ module $L_p$ is
    relatively/metrically/extremely projective in $\mathscr{L}$.
\end{theorem}
\begin{proof}
    Relative projectivity. Let $P,X,Y\in\operatorname{Ob}(\mathscr{L})$.
    Take arbitrary $\varphi\in\operatorname{Hom}_{\mathscr{L}}(P, X)$ and
    relatively admissible  epimorphism
    $\pi\in\operatorname{Hom}_{\mathscr{L}}(Y, X)$. By
    theorem~\ref{TopSurMultOpDescBtwnTwoMeasSp} we have topologically
    injective $i\in\operatorname{Hom}_{\mathscr{L}}(X, Y)$ such that
    $\pi i=1_{X}$. Then for $\psi=i\varphi$ we have
    $\operatorname{Hom}_{\mathscr{L}}(P,\pi)(\psi)=\varphi$. Hence
    $\operatorname{Hom}_{\mathscr{L}}(P,\pi)$ is surjective. This means that
    $P$ is relatively projective.

    Metric/extreme projectivity. Let
    $P,X,Y\in\operatorname{Ob}(\mathscr{L}_1)$. Take arbitrary
    $\varphi\in\operatorname{Hom}_{\mathscr{L}_1}(P, X)$ and
    metrically/extremely admissible epimorphism
    $\pi\in\operatorname{Hom}_{\mathscr{L}_1}(Y, X)$. By
    theorem~\ref{CoisomMultOpDescBtwnTwoMeasSp} we have isometric
    $i\in\operatorname{Hom}_{\mathscr{L}_1}(X, Y)$ such that
    $\pi i=1_{X}$. Then for $\psi=i\varphi$ we have
    $\operatorname{Hom}_{\mathscr{L}_1}(P,\pi)(\psi)=\varphi$ and what is more
    $\Vert\psi\Vert=\Vert\varphi\Vert$ because $i$ is isometric.
    Hence $\operatorname{Hom}_{\mathscr{L}_1}(P,\pi)$ is strictly coisometric
    and a fortiori coisometric. This means that $P$ is
    metrically/extremely projective.
\end{proof}

%-------------------------------------------------------------------------------
% Flat of modules Lp
%-------------------------------------------------------------------------------

\subsection{Flat modules \texorpdfstring{$L_p$}{Lp}}

\begin{definition} Let $\mathscr{C}$ be a category of left Banach modules over
    algebra $A$. We say that $F\in\operatorname{Ob}(\mathscr{C})$ is
    relatively/metrically/extremely flat if the functor
    $-\mathop{\operatorname{\otimes}}^A 1_F$ from
    $\mathscr{C}$/$\mathscr{C}_1$/$\mathscr{C}_1$ to
    $\mathscr{B}an$ maps relatively/metricaly/extremely admissible monomorphisms
    in $\mathscr{L}^{\operatorname{op}}$ to topologically injective/isometric/
    isometric operators.
\end{definition}

\begin{theorem} Every $B(\Omega)$ module $L_p$ is
    relatively/metrically/extremely flat in $\mathscr{L}$.
\end{theorem}
\begin{proof}
    Relative flatness. Let $F,X,Y\in\operatorname{Ob}(\mathscr{L})$.
    Take arbitrary  relatively admissible monomorphism
    $i\in\operatorname{Hom}_{\mathscr{L}^{\operatorname{op}}}(X, Y)$.
    By theorem~\ref{TopInjMultOpDescBtwnTwoMeasSp} we have topologically
    surjective
    $\pi\in\operatorname{Hom}_{\mathscr{L}^{\operatorname{op}}}(Y, X)$ such
    that $\pi i=1_{X}$. Then for arbitrary
    $u\in F\mathop{\operatorname{\otimes}}^{B(\Omega)} X$
    we have
    $$
        \Vert(1_F \mathop{\operatorname{\otimes}}^{B(\Omega)} \pi)\Vert\Vert(1_F
        \mathop{\operatorname{\otimes}}^{B(\Omega)}
        i)(u)\Vert_{F\mathop{\operatorname{\otimes}}^{B(\Omega)} Y}
        \geq
        \Vert(1_F \mathop{\operatorname{\otimes}}^{B(\Omega)} \pi)(1_F
        \mathop{\operatorname{\otimes}}^{B(\Omega)}
        i)(u)\Vert_{F\mathop{\operatorname{\otimes}}^{B(\Omega)} X} = \Vert(1_F
        \mathop{\operatorname{\otimes}}^{B(\Omega)} \pi
        i)(u)\Vert_{F\mathop{\operatorname{\otimes}}^{B(\Omega)} X}
    $$
    $$
        =\Vert(1_F \mathop{\operatorname{\otimes}}^{B(\Omega)}
        1_X)(u)\Vert_{F\mathop{\operatorname{\otimes}}^{B(\Omega)} X} =\Vert
        u\Vert_{F\mathop{\operatorname{\otimes}}^{B(\Omega)} X}
    $$
    Also note that
    $\Vert(1_F \mathop{\operatorname{\otimes}}^{B(\Omega)} \pi)\Vert
        \leq\Vert 1_F\Vert\Vert\pi\Vert$, hence
    $$
        \Vert(1_F \mathop{\operatorname{\otimes}}^{B(\Omega)}
        i)(u)\Vert_{F\mathop{\operatorname{\otimes}}^{B(\Omega)} Y}
        \geq
        \Vert\pi\Vert^{-1}\Vert
        u\Vert_{F\mathop{\operatorname{\otimes}}^{B(\Omega)} X}
    $$
    Thus $1_F \mathop{\operatorname{\otimes}}^{B(\Omega)} i$ is topologically
    injective, so $F$ is relatively flat.

    Metric/extreme projectivity. Let $F,X,Y\in\operatorname{Ob}(\mathscr{L})$.
    Take arbitrary metrically/extremely admissible monomorphism
    $i\in\operatorname{Hom}_{\mathscr{L}^{\operatorname{op}}}(X, Y)$. By
    theorem~\ref{TopInjMultOpDescBtwnTwoMeasSp} we have coisometric
    $\pi\in\operatorname{Hom}_{\mathscr{L}^{\operatorname{op}}}(Y, X)$ such
    that $\pi i=1_{X}$. Fix
    $u\in F\mathop{\operatorname{\otimes}}^{B(\Omega)} X$. Since $\pi$ is
    cosiometric then from previous paragraph we get
    $$
        \Vert(1_F \mathop{\operatorname{\otimes}}^{B(\Omega)}
        i)(u)\Vert_{F\mathop{\operatorname{\otimes}}^{B(\Omega)} Y}
        \geq
        \Vert\pi\Vert^{-1}\Vert
        u\Vert_{F\mathop{\operatorname{\otimes}}^{B(\Omega)} X}
        \geq
        \Vert u\Vert_{F\mathop{\operatorname{\otimes}}^{B(\Omega)} X}
    $$
    On the other hand for the same $u$ we have
    $$
        \Vert(1_F \mathop{\operatorname{\otimes}}^{B(\Omega)}
        i)(u)\Vert_{F\mathop{\operatorname{\otimes}}^{B(\Omega)} Y}
        \leq
        \Vert 1_F \mathop{\operatorname{\otimes}}^{B(\Omega)} i\Vert\Vert
        u\Vert_{F\mathop{\operatorname{\otimes}}^{B(\Omega)} X}
        \leq
        \Vert 1_F\Vert\Vert i\Vert \Vert
        u\Vert_{F\mathop{\operatorname{\otimes}}^{B(\Omega)} X} = \Vert
        u\Vert_{F\mathop{\operatorname{\otimes}}^{B(\Omega)} X}
    $$
    From these inequalities it follows that
    $1_F \mathop{\operatorname{\otimes}}^{B(\Omega)} i$ is isometric.
\end{proof}

\begin{thebibliography}{999}
    \bibitem{RoyJAtNonAtMeas}
    \textit{Roy A. Johnson} Atomic and nonatomic measures.
    Proc. Amer. Math. Soc. 25 (1970), 650--655.
    \bibitem{SierpConFamlFunc}
    \textit{Sierpinski, W.} (1922) Sur les fonctions d'ensemble additives
    et continues. Fundamenta Mathematicae 3: 240–246.
    \bibitem{RoyJLebDecompTh}
    \textit{Roy A. Johnson} On Lebesgue decomposition theorem.
    Proc. Amer. Math. Soc. 18 (1967), 628--632.
    \bibitem{KalAlbTopicsBanSpTh}
    \textit{Fernando Albiac, Nigel J. Kalton} Topics in Banach space theory.
    Springer Inc. 2006.
    \bibitem{HelTensProdAndMultModLp}
    \textit{Helemskii A. Ya.} Tensor products and multiplicators of modules
    $L_p$ on locally compact spaces. Preprint, 2012.
    \bibitem{HelMetFrPoj}
    \textit{Helemskii A. Ya.} Metric freedom and projectivity for classic
    and quantum normed modules. Mat. Sb., 204:7 (2013), 127–158
\end{thebibliography}

Norbert Nemesh, Faculty of Mechanics and Mathematics,
Moscow State University, Moscow 119991 Russia

\textit{E-mail address:} nemeshnorbert@yandex.ru

\end{document}
