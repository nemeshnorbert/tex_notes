% chktex-file 35
\documentclass{article}

\def\baselinestretch{0.98}

\usepackage[T2C]{fontenc}
\usepackage[cp866]{inputenc}
\usepackage[english, russian]{babel}
\usepackage[tbtags]{amsmath}
\usepackage{amsfonts,amssymb,mathrsfs,amscd,amsmath,comment}
\usepackage[english]{msb-a}
\usepackage{tabularx}
\usepackage{rotating}
\usepackage[all]{xy}
\usepackage{hypbmsec}
\usepackage{enumitem}

% workaround for 
% https://github.com/AndreyAkinshin/Russian-Phd-LaTeX-Dissertation-Template/issues/413
\makeatletter\AtBeginDocument{\let\@elt\relax}\makeatother  % chktex 21

%-------------------------------------------------------------------------------
\theoremstyle{plain}
\newtheorem{theorem}{Theorem}[section]
\newtheorem{lemma}{Lemma}[section]
\newtheorem{corollary}{Corollary}[section]
\newtheorem{proposition}{Proposition}[section]
%-------------------------------------------------------------------------------
\theoremstyle{definition}
\newtheorem{definition}{Definition}
\newtheorem{proof}{Proof}\def\theproof{}
\newtheorem{remark}{Remark}[section]
%-------------------------------------------------------------------------------
% My definitions
\newcommand{\projtens}{\mathbin{\widehat{\otimes}}}
\newcommand{\convol}{\ast}
\newcommand{\projmodtens}[1]{\mathbin{\widehat{\otimes}}_{#1}}
\newcommand{\isom}{\mathop{\mathbin{\cong}}}


%-------------------------------------------------------------------------------
\begin{document}

\title{Topologically projective, injective and flat modules of harmonic
    analysis}
\author[Norbert~T.~Nemesh]{N.\,T.~Nemesh}
\address{Faculty of Mechanics and Mathematics,\\
Moscow State University}
\email{nemeshnorbert@yandex.ru}

\doi{10.XXXX/smXXXX}

\date{XX/XXX/XXXX}

\subjclass{Primary 46M10; Secondary 22D20}
\udk{517.968.22}

\maketitle  

\markright{projective, injective and flat modules of harmonic analysis}

\begin{fulltext}

\begin{abstract}
We study homologically trivial modules of harmonic analysis on a locally compact
group $G$. For $L_1(G)$- and $M(G)$-modules $C_0(G)$, $L_p(G)$ and $M(G)$ we
give criteria of metric and topological projectivity, injectivity and flatness.
In most cases, modules with these properties must be finite-dimensional.

Bibliography: 18~titles.
\end{abstract}

\begin{keywords}
Banach module, projectivity, injectivity, flatness, harmonic analysis.
\end{keywords}


\footnotetext[0]{This research was supported by the Russian Foundation for Basic
    Research [19--01--00447--a].}

\section{Introduction}\label{SectionIntroduction}

Banach homology has a long history dating back to the 1950s. One of the main
questions of this discipline: whether a given Banach module is homologically
trivial, i.e.\ projective, injective or flat? An example of a successful answer
to this question is the work of Dales, Polyakov, Ramsden and Racher
~\cite{DalPolHomolPropGrAlg, RamsHomPropSemgroupAlg,RachInjModAndAmenGr}, where
they gave criteria of homological triviality for classical modules of harmonic
analysis. It is worth mentioning that all these studies were carried out for
relative Banach homology. We answer the same questions but for two less explored
versions of Banach homology --- topological and metric ones. Metric Banach
homology was introduced by Graven in~\cite{GravInjProjBanMod}, where he applied
modern, at that moment, homological and Banach geometric techniques to modules
of harmonic analysis. The notion of topological Banach homology appeared in the
work of White~\cite{WhiteInjmoduAlg}. Seemingly, the latter theory looks much
less restrictive then the metric one, but as we shall see this is not the case.

%-------------------------------------------------------------------------------
%    Preliminaries on Banach homology
%-------------------------------------------------------------------------------

\section{Preliminaries on Banach
    homology}\label{SectionPreliminariesOnBanachHomology} 

In what follows, we give several versions of our statements in parallel, listing

the respective options in angle brackets and separating them by slashes, like
$\langle$~$\ldots$ / $\ldots$~$\rangle$. For example, a real number $x$ is
called $\langle$~positive / non-negative~$\rangle$ if $\langle$~$x>0$ / $x\geq
0$~$\rangle$.

All Banach spaces under consideration are over the field of complex numbers. Let
$E$ be a Banach space. By $B_E$ we denote the closed unit ball of $E$. If $F$ is
another Banach space, then a bounded linear operator $T:E\to F$ is called
\emph{$\langle$~isometric / $c$-topologically injective~$\rangle$} if
$\langle$~$\Vert T(x)\Vert=\Vert x\Vert$ / $c\Vert T(x)\Vert\geq\Vert
x\Vert$~$\rangle$ for all $x\in E$. Similarly, $T$ is \emph{$\langle$~strictly
coisometric / strictly $c$-topologically surjective~$\rangle$} if
$\langle$~$T(B_E)=B_F$ / $c T(B_E)\supset B_F$~$\rangle$. In some cases, the
constant $c$ is omitted. We use the symbol $\bigoplus_p$ for an $\ell_p$-sum of
Banach spaces, and $\projtens$ for a projective tensor product of Banach spaces.


By $A$ we denote an arbitrary Banach algebra. The symbol $A_+$ stands for the
standard unitization of $A$. In what follows we shall consider Banach modules
with contractive outer action only. A Banach module $A$-module $X$ is called
$\langle$~essential / faithful~$\rangle$ if $\langle$~the linear span of $A\cdot
X$ is dense in $X$ / $a\cdot X=\{0\}$ implies $a=0$~$\rangle$. A bounded linear
operator which is also a morphism of $A$-modules is called an $A$-morphism. The
symbol $A-\mathbf{mod}$ stands for the category of left Banach $A$-modules with
$A$-morphisms. By $A-\mathbf{mod}_1$ we denote the subcategory of
$A-\mathbf{mod}$ with the same objects, but contractive $A$-morphisms only. The
similar categories of right modules are denoted by $\mathbf{mod}-A$ and
$\mathbf{mod}_1-A$, respectively. We use the symbol $\isom$ to denote an
isomorphism of two objects in a category. By $\projtens_A$ we denote the functor
of projective module tensor product and by $\operatorname{Hom}$, the usual
morphism functor. Now we can give our main definitions.

\begin{definition} A left Banach $A$-module $P$ is \emph{$\langle$~metrically /
$C$-topologically / $C$-relatively~$\rangle$ projective} if the morphism functor
\begin{center}
$\langle$~$\operatorname{Hom}_{A-\mathbf{mod}_1}(P,-)$ /
$\operatorname{Hom}_{A-\mathbf{mod}}(P,-)$ /
$\operatorname{Hom}_{A-\mathbf{mod}}(P,-)$~$\rangle$
\end{center}
maps all $\langle$~strictly coisometric morphisms / strictly $c$-topologically
surjective morph-\\isms / morphisms with right inverse operator of norm at most
$c$~$\rangle$ to $\langle$~strictly coisometric / strictly $c C$-topologically
surjective / strictly $c C$-topologically surjec-\\tive~$\rangle$ operators.
\end{definition}

\begin{definition} A right Banach $A$-module $J$ is \emph{$\langle$~metrically /
$C$-topologically / $C$-relatively~$\rangle$ injective} if the morphism functor
\begin{center}
$\langle$~$\operatorname{Hom}_{\mathbf{mod}_1-A}(-,J)$ /
$\operatorname{Hom}_{\mathbf{mod}-A}(-,J)$ /
$\operatorname{Hom}_{\mathbf{mod}-A}(-,J)$~$\rangle$    
\end{center}
maps all $\langle$~strictly isometric morphisms / $c$-topologically injective
morphisms / morphisms with left inverse operator of norm at most $c$~$\rangle$
to $\langle$~strictly coisometric / strictly $c C$-topologically surjective /
strictly $c C$-topologically surjective~$\rangle$ operators.
\end{definition}

\begin{definition} A left Banach $A$-module $F$ is \emph{$\langle$~metrically /
$C$-topologically / $C$-relatively~$\rangle$ flat} if the functor of module
tensor product $-\projtens_A F$ maps all $\langle$~isometric morphisms /
$c$-topologically injective morphisms / morphisms with left inverse operator of
norm at most $c$~$\rangle$ to $\langle$~isometric / $cC$-topologically injective
/ $cC$-topo-\\logically injective~$\rangle$ operators.
\end{definition}

We shall say that a Banach module is $\langle$~topologically /
relatively~$\rangle$ projective, injective or flat if it is
$\langle$~$C$-topologically / $C$-relatively~$\rangle$ projective, injective or
flat for some $C>0$.

These definitions were given in a slightly different form by Graven for metric
theory~\cite{GravInjProjBanMod}, by White for topological theory
~\cite{WhiteInjmoduAlg} and by Helemskii for relative
theory~\cite{HelemHomolDimNorModBanAlg}. For topologically projective, injective
and flat module White used the term strictly projective, injective and flat
respectively. It is worth mentioning that initially strictly injective and flat
modules where introduced by Helemskii in~\cite[section
VII.1]{HelBanLocConvAlg}. An overview of the basics of these theories is given
in~\cite{NemGeomProjInjFlatBanMod}. We shall heavily rely upon results of the
latter paper.


%-------------------------------------------------------------------------------
%    Preliminaries on harmonic analysis
%-------------------------------------------------------------------------------

\section{Preliminaries on harmonic
    analysis}\label{SectionPreliminariesOnHarmonicAnalysis} 

Let $G$ be a locally compact group with unit $e_G$. The left Haar measure of $G$
is denoted by $m_G$ and the symbol $\Delta_G$ stands for the modular function of
$G$. For $\langle$~an infinite and discrete / a  compact~$\rangle$ group $G$ we
choose $m_G$ as a $\langle$~counting / probability~$\rangle$ measure. In what
follows for $1\leq p\leq+\infty$ we use the notation $L_p(G)$ to denote the
Lebesgue space of functions that are $p$-integrable with respect to Haar
measure.

We regard $L_1(G)$ as a Banach algebra with convolution operator in the role of
multiplication. This Banach algebra has a contractive two-sided approximate
identity~\cite[theorem 3.3.23]{DalBanAlgAutCont}. Clearly, $L_1(G)$ is unital
if and only if $G$ is discrete. In this case, $\delta_{e_G}$, the indicator
function of $e_G$, is the identity of $L_1(G)$. Similarly, the space of complex
finite Borel regular measures $M(G)$ endowed with convolution becomes a unital
Banach algebra. The role of identity is played by Dirac delta measure
$\delta_{e_G}$ supported on $e_G$. Moreover, $M(G)$ is a coproduct, in the sense
of category theory, in $L_1(G)-\mathbf{mod}_1$ (but not in
$M(G)-\mathbf{mod}_1$) of the two-sided ideal $M_a(G)$ of measures absolutely
continuous with respect to $m_G$ and the subalgebra $M_s(G)$ of measures
singular with respect to $m_G$. Note that $M_a(G)\isom L_1(G)$ in
$M(G)-\mathbf{mod}_1$ and $M_s(G)$ is an annihilator $L_1(G)$-module. Finally,
$M(G)=M_a(G)$ if and only if $G$ is discrete. 

Now we proceed to discussion of the standard left and right modules over
algebras $L_1(G)$ and $M(G)$. The Banach algebra $L_1(G)$ can be regarded as a
two-sided ideal of $M(G)$ by means of isometric left and right $M(G)$-morphism
$i:L_1(G)\to M(G):f\mapsto f m_G$. Therefore it is enough to define all module
structures over $M(G)$. For any $1\leq p<+\infty$, $f\in L_p(G)$ and $\mu\in
M(G)$ we define
\[
(\mu\convol_p f)(s)=\int_G f(t^{-1}s)d\mu(t),
\qquad
(f \convol_p \mu)(s)=\int_G f(st^{-1}){\Delta_G(t^{-1})}^{1/p}d\mu(t).
\]
These module actions turn all Banach spaces $L_p(G)$ for $1\leq p<+\infty$ into
left and right $M(G)$-modules. Note that for $p=1$ and $\mu\in M_a(G)$ we get
the usual definition of convolution. For $1<p\leq +\infty$, $f\in L_p(G)$ and
$\mu\in M(G)$ we define module actions as
\[
(\mu\cdot_p f)(s)=\int_G {\Delta_G(t)}^{1/p}f(st)d\mu(t),
\qquad
(f\cdot_p \mu)(s)=\int_G f(ts)d\mu(t).
\]
These module actions turn all Banach spaces $L_p(G)$ for $1<p\leq+\infty$ into
left and right $M(G)$-modules too. This special choice of module structure
nicely interacts with duality. Indeed we have and ${(L_p(G),\convol_p)}^*\isom
(L_{p^*}(G),\cdot_{p^*})$ in $\mathbf{mod}_1-M(G)$ for all $1\leq p<+\infty$.
Here we set by definition $p^*=p/(p-1)$ for $1<p<+\infty$ and $p^*=\infty$ for
$p=1$. Finally, the Banach space $C_0(G)$ also becomes a left and a right
$M(G)$-module when endowed with $\cdot_\infty$ in the role of module action.
Even more, $C_0(G)$ is a closed left and right $M(G)$-submodule of $L_\infty(G)$
such that ${(C_0(G),\cdot_\infty)}^*\isom (M(G),\convol)$ in
$M(G)-\mathbf{mod}_1$.

By $\widehat{G}$ we shall denote the dual group of the group $G$. Any character
$\gamma\in\widehat{G}$ gives rise to a continuous characters  
\[
\varkappa_\gamma^L:L_1(G)\to\mathbb{C}:f\mapsto \int_G f(s)\overline{\gamma(s)}d m_G(s),
\quad
\varkappa_\gamma^M:M(G)\to\mathbb{C}:\mu\mapsto\int_{G} \overline{\gamma(s)}d\mu(s).
\]
on $L_1(G)$ and $M(G)$ respectively. By $\mathbb{C}_\gamma$ we denote left and
right augmentation $L_1(G)$- or $M(G)$-module. Its module actions are defined by
\[
f\cdot_{\gamma}z=z\cdot_{\gamma}f=\varkappa_\gamma^L(f)z,
\qquad
\mu\cdot_{\gamma}z=z\cdot_{\gamma}\mu=\varkappa_\gamma^M(\mu)z
\]
for all $f\in L_1(G)$, $\mu\in M(G)$ and $z\in\mathbb{C}$. 

One of the numerous definitions of amenable group says, that a locally compact
group $G$ is amenable if there exists an $L_1(G)$-morphism of right modules
$M:L_\infty(G)\to\mathbb{C}_{e_{\widehat{G}}}$ such that $M(\chi_G)=1$
~\cite[section 7.2.5]{HelBanLocConvAlg}. We can even assume that $M$ is
contractive~\cite[remark 7.1.54]{HelBanLocConvAlg}.

Most of the results of this section that are not supported with references are
presented in full detail in~\cite[section 3.3]{DalBanAlgAutCont}.

%-------------------------------------------------------------------------------
%    L_1(G)-modules
%-------------------------------------------------------------------------------

\section{\texorpdfstring{$L_1(G)$}{L1(G)}-modules}\label{SubSectionL1GModules}

Metric homological properties of $L_1(G)$-modules of harmonic analysis were
first studied in~\cite{GravInjProjBanMod}. We generalise these ideas for the
case of topological Banach homology. To clarify the definitions we start from a
general result on injectivity. It is instructive to prove it from the first
principles.

\begin{proposition}\label{AlgDualWithApproxIdIsMetrInj} Let $A$ be a Banach
algebra with a right  contractive approx-\\imate identity, then the right
$A$-module $A^*$ is metrically injective. 
\end{proposition}
\begin{proof} Let $\xi:Y\to X$ be an isometric $A$-morphism of right $A$-modules
$X$ and $Y$ and an arbitrary contractive $A$-morphism $\phi: X\to A^*$. By
assumption $A$ has a contractive approximate identity, say ${(e_\nu)}_{\nu\in
N}$. For each $\nu\in N$ we define a bounded linear functional
$f_\nu:Y\to\mathbb{C}:y\to \phi(y)(e_\nu)$. By Hahn-Banach theorem there exists
a bounded linear functional $g_\nu:X\to\mathbb{C}$ such that $g_\nu\xi=f_\nu$
and $\Vert g_\nu\Vert=\Vert f_\nu\Vert$. It is routine to check that $\psi_\nu:
X\to A^*:x\mapsto(a\mapsto g_\nu(x\cdot a))$ is an $A$-morphism of right modules
such that $\Vert\psi_\nu\Vert\leq\Vert\phi\Vert$ and
$\psi_\nu(\xi(x))(a)=\phi(x)(a e_\nu)$ for all $x\in X$ and $a\in A$. Since the
net ${(\psi_\nu)}_{\nu\in N}$ is norm bounded then there exists a subnet
${(\psi_\mu)}_{\mu\in M}$ with the same norm bound that converges in
strong-to-weak${}^*$ topology to some operator $\psi:X\to A^*$. Clearly, $\psi$
is a morphism of right $A$-modules such that $\psi\xi=\phi$ and
$\Vert\psi\Vert\leq\Vert\phi\Vert$. As $\phi$ is arbitrary, the map
$\operatorname{Hom}_{\mathbf{mod}_1-A}(\xi, A^*)$ is strictly coisometric. Hence
$A^*$ is metrically injective.
\end{proof}

\begin{proposition}\label{LInfIsL1MetrInj} Let $G$ be a locally compact group.
Then $L_\infty(G)$ is a metrically and topologically injective $L_1(G)$-module.
As the result, $L_1(G)$-module $L_1(G)$ is metrically and topologically flat.
\end{proposition}
\begin{proof} As $L_1(G)$ has a contractive approximate identity, then by
proposition~\ref{AlgDualWithApproxIdIsMetrInj} the right $L_1(G)$-module
${L_1(G)}^*$ is metrically injective. As a consequence it is topologically
injective~\cite[proposition 2.14]{NemGeomProjInjFlatBanMod}. Therefore it
remains to recall that $L_\infty(G)\isom {L_1(G)}^*$ in $\mathbf{mod}_1-L_1(G)$.
The result on flatness of $L_1(G)$ follows from~\cite[proposition
2.21]{NemGeomProjInjFlatBanMod}.
\end{proof}

\begin{proposition}\label{OneDimL1ModMetTopProjCharac} Let $G$ be a locally
compact group, and $\gamma\in\widehat{G}$. Then the following are equivalent:
\begin{enumerate}[label = (\roman*)]
    \item $G$ is compact;\label{OneDimL1ModMetTopProjCharac:i}
 
    \item $\mathbb{C}_\gamma$ is a metrically projective
    $L_1(G)$-module;\label{OneDimL1ModMetTopProjCharac:ii}
 
    \item $\mathbb{C}_\gamma$ is a topologically projective
    $L_1(G)$-module.\label{OneDimL1ModMetTopProjCharac:iii}
\end{enumerate}
\end{proposition}
\begin{proof}~\ref{OneDimL1ModMetTopProjCharac:i}
$\Longrightarrow$~\ref{OneDimL1ModMetTopProjCharac:ii},
~\ref{OneDimL1ModMetTopProjCharac:iii}
$\Longrightarrow$~\ref{OneDimL1ModMetTopProjCharac:i} The proof is similar
to~\cite[theorem 4.2]{GravInjProjBanMod}. 

~\ref{OneDimL1ModMetTopProjCharac:ii}
$\Longrightarrow$~\ref{OneDimL1ModMetTopProjCharac:iii} Implication follows
from~\cite[proposition 2.4]{NemGeomProjInjFlatBanMod}.
\end{proof}

\begin{proposition}\label{OneDimL1ModMetTopInjFlatCharac} Let $G$ be a locally
compact group, and $\gamma\in\widehat{G}$. Then the following are equivalent:
\begin{enumerate}[label = (\roman*)]
    \item $G$ is amenable;\label{OneDimL1ModMetTopInjFlatCharac:i}
    \item $\mathbb{C}_\gamma$ is a metrically injective 
    $L_1(G)$-module;\label{OneDimL1ModMetTopInjFlatCharac:ii}
 
    \item $\mathbb{C}_\gamma$ is a topologically injective 
    $L_1(G)$-module;\label{OneDimL1ModMetTopInjFlatCharac:iii}
 
    \item $\mathbb{C}_\gamma$ is a metrically flat 
    $L_1(G)$-module;\label{OneDimL1ModMetTopInjFlatCharac:iv}
 
    \item $\mathbb{C}_\gamma$ is a topologically flat 
    $L_1(G)$-module.\label{OneDimL1ModMetTopInjFlatCharac:v}
\end{enumerate}
\end{proposition}
\begin{proof} 
~\ref{OneDimL1ModMetTopInjFlatCharac:i}
$\Longrightarrow$~\ref{OneDimL1ModMetTopInjFlatCharac:ii},
~\ref{OneDimL1ModMetTopInjFlatCharac:iii}
$\Longrightarrow$~\ref{OneDimL1ModMetTopInjFlatCharac:i} 
The proof is similar to~\cite[theorem 4.5]{GravInjProjBanMod}.

~\ref{OneDimL1ModMetTopInjFlatCharac:ii}
$\Longrightarrow$~\ref{OneDimL1ModMetTopInjFlatCharac:iii} This implication
immediately follows from~\cite[proposition 2.14]{NemGeomProjInjFlatBanMod}.

~\ref{OneDimL1ModMetTopInjFlatCharac:ii}
$\Longrightarrow$~\ref{OneDimL1ModMetTopInjFlatCharac:iv},
~\ref{OneDimL1ModMetTopInjFlatCharac:iii}
$\Longrightarrow$~\ref{OneDimL1ModMetTopInjFlatCharac:v} Note that
$\mathbb{C}_\gamma^*\isom \mathbb{C}_\gamma$ in $\mathbf{mod}_1-L_1(G)$, so all
equivalences follow from three previous paragraphs and the fact that flat
modules are precisely the modules with injective dual~\cite[proposition
2.21]{NemGeomProjInjFlatBanMod}.
\end{proof}

In the following proposition we shall study specific ideals of the Banach
algebra $L_1(G)$, namely the ideals of the form $L_1(G)\convol\mu$ for some
idempotent measure $\mu$. In fact, this class of ideals for the case of
commutative compact groups coincides with those left ideals of $L_1(G)$ that
admit a right bounded approximate identity.

\begin{theorem}\label{CommIdealByIdemMeasL1MetTopProjCharac} Let $G$ be a
locally compact group and  $\mu\in M(G)$ be an idempotent measure, that is
$\mu\convol\mu=\mu$. Assume that the left ideal $I=L_1(G)\convol\mu$ of the
Banach algebra $L_1(G)$ is a topologically projective $L_1(G)$-module. Then
$\mu=p m_G$, for some $p\in I$.
\end{theorem}
\begin{proof} Let $\phi:I\to L_1(G)$ be an arbitrary morphism of left
$L_1(G)$-modules. Consider $L_1(G)$-morphism $\phi':L_1(G)\to
L_1(G):x\mapsto\phi(x\convol\mu)$. By Wendel's theorem~\cite[theorem
1]{WendLeftCentrzrs}, there exists a measure $\nu\in M(G)$ such that
$\phi'(x)=x\convol\nu$ for all $x\in L_1(G)$. In particular,
$\phi(x)=\phi(x\convol\mu)=\phi'(x)=x\convol\nu$ for all $x\in I$. It is clear
now that $\psi:I\to I:x\mapsto\nu\convol x$ is a morphism of right $I$-modules
satisfying $\phi(x)y=x\psi(y)$ for all $x,y\in I$. By paragraph \textup{(ii)} of
~\cite[lemma 2]{NemMetTopProjIdBanAlg} the ideal $I$ has a right identity, say
$e\in I$. Then $x\convol\mu=x\convol\mu\convol e$ for all $x\in L_1(G)$. Two
measures are equal if their convolutions with all functions of $L_1(G)$ coincide
~\cite[corollary 3.3.24]{DalBanAlgAutCont}, so $\mu=\mu\convol e m_G$. Since
$e\in I\subset L_1(G)$, then $\mu=\mu\convol e m_G\in M_a(G)$. Set $p=\mu\convol
e\in I$, then $\mu=p m_G$.
\end{proof}

We conjecture that a left ideal of the form $L_1(G)\convol \mu$ for an
idempotent measure $\mu$ is a metrically projective $L_1(G)$-module if and only
if $\mu=p m_G$ for $p\in I$ with $\Vert p\Vert=1$. 
In~\cite[theorem 4.14]{GravInjProjBanMod}, Graven gave a criterion of 
metric projectivity of $L_1(G)$-module $L_1(G)$. Now we can prove this 
fact as a mere corollary.

\begin{corollary}\label{L1ModL1MetTopProjCharac} Let $G$ be a locally compact
group. Then the following are equivalent:
\begin{enumerate}[label = (\roman*)]
    \item $G$ is discrete;\label{L1ModL1MetTopProjCharac:i}
    
    \item $L_1(G)$ is a metrically projective 
    $L_1(G)$-module;\label{L1ModL1MetTopProjCharac:ii}
 
    \item $L_1(G)$ is a topologically projective 
    $L_1(G)$-module.\label{L1ModL1MetTopProjCharac:iii}
\end{enumerate}
\end{corollary}
\begin{proof} 
~\ref{L1ModL1MetTopProjCharac:i}
$\Longrightarrow$~\ref{L1ModL1MetTopProjCharac:ii} If $G$ is discrete, then
$L_1(G)$ is unital with unit of norm $1$. From~\cite[proposition
7]{NemMetTopProjIdBanAlg} we conclude that $L_1(G)$ is metrically projective as
$L_1(G)$-module.

~\ref{L1ModL1MetTopProjCharac:ii}
$\Longrightarrow$~\ref{L1ModL1MetTopProjCharac:iii} This implication is a
direct corollary of~\cite[proposition 2.4]{NemGeomProjInjFlatBanMod}.

~\ref{L1ModL1MetTopProjCharac:iii}
$\Longrightarrow$~\ref{L1ModL1MetTopProjCharac:i} Clearly, $\delta_{e_G}$ is an
idempotent measure. Since $L_1(G)=L_1(G)\convol \delta_{e_G}$ is topologically
projective, then by proposition~\ref{CommIdealByIdemMeasL1MetTopProjCharac} we
have $\delta_{e_G}=f m_G$ for some $f\in L_1(G)$. This is possible only if $G$
is discrete.
\end{proof}

Note that $L_1(G)$-module $L_1(G)$ is relatively projective for any locally
compact group $G$~\cite[exercise 7.1.17]{HelBanLocConvAlg}.

\begin{proposition}\label{L1MetTopProjAndMetrFlatOfMeasAlg} Let $G$ be a locally
compact group. Then the following are equivalent:
\begin{enumerate}[label = (\roman*)]
    \item $G$ is discrete;\label{L1MetTopProjAndMetrFlatOfMeasAlg:i}
    \item $M(G)$ is a metrically projective 
    $L_1(G)$-module;\label{L1MetTopProjAndMetrFlatOfMeasAlg:ii}
 
    \item $M(G)$ is a topologically projective 
    $L_1(G)$-module;\label{L1MetTopProjAndMetrFlatOfMeasAlg:iii}
    
    \item $M(G)$ is a metrically flat 
    $L_1(G)$-module.\label{L1MetTopProjAndMetrFlatOfMeasAlg:iv}
\end{enumerate}
\end{proposition}
\begin{proof} 
~\ref{L1MetTopProjAndMetrFlatOfMeasAlg:i}
$\Longrightarrow$~\ref{L1MetTopProjAndMetrFlatOfMeasAlg:ii} We have $M(G)\isom
L_1(G)$ in $L_1(G)-\mathbf{mod}_1$ for discrete $G$, so the result follows from
theorem~\ref{L1ModL1MetTopProjCharac}. 

~\ref{L1MetTopProjAndMetrFlatOfMeasAlg:ii}
$\Longrightarrow$~\ref{L1MetTopProjAndMetrFlatOfMeasAlg:iii}
See~\cite[proposition 2.4]{NemGeomProjInjFlatBanMod}.

~\ref{L1MetTopProjAndMetrFlatOfMeasAlg:ii}
$\Longrightarrow$~\ref{L1MetTopProjAndMetrFlatOfMeasAlg:iv} Implication follows
from~\cite[proposition 2.26]{NemGeomProjInjFlatBanMod}.

~\ref{L1MetTopProjAndMetrFlatOfMeasAlg:iii}
$\Longrightarrow$~\ref{L1MetTopProjAndMetrFlatOfMeasAlg:i} Recall that
$M(G)\isom L_1(G)\bigoplus_1 M_s(G)$ in $L_1(G)-\mathbf{mod}_1$, so $M_s(G)$ is
topologically projective as a retract of a topologically projective
module~\cite[proposition 2.2]{NemGeomProjInjFlatBanMod}. Note that $M_s(G)$ is
also an annihilator $L_1(G)$-module, therefore the algebra $L_1(G)$ has a right
identity~\cite[proposition 3.3]{NemGeomProjInjFlatBanMod}. Recall that $L_1(G)$
also has a two-sided bounded approximate identity, so $L_1(G)$ is unital. The
latter is equivalent to $G$ being discrete.

~\ref{L1MetTopProjAndMetrFlatOfMeasAlg:iv}
$\Longrightarrow$~\ref{L1MetTopProjAndMetrFlatOfMeasAlg:i} Note that $M(G)\isom
L_1(G)\bigoplus_1 M_s(G)$ in $L_1(G)-\mathbf{mod}_1$, so $M_s(G)$ is metrically
flat as a retract of a metrically flat module~\cite[proposition
2.27]{NemGeomProjInjFlatBanMod}. Recall also that $M_s(G)$ is an annihilator
module over a non-zero algebra $L_1(G)$, therefore $M_s(G)$ must be a zero
module~\cite[proposition 3.6]{NemGeomProjInjFlatBanMod}. The latter is
equivalent to $G$ being discrete.
\end{proof}

\begin{proposition}\label{MeasAlgIsL1TopFlat} Let $G$ be a locally compact
group. Then $M(G)$ is a topologi-\\cally flat $L_1(G)$-module.
\end{proposition}
\begin{proof} Since $M(G)$ is an $L_1$-space it is a fortiori an
$\mathcal{L}_1^g$-space~\cite[paragraph 3.13, exercise
4.7(b)]{DefFloTensNorOpId}. Since $M_s(G)$ is complemented in $M(G)$, then
$M_s(G)$ is an $\mathcal{L}_1^g$-space too~\cite[corollary
23.2.1(2)]{DefFloTensNorOpId}. Moreover, since $M_s(G)$ is an annihilator
$L_1(G)$-module, hence it is a topologically flat $L_1(G)$-module
~\cite[proposition 3.6]{NemGeomProjInjFlatBanMod}. The $L_1(G)$-module $L_1(G)$
is also topologically flat by proposition~\ref{LInfIsL1MetrInj}. Note that
$M(G)\isom L_1(G)\bigoplus_1 M_s(G)$ in $L_1(G)-\mathbf{mod}_1$, so the
$L_1(G)$-module $M(G)$ is topologi-\\cally flat as a sum of topologically flat
modules~\cite[proposition 2.27]{NemGeomProjInjFlatBanMod}.
\end{proof}

%-------------------------------------------------------------------------------
%    M(G)-modules
%-------------------------------------------------------------------------------

\section{\texorpdfstring{$M(G)$}{M (G)}-modules}\label{SubSectionMGModules}

We turn to study the standard $M(G)$-modules of harmonic analysis. As we shall
see, most of the results can be derived from previous theorems and proposition
on $L_1(G)$-modules.

\begin{proposition}\label{MGMetTopProjInjFlatRedToL1} Let $G$ be a locally
compact group, and $X$ be $\langle$~an essential / a faithful / an
essential~$\rangle$ $L_1(G)$-module. Then,
\begin{enumerate}[label = (\roman*)]
    \item $X$ is a metrically $\langle$~projective / injective / flat~$\rangle$
    $M(G)$-module if and only if it is a metrically $\langle$~projective /
    injective / flat~$\rangle$ 
    $L_1(G)$-module;\label{MGMetTopProjInjFlatRedToL1:i}
 
    \item $X$ is a topologically $\langle$~projective / injective /
    flat~$\rangle$ $M(G)$-module if and only if it is a topologically
    $\langle$~projective / injective / flat~$\rangle$ 
    $L_1(G)$-module.\label{MGMetTopProjInjFlatRedToL1:ii}
\end{enumerate}
\end{proposition}
\begin{proof} Recall that $L_1(G)$ is a two-sided contractively complemented
ideal of $M(G)$. Now~\ref{MGMetTopProjInjFlatRedToL1:i}
and~\ref{MGMetTopProjInjFlatRedToL1:ii} follow from $\langle$~\cite[proposition
2.6]{NemGeomProjInjFlatBanMod} /~\cite[proposition
2.16]{NemGeomProjInjFlatBanMod} /~\cite[proposition
2.24]{NemGeomProjInjFlatBanMod}~$\rangle$.
\end{proof} 

It is worth mentioning here that $L_1(G)$-modules $C_0(G)$, $L_p(G)$ for $1\leq
p<\infty$ and $\mathbb{C}_\gamma$ for $\gamma\in\widehat{G}$ are essential and
$L_1(G)$-modules $C_0(G)$, $M(G)$, $L_p(G)$ for $1\leq p\leq \infty$ and
$\mathbb{C}_\gamma$ for $\gamma\in\widehat{G}$ are faithful. 

\begin{proposition}\label{MGModMGMetTopProjFlatCharac} Let $G$ be a locally
compact group. Then $M(G)$ is metrically and topologically projective
$M(G)$-module. As the consequence it is metrically and topologically flat
$M(G)$-module.
\end{proposition} 
\begin{proof} Since $M(G)$ is a unital algebra, then $\langle$~metric /
topological~$\rangle$ projectivity of $M(G)$ follows from~\cite[proposition
7]{NemMetTopProjIdBanAlg}, since one may regard $M(G)$ as a unital ideal of
$M(G)$. It remains to recall that any $\langle$~metrically /
topologically~$\rangle$ projective module is $\langle$~metrically /
topologically~$\rangle$ flat~\cite[proposition 2.26]{NemGeomProjInjFlatBanMod}.
\end{proof}

%-------------------------------------------------------------------------------
%    Banach geometric restrictions
%-------------------------------------------------------------------------------

\section{Banach geometric restrictions}\label{
    SubSectionBanachGeometricRestriction
}

In this section we shall show that many modules of harmonic analysis fail to be
metrically or topologically projective, injective or flat for purely Banach
geometric reasons. In metric theory for infinite dimensional $L_1(G)$-modules
$L_p(G)$, $M(G)$ and $C_0(G)$ it was done in~\cite[theorems
4.12--4.14]{GravInjProjBanMod}.

\begin{proposition}\label{StdModAreNotRetrOfL1LInf} Let $G$ be an infinite
locally compact group. Then
\begin{enumerate}[label = (\roman*)]
    \item $L_1(G)$, $C_0(G)$, $M(G)$, ${L_\infty(G)}^*$ are not topologically
    injective Banach spaces;\label{StdModAreNotRetrOfL1LInf:i}
    
    \item $C_0(G)$, $L_\infty(G)$ are not complemented in any 
    $L_1$-space.\label{StdModAreNotRetrOfL1LInf:ii}
\end{enumerate}
\end{proposition}
\begin{proof}
Since $G$ is infinite all modules in question are infinite dimensional.

~\ref{StdModAreNotRetrOfL1LInf:i} If an infinite dimensional Banach space is
topologically injective, then it contains a copy of
$\ell_\infty(\mathbb{N})$~\cite[corollary 1.1.4]{RosOnRelDisjFamOfMeas}, and
consequently a copy of $c_0(\mathbb{N})$. The Banach space $L_1(G)$ is weakly
sequentially complete~\cite[corollary III.C.14]{WojBanSpForAnalysts}, so
by~\cite[corollary 5.2.11]{KalAlbTopicsBanSpTh} it can't contain a copy of
$c_0(\mathbb{N})$. Therefore, $L_1(G)$ is not a topologically injective Banach
space. Assume, that $M(G)$ is topologically injective, then so is its
complemented subspace $M_a(G)$, which is isometrically isomorphic to $L_1(G)$.
By previous argument this is impossible, contradiction. By corollary 3 of
~\cite{LauMingComplSubspInLInfOfG} the Banach space $C_0(G)$ is not complemented
in $L_\infty(G)$, hence it can't be topologically injective. Note that $L_1(G)$
is complemented in ${L_\infty(G)}^*$ which is isometrically isomorphic to
${L_1(G)}^{**}$~\cite[proposition  B10]{DefFloTensNorOpId}. Therefore, if
${L_\infty(G)}^*$ is topologically injective as a Banach space, then so is its
retract $L_1(G)$. By previous argument this is impossible, contradiction.

~\ref{StdModAreNotRetrOfL1LInf:ii} Suppose $C_0(G)$ is a retract of
$L_1$-space, then $M(G)$, which is isometrically isomorphic to ${C_0(G)}^*$, is a
retract of $L_\infty$-space. Therefore $M(G)$ must be a topologi-\\cally
injective Banach space. This contradicts
paragraph~\ref{StdModAreNotRetrOfL1LInf:i}. Note that $\ell_\infty(\mathbb{N})$
embeds in $L_\infty(G)$, hence so does $c_0(\mathbb{N})$. If $L_\infty(G)$ is a
retract of $L_1$-space, then there exists an $L_1$-space containing a copy of
$c_0(\mathbb{N})$. This is impossible as already showed in
paragraph~\ref{StdModAreNotRetrOfL1LInf:i}.
\end{proof}

From now on by $A$ we denote either $L_1(G)$ or $M(G)$. Recall that $L_1(G)$ and
$M(G)$ are both $L_1$-spaces.

\begin{proposition}\label{StdModAreNotL1MGMetTopProjInjFlat} Let $G$ be an
infinite locally compact group. Then
\begin{enumerate}[label = (\roman*)]
    \item $C_0(G)$, $L_\infty(G)$ are neither topologically nor metrically
    projective $A$-modules;\label{StdModAreNotL1MGMetTopProjInjFlat:i}
    
    \item $L_1(G)$, $C_0(G)$, $M(G)$, ${L_\infty(G)}^*$ are neither topologically
    nor metrically injec-\\tive 
    $A$-modules;\label{StdModAreNotL1MGMetTopProjInjFlat:ii}
    
    \item $L_\infty(G)$, $C_0(G)$ are neither topologically nor metrically flat
    $A$-modules;\label{StdModAreNotL1MGMetTopProjInjFlat:iii}
    
    \item $L_p(G)$ for $1<p<\infty$ are neither topologically nor metrically
    projective.\label{StdModAreNotL1MGMetTopProjInjFlat:iv}
\end{enumerate}
injective or flat $A$-modules.
\end{proposition}
\begin{proof}~\ref{StdModAreNotL1MGMetTopProjInjFlat:i} Every metrically or
topologically projective $A$-module is complemen-\\ted in some
$L_1$-space~\cite[proposition 3.8]{NemGeomProjInjFlatBanMod}. Now the result
follows from proposition~\ref{StdModAreNotRetrOfL1LInf}
paragraph~\ref{StdModAreNotRetrOfL1LInf:ii}.

~\ref{StdModAreNotL1MGMetTopProjInjFlat:ii} Every metrically or topologically
injective $A$-module is topologically injective as a Banach
space~\cite[proposition 3.8]{NemGeomProjInjFlatBanMod}. It remains to apply
proposition~\ref{StdModAreNotRetrOfL1LInf}
paragraph~\ref{StdModAreNotRetrOfL1LInf:i}.

~\ref{StdModAreNotL1MGMetTopProjInjFlat:iii} Note that ${C_0(G)}^*\isom M(G)$ in
$\mathbf{mod}_1-A$. Now the result follows from
paragraph~\ref{StdModAreNotRetrOfL1LInf:i} and the fact that the adjoint module
of a flat module is injective~\cite[proposition
2.21]{NemGeomProjInjFlatBanMod}.

~\ref{StdModAreNotL1MGMetTopProjInjFlat:iv} Since $L_p(G)$ is reflexive for
$1<p<\infty$ the result follows from~\cite[corollary
3.14]{NemGeomProjInjFlatBanMod}.
\end{proof}

Now we consider metric and topological homological properties of $A$-modules
when $G$ is finite.

\begin{proposition}\label{LpFinGrL1MGMetrInjProjCharac} Let $G$ be a non-trivial
finite group and $1\leq p\leq \infty$. Then the $A$-module $L_p(G)$ is
metrically $\langle$~projective / injective~$\rangle$ if and only if
$\langle$~$p=1$ / $p=\infty$~$\rangle$.
\end{proposition}
\begin{proof} 
Assume, $L_p(G)$ is metrically $\langle$~projective / injective~$\rangle$ as an
$A$-module. As $L_p(G)$ is finite-dimensional, there exist isometric
isomorphisms $\langle$~$L_p(G)\isom \ell_1(\mathbb{N}_n)$ / $L_p(G)\isom
\ell_\infty(\mathbb{N}_n)$~$\rangle$~\cite[proposition 3.8, paragraphs
\textup{(i)}, \textup{(ii)}]{NemGeomProjInjFlatBanMod}, where
$n=\operatorname{Card}(G)>1$. Now we use the result of theorem 1
from~\cite{LyubIsomEmdbFinDimLp} for Banach spaces over field $\mathbb{C}$: if
for $2\leq m\leq k$ and $1\leq r,s\leq \infty$, there exists an isometric
embedding from $\ell_r(\mathbb{N}_m)$ into $\ell_s(\mathbb{N}_k)$, then either
$r=2$, $s\in 2\mathbb{N}$ or $r=s$. Therefore $\langle$~$p=1$ /
$p=\infty$~$\rangle$. The converse easily follows from
$\langle$~theorem~\ref{L1ModL1MetTopProjCharac} / proposition
~\ref{LInfIsL1MetrInj}~$\rangle$.
\end{proof}

\begin{proposition}\label{StdModFinGrL1MGMetrInjProjFlatCharac} Let $G$ be a
finite group. Then
\begin{enumerate}[label = (\roman*)]
    \item $C_0(G)$, $L_\infty(G)$ are metrically injective 
    $A$-modules;\label{StdModFinGrL1MGMetrInjProjFlatCharac:i}
 
    \item $C_0(G)$ and $L_p(G)$ for $1<p\leq\infty$ are metrically projective
    $A$-modules if and only 
    if $G$ is trivial;\label{StdModFinGrL1MGMetrInjProjFlatCharac:ii}
 
    \item $M(G)$ and $L_p(G)$ for $1\leq p<\infty$ are metrically injective
    $A$-modules if and only 
    if $G$ is trivial;\label{StdModFinGrL1MGMetrInjProjFlatCharac:iii}
 
    \item $C_0(G)$ and $L_p(G)$ for $1<p\leq\infty$ are metrically flat
    $A$-modules if and only 
    if $G$ is trivial.\label{StdModFinGrL1MGMetrInjProjFlatCharac:iv}
\end{enumerate}
\end{proposition}
\begin{proof}
~\ref{StdModFinGrL1MGMetrInjProjFlatCharac:i} Since $G$ is finite then
$C_0(G)=L_\infty(G)$. The result follows from proposition
~\ref{LInfIsL1MetrInj}.

~\ref{StdModFinGrL1MGMetrInjProjFlatCharac:ii} If $G$ is trivial, that is
$G=\{e_G\}$, then $L_p(G)=C_0(G)=L_1(G)$ and the result follows from
paragraph~\ref{StdModFinGrL1MGMetrInjProjFlatCharac:i}. If $G$ is non trivial,
then we recall that $C_0(G)=L_\infty(G)$ and use
proposition~\ref{LpFinGrL1MGMetrInjProjCharac}.

~\ref{StdModFinGrL1MGMetrInjProjFlatCharac:iii} If $G=\{e_G\}$, then
$M(G)=L_p(G)=L_\infty(G)$ and the result follows from
paragraph~\ref{StdModFinGrL1MGMetrInjProjFlatCharac:i}. If $G$ is non-trivial,
then we note that $M(G)=L_1(G)$ and use proposition
~\ref{LpFinGrL1MGMetrInjProjCharac}.

~\ref{StdModFinGrL1MGMetrInjProjFlatCharac:iv} From
paragraph~\ref{StdModFinGrL1MGMetrInjProjFlatCharac:iii} it follows that
$L_p(G)$ for $1\leq p<\infty$ is a metrically injective $A$-module if and only
if $G$ is trivial. Recall that a Banach module is flat if and only if its
adjoint is injective~\cite[proposition 2.21]{NemGeomProjInjFlatBanMod}. Now the
result for $L_p(G)$ follows from identifications ${L_p(G)}^*\isom L_{p^*}(G)$ in
$\mathbf{mod}_1-L_1(G)$ for $1\leq p^*<\infty$. Similarly, using above
characterisation of flat modules and isomorphisms ${C_0(G)}^*\isom M(G)\isom
L_1(G)$ in $\mathbf{mod}_1-L_1(G)$ we get a criterion of injectivity of $M(G)$.
\end{proof}

It is worth mentioning here that if we consider all Banach spaces over the field
of real numbers, then $L_\infty(G)$ and $L_1(G)$ will be metrically projective
and injective respectively, for the group $G$ consisting of two elements. The
reason is that $L_\infty(\mathbb{Z}_2)\isom
\mathbb{R}_{\gamma_0}\bigoplus\nolimits_1\mathbb{R}_{\gamma_1}$ in
$L_1(\mathbb{Z}_2)-\mathbf{mod}_1$ and $L_1(\mathbb{Z}_2)\isom
\mathbb{R}_{\gamma_0}\bigoplus\nolimits_\infty\mathbb{R}_{\gamma_1}$ in
$\mathbf{mod}_1-L_1(\mathbb{Z}_2)$. Here, $\mathbb{Z}_2$ denotes the unique
group of two elements and $\gamma_0,\gamma_1\in\widehat{\mathbb{Z}_2}$ are
defined by $\gamma_0(0)=\gamma_0(1)=\gamma_1(0)=-\gamma_1(1)=1$.

\begin{proposition}\label{StdModFinGrL1MGTopInjProjFlatCharac} Let $G$ be a
finite group. Then the $A$-modules $C_0(G)$, $M(G)$, $L_p(G)$ for $1\leq p\leq
\infty$ are topologically projective, injective and flat.
\end{proposition} 
\begin{proof}
For a finite group $G$ we have $M(G)=L_1(G)$ and $C_0(G)=L_\infty(G)$, so
modules $C_0(G)$ and $M(G)$ do not require special considerations. Since
$M(G)=L_1(G)$, we can restrict our considerations to the case $A=L_1(G)$. The
identity map $i:L_1(G)\to L_p(G):f\mapsto f$ is a topological isomorphism of
Banach spaces, because $L_1(G)$ and $L_p(G)$ for $1\leq p<+\infty$ are of equal
finite dimension. Since $G$ is finite, it is unimodular. Therefore, the module
actions in $(L_1(G),\convol)$ and $(L_p(G),\convol_p)$ coincide for $1\leq
p<+\infty$. Hence $i$ is an isomorphism in $L_1(G)-\mathbf{mod}$ and
$\mathbf{mod}-L_1(G)$. Similarly one can show that $(L_\infty(G),\cdot_\infty)$
and $(L_p(G),\cdot_p)$ for $1<p\leq+\infty$ are isomorphic in
$L_1(G)-\mathbf{mod}$ and $\mathbf{mod}-L_1(G)$. Finally, one can easily check
that $(L_1(G),\convol)$ and $(L_\infty(G),\cdot_\infty)$ are isomorphic in
$L_1(G)-\mathbf{mod}$ and $\mathbf{mod}-L_1(G)$ via the map $j:L_1(G)\to
L_\infty(G):f\mapsto(s\mapsto f(s^{-1}))$. Thus all the modules in question are
pairwise isomorphic. It remains to recall that $L_1(G)$ is topologically
projective and flat by theorem~\ref{L1ModL1MetTopProjCharac} and proposition
~\ref{LInfIsL1MetrInj}, meanwhile $L_\infty(G)$ is topologically injective by
proposition~\ref{LInfIsL1MetrInj}.
\end{proof}

We summarise results on homological properties of modules of harmonic analysis
into the table~\ref{HomolTrivModMetTh}. Each cell of each table contains a
condition under which the respective module has the respective property and
references to the proofs. The arrow $\Longrightarrow$ indicates that only a
necessary condition is known. We should mention that results for modules
$L_p(G)$, where $1<p<\infty$, are valid for both module actions $\convol_p$ and
$\cdot_p$. Characterisations and proofs for homologically trivial modules
$\mathbb{C}_\gamma$ in the case of relative theory are the same as in
propositions~\ref{OneDimL1ModMetTopProjCharac}
and~\ref{OneDimL1ModMetTopInjFlatCharac}, but this results are already well
known. For example, projectivity of $\mathbb{C}_\gamma$ is characterized
in~\cite[theorem IV.5.13]{HelBanLocConvAlg}, and the criterion of injectivity
was given in~\cite[theorem 2.5]{JohnCohomolBanAlg}. For algebras $L_1(G)$ and
$M(G)$ the notions of $\langle$~projectivity / injectivity / flatness~$\rangle$
coincide for all three theories when one deals with modules $\langle$~$M(G)$ and
$\mathbb{C}_\gamma$ / $L_\infty(G)$, $C_0(G)$ and $\mathbb{C}_\gamma$ / $L_1(G)$
and $\mathbb{C}_\gamma$~$\rangle$. Finally, the $M(G)$-modules $M(G)$ also have
the same characterization of flatness in metric topological and relative theory.

\begin{table}[ht]
    \centering
    \caption{Homologically trivial modules of harmonic analysis}
    \begin{tiny}
        \begin{tabular}{|c|c|c|c|c|c|c|}
        \hline
             & 
             \multicolumn{3}{c|}{$L_1(G)$-modules} & 
             \multicolumn{3}{c|}{$M(G)$-modules} \\
        \hline
             & 
             Projectivity & 
             Injectivity & 
             Flatness & 
             Projectivity & 
             Injectivity & 
             Flatness \\
        \hline
             \multicolumn{7}{c}{\textbf{Metric theory}} \\
        \hline
            $L_1(G)$ & 
            \shortstack{
                $G$ is discrete \\ 
               {\ref{L1ModL1MetTopProjCharac}}
            } & 
            \shortstack{
                $G=\{e_G\}$ \\ 
               {\ref{StdModAreNotL1MGMetTopProjInjFlat}}, 
               {\ref{StdModFinGrL1MGMetrInjProjFlatCharac}}
            } &
            \shortstack{
                $G$ is any \\ 
               {\ref{LInfIsL1MetrInj}}
            } &
            \shortstack{
                $G$ is discrete \\ 
               {\ref{L1ModL1MetTopProjCharac}},
               {\ref{MGMetTopProjInjFlatRedToL1}}
            } &
            \shortstack{
                $G=\{e_G\}$ \\
               {\ref{StdModAreNotL1MGMetTopProjInjFlat}},
               {\ref{StdModFinGrL1MGMetrInjProjFlatCharac}}
            } & 
            \shortstack{
                $G$ is any \\
               {\ref{LInfIsL1MetrInj}},
               {\ref{MGMetTopProjInjFlatRedToL1}}
            } \\
        \hline
             $L_p(G)$ & 
             \shortstack{
                $G=\{e_G\}$ \\ 
               {\ref{StdModAreNotL1MGMetTopProjInjFlat}},
               {\ref{LpFinGrL1MGMetrInjProjCharac}}
            } &
            \shortstack{
                $G=\{e_G\}$ \\ 
               {\ref{StdModAreNotL1MGMetTopProjInjFlat}},
               {\ref{LpFinGrL1MGMetrInjProjCharac}}
            } & 
            \shortstack{
                $G=\{e_G\}$ \\
               {\ref{StdModAreNotL1MGMetTopProjInjFlat}},
               {\ref{StdModFinGrL1MGMetrInjProjFlatCharac}}
            } & 
            \shortstack{
                $G=\{e_G\}$ \\ 
               {\ref{StdModAreNotL1MGMetTopProjInjFlat}},
               {\ref{LpFinGrL1MGMetrInjProjCharac}}
            } & 
            \shortstack{
                $G=\{e_G\}$ \\ 
               {\ref{StdModAreNotL1MGMetTopProjInjFlat}},
               {\ref{LpFinGrL1MGMetrInjProjCharac}}
            } & 
            \shortstack{
                $G=\{e_G\}$ \\ 
               {\ref{StdModAreNotL1MGMetTopProjInjFlat}},
               {\ref{StdModFinGrL1MGMetrInjProjFlatCharac}}
            } \\
        \hline
            $L_\infty(G)$ & 
            \shortstack{$
                G=\{e_G\}$ \\ 
               {\ref{StdModAreNotL1MGMetTopProjInjFlat}},
               {\ref{LpFinGrL1MGMetrInjProjCharac}}
            } & 
            \shortstack{
                $G$ is any  \\ 
               {\ref{LInfIsL1MetrInj}}
            } & 
            \shortstack{
                $G=\{e_G\}$ \\ 
               {\ref{StdModAreNotL1MGMetTopProjInjFlat}},
               {\ref{StdModFinGrL1MGMetrInjProjFlatCharac}}
            } & 
            \shortstack{
                $G=\{e_G\}$ \\ 
               {\ref{StdModAreNotL1MGMetTopProjInjFlat}},
               {\ref{LpFinGrL1MGMetrInjProjCharac}}
            } & 
            \shortstack{
                $G$ is any  \\ 
               {\ref{LInfIsL1MetrInj}},
               {\ref{MGMetTopProjInjFlatRedToL1}}
            } & 
            \shortstack{
                $G=\{e_G\}$ \\ 
               {\ref{StdModAreNotL1MGMetTopProjInjFlat}},
               {\ref{StdModFinGrL1MGMetrInjProjFlatCharac}}
            } \\ 
        \hline
            $M(G)$ & 
            \shortstack{
                $G$ is discrete  \\ 
               {\ref{L1MetTopProjAndMetrFlatOfMeasAlg}}
            } & 
            \shortstack{
                $G=\{e_G\}$ \\ 
               {\ref{StdModAreNotL1MGMetTopProjInjFlat}},
               {\ref{StdModFinGrL1MGMetrInjProjFlatCharac}}
            } & 
            \shortstack{
                $G$ is discrete  \\ 
               {\ref{MeasAlgIsL1TopFlat}}
            } & 
            \shortstack{
                $G$ is any  \\ 
               {\ref{MGModMGMetTopProjFlatCharac}}
            } & 
            \shortstack{
                $G=\{e_G\}$ \\ 
               {\ref{StdModAreNotL1MGMetTopProjInjFlat}},
               {\ref{StdModFinGrL1MGMetrInjProjFlatCharac}}
            } & 
            \shortstack{
                $G$ is any  \\ 
               {\ref{MGModMGMetTopProjFlatCharac}}
            } \\ 
        \hline
            $C_0(G)$ & 
            \shortstack{
                $G=\{e_G\}$ \\         
               {\ref{StdModAreNotL1MGMetTopProjInjFlat}},
               {\ref{StdModFinGrL1MGMetrInjProjFlatCharac}}
            } & 
            \shortstack{
                $G$ is finite  \\ 
               {\ref{StdModAreNotL1MGMetTopProjInjFlat}},
               {\ref{StdModFinGrL1MGMetrInjProjFlatCharac}}
            } & 
            \shortstack{
                $G=\{e_G\}$ \\ 
               {\ref{StdModAreNotL1MGMetTopProjInjFlat}},
               {\ref{StdModFinGrL1MGMetrInjProjFlatCharac}}
            } & 
            \shortstack{
                $G=\{e_G\}$ \\ 
               {\ref{StdModAreNotL1MGMetTopProjInjFlat}},
               {\ref{StdModFinGrL1MGMetrInjProjFlatCharac}}
            } & 
            \shortstack{
                $G$ is finite  \\ 
               {\ref{StdModAreNotL1MGMetTopProjInjFlat}},
               {\ref{StdModFinGrL1MGMetrInjProjFlatCharac}}
            } & 
            \shortstack{
                $G=\{e_G\}$ \\ 
               {\ref{StdModAreNotL1MGMetTopProjInjFlat}},
               {\ref{StdModFinGrL1MGMetrInjProjFlatCharac}}
            } \\ 
        \hline
            $\mathbb{C}_\gamma$ & 
            \shortstack{
                $G$ is compact  \\{\ref{OneDimL1ModMetTopProjCharac}}
            } & 
            \shortstack{
                $G$ is amenable  \\ 
            {\ref{OneDimL1ModMetTopInjFlatCharac}}
            } & 
            \shortstack{
                $G$ is amenable  \\ 
            {\ref{OneDimL1ModMetTopInjFlatCharac}}
            } & 
            \shortstack{
                $G$ is compact  \\ 
            {\ref{OneDimL1ModMetTopProjCharac}},
            {\ref{MGMetTopProjInjFlatRedToL1}}
            } & 
            \shortstack{
                $G$ is amenable  \\ 
            {\ref{OneDimL1ModMetTopInjFlatCharac}},
            {\ref{MGMetTopProjInjFlatRedToL1}}
            } & 
            \shortstack{
                $G$ is amenable  \\ 
            {\ref{OneDimL1ModMetTopInjFlatCharac}},
            {\ref{MGMetTopProjInjFlatRedToL1}}
            } \\ 
        \hline
            \multicolumn{7}{c}{\textbf{Topological theory}} \\
        \hline
            $L_1(G)$ & 
            \shortstack{
                $G$ is discrete \\ 
            {\ref{L1ModL1MetTopProjCharac}}
            } & 
            \shortstack{
                $G$ is finite \\ 
            {\ref{StdModAreNotL1MGMetTopProjInjFlat}},
            {\ref{StdModFinGrL1MGTopInjProjFlatCharac}}
            } & 
            \shortstack{
                $G$ is any \\ 
            {\ref{LInfIsL1MetrInj}}
            } & 
            \shortstack{
                $G$ is discrete \\ 
            {\ref{L1ModL1MetTopProjCharac}},
            {\ref{MGMetTopProjInjFlatRedToL1}}
            } & 
            \shortstack{
                $G$ is finite \\ 
            {\ref{StdModAreNotL1MGMetTopProjInjFlat}},
            {\ref{StdModFinGrL1MGTopInjProjFlatCharac}}
            } & 
            \shortstack{
                $G$ is any \\ 
            {\ref{LInfIsL1MetrInj}},
            {\ref{MGMetTopProjInjFlatRedToL1}}
            } \\ 
        \hline
            $L_p(G)$ & 
            \shortstack{
                $G$ is finite \\ 
            {\ref{StdModAreNotL1MGMetTopProjInjFlat}},
            {\ref{StdModFinGrL1MGTopInjProjFlatCharac}}
            } & 
            \shortstack{
                $G$ is finite \\ 
            {\ref{StdModAreNotL1MGMetTopProjInjFlat}},
            {\ref{StdModFinGrL1MGTopInjProjFlatCharac}}
            } & 
            \shortstack{
                $G$ is finite \\ 
            {\ref{StdModAreNotL1MGMetTopProjInjFlat}},
            {\ref{StdModFinGrL1MGTopInjProjFlatCharac}}
            } & 
            \shortstack{
                $G$ is finite \\ 
            {\ref{StdModAreNotL1MGMetTopProjInjFlat}},
            {\ref{StdModFinGrL1MGTopInjProjFlatCharac}}
            } & 
            \shortstack{
                $G$ is finite \\ 
            {\ref{StdModAreNotL1MGMetTopProjInjFlat}},
            {\ref{StdModFinGrL1MGTopInjProjFlatCharac}}
            } & 
            \shortstack{
                $G$ is finite \\ 
            {\ref{StdModAreNotL1MGMetTopProjInjFlat}},
            {\ref{StdModFinGrL1MGTopInjProjFlatCharac}}
            } \\ 
        \hline
            $L_\infty(G)$ & 
            \shortstack{
                $G$ is finite \\ 
            {\ref{StdModAreNotL1MGMetTopProjInjFlat}},
            {\ref{StdModFinGrL1MGTopInjProjFlatCharac}}
            } & 
            \shortstack{
                $G$ is any \\ 
            {\ref{LInfIsL1MetrInj}}
            } & 
            \shortstack{
                $G$ is finite \\ 
            {\ref{StdModAreNotL1MGMetTopProjInjFlat}},
            {\ref{StdModFinGrL1MGTopInjProjFlatCharac}}
            } & 
            \shortstack{
                $G$ is finite \\ 
            {\ref{StdModAreNotL1MGMetTopProjInjFlat}},
            {\ref{StdModFinGrL1MGTopInjProjFlatCharac}}
            } & 
            \shortstack{
                $G$ is any \\ 
            {\ref{LInfIsL1MetrInj}},
            {\ref{MGMetTopProjInjFlatRedToL1}}
            } & 
            \shortstack{
                $G$ is finite \\ 
            {\ref{StdModAreNotL1MGMetTopProjInjFlat}},
            {\ref{StdModFinGrL1MGTopInjProjFlatCharac}}
            } \\ 
        \hline
            $M(G)$ & 
            \shortstack{
                $G$ is discrete \\ 
            {\ref{L1MetTopProjAndMetrFlatOfMeasAlg}}
            } & 
            \shortstack{
                $G$ is finite \\ 
            {\ref{StdModAreNotL1MGMetTopProjInjFlat}},
            {\ref{StdModFinGrL1MGTopInjProjFlatCharac}}
            } & 
            \shortstack{
                $G$ is any \\ 
            {\ref{MeasAlgIsL1TopFlat}}
            } & 
            \shortstack{
                $G$ is any \\ 
            {\ref{MGModMGMetTopProjFlatCharac}}
            } & 
            \shortstack{
                $G$ is finite \\ 
            {\ref{StdModAreNotL1MGMetTopProjInjFlat}},
            {\ref{StdModFinGrL1MGTopInjProjFlatCharac}}
            } & 
            \shortstack{
                $G$ is any \\ 
            {\ref{MGModMGMetTopProjFlatCharac}}
            } \\ 
        \hline
            $C_0(G)$ & 
            \shortstack{
                $G$ is finite \\ 
            {\ref{StdModAreNotL1MGMetTopProjInjFlat}},
            {\ref{StdModFinGrL1MGTopInjProjFlatCharac}}
            } & 
            \shortstack{
                $G$ is finite \\ 
            {\ref{StdModAreNotL1MGMetTopProjInjFlat}},
            {\ref{StdModFinGrL1MGTopInjProjFlatCharac}}
            } & 
            \shortstack{
                $G$ is finite \\ 
            {\ref{StdModAreNotL1MGMetTopProjInjFlat}},
            {\ref{StdModFinGrL1MGTopInjProjFlatCharac}}
            } & 
            \shortstack{
                $G$ is finite \\ 
            {\ref{StdModAreNotL1MGMetTopProjInjFlat}},
            {\ref{StdModFinGrL1MGTopInjProjFlatCharac}}
            } & 
            \shortstack{
                $G$ is finite \\ 
            {\ref{StdModAreNotL1MGMetTopProjInjFlat}},
            {\ref{StdModFinGrL1MGTopInjProjFlatCharac}}
            } & 
            \shortstack{
                $G$ is finite \\ 
            {\ref{StdModAreNotL1MGMetTopProjInjFlat}},
            {\ref{StdModFinGrL1MGTopInjProjFlatCharac}}
            } \\ 
        \hline
            $\mathbb{C}_\gamma$ & 
            \shortstack{
                $G$ is compact \\ 
            {\ref{OneDimL1ModMetTopProjCharac}}
            } & 
            \shortstack{
                $G$ is amenable \\ 
            {\ref{OneDimL1ModMetTopInjFlatCharac}}
            } & 
            \shortstack{
                $G$ is amenable \\ 
            {\ref{OneDimL1ModMetTopInjFlatCharac}}
            } & 
            \shortstack{
                $G$ is compact \\ 
            {\ref{OneDimL1ModMetTopProjCharac}},
            {\ref{MGMetTopProjInjFlatRedToL1}}
            } & 
            \shortstack{
                $G$ is amenable \\ 
            {\ref{OneDimL1ModMetTopInjFlatCharac}},
            {\ref{MGMetTopProjInjFlatRedToL1}}
            } & 
            \shortstack{
                $G$ is amenable \\ 
            {\ref{OneDimL1ModMetTopInjFlatCharac}},
            {\ref{MGMetTopProjInjFlatRedToL1}}
            } \\ 
        \hline
            \multicolumn{7}{c}{\textbf{Relative theory}} \\
        \hline
            $L_1(G)$ & 
            \shortstack{
                $G$ is any \\
				{\cite{DalPolHomolPropGrAlg}, \S 6}
            } & 
            \shortstack{
                $G$ is amenable \\  and discrete \\
                {\cite{DalPolHomolPropGrAlg}, \S 6}
            } & 
            \shortstack{
                $G$ is any \\
				{\cite{DalPolHomolPropGrAlg}, \S 6}
            } & 
            \shortstack{
                $G$ is any \\
				{\cite{RamsHomPropSemgroupAlg}, \S 3.5}
            } & 
            \shortstack{
                $G$ is amenable \\  and discrete \\
                {\cite{RamsHomPropSemgroupAlg}, \S 3.5}
            } & 
            \shortstack{
                $G$ is any \\
				{\cite{RamsHomPropSemgroupAlg}, \S 3.5}
            } \\ 
        \hline
            $L_p(G)$ & 
            \shortstack{
                $G$ is compact \\
				{\cite{DalPolHomolPropGrAlg}, \S 6}
            } & 
            \shortstack{
                $G$ is amenable \\
				{\cite{RachInjModAndAmenGr}}
            } & 
            \shortstack{
                $G$ is amenable \\
				{\cite{RachInjModAndAmenGr}}
            } & 
            \shortstack{
                $G$ is compact \\
				{\cite{RamsHomPropSemgroupAlg}, \S 3.5}
            } & 
            \shortstack{
                $G$ is amenable \\
				{\cite{RamsHomPropSemgroupAlg}, \S 3.5,}
                {\cite{RachInjModAndAmenGr}}
            } & 
            \shortstack{
                $G$ is amenable \\
				{\cite{RamsHomPropSemgroupAlg}, \S 3.5}
            } \\
        \hline
            $L_\infty(G)$ & 
            \shortstack{
                $G$ is finite \\
				{\cite{DalPolHomolPropGrAlg}, \S 6}
            } & 
            \shortstack{
                $G$ is any \\
				{\cite{DalPolHomolPropGrAlg}, \S 6}
            } & 
            \shortstack{
                $G$ is amenable \\
				{\cite{DalPolHomolPropGrAlg}, \S 6}
            } & 
            \shortstack{
                $G$ is finite \\
				{\cite{RamsHomPropSemgroupAlg}, \S 3.5}
            } & 
            \shortstack{
                $G$ is any \\
				{\cite{RamsHomPropSemgroupAlg}, \S 3.5}
            } & 
            \shortstack{
                $G$ is amenable \\ 
                ($\Longrightarrow$)\cite{RamsHomPropSemgroupAlg}, \S 3.5
            } \\ 
        \hline
            $M(G)$ & 
            \shortstack{
                $G$ is discrete \\
				{\cite{DalPolHomolPropGrAlg}, \S 6}
            } & 
            \shortstack{
                $G$ is amenable \\
				{\cite{DalPolHomolPropGrAlg}, \S 6}
            } & 
            \shortstack{
                $G$ is any \\
				{\cite{RamsHomPropSemgroupAlg}, \S 3.5}
            } & 
            \shortstack{
                $G$ is any \\
				{\cite{RamsHomPropSemgroupAlg}, \S 3.5}
            } & 
            \shortstack{
                $G$ is amenable \\
				{\cite{RamsHomPropSemgroupAlg}, \S 3.5}
            } & 
            \shortstack{
                $G$ is any \\
				{\cite{RamsHomPropSemgroupAlg}, \S 3.5}
            } \\ 
        \hline
            $C_0(G)$ & 
            \shortstack{
                $G$ is compact \\
				{\cite{DalPolHomolPropGrAlg}, \S 6}
            } & 
            \shortstack{
                $G$ is finite \\
				{\cite{DalPolHomolPropGrAlg}, \S 6}
            } & 
            \shortstack{
                $G$ is amenable \\
				{\cite{DalPolHomolPropGrAlg}, \S 6}
            } & 
            \shortstack{
                $G$ is compact \\
				{\cite{RamsHomPropSemgroupAlg}, \S 3.5}
            } & 
            \shortstack{
                $G$ is finite \\
				{\cite{RamsHomPropSemgroupAlg}, \S 3.5}
            } & 
            \shortstack{
                $G$ is amenable \\
				{\cite{RamsHomPropSemgroupAlg}, \S 3.5}
            } \\ 
        \hline
            $\mathbb{C}_\gamma$ & 
            \shortstack{
                $G$ is compact \\ 
            {\ref{OneDimL1ModMetTopProjCharac}}
            } & 
            \shortstack{
                $G$ is amenable \\ 
            {\ref{OneDimL1ModMetTopInjFlatCharac}}
            } & 
            \shortstack{
                $G$ is amenable \\ 
            {\ref{OneDimL1ModMetTopInjFlatCharac}}
            } & 
            \shortstack{
                $G$ is compact \\ 
            {\ref{OneDimL1ModMetTopProjCharac}},
            {\ref{MGMetTopProjInjFlatRedToL1}}
            } & 
            \shortstack{
                $G$ is amenable \\ 
            {\ref{OneDimL1ModMetTopInjFlatCharac}},
            {\ref{MGMetTopProjInjFlatRedToL1}}
            } & 
            \shortstack{
                $G$ is amenable \\ 
            {\ref{OneDimL1ModMetTopInjFlatCharac}},
            {\ref{MGMetTopProjInjFlatRedToL1}}
            } \\                   
        \hline
        \end{tabular}
    \end{tiny}\label{HomolTrivModMetTh}
\end{table}


\end{fulltext}

\begin{thebibliography}{99}

%
\RBibitem{DalPolHomolPropGrAlg}
    \by{H.\,G.~Dales, M.\,E.~Polyakov} 
    \paper{Homological properties of modules over group algebras}
    \jour{Proc. Lond.  Math. Soc.} 
    \vol{89}
    \issue{2} 
    \yr{2004} 
    \pages{390--426}

%
\RBibitem{RamsHomPropSemgroupAlg}
    \by{P.~Ramsden} 
    \thesis{Homological properties of semigroup algebras}
    \publaddr{The University of Leeds} 
    \yr{2009}

\RBibitem{RachInjModAndAmenGr}
    \by{G.~Racher} 
    \paper{Injective modules and amenable groups}
    \jour{Comment. Math. Helv.} 
    \vol{88}
    \issue{4} 
    \yr{2013} 
    \pages{1023--1031}

\RBibitem{GravInjProjBanMod}
    \by{A.\,W.\,M.~Graven} 
    \paper{Injective and projective Banach modules}
    \jour{Indag. Math.}
    \publ{Elsevier} 
    \vol{82}
    \issue{1} 
    \yr{1979}
    \pages{253--272}

\RBibitem{WhiteInjmoduAlg}
    \by{M.\,C.~White}
    \paper{Injective modules for uniform algebras}
    \jour{Proc. London Math. Soc.} 
    \vol{73}
    \issue{1}
    \yr{1996}
    \pages{155--184}

\RBibitem{HelemHomolDimNorModBanAlg}
    \by{A.\,Ya.~Helemskii.}
    \paper{On the homological dimension of normed modules over Banach algebras}
    \jour{Mat. Sb.} 
    \vol{81}
    \issue{3}
    \yr{1970}
    \pages{430--444}

\RBibitem{NemGeomProjInjFlatBanMod}
    \by{N.\,T.~Nemesh.} 
    \paper{The Geometry of Projective, Injective, and Flat Banach Modules}
    \jour{J. of Math. Sci.}
    \vol{237}
    \issue{3}
    \yr{2016}
    \pages{445--459}

\RBibitem{HelBanLocConvAlg}
    \by{A.\,Ya.~Helemskii} 
    \book{Banach and locally convex algebras.} 
    \publ{Oxford University Press}
    \yr{1993}

\RBibitem{DalBanAlgAutCont}
    \by{H.\,G.~Dales} 
    \book{Banach algebras and automatic continuity}
    \publ{Clarendon Press}
    \yr{2000}

\RBibitem{NemMetTopProjIdBanAlg}
    \by{N.\,T.~Nemesh}
    \paper{Metrically and topologically projective ideals of Banach algebras}
    \jour{Math. Notes.}
    \vol{99} 
    \issue{4}
    \pages{523--533}
    \yr{2016}

\RBibitem{WendLeftCentrzrs}
    \by{J.\,G.~Wendel} 
    \paper{Left centralizers and isomorphisms of group algebras}
    \jour{Pacific J. Math.} 
    \vol{2}
    \issue{3}
    \yr{1952}
    \pages{251--261}

\RBibitem{DefFloTensNorOpId}
    \by{A.~Defant, K.~Floret}
    \book{Tensor norms and operator ideals} 
    \vol{176}
    \publ{Elsevier}
    \yr{1992}

\RBibitem{RosOnRelDisjFamOfMeas}
    \by{H.~Rosenthal}
    \paper{On relatively disjoint families of measures, 
    with some applications to Banach space theory}
    \jour{Stud. Math.} 
    \vol{37}
    \issue{1}
    \yr{1970}
    \pages{13--36}

\RBibitem{WojBanSpForAnalysts}
    \by{P.~Wojtaszczyk} 
    \book{Banach spaces for analysts} 
    \publ{Cambridge University Press}
    \vol{25}
    \yr{1996}

\RBibitem{KalAlbTopicsBanSpTh}
    \by{F.~Albiac, N.\,J.~Kalton} 
    \book{Topics in Banach space theory}
    \vol{233}
    \publ{Springer}
    \yr{2006}

\RBibitem{LauMingComplSubspInLInfOfG}
    \by{A.\,T.-M.~Lau,  V.~Losert}
    \paper{Complementation of certain subspaces 
    of $L_\infty(G)$ of a locally compact group}
    \jour{Pacific J. Math}
    \vol{141}
    \issue{2}
    \yr{1990}
    \pages{295--310}

\RBibitem{LyubIsomEmdbFinDimLp}
    \by{Yu.\,I.~Lyubich, O.\,A.~Shatalova}
    \paper{Isometric embeddings of finite-dimensional 
        $\ell_p$-spaces over the quaternions}
    \jour{St.~Petersburg Math. J.}
    \vol{16}
    \issue{1}
    \yr{2005}
    \pages{9--24}

\RBibitem{JohnCohomolBanAlg}
    \by{B.~Johnson}
    \book{Cohomology in Banach Algebras} 
    \publ{Memoirs Series}
    \yr{1972}

\end{thebibliography}

\end{document}