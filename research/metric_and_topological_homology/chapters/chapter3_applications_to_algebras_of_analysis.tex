% chktex-file 35
% Chapter Template
% Main chapter title
% Change X to a consecutive number; for referencing this 
% chapter elsewhere, use~\ref{ChapterX}
\chapter{
    Applications to algebras of analysis
}\label{ChapterApplicationsToAlgebrasOfAnalysis} 

% Change X to a consecutive number; this is for the header on 
% each page - perhaps a shortened title
\lhead{Chapter 3. \emph{Applications to algebras of analysis}} 

Vaguely speaking there are three types of Banach modules depending on the type
of module action: modules with pointwise multiplication, modules with
composition of operators in the role of module action and modules with
convolution. We shall investigate main examples of these types. Following the
style of Dales and Polyakov from~\cite{DalPolHomolPropGrAlg} we shall
systematize all results on classical modules of analysis, but this time for
metric and topological theory. We shall consider modules over operator algebras,
sequence algebras, algebras of continuous functions and, finally, classical
modules of harmonic analysis.


%-------------------------------------------------------------------------------
%	Applications to operator algebras
%-------------------------------------------------------------------------------

\section{
    Applications to modules over \texorpdfstring{$C^*$}{C*}-algebras
}\label{SectionApplicationsToModulesOverCStarAlgebras}

%-------------------------------------------------------------------------------
%	Spatial modules
%-------------------------------------------------------------------------------

\subsection{
    Spatial modules
}\label{SubSectionSpatialModules}

We start from the simplest  examples of modules over operator algebras --- the
spatial modules. By Gelfand-Naimark's theorem (see e.g.
[\cite{HelBanLocConvAlg}, theorem 4.7.57]) for any $C^*$-algebra $A$ there
exists a Hilbert space $H$ and an isometric ${}^*$-homomorphism
$\varrho:A\to\mathcal{B}(H)$. For Hilbert spaces that admit such homomorphism we
may consider the left $A$-module $H_\varrho$ with module action defined as
$a\cdot x=\varrho(a)(x)$. Automatically we get the structure of right $A$-module
on $H^*$ which is by Riesz's theorem is isometrically isomorphic to $H^{cc}$.
This isomorphism allows one to define the right $A$-module structure on $H^{cc}$
by $\overline{x}\cdot a=\overline{\varrho(a^*)(x)}$. For a given $x_1,x_2\in H$
we define the rank one 
operator $x_1\bigcirc x_2:H\to H:x\mapsto \langle x, x_2\rangle x_1$. 

\begin{proposition}\label{SpatModOverCStarAlgProp} Let $A$ be a $C^*$-algebra
and $\varrho:A\to\mathcal{B}(H)$ be an isometric ${}^*$-homomorphism, such that
its image contains a subspace of rank one operators of the 
form $ \{x\bigcirc x_0:x\in H \}$ for some non zero $x_0\in H$. 
Then the left $A$-module $H_\varrho$ is metrically projective and flat, 
while the  right $A$-module $H_\varrho^{cc}$ is metrically injective.
\end{proposition}
\begin{proof} Without loss of generality we may assume that $\Vert x_0\Vert=1$.
Consider linear operators 
$\pi:A_+\to H_\varrho:a\oplus_1 z\mapsto \varrho(a)(x_0)+zx_0$ 
and $\sigma:H_\varrho\to A_+:x\mapsto \varrho^{-1}(x\bigcirc x_0)$. 
It is straightforward to check that $\pi$ and
$\sigma$ are contractive $A$-morphisms such that $\pi\sigma=1_{H_\varrho}$.
Therefore $H_\varrho$ is a retract of $A_+$ in $A-\mathbf{mod}_1$. From
propositions~\ref{UnitalAlgIsMetTopProj} and~\ref{RetrMetCTopProjIsMetCTopProj} 
it follows that $H_\varrho$ is metrically projective $A$-module. From
proposition~\ref{MetTopProjIsMetTopFlat} it follows that $H_\varrho$ is
metrically flat too. Since $H_\varrho^{cc}\isom{\mathbf{mod}_1-A}H_\varrho^*$,
proposition~\ref{DualMetTopProjIsMetrInj} gives that $H_\varrho^{cc}$ is
metrically injective.
\end{proof}

In what follows we shall use the following simple application of the above
result.

\begin{proposition}\label{FinDimNHModTopProjFlat} Let $H$ be a finite
dimensional Hilbert space. Then $\mathcal{N}(H)$ is topologically projective and
hence flat as $\mathcal{B}(H)$-module.
\end{proposition}
\begin{proof} From [\cite{HelBanLocConvAlg}, proposition 0.3.38] we know that
$\mathcal{N}(H)\isom{\mathbf{Ban}_1}H\projtens H^*$. Let
$\varrho=1_{\mathcal{B}(H)}$, then we can claim a little bit more:
$\mathcal{N}(H)\isom{\mathcal{B}(H)-\mathbf{mod}_1} H_\varrho\projtens H^*$.
Since $H^*$ is finite dimensional, then
$H^*\isom{\mathbf{Ban}}\ell_1(\mathbb{N}_n)$ for $n=\dim(H)$ and as the result
$\mathcal{N}(H)\isom{\mathcal{B}(H)-\mathbf{mod}}
H_\varrho\projtens\ell_1(\mathbb{N}_n)$. By
proposition~\ref{SpatModOverCStarAlgProp} the module $H_\varrho$ it
topologically projective, so from corollary~\ref{MetTopProjTensProdWithl1} we
get that $\mathcal{N}(H)$ is topologically projective as
$\mathcal{B}(H)$-module. The last claim of theorem follows from
proposition~\ref{MetTopProjIsMetTopFlat}.
\end{proof}

%-------------------------------------------------------------------------------
%	Projective ideals of C^*-algebras
%-------------------------------------------------------------------------------

\subsection{
    Projective ideals of \texorpdfstring{$C^*$}{C*}-algebras
}\label{SubSectionProjectiveIdealsOfCStarAlgebras}

The study of homologically trivial ideals of $C^*$-algebras we start from
projectivity, but before stating the main result we need a preparatory lemma.

\begin{lemma}\label{ContFuncCalcOnIdealOfCStarAlg} Let $I$ be a left ideal of a
unital $C^*$-algebra $A$. Assume $a\in I$ is a self-adjoint element and let $E$
be the real subspace of real valued functions in $C(\operatorname{sp}_A(a))$
vanishing at zero. Then there is an isometric homomorphism
$\operatorname{RCont}_a^0:E\to I$ well defined by
$\operatorname{RCont}_a^0(f)=a$, where
$f:\operatorname{sp}_A(a)\to\mathbb{C}:t\mapsto t$.
\end{lemma}
\begin{proof} By $\mathbb{R}_0[t]$ we denote the real linear subspace of $E$
consisting of polynomials vanishing at zero. Since $I$ is an ideal of $A$ and
and $p\in\mathbb{R}_0[t]$ has no constant term then $p(a)\in I$.  Hence we have
well defined $\mathbb{R}$-linear homomorphism of algebras
$\operatorname{RPol}_a^0:\mathbb{R}_0[t]\to I:p\mapsto p(a)$. By continuous
functional calculus for any polynomial $p$ 
holds $\Vert p(a)\Vert=\Vert p|_{\operatorname{sp}_A(a)}\Vert_\infty$, so
$\Vert\operatorname{RPol}_a^0(p)\Vert
=\Vert p|_{\operatorname{sp}_A(a)}\Vert_\infty$. 
Thus $\operatorname{RPol}_a^0$ is isometric. 
As $\mathbb{R}_0[t]$ is dense in $E$ and $I$ is complete, then
$\operatorname{RPol}_a^0$ has an isometric extension
$\operatorname{RCont}_a^0:E\to I$ which is also an $\mathbb{R}$-linear
homomorphism. 
\end{proof}

The following proof is inspired by ideas of D. P. Blecher and T. Kania. In
[\cite{BleKanFinGenCStarAlgHilbMod}, lemma 2.1] they proved that any
algebraically finitely generated left ideal of $C^*$-algebras is principal.  

\begin{theorem}\label{LeftIdealOfCStarAlgMetTopProjCharac} Let $I$ be a left
ideal of a $C^*$-algebra $A$. Then the following are equivalent:

\begin{enumerate}[label = (\roman*)]
    \item $I=Ap$ for some self-adjoint idempotent $p\in I$;

    \item $I$ is metrically projective $A$-module;

    \item $I$ is topologically projective $A$-module.
\end{enumerate}
\end{theorem}
\begin{proof} $(i)\implies (ii)$ Since $p$ is a self-adjoint idempotent, then
$\Vert p\Vert=1$, so by proposition~\ref{UnIdeallIsMetTopProj} paragraph $(i)$
the ideal $I$ is metrically projective as $A$-module.

$(ii)\implies (iii)$ See
proposition~\ref{MetProjIsTopProjAndTopProjIsRelProj}.

$(iii) \implies (i)$ Let ${(e_\nu)}_{\nu\in N}$ be a right contractive
approximate identity of ideal $I$ [\cite{HelBanLocConvAlg}, theorem 4.7.79].
Since $I$ admits a right approximate identity, then it is an essential left
$I$-module, and a fortiori an essential $A$-module. By
proposition~\ref{NonDegenMetTopProjCharac} we have a right inverse $A$-morphism
$\sigma:I\to A\projtens \ell_1(B_I)$ of $\pi_I$ in $A-\mathbf{mod}$. For each
$d\in B_I$ consider $A$-morphisms $p_d:A\projtens \ell_1(B_I)\to A:a\projtens
\delta_x\mapsto \delta_x(d)a$ and $\sigma_d=p_d\sigma$. Then
$\sigma(x)=\sum_{d\in B_I}\sigma_d(x)\projtens \delta_d$ for all $x\in I$. From
identification 
$A\projtens\ell_1(B_I)\isom{\mathbf{Ban}_1}\bigoplus_1 \{ A:d\in B_I \}$, 
for all $x\in I$ we 
have $\Vert\sigma(x)\Vert=\sum_{d\in B_I} \Vert\sigma_d(x)\Vert$. 
Since $\sigma$ is a right inverse morphism of $\pi_I$ we
have $x=\pi_I(\sigma(x))=\sum_{d\in B_I}\sigma_d(x)d$ for all $x\in I$. 

For all $x\in I$ we have 
$\Vert\sigma_d(x)\Vert
=\Vert\sigma_d(\lim_\nu xe_\nu)\Vert
=\lim_\nu\Vert x\sigma_d(e_\nu)\Vert 
\leq\Vert x\Vert\liminf_\nu\Vert\sigma_d(e_\nu)\Vert$, 
so $\Vert\sigma_d\Vert\leq \liminf_\nu\Vert\sigma_d(e_\nu)\Vert$. 
Then for all $S\in\mathcal{P}_0(B_I)$ holds
$$
\sum_{d\in S}\Vert \sigma_d\Vert
\leq \sum_{d\in S}\liminf_\nu\Vert \sigma_d(e_\nu)\Vert
\leq \liminf_\nu\sum_{d\in S}\Vert \sigma_d(e_\nu)\Vert
\leq \liminf_\nu\sum_{d\in B_I}\Vert \sigma_d(e_\nu) \Vert
$$
$$
=\liminf_{\nu}\Vert\sigma(e_\nu)\Vert
\leq \Vert\sigma\Vert\liminf_{\nu}\Vert e_\nu\Vert
\leq \Vert\sigma\Vert
$$
Since $S\in \mathcal{P}_0(B_I)$ is arbitrary, then the sum 
$\sum_{d\in B_I}\Vert\sigma_d\Vert$ is finite. As the consequence, the 
sum $\sum_{d\in B_I}\Vert\sigma_d\Vert^2$ is finite too. 

Now we regard $A$ as an ideal in its unitization $A_\#$, then $I$ is an ideal of
$A_\#$ too. Fix a natural number $m\in\mathbb{N}$ and a real number
$\epsilon>0$. Then there exists a set $S\in\mathcal{P}_0(B_I)$ such that
$\sum_{d\in B_I\setminus S}\Vert\sigma_d\Vert<\epsilon$. Denote its cardinality
by $N$. Consider positive element 
$b=\sum_{d\in B_I}\Vert\sigma_d\Vert^2 d^*d\in I$. Now we perform 
a ``power trick'' by considering different powers $b^{1/m}$ of positive element 
$b$, where $m\in\mathbb{N}$. By lemma~\ref{ContFuncCalcOnIdealOfCStarAlg} we 
have that $b^{1/m}\in I$, so $b^{1/m}=\sum_{d\in B_I}\sigma_d(b^{1/m})d$. 
By continuous functional calculus we have 
$\Vert b^{1/m}\Vert
=\sup_{t\in\operatorname{sp}_{A_\#}(b)} t^{1/m}\leq\Vert b\Vert^{1/m}$, 
then $\limsup_{m\to\infty}\Vert b^{1/m}\Vert\leq 1$. 
Therefore $\Vert b^{1/m}\Vert\leq 2$ for sufficiently big $m$. Denote
$\varsigma_d=\sigma_d(b^{1/m})$, $u=\sum_{d\in S}\varsigma_d d$ and
$v=\sum_{d\in B_I\setminus S}\varsigma_d d$, so 
$$
b^{2/m}={(b^{1/m})}^*b^{1/m}=u^*u+u^*v+v^*u+v^*v
$$
Clearly, $\varsigma_d^*\varsigma_d\leq \Vert \varsigma_d\Vert^2 e_{A_\#}\leq
\Vert \sigma_d\Vert^2\Vert b^{1/m}\Vert^2 e_{A_\#}\leq 4\Vert \sigma_d\Vert^2
e_{A_\#}$. For any $x,y\in A$ we have $x^*x+y^*y-(x^*y+y^*x)={(x-y)}^*(x-y)\geq
0$, therefore 
$$
d^*\varsigma_d^* \varsigma_c c+c^*\varsigma_c^* \varsigma_d d
\leq d^*\varsigma_d^*\varsigma_d d + c^*\varsigma_c^*\varsigma_c c
\leq 4\Vert \sigma_d\Vert^2 d^*d+4\Vert \sigma_c\Vert^2 c^*c
$$
for all $c,d\in B_I$. We sum up these inequalities over $c\in S$ and $d\in S$,
then 
$$
\begin{aligned}
\sum_{c\in S}\sum_{d\in S}c^*\varsigma_c^* \varsigma_d d
&=\frac{1}{2}\left(
    \sum_{c\in S}\sum_{d\in S}d^*\varsigma_d^* \varsigma_c c
    +
    \sum_{c\in S}\sum_{d\in S}c^*\varsigma_c^* \varsigma_d d
\right)\\
&\leq\frac{1}{2}\left(4 N\sum_{d\in S} \Vert \sigma_d\Vert^2 d^*d+
4 N\sum_{c\in S} \Vert \sigma_c\Vert^2 c^*c\right)\\
&=4 N\sum_{d\in S} \Vert \sigma_d\Vert^2 d^*d.
\end{aligned}
$$
Therefore
$$
u^*u
={\left(\sum_{c\in S}\varsigma_c c\right)}^*
\left(\sum_{d\in S}\varsigma_d d\right)
=\sum_{c\in S}\sum_{d\in S}c^*\varsigma_c^* \varsigma_d d
\leq 4N\sum_{d\in S} \Vert \sigma_d\Vert^2 d^*d
= 4N b
$$
Note that
$$
\Vert u\Vert
\leq \sum_{d\in S}\Vert\varsigma_d\Vert\Vert d\Vert
\leq \sum_{d\in S}2\Vert\sigma_d\Vert
\leq 2\Vert\sigma\Vert,
\qquad
\Vert v\Vert
\leq \sum_{d\in B_I\setminus S}\Vert\varsigma_d\Vert\Vert d\Vert
\leq \sum_{d\in B_I\setminus S}2\Vert\sigma_d\Vert
\leq 2\epsilon,
$$
so $\Vert u^*v+v^*u\Vert\leq 8\Vert\sigma\Vert\epsilon$ 
and $\Vert v^*v\Vert\leq 4\epsilon^2$. Since $u^*v+v^*u$ and $v^*v$ are self 
adjoint, then $u^*v+v^*u\leq 8\Vert\sigma\Vert\epsilon e_{A_\#}$ 
and $v^*v\leq 4\epsilon^2 e_{A_\#}$ Therefore for any $\epsilon>0$ and 
sufficiently big $m$ we have 
$$
b^{2/m}
=u^*u+u^*v+v^*u+v^*v
\leq 4Nb+\epsilon(8\Vert\sigma\Vert+4\epsilon)e_{A_\#}.
$$


In other words $g_m(b)\geq 0$ for continuous function
$g_m:\mathbb{R}_+\to\mathbb{R}:t\mapsto
4Nt+\epsilon(8\Vert\sigma\Vert+4\epsilon)-t^{2/m}$. Now choose $\epsilon>0$ such
that $M:=\epsilon(8\Vert\sigma\Vert+4\epsilon)<1$.  By spectral mapping theorem
[\cite{HelLectAndExOnFuncAn}, theorem 6.4.2] we get
$g_m(\operatorname{sp}_{A_\#}(b))
=\operatorname{sp}_{A_\#}(g_m(b))\subset\mathbb{R}_+$.
It is routine to check that $g_m$ has the only one extreme point
$t_{0,m}={(2Nm)}^{\frac{m}{2-m}}$ where the minimum of $g_m$ is attained. Since
$\lim_{m\to\infty} g_m(t_{0,m})=M-1<0$, $g_m(0)=M>0$ and $\lim_{t\to\infty}
g_m(t)=+\infty$, then for sufficiently big $m$ the function $g_m$ has exactly
two roots: $0<t_{1,m}<t_{0,m}$ and $t_{2,m} > t_{0,m}$. Therefore $g_m(t)\geq 0$
for $0\leq t\leq t_{1,m}$ or $t\geq t_{2,m}$. Hence for all sufficiently large
$m$ holds $\operatorname{sp}_{A_\#}(b)\subset T_{t_{1,m},t_{2,m}}$, 
where $T_{a, b}:=\{t\in\mathbb{R} : 0\leq t \leq a \vee t \geq b\}$. 
Since $\lim_{m\to\infty} t_{0,m}=0$ then $\lim_{m\to\infty} t_{1,m}=0$ too. 
Note that $g_m(1)=4N+M-1>0$, so for sufficiently big $m$ we also 
have $t_{2,m}\leq 1$. Consequently,
$\operatorname{sp}_{A_\#}(b)\subset T_{0,1}$.

Consider a continuous function $h:\mathbb{R}_+\to\mathbb{R}:t\mapsto\min(1, t)$,
uthen from lemma~\ref{ContFuncCalcOnIdealOfCStarAlg} we get an idempotent
$p=h(b)=\operatorname{RCont}_b^0(h)\in I$ such that $\Vert
p\Vert=\sup_{t\in\operatorname{sp}_{A_\#}(b)}|h(t)|\leq 1$. Therefore $p$ is a
self-adjoint idempotent. Since $h(t)t=th(t)=t$ for all $t\in
\operatorname{sp}_{A_\#}(b)$ we have $bp=pb=b$. The last equality implies
$$
0=(e_{A_\#}-p)b(e_{A_\#}-p)
=\sum_{d\in B_I}
    {(\Vert\sigma_d\Vert d(e_{A_\#}-p))}^*\Vert\sigma_d\Vert d(e_{A_\#}-p).
$$
Since the right hand side of this equality is non negative, then $d=dp$ for all
$d\in B_I$ with $\sigma_d\neq 0$. Finally, for any $x\in I$ we obtain
$xp=\sum_{d\in B_I}\sigma_d(x)dp=\sum_{d\in B_I}\sigma_d(x)d=x$, i.e. $I=Ap$,
for self-adjoint idempotent $p\in I$.
\end{proof}

It is worth to point out here that in relative theory there no such description
for relative projectivity of left ideals of $C^*$-algebras. Though for the case
of separable $C^*$-algebras (that is $C^*$-algebras that are separable as Banach
spaces) all left ideals are relatively projective. See
[\cite{LykProjOfBanAndCStarAlgsOfContFld}, section 6] for a nice overview of the
current state of the problem.

\begin{corollary}\label{BiIdealOfCStarAlgMetTopProjCharac} Let $I$ be a
two-sided ideal of a $C^*$-algebra $A$. Then the following are equivalent:

\begin{enumerate}[label = (\roman*)]
    \item $I$ is unital;

    \item $I$ is metrically projective $A$-module;

    \item $I$ is topologically projective $A$-module.
\end{enumerate}
\end{corollary}
\begin{proof} The ideal $I$ has a contractive approximate identity
[\cite{HelBanLocConvAlg}, theorem 4.7.79]. Therefore $I$ has a right identity
iff $I$ is unital. Now all equivalences follow from
theorem~\ref{LeftIdealOfCStarAlgMetTopProjCharac}. 
\end{proof}

\begin{corollary}\label{IdealofCommCStarAlgMetTopProjCharac} Let $S$ be a
locally compact Hausdorff space, and $I$ be an ideal of $C_0(S)$. Then the
following are equivalent:

\begin{enumerate}[label = (\roman*)]
    \item $\operatorname{Spec}(I)$ is compact;

    \item $I$ is metrically projective $C_0(S)$-module;

    \item $I$ is topologically projective $C_0(S)$-module.
\end{enumerate}
\end{corollary}
\begin{proof} By Gelfand-Naimark's theorem
$I\isom{\mathbf{Ban}_1}C_0(\operatorname{Spec}(I))$, therefore $I$ is
semisimple. Now by Shilov's idempotent theorem $I$ is unital iff
$\operatorname{Spec}(I)$ is compact. It remains to apply
corollary~\ref{BiIdealOfCStarAlgMetTopProjCharac}. 
\end{proof}

It is worth to mention that the class of relatively projective ideals of
$C_0(S)$ is much larger. In fact a closed ideal $I$ of $C_0(S)$ is relatively
projective iff $\operatorname{Spec}(I)$ is paracompact
[\cite{HelHomolBanTopAlg}, chapter IV,\S\S 2--3].

%-------------------------------------------------------------------------------
%	Injective ideals of C^*-algebras
%-------------------------------------------------------------------------------

\subsection{
    Injective ideals of \texorpdfstring{$C^*$}{C*}-algebras
}\label{SubSectionInjectiveIdealsOfCStarAlgebras}

Now we proceed to injectivity of two-sided ideals of $C^*$-algebras.
Unfortunately we don't have a complete characterization at hand, but some
necessary conditions and several examples. The following proposition shows that
we may restrict investigation of injective ideals to the case of $C^*$-algebras
that $\langle$~metrically / topologically~$\rangle$ injective over themselves as
right modules.

\begin{proposition}\label{MetTopInjOvrAlgIffOvrItslf} Let $I$ be a two-sided
ideal of a $C^*$-algebra $A$, then $I$ is $\langle$~metrically /
topologically~$\rangle$ injective as $A$-module iff $I$ is $\langle$~metrically
/ topologically~$\rangle$ injective as $I$-module.
\end{proposition}
\begin{proof} Note that any two-sided ideal $I$ of a $C^*$-algebra $A$ is again
a $C^*$-algebra with contractive approximate identity [\cite{HelBanLocConvAlg},
theorem 4.7.79]. Therefore $I$ is faithful as right $I$-module. Now
proposition~\ref{ReduceInjIdToInjAlg} gives the desired equivalence.
\end{proof}

We shall say a few words on so called $AW^*$-algebras, since they are key
players here. In attempts to find a purely algebraic definition of
$W^*$-algebras Kaplanski introduced this class of $C^*$-algebras
in~\cite{KaplProjInBanAlg}. A $C^*$-algebra $A$ is called an $AW^*$-algebra if
for any subset $S\subset A$ the right algebraic annihilator
$S^{\perp A}$ is of the form $pA$ for
some self-adjoint idempotent $p\in A$. This class contains all $W^*$-algebras,
but strictly larger. Note that for the case of commutative $C^*$-algebras the
property of being an $AW^*$-algebra is equivalent to  $\operatorname{Spec}(A)$
being a Stonean space [\cite{BerbBaerStarRings}, theorem 1.7.1]. The main
reference to $AW^*$-algebras and more general Baer ${}^*$-rings
is~\cite{BerbBaerStarRings}. 

The following proposition is a combination of results by M. Hamana and M.
Takesaki.

\begin{proposition}[Hamana, Takesaki]\label{MetInjCStarAlgCharac} Let $A$ be a
$C^*$-algebra, then it is metrically injective right $A$-module iff it is a
commutative $AW^*$-algebra, that is $\operatorname{Spec}(A)$ is a Stonean space.
\end{proposition}
\begin{proof} 

If $A$ is metrically injective as $A$-module, then it has norm one left identity
by proposition~\ref{MetTopInjOfId}. But $A$ also has a contractive approximate
identity  [\cite{HelBanLocConvAlg}, theorem 4.7.79], therefore $A$ is unital.
Now by result of Hamana  [\cite{HamInjEnvBanMod}, proposition 2] the
$C^*$-algebra $A$ is a commutative $AW^*$-algebra. Hamana's argument was for
left modules, but one can easily modify his proof to get the result for right
modules.

The converse proved by Takesaki in [\cite{TakHanBanThAndJordDecomOfModMap},
theorem 2]. Although only two-sided Banach modules were treated there, the
reasoning is essentially the same for right modules.
\end{proof}

It remains to consider topological injectivity. As the following proposition
shows topologically injective $C^*$-algebras are not so far from commutative
ones. This proposition exploits the l.u.st.\ property, for its definition see
section~\ref{
    SubSectionHomologicallyTrivialModulesOverBanachAlgebrasWithSpecificGeometry
}.

\begin{proposition}\label{TopInjIdHaveLUST} Let $A$ be a $C^*$-algebra which is
topologically injective as an $A$-module. Then $A$ has the l.u.st.\ property and
as the consequence it can't contain  $\mathcal{B}(\ell_2(\mathbb{N}_n))$ as
${}^*$-subalgebra for arbitrary $n\in\mathbb{N}$.
\end{proposition}
\begin{proof} By Gelfand-Naimark's theorem [\cite{HelBanLocConvAlg}, theorem
4.7.57] there exists a Hilbert space $H$ and an isometric ${}^*$-homomorphism
$\varrho:A\to\mathcal{B}(H)$. Denote $\Lambda:=B_{H_\varrho^{cc}}$. It is easy
to check that 
$$
\rho
:A
    \to
\bigoplus\nolimits_\infty \{H_\varrho^{cc}:\overline{x}\in \Lambda \}
:a
    \mapsto 
\bigoplus\nolimits_\infty \{\overline{x}\cdot a:\overline{x}\in \Lambda \}
$$
is an isometric $A$-morphism of right $A$-modules. Since $A$ is topologically
injective $A$-module, then $\rho$ has a left inverse $A$-morphism $\tau$.
Therefore $A$ is complemented in $E:=\bigoplus_\infty
\{H_\varrho^{cc}:\overline{x}\in \Lambda \}$ via projection $\rho\tau$. Note
that $H_{\varrho}^{cc}$ is a Banach lattice as any Hilbert space, then so does
$E$. As any Banach lattice $E$ has the l.u.st.\ property [\cite{DiestAbsSumOps},
theorem 17.1], then so does $A$, because the l.u.st.\ property is inherited by
complemented subspaces.

Assume $A$ contains $\mathcal{B}(\ell_2(\mathbb{N}_n))$ as ${}^*$-subalgebra for
arbitrarily large $n\in\mathbb{N}$. In fact this copy of
$\mathcal{B}(\ell_2(\mathbb{N}_n))$ is $1$-complemented in $A$
[\cite{LauLoyWillisAmnblOfBanAndCStarAlgsOfLCG}, lemma 2.1]. Therefore we have
an inequality for local unconditional constants
$\kappa_u(\mathcal{B}(\ell_2(\mathbb{N}_n)))\leq \kappa_u(A)$. By theorem 5.1
in~\cite{GorLewAbsSmOpAndLocUncondStrct} we know that $\lim_n
\kappa_u(\mathcal{B}(\ell_2(\mathbb{N}_n)))=+\infty$, so $\kappa_u(A)=+\infty$.
This contradicts the l.u.st.\ property of $A$. Hence $A$ can't contain
$\mathcal{B}(\ell_2(\mathbb{N}_n))$ as ${}^*$-subalgebra for arbitrary
$n\in\mathbb{N}$.
\end{proof}

As the proposition~\ref{TopInjIdHaveLUST} shows $C^*$-algebras that are
topologically injective over themselves can't contain
$\mathcal{B}(\ell_2(\mathbb{N}_n))$ as ${}^*$-subalgebra for arbitrary
$n\in\mathbb{N}$. Such $C^*$-algebras are called subhomogeneous, and in fact
[\cite{BlackadarOpAlg}, proposition IV.1.4.3] they can be treated as closed
${}^*$-subalgebras of $M_n(C(K))$ for some compact Hausdorff space $K$ and some
natural number $n$. For more on subhomogeneous $C^*$-algebras see
[\cite{BlackadarOpAlg}, section IV.1.4]. 

We shall give two important examples of non commutative $C^*$-algebras that are
topologically injective over themselves.

\begin{proposition}\label{FinDimBHModTopInj} Let $H$ be a finite dimensional
Hilbert space. Then $\mathcal{B}(H)$ is topologically injective as
$\mathcal{B}(H)$-module. 
\end{proposition}
\begin{proof} Note that
$\mathcal{B}(H)\isom{\mathbf{mod}_1-\mathcal{B}(H)}{\mathcal{N}(H)}^*$, and the
the result immediately follows from propositions~\ref{FinDimNHModTopProjFlat}
and~\ref{DualMetTopProjIsMetrInj}.
\end{proof}

\begin{proposition}\label{CKMatrixModTopInj} Let $K$ be a Stonean space and
$n\in\mathbb{N}$, then $M_n(C(K))$ is topologically injective
$M_n(C(K))$-module.
\end{proposition}
\begin{proof} For a fixed $s\in K$ by $\mathbb{C}_s$ we denote the right
$C(K)$-module $\mathbb{C}$ with outer action defined by $z\cdot a=a(s)z$ for all
$a\in C(K)$ and $z\in\mathbb{C}$. By $M_n(\mathbb{C}_s)$ we denote the right
Banach $M_n(C(K))$-module $M_n(\mathbb{C})$ with outer action defined by
${(x\cdot a)}_{i,j}=\sum_{k=1}^n x_{i,k}a_{k,j}(s)$ for $a\in M_n(C(K))$ and 
$x\in M_n(\mathbb{C}_s)$. The $C^*$-algebra $M_n(C(K))$ is nuclear
[\cite{BroOzaCStarAlgFinDimApprox}, corollary 2.4.4], then by
[\cite{HaaNucCStarAlgAmen}, theorem 3.1] this algebra is relatively amenable and
even $1$-relatively amenable [\cite{RundeAmenConstFour}, example 2]. Since
$M_n(\mathbb{C}_s)$ is finite dimensional, it is an $\mathscr{L}_{1, C}^g$-space
for some constant $C\geq 1$ independent of $s$. Thus, by
proposition~\ref{MetTopEssL1FlatModAoverAmenBanAlg} the $M_n(C(K))$-module
${M_n(\mathbb{C}_s)}^*$ is $C$-topologically flat. Since the latter module is
essential, by proposition~\ref{MetCTopFlatCharac} the right $M_n(C(K))$-module
${M_n(\mathbb{C}_s)}^{**}$ is $C$-topologically injective. Note that
${M_n(\mathbb{C}_s)}^{**}$ is isometrically isomorphic to $M_n(\mathbb{C}_s)$ as
right $M_n(C(K))$-module, so from proposition~\ref{MetTopInjModProd} we get that
$\bigoplus_\infty \{M_n(\mathbb{C}_s):s\in K \}$ is topologically injective as
$M_n(C(K))$-module.

Note that by proposition~\ref{MetInjCStarAlgCharac} the $C(K)$-module $C(K)$ is
metrically injective, therefore an isometric $C(K)$-morphism
$\widetilde{\rho}
:C(K)\to\bigoplus_\infty \{ \mathbb{C}_s:s\in K \}
:x\mapsto \bigoplus_\infty \{x(s):s\in K \}$ 
admits a left inverse contractive $C(K)$-morphism 
$\widetilde{\tau}:\bigoplus_\infty \{ \mathbb{C}_s:s\in K \} \to C(K)$. 
It is routine to check now that linear operators
$$
\rho
:M_n(C(K))\to\bigoplus\nolimits_\infty \{M_n(\mathbb{C}_s):s\in K \}
:x\mapsto \bigoplus\nolimits_\infty \{
    {(x_{i,j}(s))}_{i,j\in\mathbb{N}_n}:s\in K
 \}
$$
$$
\tau
:\bigoplus\nolimits_\infty \{M_n(\mathbb{C}_s):s\in K \}\to M_n(C(K))
:y\mapsto {\left(
    \widetilde{\tau}\left(\bigoplus\nolimits_\infty \{y_{s,i,j}:s\in K \}\right)
\right)}_{i,j\in\mathbb{N}_n}
$$
are bounded $M_n(C(K))$-morphisms such that $\tau \rho=1_{M_n(C(K))}$. That is
$M_n(C(K))$ is a retract of topologically injective $M_n(C(K))$-module
$\bigoplus_\infty \{M_n(\mathbb{C}_s):s\in K \}$ in $\mathbf{mod}_1-M_n(C(K))$.
Finally, from proposition~\ref{RetrMetCTopInjIsMetCTopInj} we conclude that
$M_n(C(K))$ is topologically injective $M_n(C(K))$-module.
\end{proof}

\begin{theorem}\label{TopInjAWStarAlgCharac} Let $A$ be a $C^*$-algebra. Then
the following are equivalent:

\begin{enumerate}[label = (\roman*)]
    \item $A$ is an $AW^*$-algebra which is topologically injective 
    as $A$-module;

    \item $A
    =\bigoplus_\infty \{M_{n_\lambda}(C(K_\lambda)):\lambda\in\Lambda \}$
    for some finite families of natural 
    numbers ${(n_\lambda)}_{\lambda\in\Lambda}$ and Stonean 
    spaces ${(K_\lambda)}_{\lambda\in\Lambda}$.
\end{enumerate}
\end{theorem}
\begin{proof}$(i)\implies (ii)$ From proposition 6.6
in~\cite{SmithDecompPropCStarAlg} we know that an $AW^*$-algebra is either
isomorphic as $C^*$-algebra 
to  $\bigoplus_\infty \{M_{n_\lambda}(C(K_\lambda)):\lambda\in\Lambda \}$ 
for some finite families of natural numbers ${(n_\lambda)}_{\lambda\in\Lambda}$ 
and Stonean spaces ${(K_\lambda)}_{\lambda\in\Lambda}$ or contains 
a ${}^*$-subalgebra 
$\bigoplus_\infty \{ \mathcal{B}(\ell_2(\mathbb{N}_n)):n\in\mathbb{N} \}$. The
second option is canceled out by proposition~\ref{TopInjIdHaveLUST}.

$(ii)\implies (i)$ For each $\lambda\in\Lambda$ the algebra
$M_{n_\lambda}(C(K_\lambda))$ is unital because $K_\lambda$ is compact.
Therefore $M_{n_\lambda}(C(K_\lambda))$ is faithful as
$M_{n_\lambda}(C(K_\lambda))$-module. It is also topologically injective as
$M_{n_\lambda}(C(K_\lambda))$-module by proposition~\ref{CKMatrixModTopInj}.
Now the topological injectivity of $A$ as $A$-module follows from paragraph
$(ii)$ of proposition~\ref{MetTopProjInjFlatUnderSumOfAlg} with $p=\infty$ and
$X_\lambda=A_\lambda=M_{n_\lambda}(C(K_\lambda))$ for all $\lambda\in\Lambda$. 

For all $\lambda\in\Lambda$ the algebra $C(K_\lambda)$ is an $AW^*$-algebra,
because $K_\lambda$ is a Stonean space [\cite{BerbBaerStarRings}, theorem
1.7.1]. Therefore $M_{n_\lambda}(C(K_\lambda))$ is an $AW^*$-algebra too
[\cite{BerbBaerStarRings}, corollary 9.62.1]. Finally $A$ is an $AW^*$-algebra
as $\bigoplus_\infty$-sum of such algebras [\cite{BerbBaerStarRings},
proposition 1.10.1].
\end{proof}

It is desirable to prove that any topologically injective over itself
$C^*$-algebra is an $AW^*$-algebra, but it seems to be a challenge even in the
commutative case.

%-------------------------------------------------------------------------------
%	Flat ideals of C^*-algebras
%-------------------------------------------------------------------------------

\subsection{
    Flat ideals of \texorpdfstring{$C^*$}{C*}-algebras
}\label{SubSectionFlatIdealsOfCStarAlgebras}

By considering flatness we finalize this lengthy investigations of ideals of
$C^*$-algebras.

\begin{proposition}\label{IdealofCstarAlgisMetTopFlat} Let $I$ be a left ideal
of a $C^*$-algebra $A$. Then $I$ is metrically and topologically flat as
$A$-module.
\end{proposition}
\begin{proof} From [\cite{HelBanLocConvAlg}, proposition 4.7.78] it follows that
$I$ has a right contractive identity. It remains to apply
proposition~\ref{MetTopFlatIdealsInUnitalAlg}.
\end{proof}

\begin{proposition}\label{CStarAlgIsL1IfFinDim} Let $A$ be a $C^*$-algebra, then
$A$ is an $\langle$~$L_1$-space / $\mathscr{L}_1^g$-space~$\rangle$ iff
$\langle$~$\operatorname{dim}(A)\leq 1$ / $A$ is finite dimensional~$\rangle$.
\end{proposition}
\begin{proof} Assume $A$ is an $\mathscr{L}_1^g$-space, then $A^{**}$ is
complemented in some $L_1$-space [\cite{DefFloTensNorOpId}, corollary
23.2.1(2)]. Since $A$ isometrically embeds in its second dual via $\iota_{A}$ we
may regard $A$ as closed subspace of some $L_1$-space. The latter space is
weakly sequentially complete [\cite{WojBanSpForAnalysts}, corollary III.C.14].
The property of being weakly sequentially complete is preserved by closed
subspaces, therefore $A$ is weakly sequentially complete too. By proposition 2
in~\cite{SakWeakCompOpOnOpAlg} every weakly sequentially complete $C^*$-algebra
is finite dimensional, hence $A$ is finite dimensional. Conversely, if $A$ is
finite dimensional it is an $\mathscr{L}_1^g$-space as any finite dimensional
Banach space.

Assume $A$ is an $L_1$-space and, a fortiori, an $\mathscr{L}_1^g$-space. As was
noted above $A$ is a finite dimensional, so
$A\isom{\mathbf{Ban}_1}\ell_1(\mathbb{N}_n)$ for $n=\operatorname{dim}(A)$. On
the other hand, $A$ is a finite dimensional $C^*$-algebra, so it is
isometrically isomorphic to $\bigoplus_\infty \{
\mathcal{B}(\ell_2(\mathbb{N}_{n_k})):k\in\mathbb{N}_m \}$ for some natural
numbers ${(n_k)}_{k\in\mathbb{N}_m}$ 
[\cite{DavCSatrAlgByExmpl}, theorem III.1.1].
Assume $\operatorname{dim}(A)>1$, then $A$ contains an isometric copy of
$\ell_\infty(\mathbb{N}_2)$. Therefore we have an isometric embedding of
$\ell_\infty(\mathbb{N}_2)$ into $\ell_1(\mathbb{N}_n)$. This is impossible by
theorem 1 from~\cite{LyubIsomEmdbFinDimLp}. 
Therefore $\operatorname{dim}(A)\leq 1$. 
\end{proof}

\begin{proposition}\label{CStarAlgIsTopFlatOverItsIdeal} Let $I$ be a proper
two-sided ideal of a  $C^*$-algebra $A$. Then the following are equivalent:

\begin{enumerate}[label = (\roman*)]
    \item $A$ is $\langle$~metrically / topologically~$\rangle$ flat $I$-module;

    \item $\langle$~$\operatorname{dim}(A)=1$, $I= \{0 \}$ / $A/I$ is finite
    dimensional~$\rangle$.
\end{enumerate}
\end{proposition}
\begin{proof} We may regard $I$ as an  ideal of unitazation $A_\#$ of $A$. Since
$I$ is a two-sided ideal, then it has a contractive approximate identity
${(e_\nu)}_{\nu\in N}$ such that $0\leq e_\nu\leq e_{A_\#}$
[\cite{HelBanLocConvAlg}, proposition 4.7.79]. As a 
corollary $\sup_{\nu\in N}\Vert e_{A_\#}-e_\nu\Vert\leq 1$. Since $I$ has an 
approximate identity we also have $A_{ess}:=\operatorname{cl}_A(IA)=I$. 
Since $I$ is a two sided ideal then $A/I$ is a $C^*$-algebra 
[\cite{HelBanLocConvAlg}, theorem 4.7.81].

Assume, $A$ is a metrically flat $I$-module. Since 
$\sup_{\nu\in N}\Vert e_{A_\#}-e_\nu\Vert\leq 1$, then paragraph $(ii)$ of
proposition~\ref{DualBanModDecomp} tells us that ${(A/A_{ess})}^*={(A/I)}^*$ is 
a retract of $A^*$ in $\mathbf{mod}_1-I$. Now from
propositions~\ref{MetCTopFlatCharac} and~\ref{RetrMetCTopInjIsMetCTopInj} it
follows that $A/I$ is metrically flat $I$-module. Since this is an annihilator
module, then from proposition~\ref{MetTopFlatAnnihModCharac} it follows that 
$I= \{0 \}$ and $A/I$ is an $L_1$-space. Now from
proposition~\ref{CStarAlgIsL1IfFinDim} we get that 
$\operatorname{dim}(A/I)\leq 1$. Since $A$ contains a proper 
ideal $I= \{0 \}$, then $\operatorname{dim}(A)=1$. Conversely, 
if $I= \{0 \}$ and $\operatorname{dim}(A)=1$, then we have an 
annihilator $I$-module $A$ which is isometrically isomorphic 
to $\ell_1(\mathbb{N}_1)$. By proposition~\ref{MetTopFlatAnnihModCharac} 
it is metrically flat. 

By proposition~\ref{TopFlatModCharac} the $I$-module $A$ is topologically flat
iff $A_{ess}=I$ and $A/A_{ess}=A/I$ are topologically flat $I$-modules. By
proposition~\ref{IdealofCstarAlgisMetTopFlat} the ideal $I$ is topologically
flat $I$-module. By proposition~\ref{MetTopFlatAnnihModCharac} the annihilator
$I$-module $A/I$ is topologically flat iff it is an $\mathscr{L}_1^g$-space. By
proposition~\ref{CStarAlgIsL1IfFinDim} this is equivalent to $A/I$ being finite
dimensional.
\end{proof}

%-------------------------------------------------------------------------------
%	\mathcal{K}(H)- and \mathcal{B}(H)-modules
%-------------------------------------------------------------------------------

\subsection{
    \texorpdfstring{$\mathcal{K}(H)$}{K (H)}- and
    \texorpdfstring{$\mathcal{B}(H)$}{B (H)}-modules
}\label{SubSectionKHAndBHModules}

In this section we apply general results on ideals obtained above to classical
modules over $C^*$-algebras.  For a given Hilbert space $H$ we consider
$\mathcal{B}(H)$, $\mathcal{K}(H)$ and $\mathcal{N}(H)$ as left and right Banach
modules over $\mathcal{B}(H)$ and $\mathcal{K}(H)$. For all modules the module
action is just the composition of operators. The Schatten-von Neumann
isomorphisms $\mathcal{N}(H)\isom{\mathbf{Ban}_1}{\mathcal{K}(H)}^*$,
$\mathcal{B}(H)\isom{\mathbf{Ban}_1}{\mathcal{N}(H)}^*$ (see
[\cite{TakThOpAlgVol1}, theorems II.1.6, II.1.8]) will be of use here. They are
in fact isomorphisms of left and right $\mathcal{B}(H)$-modules and a fortiori
of $\mathcal{K}(H)$-modules.

\begin{proposition}\label{KHAndBHModBH} Let $H$ be a Hilbert space. Then

\begin{enumerate}[label = (\roman*)]
    \item $\mathcal{B}(H)$ is metrically and topologically projective and 
    flat as $\mathcal{B}(H)$-module;

    \item $\mathcal{B}(H)$ is metrically or topologically projective or flat as
    $\mathcal{K}(H)$-module iff $H$ is finite dimensional;

    \item $\mathcal{B}(H)$ is topologically injective as $\mathcal{B}(H)$- or
    $\mathcal{K}(H)$-module iff $H$ is finite dimensional;

    \item $\mathcal{B}(H)$ is metrically injective as $\mathcal{B}(H)$- or
    $\mathcal{K}(H)$-module iff $\dim(H)\leq 1$.
\end{enumerate}
\end{proposition}
\begin{proof} $(i)$ Since $\mathcal{B}(H)$ is a unital algebra it is metrically
and topologically projective as $\mathcal{B}(H)$-module by
proposition~\ref{UnitalAlgIsMetTopProj}. Both results regarding flatness follow
from proposition~\ref{MetTopProjIsMetTopFlat}.

$(ii)$ For infinite dimensional $H$ the Banach space
$\mathcal{B}(H)/\mathcal{K}(H)$ is of infinite dimension, so by
proposition~\ref{CStarAlgIsTopFlatOverItsIdeal} the module $\mathcal{B}(H)$
neither topologically nor metrically flat as $\mathcal{K}(H)$-module. Both
claims regarding projectivity follow from
proposition~\ref{MetTopProjIsMetTopFlat}. If $H$ is finite dimensional, then
$\mathcal{K}(H)=\mathcal{B}(H)$, so the result follows from paragraph $(i)$.

$(iii)$ If $H$ is infinite dimensional, then $\mathcal{B}(H)$ contains
$\mathcal{B}(\ell_2(\mathbb{N}_n))$ as ${}^*$-subalgebra for all
$n\in\mathbb{N}$. By proposition~\ref{TopInjIdHaveLUST} we get that
$\mathcal{B}(H)$ is not topologically injective as $\mathcal{B}(H)$-module. The
rest follows from paragraph $(i)$ of
proposition~\ref{MetTopInjUnderChangeOfAlg}. If $H$ is finite dimensional, then
$\mathcal{K}(H)=\mathcal{B}(H)$, so the result follows from
proposition~\ref{FinDimBHModTopInj}.

$(iv)$ If $\dim(H)>1$, then $C^*$-algebra $\mathcal{B}(H)$ is not commutative. 
By proposition~\ref{MetInjCStarAlgCharac} we get that it is not metrically
injective as $\mathcal{B}(H)$-module. Now from paragraph $(i)$
of~\ref{MetTopInjUnderChangeOfAlg} we get that $\mathcal{B}(H)$ is not
metrically injective as $\mathcal{K}(H)$-module. If $\dim(H)\leq 1$ both claims
obviously follow from~\ref{MetInjCStarAlgCharac}.
\end{proof}

\begin{proposition}\label{KHAndBHModKH} Let $H$ be a Hilbert space. Then 

\begin{enumerate}[label = (\roman*)]
    \item $\mathcal{K}(H)$ is metrically and topologically flat 
    as $\mathcal{B}(H)$- or $\mathcal{K}(H)$-module;

    \item $\mathcal{K}(H)$ is metrically or topologically projective as
    $\mathcal{B}(H)$- or $\mathcal{K}(H)$-module iff $H$ is finite dimensional;

    \item $\mathcal{K}(H)$ is topologically injective as $\mathcal{B}(H)$- 
    or $\mathcal{K}(H)$-module iff $H$ is finite dimensional;

    \item $\mathcal{K}(H)$ is metrically injective as $\mathcal{B}(H)$- 
    or $\mathcal{K}(H)$-module iff $\dim(H)\leq 1$.
\end{enumerate}
\end{proposition}
\begin{proof} Let $A$ be either $\mathcal{B}(H)$ or $\mathcal{K}(H)$. Note that
$\mathcal{K}(H)$ is a two-sided ideal of $A$. 

$(i)$ Recall that $\mathcal{K}(H)$ has a contractive approximate identity
consisting of orthogonal projections onto all finite-dimensional subspaces of
$H$. Since $\mathcal{K}(H)$ is a two-sided ideal of $A$, then the result follows
from proposition~\ref{IdealofCstarAlgisMetTopFlat}.

$(ii)$, $(iii)$, $(iv)$ If $H$ is infinite dimensional, then $\mathcal{K}(H)$ is
not unital as Banach algebra. From
corollary~\ref{BiIdealOfCStarAlgMetTopProjCharac} and
proposition~\ref{MetTopInjOfId} the $A$-module $\mathcal{K}(H)$ is neither
metrically nor topologically projective or injective. If $H$ is finite
dimensional, then $\mathcal{K}(H)=\mathcal{B}(H)$, so both results follow from
paragraphs $(i)$, $(iii)$ and $(iv)$ of proposition~\ref{KHAndBHModBH}.
\end{proof}

\begin{proposition}\label{KHAndBHModNH} Let $H$ be a Hilbert space. Then

\begin{enumerate}[label = (\roman*)]
    \item $\mathcal{N}(H)$ is metrically and topologically injective as
    $\mathcal{B}(H)$- or $\mathcal{K}(H)$-module;

    \item $\mathcal{N}(H)$ is topologically projective or flat 
    as $\mathcal{B}(H)$- or $\mathcal{K}(H)$-module 
    iff $H$ is finite dimensional;

    \item $\mathcal{N}(H)$ is metrically projective or flat 
    as $\mathcal{B}(H)$- or $\mathcal{K}(H)$-module iff $\dim(H)\leq 1$.
\end{enumerate}
\end{proposition}
\begin{proof} Let $A$ be either $\mathcal{B}(H)$ or $\mathcal{K}(H)$.

$(i)$ Note that $\mathcal{N}(H)\isom{\mathbf{mod}_1-A}{\mathcal{K}(H)}^*$, 
so the result follows from proposition~\ref{MetCTopFlatCharac} and 
paragraph $(i)$ of proposition~\ref{KHAndBHModKH}.

$(ii)$ Assume $H$ is infinite dimensional. Note that
$\mathcal{B}(H)\isom{\mathbf{mod}_1-A}{\mathcal{N}(H)}^*$, so from
proposition~\ref{DualMetTopProjIsMetrInj} and paragraph $(iii)$ of
proposition~\ref{KHAndBHModBH} we get that $\mathcal{N}(H)$ is not topologically
projective as $A$-module. Both results regarding flatness follow from
proposition~\ref{MetTopProjIsMetTopFlat}. If $H$ is finite dimensional we use
proposition~\ref{FinDimNHModTopProjFlat}.

$(iii)$ Assume $\dim(H)>1$, then by paragraph $(iv)$ of
proposition~\ref{KHAndBHModBH} the $A$-module $\mathcal{B}(H)$ is not metrically
injective. Since $\mathcal{B}(H)\isom{\mathbf{mod}_1-A}{\mathcal{N}(H)}^*$, then
from proposition~\ref{MetCTopFlatCharac} we get that $\mathcal{N}(H)$ is not
metrically flat as $A$-module. By proposition~\ref{MetTopProjIsMetTopFlat}, it
is not metrically projective as $A$-module. If $\dim(H)\leq 1$, then
$\mathcal{N}(H)=\mathcal{K}(H)=\mathcal{B}(H)$, so both results follow from
paragraph $(i)$ of proposition~\ref{KHAndBHModBH}.
\end{proof}

\begin{proposition}\label{KHAndBHModsRelTh} Let $H$ be a Hilbert space. Then
\begin{enumerate}[label = (\roman*)]
    \item as $\mathcal{K}(H)$-modules $\mathcal{N}(H)$ is relatively projective
    injective and flat, $\mathcal{K}(H)$ is relatively projective and flat, but
    relatively injective only for finite dimensional $H$, $\mathcal{B}(H)$ is
    relatively injective and flat, but relatively projective only for finite
    dimensional $H$;

    \item as $\mathcal{B}(H)$-modules $\mathcal{N}(H)$ is relatively projective
    injective and flat, $\mathcal{K}(H)$ is relatively projective and flat,
    $\mathcal{B}(H)$ is relatively projective, injective and flat.
\end{enumerate}
\end{proposition}
\begin{proof} $(i)$ Note that $H$ is relatively projective as
$\mathcal{K}(H)$-module [\cite{HelBanLocConvAlg}, theorem 7.1.27], so from
proposition 7.1.13 in~\cite{HelBanLocConvAlg} we get that
$\mathcal{N}(H)\isom{\mathcal{K}(H)-\mathbf{mod}_1}H\projtens H^*$ is also
relatively projective as $\mathcal{K}(H)$-module. By theorem IV.2.16
in~\cite{HelHomolBanTopAlg} the $\mathcal{K}(H)$-module $\mathcal{K}(H)$ is
relatively projective. A fortiori $\mathcal{N}(H)$ and $\mathcal{K}(H)$ are
relatively flat $\mathcal{K}(H)$-modules 
[\cite{HelBanLocConvAlg}, proposition 7.1.40], 
so $\mathcal{N}(H)\isom{\mathbf{mod}_1-\mathcal{K}(H)}{\mathcal{K}(H)}^*$
and  $\mathcal{B}(H)\isom{\mathbf{mod}_1-\mathcal{K}(H)}{\mathcal{N}(H)}^*$ are
relatively injective $\mathcal{K}(H)$-modules. From
[\cite{RamsHomPropSemgroupAlg}, proposition 2.2.8  (i)] we know that a Banach
algebra relatively injective over itself as a right module, necessarily has a
left identity. Therefore $\mathcal{K}(H)$ is not relatively injective
$\mathcal{K}(H)$-module for infinite dimensional $H$. If $H$ is finite
dimensional, then $\mathcal{K}(H)$-module $\mathcal{K}(H)$ is relatively
injective because $\mathcal{K}(H)=\mathcal{B}(H)$ and $\mathcal{B}(H)$ is
relatively injective $\mathcal{K}(H)$-module as was shown above. By corollary
5.5.64 from~\cite{DalBanAlgAutCont} the algebra $\mathcal{K}(H)$ is relatively
amenable, so all its left modules are relatively flat [\cite{HelBanLocConvAlg},
theorem 7.1.60]. In particular $\mathcal{B}(H)$ is relatively flat
$\mathcal{K}(H)$-module. From [\cite{HelHomolBanTopAlg}, exercise V.2.20] we
know that $\mathcal{B}(H)$ is not relatively projective as
$\mathcal{K}(H)$-module when $H$ is infinite dimensional. If $H$ is finite
dimensional then $\mathcal{B}(H)$ is relatively projective
$\mathcal{K}(H)$-module because $\mathcal{B}(H)=\mathcal{K}(H)$ and
$\mathcal{K}(H)$ is relatively projective $\mathcal{K}(H)$-module as was shown
above.

$(ii)$ From proposition~\ref{KHAndBHModBH} paragraph $(i)$ and
proposition~\ref{MetProjIsTopProjAndTopProjIsRelProj} it follows that
$\mathcal{B}(H)$ is relatively projective $\mathcal{B}(H)$-module. From
[\cite{RamsHomPropSemgroupAlg}, propositions 2.3.3, 2.3.4] we know that
$\langle$~an essential relatively projective / a faithful relatively
injective~$\rangle$ module over ideal of Banach algebra is $\langle$~relatively
projective / relative injective~$\rangle$ over algebra itself. Since
$\mathcal{K}(H)$ and $\mathcal{N}(H)$ are essential and faithful
$\mathcal{K}(H)$-modules, then from results of previous paragraph
$\mathcal{N}(H)$ is relatively projective and injective, while $\mathcal{K}(H)$
is relatively projective as $\mathcal{B}(H)$-modules. Now, by
[\cite{HelBanLocConvAlg}, proposition 7.1.40] all aforementioned modules are
relatively flat $\mathcal{B}(H)$-modules. In particular
$\mathcal{B}(H)\isom{\mathbf{mod}_1-\mathcal{B}(H)}{\mathcal{N}(H)}^*$ is
relatively injective $\mathcal{B}(H)$-module.
\end{proof}

Results of this section are summarized in the following three tables. Each cell
contains a condition under which the respective module has the respective
property and propositions where this is proved. We use ``?'' symbol to indicate
open problems. These tables confirm that the property of being homologically
trivial in metric and topological theory is too restrictive. It is easier to
mention cases where metric and topological properties coincide with relative
ones: flatness of $\mathcal{K}(H)$ as $\mathcal{B}(H)$- or
$\mathcal{K}(H)$-module, injectivity of $\mathcal{N}(H)$ as $\mathcal{B}(H)$- or
$\mathcal{K}(H)$-module, projectivity and flatness of $\mathcal{B}(H)$-module
$\mathcal{B}(H)$. In the remaining cases $H$ needs to be at least finite
dimensional in order to these properties be equivalent in metric, topological
and relative theory.


\begin{scriptsize}
    \begin{longtable}{|c|c|c|c|c|c|c|} 
    \multicolumn{7}{c}{
        \mbox{
            Homologically trivial $\mathcal{K}(H)$- 
            and $\mathcal{B}(H)$-modules in metric theory
        }
    }                                                                                                                                                                                                                                                                                                                                                                                                                                                    \\
    \hline &
    \multicolumn{3}{c|}{
        $\mathcal{K}(H)$-modules
    } & 
    \multicolumn{3}{c|}{
        $\mathcal{B}(H)$-modules
    }                                                                                                                                                                                                                       \\
    \hline & 
        \mbox{Projectivity} & 
        \mbox{Injectivity} & 
        \mbox{Flatness} & 
        \mbox{Projectivity} & 
        \mbox{Injectivity} & 
        \mbox{Flatness} \\ 
    \hline
        $\mathcal{N}(H)$ & 
        \begin{tabular}{@{}c@{}}
            $\dim(H)\leq 1$ \\
            \mbox{\ref{KHAndBHModNH}}
        \end{tabular} & 
        \begin{tabular}{@{}c@{}}
            $H$\mbox{ is any } \\
            \mbox{\ref{KHAndBHModNH}}
        \end{tabular} & 
        \begin{tabular}{@{}c@{}}
            $\dim(H)\leq 1$ \\
            \mbox{\ref{KHAndBHModNH}}
        \end{tabular} & 
        \begin{tabular}{@{}c@{}}
            $\dim(H)\leq 1$ \\
            \mbox{\ref{KHAndBHModNH}}
        \end{tabular} & 
        \begin{tabular}{@{}c@{}}
            $H$\mbox{ is any } \\
            \mbox{\ref{KHAndBHModNH}}
        \end{tabular} & 
        \begin{tabular}{@{}c@{}}
            $\dim(H)\leq 1$ \\
            \mbox{\ref{KHAndBHModNH}}
        \end{tabular} \\
    \hline
        $\mathcal{B}(H)$ & 
        \begin{tabular}{@{}c@{}}
            $\dim(H)<\aleph_0$ \\
            \mbox{\ref{KHAndBHModBH}}
        \end{tabular} & 
        \begin{tabular}{@{}c@{}}
            $\dim(H)\leq 1$ \\
            \mbox{\ref{KHAndBHModBH}}
        \end{tabular} & 
        \begin{tabular}{@{}c@{}}
            $\dim(H)<\aleph_0$ \\
            \mbox{\ref{KHAndBHModBH}}
        \end{tabular} & 
        \begin{tabular}{@{}c@{}}
            $H$\mbox{ is any } \\
            \mbox{\ref{KHAndBHModBH}}
        \end{tabular} & 
        \begin{tabular}{@{}c@{}}
            $\dim(H)\leq 1$ \\
            \mbox{\ref{KHAndBHModBH}}
        \end{tabular} & 
        \begin{tabular}{@{}c@{}}
            $H$\mbox{ is any } \\
            \mbox{\ref{KHAndBHModBH}}
        \end{tabular}          \\ 
    \hline
        $\mathcal{K}(H)$ & 
        \begin{tabular}{@{}c@{}}
            $\dim(H)<\aleph_0$ \\
            \mbox{\ref{KHAndBHModKH}}
        \end{tabular} & 
        \begin{tabular}{@{}c@{}}
            $\dim(H)\leq 1$ \\
            \mbox{\ref{KHAndBHModKH}}
        \end{tabular} & 
        \begin{tabular}{@{}c@{}}
            $H$\mbox{ is any } \\
            \mbox{\ref{KHAndBHModKH}}
        \end{tabular} & 
        \begin{tabular}{@{}c@{}}
            $\dim(H)<\aleph_0$ \\
            \mbox{\ref{KHAndBHModKH}}
        \end{tabular} & 
        \begin{tabular}{@{}c@{}}
            $\dim(H)\leq 1$ \\
            \mbox{\ref{KHAndBHModKH}}
        \end{tabular} & 
        \begin{tabular}{@{}c@{}}
            $H$\mbox{ is any } \\
            \mbox{\ref{KHAndBHModKH}}
        \end{tabular} \\ 
    \hline
    \multicolumn{7}{c}{
        \mbox{
            Homologically trivial $\mathcal{K}(H)$- 
            and $\mathcal{B}(H)$-modules in topological theory
        }
    } \\
    \hline & 
    \multicolumn{3}{c|}{
        $\mathcal{K}(H)$-modules
    } & 
    \multicolumn{3}{c|}{
        $\mathcal{B}(H)$-modules
    } \\
    \hline & 
        \mbox{Projectivity} & 
        \mbox{Injectivity} & 
        \mbox{Flatness} & 
        \mbox{Projectivity} & 
        \mbox{Injectivity} & 
        \mbox{Flatness} \\ 
    \hline
        $\mathcal{N}(H)$ & 
        \begin{tabular}{@{}c@{}}
            $\dim(H)<\aleph_0$ \\
            \mbox{\ref{KHAndBHModNH}}
        \end{tabular} & 
        \begin{tabular}{@{}c@{}}
            $H$\mbox{ is any } \\
            \mbox{\ref{KHAndBHModNH}}
        \end{tabular} & 
        \begin{tabular}{@{}c@{}}
            $\dim(H)<\aleph_0$ \\
            \mbox{\ref{KHAndBHModNH}}
        \end{tabular} & 
        \begin{tabular}{@{}c@{}}
            $\dim(H)<\aleph_0$ \\
            \mbox{\ref{KHAndBHModNH}}
        \end{tabular} & 
        \begin{tabular}{@{}c@{}}
            $H$\mbox{ is any } \\
            \mbox{\ref{KHAndBHModNH}}
        \end{tabular} & 
        \begin{tabular}{@{}c@{}}
            $\dim(H)<\aleph_0$ \\
            \mbox{\ref{KHAndBHModNH}}
        \end{tabular} \\
    \hline
        $\mathcal{B}(H)$ & 
        \begin{tabular}{@{}c@{}}
            $\dim(H)<\aleph_0$ \\
            \mbox{\ref{KHAndBHModBH}}
        \end{tabular} & 
        \begin{tabular}{@{}c@{}}
            $\dim(H)<\aleph_0$ \\
            \mbox{\ref{KHAndBHModBH}}
        \end{tabular} & 
        \begin{tabular}{@{}c@{}}
            $\dim(H)<\aleph_0$ \\
            \mbox{\ref{KHAndBHModBH}}
        \end{tabular} & 
        \begin{tabular}{@{}c@{}}
            $H$\mbox{ is any } \\
            \mbox{\ref{KHAndBHModBH}}
        \end{tabular} & 
        \begin{tabular}{@{}c@{}}
            $\dim(H)<\aleph_0$ \\
            \mbox{\ref{KHAndBHModBH}}
        \end{tabular} & 
        \begin{tabular}{@{}c@{}}
            $H$\mbox{ is any } \\
            \mbox{\ref{KHAndBHModBH}}
        \end{tabular} \\ 
    \hline
        $\mathcal{K}(H)$ &
        \begin{tabular}{@{}c@{}}
            $\dim(H)<\aleph_0$ \\
            \mbox{\ref{KHAndBHModKH}}
        \end{tabular} & 
        \begin{tabular}{@{}c@{}}
            $\dim(H)<\aleph_0$ \\
            \mbox{\ref{KHAndBHModKH}}
        \end{tabular} & 
        \begin{tabular}{@{}c@{}}
            $H$\mbox{ is any } \\
            \mbox{\ref{KHAndBHModKH}}
        \end{tabular} & 
        \begin{tabular}{@{}c@{}}
            $\dim(H)<\aleph_0$ \\
            \mbox{\ref{KHAndBHModKH}}
        \end{tabular} & 
        \begin{tabular}{@{}c@{}}
            $\dim(H)<\aleph_0$ \\
            \mbox{\ref{KHAndBHModKH}}
        \end{tabular} & 
        \begin{tabular}{@{}c@{}}
            $H$\mbox{ is any } \\
            \mbox{\ref{KHAndBHModKH}}
        \end{tabular} \\ 
    \hline
    \multicolumn{7}{c}{
        \mbox{
            Homologically trivial $\mathcal{K}(H)$- 
            and $\mathcal{B}(H)$-modules in relative theory
        }
    } \\
    \hline & 
    \multicolumn{3}{c|}{
        $\mathcal{K}(H)$-modules
    } & 
    \multicolumn{3}{c|}{
        $\mathcal{B}(H)$-modules
    } \\
    \hline & 
        \mbox{Projectivity} & 
        \mbox{Injectivity} & 
        \mbox{Flatness} & 
        \mbox{Projectivity} & 
        \mbox{Injectivity} & 
        \mbox{Flatness} \\ 
    \hline
        $\mathcal{N}(H)$ & 
        \begin{tabular}{@{}c@{}}
            $H$\mbox{ is any } \\
            \mbox{\ref{KHAndBHModsRelTh}}, (i)
        \end{tabular} & 
        \begin{tabular}{@{}c@{}}
            $H$\mbox{ is any } \\
            \mbox{\ref{KHAndBHModsRelTh}}, (i)
        \end{tabular} & 
        \begin{tabular}{@{}c@{}}
            $H$\mbox{ is any } \\
            \mbox{\ref{KHAndBHModsRelTh}}, (i)
        \end{tabular} & 
        \begin{tabular}{@{}c@{}}
            $H$\mbox{ is any } \\
            \mbox{\ref{KHAndBHModsRelTh}}, (ii)
        \end{tabular} & 
        \begin{tabular}{@{}c@{}}
            $H$\mbox{ is any } \\
            \mbox{\ref{KHAndBHModsRelTh}}, (ii)
        \end{tabular} & 
        \begin{tabular}{@{}c@{}}
            $H$\mbox{ is any } \\
            \mbox{\ref{KHAndBHModsRelTh}}, (ii)
        \end{tabular} \\
    \hline
        $\mathcal{B}(H)$ & 
        \begin{tabular}{@{}c@{}}
            $\dim(H)<\aleph_0$ \\
            \mbox{\ref{KHAndBHModsRelTh}}, (i)
        \end{tabular} & 
        \begin{tabular}{@{}c@{}}
            $H$\mbox{ is any } \\
            \mbox{\ref{KHAndBHModsRelTh}}, (i)
        \end{tabular} & 
        \begin{tabular}{@{}c@{}}
            $H$\mbox{ is any } \\
            \mbox{\ref{KHAndBHModsRelTh}}, (i)
        \end{tabular} & 
        \begin{tabular}{@{}c@{}}
            $H$\mbox{ is any } \\
            \mbox{\ref{KHAndBHModsRelTh}}, (ii)
        \end{tabular} & 
        \begin{tabular}{@{}c@{}}
            $H$\mbox{ is any } \\
            \mbox{\ref{KHAndBHModsRelTh}}, (ii)
        \end{tabular} & 
        \begin{tabular}{@{}c@{}}
            $H$\mbox{ is any } \\
            \mbox{\ref{KHAndBHModsRelTh}}, (ii)
        \end{tabular} \\
    \hline
        $\mathcal{K}(H)$ & 
        \begin{tabular}{@{}c@{}}
            $H$\mbox{ is any } \\
            \mbox{\ref{KHAndBHModsRelTh}}, (i)
        \end{tabular} & 
        \begin{tabular}{@{}c@{}}
            $\dim(H)<\aleph_0$ \\
            \mbox{\ref{KHAndBHModsRelTh}}, (i)
        \end{tabular} & 
        \begin{tabular}{@{}c@{}}
            $H$\mbox{ is any } \\
            \mbox{\ref{KHAndBHModsRelTh}}, (i)
        \end{tabular} & 
        \begin{tabular}{@{}c@{}}
            $H$\mbox{ is any } \\
            \mbox{\ref{KHAndBHModsRelTh}}, (ii)
        \end{tabular} & 
        \begin{tabular}{@{}c@{}} 
            {?}
        \end{tabular} & 
        \begin{tabular}{@{}c@{}}
            $H$\mbox{ is any } \\
            \mbox{\ref{KHAndBHModsRelTh}}, (ii)
        \end{tabular} \\
    \hline
    \end{longtable}
\end{scriptsize}

%-------------------------------------------------------------------------------
%	c_0(\Lambda)- and l_infty(\Lambda)-modules
%-------------------------------------------------------------------------------

\subsection{
    \texorpdfstring{$c_0(\Lambda)$}{c0 (Lambda)}- and
    \texorpdfstring{$\ell_\infty(\Lambda)$}{lInfty (Lambda)}-modules
}\label{SubSectionc0AndlInftyModules}

We continue our study of modules over $C^*$-algebras and move to commutative
examples. For a given index set $\Lambda$ we consider spaces $c_0(\Lambda)$ and
$\ell_p(\Lambda)$ for $1\leq p\leq+\infty$ as left and right modules over
algebras $c_0(\Lambda)$ and $\ell_\infty(\Lambda)$. For all these modules the
module action is just the pointwise multiplication. It is well known that
${c_0(\Lambda)}^*\isom{\mathbf{Ban}_1}\ell_1(\Lambda)$ and
${\ell_p(\Lambda)}^*\isom{\mathbf{Ban}_1}\ell_{p^*}(\Lambda)$ 
for $1\leq p<+\infty$. In fact these isomorphisms are isomorphisms of
$\ell_\infty(\Lambda)$- and $c_0(\Lambda)$-modules. 

For a given $\lambda\in\Lambda$ we define $\mathbb{C}_\lambda$ as left or right
$\ell_\infty(\Lambda)$- or $c_0(\Lambda)$-module $\mathbb{C}$ with module action
defined by
$$
a\cdot_\lambda z=a(\lambda)z,\qquad z\cdot_\lambda a=a(\lambda) z.
$$
for $a\in \ell_\infty(\Lambda)$ and $z\in\mathbb{C}_s$. 

\begin{proposition}\label{OneDimlInftyc0ModMetTopProjIngFlat} Let $\Lambda$ be a
set and $\lambda\in\Lambda$. Then $\mathbb{C}_\lambda$ is metrically and
topologically projective injective and flat $\ell_\infty(\Lambda)$- or
$c_0(\Lambda)$-module.
\end{proposition}
\begin{proof} Let $A$ be either $\ell_\infty(\Lambda)$ or $c_0(\Lambda)$. One
can easily check that the 
maps $\pi:A_+\to\mathbb{C}_\lambda:a\oplus_1 z\mapsto a(\lambda)+z$ 
and $\sigma:\mathbb{C}_\lambda\to A_+:z\mapsto z\delta_\lambda\oplus_1 0$ 
are contractive $A$-morphisms of left $A$-modules.
Since $\pi\sigma=1_{\mathbb{C}_\lambda}$, then $\mathbb{C}_\lambda$ is retract
of $A_+$ in $A-\mathbf{mod}_1$. From proposition~\ref{UnitalAlgIsMetTopProj}
and~\ref{RetrMetCTopProjIsMetCTopProj} it follows that $\mathbb{C}_\lambda$ is
metrically and topologically projective left $A$-module and a fortiori
metrically and topologically flat by proposition~\ref{MetTopProjIsMetTopFlat}.
By proposition~\ref{DualMetTopProjIsMetrInj} we have that $\mathbb{C}_\lambda^*$
is metrically and topologically injective as $A$-module. Now metric and
topological injectivity of $\mathbb{C}_\lambda$ follow from isomorphism
$\mathbb{C}_\lambda\isom{\mathbf{mod}_1-A}\mathbb{C}_\lambda^*$.
\end{proof}

\begin{proposition}\label{c0AndlInftyModlIfty} Let $\Lambda$ be a set. Then

\begin{enumerate}[label = (\roman*)]
    \item $\ell_\infty(\Lambda)$ is metrically and topologically projective 
    and flat as $\ell_\infty(\Lambda)$-module;

    \item $\ell_\infty(\Lambda)$ is metrically or topologically projective 
    or flat as $c_0(\Lambda)$-module iff $\Lambda$ is finite;

    \item $\ell_\infty(\Lambda)$ is metrically and topologically injective as
    $\ell_\infty(\Lambda)$- and $c_0(\Lambda)$-module.
\end{enumerate}
\end{proposition}
\begin{proof} $(i)$ Since $\ell_\infty(\Lambda)$ is a unital algebra, then it is
metrically and topologically projective as $\ell_\infty(\Lambda)$-module by
proposition~\ref{UnitalAlgIsMetTopProj}. Results on flatness follow from
proposition~\ref{MetTopProjIsMetTopFlat}.

$(ii)$ For infinite $\Lambda$ the Banach space
$\ell_\infty(\Lambda)/c_0(\Lambda)$ is of infinite dimension, so by
proposition~\ref{CStarAlgIsTopFlatOverItsIdeal} the module
$\ell_\infty(\Lambda)$ neither topologically nor metrically flat as
$c_0(\Lambda)$-module. Both claims regarding projectivity follow from
proposition~\ref{MetTopProjIsMetTopFlat}. If $\Lambda$ is finite, then
$c_0(\Lambda)=\ell_\infty(\Lambda)$, so the result follows from paragraph $(i)$.

$(iii)$ Let $A$ be either $\ell_\infty(\Lambda)$ or $c_0(\Lambda)$. Note that
$\ell_\infty(\Lambda)
\isom{A-\mathbf{mod}_1}
\bigoplus_\infty \{\mathbb{C}_\lambda:\lambda\in\Lambda \}$, then from
propositions~\ref{OneDimlInftyc0ModMetTopProjIngFlat} and~\ref{MetTopInjModProd}
it follows that $\ell_\infty(\Lambda)$ is metrically injective as $A$-module.
Topological injectivity follows from
proposition~\ref{MetInjIsTopInjAndTopInjIsRelInj}.
\end{proof}

\begin{proposition}\label{c0AndlInftyModc0} Let $\Lambda$ be a set. Then 

\begin{enumerate}[label = (\roman*)]
    \item $c_0(\Lambda)$ is metrically and topologically flat as
    $\ell_\infty(\Lambda)$- or $c_0(\Lambda)$-module;

    \item $c_0(\Lambda)$ is metrically or topologically projective as
    $\ell_\infty(\Lambda)$- or $c_0(\Lambda)$-module iff $\Lambda$ is finite;

    \item $c_0(\Lambda)$ is metrically or topologically injective as
    $\ell_\infty(\Lambda)$- or $c_0(\Lambda)$-module iff $\Lambda$ is finite.
\end{enumerate}
\end{proposition}
\begin{proof} Let $A$ be either $\ell_\infty(\Lambda)$ or $c_0(\Lambda)$. Note
that $c_0(\Lambda)$ is a two-sided ideal of $A$. 

$(i)$ Recall that $c_0(\Lambda)$ has a contractive approximate identity of the
form ${(\sum_{\lambda\in S}\delta_\lambda)}_{S\in\mathcal{P}_0(\Lambda)}$. Since
$c_0(\Lambda)$ is a two-sided ideal of $A$, then the result follows from
proposition~\ref{IdealofCstarAlgisMetTopFlat}.

$(ii)$, $(iii)$ If $\Lambda$ is infinite, then $c_0(\Lambda)$ is not unital as
Banach algebra. From corollary~\ref{BiIdealOfCStarAlgMetTopProjCharac} and
proposition~\ref{MetTopInjOfId} the $A$-module $c_0(\Lambda)$ is neither
metrically nor topologically projective or injective. If $\Lambda$ is finite,
then $c_0(\Lambda)=\ell_\infty(\Lambda)$, so both results follow from paragraphs
$(i)$ and $(iii)$ of proposition~\ref{c0AndlInftyModlIfty}.
\end{proof}

\begin{proposition}\label{c0AndlInftyModl1} Let $\Lambda$ be a set. Then

\begin{enumerate}[label = (\roman*)]
    \item $\ell_1(\Lambda)$ is metrically and topologically injective as
    $\ell_\infty(\Lambda)$- or $c_0(\Lambda)$-module;

    \item $\ell_1(\Lambda)$ is metrically and topologically projective and 
    flat as $\ell_\infty(\Lambda)$- or $c_0(\Lambda)$-module;
\end{enumerate}
\end{proposition}
\begin{proof} Let $A$ be either $\ell_\infty(\Lambda)$ or $c_0(\Lambda)$.

$(i)$ Note that $\ell_1(\Lambda)\isom{\mathbf{mod}_1-A}{c_0(\Lambda)}^*$, so the
result follows from proposition~\ref{MetCTopFlatCharac} and paragraph $(i)$ of
proposition~\ref{c0AndlInftyModc0}.

$(ii)$ Note that 
$\ell_1(\Lambda)
\isom{A-\mathbf{mod}_1}
\bigoplus_1 \{\mathbb{C}_\lambda:\lambda\in\Lambda \}$, then from
propositions~\ref{OneDimlInftyc0ModMetTopProjIngFlat}
and~\ref{MetTopProjModCoprod} it follows that $\ell_1(\Lambda)$ is metrically
projective as $A$-module. Topological projectivity follows from
proposition~\ref{MetProjIsTopProjAndTopProjIsRelProj}. Metric and topological
flatness follow from proposition~\ref{MetTopProjIsMetTopFlat}.
\end{proof}

\begin{proposition}\label{c0AndlInftyModlp} Let $\Lambda$ be a set and
$1<p<+\infty$. Then $\ell_p(\Lambda)$ is metrically or topologically projective,
injective or flat as $\ell_\infty(\Lambda)$- or $c_0(\Lambda)$-module iff
$\Lambda$ is finite.
\end{proposition}
\begin{proof} Let $A$ be either $\ell_\infty(\Lambda)$ or $c_0(\Lambda)$, then
$A$ is an $\mathscr{L}_\infty^g$-space. Since $\ell_p(\Lambda)$ is reflexive for
$1<p<+\infty$, then from corollary
~\ref{NoInfDimRefMetTopProjInjFlatModOverMthscrL1OrLInfty} it follows that
$\ell_p(\Lambda)$ is necessarily finite dimensional if it is metrically or
topologically projective injective or flat. This is equivalent to $\Lambda$
being finite. If $\Lambda$ is finite then
$\ell_p(\Lambda)\isom{A-\mathbf{mod}}\ell_1(\Lambda)$ and
$\ell_p(\Lambda)\isom{\mathbf{mod}-A}\ell_1(\Lambda)$, so topological
projectivity injectivity and flatness follow from
proposition~\ref{c0AndlInftyModl1}.
\end{proof}

\begin{proposition}\label{c0AndlInftyModsRelTh} Let $\Lambda$ be a set. Then

\begin{enumerate}[label = (\roman*)]
    \item as $c_0(\Lambda)$-modules $\ell_p(\Lambda)$ for $1\leq p<+\infty$ and
    $\mathbb{C}_\lambda$ for $\lambda\in\Lambda$ are relatively projective,
    injective and flat, $c_0(\Lambda)$ relatively projective and flat, but
    relatively injective only for finite $\Lambda$, $\ell_\infty(\Lambda)$ is
    relatively injective and flat, but relatively projective only for finite
    $\Lambda$;

    \item as $\ell_\infty(\Lambda)$-modules $\ell_p(\Lambda)$ for 
    $1\leq p\leq+\infty$ and $\mathbb{C}_\lambda$ for $\lambda\in\Lambda$ 
    are relatively projective, injective and flat, $c_0(\Lambda)$ is 
    relatively projective, injective and flat.
\end{enumerate}
\end{proposition}
\begin{proof} $(i)$ The algebra $c_0(\Lambda)$ is relatively biprojective
[\cite{HelHomolBanTopAlg}, theorem IV.5.26] and admits a contractive approximate
identity, so by [\cite{HelBanLocConvAlg}, theorem 7.1.60] all essential
$c_0(\Lambda)$-modules are projective. Thus $c_0(\Lambda)$ and $\ell_p(\Lambda)$
for $1\leq p<+\infty$ are relatively projective $c_0(\Lambda)$-modules. A
fortiori they are relatively flat as $c_0(\Lambda)$-modules
[\cite{HelBanLocConvAlg}, proposition 7.1.40]. By the same proposition
$\ell_1(\Lambda)\isom{\mathbf{mod}_1-c_0(\Lambda)}{c_0(\Lambda)}^*$ and
$\ell_{p^*}(\Lambda)\isom{\mathbf{mod}_1-c_0(\Lambda)}{\ell_p(\Lambda)}^*$ for
$1\leq p<+\infty$ are relatively injective $c_0(\Lambda)$-modules. From
[\cite{RamsHomPropSemgroupAlg}, proposition 2.2.8 (i)] we know that a Banach
algebra relatively injective over itself as right module, necessarily has a left
identity. Therefore $c_0(\Lambda)$ is not relatively injective
$c_0(\Lambda)$-module for infinite $\Lambda$. If $\Lambda$ is finite, then
$c_0(\Lambda)$-module $c_0(\Lambda)$ is relatively injective because
$c_0(\Lambda)=\ell_\infty(\Lambda)$ and $\ell_\infty(\Lambda)$ is relatively
injective $c_0(\Lambda)$-module as was shown above. From
[\cite{HelHomolBanTopAlg}, corollary V.2.16(II)] we know that
$\ell_\infty(\Lambda)$ is not relatively projective as $c_0(\Lambda)$-module
provided $\Lambda$ is infinite. If $\Lambda$ is finite then
$\ell_\infty(\Lambda)$ is relatively projective $c_0(\Lambda)$-module because
$\ell_\infty(\Lambda)=c_0(\Lambda)$ and $c_0(\Lambda)$ is relatively projective
$c_0(\Lambda)$-module as was shown above.
Propositions~\ref{OneDimlInftyc0ModMetTopProjIngFlat},
~\ref{MetProjIsTopProjAndTopProjIsRelProj},
~\ref{MetInjIsTopInjAndTopInjIsRelInj}
and~\ref{MetFlatIsTopFlatAndTopFlatIsRelFlat} give the result for modules
$\mathbb{C}_\lambda$, where $\lambda\in\Lambda$.

$(ii)$ From proposition~\ref{c0AndlInftyModlIfty} paragraph $(i)$ and
proposition~\ref{MetProjIsTopProjAndTopProjIsRelProj} it follows that
$\ell_\infty(\Lambda)$ is relatively projective $\ell_\infty(\Lambda)$-module. 
In [\cite{NemANoteOnRelInjC0ModC0}, theorem 4.4] it was shown 
that $\ell_\infty(\Lambda)$-module $c_0(\Lambda)$ is relatively injective. 
From [\cite{RamsHomPropSemgroupAlg}, propositions 2.3.3, 2.3.4] we know that
$\langle$~an essential relatively projective / a  faithful relatively
injective~$\rangle$ module over an ideal of a Banach algebra is 
$\langle$~relatively projective / relatively injective~$\rangle$ over algebra 
itself. Since $c_0(\Lambda)$ and $\ell_p(\Lambda)$ for $1\leq p<+\infty$ are
essential and faithful $c_0(\Lambda)$-modules then from results of previous
paragraph $\ell_p(\Lambda)$ for $1\leq p<+\infty$ are relatively projective and
injective as $\ell_\infty(\Lambda)$-modules. Also we get that $c_0(\Lambda)$ is
relatively projective $\ell_\infty(\Lambda)$-module. Therefore all these
$\ell_\infty(\Lambda)$-modules are relatively flat [\cite{HelBanLocConvAlg},
proposition 7.1.40]. As the consequence $\ell_\infty(\Lambda)
\isom{\mathbf{mod}_1-\ell_\infty(\Lambda)} {\ell_1(\Lambda)}^*$ is relatively
injective $\ell_\infty(\Lambda)$-module.
Propositions~\ref{OneDimlInftyc0ModMetTopProjIngFlat},
~\ref{MetProjIsTopProjAndTopProjIsRelProj},
~\ref{MetInjIsTopInjAndTopInjIsRelInj}
and~\ref{MetFlatIsTopFlatAndTopFlatIsRelFlat} give the result for modules
$\mathbb{C}_\lambda$, where $\lambda\in\Lambda$.
\end{proof}

Results of this section are summarized in the following three tables. Each cell
contains a condition under which the respective module has the respective
property and propositions where this is proved. For the case 
of $\ell_\infty(\Lambda)$- and $c_0(\Lambda)$-modules $\ell_p(\Lambda)$ 
for $1<p<+\infty$ we don't have a criterion of homological triviality in metric
theory, just a necessary condition. We indicate this fact via symbol ${}^*$.
From these table one can easily see that for modules over commutative
$C^*$-algebras, there is much more in common between relative, metric and
topological  theory. For example $\ell_1(\Lambda)$ is projective injective and
flat as $\ell_\infty(\Lambda)$- or $c_0(\Lambda)$-module in all three theories.


\begin{scriptsize}
    \begin{longtable}{|c|c|c|c|c|c|c|} 
    \multicolumn{7}{c}{
        \mbox{
            Homologically trivial $c_0(\Lambda)$- 
            and $\ell_\infty(\Lambda)$-modules in metric theory
        }
    } \\ 
    \hline & 
    \multicolumn{3}{c|}{
        $c_0(\Lambda)$-modules
    } & 
    \multicolumn{3}{c|}{
        $\ell_\infty(\Lambda)$-modules
    } \\
    \hline & 
        \mbox{Projectivity} & 
        \mbox{Injectivity} & 
        \mbox{Flatness} & 
        \mbox{Projectivity} & 
        \mbox{Injectivity} & 
        \mbox{Flatness} \\ 
    \hline
        $\ell_1(\Lambda)$ & 
        \begin{tabular}{@{}c@{}}
            $\Lambda$\mbox{ is any } \\
            \mbox{\ref{c0AndlInftyModl1}}
        \end{tabular} & 
        \begin{tabular}{@{}c@{}}
            $\Lambda$\mbox{ is any } \\
            \mbox{\ref{c0AndlInftyModl1}}
        \end{tabular} & 
        \begin{tabular}{@{}c@{}}
            $\Lambda$\mbox{ is any } \\
            \mbox{\ref{c0AndlInftyModl1}}
        \end{tabular} & 
        \begin{tabular}{@{}c@{}}
            $\Lambda$\mbox{ is any } \\
            \mbox{\ref{c0AndlInftyModl1}}
        \end{tabular} & 
        \begin{tabular}{@{}c@{}}
            $\Lambda$\mbox{ is any }  \\
            \mbox{\ref{c0AndlInftyModl1}}
        \end{tabular} & 
        \begin{tabular}{@{}c@{}}
            $\Lambda$\mbox{ is any } \\
            \mbox{\ref{c0AndlInftyModl1}}
        \end{tabular} \\
    \hline 
        $\ell_p(\Lambda)$ & 
        \begin{tabular}{@{}c@{}}
            $\operatorname{Card}(\Lambda)<\aleph_0$ \\ 
            \mbox{\ref{c0AndlInftyModlp}}${}^*$
        \end{tabular} & 
        \begin{tabular}{@{}c@{}}
            $\operatorname{Card}(\Lambda)<\aleph_0$ \\ 
            \mbox{\ref{c0AndlInftyModlp}}${}^*$
        \end{tabular} & 
        \begin{tabular}{@{}c@{}}
            $\operatorname{Card}(\Lambda)<\aleph_0$ \\ 
            \mbox{\ref{c0AndlInftyModlp}}${}^*$
        \end{tabular} & 
        \begin{tabular}{@{}c@{}}
            $\operatorname{Card}(\Lambda)<\aleph_0$ \\ 
            \mbox{\ref{c0AndlInftyModlp}}${}^*$
        \end{tabular} & 
        \begin{tabular}{@{}c@{}}
            $\operatorname{Card}(\Lambda)<\aleph_0$ \\ 
            \mbox{\ref{c0AndlInftyModlp}}${}^*$
        \end{tabular} & 
        \begin{tabular}{@{}c@{}}
            $\operatorname{Card}(\Lambda)<\aleph_0$ \\ 
            \mbox{\ref{c0AndlInftyModlp}}${}^*$
        \end{tabular} \\
    \hline
        $\ell_\infty(\Lambda)$ & 
        \begin{tabular}{@{}c@{}}
            $\operatorname{Card}(\Lambda)<\aleph_0$ \\
            \mbox{\ref{c0AndlInftyModlIfty}}
        \end{tabular} & 
        \begin{tabular}{@{}c@{}}
            $\Lambda$\mbox{ is any } \\
            \mbox{\ref{c0AndlInftyModlIfty}}
        \end{tabular} & 
        \begin{tabular}{@{}c@{}}
            $\operatorname{Card}(\Lambda)<\aleph_0$ \\
            \mbox{\ref{c0AndlInftyModlIfty}}
        \end{tabular} & 
        \begin{tabular}{@{}c@{}}
            $\Lambda$\mbox{ is any } \\
            \mbox{\ref{c0AndlInftyModlIfty}}
        \end{tabular} & 
        \begin{tabular}{@{}c@{}}
            $\Lambda$\mbox{ is any } \\
            \mbox{\ref{c0AndlInftyModlIfty}}
        \end{tabular} & 
        \begin{tabular}{@{}c@{}}
            $\Lambda$\mbox{ is any } \\
            \mbox{\ref{c0AndlInftyModlIfty}}
        \end{tabular} \\ 
    \hline
        $c_0(\Lambda)$ & 
        \begin{tabular}{@{}c@{}}
            $\operatorname{Card}(\Lambda)<\aleph_0$ \\
            \mbox{\ref{c0AndlInftyModc0}}
        \end{tabular} & 
        \begin{tabular}{@{}c@{}}
            $\operatorname{Card}(\Lambda)< \aleph_0$ \\
            \mbox{\ref{c0AndlInftyModc0}}
        \end{tabular} & 
        \begin{tabular}{@{}c@{}}
            $\Lambda$\mbox{ is any } \\
            \mbox{\ref{c0AndlInftyModc0}}
        \end{tabular} & 
        \begin{tabular}{@{}c@{}}
            $\operatorname{Card}(\Lambda)<\aleph_0$ \\
            \mbox{\ref{c0AndlInftyModc0}}
        \end{tabular} & 
        \begin{tabular}{@{}c@{}}
            $\operatorname{Card}(\Lambda)< \aleph_0$ \\
            \mbox{\ref{c0AndlInftyModc0}}
        \end{tabular} & 
        \begin{tabular}{@{}c@{}}
            $\Lambda$\mbox{ is any } \\
            \mbox{\ref{c0AndlInftyModc0}}
        \end{tabular} \\ 
    \hline
        $\mathbb{C}_\lambda$ &
        \begin{tabular}{@{}c@{}}
            $\lambda$\mbox{ is any } \\
            \mbox{\ref{OneDimlInftyc0ModMetTopProjIngFlat}}
        \end{tabular} & 
        \begin{tabular}{@{}c@{}}
            $\lambda$\mbox{ is any } \\
            \mbox{\ref{OneDimlInftyc0ModMetTopProjIngFlat}}
        \end{tabular} & 
        \begin{tabular}{@{}c@{}}
            $\lambda$\mbox{ is any } \\
            \mbox{\ref{OneDimlInftyc0ModMetTopProjIngFlat}}
        \end{tabular} &
        \begin{tabular}{@{}c@{}}
            $\lambda$\mbox{ is any } \\
            \mbox{\ref{OneDimlInftyc0ModMetTopProjIngFlat}}
        \end{tabular} & 
        \begin{tabular}{@{}c@{}}
            $\lambda$\mbox{ is any } \\
            \mbox{\ref{OneDimlInftyc0ModMetTopProjIngFlat}}
        \end{tabular} & 
        \begin{tabular}{@{}c@{}}
            $\lambda$\mbox{ is any } \\
            \mbox{\ref{OneDimlInftyc0ModMetTopProjIngFlat}}
        \end{tabular} \\
    \hline 
    \multicolumn{7}{c}{
        \mbox{
            Homologically trivial $c_0(\Lambda)$- 
            and $\ell_\infty(\Lambda)$-modules in topological theory
        }
    } \\
    \hline & 
    \multicolumn{3}{c|}{
        $c_0(\Lambda)$-modules
    } & 
    \multicolumn{3}{c|}{
        $\ell_\infty(\Lambda)$-modules
    } \\
    \hline & 
        \mbox{Projectivity} & 
        \mbox{Injectivity} & 
        \mbox{Flatness} & 
        \mbox{Projectivity} & 
        \mbox{Injectivity} & 
        \mbox{Flatness} \\ 
    \hline 
        $\ell_1(\Lambda)$ & 
        \begin{tabular}{@{}c@{}}
            $\Lambda$\mbox{ is any }  \\
            \mbox{\ref{c0AndlInftyModl1}}
        \end{tabular} & 
        \begin{tabular}{@{}c@{}}
            $\Lambda$\mbox{ is any } \\
            \mbox{\ref{c0AndlInftyModl1}}
        \end{tabular} & 
        \begin{tabular}{@{}c@{}}
            $\Lambda$\mbox{ is any } \\
            \mbox{\ref{c0AndlInftyModl1}}
        \end{tabular} & 
        \begin{tabular}{@{}c@{}}
            $\Lambda$\mbox{ is any }  \\
            \mbox{\ref{c0AndlInftyModl1}}
        \end{tabular} & 
        \begin{tabular}{@{}c@{}}
            $\Lambda$\mbox{ is any } \\
            \mbox{\ref{c0AndlInftyModl1}}
        \end{tabular} & 
        \begin{tabular}{@{}c@{}}
            $\Lambda$\mbox{ is any } \\
            \mbox{\ref{c0AndlInftyModl1}}
        \end{tabular} \\
    \hline
        $\ell_p(\Lambda)$ & 
        \begin{tabular}{@{}c@{}}
            $\operatorname{Card}(\Lambda)<\aleph_0$ \\
            \mbox{\ref{c0AndlInftyModlp}}
        \end{tabular} & 
        \begin{tabular}{@{}c@{}}
            $\operatorname{Card}(\Lambda)<\aleph_0$ \\
            \mbox{\ref{c0AndlInftyModlp}}
        \end{tabular} & 
        \begin{tabular}{@{}c@{}}
            $\operatorname{Card}(\Lambda)<\aleph_0$ \\
            \mbox{\ref{c0AndlInftyModlp}}
        \end{tabular} & 
        \begin{tabular}{@{}c@{}}
            $\operatorname{Card}(\Lambda)<\aleph_0$ \\
            \mbox{\ref{c0AndlInftyModlp}}
        \end{tabular} & 
        \begin{tabular}{@{}c@{}}
            $\operatorname{Card}(\Lambda)<\aleph_0$ \\
            \mbox{\ref{c0AndlInftyModlp}}
        \end{tabular} & 
        \begin{tabular}{@{}c@{}}
            $\operatorname{Card}(\Lambda)<\aleph_0$ \\
            \mbox{\ref{c0AndlInftyModlp}}
        \end{tabular} \\
    \hline
        $\ell_\infty(\Lambda)$ & 
        \begin{tabular}{@{}c@{}}
            $\operatorname{Card}(\Lambda)<\aleph_0$ \\
            \mbox{\ref{c0AndlInftyModlIfty}}
        \end{tabular} & 
        \begin{tabular}{@{}c@{}}
            $\Lambda$\mbox{ is any } \\
            \mbox{\ref{c0AndlInftyModlIfty}}
        \end{tabular} & 
        \begin{tabular}{@{}c@{}}
            $\operatorname{Card}(\Lambda)<\aleph_0$ \\
            \mbox{\ref{c0AndlInftyModlIfty}}
        \end{tabular} & 
        \begin{tabular}{@{}c@{}}
            $\Lambda$\mbox{ is any } \\
            \mbox{\ref{c0AndlInftyModlIfty}}
        \end{tabular} & 
        \begin{tabular}{@{}c@{}}
            $\Lambda$\mbox{ is any } \\
            \mbox{\ref{c0AndlInftyModlIfty}}
        \end{tabular} & 
        \begin{tabular}{@{}c@{}}
            $\Lambda$\mbox{ is any } \\
            \mbox{\ref{c0AndlInftyModlIfty}}
        \end{tabular} \\ 
    \hline
        $c_0(\Lambda)$ &
        \begin{tabular}{@{}c@{}}
            $\operatorname{Card}(\Lambda)<\aleph_0$ \\
            \mbox{\ref{c0AndlInftyModc0}}
        \end{tabular} & 
        \begin{tabular}{@{}c@{}}
            $\operatorname{Card}(\Lambda)<\aleph_0$ \\
            \mbox{\ref{c0AndlInftyModc0}}
        \end{tabular} & 
        \begin{tabular}{@{}c@{}}
            $\Lambda$\mbox{ is any } \\
            \mbox{\ref{c0AndlInftyModc0}}
        \end{tabular} & 
        \begin{tabular}{@{}c@{}}
            $\operatorname{Card}(\Lambda)<\aleph_0$ \\
            \mbox{\ref{c0AndlInftyModc0}}
        \end{tabular} & 
        \begin{tabular}{@{}c@{}}
            $\operatorname{Card}(\Lambda)<\aleph_0$ \\
            \mbox{\ref{c0AndlInftyModc0}}
        \end{tabular} & 
        \begin{tabular}{@{}c@{}}
            $\Lambda$\mbox{ is any } \\
            \mbox{\ref{c0AndlInftyModc0}}
        \end{tabular} \\ 
    \hline
        $\mathbb{C}_\lambda$ & 
        \begin{tabular}{@{}c@{}}
            $\lambda$\mbox{ is any } \\
            \mbox{\ref{OneDimlInftyc0ModMetTopProjIngFlat}}
        \end{tabular} & 
        \begin{tabular}{@{}c@{}}
            $\lambda$\mbox{ is any } \\
            \mbox{\ref{OneDimlInftyc0ModMetTopProjIngFlat}}
        \end{tabular} & 
        \begin{tabular}{@{}c@{}}
            $\lambda$\mbox{ is any } \\
            \mbox{\ref{OneDimlInftyc0ModMetTopProjIngFlat}}
        \end{tabular} & 
        \begin{tabular}{@{}c@{}}
            $\lambda$\mbox{ is any } \\
            \mbox{\ref{OneDimlInftyc0ModMetTopProjIngFlat}}
        \end{tabular} & 
        \begin{tabular}{@{}c@{}}
            $\lambda$\mbox{ is any } \\
            \mbox{\ref{OneDimlInftyc0ModMetTopProjIngFlat}}
        \end{tabular} & 
        \begin{tabular}{@{}c@{}}
            $\lambda$\mbox{ is any } \\
            \mbox{\ref{OneDimlInftyc0ModMetTopProjIngFlat}}
        \end{tabular} \\
    \hline
    \multicolumn{7}{c}{
        \mbox{
            Homologically trivial $c_0(\Lambda)$- 
            and $\ell_\infty(\Lambda)$-modules in relative theory
        }
    } \\
    \hline & 
    \multicolumn{3}{c|}{
        $c_0(\Lambda)$-modules
    } & 
    \multicolumn{3}{c|}{
        $\ell_\infty(\Lambda)$-modules
    } \\
    \hline & 
        \mbox{Projectivity} & 
        \mbox{Injectivity} & 
        \mbox{Flatness} & 
        \mbox{Projectivity} & 
        \mbox{Injectivity} & 
        \mbox{Flatness} \\ 
    \hline
        $\ell_1(\Lambda)$ & 
        \begin{tabular}{@{}c@{}}
            $\Lambda$\mbox{ is any } \\
            \mbox{\ref{c0AndlInftyModsRelTh}, (i)}
        \end{tabular} & 
        \begin{tabular}{@{}c@{}}
            $\Lambda$\mbox{ is any } \\
            \mbox{\ref{c0AndlInftyModsRelTh}, (i)}
        \end{tabular} & 
        \begin{tabular}{@{}c@{}}
            $\Lambda$\mbox{ is any } \\
            \mbox{\ref{c0AndlInftyModsRelTh}, (i)}
        \end{tabular} & 
        \begin{tabular}{@{}c@{}}
            $\Lambda$\mbox{ is any } \\
            \mbox{\ref{c0AndlInftyModsRelTh}, (ii)}
        \end{tabular} & 
        \begin{tabular}{@{}c@{}}
            $\Lambda$\mbox{ is any } \\
            \mbox{\ref{c0AndlInftyModsRelTh}, (ii)}
        \end{tabular} & 
        \begin{tabular}{@{}c@{}}
            $\Lambda$\mbox{ is any } \\
            \mbox{\ref{c0AndlInftyModlIfty}, (ii)}
        \end{tabular} \\
    \hline 
        $\ell_p(\Lambda)$ & 
        \begin{tabular}{@{}c@{}}
            $\Lambda$\mbox{ is any } \\
            \mbox{\ref{c0AndlInftyModsRelTh}, (i)}
        \end{tabular} & 
        \begin{tabular}{@{}c@{}}
            $\Lambda$\mbox{ is any }  \\
            \mbox{\ref{c0AndlInftyModsRelTh}, (i)}
        \end{tabular} & 
        \begin{tabular}{@{}c@{}}
            $\Lambda$\mbox{ is any } \\
            \mbox{\ref{c0AndlInftyModsRelTh}, (i)}
        \end{tabular} & 
        \begin{tabular}{@{}c@{}}
            $\Lambda$\mbox{ is any }  \\
            \mbox{\ref{c0AndlInftyModsRelTh}, (ii)}
        \end{tabular} & 
        \begin{tabular}{@{}c@{}}
            $\Lambda$\mbox{ is any } \\
            \mbox{\ref{c0AndlInftyModsRelTh}, (ii)}
        \end{tabular} & 
        \begin{tabular}{@{}c@{}}
            $\Lambda$\mbox{ is any } \\
            \mbox{\ref{c0AndlInftyModlIfty}, (ii)}
        \end{tabular} \\
    \hline
        $\ell_\infty(\Lambda)$ & 
        \begin{tabular}{@{}c@{}}
            $\operatorname{Card}(\Lambda)<\aleph_0$ \\
            \mbox{\ref{c0AndlInftyModsRelTh}, (i)}
        \end{tabular} & 
        \begin{tabular}{@{}c@{}}
            $\Lambda$\mbox{ is any } \\
            \mbox{\ref{c0AndlInftyModsRelTh}, (i)}
        \end{tabular} & 
        \begin{tabular}{@{}c@{}}
            $\Lambda$\mbox{ is any } \\
            \mbox{\ref{c0AndlInftyModsRelTh}, (i)}
        \end{tabular} & 
        \begin{tabular}{@{}c@{}}
            $\Lambda$\mbox{ is any }  \\
            \mbox{\ref{c0AndlInftyModsRelTh}, (ii)}
        \end{tabular} & 
        \begin{tabular}{@{}c@{}}
            $\Lambda$\mbox{ is any } \\
            \mbox{\ref{c0AndlInftyModsRelTh}, (ii)}
        \end{tabular} & 
        \begin{tabular}{@{}c@{}}
            $\Lambda$\mbox{ is any } \\
            \mbox{\ref{c0AndlInftyModlIfty}, (ii)}
        \end{tabular} \\
    \hline
        $c_0(\Lambda)$ &
        \begin{tabular}{@{}c@{}}
            $\Lambda$\mbox{ is any } \\
            \mbox{\ref{c0AndlInftyModsRelTh}, (i)}
        \end{tabular} & 
        \begin{tabular}{@{}c@{}}
            $\operatorname{Card}(\Lambda)<\aleph_0$ \\
            \mbox{\ref{c0AndlInftyModsRelTh}, (i) }
        \end{tabular} & 
        \begin{tabular}{@{}c@{}}
            $\Lambda$\mbox{ is any } \\
            \mbox{\ref{c0AndlInftyModsRelTh}, (i)}
        \end{tabular} & 
        \begin{tabular}{@{}c@{}}
            $\Lambda$\mbox{ is any } \\
            \mbox{\ref{c0AndlInftyModsRelTh}, (ii)}
        \end{tabular} & 
        \begin{tabular}{@{}c@{}} 
            $\Lambda$\mbox{ is any } \\
            \mbox{\ref{c0AndlInftyModsRelTh}, (ii)} 
        \end{tabular} & 
        \begin{tabular}{@{}c@{}}
            $\Lambda$\mbox{ is any }  \\
            \mbox{\ref{c0AndlInftyModlIfty}, (ii)}
        \end{tabular} \\
    \hline 
        $\mathbb{C}_\lambda$ & 
        \begin{tabular}{@{}c@{}}
            $\lambda$\mbox{ is any } \\
            \mbox{\ref{c0AndlInftyModsRelTh}, (i)}
        \end{tabular} & 
        \begin{tabular}{@{}c@{}}
            $\lambda$\mbox{ is any }  \\
            \mbox{\ref{c0AndlInftyModsRelTh}, (i)}
        \end{tabular} & 
        \begin{tabular}{@{}c@{}}
            $\lambda$\mbox{ is any } \\
            \mbox{\ref{c0AndlInftyModsRelTh}, (i)}
        \end{tabular} & 
        \begin{tabular}{@{}c@{}}
            $\lambda$\mbox{ is any }  \\
            \mbox{\ref{c0AndlInftyModsRelTh}, (ii)}
        \end{tabular} & 
        \begin{tabular}{@{}c@{}}
            $\lambda$\mbox{ is any } \\
            \mbox{\ref{c0AndlInftyModsRelTh}, (ii)}
        \end{tabular} & 
        \begin{tabular}{@{}c@{}}
            $\lambda$\mbox{ is any }  \\
            \mbox{\ref{c0AndlInftyModlIfty}, (ii)}
        \end{tabular} \\
    \hline
    \end{longtable}
\end{scriptsize}

%-------------------------------------------------------------------------------
%	C_0(S)-modules
%-------------------------------------------------------------------------------

\subsection{
    \texorpdfstring{$C_0(S)$}{C0(S)}-modules
}\label{SubSectionC0SModules}

This section is devoted to study of homological triviality of classical modules
over algebra $C_0(S)$, where $S$ is a locally compact Hausdorff space. By
classical we mean modules $C_0(S)$, $M(S)$ and $L_p(S,\mu)$ for positive measure
$\mu\in M(S)$. The pointwise multiplication plays the role of outer action for
these modules.

Further we give short preliminaries on these modules. Recall that
${C_0(S)}^*\isom{\mathbf{Ban}_1}M(S)$ and
${L_p(S,\mu)}^*\isom{\mathbf{Ban}_1}L_{p^*}(S,\mu)$ for $1\leq p<+\infty$. 
In fact these identifications are isomorphisms of left and 
right $C_0(S)$-modules. For a given positive measure $\mu\in M(S)$ 
by $M_s(S,\mu)$ we shall denote the closed $C_0(S)$-submodule of $M(S)$ 
consisting of measures strictly singular with respect to $\mu$. Then the well 
known Lebesgue decomposition theorem can be stated 
as $M(S)\isom{\mathbf{Ban}_1}L_1(S,\mu)\bigoplus_1 M_s(S,\mu)$. Even
more, this identification is an isomorphism of left and right $C_0(S)$-modules. 

We obliged to emphasize here that we consider only finite Borel regular positive
measures. This shall simplify many considerations. For example, any atom of
regular measure on a locally compact Hausdorff spaces is a point
[\cite{BourbElemMathIntegLivVI}, chapter 5, \S 5, exercise 7]. Since we consider
only finite measures, one can not say that this section simply generalizes
results of the previous one. Strictly speaking these sections are different,
though their methods have much in common.

For a fixed point $s\in S$ by $\mathbb{C}_s$ we denote left or right Banach
$C_0(S)$-module $\mathbb{C}$ with outer action defined by
$$
a\cdot_s z=a(s)z,\qquad z\cdot_s a=a(s)z
$$
\begin{proposition}\label{OneDimC0SModMetTopRelProjIngFlat} Let $S$ be a locally
compact Hausdorff space and let $s\in S$. Then 

\begin{enumerate}[label = (\roman*)]
    \item $\mathbb{C}_s$ is metrically, topologically or relatively projective 
    as $C_0(S)$-module iff $s$ is an isolated point of $S$;

    \item $\mathbb{C}_s$ is metrically, topologically and relatively flat as
    $C_0(S)$-module.

    \item $\mathbb{C}_s$ is metrically, topologically and relatively injective 
    as $C_0(S)$-module;

\end{enumerate}
\end{proposition}
\begin{proof} $(i)$ If $\mathbb{C}_s$ is metrically or topologically or
relatively projective, then by
proposition~\ref{MetProjIsTopProjAndTopProjIsRelProj} it is at least relatively
projective. Now from [\cite{HelBanLocConvAlg}, proposition 7.1.31] we know that
the latter forces $s$ to be an isolated point of $S$. Conversely, assume $s$ is
an isolated point of $S$. One can easily check that the maps
$\pi:{C_0(S)}_+\to\mathbb{C}_s:a\oplus_1 z\mapsto a(s)+z$ and
$\sigma:\mathbb{C}_s\to {C_0(S)}_+:z\mapsto z\delta_s\oplus_1 0$ are contractive
$C_0(S)$-morphisms. Since $\pi\sigma=1_{\mathbb{C}_s}$, then $\mathbb{C}_s$ is a
retract of ${C_0(S)}_+$ in $C_0(S)-\mathbf{mod}_1$. From
propositions~\ref{RetrMetCTopProjIsMetCTopProj} and~\ref{UnitalAlgIsMetTopProj} 
it follows that $\mathbb{C}_s$ is metrically and topologically projective left
$C_0(S)$-module. From~\ref{MetProjIsTopProjAndTopProjIsRelProj} we conclude that
$\mathbb{C}_s$ is also relatively projective $C_0(S)$-module.

$(ii)$ By [\cite{HelBanLocConvAlg}, theorem 7.1.87] the algebra $C_0(S)$ is
relatively amenable. Since this algebra is a $C^*$-algebra it is $1$-relatively 
amenable [\cite{RundeAmenConstFour}, example 3]. Clearly, $\mathbb{C}_s$ is a
$1$-dimensional $L_1$-space and an essential $C_0(S)$-module. Therefore, by
proposition~\ref{MetTopEssL1FlatModAoverAmenBanAlg} this module is metrically
flat. Now the result follows from
proposition~\ref{MetFlatIsTopFlatAndTopFlatIsRelFlat}.

$(iii)$ From paragraph $(ii)$ and proposition~\ref{MetCTopFlatCharac} it follows
that $\mathbb{C}_s^*$ is metrically injective. By
proposition~\ref{MetFlatIsTopFlatAndTopFlatIsRelFlat} it is topologically and
relatively injective too. It remains to note that
$\mathbb{C}_s\isom{\mathbf{mod}_1-A}\mathbb{C}_s^*$. 
\end{proof}

\begin{proposition}\label{C0SC0SModMetTopRelProjInjFlat} Let $S$ be a locally
compact Hausdorff space and let $s\in S$. Then

\begin{enumerate}[label = (\roman*)]
    \item $C_0(S)$ is $\langle$~metrically / 
    topologically / relatively~$\rangle$
    projective as $C_0(S)$-module iff $S$ is $\langle$~compact / compact /
    paracompact~$\rangle$;

    \item $C_0(S)$ is metrically injective as $C_0(S)$-module iff $S$ is 
    a Stonean space, if $C_0(S)$ is topologically or relatively injective then
    $S$ is compact and $S=\beta(S\setminus \{ s\})$ for any limit point $s$;

    \item $C_0(S)$ is metrically, topologically and relatively flat as
    $C_0(S)$-module.
\end{enumerate}
\end{proposition}
\begin{proof} We regard $C_0(S)$ as a two-sided ideal of $C_0(S)$. Recall that
$\operatorname{Spec}(C_0(S))$ is homeomorphic to $S$ [\cite{HelHomolBanTopAlg},
corollary 3.1.6].

$(i)$ It is enough to note that by
$\langle$~proposition~\ref{IdealofCommCStarAlgMetTopProjCharac} /
proposition~\ref{IdealofCommCStarAlgMetTopProjCharac} /
[\cite{HelHomolBanTopAlg}, chapter IV,\S\S 2--3]~$\rangle$ the spectrum of
$C_0(S)$ is $\langle$~compact / compact / paracompact~$\rangle$. 

$(ii)$ The result on metric injectivity is a weakened version of
proposition~\ref{MetInjCStarAlgCharac}. The result on relatively injectivity 
follows from~[\cite{NemANoteOnRelInjC0ModC0}, theorem 4.4]. It remains to recall
that by proposition~\ref{MetInjIsTopInjAndTopInjIsRelInj} metrically injective
module is relatively injective.

$(iii)$ From proposition~\ref{IdealofCstarAlgisMetTopFlat} it immediately follows
that $C_0(S)$-module $C_0(S)$ is metrically and topologically flat. By
proposition~\ref{MetFlatIsTopFlatAndTopFlatIsRelFlat} it is also relatively
flat.
\end{proof}

\begin{proposition}\label{AtomsOfRelProjLpMod} Let $S$ be a locally compact
Hausdorff space, $\mu$ be a finite Borel regular positive measure on $S$. Assume
$1\leq p\leq+\infty$ and $C_0(S)$-module $L_p(S,\mu)$ is relatively projective.
Then any atom of $\mu$ is an isolated point in $S$.
\end{proposition} 
\begin{proof} Assume $\mu$ has at least one atom, otherwise there is nothing to
prove. From [\cite{BourbElemMathIntegLivVI}, chapter 5, \S 5, exercise 7] we
know that any atom of $\mu$ is a point. Call it $s$. Consider well defined
linear maps $\pi:L_p(\Omega,\mu)\to\mathbb{C}_s:f\mapsto f(s)$ and
$\sigma:\mathbb{C}_s\to L_p(\Omega,\mu):z\mapsto z\delta_s$. One can easily
check that these maps are $C_0(S)$-morphisms and $\pi\sigma=1_{\mathbb{C}_s}$.
Therefore, $\mathbb{C}_s$ is a retract of $L_p(S,\mu)$ in $C_0(S)-\mathbf{mod}$.
By assumption, the latter module is relatively projective, so by
[\cite{HelBanLocConvAlg}, proposition 7.1.6] the $C_0(S)$-module $\mathbb{C}_s$
is relatively projective. By paragraph $(i)$ of
proposition~\ref{C0SC0SModMetTopRelProjInjFlat} we see that $s$ is an isolated
point of $S$.
\end{proof}

\begin{proposition}\label{L1C0SModMetTopRelProjInjFlat} Let $S$ be a locally
compact Hausdorff space and $\mu$ be a finite Borel regular positive measure on
$S$. Then 

\begin{enumerate}[label = (\roman*)]
    \item $L_1(S,\mu)$ is metrically or topologically or relatively projective 
    as $C_0(S)$-module iff $\mu$ is purely atomic and all its atoms are 
    isolated points in $S$;

    \item $L_1(S,\mu)$ is metrically, topologically and relatively injective as
    $C_0(S)$-module;

    \item $L_1(S,\mu)$ is metrically, topologically and relatively flat as
    $C_0(S)$-module.
\end{enumerate}
\end{proposition}
\begin{proof} $(i)$ If $L_1(S,\mu)$ is metrically or topologically or relatively
projective, then by proposition~\ref{MetProjIsTopProjAndTopProjIsRelProj} it is
at least relatively projective. Now from [\cite{NemRelProjModLp}, theorem 2] 
the measure $\mu$ is purely atomic and all atoms are isolated  points.
Conversely, assume that $\mu$ is purely atomic and all atoms are isolated
points. By $S_a^{\mu}$ we denote the set of these atoms. Now one can easily show
that the linear map 
$i:L_1(S,\mu)\to\bigoplus_1 \{\mathbb{C}_s:s\in S_a^{\mu} \}
:f\mapsto \bigoplus_1 \{\mu( \{s \})f(s):s\in S_a^{\mu} \}$ is an isometric
isomorphism of $C_0(S)$-modules. By paragraphs $(i)$ of
propositions~\ref{MetTopProjModCoprod}
and~\ref{OneDimC0SModMetTopRelProjIngFlat} the $C_0(S)$-module $\bigoplus_1
\{\mathbb{C}_s:s\in S_a^{\mu} \}$ is metrically projective. Therefore so does
$L_1(S,\mu)$. By proposition~\ref{MetProjIsTopProjAndTopProjIsRelProj} it is
also topologically and relatively projective.

$(ii)$ By paragraph $(iii)$ of proposition~\ref{C0SC0SModMetTopRelProjInjFlat} 
the $C_0(S)$-module $C_0(S)$ is metrically flat. From
proposition~\ref{DualMetTopProjIsMetrInj} we get that
$M(S)\isom{\mathbf{mod}_1-C_0(S)}{C_0(S)}^*$ is metrically injective. Since
$M(S)\isom{\mathbf{mod}_1-C_0(S)}L_1(S,\mu)\bigoplus_1 M_s(S,\mu)$, then
$L_1(S,\mu)$ is a retract of $M(S)$ in $\mathbf{mod}_1-C_0(S)$. So by
proposition~\ref{RetrMetCTopInjIsMetCTopInj} the $C_0(S)$-module $L_1(S,\mu)$ is
metrically injective. Relative and topological injectivity of $L_1(S,\mu)$
follows from proposition~\ref{MetInjIsTopInjAndTopInjIsRelInj}.

$(iii)$ By [\cite{HelBanLocConvAlg}, theorem 7.1.87] the algebra $C_0(S)$ is
relatively amenable. Since this algebra is a $C^*$-algebra it is $1$-relatively 
amenable [\cite{RundeAmenConstFour}, example 3]. Since $L_1(S,\mu)$ is 
an essential $C_0(S)$-module which tautologically an $L_1$-space, then by
proposition~\ref{MetTopEssL1FlatModAoverAmenBanAlg} this module is metrically
flat. From proposition~\ref{MetFlatIsTopFlatAndTopFlatIsRelFlat} the
$C_0(S)$-module $L_1(S,\mu)$ is also topologically and relatively flat.
\end{proof}

\begin{proposition}\label{LpC0SModMetTopRelProjIngFlat} Let $S$ be a locally
compact Hausdorff space and $\mu$ be a finite Borel regular positive measure on
$S$. Assume $1<p<+\infty$, then 

\begin{enumerate}[label = (\roman*)]
    \item $L_p(S,\mu)$ is relatively injective and flat, but relatively 
    projective iff $\mu$ is purely atomic and all atoms are isolated points;

    \item $L_p(S,\mu)$ is topologically projective or injective or flat 
    iff $\mu$ is purely atomic with finitely many atoms;

    \item if $L_p(S,\mu)$ is metrically projective or injective or flat, 
    then $\mu$ is purely atomic with finitely many atoms.
\end{enumerate}
\end{proposition}
\begin{proof} $(i)$ By [\cite{HelBanLocConvAlg}, theorem 7.1.87] the algebra
$C_0(S)$ is relatively amenable. Now from [\cite{HelBanLocConvAlg}, theorem
7.1.60] it follows that $L_p(S,\mu)$ is relatively flat for all $1<p<+\infty$.
Note that $L_p(S,\mu)\isom{\mathbf{mod}_1-C_0(S)}{L_{p^*}(S,\mu)}^*$. Then from
[\cite{HelBanLocConvAlg}, proposition 7.1.42] we get that $L_p(S,\mu)$ is
relatively injective for any $1<p<+\infty$. Now assume that $L_p(S,\mu)$ is
relatively projective, then by [\cite{NemRelProjModLp}, theorem 2]
the measure $\mu$ is purely atomic and all its atoms are isolated points.
Conversely, let $\mu$ be purely atomic with all atoms isolated. Denote by
$S_a^{\mu}$ the set of these atoms. Since $S_a^{\mu}$ is discrete, then
$C_0(S_a^{\mu})$ is relatively biprojective [\cite{HelHomolBanTopAlg}, theorem
4.5.26]. Then $L_p(S,\mu)$ is relatively projective $C_0(S_a^{\mu})$-module
because it is essential module over relatively biprojective algebra with
two-sided bounded approximate identity. Clearly $C_0(S_a^{\mu})$ is a two-sided
ideal of $C_0(S)$, so from [\cite{RamsHomPropSemgroupAlg}, proposition 2.3.2(i)]
we get that $L_p(S,\mu)$ is relatively projective as $C_0(S)$-module.

$(ii), (iii)$ Assume that $L_p(S,\mu)$ is metrically or topologically projective
or injective or flat $C_0(S)$-module. Since $L_p(S,\mu)$ is reflexive and
$C_0(S)$ is an $\mathscr{L}_\infty^g$-space, then $L_p(S,\mu)$ is finite
dimensional by
corollary~\ref{NoInfDimRefMetTopProjInjFlatModOverMthscrL1OrLInfty}. The latter
is equivalent to measure $\mu$ being purely atomic with finitely many atoms. On
the other hand, if $\mu$ is purely atomic with finitely many atoms, then
$L_p(S,\mu)$ is topologically isomorphic to $L_1(S,\mu)$ as left or right
$C_0(S)$-module. The latter module is topologically projective, injective and
flat for our measure by proposition~\ref{L1C0SModMetTopRelProjInjFlat}. Hence so
does $L_p(S,\mu)$.
\end{proof}

\begin{proposition}\label{LinftyC0SModMetTopRelProjIngFlat} Let $S$ be a locally
compact Hausdorff space and $\mu$ be a finite Borel regular positive measure on
$S$. Then

\begin{enumerate}[label = (\roman*)]
    \item if $L_\infty(S,\mu)$ is metrically, topologically or relatively 
    projective, then $\mu$ is normal and with is pseudocompact support; 

    \item $L_\infty(S,\mu)$ is metrically, topologically and relatively 
    injective as $C_0(S)$-module;

    \item $L_\infty(S,\mu)$ is relatively flat $C_0(S)$-module.
\end{enumerate}
\end{proposition}
\begin{proof} $(i)$ From [\cite{NemRelProjModLp}, theorem 3] it follows 
that $\operatorname{supp}(\mu)$ is pseudocompact and $\mu$ is inner open 
regular. Since $\mu$ is regular and finite it is 
normal [\cite{NemRelProjModLp}, proposition 9].
    
$(ii)$ Since
$L_\infty(S,\mu)\isom{\mathbf{mod}_1-C_0(S)}{L_1(S,\mu)}^*$, then the result
immediately follows from proposition~\ref{DualMetTopProjIsMetrInj} and paragraph
$(iii)$ of proposition~\ref{L1C0SModMetTopRelProjInjFlat}.

$(iii)$ By [\cite{HelBanLocConvAlg}, theorem 7.1.87] the algebra $C_0(S)$ is
relatively amenable. Any left Banach module over relatively amenable Banach
algebra is relatively flat [\cite{HelBanLocConvAlg}, theorem 7.1.60]. In
particular $L_\infty(S,\mu)$ is relatively flat $C_0(S)$-module.
\end{proof}

\begin{proposition}\label{MSC0SModMetTopRelProjIngFlat} Let $S$ be a locally
compact Hausdorff space and $\mu$ be a finite Borel regular positive measure on
$S$. Then

\begin{enumerate}[label = (\roman*)]
    \item $M(S)$ is metrically or topologically or relatively projective as
    $C_0(S)$-module iff $S$ is discrete; 

    \item $M(S)$ is metrically, topologically and relatively injective as
    $C_0(S)$-module; 

    \item $M(S)$ is metrically, topologically and relatively flat as
    $C_0(S)$-module.
\end{enumerate}
\end{proposition}
\begin{proof} $(i)$ If $M(S)$ is metrically or topologically or relatively
projective, then by proposition~\ref{MetProjIsTopProjAndTopProjIsRelProj} it is
at least relatively projective. For arbitrary $s\in S$ consider measure
$\mu=\delta_s$ and recall the decomposition
$M(S)\isom{C_0(S)-\mathbf{mod}_1}L_1(S,\mu)\bigoplus_1 M_s(S,\mu)$. Then
$L_1(S,\mu)$ is a retract of $M(S)$ in $C_0(S)-\mathbf{mod}_1$. So from
[\cite{HelBanLocConvAlg}, proposition 7.1.6] we get that $L_1(S,\mu)$ is
relatively projective $C_0(S)$-module. Since $s$ is the only atom of $\mu$, then
from proposition~\ref{L1C0SModMetTopRelProjInjFlat} it follows that $s$ is an
isolated point in $S$. Since $s\in S$ is arbitrary, then $S$ is discrete.
Conversely, assume $S$ is discrete. Then $C_0(S)=c_0(S)$, and
$M(S)
\isom{C_0(S)-\mathbf{mod}_1}{C_0(S)}^*
\isom{C_0(S)-\mathbf{mod}_1}\ell_1(S)
\isom{C_0(S)-\mathbf{mod}_1}
\bigoplus_1 \{\mathbb{C}_s:s\in S \}$. The latter $C_0(S)$-module is metrically
projective by paragraphs $(i)$ of propositions~\ref{MetTopProjModCoprod}
and~\ref{OneDimC0SModMetTopRelProjIngFlat}. Therefore $M(S)$ is metrically
projective $C_0(S)$-module too. By
proposition~\ref{MetProjIsTopProjAndTopProjIsRelProj} it is also topologically
and relatively projective.

$(ii)$ Since $M(S)\isom{\mathbf{mod}_1-C_0(S)}{C_0(S)}^*$, then the result
immediately follows from proposition~\ref{DualMetTopProjIsMetrInj} and paragraph
$(iii)$ of proposition~\ref{C0SC0SModMetTopRelProjInjFlat}.

$(iii)$ By [\cite{HelBanLocConvAlg}, theorem 7.1.87] the algebra $C_0(S)$ is
relatively amenable. Since this algebra is a $C^*$-algebra it is $1$-relatively 
amenable [\cite{RundeAmenConstFour}, example 2]. Note that $M(S)$ is 
an essential $C_0(S)$-module which as Banach space is an $L_1$-space
[\cite{DalLauSecondDualOfMeasAlg}, discussion after proposition 2.14]. Then by
proposition~\ref{MetTopEssL1FlatModAoverAmenBanAlg} this module is metrically
flat. From proposition~\ref{MetFlatIsTopFlatAndTopFlatIsRelFlat} it is also
topologically and relatively flat.
\end{proof}

Results of this section are summarized in the following three tables. Each cell
contains a condition under which the respective module has the respective
property and propositions where it is proved. We use ``?'' symbol to indicate 
open problems. Open problems of this section are divided into three parts:
injectivity of $C_0(S)$, projectivity of $L_\infty(S,\mu)$ and flatness of
$L_\infty(S,\mu)$. Complete description of relatively and topologically
injective $C_0(S)$-modules $C_0(S)$ seems quite a challenge for one simple
reason --- still there is no standard category of functional analysis where even
topologically injective objects were fully understood. The question of relative
projectivity of $C_0(S)$-module $L_\infty(S,\mu)$ is rather old. It seems that
even relative projectivity of $L_\infty(S,\mu)$ is a rare property. Our
conjecture that $\mu$ must be purely atomic with finitely many atoms. Finally we
presume that a necessary condition for metric and topological flatness of
$C_0(S)$-module $L_\infty(S,\mu)$ is compactness of $S$.

This section cotaines many other unsolved problems. Usually we have a strong
necessary condition. We use ${}^{*}$ to indicate that. As for partial results, 
we don't have a criterion of homological triviality of $C_0(S)$-modules 
$L_p(S,\mu)$ in metric theory for $1<p<+\infty$. Using advanced
Banach geometric techniques on factorization constants through finite
dimensional Hilbert spaces one may show that atoms count for metrically
projective modules $L_p(S,\mu)$ doesn't exceed some universal constant. It seems
that $L_p(S,\mu)$ is homologically trivial $C_0(S)$-module in metric theory only
for purely atomic measures with unique atom. 

\begin{scriptsize}
    \begin{longtable}{|c|c|c|c|} 
    \multicolumn{4}{c}{
        \mbox{
            Homologically trivial $C_0(S)$-modules in metric theory
        }
    }                                                                                                                                                                                                                                                                                                                                                                                                                               \\
    \hline & 
    \mbox{Projectivity} & 
    \mbox{Injectivity} & 
    \mbox{Flatness} \\
    \hline
        $L_1(S,\mu)$ & 
        \begin{tabular}{@{}c@{}}
            $\mu$\mbox{ is purely atomic, all } \\ 
            \mbox{ atoms are isolated points } \\
            \mbox{\ref{L1C0SModMetTopRelProjInjFlat}} (i)
        \end{tabular} & 
        \begin{tabular}{@{}c@{}}
            $\mu$\mbox{ is any }  \\
            \mbox{\ref{L1C0SModMetTopRelProjInjFlat}} (ii)
        \end{tabular} & 
        \begin{tabular}{@{}c@{}}
            $\mu$\mbox{ is any }  \\
            \mbox{\ref{L1C0SModMetTopRelProjInjFlat}} (iii)
        \end{tabular} \\
    \hline
        $L_p(S,\mu)$ & 
        \begin{tabular}{@{}c@{}}
            $\mu$\mbox{ is purely atomic } \\ 
            \mbox{ with finitely many atoms } \\ 
            \mbox{\ref{LpC0SModMetTopRelProjIngFlat}} (iii)${}^{*}$
        \end{tabular} & 
        \begin{tabular}{@{}c@{}}
            $\mu$\mbox{ is purely atomic } \\ 
            \mbox{ with finitely many atoms } \\ 
            \mbox{\ref{LpC0SModMetTopRelProjIngFlat}} (iii)${}^{*}$
        \end{tabular} & 
        \begin{tabular}{@{}c@{}}
            $\mu$\mbox{ is purely atomic } \\ 
            \mbox{ with finitely many atoms } \\ 
            \mbox{\ref{LpC0SModMetTopRelProjIngFlat}} (iii)${}^{*}$
        \end{tabular} \\
    \hline
        $L_\infty(S,\mu)$ & 
        \begin{tabular}{@{}c@{}} 
            $\mu$ is normal, with \\
            pseudocompact support \\
            \mbox{\ref{LinftyC0SModMetTopRelProjIngFlat}} (i)${}^{*}$
        \end{tabular} & 
        \begin{tabular}{@{}c@{}}
            $\mu$\mbox{ is any } \\
            \mbox{\ref{LinftyC0SModMetTopRelProjIngFlat}} (ii)
        \end{tabular} & 
        \begin{tabular}{@{}c@{}} 
            {?}
        \end{tabular} \\
    \hline
        $M(S)$ & 
        \begin{tabular}{@{}c@{}}
            $S$\mbox{ is discrete } \\
            \mbox{\ref{MSC0SModMetTopRelProjIngFlat}}
        \end{tabular} & 
        \begin{tabular}{@{}c@{}}
            $S$\mbox{ is any } \\
            \mbox{\ref{MSC0SModMetTopRelProjIngFlat}}
        \end{tabular} & 
        \begin{tabular}{@{}c@{}}
            $S$\mbox{ is any } \\
            \mbox{\ref{MSC0SModMetTopRelProjIngFlat}}
        \end{tabular} \\
    \hline
        $C_0(S)$ & 
        \begin{tabular}{@{}c@{}}
            $S$\mbox{ is compact } \\
            \mbox{\ref{C0SC0SModMetTopRelProjInjFlat}} (i)
        \end{tabular} & 
        \begin{tabular}{@{}c@{}}
            $S$\mbox{ is Stonean } \\
            \mbox{\ref{C0SC0SModMetTopRelProjInjFlat}} (ii) 
        \end{tabular} & 
        \begin{tabular}{@{}c@{}}
            $S$\mbox{ is any } \\
            \mbox{\ref{C0SC0SModMetTopRelProjInjFlat}} (iii)
        \end{tabular} \\
    \hline
        $\mathbb{C}_s$ & 
        \begin{tabular}{@{}c@{}}
            $s$\mbox{ is an isolated point } \\
            \mbox{\ref{OneDimC0SModMetTopRelProjIngFlat}}
        \end{tabular} & 
        \begin{tabular}{@{}c@{}}
            $s$\mbox{ is any } \\
            \mbox{\ref{OneDimC0SModMetTopRelProjIngFlat}}
        \end{tabular} &
        \begin{tabular}{@{}c@{}}
            $s$\mbox{ is any } \\
            \mbox{\ref{OneDimC0SModMetTopRelProjIngFlat}}
        \end{tabular} \\
    \hline
    \multicolumn{4}{c}{
        \mbox{
            Homologically trivial $C_0(S)$-modules in topological theory
        }
    } \\
    \hline & 
        \mbox{Projectivity} & 
        \mbox{Injectivity} & 
        \mbox{Flatness} \\
    \hline
        $L_1(S,\mu)$ & 
        \begin{tabular}{@{}c@{}}
            $\mu$\mbox{ is purely atomic, all } \\ 
            \mbox{ atoms are isolated points } \\
            \mbox{\ref{L1C0SModMetTopRelProjInjFlat}}
        \end{tabular} & 
        \begin{tabular}{@{}c@{}}
            $\mu$\mbox{ is any }  \\
            \mbox{\ref{L1C0SModMetTopRelProjInjFlat}}
        \end{tabular} & 
        \begin{tabular}{@{}c@{}}
            $\mu$\mbox{ is any } \\
            \mbox{\ref{L1C0SModMetTopRelProjInjFlat}}
        \end{tabular} \\
    \hline
        $L_p(S,\mu)$ & 
        \begin{tabular}{@{}c@{}}
            $\mu$\mbox{ is purely atomic } \\ 
            \mbox{ with finitely many atoms } \\
            \mbox{\ref{LpC0SModMetTopRelProjIngFlat}} (ii)
        \end{tabular} & 
        \begin{tabular}{@{}c@{}}
            $\mu$\mbox{ is purely atomic } \\ 
            \mbox{ with finitely many atoms } \\
            \mbox{\ref{LpC0SModMetTopRelProjIngFlat}} (ii)
        \end{tabular} & 
        \begin{tabular}{@{}c@{}}
            $\mu$\mbox{ is purely atomic } \\ 
            \mbox{ with finitely many atoms } \\
            \mbox{\ref{LpC0SModMetTopRelProjIngFlat}} (ii)
        \end{tabular} \\
    \hline
        $L_\infty(S,\mu)$ & 
        \begin{tabular}{@{}c@{}} 
            $\mu$ is normal, with \\
            pseudocompact support \\
            \mbox{\ref{LinftyC0SModMetTopRelProjIngFlat}} (i)${}^{*}$
        \end{tabular} & 
        \begin{tabular}{@{}c@{}}
            $\mu$\mbox{ is any } \\
            \mbox{\ref{LinftyC0SModMetTopRelProjIngFlat}} (ii)
        \end{tabular} & 
        \begin{tabular}{@{}c@{}}
            {?}
        \end{tabular} \\
    \hline
        $M(S)$ & 
        \begin{tabular}{@{}c@{}}
            $S$\mbox{ is discrete } \\
            \mbox{\ref{MSC0SModMetTopRelProjIngFlat}} (i)
        \end{tabular} & 
        \begin{tabular}{@{}c@{}}
            $S$\mbox{ is any } \\
            \mbox{\ref{MSC0SModMetTopRelProjIngFlat}} (ii)
        \end{tabular} & 
        \begin{tabular}{@{}c@{}}
            $S$\mbox{ is any } \\
            \mbox{\ref{MSC0SModMetTopRelProjIngFlat}} (iii)
        \end{tabular} \\
    \hline
        $C_0(S)$ & 
        \begin{tabular}{@{}c@{}}
            $S$\mbox{ is compact } \\
            \mbox{\ref{C0SC0SModMetTopRelProjInjFlat}} (i)
        \end{tabular} & 
        \begin{tabular}{@{}c@{}} 
            $S=\beta(S\setminus \{s \})$ \\
            for any limit point $s$ \\
            \mbox{\ref{C0SC0SModMetTopRelProjInjFlat}} (ii)${}^{*}$
        \end{tabular} & 
        \begin{tabular}{@{}c@{}}
            $S$\mbox{ is any } \\
            \mbox{\ref{C0SC0SModMetTopRelProjInjFlat}} (iii)
        \end{tabular} \\
    \hline
        $\mathbb{C}_s$ & 
        \begin{tabular}{@{}c@{}}
            $s$\mbox{ is an isolated point } \\
            \mbox{\ref{OneDimC0SModMetTopRelProjIngFlat}}
        \end{tabular} & 
        \begin{tabular}{@{}c@{}}
            $s$\mbox{ is any } \\
            \mbox{\ref{OneDimC0SModMetTopRelProjIngFlat}}
        \end{tabular} & 
        \begin{tabular}{@{}c@{}}
            $s$\mbox{ is any } \\
            \mbox{\ref{OneDimC0SModMetTopRelProjIngFlat}}
        \end{tabular} \\
    \hline
    \multicolumn{4}{c}{
        \mbox{
            Homologically trivial $C_0(S)$-modules in relative theory
        }
    } \\
    \hline & 
        \mbox{Projectivity} & 
        \mbox{Injectivity} & 
        \mbox{Flatness} \\
    \hline
        $L_1(S,\mu)$ & 
        \begin{tabular}{@{}c@{}}
            $\mu$\mbox{ is purely atomic, all } \\ 
            \mbox{ atoms are isolated points } \\
            \mbox{\ref{L1C0SModMetTopRelProjInjFlat}}
        \end{tabular} & 
        \begin{tabular}{@{}c@{}}
            $\mu$\mbox{ is any }  \\
            \mbox{\ref{L1C0SModMetTopRelProjInjFlat}}
        \end{tabular} & 
        \begin{tabular}{@{}c@{}}
            $\mu$\mbox{ is any } \\
            \mbox{\ref{L1C0SModMetTopRelProjInjFlat}}
        \end{tabular} \\
    \hline
        $L_p(S,\mu)$ & 
        \begin{tabular}{@{}c@{}}
            $\mu$\mbox{ is purely atomic, all } \\ 
            \mbox{ atoms are isolated points } \\
            \mbox{\ref{LpC0SModMetTopRelProjIngFlat}} (i)
        \end{tabular} & 
        \begin{tabular}{@{}c@{}}
            $\mu$\mbox{ is any } \\
            \mbox{\ref{LpC0SModMetTopRelProjIngFlat}} (i)
        \end{tabular} & 
        \begin{tabular}{@{}c@{}}
            $\mu$\mbox{ is any } \\
            \mbox{\ref{LpC0SModMetTopRelProjIngFlat}} (i)
        \end{tabular} \\
    \hline
        $L_\infty(S,\mu)$ & 
        \begin{tabular}{@{}c@{}} 
            $\mu$ is normal, with \\
            pseudocompact support \\
            \mbox{\ref{LinftyC0SModMetTopRelProjIngFlat}} (i)${}^{*}$
        \end{tabular} & 
        \begin{tabular}{@{}c@{}}
            $\mu$\mbox{ is any } \\
            \mbox{\ref{LinftyC0SModMetTopRelProjIngFlat}} (ii)
        \end{tabular} & 
        \begin{tabular}{@{}c@{}}
            $\mu$\mbox{ is any } \\
            \mbox{\ref{LinftyC0SModMetTopRelProjIngFlat}} (iii)
        \end{tabular} \\
    \hline
        $M(S)$ & 
        \begin{tabular}{@{}c@{}}
            $S$\mbox{ is discrete } \\
            \mbox{\ref{MSC0SModMetTopRelProjIngFlat}} (i)
        \end{tabular} & 
        \begin{tabular}{@{}c@{}}
            $S$\mbox{ is any } \\
            \mbox{\ref{MSC0SModMetTopRelProjIngFlat}} (ii)
        \end{tabular} & 
        \begin{tabular}{@{}c@{}}
            $S$\mbox{ is any } \\
            \mbox{\ref{MSC0SModMetTopRelProjIngFlat}} (iii)
        \end{tabular} \\
    \hline
        $C_0(S)$ & 
        \begin{tabular}{@{}c@{}}
            $S$\mbox{ is paracompact } \\
            \mbox{\ref{C0SC0SModMetTopRelProjInjFlat}} (i)
        \end{tabular} & 
        \begin{tabular}{@{}c@{}} 
            $S=\beta(S\setminus \{s \})$ \\
            for any limit point $s$ \\
            \mbox{\ref{C0SC0SModMetTopRelProjInjFlat}} (ii)${}^{*}$
        \end{tabular} & 
        \begin{tabular}{@{}c@{}}
            $S$\mbox{ is any } \\
            \mbox{\ref{C0SC0SModMetTopRelProjInjFlat}} (iii)
        \end{tabular} \\
    \hline  
        $\mathbb{C}_s$ & 
        \begin{tabular}{@{}c@{}}
            $s$\mbox{ is an isolated point } \\
            \mbox{\ref{OneDimC0SModMetTopRelProjIngFlat}}
        \end{tabular} & 
        \begin{tabular}{@{}c@{}}
            $s$\mbox{ is any } \\
            \mbox{\ref{OneDimC0SModMetTopRelProjIngFlat}}
        \end{tabular} & 
        \begin{tabular}{@{}c@{}}
            $s$\mbox{ is any } \\
            \mbox{\ref{OneDimC0SModMetTopRelProjIngFlat}}
        \end{tabular} \\
    \hline
    \end{longtable}
\end{scriptsize}


%-------------------------------------------------------------------------------
%	Applications to harmonic analysis
%-------------------------------------------------------------------------------

\section{
    Applications to modules of harmonic analysis
}\label{SectionApplicationsToModulesOfHarmonicAnalysis}

%-------------------------------------------------------------------------------
%	Preliminaries on harmonic analysis
%-------------------------------------------------------------------------------

\subsection{
    Preliminaries on harmonic analysis
}\label{SectionPreliminariesOnHarmonicAnalysis} 

Let $G$ be a locally compact group. Its identity we shall denote by $e_G$. By
well known Haar's theorem [\cite{HewRossAbstrHarmAnalVol1},section 15.8] there
exists a unique up to positive constant Borel regular measure $m_G$ which is
finite on all compact sets, positive on all open sets and left translation
invariant, that is $m_G(sE)=m_G(E)$ for all $s\in G$ and $E\in Bor(G)$. It is
called the left Haar measure of group $G$. If $G$ is compact we assume
$m_G(G)=1$. If $G$ is infinite and discrete we choose $m_G$ as counting measure.
For each $s\in G$ the map $m:Bor(G)\to[0,+\infty]:E\mapsto m_G(Es)$ is also a
left Haar measure, so from uniqueness we infer that $m(E)=\Delta_G(s)m_G(E)$ for
some $\Delta_G(s)>0$. The function $\Delta_G:G\to(0,+\infty)$ is called the
modular function of the group $G$. It is clear that
$\Delta_G(st)=\Delta_G(s)\Delta_G(t)$ for all $s,t\in G$. Groups with modular
function equal to one are called unimodular. Examples of groups with unimodular
function include compact groups, commutative groups and discrete groups. In what
follows we use the notation $L_p(G)$ instead of $L_p(G,m_G)$ 
for $1\leq p\leq+\infty$. For a fixed $s\in G$ we define the left shift operator
$L_s:L_1(G)\to L_1(G):f\mapsto(t\mapsto f(s^{-1}t))$ and the right shift
operator $R_s:L_1(G)\to L_1(G):f\mapsto (t\mapsto f(ts))$. 

Group structure of $G$ allows us to introduce the Banach algebra structure on
$L_1(G)$. For a given $f,g\in L_1(G)$ we define their convolution as
$$
(f\convol g)(s)=\int_G f(t)g(t^{-1}s)dm_G(t)=\int_G f(st)g(t^{-1})dm_G(t)
$$
$$=\int_G f(st^{-1})g(t)\Delta_G(t^{-1})dm_G(t)
$$
for almost all $s\in G$. In this case $L_1(G)$ endowed with convolution 
product becomes a Banach algebra. The Banach algebra $L_1(G)$ has a 
contractive two-sided approximate identity consisting of positive compactly 
supported continuous functions. The algebra $L_1(G)$ is unital iff $G$ is 
discrete, and in this case $\delta_{e_G}$ is the identity of $L_1(G)$. The 
group structure of $G$ allows us to turn the Banach space of complex finite 
Borel regular measures $M(G)$ into the Banach algebra too. We define 
convolution of two measures $\mu,\nu\in M(G)$ as
$$
(\mu\convol \nu)(E)=\int_G\nu(s^{-1}E)d\mu(s)=\int_G\mu(Es^{-1})d\nu(s)
$$
for all $E\in Bor(G)$. The Banach space $M(G)$ along with this convolution is 
a unital Banach algebra. The role of identity is played by Dirac delta 
measure $\delta_{e_G}$ supported on $e_G$. In fact $M(G)$ is a coproduct 
in $L_1(G)-\mathbf{mod}_1$ (but not in $M(G)-\mathbf{mod}_1$) of two-sided 
ideal $M_a(G)$ of measures absolutely continuous with respect to $m_G$ and 
subalgebra $M_s(G)$ of measures singular with respect to $m_G$. Note 
that $M_a(G)\isom{M(G)-\mathbf{mod}_1}L_1(G)$ and $M_s(G)$ is an 
annihilator $L_1(G)$-module. Finally, $M(G)=M_a(G)$ iff $G$ is discrete. 

Now we proceed to the discussion of standard left and right modules 
over $L_1(G)$ and $M(G)$. Since $L_1(G)$ can be regarded as two-sided ideal 
of $M(G)$ because of isometric left and 
right $M(G)$-morphism $i:L_1(G)\to M(G):f\mapsto f m_G$ it is enough to define 
module structure over $M(G)$. For $1\leq p<+\infty$ and 
any $f\in L_p(G)$, $\mu\in M(G)$ we define
$$
(\mu\convol_p f)(s)=\int_G f(t^{-1}s)d\mu(t), \qquad\qquad (f\convol_p
\mu)(s)=\int_G f(st^{-1}){\Delta_G(t^{-1})}^{1/p}d\mu(t)
$$
These module actions turn any Banach space $L_p(G)$ for $1\leq p<+\infty$ into 
the left and right $M(G)$-module. Note that for $p=1$ and $\mu\in M_a(G)$ we 
get the usual definition of convolution. For $1<p\leq +\infty$ and 
any $f\in L_p(G)$, $\mu\in M(G)$ we define module actions
$$
(\mu\cdot_p f)(s)=\int_G {\Delta_G(t)}^{1/p}f(st)d\mu(t), \qquad\qquad (f\cdot_p
\mu)(s)=\int_G f(ts)d\mu(t)
$$
These module actions turn any Banach space $L_p(G)$ for $1<p\leq+\infty$ into 
the left and right $M(G)$-module too. This special choice of module structure 
nicely interacts with duality. Indeed we have 
and ${(L_p(G),\convol_p)}^*\isom{\mathbf{mod}_1-M(G)}(L_{p^*}(G),\cdot_{p^*})$ 
for all $1\leq p<+\infty$. Finally, the Banach space $C_0(G)$ also becomes left 
and right $M(G)$-module when endowed with $\cdot_\infty$ in the role of module 
action. Even more, $C_0(G)$ is a closed left and right $M(G)$-submodule 
of $L_\infty(G)$ 
and ${(C_0(G),\cdot_\infty)}^*\isom{M(G)-\mathbf{mod}_1}(M(G),\convol)$.

A character on a locally compact group $G$ is by definition a continuous 
homomorphism from $G$ to $\mathbb{T}$. The set of characters on $G$ forms a 
group denoted by $\widehat{G}$. It becomes a locally compact group when 
considered with compact open topology. Any character $\gamma\in\widehat{G}$ 
gives rise to the continuous character 
$\varkappa_\gamma^L
:L_1(G)\to\mathbb{C}
:f\mapsto \int_G f(s)\overline{\gamma(s)}dm_G(s)$ on $L_1(G)$. In fact all 
characters of $L_1(G)$ arise this way. This result is due to 
Gelfand [\cite{KaniBanAlg}, theorems 2.7.2, 2.7.5]. Similarly, for 
each $\gamma\in\widehat{G}$ we have a character on $M(G)$ defined by 
$\varkappa_\gamma^M
:M(G)\to\mathbb{C}
:\mu\mapsto\int_{G} \overline{\gamma(s)}d\mu(s)$. By $\mathbb{C}_\gamma$ we 
denote the respective augmentation left and right $L_1(G)$- or $M(G)$-module. 
Their module actions are defined by
$$
f\cdot_{\gamma}z=z\cdot_{\gamma}f=\varkappa_\gamma^L(f)z \qquad\qquad
\mu\cdot_{\gamma}z=z\cdot_{\gamma}\mu=\varkappa_\gamma^M(\mu)z
$$
for all $f\in L_1(G)$, $\mu\in M(G)$ and $z\in\mathbb{C}$. 

One of the numerous definitions of amenable group says, that a locally compact 
group $G$ is amenable if there exists an $L_1(G)$-morphism of right 
modules $M:L_\infty(G)\to\mathbb{C}_{e_{\widehat{G}}}$ such 
that $M(\chi_G)=1$ [\cite{HelBanLocConvAlg}, section VII.2.5]. We can even 
assume that $M$ is contractive [\cite{HelBanLocConvAlg}, remark 7.1.54].

Most of results of this section that not supported with references are 
presented in a full detail in [\cite{DalBanAlgAutCont}, section 3.3].

%-------------------------------------------------------------------------------
%	L_1(G)-modules
%-------------------------------------------------------------------------------

\subsection{
    \texorpdfstring{$L_1(G)$}{L1(G)}-modules
}\label{SubSectionL1GModules}

Metric homological properties of most of the standard $L_1(G)$-modules of 
harmonic analysis are studied in~\cite{GravInjProjBanMod}. We borrow these 
ideas to unify approaches to metrical and topological homological properties 
of modules over group algebras.

\begin{proposition}\label{LInfIsL1MetrInj} Let $G$ be a locally compact group. 
Then $L_1(G)$ is metrically and topologically flat $L_1(G)$-module, 
i.e. $L_1(G)$-module $L_\infty(G)$ is metrically and topologically injective.
\end{proposition} 
\begin{proof} Since $L_1(G)$ has contractive approximate identity, 
then $L_1(G)$ is metrically and topologically flat $L_1(G)$-module 
by proposition~\ref{MetTopFlatIdealsInUnitalAlg}. 
Since $L_\infty(G)\isom{\mathbf{mod}_1-L_1(G)}{L_1(G)}^*$, then by 
proposition~\ref{MetCTopFlatCharac} it is metrically and topologically injective.
\end{proof}

\begin{proposition}\label{OneDimL1ModMetTopProjCharac} Let $G$ be a locally 
compact group, and $\gamma\in\widehat{G}$. Then the following are equivalent:

\begin{enumerate}[label = (\roman*)]
    \item $G$ is compact;

    \item $\mathbb{C}_\gamma$ is metrically projective $L_1(G)$-module;

    \item $\mathbb{C}_\gamma$ is topologically projective $L_1(G)$-module.
\end{enumerate}
\end{proposition}
\begin{proof} $(i)\implies (ii)$ Consider $L_1(G)$-morphisms 
$\sigma^+:\mathbb{C}_\gamma\to {L_1(G)}_+:z\mapsto z\gamma \oplus_1 0$ 
and $\pi^+:{L_1(G)}_+\to\mathbb{C}_\gamma: f\oplus_1 w\to f\cdot_{\gamma}1+w$. 
One can easily check 
that $\Vert\pi^+\Vert=\Vert\sigma^+\Vert=1$ 
and $\pi^+\sigma^+=1_{\mathbb{C}_\gamma}$. Therefore $\mathbb{C}_\gamma$ is a 
retract of ${L_1(G)}_+$ in $L_1(G)-\mathbf{mod}_1$. From 
propositions~\ref{UnitalAlgIsMetTopProj} 
and~\ref{RetrMetCTopProjIsMetCTopProj} it follows 
that $\mathbb{C}_\gamma$ is metrically projective.

$(ii)\implies (iii)$ See
proposition~\ref{MetProjIsTopProjAndTopProjIsRelProj}.

$(iii)\implies (i)$ Consider $L_1(G)$-morphism
$\pi:L_1(G)\to\mathbb{C}_\gamma:f\mapsto f\cdot_{\gamma} 1$. It is easy to see
that $\pi$ is strictly coisometric. Since $\mathbb{C}_\gamma$ is topologically
projective, then there exists an $L_1(G)$-morphism $\sigma:\mathbb{C}_\gamma\to
L_1(G)$ such that $\pi\sigma=1_{\mathbb{C}_\gamma}$. Let $f=\sigma(1)\in L_1(G)$
and ${(e_\nu)}_{\nu\in N}$ be a standard approximate identity of $L_1(G)$. Since
$\sigma$ is an $L_1(G)$-morphism, then for all $s,t\in G$ we have 
$$
f(s^{-1}t)
=L_s(f)(t)
=\lim_\nu L_s(e_\nu\convol \sigma(1))(t)
=\lim_\nu((\delta_s\convol e_\nu)\convol \sigma(1))(t)
=\lim_\nu\sigma((\delta_s\convol e_\nu)\cdot_{\gamma} 1)(t)
$$
$$
=\lim_\nu\sigma(\varkappa_\gamma^L(\delta_s\convol e_\nu))(t)
=\lim_\nu\varkappa_\gamma^L(\delta_s\convol e_\nu)\sigma(1)(t)
=\lim_\nu(e_\nu\convol\gamma)(s^{-1})f(t)
=\gamma(s^{-1})f(t).
$$
Therefore, for the function $g(t):=\gamma(t^{-1})f(t)$ in $L_1(G)$ we have
$g(st)=g(t)$ for all $s,t\in G$. Thus $g$ is a constant function in $L_1(G)$,
which is possible only for compact group $G$.
\end{proof}

\begin{proposition}\label{OneDimL1ModMetTopInjFlatCharac} Let $G$ be a locally
compact group, and $\gamma\in\widehat{G}$. Then the following are equivalent:

\begin{enumerate}[label = (\roman*)]
    \item $G$ is amenable;

    \item $\mathbb{C}_\gamma$ is metrically injective $L_1(G)$-module;

    \item $\mathbb{C}_\gamma$ is topologically injective $L_1(G)$-module.

    \item $\mathbb{C}_\gamma$ is metrically flat $L_1(G)$-module;

    \item $\mathbb{C}_\gamma$ is topologically flat $L_1(G)$-module.
\end{enumerate}
\end{proposition}
\begin{proof} $(i)\implies (ii)$ Since $G$ is amenable, then we have
contractive $L_1(G)$-morphism $M:L_\infty(G)\to\mathbb{C}_{e_{\widehat{G}}}$
with $M(\chi_G)=1$. Consider linear 
operators $\rho:\mathbb{C}_\gamma\to L_\infty(G):z\mapsto z\overline{\gamma}$ 
and $\tau:L_\infty(G)\to\mathbb{C}_\gamma:f\mapsto M(f\gamma)$. These are
$L_1(G)$-morphisms of right $L_1(G)$-modules. We shall check this for operator
$\tau$: for all $f\in L_\infty(G)$ and $g\in L_1(G)$ we have
$$
\tau(f\cdot_\infty g)
=M((f\cdot_\infty g)\gamma)
=M(f\gamma\cdot_\infty g\overline{\gamma})
=M(f\gamma)\cdot_{e_{\widehat{G}}} g\overline{\gamma}
=M(f\gamma)\varkappa_\gamma^L(g)
=\tau(f)\cdot_{\gamma} g.
$$  
It is easy to check that $\rho$ and $\tau$ are contractive and
$\tau\rho=1_{\mathbb{C}_\gamma}$. Therefore $\mathbb{C}_\gamma$ is a retract of
$L_\infty(G)$ in $\mathbf{mod}_1-L_1(G)$. From
propositions~\ref{LInfIsL1MetrInj} and~\ref{RetrMetCTopInjIsMetCTopInj} it 
follows that $\mathbb{C}_\gamma$ is metrically injective as $L_1(G)$-module.

$(ii)\implies (iii)$ See proposition~\ref{MetInjIsTopInjAndTopInjIsRelInj}.

$(iii) \implies (i)$ Since $\rho$ is an isometric $L_1(G)$-morphism of right
$L_1(G)$-modules and $\mathbb{C}_\gamma$ is topologically injective as
$L_1(G)$-module, then $\rho$ is a coretraction in $\mathbf{mod}-L_1(G)$. Denote
its left inverse morphism by $\pi$, then
$\pi(\overline{\gamma})=\pi(\rho(1))=1$. Consider bounded linear functional
$M:L_\infty(G)\to\mathbb{C}_\gamma:f\mapsto \pi(f\overline{\gamma})$. For all
$f\in L_\infty(G)$ and $g\in L_1(G)$ we have
$$
M(f\cdot_\infty g)
=\pi((f\cdot_\infty g)\overline{\gamma})
=\pi(f\overline{\gamma}\cdot_\infty g\gamma)
=\pi(f\overline{\gamma})\cdot_{\gamma} g\gamma
=M(f)\varkappa_\gamma^L(g\gamma)
=M(f)\cdot_{e_{\widehat{G}}}g.
$$
Therefore $M$ is an $L_1(G)$-morphism, but we also have
$M(\chi_G)=\pi(\overline{\gamma})=1$. Therefore $G$ is amenable.

$(ii) \Longleftrightarrow (iv)$, $(iii) \Longleftrightarrow (v)$ Note that
$\mathbb{C}_\gamma^*\isom{\mathbf{mod}_1-L_1(G)}\mathbb{C}_\gamma$, so all
equivalences  follow from three previous paragraphs and
proposition~\ref{MetCTopFlatCharac}.
\end{proof}

In the next proposition we shall study specific ideals of Banach algebra
$L_1(G)$. They are of the form $L_1(G)\convol\mu$ for some idempotent measure
$\mu$. In fact, this class of ideals in case of commutative compact groups $G$
coincides with those left ideals of $L_1(G)$ that admit a right bounded
approximate identity.

\begin{proposition}\label{CommIdealByIdemMeasL1MetTopProjCharac} Let $G$ be a
locally compact group and  $\mu\in M(G)$ be an idempotent measure, that is
$\mu\convol\mu=\mu$. If the left ideal $I=L_1(G)\convol\mu$ of Banach algebra
$L_1(G)$ is topologically projective $L_1(G)$-module, then $\mu=p m_G$, for some
$p\in I$.
\end{proposition}
\begin{proof} Let $\phi:I\to L_1(G)$ be arbitrary morphism of left
$L_1(G)$-modules. Consider $L_1(G)$-morphism $\phi':L_1(G)\to
L_1(G):x\mapsto\phi(x\convol\mu)$. By Wendel's theorem [\cite{WendLeftCentrzrs},
theorem 1], there exists a measure $\nu\in M(G)$ such that
$\phi'(x)=x\convol\nu$ for all $x\in L_1(G)$. In particular,
$\phi(x)=\phi(x\convol\mu)=\phi'(x)=x\convol\nu$ for all $x\in I$. It is now
clear that $\psi:I\to I:x\mapsto\nu\convol x$ is a morphism of right $I$-modules
satisfying $\phi(x)y=x\psi(y)$ for all $x,y\in I$. By paragraph $(ii)$ of
lemma~\ref{GoodIdealMetTopProjIsUnital} the ideal $I$ has a right identity, say
$e\in I$. Then $x\convol\mu=x\convol\mu\convol e$ for all $x\in L_1(G)$. Two
measures are equal if their convolutions with all functions of $L_1(G)$ coincide
[\cite{DalBanAlgAutCont}, corollary 3.3.24], so $\mu=\mu\convol e m_G$. Since
$e\in I\subset L_1(G)$, then $\mu=\mu\convol e m_G\in M_a(G)$. 
Set $p=\mu\convol e\in I$, then $\mu=p m_G$.
\end{proof}

We conjecture that the left ideal $L_1(G)\convol \mu$ for idempotent measure
$\mu$ is metrically projective $L_1(G)$-module iff $\mu=p m_G$ where $p\in I$
and $\Vert p\Vert=1$.

\begin{theorem}\label{L1ModL1MetTopProjCharac} Let $G$ be a locally compact
group. Then the following are equivalent:

\begin{enumerate}[label = (\roman*)]
    \item $G$ is discrete;

    \item $L_1(G)$ is metrically projective $L_1(G)$-module;

    \item $L_1(G)$ is topologically projective $L_1(G)$-module.
\end{enumerate}
\end{theorem}
\begin{proof} $(i)\implies (ii)$ If $G$ is discrete, then $L_1(G)$ is unital
with unit of norm $1$. By  proposition~\ref{UnIdeallIsMetTopProj} we see that
$L_1(G)$ is metrically projective as $L_1(G)$-module.

$(ii)\implies (iii)$ See
proposition~\ref{MetProjIsTopProjAndTopProjIsRelProj}.

$(iii) \implies (i)$ Clearly, $\delta_{e_G}$ is an idempotent measure. Since
$L_1(G)=L_1(G)\convol \delta_{e_G}$ is topologically projective, then by
proposition~\ref{CommIdealByIdemMeasL1MetTopProjCharac} 
we have $\delta_{e_G}=f m_G$ for some $f\in L_1(G)$. This is possible only 
if $G$ is discrete.
\end{proof}

Note that $L_1(G)$-module $L_1(G)$ is relatively projective for any locally
compact group $G$ [\cite{HelBanLocConvAlg}, exercise 7.1.17].

\begin{proposition}\label{L1MetTopProjAndMetrFlatOfMeasAlg} Let $G$ be 
a locally compact group. Then the following are equivalent:

\begin{enumerate}[label = (\roman*)]
    \item $G$ is discrete;

    \item $M(G)$ is metrically projective $L_1(G)$-module;

    \item $M(G)$ is topologically projective $L_1(G)$-module;

    \item $M(G)$ is metrically flat $L_1(G)$-module.
\end{enumerate}
\end{proposition}
\begin{proof} 
$(i)\implies (ii)$ We have $M(G)\isom{L_1(G)-\mathbf{mod}_1}L_1(G)$ for
discrete $G$, so the result follows from theorem~\ref{L1ModL1MetTopProjCharac}. 

$(ii)\implies (iii)$ See
proposition~\ref{MetProjIsTopProjAndTopProjIsRelProj}.

$(ii)\implies (iv)$ See proposition~\ref{MetTopProjIsMetTopFlat}.

$(iii)\implies (i)$ Note that $M(G)\isom{L_1(G)-\mathbf{mod}_1}
L_1(G)\bigoplus_1 M_s(G)$, so $M_s(G)$ is topologically projective by
proposition~\ref{MetTopProjModCoprod}. Note that $M_s(G)$ is an annihilator
$L_1(G)$-module, then by proposition~\ref{MetTopProjOfAnnihModCharac} the
algebra $L_1(G)$ has a right identity. Recall that $L_1(G)$ also has a two-sided
bounded approximate identity, so $L_1(G)$ is unital. The last is equivalent to
$G$ being discrete.

$(iv)\implies (i)$ Note that $M(G)\isom{L_1(G)-\mathbf{mod}_1}
L_1(G)\bigoplus_1 M_s(G)$, so $M_s(G)$ is metrically flat by
proposition~\ref{MetTopFlatModCoProd}. Note that $M_s(G)$ is an annihilator
$L_1(G)$-module, then by proposition~\ref{MetTopFlatAnnihModCharac} it is equal
to zero. The last is equivalent to $G$ being discrete.
\end{proof}

\begin{proposition}\label{MeasAlgIsL1TopFlat} Let $G$ be a locally compact
group. Then $M(G)$ is topologically flat $L_1(G)$-module.
\end{proposition}
\begin{proof} Since $M(G)$ is an $L_1$-space it is a fortiori an
$\mathscr{L}_1^g$-space. Since $M_s(G)$ is complemented in $M(G)$, then $M_s(G)$
is an $\mathscr{L}_1^g$-space too [\cite{DefFloTensNorOpId}, corollary
23.2.1(2)]. Since $M_s(G)$ is an annihilator $L_1(G)$-module, then from
proposition~\ref{MetTopFlatAnnihModCharac} we have that $M_s(G)$ is
topologically flat $L_1(G)$-module. The $L_1(G)$-module $L_1(G)$ is also
topologically flat by proposition~\ref{LInfIsL1MetrInj}. Since
$M(G)\isom{L_1(G)-\mathbf{mod}_1}L_1(G)\bigoplus_1 M_s(G)$, then $M(G)$ is
topologically flat $L_1(G)$-module by proposition~\ref{MetTopFlatModCoProd}.
\end{proof}

%-------------------------------------------------------------------------------
%	M(G)-modules
%-------------------------------------------------------------------------------

\subsection{
    \texorpdfstring{$M(G)$}{M (G)}-modules
}\label{SubSectionMGModules}

We turn to the study of standard $M(G)$-modules of harmonic analysis. As we
shall see most of results can be derived from results on $L_1(G)$-modules.

\begin{proposition}\label{MGMetTopProjInjFlatRedToL1} Let $G$ be a locally
compact group, and $X$ be $\langle$~essential / faithful / essential~$\rangle$
$L_1(G)$-module. Then

\begin{enumerate}[label = (\roman*)]
    \item $X$ is metrically $\langle$~projective / injective / flat~$\rangle$
    $M(G)$-module iff it is metrically $\langle$~projective / injective /
    flat~$\rangle$ $L_1(G)$-module;

    \item $X$ is topologically $\langle$~projective / injective / flat~$\rangle$
    $M(G)$-module iff it is topologically $\langle$~projective / injective /
    flat~$\rangle$ $L_1(G)$-module.
\end{enumerate}
\end{proposition}
\begin{proof} Recall that $L_1(G)\isom{L_1(G)-\mathbf{mod}_1}M_a(G)$ is a
two-sided complemented in $\mathbf{Ban}_1$ ideal of $M(G)$. Now $(i)$ and $(ii)$
follow from proposition $\langle$~\ref{MetTopProjUnderChangeOfAlg}
/~\ref{MetTopInjUnderChangeOfAlg}  /~\ref{MetTopFlatUnderChangeOfAlg}~$\rangle$.
\end{proof} 

It is worth to mention here that the $L_1(G)$-modules $C_0(G)$, $L_p(G)$ for
$1\leq p<\infty$ and $\mathbb{C}_\gamma$ for $\gamma\in\widehat{G}$ are
essential and $L_1(G)$-modules $C_0(G)$, $M(G)$, $L_p(G)$ for $1\leq p\leq
\infty$ and $\mathbb{C}_\gamma$ for $\gamma\in\widehat{G}$ are faithful. 

\begin{proposition}\label{MGModMGMetTopProjFlatCharac} Let $G$ be a locally
compact group. Then $M(G)$ is metrically and topologically projective
$M(G)$-module. As the consequence it is metrically and topologically flat
$M(G)$-module.
\end{proposition} 
\begin{proof} Since $M(G)$ is a unital algebra, the metric and topological
projectivity of $M(G)$ follow from proposition~\ref{UnitalAlgIsMetTopProj}. It
remains to apply proposition~\ref{MetTopProjIsMetTopFlat}.
\end{proof}

%-------------------------------------------------------------------------------
%	Banach geometric restrictions
%-------------------------------------------------------------------------------

\subsection{
    Banach geometric restriction
}\label{SubSectionBanachGeometricRestriction}

In this section we shall show that many modules of harmonic analysis are fail to
be metrically or topologically projective, injective or flat for purely Banach
geometric reasons. 

\begin{proposition}\label{StdModAreNotRetrOfL1LInf} Let $G$ be an infinite
locally compact group. Then

\begin{enumerate}[label = (\roman*)]
    \item $L_1(G)$, $C_0(G)$, $M(G)$, ${L_\infty(G)}^*$ are not 
    topologically injective Banach spaces;

    \item $C_0(G)$, $L_\infty(G)$ are not complemented in any $L_1$-space.
\end{enumerate}
\end{proposition}
\begin{proof}
Since $G$ is infinite all modules in question are infinite dimensional.

$(i)$ If an infinite dimensional Banach space is topologically injective, then it
contains a copy of $\ell_\infty(\mathbb{N})$ [\cite{RosOnRelDisjFamOfMeas},
corollary 1.1.4], and consequently a copy of $c_0(\mathbb{N})$. The Banach space
$L_1(G)$ is weakly sequentially complete [\cite{WojBanSpForAnalysts}, corollary
III.C.14], so by corollary 5.2.11 in~\cite{KalAlbTopicsBanSpTh} it can't contain
a copy of $c_0(\mathbb{N})$. Therefore, $L_1(G)$ is not topologically injective
Banach space.  If $M(G)$ is topologically injective Banach space, then so does
its complemented subspace $M_a(G)\isom{\mathbf{Ban}_1}L_1(G)$. By previous
argument this is impossible. So $M(G)$ is not topologically injective as Banach
space. By corollary 3 of~\cite{LauMingComplSubspInLInfOfG} the space $C_0(G)$ is
not complemented in $L_\infty(G)$. Then $C_0(G)$ can't be topologically
injective either. The Banach space $L_1(G)$ is complemented in
${L_\infty(G)}^*\isom{\mathbf{Ban}_1}{L_1(G)}^{**}$ [\cite{DefFloTensNorOpId},
proposition  B10]. Therefore if ${L_\infty(G)}^*$ is topologically injective as
Banach space, then so does its retract $L_1(G)$. By previous arguments this is
impossible, so ${L_\infty(G)}^*$ is not topologically injective Banach space.

$(ii)$ If $C_0(G)$ is a retract of $L_1$-space, then
$M(G)\isom{\mathbf{Ban}_1}{C_0(G)}^*$ is a retract of $L_\infty$-space, 
so it must be a topologically injective Banach space. This contradicts 
paragraph $(i)$, so $C_0(G)$ is not a retract of $L_1$-space. 
Note that $\ell_\infty(\mathbb{N})$ embeds in $L_\infty(G)$, then so 
does $c_0(\mathbb{N})$. So if $L_\infty(G)$ is
a retract of $L_1$-space, then there would exist an $L_1$-space containing a
copy of $c_0(\mathbb{N})$. This is impossible as already showed in paragraph
$(i)$.
\end{proof}

From now on by $A$ we denote either $L_1(G)$ or $M(G)$. Recall that $L_1(G)$ and
$M(G)$ are both $L_1$-spaces.

\begin{proposition}\label{StdModAreNotL1MGMetTopProjInjFlat} Let $G$ be an
infinite locally compact group. Then

\begin{enumerate}[label = (\roman*)]
    \item $C_0(G)$, $L_\infty(G)$ are neither topologically nor metrically 
    projective $A$-modules;

    \item $L_1(G)$, $C_0(G)$, $M(G)$, ${L_\infty(G)}^*$ are neither 
    topologically nor metrically injective $A$-modules;

    \item $L_\infty(G)$, $C_0(G)$ are neither topologically nor metrically flat
    $A$-modules.
\end{enumerate}
\end{proposition}

$(iv)$ $L_p(G)$ for $1<p<\infty$ are neither topologically nor metrically
projective, injective or flat $A$-flat.

\begin{proof} $(i)$ The result follows from
propositions~\ref{TopProjInjFlatModOverL1Charac} paragraph $(i)$ and
~\ref{StdModAreNotRetrOfL1LInf} paragraph $(ii)$.

$(ii)$ The result follows from propositions~\ref{TopProjInjFlatModOverL1Charac}
paragraph $(ii)$ and~\ref{StdModAreNotRetrOfL1LInf}.

$(iii)$ Note that ${C_0(G)}^*\isom{\mathbf{mod}_1-A}M(G)$. Now the result 
follows from paragraph $(i)$ and proposition~\ref{MetCTopFlatCharac}.

$(iv)$ Since $L_p(G)$ is reflexive for $1<p<\infty$ the result follows
from~\ref{NoInfDimRefMetTopProjInjFlatModOverMthscrL1OrLInfty}.
\end{proof}

It remains to consider metric and topological homological properties of
$A$-modules when $G$ is finite.

\begin{proposition}\label{LpFinGrL1MGMetrInjProjCharac} Let $G$ be a non trivial
finite group and $1\leq p\leq \infty$. Then the $A$-module $L_p(G)$ is
metrically $\langle$~projective / injective~$\rangle$ iff $\langle$~$p=1$ /
$p=\infty$~$\rangle$
\end{proposition}
\begin{proof} 
Assume $L_p(G)$ is metrically $\langle$~projective / injective~$\rangle$ as
$A$-module. Since $L_p(G)$ is finite dimensional, then by paragraphs $(i)$ and
$(ii)$ of proposition~\ref{TopProjInjFlatModOverL1Charac} we have 
identifications $\langle$~$L_p(G)\isom{\mathbf{Ban}_1}\ell_1(\mathbb{N}_n)$ /
$L_p(G)
\isom{\mathbf{Ban}_1}
C(\mathbb{N}_n)
\isom{\mathbf{Ban}_1}
\ell_\infty(\mathbb{N}_n)$~$\rangle$, 
where $n=\operatorname{Card}(G)>1$. Now we use the result of theorem
1 from~\cite{LyubIsomEmdbFinDimLp} for Banach spaces over field $\mathbb{C}$: if
for $2\leq m\leq k$ and $1\leq r,s\leq \infty$, there exists an isometric
embedding from $\ell_r(\mathbb{N}_m)$ into $\ell_s(\mathbb{N}_k)$, then either
$r=2$, $s\in 2\mathbb{N}$ or $r=s$. Therefore $\langle$~$p=1$ /
$p=\infty$~$\rangle$. The converse easily follows from
$\langle$~theorem~\ref{L1ModL1MetTopProjCharac} /
proposition~\ref{LInfIsL1MetrInj}~$\rangle$
\end{proof}

\begin{proposition}\label{StdModFinGrL1MGMetrInjProjFlatCharac} Let $G$ be a
finite group. Then

\begin{enumerate}[label = (\roman*)]
    \item $C_0(G)$, $L_\infty(G)$ are metrically injective $A$-modules;

    \item $C_0(G)$, $L_p(G)$ for $1<p\leq\infty$ are metrically projective
    $A$-modules iff $G$ is trivial;

    \item $M(G)$, $L_p(G)$ for $1\leq p<\infty$ are metrically injective
    $A$-modules iff $G$ is trivial;

    \item $C_0(G)$, $L_p(G)$ for $1<p\leq\infty$ are metrically 
    flat $A$-modules iff $G$ is trivial.
\end{enumerate}
\end{proposition}
\begin{proof}
$(i)$ Since $G$ is finite then $C_0(G)=L_\infty(G)$. The result follows from
proposition~\ref{LInfIsL1MetrInj}.

$(ii)$ If $G$ is trivial, that is $G= \{e_G \}$, then $L_p(G)=C_0(G)=L_1(G)$ and
the result follows from paragraph $(i)$. If $G$ is non trivial, then we recall
that $C_0(G)=L_\infty(G)$ and use
proposition~\ref{LpFinGrL1MGMetrInjProjCharac}.

$(iii)$ If $G= \{e_G \}$, then $M(G)=L_p(G)=L_\infty(G)$ and the result follows
from paragraph $(i)$. If $G$ is non trivial, then we note that $M(G)=L_1(G)$ and
use proposition~\ref{LpFinGrL1MGMetrInjProjCharac}.

$(iv)$ From paragraph $(iii)$ it follows that $L_p(G)$ for $1\leq p<\infty$ is
metrically injective $A$-module iff $G$ is trivial. Now the result follows from
proposition~\ref{MetCTopFlatCharac} and the facts that
${C_0(G)}^*\isom{\mathbf{mod}_1-L_1(G)}M(G)\isom{\mathbf{mod}_1-L_1(G)}L_1(G)$,
${L_p(G)}^*\isom{\mathbf{mod}_1-L_1(G)}L_{p^*}(G)$ for $1\leq p^*<\infty$.
\end{proof}

It is worth to mention here that if we would consider all Banach spaces over the
field of real numbers, then $L_\infty(G)$ and $L_1(G)$ would be metrically
projective and injective respectively,  additionally for $G$ consisting of two
elements, because
$$
L_\infty(\mathbb{Z}_2)
\isom{L_1(\mathbb{Z}_2)-\mathbf{mod}_1}
\mathbb{R}_{\gamma_0}\bigoplus\nolimits_1\mathbb{R}_{\gamma_1},
\qquad
L_1(\mathbb{Z}_2)
\isom{L_1(\mathbb{Z}_2)-\mathbf{mod}_1}
\mathbb{R}_{\gamma_0}\bigoplus\nolimits_\infty\mathbb{R}_{\gamma_1}
$$
for $\gamma_0,\gamma_1\in\widehat{\mathbb{Z}_2}$ defined by
$\gamma_0(0)=\gamma_0(1)=\gamma_1(0)=-\gamma_1(1)=1$. Here $\mathbb{Z}_2$
denotes the unique group of two elements.

\begin{proposition}\label{StdModFinGrL1MGTopInjProjFlatCharac} Let $G$ be a
finite group. Then the $A$-modules $C_0(G)$, $M(G)$, $L_p(G)$ 
for $1\leq p\leq \infty$ are both topologically projective, injective and flat.
\end{proposition} 
\begin{proof}
For finite group $G$ we have $M(G)=L_1(G)$ and $C_0(G)=L_\infty(G)$, so these
modules do not require special considerations. Since $M(G)=L_1(G)$, we can
restrict our considerations to the case $A=L_1(G)$. The identity map
$i:L_1(G)\to L_p(G):f\mapsto f$ is a topological isomorphism of Banach spaces,
because $L_1(G)$ and $L_p(G)$ for $1\leq p<+\infty$ are of equal finite
dimension. Since $G$ is finite, it is unimodular. Therefore, the module actions
in $(L_1(G),\convol)$ and $(L_p(G),\convol_p)$ coincide for $1\leq p<+\infty$
and $i$ is an isomorphism in $L_1(G)-\mathbf{mod}$ and $\mathbf{mod}-L_1(G)$.
Similarly one can show that $(L_\infty(G),\cdot_\infty)$ and $(L_p(G),\cdot_p)$
for $1<p\leq+\infty$ are isomorphic in $L_1(G)-\mathbf{mod}$ and
$\mathbf{mod}-L_1(G)$. Finally, one can easily check that $(L_1(G),\convol)$ and
$(L_\infty(G),\cdot_\infty)$ are isomorphic in $L_1(G)-\mathbf{mod}$ and
$\mathbf{mod}-L_1(G)$ via the 
map $j:L_1(G)\to L_\infty(G):f\mapsto(s\mapsto f(s^{-1}))$. Therefore all 
the discussed modules are isomorphic. It remains to
recall that $L_1(G)$ is topologically projective and flat by
theorem~\ref{L1ModL1MetTopProjCharac} and proposition~\ref{LInfIsL1MetrInj},
while $L_\infty(G)$ is topologically injective by
proposition~\ref{LInfIsL1MetrInj}.
\end{proof}

Now we can summarize results on homological properties of modules of harmonic
analysis into three tables. Each cell of the table contains a condition under
which the respective module has respective property and propositions where this
is proved. We shall mention that results for modules $L_p(G)$ are valid for both
module actions $\convol_p$ and $\cdot_p$. Characterization and proofs for
homologically trivial modules $\mathbb{C}_\gamma$ in case of relative theory is
the same as in
propositions~\ref{OneDimL1ModMetTopProjCharac},
~\ref{OneDimL1ModMetTopInjFlatCharac}
and~\ref{OneDimL1ModMetTopInjFlatCharac}. As usually, we use ${}^{*}$
indicates that only a necessary conditions is known. As we showed above even
topological theory is too restrictive for $L_1(G)$ to be projective as
$L_1(G)$-module. Similarly a Banach space is topologically projective iff it is
an $L_1$-space, and the underlying measure space is atomic. This analogy
confirms important role of Banach geometry in metric and topological Banach
homology.

\begin{scriptsize}
    \begin{longtable}{|c|c|c|c|c|c|c|} 
    \multicolumn{7}{c}{
        \mbox{
            Homologically trivial $L_1(G)$- and $M(G)$-modules in metric theory
        }
    }                                                                                                                                                                                                                                                                                                                                                                                                                                                                                                                                                                                                                                                                                                                                                                                                                                                                                                                                             \\
    \hline & 
        \multicolumn{3}{c|}{
            $L_1(G)$-modules
        } & 
        \multicolumn{3}{c|}{
            $M(G)$-modules
        } \\
    \hline & 
        \mbox{Projectivity} & 
        \mbox{Injectivity} & 
        \mbox{Flatness} & 
        \mbox{Projectivity} &
        \mbox{Injectivity} & 
        \mbox{Flatness} \\ 
    \hline
        $L_1(G)$ & 
        \begin{tabular}{@{}c@{}}
            $G$\mbox{ is discrete } \\
            \mbox{\ref{L1ModL1MetTopProjCharac}}
        \end{tabular} & 
        \begin{tabular}{@{}c@{}}
            $G= \{e_G \}$ \\
            \mbox{\ref{StdModAreNotL1MGMetTopProjInjFlat}},
            \mbox{\ref{StdModFinGrL1MGMetrInjProjFlatCharac}}
        \end{tabular} & 
        \begin{tabular}{@{}c@{}}
            $G$\mbox{ is any } \\
            \mbox{\ref{LInfIsL1MetrInj}}
        \end{tabular} & 
        \begin{tabular}{@{}c@{}}
            $G$\mbox{ is discrete } \\
            \mbox{\ref{L1ModL1MetTopProjCharac}},
            \mbox{\ref{MGMetTopProjInjFlatRedToL1}}
        \end{tabular} & 
        \begin{tabular}{@{}c@{}}
            $G= \{e_G \}$ \\
            \mbox{\ref{StdModAreNotL1MGMetTopProjInjFlat}},
            \mbox{\ref{StdModFinGrL1MGMetrInjProjFlatCharac}}
        \end{tabular} & 
        \begin{tabular}{@{}c@{}}
            $G$\mbox{ is any } \\
            \mbox{\ref{LInfIsL1MetrInj}},
            \mbox{\ref{MGMetTopProjInjFlatRedToL1}}
        \end{tabular} \\
    \hline 
        $L_p(G)$ & 
        \begin{tabular}{@{}c@{}}
            $G= \{e_G \}$ \\
            \mbox{\ref{StdModAreNotL1MGMetTopProjInjFlat}},
            \mbox{\ref{LpFinGrL1MGMetrInjProjCharac}}
        \end{tabular} &
        \begin{tabular}{@{}c@{}}
            $G= \{e_G \}$ \\
            \mbox{\ref{StdModAreNotL1MGMetTopProjInjFlat}},
            \mbox{\ref{LpFinGrL1MGMetrInjProjCharac}}
        \end{tabular} & 
        \begin{tabular}{@{}c@{}}
            $G= \{e_G \}$ \\
            \mbox{\ref{StdModAreNotL1MGMetTopProjInjFlat}},
            \mbox{\ref{StdModFinGrL1MGMetrInjProjFlatCharac}}
        \end{tabular} & 
        \begin{tabular}{@{}c@{}}
            $G= \{e_G \}$ \\
            \mbox{\ref{StdModAreNotL1MGMetTopProjInjFlat}},
            \mbox{\ref{LpFinGrL1MGMetrInjProjCharac}}
        \end{tabular} & 
        \begin{tabular}{@{}c@{}}
            $G= \{e_G \}$ \\
            \mbox{\ref{StdModAreNotL1MGMetTopProjInjFlat}},
            \mbox{\ref{LpFinGrL1MGMetrInjProjCharac}}
        \end{tabular} & 
        \begin{tabular}{@{}c@{}}
            $G= \{e_G \}$ \\
            \mbox{\ref{StdModAreNotL1MGMetTopProjInjFlat}},
            \mbox{\ref{StdModFinGrL1MGMetrInjProjFlatCharac}}
        \end{tabular} \\
    \hline
        $L_\infty(G)$ & 
        \begin{tabular}{@{}c@{}}
            $G= \{e_G \}$ \\
            \mbox{\ref{StdModAreNotL1MGMetTopProjInjFlat}},
            \mbox{\ref{LpFinGrL1MGMetrInjProjCharac}}
        \end{tabular} & 
        \begin{tabular}{@{}c@{}}
            $G$\mbox{ is any } \\
            \mbox{\ref{LInfIsL1MetrInj}}
        \end{tabular} &
        \begin{tabular}{@{}c@{}}
            $G= \{e_G \}$ \\
            \mbox{\ref{StdModAreNotL1MGMetTopProjInjFlat}},
            \mbox{\ref{StdModFinGrL1MGMetrInjProjFlatCharac}}
        \end{tabular} & 
        \begin{tabular}{@{}c@{}}
            $G= \{e_G \}$ \\
            \mbox{\ref{StdModAreNotL1MGMetTopProjInjFlat}},
            \mbox{\ref{LpFinGrL1MGMetrInjProjCharac}}
        \end{tabular} & 
        \begin{tabular}{@{}c@{}}
            $G$\mbox{ is any } \\
            \mbox{\ref{LInfIsL1MetrInj}},
            \mbox{\ref{MGMetTopProjInjFlatRedToL1}}
        \end{tabular} & 
        \begin{tabular}{@{}c@{}}
            $G= \{e_G \}$ \\
            \mbox{\ref{StdModAreNotL1MGMetTopProjInjFlat}},
            \mbox{\ref{StdModFinGrL1MGMetrInjProjFlatCharac}}
        \end{tabular} \\ 
    \hline
        $M(G)$ & 
        \begin{tabular}{@{}c@{}}
            $G$\mbox{ is discrete } \\
            \mbox{\ref{L1MetTopProjAndMetrFlatOfMeasAlg}}
        \end{tabular} & 
        \begin{tabular}{@{}c@{}}
            $G= \{e_G \}$ \\
            \mbox{\ref{StdModAreNotL1MGMetTopProjInjFlat}},
            \mbox{\ref{StdModFinGrL1MGMetrInjProjFlatCharac}}
        \end{tabular} & 
        \begin{tabular}{@{}c@{}}
            $G$\mbox{ is discrete } \\
            \mbox{\ref{MeasAlgIsL1TopFlat}}
        \end{tabular} & 
        \begin{tabular}{@{}c@{}}
            $G$\mbox{ is any } \\
            \mbox{\ref{MGModMGMetTopProjFlatCharac}}
        \end{tabular} & 
        \begin{tabular}{@{}c@{}}
            $G= \{e_G \}$ \\
            \mbox{\ref{StdModAreNotL1MGMetTopProjInjFlat}},
            \mbox{\ref{StdModFinGrL1MGMetrInjProjFlatCharac}}
        \end{tabular} & 
        \begin{tabular}{@{}c@{}}
            $G$\mbox{ is any } \\
            \mbox{\ref{MGModMGMetTopProjFlatCharac}}
        \end{tabular} \\ 
    \hline
        $C_0(G)$ & 
        \begin{tabular}{@{}c@{}}
            $G= \{e_G \}$ \\
            \mbox{\ref{StdModAreNotL1MGMetTopProjInjFlat}},
            \mbox{\ref{StdModFinGrL1MGMetrInjProjFlatCharac}}
        \end{tabular} & 
        \begin{tabular}{@{}c@{}}
            $G$\mbox{ is finite } \\
            \mbox{\ref{StdModAreNotL1MGMetTopProjInjFlat}},
            \mbox{\ref{StdModFinGrL1MGMetrInjProjFlatCharac}}
        \end{tabular} & 
        \begin{tabular}{@{}c@{}}
            $G= \{e_G \}$ \\
            \mbox{\ref{StdModAreNotL1MGMetTopProjInjFlat}},
            \mbox{\ref{StdModFinGrL1MGMetrInjProjFlatCharac}}
        \end{tabular} & 
        \begin{tabular}{@{}c@{}}
            $G= \{e_G \}$ \\
            \mbox{\ref{StdModAreNotL1MGMetTopProjInjFlat}},
            \mbox{\ref{StdModFinGrL1MGMetrInjProjFlatCharac}}
        \end{tabular} & 
        \begin{tabular}{@{}c@{}}
            $G$\mbox{ is finite } \\
            \mbox{\ref{StdModAreNotL1MGMetTopProjInjFlat}},
            \mbox{\ref{StdModFinGrL1MGMetrInjProjFlatCharac}}
        \end{tabular} & 
        \begin{tabular}{@{}c@{}}
            $G= \{e_G \}$ \\
            \mbox{\ref{StdModAreNotL1MGMetTopProjInjFlat}},
            \mbox{\ref{StdModFinGrL1MGMetrInjProjFlatCharac}}
        \end{tabular} \\ 
    \hline 
        $\mathbb{C}_\gamma$ & 
        \begin{tabular}{@{}c@{}}
            $G$\mbox{ is compact } \\
            \mbox{\ref{OneDimL1ModMetTopProjCharac}}
        \end{tabular} & 
        \begin{tabular}{@{}c@{}}
            $G$\mbox{ is amenable } \\
            \mbox{\ref{OneDimL1ModMetTopInjFlatCharac}}
        \end{tabular} & 
        \begin{tabular}{@{}c@{}}
            $G$\mbox{ is amenable } \\
            \mbox{\ref{OneDimL1ModMetTopInjFlatCharac}}
        \end{tabular} & 
        \begin{tabular}{@{}c@{}}
            $G$\mbox{ is compact } \\
            \mbox{\ref{OneDimL1ModMetTopProjCharac}},
            \mbox{\ref{MGMetTopProjInjFlatRedToL1}}
        \end{tabular} & 
        \begin{tabular}{@{}c@{}}
            $G$\mbox{ is amenable } \\
            \mbox{\ref{OneDimL1ModMetTopInjFlatCharac}},
            \mbox{\ref{MGMetTopProjInjFlatRedToL1}}
        \end{tabular} & 
        \begin{tabular}{@{}c@{}}
            $G$\mbox{ is amenable } \\
            \mbox{\ref{OneDimL1ModMetTopInjFlatCharac}},
            \mbox{\ref{MGMetTopProjInjFlatRedToL1}}
        \end{tabular} \\ 
    \hline
        \multicolumn{7}{c}{
            \mbox{
                Homologically trivial $L_1(G)$- and $M(G)$-modules 
                in topological theory
            }
        } \\
    \hline & 
        \multicolumn{3}{c|}{
            $L_1(G)$-modules
        } & 
        \multicolumn{3}{c|}{
            $M(G)$-modules
        } \\
    \hline & 
        \mbox{Projectivity} & 
        \mbox{Injectivity} & 
        \mbox{Flatness} & 
        \mbox{Projectivity} & 
        \mbox{Injectivity} & 
        \mbox{Flatness} \\ 
    \hline
        $L_1(G)$ & 
        \begin{tabular}{@{}c@{}}
            $G$\mbox{ is discrete } \\
            \mbox{\ref{L1ModL1MetTopProjCharac}}
        \end{tabular} & 
        \begin{tabular}{@{}c@{}}
            $G$\mbox{ is finite } \\
            \mbox{\ref{StdModAreNotL1MGMetTopProjInjFlat}},
            \mbox{\ref{StdModFinGrL1MGTopInjProjFlatCharac}}
        \end{tabular} & 
        \begin{tabular}{@{}c@{}}
            $G$\mbox{ is any } \\
            \mbox{\ref{LInfIsL1MetrInj}}
        \end{tabular} & 
        \begin{tabular}{@{}c@{}}
            $G$\mbox{ is discrete } \\
            \mbox{\ref{L1ModL1MetTopProjCharac}},
            \mbox{\ref{MGMetTopProjInjFlatRedToL1}}
        \end{tabular} & 
        \begin{tabular}{@{}c@{}}
            $G$\mbox{ is finite } \\
            \mbox{\ref{StdModAreNotL1MGMetTopProjInjFlat}},
            \mbox{\ref{StdModFinGrL1MGTopInjProjFlatCharac}}
        \end{tabular} & 
        \begin{tabular}{@{}c@{}}
            $G$\mbox{ is any } \\
            \mbox{\ref{LInfIsL1MetrInj}},
            \mbox{\ref{MGMetTopProjInjFlatRedToL1}}
        \end{tabular} \\ 
    \hline
        $L_p(G)$ & 
        \begin{tabular}{@{}c@{}}
            $G$\mbox{ is finite } \\
            \mbox{\ref{StdModAreNotL1MGMetTopProjInjFlat}},
            \mbox{\ref{StdModFinGrL1MGTopInjProjFlatCharac}}
        \end{tabular} & 
        \begin{tabular}{@{}c@{}}
            $G$\mbox{ is finite } \\
            \mbox{\ref{StdModAreNotL1MGMetTopProjInjFlat}},
            \mbox{\ref{StdModFinGrL1MGTopInjProjFlatCharac}}
        \end{tabular} & 
        \begin{tabular}{@{}c@{}}
            $G$\mbox{ is finite } \\
            \mbox{\ref{StdModAreNotL1MGMetTopProjInjFlat}},
            \mbox{\ref{StdModFinGrL1MGTopInjProjFlatCharac}}
        \end{tabular} & 
        \begin{tabular}{@{}c@{}}
            $G$\mbox{ is finite } \\
            \mbox{\ref{StdModAreNotL1MGMetTopProjInjFlat}},
            \mbox{\ref{StdModFinGrL1MGTopInjProjFlatCharac}}
        \end{tabular} & 
        \begin{tabular}{@{}c@{}}
            $G$\mbox{ is finite } \\
            \mbox{\ref{StdModAreNotL1MGMetTopProjInjFlat}},
            \mbox{\ref{StdModFinGrL1MGTopInjProjFlatCharac}}
        \end{tabular} & 
        \begin{tabular}{@{}c@{}}
            $G$\mbox{ is finite } \\
            \mbox{\ref{StdModAreNotL1MGMetTopProjInjFlat}},
            \mbox{\ref{StdModFinGrL1MGTopInjProjFlatCharac}}
        \end{tabular} \\ 
    \hline
        $L_\infty(G)$ &
        \begin{tabular}{@{}c@{}}
            $G$\mbox{ is finite } \\
            \mbox{\ref{StdModAreNotL1MGMetTopProjInjFlat}},
            \mbox{\ref{StdModFinGrL1MGTopInjProjFlatCharac}}
        \end{tabular} & 
        \begin{tabular}{@{}c@{}}
            $G$\mbox{ is any } \\
            \mbox{\ref{LInfIsL1MetrInj}}
        \end{tabular} & 
        \begin{tabular}{@{}c@{}}
            $G$\mbox{ is finite } \\
            \mbox{\ref{StdModAreNotL1MGMetTopProjInjFlat}},
            \mbox{\ref{StdModFinGrL1MGTopInjProjFlatCharac}}
        \end{tabular} & 
        \begin{tabular}{@{}c@{}}
            $G$\mbox{ is finite } \\
            \mbox{\ref{StdModAreNotL1MGMetTopProjInjFlat}},
            \mbox{\ref{StdModFinGrL1MGTopInjProjFlatCharac}}
        \end{tabular} & 
        \begin{tabular}{@{}c@{}}
            $G$\mbox{ is any } \\
            \mbox{\ref{LInfIsL1MetrInj}},
            \mbox{\ref{MGMetTopProjInjFlatRedToL1}}
        \end{tabular} & 
        \begin{tabular}{@{}c@{}}
            $G$\mbox{ is finite } \\
            \mbox{\ref{StdModAreNotL1MGMetTopProjInjFlat}},
            \mbox{\ref{StdModFinGrL1MGTopInjProjFlatCharac}}
        \end{tabular} \\ 
    \hline
        $M(G)$ & 
        \begin{tabular}{@{}c@{}}
            $G$\mbox{ is discrete } \\
            \mbox{\ref{L1MetTopProjAndMetrFlatOfMeasAlg}}
        \end{tabular} & 
        \begin{tabular}{@{}c@{}}
            $G$\mbox{ is finite } \\
            \mbox{\ref{StdModAreNotL1MGMetTopProjInjFlat}},
            \mbox{\ref{StdModFinGrL1MGTopInjProjFlatCharac}}
        \end{tabular} & 
        \begin{tabular}{@{}c@{}}
            $G$\mbox{ is any } \\
            \mbox{\ref{MeasAlgIsL1TopFlat}}
        \end{tabular} & 
        \begin{tabular}{@{}c@{}}
            $G$\mbox{ is any } \\
            \mbox{\ref{MGModMGMetTopProjFlatCharac}}
        \end{tabular} & 
        \begin{tabular}{@{}c@{}}
            $G$\mbox{ is finite } \\
            \mbox{\ref{StdModAreNotL1MGMetTopProjInjFlat}},
            \mbox{\ref{StdModFinGrL1MGTopInjProjFlatCharac}}
        \end{tabular} & 
        \begin{tabular}{@{}c@{}}
            $G$\mbox{ is any } \\
            \mbox{\ref{MGModMGMetTopProjFlatCharac}}
        \end{tabular} \\ 
    \hline
        $C_0(G)$ & 
        \begin{tabular}{@{}c@{}}
            $G$\mbox{ is finite } \\
            \mbox{\ref{StdModAreNotL1MGMetTopProjInjFlat}},
            \mbox{\ref{StdModFinGrL1MGTopInjProjFlatCharac}}
        \end{tabular} & 
        \begin{tabular}{@{}c@{}}
            $G$\mbox{ is finite } \\
            \mbox{\ref{StdModAreNotL1MGMetTopProjInjFlat}},
            \mbox{\ref{StdModFinGrL1MGTopInjProjFlatCharac}}
        \end{tabular} & 
        \begin{tabular}{@{}c@{}}
            $G$\mbox{ is finite } \\
            \mbox{\ref{StdModAreNotL1MGMetTopProjInjFlat}},
            \mbox{\ref{StdModFinGrL1MGTopInjProjFlatCharac}}
        \end{tabular} & 
        \begin{tabular}{@{}c@{}}
            $G$\mbox{ is finite } \\
            \mbox{\ref{StdModAreNotL1MGMetTopProjInjFlat}},
            \mbox{\ref{StdModFinGrL1MGTopInjProjFlatCharac}}
        \end{tabular} & 
        \begin{tabular}{@{}c@{}}
            $G$\mbox{ is finite } \\
            \mbox{\ref{StdModAreNotL1MGMetTopProjInjFlat}},
            \mbox{\ref{StdModFinGrL1MGTopInjProjFlatCharac}}
        \end{tabular} & 
        \begin{tabular}{@{}c@{}}
            $G$\mbox{ is finite } \\
            \mbox{\ref{StdModAreNotL1MGMetTopProjInjFlat}},
            \mbox{\ref{StdModFinGrL1MGTopInjProjFlatCharac}}
        \end{tabular} \\ 
    \hline
        $\mathbb{C}_\gamma$ & 
        \begin{tabular}{@{}c@{}}
            $G$\mbox{ is compact } \\
            \mbox{\ref{OneDimL1ModMetTopProjCharac}}
        \end{tabular} & 
        \begin{tabular}{@{}c@{}}
            $G$\mbox{ is amenable } \\
            \mbox{\ref{OneDimL1ModMetTopInjFlatCharac}}
        \end{tabular} & 
        \begin{tabular}{@{}c@{}}$G$\mbox{ is amenable } \\
            \mbox{\ref{OneDimL1ModMetTopInjFlatCharac}}
        \end{tabular} & 
        \begin{tabular}{@{}c@{}}
            $G$\mbox{ is compact } \\
            \mbox{\ref{OneDimL1ModMetTopProjCharac}},
            \mbox{\ref{MGMetTopProjInjFlatRedToL1}}
        \end{tabular} & 
        \begin{tabular}{@{}c@{}}
            $G$\mbox{ is amenable } \\
            \mbox{\ref{OneDimL1ModMetTopInjFlatCharac}},
            \mbox{\ref{MGMetTopProjInjFlatRedToL1}}
        \end{tabular} & 
        \begin{tabular}{@{}c@{}}
            $G$\mbox{ is amenable } \\
            \mbox{\ref{OneDimL1ModMetTopInjFlatCharac}},
            \mbox{\ref{MGMetTopProjInjFlatRedToL1}}
        \end{tabular} \\ 
    \hline
        \multicolumn{7}{c}{
            \mbox{
                Homologically trivial $L_1(G)$- and $M(G)$-modules 
                in relative theory
            }
        } \\
    \hline & 
    \multicolumn{3}{c|}{
        $L_1(G)$-modules
    } & 
    \multicolumn{3}{c|}{
        $M(G)$-modules
    } \\
    \hline & 
        \mbox{Projectivity} & 
        \mbox{Injectivity} & 
        \mbox{Flatness} & 
        \mbox{Projectivity} & 
        \mbox{Injectivity} & 
        \mbox{Flatness} \\ 
    \hline 
        $L_1(G)$ & 
        \begin{tabular}{@{}c@{}}
            $G$\mbox{ is any } \\
            \mbox{\cite{DalPolHomolPropGrAlg}, \S 6}
        \end{tabular} & 
        \begin{tabular}{@{}c@{}}
            $G$\mbox{ is amenable } \\
            \mbox{ and discrete } \\
            \mbox{\cite{DalPolHomolPropGrAlg}, \S 6}
        \end{tabular} & 
        \begin{tabular}{@{}c@{}}
            $G$\mbox{ is any } \\
            \mbox{\cite{DalPolHomolPropGrAlg}, \S 6}
        \end{tabular} &
        \begin{tabular}{@{}c@{}}
            $G$\mbox{ is any } \\
            \mbox{\cite{RamsHomPropSemgroupAlg}, \S 3.5}
        \end{tabular} &
        \begin{tabular}{@{}c@{}}
            $G$\mbox{ is amenable } \\
            \mbox{ and discrete } \\
            \mbox{\cite{RamsHomPropSemgroupAlg}, \S 3.5}
        \end{tabular} &
        \begin{tabular}{@{}c@{}}
            $G$\mbox{ is any } \\
            \mbox{\cite{RamsHomPropSemgroupAlg}, \S 3.5}
        \end{tabular} \\ 
    \hline
        $L_p(G)$ & 
        \begin{tabular}{@{}c@{}}
            $G$\mbox{ is compact } \\
            \mbox{\cite{DalPolHomolPropGrAlg}, \S 6}
        \end{tabular} & 
        \begin{tabular}{@{}c@{}}
            $G$\mbox{ is amenable } \\
            \mbox{\cite{RachInjModAndAmenGr}}
        \end{tabular} & 
        \begin{tabular}{@{}c@{}}
            $G$\mbox{ is amenable } \\
            \mbox{\cite{RachInjModAndAmenGr}}
        \end{tabular} & 
        \begin{tabular}{@{}c@{}}
            $G$\mbox{ is compact } \\
            \mbox{\cite{RamsHomPropSemgroupAlg}, \S 3.5}
        \end{tabular} & 
        \begin{tabular}{@{}c@{}}
            $G$\mbox{ is amenable } \\
            \mbox{\cite{RamsHomPropSemgroupAlg}, \S 3.5},
            \mbox{\cite{RachInjModAndAmenGr}}
        \end{tabular} &
        \begin{tabular}{@{}c@{}}
            $G$\mbox{ is amenable } \\
            \mbox{\cite{RamsHomPropSemgroupAlg}, \S 3.5}
        \end{tabular} \\
    \hline
        $L_\infty(G)$ & 
        \begin{tabular}{@{}c@{}}
            $G$\mbox{ is finite } \\
            \mbox{\cite{DalPolHomolPropGrAlg}, \S 6}
        \end{tabular} & 
        \begin{tabular}{@{}c@{}}
            $G$\mbox{ is any } \\
            \mbox{\cite{DalPolHomolPropGrAlg}, \S 6}
        \end{tabular} & 
        \begin{tabular}{@{}c@{}}
            $G$\mbox{ is amenable } \\
            \mbox{\cite{DalPolHomolPropGrAlg}, \S 6}
        \end{tabular} & 
        \begin{tabular}{@{}c@{}}
            $G$\mbox{ is finite } \\
            \mbox{\cite{RamsHomPropSemgroupAlg}, \S 3.5}
        \end{tabular} & 
        \begin{tabular}{@{}c@{}}
            $G$\mbox{ is any } \\
            \mbox{\cite{RamsHomPropSemgroupAlg}, \S 3.5}
        \end{tabular} & 
        \begin{tabular}{@{}c@{}}
            $G$\mbox{ is amenable } \\ 
            \mbox{\cite{RamsHomPropSemgroupAlg}, \S 3.5}${}^{*}$
        \end{tabular} \\ 
    \hline
        $M(G)$ & 
        \begin{tabular}{@{}c@{}}
            $G$\mbox{ is discrete } \\
            \mbox{\cite{DalPolHomolPropGrAlg}, \S 6}
        \end{tabular} & 
        \begin{tabular}{@{}c@{}}
            $G$\mbox{ is amenable }\\
            \mbox{\cite{DalPolHomolPropGrAlg}, \S 6}
        \end{tabular} & 
        \begin{tabular}{@{}c@{}}
            $G$\mbox{ is any } \\
            \mbox{\cite{RamsHomPropSemgroupAlg}, \S 3.5}
        \end{tabular} & 
        \begin{tabular}{@{}c@{}}
            $G$\mbox{ is any } \\ 
            \mbox{\cite{RamsHomPropSemgroupAlg}, \S 3.5}
        \end{tabular} & 
        \begin{tabular}{@{}c@{}}
            $G$\mbox{ is amenable } \\
            \mbox{\cite{RamsHomPropSemgroupAlg}, \S 3.5}
        \end{tabular} & 
        \begin{tabular}{@{}c@{}}
            $G$\mbox{ is any } \\
            \mbox{\cite{RamsHomPropSemgroupAlg}, \S 3.5}
        \end{tabular} \\ 
    \hline 
        $C_0(G)$ & 
        \begin{tabular}{@{}c@{}}
            $G$\mbox{ is compact } \\ 
            \mbox{\cite{DalPolHomolPropGrAlg}, \S 6}
        \end{tabular} & 
        \begin{tabular}{@{}c@{}}
            $G$\mbox{ is finite } \\ 
            \mbox{\cite{DalPolHomolPropGrAlg}, \S 6}
        \end{tabular} & 
        \begin{tabular}{@{}c@{}}
            $G$\mbox{ is amenable } \\ 
            \mbox{\cite{DalPolHomolPropGrAlg}, \S 6}
        \end{tabular} & 
        \begin{tabular}{@{}c@{}}
            $G$\mbox{ is compact } \\ 
            \mbox{\cite{RamsHomPropSemgroupAlg}, \S 3.5}
        \end{tabular} & 
        \begin{tabular}{@{}c@{}}
            $G$\mbox{ is finite } \\
            \mbox{\cite{RamsHomPropSemgroupAlg}, \S 3.5}
        \end{tabular} & 
        \begin{tabular}{@{}c@{}}
            $G$\mbox{ is amenable } \\
            \mbox{\cite{RamsHomPropSemgroupAlg}, \S 3.5}
        \end{tabular} \\ 
    \hline
        $\mathbb{C}_\gamma$ & 
        \begin{tabular}{@{}c@{}}
            $G$\mbox{ is compact } \\
            \mbox{\ref{OneDimL1ModMetTopProjCharac}}
        \end{tabular} & 
        \begin{tabular}{@{}c@{}}
            $G$\mbox{ is amenable } \\
            \mbox{\ref{OneDimL1ModMetTopInjFlatCharac}}
        \end{tabular} & 
        \begin{tabular}{@{}c@{}}
            $G$\mbox{ is amenable } \\
            \mbox{\ref{OneDimL1ModMetTopInjFlatCharac}}
        \end{tabular} & 
        \begin{tabular}{@{}c@{}}
            $G$\mbox{ is compact } \\
            \mbox{\ref{OneDimL1ModMetTopProjCharac}},
            \mbox{\ref{MGMetTopProjInjFlatRedToL1}}
        \end{tabular} & 
        \begin{tabular}{@{}c@{}}
            $G$\mbox{ is amenable } \\
            \mbox{\ref{OneDimL1ModMetTopInjFlatCharac}},
            \mbox{\ref{MGMetTopProjInjFlatRedToL1}}
        \end{tabular} & 
        \begin{tabular}{@{}c@{}}
            $G$\mbox{ is amenable } \\
            \mbox{\ref{OneDimL1ModMetTopInjFlatCharac}},
            \mbox{\ref{MGMetTopProjInjFlatRedToL1}}
        \end{tabular} \\
    \hline
    \end{longtable}
\end{scriptsize}
