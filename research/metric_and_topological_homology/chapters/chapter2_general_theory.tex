% chktex-file 35
% Chapter Template

% Main chapter title
% Change X to a consecutive number; for referencing 
% this chapter elsewhere, use~\ref{ChapterX}
\chapter{General theory}\label{ChapterGeneralTheory} 

% Change X to a consecutive number; this is for the header 
% on each page - perhaps a shortened title
\lhead{Chapter 2. \emph{General theory}} 

%-------------------------------------------------------------------------------
%    Projectivity, injectivity and flatness
%-------------------------------------------------------------------------------

\section{
    Projectivity, injectivity and flatness}\label{SectionProjectivityInjectivityAndFlatness}


%-------------------------------------------------------------------------------
%    Metric and topological projectivity
%-------------------------------------------------------------------------------

\subsection{
    Metric and topological projectivity}\label{SubSectionMetricAndTopologicalProjectivity}

In what follows $A$ denotes a not necessary unital Banach algebra. We
immediately start from the most important definitions in this book.

\begin{definition}[\cite{HelMetrFrQMod}, 
    definition 1.4;~\cite{WhiteInjmoduAlg}, 
    definition 2.4]\label{MetCTopProjMod} 
An $A$-module $P$ is 
called $\langle$~metrically / $C$-topologically~$\rangle$ projective if for any 
strictly $\langle$~coisometric / $c$-topologically surjective~$\rangle$
$A$-morphism $\xi:X\to Y$ and for any $A$-morphism $\phi:P\to Y$ 
there exists an $A$-morphism $\psi:P\to X$ such that $\xi\psi=\phi$  and
$\langle$~$\Vert\psi\Vert\leq\Vert\phi\Vert$ / 
$\Vert \psi\Vert\leq c C\Vert\phi\Vert$~$\rangle$. We say that an $A$-module 
$P$ is topologically projective if it is $C$-topologically projective for 
some $C\geq 1$.
\end{definition}

The task of constructing an $A$-morphism $\psi$ for a given $A$-morphisms $\phi$ 
and $\xi$ in the definition~\ref{MetCTopProjMod} is called a lifting problem 
and $\psi$ is called a lifting of $\phi$ along $\xi$.

A short but more involved equivalent definition 
of $\langle$~metric / $C$-topological~$\rangle$ projectivity is the following: 
an $A$-module $P$ is called $\langle$~metrically / $C$-topologically~$\rangle$ 
projective, if the functor
$\langle$~$\operatorname{Hom}_{A-\mathbf{mod}_1}(P,-)
:A-\mathbf{mod}_1\to\mathbf{Ban}_1$
/
$\operatorname{Hom}_{A-\mathbf{mod}}(P,-)
:A-\mathbf{mod}\to\mathbf{Ban}$~$\rangle$
maps strictly $\langle$~coisometric / $c$-topologically surjective~$\rangle$
$A$-morphisms into strictly $\langle$~coisometric / 
$c C$-topologically surjective~$\rangle$ operators. 

In category $\langle$~$A-\mathbf{mod}_1$ / $A-\mathbf{mod}$~$\rangle$ there
is a special class of $\langle$~metrically / $1$-topologically~$\rangle$
projective modules of the form $A_+\projtens \ell_1(\Lambda)$ for 
some set $\Lambda$. They are called free modules. These modules play a crucial 
role in our studies of projectivity.

\begin{proposition}[\cite{WhiteInjmoduAlg}, 
    lemma 2.6]\label{MetCTopFreeMod} Let $\Lambda$ be an arbitrary 
set, then the left $A$-modules $A_+ \projtens \ell_1(\Lambda)$ 
and $A_\times \projtens \ell_1(\Lambda)$ 
are $\langle$~metrically / $1$-topologically~$\rangle$ projective. 
\end{proposition}
\begin{proof} By $A_\bullet$ we denote either $A_+$ or $A_\times$. 
Consider arbitrary $A$-morphism 
$\phi:A_\bullet\projtens\ell_1(\Lambda)\to X$ and a 
strictly $\langle$~coisometric / $c$-topologically surjective~$\rangle$ 
$A$-morphism $\xi:X\to Y$. Fix arbitrary $\lambda\in\Lambda$ and 
consider $y_\lambda=\phi(e_{A_\bullet}\projtens \delta_\lambda)$. 
Clearly $\Vert y_\lambda\Vert\leq \Vert\phi\Vert$. Since $\xi$ is 
strictly $\langle$~coisometric / $c$-topologically surjective~$\rangle$, 
then there exists $x_\lambda\in X$ such that $\xi(x_\lambda)=y_\lambda$
and $\Vert x_\lambda\Vert\leq K\Vert y_\lambda\Vert$ for $\langle$~$K=1$ / 
$K=c$~$\rangle$. Now we can define a bounded linear operator
$\psi:A_\bullet\projtens \ell_1(\Lambda)\to X$ such that 
$\psi(a\projtens \delta_\lambda)=a\cdot x_\lambda$ for $\lambda\in\Lambda$. 
It is routine to check that $\psi$ is an $A$-morphism with $\xi\psi=\phi$ 
and $\Vert\psi\Vert\leq K\Vert\phi\Vert$. Thus, for a given $\phi$ and $\xi$
we have constructed an $A$-morphism $\psi$ such that $\xi\psi=\phi$ and
$\langle$~$\Vert\psi\Vert\leq \Vert\phi\Vert$ / $\Vert\psi\Vert\leq
c\Vert\phi\Vert$~$\rangle$. Hence, the $A$-module $A_\bullet\projtens
\ell_1(\Lambda)$ is $\langle$~metrically / $1$-topologically~$\rangle$
projective.
\end{proof}


\begin{proposition}\label{UnitalAlgIsMetTopProj} The left $A$-module $A_\times$
is $\langle$~metrically / $1$-topologically~$\rangle$ projective.
\end{proposition} 
\begin{proof} Consider set $\Lambda=\mathbb{N}_1$. 
By proposition~\ref{MetCTopFreeMod} the 
$A$-module $A_\times\projtens\ell_1(\Lambda)$ is metrically 
and $1$-topologically projective. Now it remains to note that 
$A_\times\projtens\ell_1(\Lambda)
\isom{A-\mathbf{mod}_1}
A_\times\projtens \mathbb{C}
\isom{A-\mathbf{mod}_1}
A_\times$.
\end{proof}

\begin{proposition}[\cite{WhiteInjmoduAlg}, 
    lemma 2.7]\label{RetrMetCTopProjIsMetCTopProj} Any 
$\langle$~$1$-retract / $c$-retract~$\rangle$ of
$\langle$~metrically / $C$-topologically~$\rangle$ projective module is
$\langle$~metrically / $c C$-topologically~$\rangle$ projective.
\end{proposition}
\begin{proof} Suppose that $P$ is a $c$-retract of 
$\langle$~metrically / $C$-topologically~$\rangle$ projective $A$-module $P'$.
Then there exist $A$-morphisms $\eta:P\to P'$ and $\zeta: P'\to P$ such that
$\zeta\eta=1_{P}$ and $\Vert\zeta\Vert\Vert\eta\Vert\leq c$ 
for $\langle$~$c=1$ / $c\geq 1$~$\rangle$. Consider arbitrary
strictly $\langle$~coisometric / $c'$-topologically surjective~$\rangle$ 
$A$-morphism $\xi:X\to Y$ and an arbitrary $A$-morphism $\phi:P\to Y$. 
Consider $A$-morphism $\phi'=\phi\zeta$. Since $P'$ is 
$\langle$~metrically / $C$-topologically~$\rangle$ projective, then there 
exists an $A$-morphism $\psi':P'\to X$ such that $\phi'=\xi\psi'$ 
and $\Vert\psi'\Vert \leq K\Vert\phi'\Vert$ 
for $\langle$~$K=1$ / $K=c' C$~$\rangle$. Now it is routine to check that 
for the $A$-morphism $\psi=\psi'\eta$ 
holds $\xi\psi=\phi$ and $\Vert\psi\Vert \leq cK\Vert\phi\Vert$. 
Thus, for a given $\phi$ and $\xi$ we have
constructed an $A$-morphism $\psi$ such that $\xi\psi=\phi$  and
$\langle$~$\Vert\psi\Vert\leq\Vert\phi\Vert$ / 
$\Vert \psi\Vert\leq c C\Vert\phi\Vert$~$\rangle$. Hence, $P$ 
is $\langle$~metrically / $c C$-topologically~$\rangle$ projective.
\end{proof}

It is easy to show that, for any $A$-module $X$ there exists 
a strictly $\langle$~coisometric / $1$-topologically surjective~$\rangle$ 
$A$-morphism
$$
\pi_X^+:A_+\projtens \ell_1(B_X)\to X:a\projtens \delta_x\mapsto a\cdot x.
$$

\begin{proposition}[\cite{WhiteInjmoduAlg}, proposition 2.10]\label{MetCTopProjModViaCanonicMorph} 
An $A$-module $P$ is
$\langle$~metrically / $C$-topologically~$\rangle$ projective iff $\pi_P^+$ is 
a $\langle$~$1$-retraction / $C$-retraction~$\rangle$ in $A-\mathbf{mod}$.
\end{proposition}
\begin{proof}
Suppose $P$ is 
$\langle$~metrically / $C$-topologically~$\rangle$ projective, then consider 
strictly $\langle$~coisometric / $1$-topologically surjective~$\rangle$
$A$-morphism $\pi_P^+$. Consider lifting problem with $\phi=1_P$ 
and $\xi=\pi_P^+$, then from $\langle$~metric / $C$-topological~$\rangle$ 
projectivity of $P$ we get an $A$-morphism $\sigma^+$ such 
that $\pi_P^+\sigma^+=1_P$
and $\Vert\sigma^+\Vert\leq K$ for $\langle$~$K=1$ / $K=C$~$\rangle$.
Since $\Vert\pi_P^+\Vert\Vert \sigma^+\Vert\leq K$ 
we conclude that $\pi_P^+$ is a
$\langle$~$1$-retraction / $C$-retraction~$\rangle$ in $A-\mathbf{mod}$.

Conversely, assume that $\pi_P^+$ is 
a $\langle$~$1$-retraction / $C$-retraction~$\rangle$. In other words $P$ is 
a $\langle$~$1$-retract / $C$-retract~$\rangle$ of $A_+\projtens \ell_1(B_P)$. 
By proposition~\ref{MetCTopFreeMod} the $A$-module $A_+\projtens \ell_1(B_P)$ 
is $\langle$~metrically / $1$-topologically~$\rangle$ projective. So from 
proposition~\ref{RetrMetCTopProjIsMetCTopProj} 
its $\langle$~$1$-retract / $C$-retract~$\rangle$ $P$ is 
$\langle$~metrically / $C$-topologically~$\rangle$ projective.
\end{proof}

\begin{proposition}[\cite{WhiteInjmoduAlg}, 
    proposition 2.9]\label{MetProjIsTopProjAndTopProjIsRelProj} Every metrically
projective module is $1$-topologically projective and every $C$-topologically 
projective module is $C$-relatively projective.
\end{proposition}
\begin{proof} By proposition~\ref{MetCTopProjModViaCanonicMorph} every 
metrically projective $A$-module $P$ is a $1$-retract 
of $A_+\projtens \ell_1(B_P)$. Hence, by the same proposition $P$ 
is $1$-topologically projective. Again by 
proposition~\ref{MetCTopProjModViaCanonicMorph} every $C$-topologically 
projective $A$-module $P$ is a $C$-retract of $A$-module $A_+\projtens \ell_1(B_P)$. 
In other words $P$ is a $C$-retract of the module $A\projtens E$ for some Banach
space $E$. Therefore, $P$ is $C$-relatively projective.
\end{proof}


Clearly, every $C$-topologically projective module is $C'$-topologically 
projective for $C'\geq C$. But we can state even more.

\begin{proposition}\label{MetProjIsOneTopProj} An $A$-module is metrically
projective iff it is $1$-topologically projective.
\end{proposition}
\begin{proof} The result immediately follows 
from propositions~\ref{MetProjIsTopProjAndTopProjIsRelProj} 
and~\ref{MetCTopProjModViaCanonicMorph}.
\end{proof}


Let us proceed to examples. Note that the category of Banach spaces may be
regarded as the category of left Banach modules over zero algebra. As the
results we get the definition of $\langle$~metrically / topologically~$\rangle$
projective Banach space. All the results mentioned above hold for this type of
projectivity. Both types of projective objects are described by now.
In~\cite{KotheTopProjBanSp} K{\"o}the proved that all topologically projective
Banach spaces are topologically isomorphic to $\ell_1(\Lambda)$ for some index
set $\Lambda$. Using result of Grothendieck from~\cite{GrothMetrProjFlatBanSp}
Helemskii showed that metrically projective Banach spaces are isometrically
isomorphic to $\ell_1(\Lambda)$ for some index set $\Lambda$
[\cite{HelMetrFrQMod}, proposition 3.2]. Thus, the zoo of projective Banach
spaces is wide but conformed.

\begin{proposition}\label{NonDegenMetTopProjCharac}  Let $P$ be an essential
$A$-module. Then $P$ is $\langle$~metrically / $C$-topologically~$\rangle$
projective iff the map 
$\pi_P:A\projtens\ell_1(B_P):a\projtens\delta_x\mapsto a\cdot x$ 
is a $\langle$~$1$-retraction / $C$-retraction~$\rangle$ in $A-\mathbf{mod}$.
\end{proposition} 
\begin{proof}
If $P$ is $\langle$~metrically / $C$-topologically~$\rangle$ projective, then by
proposition~\ref{MetCTopProjModViaCanonicMorph} the morphism 
$\pi_P^+$ has a right inverse morphism $\sigma^+$ of norm 
$\langle$~at most $1$ / at most $C$~$\rangle$. Then 
$$
\sigma^+(P)
=\sigma^+(\operatorname{cl}_{A_+\projtens\ell_1(B_P)}(AP))
\subset \operatorname{cl}_{A_+\projtens\ell_1(B_P)}(A\cdot\sigma(P))=
$$
$$
\operatorname{cl}_{A_+\projtens\ell_1(B_P)}(A\cdot(A_+\projtens\ell_1(B_P)))
=A\projtens\ell_1(B_P).
$$ 
So we have well a defined corestriction $\sigma:P\to A\projtens\ell_1(B_P)$
which is also an $A$-morphism with norm $\langle$~at most $1$ / at most
$C$~$\rangle$. Clearly, $\pi_P\sigma=1_P$, so $\pi_P$ is a
$\langle$~$1$-retraction / $C$-retraction~$\rangle$ in $A-\mathbf{mod}$.

Conversely, assume $\pi_P$ has a right inverse morphism $\sigma$ of norm
$\langle$~at most $1$ / at most $C$~$\rangle$. Then its coextension $\sigma^+$
also is a right inverse morphism to $\pi_P^+$ with the same norm. Again, by
proposition~\ref{MetCTopProjModViaCanonicMorph} the module $P$ is
$\langle$~metrically / $C$-topologically~$\rangle$ projective. 
\end{proof}

It is worth to mention that $\langle$~an arbitrary / only finite~$\rangle$
family of objects in $\langle$~$A-\mathbf{mod}_1$ / $A-\mathbf{mod}$~$\rangle$
have the categorical coproduct which is in fact their $\bigoplus_1$-sum. This is
reason why we make additional assumption in the second paragraph of the next
proposition.

\begin{proposition}\label{MetTopProjModCoprod} Let
${(P_\lambda)}_{\lambda\in\Lambda}$ be a family of $A$-modules. Then 
\begin{enumerate}[label = (\roman*)]
    \item $\bigoplus_1 \{P_\lambda:\lambda\in\Lambda \}$ is metrically
    projective iff for all $\lambda\in\Lambda$ the $A$-module $P_\lambda$ is
    metrically projective;

    \item $\bigoplus_1 \{P_\lambda:\lambda\in\Lambda \}$ is $C$-topologically
    projective iff for all $\lambda\in\Lambda$ the $A$-module $P_\lambda$ is
    $C$-topologically projective.
\end{enumerate}
\end{proposition}
\begin{proof} Denote $P:=\bigoplus_1 \{P_\lambda:\lambda\in\Lambda \}$.

$(i)$ The proof is literally the same as in paragraph $(ii)$.

$(ii)$ Assume that $P$ is $C$-topologically projective. Note that, for any
$\lambda\in\Lambda$ the $A$-module $P_\lambda$ is a $1$-retract of $P$ via
natural projection $p_\lambda:P\to P_\lambda$. By
proposition~\ref{RetrMetCTopProjIsMetCTopProj} the $A$-module $P_\lambda$ is
$C$-topologically projective.

Conversely, let each $A$-module $P_\lambda$ be $C$-topologically projective. By
proposition~\ref{MetCTopProjModViaCanonicMorph} we have a family 
of $C$-retractions $\pi_\lambda:A_+\projtens\ell_1(S_\lambda)\to P_\lambda$. 
It follows that
$\bigoplus_1 \{\pi_\lambda:\lambda\in\Lambda \}$ is a $C$-retraction in
$A-\mathbf{mod}$. As the result $P$ is a $C$-retract of 
$$
\bigoplus\nolimits_1\left \{A_+\projtens
\ell_1(S_\lambda):\lambda\in\Lambda\right \} \isom{A-\mathbf{mod}_1}
\bigoplus\nolimits_1\left \{\bigoplus\nolimits_1 \{A_+:s\in S_\lambda
\}:\lambda\in\Lambda\right \}
$$
$$
\isom{A-\mathbf{mod}_1} \bigoplus\nolimits_1 \{A_+:s\in S \}
\isom{A-\mathbf{mod}_1} A_+ \projtens \ell_1(S)
$$
in $A-\mathbf{mod}$ where $S=\bigsqcup_{\lambda\in\Lambda}S_\lambda$. Clearly,
the latter module is $1$-topologically projective, so by
proposition~\ref{RetrMetCTopProjIsMetCTopProj} the $A$-module $P$ is 
$C$-topologically projective.
\end{proof}

\begin{corollary}\label{MetTopProjTensProdWithl1} Let $P$ be an $A$-module and
$\Lambda$ be an arbitrary set. Then $P\projtens \ell_1(\Lambda)$ is
$\langle$~metrically / $C$-topologically~$\rangle$ projective iff $P$ is
$\langle$~metrically / $C$-topologically~$\rangle$ projective.
\end{corollary}
\begin{proof} 
Note that 
$P\projtens \ell_1(\Lambda)
\isom{A-\mathbf{mod}_1}
\bigoplus_1 \{P:\lambda\in\Lambda \}$. It remains to set $P_\lambda=P$ for all
$\lambda\in\Lambda$ and apply proposition~\ref{MetTopProjModCoprod}.
\end{proof}

The property of being metrically, topologically or relatively projective module 
puts some restrictions on the Banach geometric structure of the module.

\begin{proposition}[\cite{RamsHomPropSemgroupAlg}, 
    proposition 2.1.1]\label{MetTopRelProjModCompIdealPartCompl} Let $P$ 
be a $\langle$~metrically / $C$-topologically / $C$-relatively~$\rangle$ 
projective $A$-moudle, and let $I$ 
be a $\langle$~$1$-complemented / $c$-complemented / $c$-complemented~$\rangle$
right ideal of $A$. Then $\operatorname{cl}_P(I P)$ 
is $\langle$~$1$-complemented / $cC$-complemented / $cC$-complemented~$\rangle$ 
in $P$.
\end{proposition}
\begin{proof} Since $A$ is $1$-complemented in $A_+$, then $I$ 
is $\langle$~$1$-complemented / $c$-complemented / $c$-complemented~$\rangle$
in $A_+$. Hence, there exists a bounded linear operator $r:A_+\to I$ of 
$\langle$~norm $1$ / norm $c$ / norm $c$~$\rangle$ such that $r|_I=1_I$. 
Consider operator $R=r\projtens 1_P$, then $R|_{I\projtens P}=1_{I\projtens P}$ 
and $\Vert R\Vert\leq\Vert r\Vert$. Since $P$ 
is $\langle$~metrically / $C$-topologically / $C$-relatively~$\rangle$ 
projective, by proposition~\ref{MetCTopProjModViaCanonicMorph} the 
$A$-morphism $\pi_P^+$ has a right inverse morphism $\sigma^+$ of norm
$\langle$~at most $1$ / at most $C$ / at most $C$~$\rangle$. Now consider 
bounded linear operator $p=\pi_P^+ R \sigma^+$. Clearly,
$\operatorname{Im}(p)=\pi_P^+(R(\sigma^+(P)))
\subset\pi_P^+(R(A_+\projtens P))
\subset \pi_P^+(I\projtens P)
\subset \operatorname{cl}_P(I P)$.
Since $I$ is a right ideal of $A$ and $\sigma^+$ is an $A$-morphism, then 
$$
\sigma^+(\operatorname{cl}_P(I P))
\subset \operatorname{cl}_{A_+\projtens P}(\sigma^+(I P))
\subset \operatorname{cl}_{A_+\projtens P}(I\sigma^+(P))
\subset \operatorname{cl}_{A_+\projtens P}(I (A_+\projtens P))
\subset I \projtens P.
$$
In particular, for all $x\in \operatorname{cl}_P(I P)$, we have 
$p(x)
=\pi_P^+(R(\sigma^+(x)))
=\pi_P^+(1_{I\projtens P}(\sigma^+(x)))
=\pi_P^+(\sigma^+(x))
=x$. 
Thus, $p$ is projection onto $\operatorname{cl}_P(I P)$ with norm at 
most $\Vert \sigma^+\Vert\Vert r\Vert$. Hence, $\operatorname{cl}_P(I P)$ 
is $\langle$~$1$-complemented / $cC$-complemented / 
$cC$-complemented~$\rangle$ in $P$.
\end{proof}

\begin{corollary}\label{MetTopRelProjModPartCompl} Let $P$ 
be a $\langle$~metrically / $C$-topologically / $C$-relatively~$\rangle$ 
projective $A$-moudle. Then $P_{ess}$ 
is $\langle$~$1$-complemented / $C$-complemented / $C$-complemented~$\rangle$ 
in $P$.
\end{corollary}
\begin{proof} The result directly follows from 
proposition~\ref{MetTopRelProjModCompIdealPartCompl}.
\end{proof}

%-------------------------------------------------------------------------------
%    Metric and topological projectivity of ideals and cyclic modules
%-------------------------------------------------------------------------------

\subsection{
    Metric and topological projectivity of ideals and cyclic modules}
\label{SubSectionMetricAndTopologicalProjectivityOfIdealsAndCyclicModules}

As we shall see idempotents play a significant role in the study of metric and
topological projectivity, so we shall recall one of the corollaries of Shilov's
idempotent theorem [\cite{KaniBanAlg}, section 3.5]: every semisimple
commutative Banach algebra with compact spectrum admits an identity, but not
necessarily of norm 1. 

\begin{proposition}\label{UnIdeallIsMetTopProj} Let $I$ be a left ideal of a
Banach algebra $A$. Then

\begin{enumerate}[label = (\roman*)]
    \item if $I=Ap$ for some $\langle$~norm one idempotent /
    idempotent~$\rangle$ $p\in I$, then $I$ is $\langle$~metrically / 
    $\Vert p\Vert$-topologically~$\rangle$ projective $A$-module;

    \item if $I$ is commutative, semisimple and $\operatorname{Spec}(I)$ is
    compact then $I$ is topologically projective $A$-module.
\end{enumerate}
\end{proposition}
\begin{proof} 
$(i)$ For $A$-module maps $\pi:A_\times\to I:x\mapsto xp$ and 
$\sigma:I\to A_\times:x\mapsto x$ we clearly have $\pi\sigma=1_I$. 
Therefore, $I$ is a $\langle$~$1$-retract / $\Vert p\Vert$-retract~$\rangle$ 
of $A_\times$. Now the result follows from 
propositions~\ref{RetrMetCTopProjIsMetCTopProj} and~\ref{UnitalAlgIsMetTopProj}.

$(ii)$ By Shilov's idempotent theorem the ideal $I$ is unital. In general the
norm of this unit is not less than $1$. By paragraph $(i)$ the ideal $I$ is
topologically projective.
\end{proof}

The assumption of semisimplicity in~\ref{UnIdeallIsMetTopProj} is not necessary.
From [\cite{DalesIntroBanAlgOpHarmAnal}, exercise 2.3.7] we know, that there
exists a commutative non semisimple unital Banach algebra $A$. By
proposition~\ref{UnIdeallIsMetTopProj} it is topologically projective as
$A$-module. To prove the main result of this section we need two preparatory
lemmas.

\begin{lemma}\label{ImgOfAMorphFromBiIdToA} Let $I$ be a two-sided ideal of a
Banach algebra $A$ which is essential as left $I$-module and let $\phi:I\to A$
be an $A$-morphism. Then $\operatorname{Im}(\phi)\subset I$.
\end{lemma}
\begin{proof} Since $I$ is a right ideal, then $\phi(ab)=a\phi(b)\in I$ for all
$a,b\in I$. So $\phi(I\cdot I)\subset I$. Since $I$ is an essential left
$I$-module then $I=\operatorname{cl}_A(\operatorname{span}(I\cdot I))$ and
$\operatorname{Im}(\phi)
\subset\operatorname{cl}_A(\operatorname{span}\phi(I\cdot I))
=\operatorname{cl}_A(\operatorname{span}I)
=I$.
\end{proof}

\begin{lemma}\label{GoodIdealMetTopProjIsUnital} Let $I$ be a left ideal of a
Banach algebra $A$, which is $\langle$~metrically / $C$-topologically~$\rangle$
projective as $A$-module. Then the following holds:

\begin{enumerate}[label = (\roman*)]
    \item Assume $I$ has a left $\langle$~contractive / $c$-bounded~$\rangle$
    approximate identity and for each  morphism $\phi:I\to A$ of left
    $A$-modules there exists a morphism $\psi:I\to I$ of right $I$-modules such
    that $\phi(x)y=x\psi(y)$ for all $x,y\in I$. Then $I$ has the identity of
    norm $\langle$~at most $1$ / at most $c$~$\rangle$;

    \item Assume $I$ has a right $\langle$~contractive / $c$-bounded~$\rangle$
    approximate identity and for $\langle$~$k=1$ / some $k\geq 1$~$\rangle$ and
    each morphism $\phi:I\to A$ of left $A$-modules there exists a morphism
    $\psi:I\to I$ of right $I$-modules such that $\Vert\psi\Vert\leq
    k\Vert\phi\Vert$ and $\phi(x)y=x\psi(y)$ for all $x,y\in I$. Then $I$ has a
    right identity of norm $\langle$~at most $1$ / at most $ckC$~$\rangle$.
\end{enumerate}
\end{lemma} 
\begin{proof} If either $(i)$ or $(ii)$ holds then $I$ has a one-sided bounded
approximate identity. So $I$ is an essential left $I$-module, and a fortiori an
essential $A$-module. By proposition~\ref{NonDegenMetTopProjCharac} we have a
right inverse $A$-morphism $\sigma:I\to A\projtens \ell_1(B_I)$ of $\pi_I$ with
norm  $\langle$~at most $1$ / at most $C$~$\rangle$. For each $d\in B_I$
consider $A$-morphisms $p_d:A\projtens \ell_1(B_I)\to A:a\projtens
\delta_x\mapsto \delta_x(d)a$ and $\sigma_d=p_d\sigma$. Then
$\sigma(x)=\sum_{d\in B_I}\sigma_d(x)\projtens \delta_d$ for all $x\in I$. From
identification 
$A\projtens\ell_1(B_I)\isom{\mathbf{Ban}_1}\bigoplus_1 \{A:d\in B_I \}$ 
we have $\Vert\sigma(x)\Vert=\sum_{d\in B_I} \Vert\sigma_d(x)\Vert$ for
all $x\in I$. Since $\sigma$ is a right inverse morphism of $\pi_I$, then
$x=\pi_I(\sigma(x))=\sum_{d\in B_I}\sigma_d(x)d$ for all $x\in I$. 

From assumption, for each $d\in B_I$ there exists a morphism of right
$I$-modules $\tau_d:I\to I$ such that $\sigma_d(x)d=x\tau_d(d)$ for all 
$x\in I$. 

Assume $(i)$ holds. From assumption, for each $d\in B_I$ there exists a morphism
of right $I$-modules $\tau_d:I\to I$ such that $\sigma_d(x)d=x\tau_d(d)$ for all
$x\in I$.  Let ${(e_\nu)}_{\nu\in N}$ be a left $\langle$~contractive /
bounded~$\rangle$ approximate identity of $I$ bounded in norm by constant
$\langle$~$D=1$ / $D=c$~$\rangle$. Since $\tau_d(d)\in I$ for all $d\in B_I$,
then for all $S\in\mathcal{P}_0(B_I)$ holds
$$
\sum_{d\in S}\Vert \tau_d(d)\Vert
=\sum_{d\in S}\lim_{\nu}\Vert e_\nu \tau_d(d) \Vert
=\lim_{\nu}\sum_{d\in S}\Vert e_\nu \tau_d(d)\Vert
=\lim_{\nu}\sum_{d\in S}\Vert \sigma_d(e_\nu)d \Vert
$$
$$
\leq\liminf_{\nu}\sum_{d\in S}\Vert\sigma_d(e_\nu)\Vert\Vert d\Vert 
\leq\liminf_{\nu}\sum_{d\in S}\Vert\sigma_d(e_\nu)\Vert
\leq\liminf_{\nu}\sum_{d\in B_I}\Vert\sigma_d(e_\nu)\Vert
$$
$$
=\liminf_{\nu}\Vert\sigma(e_\nu)\Vert
\leq\Vert\sigma\Vert\liminf_{\nu}\Vert e_\nu\Vert
\leq D\Vert\sigma\Vert 
$$
Since $S\in \mathcal{P}_0(B_I)$ is arbitrary we have well-defined element
$p=\sum_{d\in B_I}\tau_d(d)$ with norm $\langle$ at most $1$ / at most
$cC$~$\rangle$. For all $x\in I$ we have $x=\sum_{d\in
B_I}\sigma_d(x)d=\sum_{d\in B_I}x\tau_d(d)=xp$, i.e. $p$ is a right identity for
$I$. But $I$ admits a left $\langle$~contractive / $c$-bounded~$\rangle$
approximate identity, so $p$ is the identity of $I$ with $\Vert
p\Vert=\lim_\nu\Vert e_\nu\Vert$. Therefore, the norm of $p$ is $\langle$~at most
$1$ / at most $c$~$\rangle$.

Assume $(ii)$ holds. From assumption, for each $d\in B_I$ there exists a
morphism of right $I$-modules $\tau_d:I\to I$ such that
$\sigma_d(x)d=x\tau_d(d)$ for all $x\in I$ and 
$\Vert\tau_d\Vert\leq k\Vert\sigma_d\Vert$. 
Let ${(e_\nu)}_{\nu\in N}$ be a right $\langle$~contractive / bounded~$\rangle$ 
approximate identity of $I$ bounded in norm by constant 
$\langle$~$D=1$ / $D=c$~$\rangle$. For all $x\in I$ we have
$$
\Vert\sigma_d(x)\Vert
=\Vert\sigma_d(\lim_\nu x e_\nu)\Vert
=\lim_\nu\Vert x\sigma_d(e_\nu)\Vert
\leq\Vert x\Vert\liminf_\nu\Vert\sigma_d(e_\nu)\Vert
$$
so $\Vert\sigma_d\Vert\leq \liminf_\nu\Vert\sigma_d(e_\nu)\Vert$. 
Then for all $S\in\mathcal{P}_0(B_I)$ holds
$$
\sum_{d\in S}\Vert \tau_d(d)\Vert
\leq \sum_{d\in S}\Vert \tau_d\Vert\Vert d\Vert
\leq k\sum_{d\in S}\Vert \sigma_d\Vert
\leq k\sum_{d\in S}\liminf_\nu \Vert \sigma_d(e_\nu)\Vert
\leq k\liminf_{\nu}\sum_{d\in S}\Vert \sigma_d(e_\nu) \Vert
$$
$$
\leq k\liminf_{\nu}\sum_{d\in B_I}\Vert \sigma_d(e_\nu) \Vert
=k\liminf_{\nu}\Vert\sigma(e_\nu)\Vert
\leq k\Vert\sigma\Vert\liminf_{\nu}\Vert e_\nu\Vert
\leq kD\Vert\sigma\Vert
$$
Since $S\in \mathcal{P}_0(B_I)$ is arbitrary we have well-defined element
$p=\sum_{d\in B_I}\tau_d(d)$ with norm $\langle$ at most $1$ / at most
$ckC$~$\rangle$. For all $x\in I$ 
we have $x=\sum_{d\in B_I}\sigma_d(x)d=\sum_{d\in B_I}x\tau_d(d)=xp$, 
i.e. $p$ is a right identity for $I$.
\end{proof}

\begin{theorem}\label{GoodCommIdealMetTopProjIsUnital} Let $I$ be an ideal of a
commutative Banach algebra $A$ and $I$ has a $\langle$~contractive /
$c$-bounded~$\rangle$ approximate identity. Then $I$ is $\langle$~metrically /
$c$-topologically~$\rangle$ projective as $A$-module iff $I$ has the identity of
norm $\langle$~at most $1$ / at most $c$~$\rangle$.
\end{theorem} 
\begin{proof} Assume $I$ is $\langle$~metrically / $c$-topologically~$\rangle$
projective as $A$-module. Since $A$ is commutative, then for any $A$-morphism
$\phi:I\to A$ and $x,y\in I$ we have $\phi(x)y=x\phi(y)$. Since $I$ has a 
bounded approximate identity and $I$ is commutative we can apply
lemma~\ref{ImgOfAMorphFromBiIdToA} to get that $\phi(y)\in I$. Now by paragraph
$(i)$ of lemma~\ref{GoodIdealMetTopProjIsUnital} we get that $I$ has the
identity of norm $\langle$~at most $1$ / at most $c$~$\rangle$.

The converse immediately follows from proposition~\ref{UnIdeallIsMetTopProj}.
\end{proof}

There is no analogous criterion of this theorem in relative theory. The most
general result of this kind gives only a necessary condition: any ideal in a
commutative Banach algebra $A$ which is relatively projective as $A$-module has
a paracompact spectrum. This result is due to Helemskii
[\cite{HelHomolBanTopAlg}, theorem IV.3.6]. 

Note that existence of bounded approximate identity is not necessary for an
ideal of a commutative Banach algebra to be even topologically projective.
Indeed, consider Banach algebra  $A_0(\mathbb{D})$ --- the ideal of the disk
algebra consisting of functions vanishing at zero. By combination of 
propositions 4.3.5 and 4.3.13 paragraph $(iii)$ from~\cite{DalBanAlgAutCont} 
we get that $A_0(\mathbb{D})$ has no bounded approximate identities. 
On the other hand, from [\cite{HelBanLocConvAlg}, example IV.2.2] we know that
$A_0(\mathbb{D})\isom{A_0(\mathbb{D})-\mathbf{mod}} {A_0(\mathbb{D})}_+$, so
$A_0(\mathbb{D})$ is topologically projective by
proposition~\ref{UnitalAlgIsMetTopProj}.

Next proposition is an obvious adaptation of purely algebraic argument on
projective cyclic modules. It is almost identical to [\cite{WhiteInjmoduAlg},
proposition 2.11].

\begin{proposition}\label{MetTopProjCycModCharac} Let $I$ be a left ideal in
$A_\times $. Assume the natural projection $\pi:A_\times\to A_\times/I$ is
strictly $\langle$~coisometric / $c$-topologically surjective~$\rangle$. 
Then the following holds:

\begin{enumerate}[label = (\roman*)]
    \item If $A_\times /I$ is $\langle$~metrically / $C$-topologically~$\rangle$
    projective as $A$-module, then there exists an idempotent $p\in I$ such that
    $I=Ap$ and $\Vert e_{A_\times}-p\Vert$ is $\langle$~at most $1$ / at most
    $cC$~$\rangle$;

    \item If there exists an idempotent $p\in I$ such that $I=A_\times  p$ and
    $\Vert e_{A_\times }-p\Vert$ is $\langle$~at most $1$ / at most
    $C$~$\rangle$, then $A_\times/I$ is $\langle$~metrically /
    $C$-topologically~$\rangle$ projective.
\end{enumerate}
\end{proposition}
\begin{proof} $(i)$ Since the natural quotient map $\pi$ is strictly 
$\langle$~coisometric / $c$-topologically surjective~$\rangle$ and $A_\times /I$
is $\langle$~metrically / $C$-topologically~$\rangle$ projective, then $\pi$ has
a right inverse $A$-morphism $\sigma$ with norm $\langle$~at most $1$ / at most
$cC$~$\rangle$. We set $e_{A_\times }-p=(\sigma\pi)(e_{A_\times })$, then
$(\sigma\pi)(a)=a(e_{A_\times }-p)$. By construction, $\pi\sigma=1_{A_\times }$,
so  
$$
e_{A_\times }-p
=(\sigma\pi)(e_{A_\times })
=(\sigma\pi)(\sigma\pi)(e_{A_\times })
=(\sigma\pi)(e_{A_\times }-p)
=(e_{A_\times }-p)(\sigma\pi)(e_{A_\times })
={(e_{A_\times }-p)}^2
$$
This equality shows that $p^2=p$. Therefore, $A_\times
p=\operatorname{Ker}(\sigma\pi)$ because $(\sigma\pi)(a)=a-ap$. Since $\sigma$
is injective this is equivalent to $A_\times p=\operatorname{Ker}(\pi)$ which
equals to $I$. Finally, note that 
$\Vert e_{A_\times }-p\Vert
=\Vert(\sigma\pi)(e_{A_\times})\Vert
\leq\Vert\sigma\Vert\Vert\pi\Vert\Vert e_{A_\times }\Vert
=\Vert\sigma\Vert$.

$(ii)$ Since $p^2=p$, then we have a well-defined left ideal $I=A_\times p$ and
an $A$-module map $\sigma:A_\times /I\to A_\times:a+I\mapsto a-ap$. It is easy
to check that $\pi\sigma=1_{A_\times/I}$ 
and $\Vert\sigma\Vert\leq\Vert e_{A_\times }-p\Vert$. 
This means that $\pi:A_\times \to A_\times /I$ is a
$\langle$~$1$-retraction / $C$-retraction~$\rangle$. From
propositions~\ref{UnitalAlgIsMetTopProj} 
and~\ref{RetrMetCTopProjIsMetCTopProj} it follows that $A_\times /I$ is
$\langle$~metrically / $C$-topologically~$\rangle$ projective.
\end{proof} 

In contrast with topological theory, there is no description of relatively
projective cyclic modules. There are partial answers under additional
assumptions. For example, if an ideal $I$ is complemented as Banach space in
$A_\times$, then almost the same criterion as in previous proposition holds in
relative theory [\cite{HelBanLocConvAlg}, proposition 7.1.29]. There are other
characterizations of relatively projective cyclic modules under more mild
assumptions on Banach geometry. For example, Selivanov proved that if $I$ is a
two-sided ideal and either $A/I$ has the approximation property or all
irreducible $A$-modules have the approximation property, then $A/I$ is
relatively projective iff $A_\times\isom{A-\mathbf{mod}}I\bigoplus_1 I'$ for
some left ideal $I'$ of $A$. For details see [\cite{HelHomolBanTopAlg}, chapter
IV, \S 4].



%-------------------------------------------------------------------------------
%    Metric and topological injectivity
%-------------------------------------------------------------------------------

\subsection{
    Metric and topological injectivity}\label{SubSectionMetricAndTopologicalInjectivity}

Unless otherwise stated we shall consider injectivity of right modules.

\begin{definition}[\cite{HelMetrFrQMod}, 
    definition 4.3;~\cite{WhiteInjmoduAlg}, 
    definition 3.4]\label{MetCTopInjMod} 
An $A$-module $J$ is 
called $\langle$~metrically / $C$-topologically~$\rangle$ injective if for any 
$\langle$~isometric / $c$-topologically injective~$\rangle$
$A$-morphism $\xi:Y\to X$ and for any $A$-morphism $\phi:Y\to J$ 
there exists an $A$-morphism $\psi:X\to J$ such that $\psi\xi=\phi$  and
$\langle$~$\Vert\psi\Vert\leq\Vert\phi\Vert$ / 
$\Vert \psi\Vert\leq c C\Vert\phi\Vert$~$\rangle$. We say that an $A$-module 
$J$ is topologically injective if it is $C$-topologically injective for 
some $C\geq 1$.
\end{definition}

The task of constructing an $A$-morphism $\psi$ for a given $A$-morphisms $\phi$ 
and $\xi$ in the definition~\ref{MetCTopInjMod} is called an extension problem 
and $\psi$ is called an extension of $\phi$ along $\xi$.

A short but more involved equivalent definition 
of $\langle$~metric / $C$-topological~$\rangle$ injectivity is the following: 
an $A$-module $J$ is called $\langle$~metrically / $C$-topologically~$\rangle$ 
injective, if the functor
$\langle$~$\operatorname{Hom}_{A-\mathbf{mod}_1}(-,J)
:A-\mathbf{mod}_1\to\mathbf{Ban}_1$
/
$\operatorname{Hom}_{A-\mathbf{mod}}(-,J)
:A-\mathbf{mod}\to\mathbf{Ban}$~$\rangle$
maps $\langle$~isometric / $c$-topologically injective~$\rangle$
$A$-morphisms into strictly $\langle$~coisometric / 
$c C$-topologically surjective~$\rangle$ operators. 

In category $\langle$~$A-\mathbf{mod}_1$ / $A-\mathbf{mod}$~$\rangle$ there
is a special class of $\langle$~metrically / $1$-topologically~$\rangle$
injective modules of the form $\mathcal{B}(A_+, \ell_\infty(\Lambda))$ for 
some set $\Lambda$. They are called cofree modules. These modules play a crucial 
role in our studies of injectivity.

\begin{proposition}[\cite{WhiteInjmoduAlg}, 
    lemma 3.6]\label{MetCTopCofreeMod} Let $\Lambda$ be an arbitrary 
set, then the left $A$-modules $\mathcal{B}(A_+, \ell_\infty(\Lambda))$ 
and $\mathcal{B}(A_\times, \ell_\infty(\Lambda))$
are $\langle$~metrically / $1$-topologically~$\rangle$ injective. 
\end{proposition}
\begin{proof} By $A_\bullet$ we denote either $A_+$ or $A_\times$.
Consider arbitrary $A$-morphism 
$\phi:Y\to \mathcal{B}(A_\bullet, \ell_\infty(\Lambda))$ and 
$\langle$~an isometric / $c$-topologically injective~$\rangle$ 
$A$-morphism $\xi:Y\to X$. Fix arbitrary $\lambda\in\Lambda$ and define 
a bounded linear functional 
$h_\lambda:Y\to\mathbb{C}:y\mapsto \phi(y)(e_{A_\times})(\lambda)$. 
Clearly, $\Vert h_\lambda\Vert\leq\Vert \phi\Vert$.
Denote $X_0=\operatorname{Im}(\xi)$ and consider 
$A$-morphism $\eta=\xi|^{X_0}$. Since $\xi$ is 
$\langle$~an isometric / $c$-topologically injective~$\rangle$, then $X_0$ is
closed and $\eta$ has a left inverse bounded linear operator $\zeta:X_0\to Y$, 
such that $\zeta$ has norm at most $\langle$~$K=1$ / $K=c$~$\rangle$. 
Now consider a bounded linear functional  $f_\lambda=h_\lambda\zeta\in X_0^*$. 
By Hahn-Banach theorem we can extend $f_\lambda$ to some bounded linear functional
$g_\lambda:X\to\mathbb{C}$ with norm  
$\Vert g_\lambda\Vert
=\Vert f_\lambda\Vert
\leq\Vert h_\lambda\Vert\Vert\zeta\Vert
\leq K\Vert \phi\Vert$. 
Consider $A$-morphism 
$\psi
:X\to \mathcal{B}(A_\bullet, \ell_\infty(\Lambda))
:x\mapsto (a\mapsto (\lambda\mapsto g_\lambda(x\cdot a)))$. 
It is easy to check that $\psi\xi=\phi$ and 
$\Vert\psi\Vert\leq K\Vert\phi\Vert$. Thus, for a given $\phi$ 
and $\xi$ we have constructed an $A$-morphism $\psi$ such that $\psi\xi=\phi$ 
and $\langle$~$\Vert\psi\Vert\leq\Vert\phi\Vert$ /
$\Vert\psi\Vert\leq c\Vert\phi\Vert$~$\rangle$. Hence, the 
$A$-module $\mathcal{B}(A_\bullet, \ell_\infty(\Lambda))$ is 
$\langle$~metrically / $1$-topologically~$\rangle$ injective.
\end{proof}


\begin{proposition}\label{DualOfUnitalAlgIsMetTopInj} The right $A$-module
$A_\times^*$ is metrically and $1$-topologically injective.
\end{proposition}
\begin{proof} Consider set $\Lambda=\mathbb{N}_1$. 
By proposition~\ref{MetCTopCofreeMod} the 
$A$-module $\mathcal{B}(A_\times, \ell_\infty(\Lambda))$ is metrically 
and $1$-topologically injective. Now it remains to note that 
$\mathcal{B}(A_\times, \ell_\infty(\Lambda))
\isom{\mathbf{mod}_1-A}\\
\mathcal{B}(A_\times, \mathbb{C})
\isom{\mathbf{mod}_1-A}
A_\times^*$.
\end{proof}

\begin{proposition}[\cite{WhiteInjmoduAlg}, 
    lemma 3.7]\label{RetrMetCTopInjIsMetCTopInj} Any 
$\langle$~$1$-retract / $c$-retract~$\rangle$ of
$\langle$~metrically / $C$-topologically~$\rangle$ injective module is
$\langle$~metrically / $c C$-topologically~$\rangle$ injective.
\end{proposition}
\begin{proof} Suppose that $J$ is a $c$-retract of 
$\langle$~metrically / $C$-topologically~$\rangle$ injective $A$-module $J'$.
Then there exist $A$-morphisms $\eta:J\to J'$ and $\zeta: J'\to J$ such that
$\zeta\eta=1_{J}$ and $\Vert\zeta\Vert\Vert\eta\Vert\leq c$ 
for $\langle$~$c=1$ / $c\geq 1$~$\rangle$. Consider 
arbitrary $\langle$~isometric / $c'$-topologically injective~$\rangle$ 
$A$-morphism $\xi:Y\to X$ and an arbitrary $A$-morphism $\phi:Y\to J$. 
Consider $A$-morphism $\phi'=\eta\phi$. Since $J'$ is 
$\langle$~metrically / $C$-topologically~$\rangle$ injective, then there 
exists an $A$-morphism $\psi':X\to J'$ such that $\phi'=\psi'\xi$ 
and $\Vert\psi'\Vert \leq K\Vert\phi'\Vert$ 
for $\langle$~$K=1$ / $K=c' C$~$\rangle$. Now it is routine to check that 
for the $A$-morphism $\psi=\zeta\psi'$ 
holds $\psi\xi=\phi$ and $\Vert\psi\Vert \leq cK\Vert\phi\Vert$. 
Thus, for a given $\phi$ and $\xi$ we have
constructed an $A$-morphism $\psi$ such that $\psi\xi=\phi$  and
$\langle$~$\Vert\psi\Vert\leq\Vert\phi\Vert$ / 
$\Vert \psi\Vert\leq c C\Vert\phi\Vert$~$\rangle$. Hence, $J$ 
is $\langle$~metrically / $c C$-topologically~$\rangle$ injective.
\end{proof}


It is easy to show that, for any $A$-module $X$ there exists 
$\langle$~an isometric / a $1$-topologically injective~$\rangle$ 
$A$-morphism
$$
\rho_X^+
:X\to\mathcal{B}(A_+, \ell_\infty(B_{X^*}))
:x\mapsto(a\mapsto(f\mapsto f(x\cdot a)))
$$

\begin{proposition}[\cite{WhiteInjmoduAlg}, 
    proposition 3.10]\label{MetCTopInjModViaCanonicMorph} An $A$-module $J$ is
$\langle$~metrically / $C$-topologically~$\rangle$ injective iff $\rho_J^+$ is 
a $\langle$~$1$-retraction / $C$-retraction~$\rangle$ in $A-\mathbf{mod}$.
\end{proposition}
\begin{proof}
Suppose $J$ is 
$\langle$~metrically / $C$-topologically~$\rangle$ injective, then consider 
$\langle$~isometric / $1$-topologically injective~$\rangle$
$A$-morphism $\rho_J^+$. Consider extension problem with $\phi=1_J$ 
and $\xi=\rho_J^+$, then from $\langle$~metric / $C$-topological~$\rangle$ 
injectivity of $J$ we get an $A$-morphism $\tau^+$ such 
that $\tau^+\rho_J^+=1_J$
and $\Vert\tau^+\Vert\leq K$ for $\langle$~$K=1$ / $K=C$~$\rangle$. 
Since $\Vert\rho_J^+\Vert\Vert \tau^+\Vert\leq K$ 
we conclude that $\rho_J^+$ is 
a $\langle$~$1$-retraction / $C$-retraction~$\rangle$ in $A-\mathbf{mod}$.

Conversely, assume that $\rho_J^+$ is 
a $\langle$~$1$-retraction / $C$-retraction~$\rangle$. In other words $J$ is 
a $\langle$~$1$-retract / $C$-retract~$\rangle$ of 
$\mathcal{B}(A_+, \ell_\infty(B_{J^*}))$. By proposition~\ref{MetCTopCofreeMod} 
the $A$-module $\mathcal{B}(A_+, \ell_\infty(B_{J^*}))$ 
is $\langle$~metrically / $1$-topologically~$\rangle$ injective. So from 
proposition~\ref{RetrMetCTopInjIsMetCTopInj} 
its $\langle$~$1$-retract / $C$-retract~$\rangle$ $P$ is 
$\langle$~metrically / $C$-topologically~$\rangle$ injective.
\end{proof}

\begin{proposition}[\cite{WhiteInjmoduAlg}, 
    proposition 3.9]\label{MetInjIsTopInjAndTopInjIsRelInj} Every metrically
injective module is $1$-topologically injective and every $C$-topologically 
injective module is $C$-relatively injective.
\end{proposition}
\begin{proof} By proposition~\ref{MetCTopInjModViaCanonicMorph} every 
metrically injective $A$-module $J$ is a $1$-retract 
of $\mathcal{B}(A_+, \ell_\infty(B_{J^*}))$. Hence, by the same proposition $J$ 
is $1$-topologically injective. Again by 
proposition~\ref{MetCTopInjModViaCanonicMorph} every $C$-topologically 
injective $A$-module $J$ is a $C$-retract 
of $A$-module $\mathcal{B}(A_+, \ell_\infty(B_{J^*}))$. 
In other words $J$ is a $C$-retract of the module $\mathcal{B}(A_+, E)$ for 
some Banach space $E$. Therefore, $J$ is $C$-relatively injective.
\end{proof}

Clearly, every $C$-topologically injective module is $C'$-topologically 
injective for $C'\geq C$. But we can state even more.

\begin{proposition}\label{MetInjIsOneTopInj} An $A$-module is metrically
injective iff it is $1$-topologically injective.
\end{proposition}
\begin{proof} The result immediately follows 
from propositions~\ref{MetInjIsTopInjAndTopInjIsRelInj} 
and~\ref{MetCTopInjModViaCanonicMorph}.
\end{proof}

Let us proceed to examples. If we regard the category of Banach spaces as the
category of right Banach modules over zero algebra, we may speak of
$\langle$~metrically / topologically~$\rangle$ injective Banach spaces. All
results mentioned above hold for this type of injectivity. An equivalent
definition says that a Banach space is $\langle$~metrically /
topologically~$\rangle$ injective if it is $\langle$~contractively complemented
/ complemented~$\rangle$ in any ambient Banach space. The typical examples of
metrically injective Banach spaces are $L_\infty$-spaces. Only metrically
injective Banach spaces are completely understood --- these spaces are
isometrically isomorphic to $C(K)$-space for some extremely disconnected compact
Hausdorff space $K$ [\cite{LaceyIsomThOfClassicBanSp}, theorem 3.11.6]. Usually
such topological spaces are referred to as Stonean spaces.  For the contemporary
results on topologically injective Banach spaces see
[\cite{JohnLinHandbookGeomBanSp}, chapter 40].

\begin{proposition}\label{NonDegenMetTopInjCharac}  Let $J$ be a faithful
$A$-module. Then $J$ is $\langle$~metrically / $C$-topologically~$\rangle$
injective iff the map
$\rho_J:J\to\mathcal{B}(A,\ell_\infty(B_{J^*})):x\mapsto(a\mapsto(f\mapsto
f(x\cdot a)))$ is a $\langle$~$1$-coretraction / $C$-coretraction~$\rangle$ in
$\mathbf{mod}-A$.
\end{proposition} 
\begin{proof}
If $J$ is $\langle$~metrically / $C$-topologically~$\rangle$ injective, then by
proposition~\ref{MetCTopInjModViaCanonicMorph} the $A$-morphism $\rho_J^+$ has
right inverse morphism $\tau^+$, with norm $\langle$~at most $1$ / at most
$C$~$\rangle$. Assume we are given an operator $T\in
\mathcal{B}(A_+,\ell_\infty(B_{J^*}))$, such that $T|_A=0$. Fix $a\in A$, then
$T\cdot a=0$, and so $\tau^+(T)\cdot a=\tau^+(T\cdot a)=0$. Since $J$ is
faithful and $a\in A$ is arbitrary, then $\tau^+(T)=0$. Define $p:A_+\to A$ be
the natural projection from $A_+$ onto $A$, then define $A$-morphisms
$j=\mathcal{B}(p,\ell_\infty(B_{J^*}))$ and $\tau =\tau^+ j$. For any $a\in A$
and $T\in\mathcal{B}(A,\ell_\infty(B_{J^*}))$ 
we have $\tau (T\cdot a)-\tau (T)\cdot a=\tau^+(j(T\cdot a)-j(T)\cdot a)=0$, 
because $j(T\cdot a)-j(T)\cdot a|_A=0$. Therefore, $\tau $ is an $A$-morphism. 
Note that $\Vert\tau \Vert\leq\Vert\tau^+\Vert\Vert j\Vert\leq\Vert\tau^+\Vert$.
Therefore, $\tau$ has norm $\langle$~at most $1$ / at most $C$~$\rangle$. 
Obviously, for all $x\in J$ we have $\rho_J^+(x)-j(\rho_J(x))|_A=0$, 
so $\tau^+(\rho_J^+(x)-j(\rho_J(x)))=0$. 
As a consequence $\tau (\rho_J(x))=\tau^+(j(\rho_J(x)))=\tau^+(\rho_J^+(x))=x$ 
for all $x\in J$. Since $\tau \rho_J=1_J$, then $\rho_J$ 
is a  $\langle$~$1$-coretraction / $C$-coretraction~$\rangle$ 
in $\mathbf{mod}-A$.

Conversely, assume $\rho_J$ is a $\langle$~$1$-coretraction /
$C$-coretraction~$\rangle$, that is has a right inverse morphism $\tau $ with
norm $\langle$~at most $1$ / at most $C$~$\rangle$. Define $i:A\to A_+$ to be
the natural embedding of $A$ into $A_+$ and define $A$-morphism
$q=\mathcal{B}(i,\ell_\infty(B_{J^*}))$. Obviously, $\rho_J=q\rho_J^+$. Consider
$A$-morphism $\tau^+=\tau q$. 
Note that $\Vert\tau^+\Vert\leq\Vert\tau \Vert\Vert q\Vert\leq \Vert\tau \Vert$.
Therefore, $\tau^+$ has norm $\langle$~at most $1$ / at most $C$~$\rangle$. 
Clearly $\tau^+\rho_J^+=\tau q\rho_J^+=\tau \rho_J=1_J$. So $\rho_J^+$ 
is a $\langle$~$1$-coretraction / $C$-coretraction~$\rangle$ and by 
proposition~\ref{MetCTopInjModViaCanonicMorph} the $A$-module $J$ is
$\langle$~metrically / $C$-topologically~$\rangle$ injective.
\end{proof}

It is worth to mention that $\langle$~arbitrary / only finite~$\rangle$ family
of objects in $\langle$~$\mathbf{mod}_1-A$ / $\mathbf{mod}-A$~$\rangle$ have the
categorical product which in fact is their $\bigoplus_\infty$-sum. This is the
reason why we make additional assumption in the second paragraph of the next
proposition.

\begin{proposition}\label{MetTopInjModProd} Let
${(J_\lambda)}_{\lambda\in\Lambda}$ be a family of $A$-modules. Then 

\begin{enumerate}[label = (\roman*)]
    \item $\bigoplus_\infty \{J_\lambda:\lambda\in\Lambda \}$ is metrically
    injective iff for all $\lambda\in\Lambda$ the $A$-module $J_\lambda$ is
    metrically injective;

    \item $\bigoplus_\infty \{J_\lambda:\lambda\in\Lambda \}$ is
    $C$-topologically injective iff for all $\lambda\in\Lambda$ the $A$-module
    $J_\lambda$ is a $C$-topologically injective.
\end{enumerate}
\end{proposition}
\begin{proof} Denote $J:=\bigoplus_\infty \{J_\lambda:\lambda\in\Lambda \}$.

$(i)$ The proof is literally the same as in paragraph $(ii)$.

$(ii)$ Assume that $J$ is $C$-topologically injective. Note that, for any
$\lambda\in\Lambda$ the $A$-module $J_\lambda$ is a $1$-retract of $J$ via
natural projection $p_\lambda:J\to J_\lambda$. By
proposition~\ref{RetrMetCTopInjIsMetCTopInj} the $A$-module $J_\lambda$ is
$C$-topologically injective.

Conversely, let each $A$-module $J_\lambda$ be $C$-topologically injective. By
proposition~\ref{MetCTopInjModViaCanonicMorph} we have a family of
$C$-coretractions
$\rho_\lambda:J_\lambda\to\mathcal{B}(A_+,\ell_\infty(S_\lambda))$. It follows
that $\bigoplus_\infty \{\rho_\lambda:\lambda\in\Lambda \}$ is a
$C$-coretraction in $A-\mathbf{mod}$. As the result $J$ is a $C$-retract of 
$$
\bigoplus\nolimits_\infty \{
    \mathcal{B}(A_+,\ell_\infty(S_\lambda)):\lambda\in\Lambda
 \}
\isom{\mathbf{mod}_1-A}
\bigoplus\nolimits_\infty\left \{
    \bigoplus\nolimits_\infty \{ A_+^*:s\in S_\lambda \}:\lambda\in\Lambda
\right \}
\isom{\mathbf{mod}_1-A}
$$
$$
\bigoplus\nolimits_\infty \{A_+^*:s\in S \}
\isom{\mathbf{mod}_1-A}
\mathcal{B}(A_+,\ell_\infty(S))
$$
in $\mathbf{mod}-A$, where $S=\bigsqcup_{\lambda\in\Lambda}S_\lambda$. Clearly,
the latter module is $1$-topologically injective, so by
proposition~\ref{RetrMetCTopInjIsMetCTopInj} the $A$-module $J$ is $C$-topologically
injective.
\end{proof}

\begin{corollary}\label{MetTopInjlInftySum} Let $J$ be an $A$-module and
$\Lambda$ be an arbitrary set. Then $\bigoplus_\infty \{J:\lambda\in\Lambda \}$
is $\langle$~metrically / $C$-topologically~$\rangle$ injective iff $J$ is
$\langle$~metrically / $C$-topologically~$\rangle$ injective.
\end{corollary}
\begin{proof} The result immediately follows from
proposition~\ref{MetTopInjModProd} if one set $J_\lambda=J$ 
for all $\lambda\in\Lambda$.
\end{proof}

\begin{proposition}\label{MapsFroml1toMetTopInj} Let $J$ be an $A$-module and
$\Lambda$ be an arbitrary set. Then $\mathcal{B}(\ell_1(\Lambda),J)$ is
$\langle$~metrically / $C$-topologically~$\rangle$ injective iff $J$ is
$\langle$~metrically / $C$-topologically~$\rangle$ injective.
\end{proposition}
\begin{proof} 
Assume $\mathcal{B}(\ell_1(\Lambda), J)$ is $\langle$~metrically /
$C$-topologically~$\rangle$ injective. Take any $\lambda\in\Lambda$ and consider
contractive $A$-morphisms
$i_\lambda:J\to\mathcal{B}(\ell_1(\Lambda),J):x\mapsto(f\mapsto f(\lambda)x)$
and $p_\lambda:\mathcal{B}(\ell_1(\Lambda),J)\to J:T\mapsto T(\delta_\lambda)$.
Clearly, $p_\lambda i_\lambda=1_J$, so by 
proposition~\ref{RetrMetCTopInjIsMetCTopInj} the $A$-module $J$ 
is $\langle$~metrically / $C$-topologically~$\rangle$ injective as $1$-retract 
of $\langle$~metrically / $1$-topologically~$\rangle$ injective 
$A$-module $\mathcal{B}(\ell_1(\Lambda),J)$.

Conversely, since $J$ is $\langle$~metrically / $C$-topologically~$\rangle$
injective, by proposition~\ref{MetCTopInjModViaCanonicMorph} 
the $A$-morphism $\rho_J^+$ is a
$\langle$~$1$-coretraction / $C$-coretraction~$\rangle$. Apply the functor
$\mathcal{B}(\ell_1(\Lambda),-)$ to this coretraction to get another
$\langle$~$1$-coretraction / $C$-coretraction~$\rangle$ denoted by
$\mathcal{B}(\ell_1(\Lambda),\rho_J^+)$. Note that 
$$
\mathcal{B}(\ell_1(\Lambda),\ell_\infty(B_{J^*}))
\isom{\mathbf{Ban}_1}
{(\ell_1(\Lambda)\projtens \ell_1(B_{J^*}))}^*
\isom{\mathbf{Ban}_1}
{\ell_1(\Lambda\times B_{J^*})}^*
\isom{\mathbf{Ban}_1}
\ell_\infty(\Lambda\times B_{J^*}),
$$ 
so we have isometric isomorphisms of Banach modules
$$
\mathcal{B}(\ell_1(\Lambda),\mathcal{B}(A_+,\ell_\infty(B_{J^*})))
\isom{\mathbf{mod}_1-A}
\mathcal{B}(A_+,\mathcal{B}(\ell_1(\Lambda),\ell_\infty(B_{J^*}))
\isom{\mathbf{mod}_1-A}
\mathcal{B}(A_+,\ell_\infty(\Lambda\times B_{J^*})).
$$ 
Therefore, $\mathcal{B}(\ell_1(\Lambda),J)$ is a $\langle$~$1$-retract /
$C$-retract~$\rangle$ of $\langle$~metrically / $1$-topologically~$\rangle$
injective $A$-module $\mathcal{B}(A_+,\ell_\infty(\Lambda\times B_{J^*}))$. By
proposition~\ref{RetrMetCTopInjIsMetCTopInj} the $A$-module
$\mathcal{B}(\ell_1(\Lambda), J)$ is $\langle$~metrically /
$C$-topologically~$\rangle$ injective.
\end{proof}

The property of being metrically, topologically or relatively injective module 
puts some restrictions on the Banach geometric structure of the module.

\begin{proposition}[\cite{RamsHomPropSemgroupAlg}, 
    proposition 2.2.1]\label{MetTopRelInjModCompIdealAnnihCompl} Let $J$ 
be a $\langle$~metrically / $C$-topologically / $C$-relatively~$\rangle$ 
injective $A$-moudle, and let $I$ 
be a $\langle$~$1$-complemented / $c$-complemented / $c$-complemented~$\rangle$
right ideal of $A$. Then $J^{\perp I}$ 
is $\langle$~$2$-complemented / $1+cC$-complemented / 
$1+cC$-complemented~$\rangle$ in $J$.
\end{proposition}
\begin{proof} Since $A$ is $1$-complemented in $A_+$, then $I$ 
is $\langle$~$1$-complemented / $c$-complemented / $c$-complemented~$\rangle$
in $A_+$. Hence, there exists a bounded linear operator $r:A_+\to I$ of 
$\langle$~norm $1$ / norm $c$ / norm $c$~$\rangle$ such that $r|_I=1_I$. 
Since $J$ is $\langle$~metrically / $C$-topologically / $C$-relatively~$\rangle$ 
injective, by proposition~\ref{MetCTopInjModViaCanonicMorph} the 
morphism $\rho_J^+$ has a left inverse $A$-morphism $\tau^+$ of norm 
$\langle$~at most $1$ / at most $C$ / at most $C$~$\rangle$. Now consider 
a bounded linear operator $p:A\to A:x\mapsto x-\tau^+(\rho_J^+(x)r)$. Clearly,
$\Vert p\Vert\leq 1+\Vert \tau^+\Vert\Vert r\Vert$. 
Since $\operatorname{Im}(r)\subset I$, then for all $x\in J^{\perp I}$ we 
have $\rho_J^+(x)r=0$. Hence, $p(x)=x$ for all $x\in J^{\perp I}$. Since $I$ is 
a right ideal, then one can check that for all $x\in J$ and $a\in I$ holds 
$(\rho_J^+(x)r)\cdot a=\rho_J^+(x\cdot a)$. As a consequence 
$$
p(x)\cdot a
=x\cdot a-\tau^+(\rho_J^+(x)r)\cdot a
=x\cdot a-\tau^+((\rho_J^+(x)r)\cdot a)
=x\cdot a-\tau^+(\rho_J^+(x\cdot a))
=0.
$$
In other words $\operatorname{Im}(p)\subset J^{\perp I}$. Thus, $p$ is 
projection onto $J^{\perp I}$ with norm at 
most $1+\Vert \tau^+\Vert\Vert r\Vert$. Hence, $J^{\perp I}$ 
is $\langle$~$2$-complemented / $1+cC$-complemented / 
$1+cC$-complemented~$\rangle$ complemented in $J$.
\end{proof}


\begin{corollary}\label{MetTopRelInjModAnnihCompl} Let $J$ 
be a $\langle$~metrically / $C$-topologically / $C$-relatively~$\rangle$ 
injective $A$-moudle. Then $J_{ann}$ 
is $\langle$~$2$-complemented / $1+C$-complemented / 
$1+C$-complemented~$\rangle$ in $J$.
\end{corollary}
\begin{proof} The result directly follows from 
proposition~\ref{MetTopRelInjModCompIdealAnnihCompl}.
\end{proof}

%-------------------------------------------------------------------------------
%    Metric and topological flatness
%-------------------------------------------------------------------------------

\subsection{
    Metric and topological flatness}\label{SubSectionMetricAndTopologicalFlatness}

The definition of metrically flat modules was given by Helemeskii 
in~\cite{HelMetrFlatNorMod} under the name of extremely flat modules. The 
notion of topological flatness was first studied by 
White in~\cite{WhiteInjmoduAlg}. We use a somewhat different, but equivalent  
terminology. Unfortunately, his definition was erroneous, meanwhile the results 
were correct. So we take responsibility to fix the definition.

\begin{definition}[\cite{HelMetrFlatNorMod}, 
    definition I;~\cite{WhiteInjmoduAlg}, definition 4.4]\label{MetCTopFlatMod} 
A left $A$-module $F$ is 
called $\langle$~metrically / $C$-topologically~$\rangle$ 
flat if for each $\langle$~isometric / $c$-topologically injective~$\rangle$ 
$A$-morphism $\xi:X\to Y$ of right $A$-modules the 
operator $\xi\projmodtens{A} 1_F:X\projmodtens{A} F\to Y\projmodtens{A} F$ 
is $\langle$~isometric / $c C$-topologically injective~$\rangle$. We say that an 
$A$-module $F$ is topologically flat if it is $C$-topologically flat for 
some $C\geq 1$.
\end{definition}

A short but more involved definition is the following: an $A$-module $F$ is
called $\langle$~metrically / $C$-topologically~$\rangle$ flat, if the functor
$\langle$~$-\projmodtens{A}F:A-\mathbf{mod}_1\to\mathbf{Ban}_1$ /
$-\projmodtens{A}F:A-\mathbf{mod}\to\mathbf{Ban}$~$\rangle$ maps
$\langle$~isometric / $c$-topologically injective~$\rangle$ $A$-morphisms into
$\langle$~isometric / $c C$-topologically injective~$\rangle$ operators.

The key result in the study of flatness is the following.

\begin{proposition}[\cite{WhiteInjmoduAlg}, lemma 4.10]\label{MetCTopFlatCharac}
An $A$-module $F$ is
$\langle$~metrically / $C$-topologically~$\rangle$ flat iff $F^*$ is
$\langle$~metrically / $C$-topologically~$\rangle$ injective.
\end{proposition}
\begin{proof} Consider any $\langle$~isometric / $c$-topologically
injective~$\rangle$ morphism of right $A$-modules, call it $\xi:X\to Y$. The
operator $\xi\projmodtens{A} 1_F$ is $\langle$~isometric / $c C$-topologically
injective~$\rangle$ iff the adjoint operator ${(\xi\projmodtens{A} 1_F)}^*$ is
strictly $\langle$~coisometric / $c C$-topologically surjective~$\rangle$. 
Since operators ${(\xi\projmodtens{A} 1_F)}^*$ and $\mathcal{B}_A(\xi,F^*)$ 
are equivalent in $\mathbf{Ban}_1$ via universal property of projective module 
tensor product, then we get that $\xi\projmodtens{A} 1_F$ 
is $\langle$~isometric / $c C$-topologically injective~$\rangle$ 
iff $\mathcal{B}_A(\xi,F^*)$ is 
strictly $\langle$~coisometric / $c C$-topologically surjective~$\rangle$. 
Since $\xi$ is arbitrary we conclude that $F$ 
is $\langle$~metrically / $C$-topologically~$\rangle$ flat iff
$F^*$ is $\langle$~metrically / $C$-topologically~$\rangle$ injective.
\end{proof}

This characterization allows one to prove many properties of flat modules by 
considering respective duals.

\begin{proposition}\label{RetrMetCTopFlatIsMetCTopFlat} 
Any $\langle$~$1$-retract / $c$-retract~$\rangle$ of a
$\langle$~metrically / $C$-topologically~$\rangle$ flat module is 
$\langle$~metrically / $cC$-topologically~$\rangle$ flat.
\end{proposition}
\begin{proof} The result follows from propositions~\ref{MetCTopFlatCharac} 
and~\ref{RetrMetCTopInjIsMetCTopInj}
\end{proof}

\begin{proposition}\label{MetFlatIsTopFlatAndTopFlatIsRelFlat} Every metrically
flat module is $1$-topologically flat and every $C$-topologically flat module is
$C$-relatively flat.
\end{proposition}
\begin{proof} The result follows from propositions~\ref{MetCTopFlatCharac} 
and~\ref{MetInjIsTopInjAndTopInjIsRelInj}.
\end{proof}

\begin{proposition}\label{MetFlatIsOneTopFlat} An $A$-module is metrically flat
iff it is $1$-topologically flat.
\end{proposition}
\begin{proof} The result follows from propositions~\ref{MetCTopFlatCharac}
and~\ref{MetInjIsOneTopInj}.
\end{proof}

Loosely speaking flatness is ``projectivity with respect to second duals''. 
We can give this statement a precise meaning.

\begin{proposition}\label{MetTopFlatSecondDualCharac} Let $F$ be a left Banach
$A$-module. Then the following are equivalent:

\begin{enumerate}[label = (\roman*)]
    \item $F$ is $\langle$~metrically / $C$-topologically~$\rangle$ flat;

    \item for any strictly $\langle$~coisometric / $c$-topologically
    surjective~$\rangle$ $A$-morphism $\xi:X\to Y$ and for any $A$-morphism
    $\phi:F\to Y$ there exists an $A$-morphism $\psi:P\to X^{**}$ such that
    $\xi^{**}\psi=\iota_Y\phi$ and $\langle$~$\Vert\psi\Vert=\Vert\phi\Vert$ /
    $\Vert\psi\Vert\leq cC\Vert\phi\Vert$~$\rangle$;

    \item there exists an $A$-morphism $\sigma:F\to {(A_+\projtens
    \ell_1(B_F))}^{**}$ with norm $\langle$~at most $1$ / at most $C$~$\rangle$
    such that ${(\pi_F^+)}^{**}\sigma=\iota_F$.
\end{enumerate}
\end{proposition}
\begin{proof} $(i)\implies (ii)$ Again, consider arbitrary strictly 
$\langle$~coisometric / $c$-topologically surjective~$\rangle$ $A$-morphism
$\xi:X\to Y$ and arbitrary $A$-morphism $\phi:F\to Y$. By
[\cite{HelLectAndExOnFuncAn}, exercise 4.4.6] we know that $\xi^*$ is
$\langle$~isometric / $c$-topologically injective~$\rangle$. Since $F$ is
$\langle$~metrically / $C$-topologically~$\rangle$ flat then
$(\xi^*\projmodtens{A}1_F)$ is $\langle$~isometric / $cC$-topologically
injective~$\rangle$ too. Therefore, ${(\xi^*\projmodtens{A}1_F)}^*$ is
$\langle$~strictly coisometric / strictly $cC$-topologically
surjective~$\rangle$ by [\cite{HelLectAndExOnFuncAn}, exercise 4.4.7]. Note that
${(\xi^*\projmodtens{A}1_F)}^*$ and $\mathcal{B}_A(F,\xi^{**})$ are equivalent
in $\mathbf{Ban}_1$ thanks to the law of adjoint associativity. So
$\mathcal{B}_A(F,\xi^{**})$ is strictly $\langle$~coisometric / 
$cC$-topologically surjective~$\rangle$ too. The latter implies that for the
operator $\iota_Y\phi$ we can find an $A$-morphism $\psi:Y\to X^{**}$ such that
$\xi^{**}\psi=\iota_Y\phi$ and
$\langle$~$\Vert\psi\Vert=\Vert\iota_Y\phi\Vert=\Vert\phi\Vert$ /
$\Vert\psi\Vert\leq cC\Vert\iota_Y\phi\Vert=cC\Vert\phi\Vert$~$\rangle$

$(ii)\implies (iii)$ Set $\xi=\pi_F^+$ and $\phi=1_F$. Since $\xi$ is
strictly $\langle$~coisometric/ $1$-topologically surjective~$\rangle$,
then from assumption we get an $A$-morphism $\sigma:F\to
{(A_+\projtens\ell_1(B_F))}^{**}$ such that 
${(\pi_F^+)}^{**}\sigma
=\iota_F 1_F
=\iota_F$ 
and $\langle$~$\Vert\sigma\Vert\leq \Vert\phi\Vert=1$ /
$\Vert\sigma\Vert\leq 1\cdot C\Vert\phi\Vert=C$~$\rangle$.

$(iii)\implies (i)$ Let $\sigma$ be a right inverse $A$-morphism for
${(\pi_F^+)}^{**}$ with norm $\langle$~at most $1$ / at most $C$~$\rangle$.
Consider $A$-morphism $\tau=\sigma^*\iota_{{(A_+\projtens\ell_1(B_F))}^*}$.
Clearly, its norm is $\langle$~at most $1$ / at most $C$~$\rangle$. For any
$f\in F^*$ and $x\in F$ we have
$$
({\tau(\pi_F^+)}^*)(f)(x)
=\sigma^*(\iota_{{(A_+\projtens\ell_1(B_F))}^*}({(\pi_F^+)}^*(f)))(x)
=\iota_{{(A_+\projtens\ell_1(B_F))}^*}({(\pi_F^+)}^*(f))(\sigma(x))
$$
$$
=\sigma(x)({(\pi_F^+)}^*(f))
={(\pi_F^+)}^{**}(\sigma(x))(f)
=\iota_F(x)(f)
=f(x)
$$
So ${\tau(\pi_F^+)}^*=1_{F^*}$, which means $F^*$ is a with norm
$\langle$~$1$-retract / $C$-retract~$\rangle$ of
${(A_+\projtens\ell_1(B_F))}^*$. The latter module is $\langle$~metrically /
$1$-topologically~$\rangle$ injective, because 
$$
{(A_+\projtens\ell_1(B_F))}^*
\isom{\mathbf{mod}_1-A}
\mathcal{B}(A_+,\ell_\infty(B_F)).
$$ 
By proposition~\ref{RetrMetCTopInjIsMetCTopInj} the $A$-module $F^*$ is
$\langle$~metrically / $C$-topologically~$\rangle$ injective. By 
proposition~\ref{MetCTopFlatCharac} this is
equivalent to $\langle$~metric / $C$-topological~$\rangle$ flatness of $F$.
\end{proof}

Let us proceed to examples. Consider the category of Banach spaces as 
the category of left Banach modules over zero algebra, then we get the 
definition of $\langle$~metrically / topologically~$\rangle$ flat Banach space. 
From Grothendieck's paper~\cite{GrothMetrProjFlatBanSp} it follows that any 
metrically flat Banach space is isometrically isomorphic to $L_1(\Omega,\mu)$ 
for some measure space $(\Omega,\Sigma,\mu)$. For topologically flat Banach 
spaces, in contrast with topologically injective ones, we also have a criterion
[\cite{DefFloTensNorOpId}, corollary 23.5(1)]: a Banach space is topologically
flat iff it is an $\mathscr{L}_1^g$-space.

\begin{proposition}\label{MetTopFlatModCoProd} Let
${(F_\lambda)}_{\lambda\in\Lambda}$ be family of $A$-modules. Then 

\begin{enumerate}[label = (\roman*)]
    \item $\bigoplus_1 \{F_\lambda:\lambda\in\Lambda \}$ is metrically flat iff
    for all $\lambda\in\Lambda$ the $A$-module $F_\lambda$ is metrically flat;

    \item $\bigoplus_1 \{F_\lambda:\lambda\in\Lambda \}$ is $C$-topologically
    flat iff for all $\lambda\in\Lambda$ the $A$-module $F_\lambda$ is
    $C$-topologically flat.
\end{enumerate}
\end{proposition}
\begin{proof} By proposition~\ref{MetCTopFlatCharac} an $A$-module $F$ 
is $\langle$~metrically / $C$-topologically~$\rangle$ flat iff $F^*$ 
is $\langle$~metrically / $C$-topologically~$\rangle$ injective. 
It remains to apply
proposition~\ref{MetTopInjModProd} with $J_\lambda=F_\lambda^*$ for all
$\lambda\in\Lambda$ and recall that 
${\left(\bigoplus_1 \{
    F_\lambda:\lambda\in\Lambda \}
\right)}^*
\isom{\mathbf{mod}_1-A}
\bigoplus_\infty \{ F_\lambda^*:\lambda\in\Lambda \}$.
\end{proof}

The following propositions demonstrate a close relationship between flatness and
projectivity.

\begin{proposition}\label{DualMetTopProjIsMetrInj} Let $P$ be a
$\langle$~metrically / $C$-topologically~$\rangle$ projective $A$-module, and
$\Lambda$ be an arbitrary set. Then $\mathcal{B}(P,\ell_\infty(\Lambda))$ is
$\langle$~metrically / $C$-topologically~$\rangle$ injective $A$-module. In
particular, $P^*$ is $\langle$~metrically / $C$-topologically~$\rangle$
injective $A$-module.
\end{proposition}
\begin{proof} From proposition~\ref{MetCTopProjModViaCanonicMorph} we know 
that $\pi_P^+$ is a $\langle$~$1$-retraction / $C$-retraction~$\rangle$. 
Then the $A$-morphism
$\rho^+=\mathcal{B}(\pi_P^+,\ell_\infty(\Lambda))$ is a
$\langle$~$1$-coretraction / $C$-coretraction~$\rangle$. Note that, 
$$
\mathcal{B}(A_+\projtens\ell_1(B_P),\ell_\infty(\Lambda))
\isom{\mathbf{mod}_1-A}
\mathcal{B}(A_+,\mathcal{B}(\ell_1(B_P),\ell_\infty(\Lambda)))
\isom{\mathbf{mod}_1-A}
\mathcal{B}(A_+,\ell_\infty(B_P\times\Lambda)).
$$ 
Thus, we showed that $\rho^+$ is a $\langle$~$1$-coretraction /
$C$-coretraction~$\rangle$ from $\mathcal{B}(P,\ell_\infty(\Lambda))$ into
$\langle$~metrically / $1$-topologically~$\rangle$ injective $A$-module. By
proposition~\ref{RetrMetCTopInjIsMetCTopInj} the $A$-module
$\mathcal{B}(P,\ell_\infty(\Lambda))$ is $\langle$~metrically /
$C$-topologically~$\rangle$ injective. To prove the last claim, just set
$\Lambda=\mathbb{N}_1$.
\end{proof}

\begin{proposition}\label{MetTopProjIsMetTopFlat} Every $\langle$~metrically /
$C$-topologically~$\rangle$ projective module is $\langle$~metrically /
$C$-topologically~$\rangle$ flat.
\end{proposition}
\begin{proof} The result follows from propositions~\ref{MetCTopFlatCharac}
and~\ref{DualMetTopProjIsMetrInj}.
\end{proof}

The property of being metrically, topologically or relatively flat module 
puts some restrictions on the Banach geometric structure of the module.

\begin{proposition}[\cite{RamsHomPropSemgroupAlg}, 
    corollary 2.2.2]\label{MetTopRelFlatModCompIdealPartCompl} Let $F$ 
be a $\langle$~metrically / $C$-topologically / $C$-relatively~$\rangle$ 
flat $A$-moudle, and let $I$ 
be a $\langle$~$1$-complemented / $c$-complemented / $c$-complemented~$\rangle$
right ideal of $A$. Then $\operatorname{cl}_F(I F)$ 
is weakly $\langle$~$2$-complemented / $1+cC$-complemented / 
$1+cC$-complemented~$\rangle$  in $F$.
\end{proposition}
\begin{proof} From 
propositions~\ref{MetCTopFlatCharac},~\ref{MetTopRelInjModCompIdealAnnihCompl} 
it follows that ${(F^*)}^{\perp I}$ is complemented in $F^*$. It remains to 
recall that ${(F^*)}^{\perp I}=\operatorname{cl}_F(I F)$. 
\end{proof}

\begin{corollary}\label{MetTopRelFlatModPartCompl} Let $F$ 
be a $\langle$~metrically / $C$-topologically / $C$-relatively~$\rangle$ 
flat $A$-moudle. Then $F_{ess}$ 
is weakly $\langle$~$2$-complemented / $1+C$-complemented / 
$1+C$-complemented~$\rangle$ in $F$.
\end{corollary}
\begin{proof} The result directly follows from 
proposition~\ref{MetTopRelFlatModCompIdealPartCompl}.
\end{proof}

%-------------------------------------------------------------------------------
%    Metric and topological flatness of ideals and cyclic modules
%-------------------------------------------------------------------------------

\subsection{
    Metric and topological flatness of ideals and cyclic modules}\label{SubSectionMetricAndTopologicalFlatnessOfIdealsAndCyclicModules}

In this section we study conditions under which ideals and cyclic modules are
metrically or topologically flat. The proofs are somewhat similar to those used
in the study of relative flatness of ideals and cyclic modules.

\begin{proposition}\label{MetTopFlatIdealsInUnitalAlg} Let $I$ be a left ideal
of $A_\times $ and $I$ has a right $\langle$~contractive / $c$-bounded~$\rangle$
approximate identity. Then $I$ is $\langle$~metrically /
$c$-topologically~$\rangle$ flat.
\end{proposition}
\begin{proof} Let $\mathfrak{F}$ be the section filter on $N$ and let
$\mathfrak{U}$ be an ultrafilter dominating $\mathfrak{F}$. 
For a fixed $f\in I^*$ 
and $a\in A_\times $ we have 
$|f(a e_\nu)|
\leq\Vert f\Vert\Vert a\Vert\Vert e_\nu\Vert
\leq c\Vert f\Vert\Vert a\Vert$ 
i.e. ${(f(ae_\nu))}_{\nu\in N}$ is a bounded net of complex numbers. 
Therefore, we have a well-defined limit $\lim_{\mathfrak{U}}f(ae_\nu)$ along 
ultrafilter $\mathfrak{U}$. Now it is routine to check that 
$\sigma:A_\times ^*\to I^*:f\mapsto (a\mapsto \lim_{\mathfrak{U}}f(ae_\nu))$ 
is an $A$-morphism with norm $\langle$~at most $1$ / at most $c$~$\rangle$. 
Let $\rho:I\to A_\times$ be the natural embedding, then for 
all $f\in A_\times^*$ and $a\in I$ holds
$$
\rho^*(\sigma(f))(a)
=\sigma(f)(\rho(a))
=\sigma(f)(a)
=\lim_{\mathfrak{U}}f(a e_\nu)
=\lim_{\nu}f(a e_\nu)
=f(\lim_{\nu}a e_\nu)
=f(a)
$$
i.e. $\sigma:I^*\to A_\times^*$ is a $\langle$~$1$-coretraction /
$c$-coretraction~$\rangle$. The right $A$-module $A_\times ^*$ is
$\langle$~metrically / $1$-topologically~$\rangle$ injective by
proposition~\ref{DualOfUnitalAlgIsMetTopInj}, hence its $\langle$~$1$-retract /
$c$-retract~$\rangle$ $I^*$ is $\langle$~metrically /
$c$-topologically~$\rangle$ injective. Now from 
proposition~\ref{MetCTopFlatCharac} we conclude that the $A$-module $I$ 
is $\langle$~metrically / $c$-topologically~$\rangle$ flat.
\end{proof}

Note that the same sufficient condition holds for relative flatness for ideals
of $A_\times$ [\cite{HelBanLocConvAlg}, proposition 7.1.45]. Now we are able to
give an example of a metrically flat module which is not even topologically
projective. Clearly $\ell_\infty(\mathbb{N})$-module $c_0(\mathbb{N})$ is not
unital as ideal but admits a contractive approximate identity. By
theorem~\ref{GoodCommIdealMetTopProjIsUnital} it is not topologically
projective, but it is metrically flat by
proposition~\ref{MetTopFlatIdealsInUnitalAlg}.

The ``metric'' part of the following proposition is a slight modification of
[\cite{WhiteInjmoduAlg}, proposition 4.11]. The case of topological flatness was
solved by Helemskii in [\cite{HelHomolBanTopAlg}, theorem VI.1.20].

\begin{proposition}\label{MetTopFlatCycModCharac} Let $I$ be a left proper ideal
of $A_\times $. Then the following are equivalent:

\begin{enumerate}[label = (\roman*)]
    \item $A_\times /I$ is $\langle$~metrically / $C$-topologically~$\rangle$
    flat $A$-module;

    \item $I$ has a right bounded approximate identity ${(e_\nu)}_{\nu\in N}$
    with $\sup_{\nu\in N}\Vert e_{A_\times }-e_\nu\Vert$ $\langle$~at most $1$ /
    at most $C$~$\rangle$
\end{enumerate}
\end{proposition}
\begin{proof} $(i)\implies (ii)$  Since $A_\times /I$ is $\langle$~metrically
/ $C$-topologically~$\rangle$ flat, then by proposition~\ref{MetCTopFlatCharac} 
the right $A$-module ${(A_\times /I)}^*$ is $\langle$~metrically /
$C$-topologically~$\rangle$ injective. Let $\pi:A_\times \to A_\times /I$ be the
natural quotient map, then $\pi^*:{(A_\times /I)}^*\to A_\times ^*$ is an
isometry. Since ${(A_\times /I)}^*$ is $\langle$~metrically /
$C$-topologically~$\rangle$ injective, then $\pi^*$ is a coretraction, i.e.\
there exists a $\langle$~strictly coisometric / topologically
surjective~$\rangle$ $A$-morphism $\tau:A_\times ^*\to {(A_\times /I)}^*$ of
norm $\langle$~at most $1$ / at most $C$~$\rangle$ such that
$\tau\pi^*=1_{{(A_\times /I)}^*}$. Consider $p\in A^{**}$ such that
$\iota_{A_\times }(e_{A_\times })-p
=\tau^*(\pi^{**}(\iota_{A_\times }(e_{A_\times })))$. 
Fix $f\in I^\perp$. Since $I^\perp=\pi^*({(A_\times /I)}^*)$, 
then there exists $g\in {(A_\times /I)}^*$ such that $f=\pi^*(g)$. Thus,
$$
(\iota_{A_\times }(e_{A_\times })-p)(f)
=\tau^*(\pi^{**}(\iota_{A_\times }(e_{A_\times })))(\pi^*(g))
=\pi^{**}(\iota_{A_\times }(e_{A_\times }))(\tau(\pi^*(g)))
$$
$$
=\pi^{**}(\iota_{A_\times }(e_{A_\times }))(g)
=\iota_{A_\times }(e_{A_\times })(\pi^*(g))
=\iota_{A_\times }(e_{A_\times })(f).
$$
Therefore, $p(f)=0$ for all $f\in I^\perp$, i.e. $p\in I^{\perp\perp}$. Recall
that $I^{\perp\perp}$ is the weak${}^*$ closure of $I$ in $A^{**}$, so we can
choose a net ${(e_\nu'')}_{\nu\in N''}\subset I$ such that
${(\iota_I(e_\nu''))}_{\nu\in N''}$ converges to $p$ in weak${}^*$ topology.
Clearly ${(\iota_{A_\times }(e_{A_\times }-e_\nu''))}_{\nu\in N''}$ converges to
$\iota_{A_\times }(e_{A_\times })-p$ in the same topology. By
[\cite{PosAndApproxIdinBanAlg}, lemma 1.1] there exists a net in the convex hull
$\operatorname{conv}{
    (\iota_{A_\times }(e_{A_\times }-e_\nu''))
}_{\nu\in N''}
=\iota_{A_\times }(e_{A_\times })
-\operatorname{conv}{(\iota_{A_\times}(e_\nu''))}_{\nu\in N''}$ 
that weak${}^*$ converges to $\iota_{A_\times }(e_{A_\times })-p$ with 
norm bound $\Vert \iota_{A_\times }(e_{A_\times})-p\Vert$. 
Denote this net as 
${(\iota_{A_\times }(e_{A_\times })-\iota_{A_\times }(e_\nu'))}_{\nu\in N'}$, 
then ${(\iota_{A_\times }(e_\nu'))}_{\nu\in N'}$ weak${}^*$ converges to $p$. 
For any $a\in I$ and $f\in I^*$ we have
$$
\lim_{\nu}f(ae_\nu')
=\lim_{\nu}\iota_{A_\times }(e_\nu')(f\cdot a)
=p(f\cdot a)
=\iota_{A_\times }(e_{A_\times })(f\cdot a)
-\tau^*(\pi^{**}(\iota_{A_\times }(e_{A_\times })))(f\cdot a)
$$
$$
=f(a)-\iota_{A_\times }(e_{A_\times })(\pi^*(\tau(f\cdot a)))
=f(a)-\pi^*(\tau(f)\cdot a)(e_{A_\times })
=f(a)-\tau(f)(\pi(a))
=f(a)
$$
hence ${(e_\nu')}_{\nu\in N'}$ is a weak right bounded approximate identity for
$I$. By [\cite{AppIdAndFactorInBanAlg}, proposition 33.2] there is a net
${(e_\nu)}_{\nu\in N}\subset\operatorname{conv}{(e_\nu')}_{\nu\in N'}$ which is
a right bounded approximate identity for $I$. For any $\nu\in N$ we have
$e_{A_\times }-e_\nu\in\operatorname{conv}{(e_{A_\times }-e_\nu')}_{\nu\in N'}$,
so taking into account the norm bound 
on ${(\iota_{A_\times }(e_{A_\times}-e_\nu'))}_{\nu\in N'}$ we get 
$$
\sup_{\nu\in N}\Vert e_{A_\times }-e_\nu\Vert
\leq\Vert \iota_{A_\times }(e_{A_\times })-p\Vert
\leq\Vert\tau^*(\pi^{**}(\iota_{A_\times }(e_{A_\times })))\Vert
\leq\Vert\tau^*\Vert\Vert\pi^{**}\Vert\Vert\iota_{A_\times }(e_{A_\times })\Vert
=\Vert\tau\Vert
$$
Since $\tau$ has norm $\langle$~at most $1$ / at most $C$~$\rangle$ we get the
desired bound. By construction, ${(e_\nu)}_{\nu\in N}$ is a right bounded
approximate identity for $I$.

$(ii)\implies (i)$ Denote $D=\sup_{\nu\in N}\Vert e_{A_\times }-e_\nu\Vert$.
Let $\mathfrak{F}$ be the section filter on $N$ and let $\mathfrak{U}$ be an
ultrafilter dominating $\mathfrak{F}$. For a fixed $f\in A_\times ^*$ 
and $a\in A_\times $ we have 
$|f(a-a e_\nu)|=|f(a(e_{A_\times }-e_\nu))|
\leq\Vert f\Vert\Vert a\Vert\Vert e_{A_\times }-e_\nu\Vert
\leq D\Vert f\Vert\Vert a\Vert$
i.e. ${(f(a-ae_\nu))}_{\nu\in N}$ is a bounded net of complex numbers. Therefore,
we have a well-defined limit $\lim_{\mathfrak{U}}f(a-ae_\nu)$ along ultrafilter
$\mathfrak{U}$. Since ${(e_\nu)}_{\nu\in N}$ is a right approximate identity for
$I$ and $\mathfrak{U}$ contains section filter then for all $a\in I$ we have
$\lim_{\mathfrak{U}}f(a-ae_\nu)=\lim_{\nu}f(a-ae_\nu)=0$. Therefore, for each
$f\in A_\times ^*$ we have well a defined map $\tau(f):A_\times /I\to
\mathbb{C}:a+I\mapsto \lim_{\mathfrak{U}} f(a-ae_\nu)$. Clearly, this is a
linear functional and from inequalities above we see its norm bounded 
by $D\Vert f\Vert$. Now it is routine to check 
that $\tau:A_\times ^*\to {(A_\times /I)}^*:f\mapsto \tau(f)$ is 
an $A$-morphism with norm $\langle$~at most $1$ / at most $C$~$\rangle$. 
For all $g\in{(A_\times /I)}^*$ and $a+I\in A_\times /I$ holds
$$
\tau(\pi^*(g))(a+I)
=\lim_{\mathfrak{U}}\pi^*(g)(a-ae_\nu)
=\lim_{\mathfrak{U}} g(\pi(a-ae_\nu))
=\lim_{\mathfrak{U}} g(a+I)
=g(a+I)
$$
i.e. $\tau:A_\times ^*\to {(A_\times /I)}^*$ is a retraction. The right
$A$-module $A_\times ^*$ is $\langle$~metrically / $1$-topologically~$\rangle$
injective by proposition~\ref{DualOfUnitalAlgIsMetTopInj}, hence its
$\langle$~$1$-retract / $C$-retract~$\rangle$ ${(A_\times /I)}^*$ is
$\langle$~metrically / $C$-topologically~$\rangle$ injective. 
Proposition~\ref{MetCTopFlatCharac} gives that the
module $A_\times /I$ is $\langle$~metrically / $C$-topologically~$\rangle$ flat.
\end{proof}

It is worth to mention that every operator algebra $A$ (not necessary self
adjoint) with contractive approximate identity has a contractive approximate
identity ${(e_\nu)}_{\nu\in N}$ such 
that $\sup_{\nu\in N}\Vert e_{A_\#}-e_\nu\Vert\leq 1$ and 
even $\sup_{\nu\in N}\Vert e_{A_\#}-2e_\nu\Vert\leq 1$. Here $A_\#$ is 
a unitization of $A$ as operator algebra. For details
see~\cite{PosAndApproxIdinBanAlg},~\cite{BleContrAppIdInOpAlg}.

Again we shall compare our result on metric and topological flatness of cyclic
modules with their relative counterpart. Helemeskii and Sheinberg showed
[\cite{HelHomolBanTopAlg}, theorem VII.1.20] that a cyclic module is relatively
flat if $I$ admits a right bounded approximate identity. In case when $I^\perp$
is complemented in $A_\times^*$ the converse is also true. In topological theory
we don't need this assumption, so we have a criterion. Metric flatness of cyclic
modules is a much stronger property due to specific restriction on the norm of
approximate identity. As we shall see in the next section, it is so restrictive
that it doesn't allow to construct any non-zero annihilator metrically
projective, injective or flat module over a non-zero Banach algebra.

%-------------------------------------------------------------------------------
%    The impact of Banach geometry
%-------------------------------------------------------------------------------

\section{
    The impact of Banach geometry}\label{SectionTheImpactOfBanachGeometry}


%-------------------------------------------------------------------------------
%    Homologically trivial annihilator modules
%-------------------------------------------------------------------------------

\subsection{
    Homologically trivial annihilator modules}\label{SubSectionHomoligicallyTrivialAnnihilatorModules}

In this section we concentrate on the study of metrically and topologically
projective, injective and flat annihilator modules. Unless otherwise stated, all
Banach spaces in this section are regarded as annihilator modules. Note the
obvious fact that we shall often use in this section: any bounded linear
operator between annihilator $A$-modules is an $A$-morphism.

\begin{proposition}\label{AnnihCModIsRetAnnihMod} Let $X$ be a non-zero
annihilator $A$-module. Then $\mathbb{C}$ is a $1$-retract of $X$ in
$A-\mathbf{mod}_1$.
\end{proposition}
\begin{proof} Take any $x_0\in X$ with $\Vert x_0\Vert=1$ and using Hahn-Banach
theorem choose $f_0\in X^*$ such that $\Vert f_0\Vert=f_0(x_0)=1$. Consider
contractive linear operators $\pi:X\to \mathbb{C}:x\mapsto f_0(x)$,
$\sigma:\mathbb{C}\to X:z\mapsto zx_0$. It is easy to check that $\pi$ and
$\sigma$ are contractive $A$-morphisms and what is more
$\pi\sigma=1_\mathbb{C}$. In other words $\mathbb{C}$ is a $1$-retract of $X$ in
$A-\mathbf{mod}_1$.
\end{proof}

Now it is time to recall that any Banach algebra $A$ can always be regarded as
proper maximal ideal of $A_+$, and what is more
$\mathbb{C}\isom{A-\mathbf{mod}_1} A_+/A$. If we regard $\mathbb{C}$ as a right
annihilator $A$-module we also have
$\mathbb{C}\isom{\mathbf{mod}_1-A}{(A_+/A)}^*$. 

\begin{proposition}\label{MetTopProjModCCharac} An annihilator $A$-module
$\mathbb{C}$ is $\langle$~metrically / $C$-topologically~$\rangle$ projective
iff $\langle$~$A= \{0 \}$ / $A$ has a right identity of norm at most
$C-1$~$\rangle$.
\end{proposition}
\begin{proof} 
It is enough to study $\langle$~metric / $C$-topological~$\rangle$ projectivity
of $A_+/A$. Since the natural quotient map $\pi:A_+\to A_+/A$ is a strict
coisometry, then by proposition~\ref{MetTopProjCycModCharac} $\langle$~metric /
$C$-topological~$\rangle$ projectivity of $A_+/A$ is equivalent to existence of
idempotent $p\in A$ such that $A=A_+p$ and $e_{A_+}-p$ has norm $\langle$~at
most $1$ / at most $C$~$\rangle$. It remains to note that this norm bound holds
iff $\langle$~$p=0$ and therefore $A=A_+p= \{0 \}$ / $p$ has norm at most
$C-1$~$\rangle$.
\end{proof}

\begin{proposition}\label{MetTopProjOfAnnihModCharac} Let $P$ be a non-zero
annihilator $A$-module. Then the following are equivalent:

\begin{enumerate}[label = (\roman*)]
    \item $P$ is $\langle$~metrically / $C$-topologically~$\rangle$ projective
    $A$-module;

    \item $\langle$~$A= \{0 \}$ / $A$ has a right identity of norm at most
    $C-1$~$\rangle$ and $P$ is a $\langle$~metrically /
    $C$-topologically~$\rangle$ projective Banach space. As the consequence
    $\langle$~$P\isom{\mathbf{Ban}_1}\ell_1(\Lambda)$ /
    $P\isom{\mathbf{Ban}}\ell_1(\Lambda)$~$\rangle$ for some set $\Lambda$.
\end{enumerate}
\end{proposition}
\begin{proof} $(i)\implies (ii)$ By 
propositions~\ref{RetrMetCTopProjIsMetCTopProj} 
and~\ref{AnnihCModIsRetAnnihMod} the
$A$-module $\mathbb{C}$ is $\langle$~metrically / $C$-topologically~$\rangle$
projective as $1$-retract of $\langle$~metrically / $C$-topologically~$\rangle$
projective module $P$. Proposition~\ref{MetTopProjModCCharac} gives that
$\langle$~$A= \{0 \}$ / $A$ has right identity of norm at most $C-1$~$\rangle$.
By corollary~\ref{MetTopProjTensProdWithl1} the annihilator $A$-module
$\mathbb{C}\projtens\ell_1(B_P)\isom{A-\mathbf{mod}_1}\ell_1(B_P)$ is
$\langle$~metrically / $C$-topologically~$\rangle$ projective. Consider strict
coisometry $\pi:\ell_1(B_P)\to P$ well-defined by equality $\pi(\delta_x)=x$.
Since $P$ and $\ell_1(B_P)$ are annihilator modules, then $\pi$ is also an
$A$-module map. Since $P$ is $\langle$~metrically / $C$-topologically~$\rangle$
projective, then the $A$-morphism $\pi$ has a right inverse morphism $\sigma$ of
norm  $\langle$~at most $1$ / at most $C-1$~$\rangle$. Therefore, $P$ is a
$\langle$~$1$-retract / $C$-retract~$\rangle$ of $\langle$~metrically /
$1$-topologically~$\rangle$ projective Banach space $\ell_1(B_P)$. By
proposition~\ref{RetrMetCTopProjIsMetCTopProj} the Banach space $P$ is
$\langle$~metrically / $C$-topologically~$\rangle$ projective. Now from
$\langle$~[\cite{HelMetrFrQMod}, proposition 3.2] / results
of~\cite{KotheTopProjBanSp}~$\rangle$ we get that $P$ is isomorphic to
$\ell_1(\Lambda)$ in $\langle$~$\mathbf{Ban}_1$ / $\mathbf{Ban}$~$\rangle$ for
some set $\Lambda$. 

$(ii)\implies (i)$ By proposition~\ref{MetTopProjModCCharac} the annihilator
$A$-module $\mathbb{C}$ is $\langle$~metrically / $C$-topologically~$\rangle$
projective. Therefore, by corollary~\ref{MetTopProjTensProdWithl1} the
annihilator $A$-module
$\mathbb{C}\projtens\ell_1(\Lambda)\isom{A-\mathbf{mod}_1}\ell_1(\Lambda)$ is
$\langle$~metrically / $C$-topologically~$\rangle$ projective too.
\end{proof}

\begin{proposition}\label{MetTopInjModCCharac} A right annihilator $A$-module
$\mathbb{C}$ is $\langle$~metrically / $C$-topologically~$\rangle$ injective iff
$\langle$~$A= \{0 \}$ / $A$ has right $(C-1)$-bounded approximate
identity~$\rangle$.
\end{proposition}
\begin{proof} Because of proposition~\ref{MetCTopFlatCharac} it is enough 
to study $\langle$~metric / $C$-topological~$\rangle$ flatness of $A_+/A$. By
proposition~\ref{MetTopFlatCycModCharac} this is equivalent to existence of
right bounded approximate identity ${(e_\nu)}_{\nu\in N}$ in $A$ with
$\langle$~$\sup_{\nu\in N}\Vert e_{A_+}-e_\nu\Vert\leq 1$ / 
$\sup_{\nu\in N}\Vert e_{A_+}-e_\nu\Vert\leq C$~$\rangle$. 
It remains to note that the latter inequality holds 
iff $\langle$~iff $e_\nu=0$ and therefore $A= \{0 \}$. /
${(e_\nu)}_{n\in N}$ is a right $C-1$-bounded approximate identity~$\rangle$.
\end{proof}

\begin{proposition}\label{MetTopInjOfAnnihModCharac} Let $J$ be a non-zero right
annihilator $A$-module. Then the following are equivalent:

\begin{enumerate}[label = (\roman*)]
    \item $J$ is $\langle$~metrically / $C$-topologically~$\rangle$ injective
    $A$-module;

    \item $\langle$~$A= \{0 \}$ / $A$ has a right $(C-1)$-bounded approximate
    identity~$\rangle$ and $J$ is a $\langle$~metrically /
    $C$-topologically~$\rangle$ injective Banach space. $\langle$~As the
    consequence $J\isom{\mathbf{Ban}_1}C(K)$ for some Stonean space $K$
    /~$\rangle$.
\end{enumerate}
\end{proposition}
\begin{proof} $(i)\implies (ii)$ By 
propositions~\ref{RetrMetCTopInjIsMetCTopInj} and~\ref{AnnihCModIsRetAnnihMod} 
the $A$-module $\mathbb{C}$ is
$\langle$~metrically / $C$-topologically~$\rangle$ injective as $1$-retract of
$\langle$~metrically / $C$-topologically~$\rangle$ injective module $J$.
Proposition~\ref{MetTopInjModCCharac} gives that $\langle$~$A= \{0 \}$ / $A$ has
a right $(C-1)$-bounded approximate identity~$\rangle$. By
proposition~\ref{MapsFroml1toMetTopInj} the annihilator $A$-module
$\mathcal{B}(\ell_1(B_{J^*}),\mathbb{C})
\isom{\mathbf{mod}_1-A}
\ell_\infty(B_{J^*})$ is $\langle$~metrically / $C$-topologically~$\rangle$ 
injective. Consider isometry $\rho:J\to\ell_\infty(B_{J^*})$ 
well-defined by $\rho(x)(f)=f(x)$. Since $J$ and $\ell_\infty(B_{J^*})$ are 
annihilator modules, then $\rho$ is also an $A$-module map. 
Since $J$ is $\langle$~metrically / $C$-topologically~$\rangle$ injective, 
then the $A$-morphism $\rho$ has a left inverse morphism $\tau$ with 
norm $\langle$~at most $1$ / at most $C$~$\rangle$.
Therefore, $J$ is $\langle$~$1$-retract / $C$-retract~$\rangle$ of
$\langle$~metrically / $1$-topologically~$\rangle$ injective Banach space
$\ell_\infty(B_{J^*})$. By 
proposition $\langle$~\ref{RetrMetCTopInjIsMetCTopInj} the Banach space $J$ is
$\langle$~metrically / $C$-topologically~$\rangle$ injective. $\langle$~From
[\cite{LaceyIsomThOfClassicBanSp}, theorem 3.11.6] the Banach space $J$ is
isometrically isomorphic to $C(K)$ for some Stonean space $K$. /~$\rangle$ 

$(ii)\implies (i)$ By proposition~\ref{MetTopInjModCCharac} the annihilator
$A$-module $\mathbb{C}$ is $\langle$~metrically / $C$-topologically~$\rangle$
injective. Even more, by proposition~\ref{MapsFroml1toMetTopInj} 
the annihilator $A$-module
$\mathcal{B}(\ell_1(B_{J^*}),\mathbb{C})
\isom{\mathbf{mod}_1-A}
\ell_\infty(B_{J^*})$ is $\langle$~metrically / $C$-topologically~$\rangle$ 
injective too. Since $J$ is a $\langle$~metrically / $C$-topologically~$\rangle$
injective Banach space and there an isometric 
embedding $\rho:J\to \ell_\infty(B_{J^*})$, then as
Banach space $J$ is a $\langle$~$1$-retract / $C$-retract~$\rangle$ of
$\ell_\infty(B_{J^*})$. Recall, that $J$ and $\ell_\infty(B_{J^*})$ are
annihilator modules, so in fact we have a retraction in
$\langle$~$\mathbf{mod}_1-A$ / $\mathbf{mod}-A$~$\rangle$. By 
proposition~\ref{RetrMetCTopInjIsMetCTopInj} the $A$-module $J$ 
is $\langle$~metrically / $C$-topologically~$\rangle$ injective.
\end{proof}

\begin{proposition}\label{MetTopFlatAnnihModCharac} Let $F$ be a non-zero
annihilator $A$-module. Then the following are equivalent:

\begin{enumerate}[label = (\roman*)]
    \item $F$ is $\langle$~metrically / $C$-topologically~$\rangle$ flat
    $A$-module;

    \item $\langle$~$A= \{0 \}$ / $A$ has a right $(C-1)$-bounded approximate
    identity~$\rangle$ and $F$ is a $\langle$~metrically /
    $C$-topologically~$\rangle$ flat Banach space, that is
    $\langle$~$F\isom{\mathbf{Ban}_1}L_1(\Omega,\mu)$ for some measure space
    $(\Omega, \Sigma, \mu)$ / $F$ is an $\mathscr{L}_{1,C}^g$-space~$\rangle$.
\end{enumerate}
\end{proposition}
\begin{proof} By $\langle$~[\cite{GrothMetrProjFlatBanSp}, theorem 1] /
[\cite{DefFloTensNorOpId}, corollary 23.5(1)]~$\rangle$ the Banach space $F$ is
$\langle$~metrically / $C$-topologically~$\rangle$ flat iff
$\langle$~$F\isom{\mathbf{Ban}_1}L_1(\Omega,\mu)$ for some measure space
$(\Omega, \Sigma, \mu)$ / $F$ is an $\mathscr{L}_{1,C}^g$-space~$\rangle$. Now
the equivalence follows from propositions~\ref{MetTopInjOfAnnihModCharac}
and~\ref{MetCTopFlatCharac}.
\end{proof}

We obliged to compare these results with similar ones in relative theory. From
$\langle$~[\cite{RamsHomPropSemgroupAlg}, proposition 2.1.7] /
[\cite{RamsHomPropSemgroupAlg}, proposition 2.1.10]~$\rangle$ we know that an
annihilator $A$-module over Banach algebra $A$ is relatively
$\langle$~projective / flat~$\rangle$ iff $A$ has $\langle$~a right identity / a
right bounded approximate identity~$\rangle$.   
In metric and topological theory, in comparison with relative one, homological
triviality of annihilator modules puts restrictions not only on the algebra
itself but on the geometry of the module too. These geometric restrictions
forbid existence of certain homologically excellent algebras. One of the most
important properties of relatively $\langle$~contractible / amenable~$\rangle$
Banach algebra is $\langle$~projectivity / flatness~$\rangle$ of all (and in
particular of all annihilator) left Banach modules over it. In a sharp contrast
in metric and topological theories such algebras can't exist.

\begin{proposition} There is no Banach algebra $A$ such that all $A$-modules are
$\langle$~metrically / topologically~$\rangle$ flat. A fortiori, there is no
such Banach algebras that all $A$-modules are $\langle$~metrically /
topologically~$\rangle$ projective.
\end{proposition}
\begin{proof} Consider any infinite dimensional $\mathscr{L}_\infty^g$-space $X$
(say $\ell_\infty(\mathbb{N})$) as an annihilator $A$-module. From remark right
after [\cite{DefFloTensNorOpId}, corollary 23.3] we know that $X$ is not an
$\mathscr{L}_1^g$-space. Therefore, by proposition~\ref{MetTopFlatAnnihModCharac}
the $A$-module $X$ is not topologically flat. By
proposition~\ref{MetFlatIsTopFlatAndTopFlatIsRelFlat} it is not metrically flat.
Now from proposition~\ref{MetTopProjIsMetTopFlat} we see that $X$ is neither
metrically nor topologically projective.
\end{proof}

%-------------------------------------------------------------------------------
%    Homologically trivial modules over Banach algebras with specific geometry
%-------------------------------------------------------------------------------

\subsection{
    Homologically trivial modules over Banach algebras with specific geometry}\label{
    SubSectionHomologicallyTrivialModulesOverBanachAlgebrasWithSpecificGeometry
}

The purpose of this section is to convince our reader that homologically trivial
modules over certain Banach algebras have similar geometric structure of those
algebras. For the case of metric theory the following proposition was proved by
Graven in~\cite{GravInjProjBanMod}.

\begin{proposition}\label{TopProjInjFlatModOverL1Charac} Let $A$ be a Banach
algebra which is as Banach space isometrically isomorphic to $L_1(\Theta,\nu)$
for some measure space $(\Theta,\Sigma,\nu)$. Then

\begin{enumerate}[label = (\roman*)]
    \item if $P$ is a $\langle$~metrically / topologically~$\rangle$ projective
    $A$-module, then $P$ is $\langle$~an $L_1$-space / complemented in some
    $L_1$-space~$\rangle$.

    \item if $J$ is a $\langle$~metrically / topologically~$\rangle$ injective
    $A$-module, then  $J$ is a $\langle$~$C(K)$-space for some Stonean space $K$
    / topologically injective Banach space~$\rangle$.

    \item if $F$ is a $\langle$~metrically / topologically~$\rangle$ flat
    $A$-module, then $F$ is an $\langle$~$L_1$-space /
    $\mathscr{L}_1^g$-space~$\rangle$.
\end{enumerate}
\end{proposition}
\begin{proof} 

Denote by $(\Theta',\Sigma',\nu')$ the measure space $(\Theta,\Sigma,\nu)$ with
singleton atom adjoined, then $A_+\isom{\mathbf{Ban}_1} L_1(\Theta',\nu')$.

$(i)$ Since $P$ is a $\langle$~metrically / topologically~$\rangle$ projective
$A$-module, then by proposition~\ref{MetCTopProjModViaCanonicMorph} it is a
retract of $A_+\projtens \ell_1(B_P)$ in $\langle$~$A-\mathbf{mod}_1$ /
$A-\mathbf{mod}$~$\rangle$. Let $\mu_c$ be the counting measure on $B_P$, then
by Grothendieck's theorem [\cite{HelLectAndExOnFuncAn}, theorem 2.7.5]
$$
A_+\projtens\ell_1(B_P)
\isom{\mathbf{Ban}_1}L_1(\Theta',\nu')\projtens L_1(B_P,\mu_c)
\isom{\mathbf{Ban}_1}L_1(\Theta'\times B_P,\nu'\times \mu_c)
$$
Therefore $P$ is $\langle$~$1$-complemented / complemented~$\rangle$ in some
$L_1$-space. It remains to recall that any $1$-complemented subspace of
$L_1$-space is again an $L_1$-space [\cite{LaceyIsomThOfClassicBanSp}, theorem
6.17.3].

$(ii)$ Since $J$ is $\langle$~metrically / topologically~$\rangle$ injective
$A$-module, then by proposition~\ref{MetCTopInjModViaCanonicMorph} it is a
retract of $\mathcal{B}(A_+,\ell_\infty(B_{J^*}))$ in
$\langle$~$\mathbf{mod}_1-A$ / $\mathbf{mod}-A$~$\rangle$. Let $\mu_c$ be the
counting measure on $B_{J^*}$, then by Grothendieck's theorem
[\cite{HelLectAndExOnFuncAn}, theorem 2.7.5]
$$
\mathcal{B}(A_+,\ell_\infty(B_{J^*}))
\isom{\mathbf{Ban}_1}
{(A_+\projtens \ell_1(B_{J^*}))}^*
\isom{\mathbf{Ban}_1}
{(L_1(\Theta',\nu')\projtens L_1(B_P,\mu_c))}^*
$$
$$
\isom{\mathbf{Ban}_1}{L_1(\Theta'\times B_P,\nu'\times \mu_c)}^*
\isom{\mathbf{Ban}_1}L_\infty(\Theta'\times B_P,\nu'\times \mu_c)
$$
Therefore $J$ is $\langle$~$1$-complemented / complemented~$\rangle$ in some
$L_\infty$-space. Since $L_\infty$-space is $\langle$~metrically /
topologically~$\rangle$ injective Banach space, then so does $J$. It remains to
recall that every metrically injective Banach space is a $C(K)$-space for some
Stonean space $K$ [\cite{LaceyIsomThOfClassicBanSp}, theorem 3.11.6].

$(iii)$  By $\langle$~[\cite{GrothMetrProjFlatBanSp}, theorem 1] / remark after
[\cite{DefFloTensNorOpId}, corollary 23.5(1)]~$\rangle$ the Banach space $F^*$
is $\langle$~metrically / topologically~$\rangle$ injective iff $F$ is an
$\langle$~$L_1$-space / $\mathscr{L}_1^g$-space~$\rangle$. Now the implication
follows from paragraph $(ii)$ and proposition~\ref{MetCTopFlatCharac}.
\end{proof}

\begin{proposition}\label{TopProjInjFlatModOverMthscrL1SpCharac} Let $A$ be a
Banach algebra which is topologically isomorphic as Banach space to some
$\mathscr{L}_1^g$-space. Then any topologically $\langle$~projective / injective
/ flat~$\rangle$ $A$-module is an $\langle$~$\mathscr{L}_1^g$-space /
$\mathscr{L}_\infty^g$-space / $\mathscr{L}_1^g$-space~$\rangle$.
\end{proposition}
\begin{proof} If $A$ is an $\mathscr{L}_1^g$-space, then so does $A_+$. 

Let $P$ be a topologically projective $A$-module. Then by
proposition~\ref{MetCTopProjModViaCanonicMorph} it is a retract 
of $A_+\projtens \ell_1(B_P)$ in $A-\mathbf{mod}$ and a fortiori 
in $\mathbf{Ban}$. Since
$\ell_1(B_P)$ is an $\mathscr{L}_1^g$-space, then so does
$A_+\projtens\ell_1(B_P)$ as projective tensor product of
$\mathscr{L}_1^g$-spaces [\cite{DefFloTensNorOpId}, exercise 23.17(c)].
Therefore, $P$ is an $\mathscr{L}_1^g$-space as complemented subspace of
$\mathscr{L}_1^g$-space [\cite{DefFloTensNorOpId}, corollary 23.2.1(2)].

Let $J$ be a topologically injective $A$-module, then by
proposition~\ref{MetCTopInjModViaCanonicMorph} it is a retract of
$\mathcal{B}(A_+,\ell_\infty(B_{J^*}))
\isom{\mathbf{mod}_1-A}
{(A_+\projtens\ell_1(B_{J^*}))}^*$
in $\mathbf{mod}-A$ and a fortiori in $\mathbf{Ban}$. As we showed in the
previous paragraph $A_+\projtens\ell_1(B_{J^*})$ is an $\mathscr{L}_1^g$-space,
therefore its dual $\mathcal{B}(A_+,\ell_\infty(B_{J^*}))$ is an
$\mathscr{L}_\infty^g$-space [\cite{DefFloTensNorOpId}, corollary 23.2.1(1)]. It
remains to recall that any complemented subspace of $\mathscr{L}_\infty^g$-space
is again an $\mathscr{L}_\infty^g$-space [\cite{DefFloTensNorOpId}, corollary
23.2.1(2)].

Finally, let $F$ be a topologically flat $A$-module, then $F^*$ is topologically
injective $A$-module by proposition~\ref{MetCTopFlatCharac}. From previous
paragraph it follows that $F^*$ is an $\mathscr{L}_\infty^g$-space. By
[\cite{DefFloTensNorOpId}, corollary 23.5(1)] we get that $F$ is an
$\mathscr{L}_1^g$-space.
\end{proof}

We proceed to the discussion of the Dunford-Pettis property for homologically
trivial modules.   

\begin{proposition}\label{C0SumOfL1SpHaveDPP} Let $(\Omega, \Sigma, \mu)$ be a
measure spaces and $\Lambda$ be an arbitrary set. Then the Banach space
$\bigoplus_0 \{L_1(\Omega,\mu):\lambda\in\Lambda \}$ and all its duals has the
Dunford-Pettis property. In particular, $\bigoplus_1
\{L_\infty(\Omega,\mu):\lambda\in\Lambda \}$ and $\bigoplus_\infty
\{L_1(\Omega,\mu):\lambda\in\Lambda \}$ have this property.
\end{proposition}
\begin{proof} Consider one point compactification $\alpha\Lambda:=\Lambda\cup
\{\Lambda \}$ of the set $\Lambda$ with discrete topology. From
[\cite{BourgOnTheDPP}, corollary 7] we know 
that $C(\alpha\Lambda, L_1(\Omega, \mu))$ and all its duals have 
the Dunford-Pettis property. Since $c_0(\Lambda)$ is complemented 
in $C(\alpha\Lambda)$ via projection 
$P:C(\alpha\Lambda)\to C(\alpha\Lambda):x\mapsto x(\lambda)-x( \{\Lambda \})$, 
then $E:=c_0(\Lambda, L_1(\Omega,\mu))$ is complemented 
in $C(\alpha\Lambda, L_1(\Omega, \mu))$. The same statement holds for 
any $n$-th dual of $E$, because we can take $n$-th
adjoint of $P$ in the role of projection. Now it remains to note that
$E=\bigoplus_0 \{L_1(\Omega,\mu):\lambda\in\Lambda \}$ and that the
Dunford-Pettis property is inherited by complemented subspaces
[\cite{FabHabBanSpTh}, proposition 13.44]. 

As the consequence of previous paragraph the Banach spaces
$E^*
\isom{\mathbf{Ban}_1}
\bigoplus_1 \{
    L_\infty(\Omega,\mu):\lambda\in\Lambda
 \}$ 
and $E^{**}
\isom{\mathbf{Ban}_1}
    \bigoplus_\infty \{{L_1(\Omega,\mu)}^{**}:\lambda\in\Lambda
 \}$ 
have the Dunford-Pettis property.
From [\cite{DefFloTensNorOpId}, proposition B10] we know that any $L_1$-space is
contractively complemented in its second dual. By $Q$ we denote the respective
projection. Therefore, the Banach space $\bigoplus_\infty
\{L_1(\Omega,\mu):\lambda\in\Lambda \}$ is contractively complemented in
$E^{**}$ via projection $\bigoplus_\infty  \{Q:\lambda\in\Lambda \}$. Since
$E^{**}$ has the Dunford-Pettis property, then by [\cite{FabHabBanSpTh},
proposition 13.44] so does its complemented 
subspace $\bigoplus_\infty \{L_1(\Omega,\mu):\lambda\in\Lambda \}$.
\end{proof}

\begin{proposition}\label{ProdOfL1SpHaveDPP} Let $
\{(\Omega_\lambda,\Sigma_\lambda,\mu_\lambda):\lambda\in\Lambda \}$ be a family
of measure spaces. 
Then 
$\bigoplus_0 \{
    L_1(\Omega_\lambda, \mu_\lambda):\lambda\in\Lambda
 \}$, 
 $\bigoplus_1 \{
     L_\infty(\Omega_\lambda, \mu_\lambda):\lambda\in\Lambda
 \}$ 
and $\bigoplus_\infty \{
    L_1(\Omega_\lambda,\mu_\lambda):\lambda\in\Lambda
 \}$ have the Dunford-Pettis property.
\end{proposition}
\begin{proof} Let $(\Omega, \Sigma, \mu)$ be a direct sum of $
\{(\Omega_\lambda,\Sigma_\lambda,\mu_\lambda):\lambda\in\Lambda \}$. By
construction each Banach space $L_1(\Omega_\lambda,\mu_\lambda)$ for
$\lambda\in\Lambda$ is $1$-complemented in $L_1(\Omega, \mu)$. Therefore, their
$\bigoplus_0$-, $\bigoplus_1$- and $\bigoplus_\infty$-sums are contractively
complemented in $\bigoplus_0 \{L_1(\Omega, \mu):\lambda\in\Lambda \}$,
$\bigoplus_1 \{L_\infty(\Omega, \mu):\lambda\in\Lambda \}$ 
and $\bigoplus_\infty \{L_1(\Omega,\mu):\lambda\in\Lambda \}$ respectively. 
It remains to combine proposition~\ref{C0SumOfL1SpHaveDPP} and the fact 
that the Dunford-Pettis property is preserved by complemented 
subspaces [\cite{FabHabBanSpTh}, proposition 13.44].
\end{proof}

\begin{proposition}\label{ProdOfDualsOfMthscrLInftySpHaveDPP} Let $E$ be an
$\mathscr{L}_\infty^g$-space and $\Lambda$ be an arbitrary set. Then
$\bigoplus_\infty \{E^*:\lambda\in\Lambda \}$ has the Dunford-Pettis property.
\end{proposition}
\begin{proof} Since $E$ is an $\mathscr{L}_\infty^g$-space, then $E^*$ is a
$\mathscr{L}_{1}^g$-space [\cite{DefFloTensNorOpId}, corollary 23.2.1(1)]. Then
from [\cite{DefFloTensNorOpId}, corollary 23.2.1(3)] it follows that $E^{***}$
is a retract of $L_1$-space. Recall that $E^*$ is complemented in $E^{***}$ via
Dixmier projection, so $E^*$ is complemented in some $L_1$-space too. Thus, we
have a bounded linear projection $P:L_1(\Omega,\mu)\to L_1(\Omega,\mu)$ with
image topologically isomorphic to $E$. In this 
case $\bigoplus_\infty \{E^*:\lambda\in\Lambda \}$ is complemented 
in $\bigoplus_\infty \{ L_1(\Omega,\mu):\lambda\in\Lambda \}$ via 
projection $\bigoplus_\infty \{P:\lambda\in\Lambda \}$. The 
space $\bigoplus_\infty \{ L_1(\Omega,\mu):\lambda\in\Lambda \}$ has 
the Dunford-Pettis property by proposition~\ref{ProdOfL1SpHaveDPP}. By 
proposition 13.44 in~\cite{FabHabBanSpTh} so does 
$\bigoplus_\infty \{ E^*:\lambda\in\Lambda \}$ as complemented subspace 
of $\bigoplus_\infty \{ L_1(\Omega,\mu):\lambda\in\Lambda \}$.
\end{proof}

\begin{proposition}\label{MthscrL1LInftyHaveDPP} Any
$\langle$~$\mathscr{L}_1^g$-space / $\mathscr{L}_\infty^g$-space~$\rangle$
admits the Dunford-Pettis property.
\end{proposition}
\begin{proof} Assume $E$ is an $\langle$~$\mathscr{L}_1^g$-space /
$\mathscr{L}_\infty^g$-space~$\rangle$. Then $E^{**}$ is complemented in some
$\langle$~$L_1$-space / $L_\infty$-space~$\rangle$ [\cite{DefFloTensNorOpId},
corollary 23.2.1(3)]. Since any $\langle$~$L_1$-space /
$L_\infty$-space~$\rangle$ admits the Dunford-Pettis
property~\cite{GrothApllFaiblCompSpCK}, then so does $E^{**}$ as complemented
subspace [\cite{FabHabBanSpTh}, proposition 13.44]. It remains to recall that a
Banach space have the Dunford-Pettis property whenever so does its dual.
\end{proof}

\begin{theorem}\label{TopProjInjFlatModOverMthscrL1OrLInftySpHaveDPP} Let $A$ be
a Banach algebra which is an $\mathscr{L}_1^g$-space or
$\mathscr{L}_\infty^g$-space as Banach space. Then any topologically projective,
injective or flat $A$-module has the Dunford-Pettis property.
\end{theorem}
\begin{proof} If $A$ is an $\mathscr{L}_1^g$-space, we just need to combine
propositions~\ref{MthscrL1LInftyHaveDPP}
and~\ref{TopProjInjFlatModOverMthscrL1SpCharac}. 

Assume $A$ is an $\mathscr{L}_\infty^g$-space, then so does $A_+$. Let $J$ be a
topologically injective $A$-module, then by
proposition~\ref{MetCTopInjModViaCanonicMorph} it is a retract of 
$$
\mathcal{B}(A_+,\ell_\infty(B_{J^*}))
\isom{\mathbf{mod}_1-A}
{(A_+\projtens\ell_1(B_{J^*}))}^*
\isom{\mathbf{mod}_1-A}
{\left(\bigoplus\nolimits_1 \{ A_+:\lambda\in B_{J^*} \}\right)}^*
$$
$$
\isom{\mathbf{mod}_1-A}
\bigoplus\nolimits_\infty \{ A_+^*:\lambda\in B_{J^*} \}
$$ 
in $\mathbf{mod}-A$ and a fortiori in $\mathbf{Ban}$. By
proposition~\ref{ProdOfDualsOfMthscrLInftySpHaveDPP} this space has the
Dunford-Pettis property. As $J$ is its retract, then $J$ also has this property
[\cite{FabHabBanSpTh}, proposition 13.44]. 

If $F$ is a topologically flat $A$-module, then $F^*$ is a topologically
injective $A$-module by proposition~\ref{MetCTopFlatCharac}. By previous
paragraph $F^*$ has the Dunford-Pettis property and so does $F$.

If $P$ is a topologically projective $A$-module, it is also topologically flat
by proposition~\ref{MetTopProjIsMetTopFlat}. From previous paragraph it follows
that $P$ has the Dunford-Pettis property.
\end{proof}

\begin{corollary}\label{NoInfDimRefMetTopProjInjFlatModOverMthscrL1OrLInfty} Let
$A$ be a Banach algebra which $\mathscr{L}_1^g$-space or
$\mathscr{L}_\infty^g$-space as Banach space. Then there is no topologically
projective, injective or flat infinite dimensional reflexive $A$-modules. A
fortiori there is no metrically projective, injective or flat infinite
dimensional reflexive $A$-modules.
\end{corollary}
\begin{proof} From theorem~\ref{TopProjInjFlatModOverMthscrL1OrLInftySpHaveDPP}
we know that any topologically injective $A$-module has the Dunford-Pettis
property. On the other hand there is no infinite dimensional reflexive Banach
spaces with the Dunford-Pettis property. Thus, we get the desired result
regarding topological injectivity. Since dual of reflexive module is reflexive,
from proposition~\ref{MetCTopFlatCharac} we get the result for topological
flatness. It remains to recall that by proposition~\ref{MetTopProjIsMetTopFlat}
every topologically projective module is topologically flat. To prove the last
claim note that metric $\langle$~projectivity / injectivity / flatness~$\rangle$
implies topological $\langle$~projectivity / injectivity / flatness~$\rangle$ by
proposition $\langle$~\ref{MetProjIsTopProjAndTopProjIsRelProj}
/~\ref{MetInjIsTopInjAndTopInjIsRelInj}
/~\ref{MetFlatIsTopFlatAndTopFlatIsRelFlat}~$\rangle$.
\end{proof}

Note that in relative theory there are examples of infinite dimensional
relatively projective injective and flat reflexive modules over Banach algebras
that are $\mathscr{L}_1^g$- or $\mathscr{L}_\infty^g$-spaces. Here are two
examples. The first one is about convolution algebra $L_1(G)$ on a locally
compact group $G$ with Haar measure. It is an $\mathscr{L}_1^g$-space. In
[\cite{DalPolHomolPropGrAlg}, \S6] and~\cite{RachInjModAndAmenGr} it was proved
that for $1<p<+\infty$ the $L_1(G)$-module $L_p(G)$ is relatively
$\langle$~projective / injective / flat~$\rangle$ iff $G$ is $\langle$~compact /
amenable / amenable~$\rangle$. Note that any compact group is amenable
[\cite{PierAmenLCA}, proposition 3.12.1], so in case $G$ is compact $L_p(G)$ is
relatively projective injective and flat for all $1<p<+\infty$.  The second
example is about $\mathscr{L}_\infty^g$-spaces $c_0(\Lambda)$ and
$\ell_\infty(\Lambda)$ for an infinite set $\Lambda$. In
proposition~\ref{c0AndlInftyModsRelTh} we shall show that $\ell_p(\Lambda)$ for
$1<p<\infty$ is relatively projective injective and flat as $c_0(\Lambda)$- or
$\ell_\infty(\Lambda)$-module. 

We finalize this lengthy section by short remark on the l.u.st.\ property of
topologically projective, injective and flat modules. 

\begin{proposition} Let $A$ be a Banach algebra which as Banach space has the
l.u.st.\ property. Then any topologically projective, injective or flat
$A$-module has the l.u.st.\ property.
\end{proposition}
\begin{proof} 

If $J$ is topologically injective $A$-module, then by
proposition~\ref{MetCTopInjModViaCanonicMorph} it is a retract of
$\mathcal{B}(A_+,\ell_\infty(B_{J^*}))
\isom{\mathbf{mod}_1-A}
\bigoplus_\infty \{ A_+^*:\lambda\in B_{J^*} \}$ 
in $\mathbf{mod}-A$ and a fortiori in $\mathbf{Ban}$. If $A$ has 
the l.u.st.\ property, then $A^{**}$ is complemented
in some Banach lattice $E$ [\cite{DiestAbsSumOps}, theorem 17.5]. As the
consequence $A_+^{***}$ is complemented in the Banach lattice
$F:={\left(E\bigoplus_1\mathbb{C}\right)}^*$ via some bounded projection 
$P:F\to F$. Thus, $\bigoplus_\infty \{A_+^{***}:\lambda\in B_{J^*} \}$ is 
complemented in the Banach lattice $\bigoplus_\infty \{F:\lambda\in B_{J^*} \}$ 
via projection $\bigoplus_\infty \{ P:\lambda\in B_{J^*} \}$. Any Banach 
lattice has the l.u.st.\ property [\cite{DiestAbsSumOps}, theorem 17.1]. 
This property is inherited by complemented subspaces, 
so $\bigoplus_\infty \{A_+^{***}:\lambda\in B_{J^*} \}$ has 
the l.u.st.\ property too. Note that $A_+^*$ is contractively
complemented in $A_+^{***}$ by Dixmier projection $Q$, therefore
$\bigoplus_\infty \{A_+^*:\lambda\in B_{J^*} \}$ is complemented in
$\bigoplus_\infty \{A_+^{***}:\lambda\in B_{J^*} \}$ via contractive projection
$\bigoplus_\infty \{Q:\lambda\in B_{J^*} \}$. Since the latter space has the
l.u.st.\ property, then so does its 
retract $\bigoplus_\infty \{A_+^*:\lambda\in B_{J^*} \}$. Finally, $J$ is a 
retract of the $\bigoplus_\infty \{A_+^*:\lambda\in B_{J^*} \}$, therefore 
also has this property.

If $F$ is topologically flat $A$-module, then $F^*$ is topologically injective
by proposition~\ref{MetCTopFlatCharac}. By previous paragraph $F^*$ has the
l.u.st.\ property. Corollary 17.6 from~\cite{DiestAbsSumOps} gives that $F$ has
this property too.

If $P$ is topologically projective $A$-module, it is topologically flat by
proposition~\ref{MetTopProjIsMetTopFlat}. So $P$ has the l.u.st.\ property by
previous paragraph.
\end{proof}

%-------------------------------------------------------------------------------
%    Further properties of projective injective and flat modules
%-------------------------------------------------------------------------------

\section{
    Further properties of projective injective and flat modules}\label{
SectionFurtherPropertiesOfProjectiveInjectiveAndFlatModules}

%----------------------------------------------------------------------------------------
%    Homological triviality of modules under change of algebra
%----------------------------------------------------------------------------------------

\subsection{
    Homological triviality of modules under change of algebra}\label{
SubSectionHomologicalTrivialityOfModulesUnderChangeOfAlgebra}

The following three propositions are metric and topological versions of
propositions 2.3.2, 2.3.3 and 2.3.4 in~\cite{RamsHomPropSemgroupAlg}.

\begin{proposition}\label{MorphCoincide} Let $X$ and $Y$ be $\langle$~left /
right~$\rangle$ $A$-modules. Assume one of the following holds

\begin{enumerate}[label = (\roman*)]
    \item $I$ is a $\langle$~left / right~$\rangle$ ideal of $A$ and $X$ is an
    essential $I$-module;

    \item $I$ is a $\langle$~right / left~$\rangle$ ideal of $A$ and $Y$ is a
    faithful $I$-module. 
\end{enumerate}

Then $\langle$~${}_A\mathcal{B}(X,Y)={}_I\mathcal{B}(X,Y)$ /
$\mathcal{B}_A(X,Y)=\mathcal{B}_I(X,Y)$~$\rangle$.
\end{proposition}
\begin{proof} We shall prove both statements only for left modules, since their
right counterparts are proved with minimal modifications. 
Fix $\phi\in{}_I\mathcal{B}(X,Y)$.

$(i)$ Take $x\in I\cdot X$, then $x=a'\cdot x'$ for some $a'\in I$, $x'\in X$.
For any $a\in A$ we have $\phi(a\cdot x)=\phi(aa'\cdot
x')=aa'\cdot\phi(x')=a\cdot\phi(a'\cdot x')=a\cdot\phi(x)$. Therefore,
$\phi(a\cdot x)=a\cdot\phi(x)$ for all $a\in A$ 
and $x\in \operatorname{cl}_X(IX)=X$. Hence, $\phi\in {}_A\mathcal{B}(X,Y)$.

$(ii)$ For any $a\in I$ and $a'\in A$, $x\in X$ we have $a\cdot(\phi(a'\cdot
x)-a'\cdot\phi(x))=\phi(aa'\cdot x)-aa'\cdot\phi(x)=0$. Since $Y$ is faithful
$I$-module we have $\phi(a'\cdot x)=a'\cdot \phi(x)$ for all $x\in X$, 
$a'\in A$. Hence, $\phi\in{}_A\mathcal{B}(X,Y)$.

In both cases we proved that $\phi\in{}_A\mathcal{B}(X,Y)$ for any
$\phi\in{}_I\mathcal{B}(X,Y)$, 
therefore ${}_I\mathcal{B}(X,Y)\subset {}_A\mathcal{B}(X,Y)$. 
The reverse inclusion is obvious.
\end{proof}

\begin{proposition}\label{MetTopProjUnderChangeOfAlg} Let $I$ be a closed
subalgebra of $A$ and $P$ be an $A$-module which is essential as $I$-module.
Then

\begin{enumerate}[label = (\roman*)]
    \item if $I$ is a left ideal of $A$ and $P$ is $\langle$~metrically /
    $C$-topologically~$\rangle$  projective $I$-module, then $P$ is
    $\langle$~metrically / $C$-topologically~$\rangle$ projective $A$-module;

    \item if $I$ is a $\langle$~$1$-complemented / $c$-complemented~$\rangle$
    right ideal of $A$ and $P$ is $\langle$~metrically /
    $C$-topologically~$\rangle$ projective $A$-module, then $P$ is
    $\langle$~metrically / $cC$-topologically~$\rangle$ projective $I$-module.
\end{enumerate}
\end{proposition}
\begin{proof} By $\widetilde{\pi}_P: I\projtens \ell_1(B_P)\to P$ and
$\pi_P:A\projtens \ell_1(B_P)\to P$ we will denote the standard epimorphisms.

$(i)$ By proposition~\ref{NonDegenMetTopProjCharac} the morphism
$\widetilde{\pi}_P$ has a right inverse morphism in $\langle$~$I-\mathbf{mod}_1$
/ $I-\mathbf{mod}$~$\rangle$, say $\widetilde{\sigma}$ of norm $\langle$~at most
$1$ / at most $C$~$\rangle$. Let $i:I\to A$ be the natural embedding, then
consider $\langle$~contractive / bounded~$\rangle$ $I$-morphism
$\sigma=(i\projtens 1_{\ell_1(B_P)})\widetilde{\sigma}$. By paragraph $(i)$ of
proposition~\ref{MorphCoincide} we have that $\sigma$ is an $A$-morphism.
Clearly, $\sigma$ has norm $\langle$~at most $1$ / at most $C$~$\rangle$. For
$\pi_P:A\projtens \ell_1(B_P)\to P$ we obviously 
have $\pi_P(i\projtens 1_{\ell_1(B_P)})=\widetilde{\pi}_P$, 
hence 
$\pi_P\sigma
=\pi_P(i\projtens 1_{\ell_1(B_P)})\widetilde{\sigma}
=\widetilde{\pi}_P\widetilde{\sigma}
=1_P$.
Thus, $\pi_P$ is a $\langle$~$1$-retraction / $C$-retraction~$\rangle$ in
$\langle$~$A-\mathbf{mod}_1$ / $A-\mathbf{mod}$~$\rangle$. So by
proposition~\ref{NonDegenMetTopProjCharac} the $A$-module $P$ is
$\langle$~metrically / $C$-topologically~$\rangle$ projective.

$(ii)$ Since $P$ is an essential $I$-module it is a fortiori an essential
$A$-module. By proposition~\ref{NonDegenMetTopProjCharac} the morphism $\pi_P$
has a right inverse morphism $\sigma$ in $\langle$~$A-\mathbf{mod}_1$ /
$A-\mathbf{mod}$~$\rangle$ with norm $\langle$~at most $1$ / at most
$C$~$\rangle$. Obviously $\sigma$ is a right inverse for $\pi_P$ in
$\langle$~$I-\mathbf{mod}_1$ / $I-\mathbf{mod}$~$\rangle$ too. By $i:I\to A$ we
denote the natural embedding, and by $r:A\to I$ the $\langle$~contractive /
bounded~$\rangle$ left inverse. By assumption $\Vert r\Vert\leq c$. Consider
$\langle$~contractive / bounded~$\rangle$ linear operator
$\widetilde{\sigma}=(r\projtens 1_{\ell_1(B_P)})\sigma$. Clearly, its norm is
$\langle$~at most $1$ / at most $cC$~$\rangle$. Since $I$ is a right ideal of
$A$ and $P$ is an essential $I$-module then
$\sigma(P)
=\sigma(\operatorname{cl}_P(IP))
=\operatorname{cl}_{A\projtens \ell_1(B_P)}(I\cdot (A\projtens \ell_1(B_P)))
=I\projtens \ell_1(B_P)$, so
$\sigma=(ir\projtens 1_{\ell_1(B_P)})\sigma$. Even more, 
since $\sigma(P)\subset I\projtens\ell_1(B_P)$ and $r|_I=1_I$, 
then $\sigma$ is an $I$-morphism.
Clearly, $\pi_P(i\projtens 1_{\ell_1(B_P)})=\widetilde{\pi}_P$, so
$$
\widetilde{\pi}_P\widetilde{\sigma}
=\pi_P(i\projtens 1_{\ell_1(B_P)})(r \projtens 1_{\ell_1(B_P)})\sigma
=\pi_P(ir\projtens 1_{\ell_1(B_P)})\sigma
=\pi_P\sigma
=1_P
$$ 
Thus $\widetilde{\pi}_P$ is 
a $\langle$~$1$-retraction / $cC$-retraction~$\rangle$ 
in $\langle$~$I-\mathbf{mod}_1$ / $I-\mathbf{mod}$~$\rangle$, so by
proposition~\ref{NonDegenMetTopProjCharac} the $I$-module $P$ is
$\langle$~metrically / $cC$-topologically~$\rangle$ projective.
\end{proof}

\begin{proposition}\label{MetTopInjUnderChangeOfAlg} Let $I$ be a closed
subalgebra of $A$ and $J$ be a right $A$-module which is faithful as $I$-module.
Then

\begin{enumerate}[label = (\roman*)]
    \item if $I$ is a left ideal of $A$ and $J$ is $\langle$~metrically /
    $C$-topologically~$\rangle$  injective $I$-module, then $J$ is
    $\langle$~metrically / $C$-topologically~$\rangle$ injective $A$-module;

    \item if $I$ is a weakly $\langle$~$1$-complemented /
    $c$-complemented~$\rangle$ right ideal of $A$ and $J$ is
    $\langle$~metrically / $C$-topologically~$\rangle$ injective $A$-module,
    then $J$ is $\langle$~metrically / $cC$-topologically~$\rangle$ injective
    $I$-module.
\end{enumerate}
\end{proposition}
\begin{proof} By $\widetilde{\rho}_J:J\to\mathcal{B}(I,\ell_\infty(B_{J^*}))$
and $\rho_J:J\to\mathcal{B}(A,\ell_\infty(B_{J^*}))$ we will denote the standard
monomorphisms.

$(i)$ By proposition~\ref{NonDegenMetTopInjCharac} the morphism
$\widetilde{\rho}_J: J\to\mathcal{B}(I,\ell_\infty(B_{J^*}))$ has a left inverse
morphism in $\langle$~$\mathbf{mod}_1-I$ / $\mathbf{mod}-I$~$\rangle$, say
$\widetilde{\tau}$ of norm $\langle$~at most $1$ / at most $C$~$\rangle$. Let
$i:I\to A$ be the natural embedding, and define $I$-morphism
$q=\mathcal{B}(i,\ell_\infty(B_{J^*}))$. Obviously
$\widetilde{\rho}_J=q\rho_J$. Consider $I$-morphism $\tau =\widetilde{\tau} q$.
By paragraph $(ii)$ of proposition~\ref{MorphCoincide} it is also an
$A$-morphism. Note that 
$\Vert\tau \Vert
\leq\Vert\widetilde{\tau}\Vert\Vert q\Vert
\leq\Vert\widetilde{\tau}\Vert$, 
so $\tau$ has norm $\langle$~at most $1$ / at most $C$~$\rangle$. 
Clearly, 
$\tau \rho_J=\widetilde{\tau} q\rho_J=\widetilde{\tau}\widetilde{\rho}_J=1_J$. 
Thus, $\rho_J$ is a $\langle$~$1$-coretraction / $C$-coretraction~$\rangle$, 
so by proposition~\ref{NonDegenMetTopInjCharac} the $A$-module $J$ is
$\langle$~metrically / $C$-topologically~$\rangle$ injective.

$(ii)$ If $J$ is $\langle$~metrically / $C$-topologically~$\rangle$ injective as
$A$-module, then by proposition~\ref{NonDegenMetTopInjCharac} the $A$-morphism
$\rho_J$ has a left inverse in $\langle$~$\mathbf{mod}_1-A$ /
$\mathbf{mod}-A$~$\rangle$, say $\tau $ of norm $\langle$~at most $1$ / at most
$C$~$\rangle$. Assume we are given an operator $T\in
\mathcal{B}(A,\ell_\infty(B_{J^*}))$, such that $T|_I=0$. Fix $a\in I$, then
$T\cdot a=0$, and so $\tau (T)\cdot a=\tau (T\cdot a)=0$. Since $J$ is faithful
$I$-module and $a\in I$ is arbitrary, then $\tau (T)=0$. Since $I$ is weakly 
$\langle$~$1$-complemented /$c$-complemented~$\rangle$ then $i^*$ has left 
inverse $r:I^*\to A^*$ with norm $\langle$~at most $1$ / at most $c$~$\rangle$.
For a given $f\in B_{J^*}$ we define a bounded linear operator 
$g_f:\mathcal{B}(I,\ell_\infty(B_{J^*}))\to I^*:T\mapsto(x\mapsto T(x)(f))$. 
Now we can define two more bounded linear operators
$$
j
:\mathcal{B}(I,\ell_\infty(B_{J^*}))\to \mathcal{B}(A,\ell_\infty(B_{J^*}))
:T\mapsto (a\mapsto (f\mapsto r(g_{f}(T))(a)))
$$ 
and $\widetilde{\tau}=\tau  j$. Fix $a\in I$ 
and $T\in\mathcal{B}(I,\ell_\infty(B_{J^*}))$. Since $r$ is a left inverse 
of $i^*$ we have $r(h)(a)=h(a)$ for all $h\in I^*$. Now it is routine to check 
that $(j(T\cdot a)-j(T)\cdot a)|_I=0$. As we have shown earlier this implies
that $\widetilde{\tau}(T\cdot a)-\widetilde{\tau}(T)\cdot a
=\tau (j(T\cdot a)-j(T)\cdot a)=0$. Since $a\in I$ 
and $T\in\mathcal{B}(I,\ell_\infty(B_{J^*}))$ are arbitrary the 
map $\widetilde{\tau}$ is an $I$-morphism.
Note that 
$\Vert\widetilde{\tau}\Vert\leq\Vert\tau \Vert\Vert j\Vert\leq \Vert
\tau\Vert\Vert r\Vert$, so $\widetilde{\tau}$ has 
norm $\langle$~at most $1$ / at most $cC$~$\rangle$.
In the same way one can show that, for all $x\in J$ 
holds $\rho_J(x)-j(\widetilde{\rho}_J(x))|_I=0$,
so $\tau (\rho_J(x)-j(\widetilde{\rho}_J(x)))=0$. As a consequence,
$\widetilde{\tau}(\widetilde{\rho}_J(x))
=\tau (j(\widetilde{\rho}_J(x)))=\tau(\rho_J(x))=x$ 
for all $x\in J$. Since $\widetilde{\tau}\widetilde{\rho}_J=1_J$,
then $\widetilde{\rho}_J$ is a  $\langle$~$1$-coretraction /
$cC$-coretraction~$\rangle$ in $\langle$~$\mathbf{mod}_1-I$ /
$\mathbf{mod}-I$~$\rangle$, so by proposition~\ref{NonDegenMetTopInjCharac} the
$I$-module $J$ is $\langle$~metrically / $cC$-topologically~$\rangle$ injective.
\end{proof}


\begin{proposition}\label{MetTopFlatUnderChangeOfAlg} Let $I$ be a closed
subalgebra of $A$ and $F$ be an $A$-module which is essential as $I$-module.
Then

\begin{enumerate}[label = (\roman*)]
    \item if $I$ is a left ideal of $A$ and $F$ is $\langle$~metrically /
    $C$-topologically~$\rangle$  flat $I$-module, then $F$ is
    $\langle$~metrically / $C$-topologically~$\rangle$ flat $A$-module;

    \item if $I$ is a weakly $\langle$~$1$-complemented / 
    $c$-complemented~$\rangle$ right ideal of $A$ and $F$ 
    is $\langle$~metrically / $C$-topologically~$\rangle$ flat $A$-module, 
    then $F$ is $\langle$~metrically / $cC$-topologically~$\rangle$ 
    flat $I$-module.
\end{enumerate}
\end{proposition}
\begin{proof} Note that the dual of essential module is faithful. Now the result
follows from propositions~\ref{MetCTopFlatCharac} 
and~\ref{MetTopInjUnderChangeOfAlg}.
\end{proof}    

\begin{proposition}\label{MetTopProjInjFlatUnderSumOfAlg} Let
${(A_\lambda)}_{\lambda\in\Lambda}$ be a family of Banach algebras and for each
$\lambda\in\Lambda$ let $X_\lambda$ be $\langle$~an essential / a faithful / an
essential~$\rangle$ $A_\lambda$-module. Denote $A=\bigoplus_p
\{A_\lambda:\lambda\in\Lambda \}$ for $1\leq p\leq +\infty$ or $p=0$. Let $X$
denote $\langle$~$\bigoplus_1 \{X_\lambda:\lambda\in\Lambda \}$ /
$\bigoplus_\infty \{X_\lambda:\lambda\in\Lambda \}$ / 
$\bigoplus_1 \{X_\lambda:\lambda\in\Lambda \}$~$\rangle$. Then

\begin{enumerate}[label = (\roman*)]
    \item $X$ is metrically $\langle$~projective / injective / flat~$\rangle$
    $A$-module iff for all $\lambda\in\Lambda$ the $A_\lambda$-module
    $X_\lambda$ is metrically $\langle$~projective / injective / flat~$\rangle$;

    \item $X$ is $C$-topologically $\langle$~projective / injective /
    flat~$\rangle$ $A$-module iff for all $\lambda\in\Lambda$ the
    $A_\lambda$-module $X_\lambda$ is $C$-topologically $\langle$~projective /
    injective / flat~$\rangle$.
\end{enumerate}
\end{proposition}
\begin{proof} Note that for each $\lambda\in\Lambda$ the natural embedding
$i_\lambda:A_\lambda\to A$ allows regarding $A_\lambda$ as complemented in
$\mathbf{Ban}_1$ two-sided ideal of $A$.

$(i)$ The proof is literally the same as in paragraph $(ii)$.

$(ii)$ Assume $X_\lambda$ is $C$-topologically $\langle$~projective / injective
/ flat~$\rangle$ $A_\lambda$-module for all $\lambda\in\Lambda$, then by
paragraph $(i)$ of proposition $\langle$~\ref{MetTopProjUnderChangeOfAlg}
/~\ref{MetTopInjUnderChangeOfAlg} /~\ref{MetTopFlatUnderChangeOfAlg}~$\rangle$
it is $C$-topologically $\langle$~projective / injective / flat~$\rangle$ as
$A$-module. It remains to apply the proposition
$\langle$~\ref{MetTopProjModCoprod} /~\ref{MetTopInjModProd}
/~\ref{MetTopFlatModCoProd}~$\rangle$. 

Conversely, assume that $X$ is $C$-topologically $\langle$~projective /
injective / flat~$\rangle$ as $A$-module. Fix arbitrary $\lambda\in\Lambda$.
Clearly, we may regard $X_\lambda$ as $A$-module and even more $X_\lambda$ is a
$1$-retract of $X$ in $\langle$~$A-\mathbf{mod}_1$ / $\mathbf{mod}_1-A$ /
$A-\mathbf{mod}_1$~$\rangle$. By proposition
$\langle$~\ref{RetrMetCTopProjIsMetCTopProj} /~\ref{RetrMetCTopInjIsMetCTopInj}
/~\ref{RetrMetCTopFlatIsMetCTopFlat}~$\rangle$ we get that $X_\lambda$ is
$C$-topologically $\langle$~projective / injective / flat~$\rangle$ as
$A$-module. It remains to apply paragraph $(ii)$ of proposition
$\langle$~\ref{MetTopProjUnderChangeOfAlg} /~\ref{MetTopInjUnderChangeOfAlg}
/~\ref{MetTopFlatUnderChangeOfAlg}~$\rangle$.
\end{proof} 

%----------------------------------------------------------------------------------------
%    Further properties of flat modules
%----------------------------------------------------------------------------------------

\subsection{
    Further properties of flat modules}\label{SubSectionFurtherPropertiesOfFlatModules}

Based on results obtained above, we collect more interesting facts on metric and
topological injectivity and flatness of Banach modules.

\begin{proposition}\label{DualBanModDecomp} Let $B$ be a unital Banach algebra,
$A$ be its subalgebra with two-sided bounded approximate identity
${(e_\nu)}_{\nu\in N}$ and $X$ be a left $B$-module. 
Denote $c_1=\sup_{\nu\in N}\Vert e_\nu\Vert$, 
$c_2=\sup_{\nu\in N}\Vert e_B-e_\nu\Vert$ and $X_{ess}=\operatorname{cl}_X(AX)$.
Then 

\begin{enumerate}[label = (\roman*)]
    \item $X^*$ is $c_2(c_1+1)$-isomorphic as a right $A$-module to
    $X_{ess}^*\bigoplus_\infty {(X/X_{ess})}^*$;

    \item $\langle$~$X_{ess}^*$ / ${(X/X_{ess})}^*$~$\rangle$ is a $\langle$
    $c_1$-retract / $c_2$-retract~$\rangle$ of $A$-module $X^*$;

    \item if $X$ is an $\mathscr{L}_{1,C}^g$-space, then $\langle$~$X_{ess}$ /
    $X/X_{ess}$~$\rangle$ is an $\langle$~$\mathscr{L}_{1,c_1C}^g$-space /
    $\mathscr{L}_{1,c_2C}^g$-space~$\rangle$.
\end{enumerate}

\end{proposition}
\begin{proof} $(i)$ Define the natural embedding $\rho:X_{ess}\to X:x\mapsto x$
and the quotient map  $\pi:X\to X/X_{ess}:x\mapsto x+X_{ess}$. Let
$\mathfrak{F}$ be the section filter on $N$ and let $\mathfrak{U}$ be an
ultrafilter dominating $\mathfrak{F}$. For a fixed $f\in X ^*$ and $x\in X $ we
have 
$|f(x-e_\nu\cdot x)|
\leq\Vert f\Vert\Vert e_B - e_\nu\Vert\Vert x\Vert
\leq c_2\Vert f\Vert\Vert x\Vert$ i.e. ${(f(x-e_\nu\cdot x))}_{\nu\in N}$ is a
bounded net of complex numbers. Therefore, we have a well-defined limit
$\lim_{\mathfrak{U}}f(x-e_\nu\cdot x)$ along ultrafilter $\mathfrak{U}$. Since
${(e_\nu)}_{\nu\in N}$ is a two-sided approximate identity for $A$ and
$\mathfrak{U}$ contains section filter then for all $x\in X_{ess}$ we have
$\lim_{\mathfrak{U}}f(x-e_\nu\cdot x)=\lim_{\nu}f(x-e_\nu\cdot x)=0$. Therefore,
for each $f\in X ^*$ we have a well-defined map 
$\tau(f)
:X /X_{ess}\to \mathbb{C}
:x+X_{ess}\mapsto \lim_{\mathfrak{U}} f(x-e_\nu\cdot x)$. 
Clearly this is a linear functional, and from inequalities above we see 
its norm is bounded by $c_2\Vert f\Vert$. Now it is routine to check 
that $\tau:X^*\to {(X/ X_{ess})}^*:f\mapsto \tau(f)$ is an $A$-morphism 
with norm not greater than $c_2$. Similarly, one can show that 
$\sigma
:X_{ess}^*\to X^*
:h\mapsto(x\mapsto \lim_{\mathfrak{U}}h(e_\nu\cdot x))$ is an $A$-morphism 
with norm not greater than $c_1$. For any $f\in X^*$, $g\in {(X/X_{ess})}^*$, 
$h\in X_{ess}^*$ and $x\in X$, $y\in X_{ess}$ we have
$$
\sigma(h)(y)
=\lim_{\mathfrak{U}}h(e_\nu\cdot y)
=\lim_{\nu}h(e_\nu\cdot y)
=h(y),
\qquad
(\rho^*\sigma)(h)(y)
=\sigma(h)(\rho(y))
\sigma(h)(y)
=h(y),
$$
$$
(\tau\pi^*)(g)(x+X_{ess})
=\lim_{\mathfrak{U}}\pi^*(g)(x-e_\nu\cdot x)
=\lim_{\mathfrak{U}}g(x+X_{ess})
=g(x+X_{ess}),
$$
$$
(\tau\sigma)(h)(x+X_{ess})
=\lim_{\mathfrak{U}}\sigma(h)(x-e_\nu\cdot x)
=\lim_{\mathfrak{U}}(\sigma(h)(x)-h(e_\nu\cdot x))
=\sigma(h)(x)-\lim_{\mathfrak{U}}h(e_\nu\cdot x)=0,
$$
$$
(\pi^*\tau + \sigma\rho^*)(f)(x)
=\tau(f)(x+X_{ess})+\lim_{\mathfrak{U}}\rho^*(f)(e_\nu\cdot x)
=\lim_{\mathfrak{U}}f(x - e_\nu\cdot x)+\lim_{\mathfrak{U}}f(e_\nu\cdot x)
=f(x).
$$
Therefore, $\tau \pi^*=1_{{(X/X_{ess})}^*}$, $\rho^*\sigma=1_{X_{ess}^*}$ and
$\pi^*\tau+\sigma\rho^*=1_{X^*}$. Now it is easy to check that the linear maps
$$
\xi
:X^*\to X_{ess}^*\bigoplus{}_\infty {(X/X_{ess})}^*
:f\mapsto \rho^*(f)\bigoplus{}_\infty \tau(f)
$$
$$
\eta
:X_{ess}^*\bigoplus{}_\infty {(X/X_{ess})}^*\to X^*
:h\bigoplus{}_\infty g\mapsto \pi^*(h)+\sigma(g)
$$
are isomorphism of right $A$-modules with $\Vert\xi \Vert\leq c_2$ 
and $\Vert \eta\Vert\leq c_1+1$. Hence, $X^*$ is $c_2(c_1+1)$-isomorphic 
in $\mathbf{mod}-A$ to $X_{ess}^*\bigoplus_\infty {(X/X_{ess})}^*$.

$(ii)$ Both results immediately follow from equalities
$\rho^*\sigma=1_{X_{ess}^*}$, $\tau \pi^*=1_{{(X/X_{ess})}^*}$ and estimates
$\Vert \rho^*\Vert\Vert \sigma\Vert\leq c_1$, 
$\Vert\tau\Vert\Vert \pi^*\Vert\leq c_2$.

$(iii)$ Now consider case when $X$ is an $\mathscr{L}_{1,C}^g$-space. Then $X^*$
is an $\mathscr{L}_{\infty,C}^g$-space [\cite{DefFloTensNorOpId}, corollary
23.2.1(1)]. As $\langle$~$X_{ess}^*$ / ${(X/X_{ess})}^*$~$\rangle$ is
$\langle$~$c_1$-complemented / $c_2$-complemented~$\rangle$ in $X^*$ it is an
$\langle$~$\mathscr{L}_{\infty,c_1C}^g$-space /
$\mathscr{L}_{\infty,c_2C}^g$-space~$\rangle$ by [\cite{DefFloTensNorOpId},
corollary 23.2.1(1)]. Again we apply [\cite{DefFloTensNorOpId}, corollary
23.2.1(1)] to conclude that $\langle$~$X_{ess}$  / $X/X_{ess}$~$\rangle$ is an
$\langle$~$\mathscr{L}_{1,c_1C}^g$-space /
$\mathscr{L}_{1,c_2C}^g$-space~$\rangle$.
\end{proof}

The following proposition is an analog 
of [\cite{RamsHomPropSemgroupAlg}, proposition 2.1.11].

\begin{proposition}\label{TopFlatModCharac} Let $A$ be a Banach algebra with
two-sided $c$-bounded approximate identity, and $F$ be a left $A$-module. Then

\begin{enumerate}[label = (\roman*)]
    \item if $F$ is $C$-topologically flat $A$-module, then $F_{ess}$ is
    $(1+c)C$-topologically flat $A$-module and $F/F_{ess}$ is an
    $\mathscr{L}_{1,(1+c)C}^g$-space;

    \item if $F_{ess}$ is $C_1$-topologically flat $A$-module and $F/F_{ess}$ is
    an $\mathscr{L}_{1,C_1}^g$-space, then $F$ is ${(1+c)}^2\max(C_1,
    C_2)$-topologically flat $A$-module.

    \item $F$ is topologically flat $A$-module iff $F_{ess}$  is topologically
    flat $A$-module and $F/F_{ess}$ is an $\mathscr{L}_1^g$-space.
\end{enumerate}
\end{proposition}
\begin{proof} We regard $A$ as closed subalgebra of unital Banach algebra
$B:=A_+$. Then $F$ is unital left $B$-module. Using notation of
proposition~\ref{DualBanModDecomp} we may say that $c_1=c$ and $c_2=1+c$, so the
right $A$-modules $F_{ess}^*$ and ${(F/F_{ess})}^*$ are $(1+c)$-retracts of
$F^*$.

$i)$ By proposition~\ref{MetCTopFlatCharac} the right $A$-module $F^*$ is
$C$-topologically injective. Therefore, from
propositions~\ref{RetrMetCTopInjIsMetCTopInj},~\ref{MetCTopFlatCharac} the modules
$F_{ess}$ and $F/F_{ess}$ are $(1+c)C$-topologically flat. It remains to note
that $F/F_{ess}$ is an annihilator $A$-module, so by
proposition~\ref{MetTopFlatAnnihModCharac} it is an
$\mathscr{L}_{1,(1+c)C}^g$-space.

$(ii)$ Again, by proposition~\ref{MetCTopFlatCharac} the right $A$-modules
$F_{ess}^*$ and ${(F/F_{ess})}^*$ are $C_1$- and $C_2$-topologically injective
respectively. So from proposition~\ref{MetTopInjModProd} their product is
$\max(C_1,C_2)$-topologically injective. By proposition~\ref{DualBanModDecomp}
this product is ${(1+c)}^2$-isomorphic to $F^*$ in $\mathbf{mod}-A$. Therefore,
$F^*$ is ${(1+c)}^2\max(C_1, C_2)$-topologically injective $A$-module. Now the
result follows from proposition~\ref{MetCTopFlatCharac}.

$(iii)$ The result immediately follows from paragraphs $(i)$ and $(ii)$.
\end{proof}

\begin{proposition}\label{MetTopEssL1FlatModAoverAmenBanAlg} Let $A$ be a
$\langle$~$1$-relatively / $c$-relatively~$\rangle$ amenable Banach algebra and
$F$ be an essential Banach $A$-module which is an $\langle$~$L_1$-space /
$\mathscr{L}_{1,C}^g$-space~$\rangle$. Then $F$ is a $\langle$~metrically /
$c^2C$-topologically~$\rangle$ flat $A$-module.
\end{proposition}
\begin{proof} We may assume that $A$ is $c$-relatively amenable for
$\langle$~$c=1$ / $c\geq 1$~$\rangle$. Let ${(d_\nu)}_{\nu\in N}$ be an
approximate diagonal for $A$ with norm bound at most $c$. Recall, that
${(\Pi_A(d_\nu))}_{\nu\in N}$ is a two-sided $\langle$~contractive /
bounded~$\rangle$ approximate identity for $A$. Since $F$ is essential left
$A$-module, then $\lim_{\nu}\Pi_A(d_\nu)\cdot x=x$ for all $x\in F$
[\cite{HelHomolBanTopAlg}, proposition 0.3.15]. As the consequence
$c\pi_F(B_{A\projtens\ell_1(B_F)})$ is dense in $B_F$. Then for all $f\in F^*$
we have
$$
\Vert\pi_F^*(f)\Vert
=\sup \{|f(\pi_F(u))|:u\in B_{A\projtens\ell_1(B_F)} \}
=\sup \{|f(x)|:x\in \operatorname{cl}_F(\pi_F(B_{A\projtens\ell_1(B_F)})) \}
$$
$$
\geq\sup \{c^{-1}|f(x)|:x\in B_F \}=c^{-1}\Vert f\Vert.
$$
This means, that $\pi_F^*$ is $c$-topologically injective. By assumption $F$ is
an $\langle$~$L_1$-space / $\mathscr{L}_{1,C}^g$-space~$\rangle$, then by
$\langle$~[\cite{GrothMetrProjFlatBanSp}, theorem 1] / remark after
[\cite{DefFloTensNorOpId}, corollary 23.5(1)]~$\rangle$ the Banach space $F^*$
is $\langle$~metrically / $C$-topologically~$\rangle$ injective. Since operator
$\pi_F^*$ is $\langle$~isometric / $c$-topologically injective~$\rangle$, then
there exists a linear operator $R:{(A\projtens\ell_1(B_F))}^*\to F^*$ of norm
$\langle$~at most $1$ / at most $cC$~$\rangle$ such that $R\pi_F^*=1_{F^*}$.

Fix $h\in {(A\projtens\ell_1(B_F))}^*$ and $x\in F$. Consider bilinear
functional $M_{h,x}:A\times A\to\mathbb{C}:(a,b)\mapsto R(h\cdot a)(b\cdot x)$.
Clearly, $\Vert M_{h,x}\Vert\leq\Vert R\Vert\Vert h\Vert\Vert x\Vert$. By
universal property of projective tensor product we have a bounded linear
functional 
$m_{h,x}:A\projtens A\to\mathbb{C}:a\projtens b\mapsto R(h\cdot a)(b\cdot x)$. 
Note that $m_{h,x}$ is linear in $h$ and $x$. Even more, for any
$u\in A\projtens A$, $a\in A$ and $f\in F^*$ we have
$m_{\pi_F^*(f),x}(u)=f(\Pi_A(u)\cdot x)$, $m_{h\cdot a,x}(u)=m_{h,x}(a\cdot u)$,
$m_{h,a\cdot x}(u)=m_{h,x}(u\cdot a)$. It easily checked for elementary tensors.
Then it is enough to recall that their linear span is dense in $A\projtens A$.

Let $\mathfrak{F}$ be the section filter on $N$ and let $\mathfrak{U}$ be an
ultrafilter dominating $\mathfrak{F}$. For any 
$h\in {(A\projtens\ell_1(B_F))}^*$ and $x\in F$ we 
have $|m_{h,x}(d_\nu)|\leq c\Vert R\Vert\Vert h\Vert\Vert x\Vert$, 
i.e. ${(m_{h,x}(d_\nu))}_{\nu\in N}$ is a
bounded net of complex numbers. Therefore, we have a well-defined limit
$\lim_{\mathfrak{U}}m_{h,x}(d_\nu)$ along ultrafilter $\mathfrak{U}$. Consider
linear operator 
$\tau
:{(A\projtens\ell_1(B_F))}^*\to F^*
:h\mapsto(x\mapsto\lim_{\mathfrak{U}}m_{h,x}(d_\nu))$. From 
norm estimates for $m_{h,x}$ it follows that $\tau$ is bounded 
with $\Vert\tau\Vert\leq c\Vert R\Vert$. For all $a\in A$, $x\in F$ 
and $h\in {(A\projtens\ell_1(B_F))}^*$ we have
$$
\tau(h\cdot a)(x)-(\tau(h)\cdot a)(x)
=\tau(h\cdot a)(x)-\tau(h)(a\cdot x)
=\lim_{\mathfrak{U}}m_{h\cdot a,x}(d_\nu)
-\lim_{\mathfrak{U}}m_{h,a\cdot x}(d_\nu).
$$
$$
=\lim_{\mathfrak{U}}m_{h,x}(a\cdot d_\nu)-m_{h,x}(d_\nu\cdot a)
=m_{h,x}\left(\lim_{\mathfrak{U}}(a\cdot d_\nu-d_\nu\cdot a)\right)
$$
$$
=m_{h,x}\left(\lim_{\nu}(a\cdot d_\nu-d_\nu\cdot a)\right)
=m_{h,x}(0)
=0.
$$
Therefore, $\tau$ is a morphism of right $A$-modules. Now for all $f\in F^*$ and
$x\in F$ we have
$$
(\tau(\pi_F^*)(f))(x)
=\lim_{\mathfrak{U}}m_{\pi_F^*(f),x}(d_\nu)
=\lim_{\mathfrak{U}}f(\Pi_A(d_\nu)\cdot x)
=\lim_{\nu}f(\Pi_A(d_\nu)\cdot x)
$$
$$
=f\left(\lim_{\nu}\Pi_A(d_\nu)\cdot x\right)
=f(x).
$$
So $\tau\pi_F^*=1_{F^*}$. This means that $F^*$ is a $\langle$~$1$-retract /
 $c^2 C$-retract~$\rangle$ of ${(A\projtens\ell_1(B_F))}^*$ in
 $\langle$~$\mathbf{mod}_1-A$ / $\mathbf{mod}-A$~$\rangle$. The latter
 $A$-module is $\langle$~metrically / topologically~$\rangle$ injective, because
 ${(A_+\projtens\ell_1(B_F))}^*
 \isom{\mathbf{mod}_1-A}
 \mathcal{B}(A_+,\ell_\infty(B_F))$
 and by proposition~\ref{RetrMetCTopInjIsMetCTopInj} so does its 
 retract $F^*$. By proposition~\ref{MetCTopFlatCharac} this is equivalent 
 to $\langle$~metric / $c^2 C$-topological~$\rangle$ flatness of $F$.
\end{proof}

\begin{theorem}\label{TopL1FlatModAoverAmenBanAlg} Let $A$ be a $c$-relatively
amenable Banach algebra and $F$ be a left Banach $A$-module which as Banach
space is an $\mathscr{L}_{1, C}^g$-space. Then $F$ is a
${(1+c)}^2C\max(c^2,(1+c))$-topologically flat $A$-module.
\end{theorem}
\begin{proof} Since $A$ is amenable it admits a two-sided $c$-bounded
approximate identity. By proposition~\ref{DualBanModDecomp} the annihilator
$A$-module $F/F_{ess}$ is an $\mathscr{L}_{1,1+c}^g$-space. From
proposition~\ref{MetTopEssL1FlatModAoverAmenBanAlg} we get that the essential
$A$-module $F_{ess}$ is $c^2 C$-topologically flat. Now the result follows from
proposition~\ref{TopFlatModCharac}.
\end{proof}

We must point out here that in relative Banach homology any left Banach module
over relatively amenable Banach algebra is relatively flat
[\cite{HelBanLocConvAlg}, theorem 7.1.60]. Even topological theory is so
restrictive that in some cases, as the following proposition shows, we can
obtain complete characterization of all flat modules.

\begin{proposition}\label{TopFlatModAoverAmenL1BanAlgCharac} Let $A$ be a
relatively amenable Banach algebra which as Banach space is an
$\mathscr{L}_1^g$-space. Then for a Banach $A$-module $F$ the following are
equivalent:

\begin{enumerate}[label = (\roman*)]
    \item $F$ is topologically flat $A$-module; 

    \item $F$ is an $\mathscr{L}_1^g$-space.
\end{enumerate}
\end{proposition}
\begin{proof} The equivalence follows from
proposition~\ref{TopProjInjFlatModOverMthscrL1SpCharac} and
theorem~\ref{TopL1FlatModAoverAmenBanAlg}.
\end{proof}

Finally, we are able to give an example of relatively flat, but not topologically
flat ideal in a Banach algebra. Consider $A=L_1(\mathbb{T})$. It is known, that
$A$ has a translation invariant infinite dimensional closed subspace $I$
isomorphic to a Hilbert space [\cite{RosProjTransInvSbspLpG}, p.52]. By
[\cite{KaniBanAlg}, proposition 1.4.7] we have that $I$ is a two-sided ideal of
$A$, as any translation invariant subspace of $A$. By [\cite{DefFloTensNorOpId},
section 23.3] this ideal is not an $\mathscr{L}_1^g$-space. So from
proposition~\ref{TopFlatModAoverAmenL1BanAlgCharac} we get that $I$ is not
topologically flat as $A$-module. We claim it is still relatively flat. Since
$\mathbb{T}$ is a compact group, then it is amenable [\cite{PierAmenLCA},
proposition 3.12.1]. Thus, $A$ is relatively amenable [\cite{HelBanLocConvAlg},
proposition VII.1.86], so all left ideals of $A$ are relatively flat
[\cite{HelBanLocConvAlg}, proposition VII.1.60(I)]. In particular, $I$ is
relatively flat.

%-------------------------------------------------------------------------------
%    Injectivity of ideals
%-------------------------------------------------------------------------------

\subsection{
    Injectivity of ideals}\label{SubSectionInjectivityOfIdeals}

Injective ideals are rare creatures, but we need to say a few words about them.
Results of this section needed for the study of metric and topological
injectivity of $C^*$-algebras.

\begin{proposition}\label{MetTopInjOfId} Let $I$ be a right ideal of a Banach
algebra $A$. Assume $I$ is $\langle$~metrically / $C$-topologically~$\rangle$
injective $A$-module. Then $I$ has a left identity of norm $\langle$~at most $1$
/ at most $C$~$\rangle$ and is a $\langle$~$1$-retract / $C$-retract~$\rangle$
of $A$ in $\mathbf{mod}-A$.
\end{proposition}
\begin{proof} Consider isometric embedding $\rho^+:I\to A_+$ of $I$ into $A_+$.
Clearly, this is an $A$-morphism. Since $I$ is $\langle$~metrically /
$C$-topologically~$\rangle$ injective, then $\rho^+$ has left inverse
$A$-morphism $\tau^+:A_+\to I$ with norm $\langle$~at most $1$ / at most
$C$~$\rangle$. Now for all $x\in I$ we have
$x
=\tau^+(\rho^+(x))
=\tau^+(e_{A_+}\rho^+(x))
=\tau(e_{A_+})\rho^+(x)=\tau^+(e_{A_+})x$.
In other words $p=\tau^+(e_{A_+})\in I$ is a left unit for $I$. Clearly, 
$\Vert p\Vert
\leq\Vert\tau^+\Vert\Vert e_{A_+}\Vert
\leq\Vert\tau^+\Vert$. Consider maps
$\rho:I\to A:x\mapsto x$ and $\tau:A\to I:x\mapsto p x$. Clearly, they are
morphisms of right $A$-modules and $\tau\rho=1_I$. Hence, $I$ is a
$\langle$~$1$-retract / $C$-retract~$\rangle$ of $A$ in $\mathbf{mod}-A$.
\end{proof}

\begin{proposition}\label{ReduceInjIdToInjAlg} Let $I$ be a two-sided ideal of
a Banach algebra $A$, which is faithful as right $I$-module. Then

\begin{enumerate}[label = (\roman*)]
    \item if $I$ is $\langle$~metrically / $C$-topologically~$\rangle$ injective
    $I$-module, then $I$ is $\langle$~metrically / $C$-topologically~$\rangle$
    injective $A$-module; 

    \item if $I$ is $\langle$~metrically / $C$-topologically~$\rangle$ injective
    $A$-module, then $I$ is $\langle$~metrically / $C^2$-topologically~$\rangle$
    injective $I$-module.
\end{enumerate}
\end{proposition}
\begin{proof} $(i)$ The result immediately follows from paragraph $(i)$ of
proposition~\ref{MetTopInjUnderChangeOfAlg}.

$(ii)$ By proposition~\ref{MetTopInjOfId} the $A$-module $I$ is
$\langle$~$1$-complemented / $C$-complemented~$\rangle$ in $A$. By paragraph
$(ii)$ of proposition~\ref{MetTopInjUnderChangeOfAlg} the $I$-module $I$ is
$\langle$~metrically / $C^2$-topologically~$\rangle$ injective.
\end{proof}
