%\title{Metric and topological freedom for sequential operator spaces}
\documentclass[12pt]{article}
\usepackage[left=2cm,right=2cm,
top=2cm,bottom=2cm,bindingoffset=0cm]{geometry}
\usepackage{amssymb,amsmath}
\usepackage[T1,T2A]{fontenc}
\usepackage[utf8]{inputenc}
\usepackage[matrix,arrow,curve]{xy}
\usepackage[russian]{babel} 
\usepackage[final]{graphicx} 
\usepackage{mathrsfs}
\usepackage[colorlinks=true, urlcolor=blue, linkcolor=blue, citecolor=blue, pdfborder={0 0 0}]{hyperref}
\usepackage{yhmath}


\newtheorem{theorem}{Теорема}[subsection]
\newtheorem{lemma}[theorem]{Лемма}
\newtheorem{proposition}[theorem]{Предложение}
\newtheorem{remark}[theorem]{Замечание}
\newtheorem{corollary}[theorem]{Следствие}
\newtheorem{definition}[theorem]{Определение}
\newtheorem{example}[theorem]{Пример}

\newenvironment{proof}{\par $\triangleleft$}{$\triangleright$}

\pagestyle{plain}

\begin{document}

\begin{center}

\Large \textbf{Метрическая и топологическая свобода для секвенциальных операторных пространств}\\[0.5cm]
\small {Норберт Немеш, Сергей Штейнер}\\[0.5cm]

\end{center}
\thispagestyle{empty}

\begin{abstract}
Цель данной статьи --- описание свободных и косвободных объектов для различных версий относительной гомологии в категории секвенциальных операторных пространств. Сначала мы докажем, что для данной категории есть теория двойственности аналогичная случаю нормированных пространств. Затем, основываясь на этих результатах, мы дадим полное описание метрически и тополоически свободных и косвободных объектов.
\end{abstract}


\section{Предварительные сведения}

\subsection{Двойственность для нормированных пространств}

\begin{definition}[\cite{HelQFA}, 0.0.1, 4.4.1]\label{DefDuality}
Пусть $E$, $F$ и $G$ --- нормированные пространства и $\mathcal{D}:E\times F\to G$ ограниченный билинейный оператор, тогда
\newline
1) $\mathcal{D}$ называется невырожденным слева (справа) если оператор
$$
{}^E\mathcal{D}:E\mapsto\mathcal{B}(F,G):x\mapsto(y\mapsto\mathcal{D}(x,y))\qquad
(\mathcal{D}^F:F\mapsto\mathcal{B}(E,G):y\mapsto(x\mapsto\mathcal{D}(x,y)))
$$ 
инъективен
\newline
2) $\mathcal{D}$ называется изометрическим слева (справа) если ${}^E\mathcal{D}$ ($\mathcal{D}^F$)  изометричен
\newline
3) $\mathcal{D}$ называется векторной двойственностью если он невырожден слева и справа
\newline
4) если $G=\mathbb{C}$, то векторная двойственность $\mathcal{D}$ называется скалярной двойственностью
\end{definition}
Билинейный операторы вида
$$
\mathcal{D}_{E,E^*}:E\times E^*\to\mathbb{C}:(x,f)\mapsto f(x)
\qquad
\mathcal{D}_{E^*,E}:E^*\times E\to\mathbb{C}:(f,x)\mapsto f(x)
$$будем называть стандартными скалярными двойственностями. Отметим, что для всех $x\in E$ и $f\in E^*$ имеют место равенства
$$
\Vert x\Vert=\sup\{|\mathcal{D}_{E,E^*}(x,f)|:f\in B_{E^*}\}
\qquad
\Vert f\Vert=\sup\{|\mathcal{D}_{E,E^*}(x,f)|:x\in  B_E\}
$$
Первое равенство является следствием теоремы Хана-Банаха, а второе опеределением нормы функционала. Ометим что $\mathcal{D}_{E,E^*}^E$ есть ничто иное как вложение $\iota_E$ во второе сопряженное пространство.
Если $T\in \mathcal{B}(E,F)$, то имеет место равенство $\mathcal{D}_{F,F^*}(T(x),g)=\mathcal{D}_{E,E^*}(x,T^*(g))$, где $x\in E$ и $g\in F^*$. 

\begin{definition} Пусть $\mathcal{D}:E\times F\to G$ --- векторная двойственность между нормированными пространствами $E$, $F$ и $G$. Будем говорить, что $(y_\nu)_{\nu\in N}\subset F$ $\mathcal{D}$-сходится к $y\in F$ если для любого $x\in E$ направленность $(\mathcal{D}(x,y_\nu-y))_{\nu\in N}$ сходится к $0$. Топологию задаваемую этим типом сходимости обозначим $\sigma_\mathcal{D}(F,E)$.
\end{definition}

Многие типы сходимости в функциональном анализе могут быть сформулированы в терминах $\mathcal{D}$-сходимости, напрмер слабая сходимость есть ничто иное как $\mathcal{D}_{X^*,X}$-сходимость. 

Для любого $p\in[1,+\infty]\cup\{0\}$ через $p'$ мы обозначаем сопряженное число, т.е. $p'=p/(p-1)$ для $p\in(1,+\infty)$ причем $1'=\infty$ и $0'=\infty'=1$. Напомним стандартный факт.

\begin{proposition}\label{PrSumDuality}
Пусть $\{E_\lambda:\lambda\in \Lambda\}$ --- семейство нормированных пространств и $p\in[1,+\infty]\cup\{0\}$, тогда для скалярной двойственности
$$
\mathcal{D}:\bigoplus{}_p^0\{E_\lambda:\lambda\in \Lambda\}\times \bigoplus{}_{p'}\{E_\lambda^*:\lambda\in \Lambda\}\to \mathbb{C}: (x,f)\mapsto\sum\limits_{\lambda\in \Lambda} f_\lambda(x_\lambda)
$$
линейный оператор $\mathcal{D}^{\bigoplus{}_{p'}\{E_\lambda^*:\lambda\in \Lambda\}}$ изометричен. Если $p\neq\infty$, то он изометрический изоморфизм.
\end{proposition}

Аналоичный результат верен для $\bigoplus{}_p$-сумм.






















\subsection{Операторы между нормированными пространствами}

\begin{definition}\label{DefNorOpType} Пусть $ T:E\to F$ --- ограниченный линейный оператор между нормированными пространствами, тогда $ T$ называется
\newline
1) сжимающим, если $\Vert T\Vert\leq 1$
\newline
2) \textit{$c$-топологически инъективным}, если существует $c > 0$ такая, что для любого $x \in E$ выполнено $\Vert x\Vert\leq c\Vert  T(x)\Vert$. Если упоминание константы $c$ не нужно то будем говорить, 
что $ T$ топологически инъективный.
\newline
3) \textit{(строго) $c$-топологически сюрьективным}, если любого $c'>c$ и любого $y\in F$ существует $x \in E$ такой, что $ T(x) = y$ и $\Vert x \Vert < c' \Vert y \Vert$ ($\Vert x \Vert \leq c \Vert y \Vert$). 
Если упоминание константы $c$ не нужно то будем говорить, что $ T$ (строго) топологически сюръективный
\newline
4) (строго) коизометрическим, если он сжимающий и (строго) $1$-топологически сюръективный
\end{definition}

\begin{proposition}\label{PrEquivDescOfIsomCoisomOp}
Пусть $E$, $F$ --- нормированные пространства,  и $T:E\to F$ ограниченный оператор. Тогда,
\newline
1) $T$ (строго) $c$-топологически сюръективен $\Longleftrightarrow$ $T(B_E^\circ)\supset c^{-1}B_F^\circ$ ($T(B_E)\supset c^{-1}B_F$) 
\newline
2) $T$ (строго) коизометричен $\Longleftrightarrow$ $T(B_E^\circ)=B_F^\circ$ ($T(B_E)=B_F$)
\end{proposition}
\begin{proof}
1) Пусть $T$ $c$-топологически сюръективен. Пусть $y\in c^{-1}B_F^\circ$, тогда существует $k'>1$ такой что $k'y\in c^{-1}B_F^\circ$. Рассмотрим $c'=k'c>c$. По предположению существует $x\in E$ такой что $T(x)=y$ и 
$\Vert x\Vert< c'\Vert y\Vert=c\Vert k'y\Vert<1$. Так как $y\in c^{-1}B_F^\circ$ произвольно, то $T(B_E^\circ)\supset c^{-1}B_F^\circ$. Обратно, пусть $T(B_E^\circ)\supset c^{-1}B_F^\circ$. Пусть $y\in F$ и $c'>c$, тогда 
$\tilde{y}=(c')^{-1}\Vert y\Vert^{-1}y\in c^{-1}B_F^\circ$. По предположению существует $\tilde{x}\in B_E^\circ$ такой что $T(\tilde{x})=\tilde{y}$. В этом случае для вектора $x:=c'\Vert y\Vert\tilde{x}$ имеем 
$\Vert x \Vert=c'\Vert y\Vert\Vert\tilde{x}\Vert< c'\Vert y\Vert$ и $T(x)=c'\Vert\ y\Vert T(\tilde{x})=c'\Vert y\Vert\tilde{y}=y$. Так как $y\in F$ и $c'>c$ произвольны, то $T$ $c$-топологически сюръективный оператор.
\newline
Пусть $T$ строго $c$-топологически сюръективен. Пусть $y\in c^{-1}B_F$, тогда по 
предположению существует $x\in E$ такой что $T(x)=y$ и $\Vert x\Vert\leq c\Vert 
y\Vert=1$. Так как $y\in c^{-1}B_F^\circ$ произвольно, то $T(B_E)\supset c^{-1}B_F$. 
Обратно, пусть $T(B_E)\supset c^{-1}B_F$. Пусть $y\in F$, тогда 
$\tilde{y}=c^{-1}\Vert y\Vert^{-1}y\in c^{-1}B_F$. По предположению существует 
$\tilde{x}\in B_E$ такой что $T(\tilde{x})=\tilde{y}$. В этом случае для вектора 
$x:=c\Vert y\Vert\tilde{x}$ имеем $\Vert x \Vert=c'\Vert 
y\Vert\Vert\tilde{x}\Vert\leq c\Vert y\Vert$ и $T(x)=c'\Vert\ y\Vert 
T(\tilde{x})=c'\Vert y\Vert\tilde{y}=y$. Так как $y\in F$ произволен, то $T$ строго 
$c$-топологически сюръективен.
\newline
2) Пусть $T$ коизометричен. Тогда $\Vert T\Vert\leq 1$ и поэтому $T(B_E^\circ)\subset B_F^\circ$. Из пункта 1) также следует, что $T(B_E^\circ)\supset B_F^\circ$. Учитывая обратное включение получаем 
$T(B_E^\circ)=B_F^\circ$. Обратно, если $T(B_E^\circ)=B_F^\circ$, то в частности $\Vert T\Vert\leq 1$ и $T(B_E^\circ)\supset B_F^\circ$. Тогда из пункта 1) заключаем что $T$ $1$-топологически сюръективен. 
Следовательно, $T$ коизометричен. Для строго коизометрического оператора рассуждения аналогичны.
\end{proof}


\begin{proposition}\label{PrDualOps} Пусть $ T:E\to F$ ограниченный оператор между нормированными пространствами и $c>0$, тогда
\newline
1) если $ T$ (строго) $c$-топологически сюръективен, то $ T^*$ $c$-топологически инъективен
\newline
2) если $ T$ $c$-топологически инъективен, то $ T^*$ строго $c$-топологически сюръективен
\newline
3) если $ T^*$ (строго) $c$-топологически сюръективен, то $ T$ $c$-топологически инъективен
\newline
4) если $ T^*$ $c$-топологически инъективен и $E$ полно, то $ T$ $c$-топологически сюръективен
\end{proposition}
\begin{proof}
1) Условие $c$-топологической сюръективности означает что $c^{-1}B_F^\circ\subset T(B_E^\circ)$, поэтому для любого $g\in F^*$ имеем
$$
\Vert  T^*(g)\Vert
=\sup\{|g( T(x))|:x\in B_E^\circ\}
=\sup\{|g(y)|: y\in T(B_E^\circ)\}
\geq\sup\{|g(y)|: y\in c^{-1}B_F^\circ\}
$$
$$
=\sup\{|g(c^{-1}y)|: y\in B_F^\circ\}
=c^{-1}\sup\{|g(y)|: y\in B_F^\circ\}
=c^{-1}\Vert g\Vert
$$
Так как $g\in F^*$ произвольно, то $ T^*$ $c$-топологически инъективно. Рассуждение для строго $c$-топологически сюръективного оператора аналогично.
\newline
2) Пусть $g\in E^*$. Так как $ T$ $c$-топологически инъективно, то $\tilde{ T}:= T|^{\mathrm{Im}( T)}$ топологический изоморфизм. Обозначим через $i:\mathrm{Im}( T)\to E$ естественное вложение, тогда 
$ T=i\tilde{ T}$. Рассмотрим функционал $f_0:=g\tilde{ T}^{-1}\in F^*$. По теореме Хана-Банаха существует функционал $f\in F^*$ такой что $\Vert f\Vert=\Vert f_0\Vert$ и $f_0=fi$. Тогда
$g=f_0\tilde{ T}=f_0 i\tilde{ T}=f T= T^*(f)$. Так как $ T$ $c$-топологически инъективен, то для любого $x\in F$ имеем
$$
|f(x)|=|g(\tilde{ T}^{-1}(x))|
\leq\Vert g\Vert\Vert \tilde{ T}^{-1}(x)\Vert
\leq\Vert g\Vert c\Vert  T(\tilde{ T}^{-1}(x))\Vert
\leq c\Vert g\Vert\Vert x\Vert
$$
Откуда $\Vert f\Vert\leq c\Vert g\Vert$. Так как $g\in E^*$ произовльно, то $ T^*$ строго $c$-топологически сюръективен.
\newline
3) Из пункта $1)$ следует, что $ T^{**}$ $c$-топологически инъективен. Так как естественное вложение во второе сопряженное пространство изометрични и выполнено равенство $\iota_F  T = T^{**}\iota_E$, 
то для любого $x\in E$ имеем
$$
\Vert T(x)\Vert
=\Vert \iota_F( T(x))\Vert
=\Vert T^{**}(\iota_E(x))\Vert
\geq c^{-1}\Vert \iota_E(x)\Vert
=c^{-1}\Vert x\Vert
$$
Так как $x\in E$ произвольно, то $ T$ $c$-топологически инъективен.
\newline
4) Допустим, что $c^{-1}B_F^\circ\not\subset\mathrm{cl}_F( T(B_E^\circ))$, тогда существует $y_0\in c^{-1}B_F^\circ\setminus\mathrm{cl}_F( T(B_E^\circ))$. В частности, $\Vert y_0\Vert<c^{-1}$. Рассмотрим множеcтва $A=\{y_0\}$, 
$B=\mathrm{cl}_F( T(B_E^\circ))$. Очевидно, $A$ компактно и выпукло. Так как $B_E^\circ$ выпукло, а $ T$ линейно, то $ T(B_E^\circ)$ выпукло и как следствие $B$ выпукло и замкнуто. По предложению 
3.4 \cite{RudinFA} существуют $g\in F^*$ и $\gamma_1,\gamma_2\in\mathbb{R}$ такие, что для любого $y\in B$ выполнено $\mathrm{Re}(g(y_0))>\gamma_2>\gamma_1>\mathrm{Re}(g(y))$. Без ограничения общности 
можно считать, что $\gamma_1>\gamma_2=1$. В частности, для любого $x\in B_E^\circ$ имеем $\mathrm{Re}(g(y_0))>1>\mathrm{Re}(g( T(x)))$. Заметим, что для любого $x\in B_E^\circ$ существует $\alpha\in\mathbb{C}$ 
такое что $|\alpha|<1$ и $|g( T(x))|=\mathrm{Re}(g(T(\alpha x)))$. Так как $|\alpha|\leq 1$, то $\alpha x\in B_E^\circ$ и $| T^*(g)(x)|=| T(g(x))|=\mathrm{Re}(g( T(\alpha x)))<1$. Так как $x\in B_E^\circ$ 
произвольно, то $\Vert T^*(g)\Vert\leq 1$. Далее $\Vert g\Vert\geq|g(y_0)|/\Vert y_0\Vert\geq c\mathrm{Re}(g(y_0))>c$, но $ T^*$ $c$-топологически инъективно. Следовательно, 
$\Vert g\Vert\leq c\Vert T^*(g)\Vert\leq c$. Противоречие, значит $c^{-1}B_F^\circ\subset \mathrm{cl}_F( 
T(B_E^\circ))$. Так как $E$ полно, то по предложению 4.4.1 \cite{HelFA} имеем $c^{-1}B_F^\circ\subset T(B_E^\circ)$. Это равносильно $c$-топологической сюръектиности оператора $T$.
\end{proof}

























\section{Секвенциальные операторные пространства}

\subsection {Матричные обозначения}

\begin{definition}\label{DefMatrNot}
Пусть $n,k\in\mathbb{N}$, тогда через $M_{n,k}$ мы будем обозначать линейное пространство комплекснозначных матриц размера $n\times k$. Если $E$ линейное пространство, то через $E^k$ мы будем обозначать 
линейное пространство столбцов высоты $k$ с элементами из $E$.
\end{definition}

Для $\alpha\in M_{n,k}$ и $x\in E^k$ через $\alpha x$ будем обозначать столбец из $E^n$ такой, что
$$
(\alpha x)_i=\sum\limits_{j=1}^n \alpha_{ij} x_j
$$
Эта  формула есть естественное обощение матричного умножения.

Пространство $M_{n,k}$, по умолчанию, наделяется операторной нормой $\Vert\cdot\Vert$, но нам также понадобится норма Гильберта-Шмидта. Пусть $\alpha\in M_{n,k}$, тогда норму Гильберта-Шмидта определим равенством
$$
\Vert\alpha\Vert_{hs}=\operatorname{trace}(|\alpha|^2)^{1/2}
$$
где $|\alpha|=(\alpha^*\alpha)^{1/2}$. Отметим, что всегда выполнены соотношения $\Vert\alpha\Vert\leq\Vert\alpha\Vert_{hs}$ и $\Vert|\alpha|\Vert_{hs}=\Vert|\alpha^*|\Vert=\Vert\alpha\Vert_{hs}$ (\cite{EROpSp}, 1.2).
Через $\operatorname{diag}_n(\lambda_1,\ldots,\lambda_n)$ будем обозначать диагональную матрицу размера $n\times n$ с числами $\lambda_1,\ldots,\lambda_n$ на главной диагонали. Мы также будем использовать обозначение  $\operatorname{diag}_n(\lambda):=\operatorname{diag}_n(\lambda,\ldots,\lambda)$. По заданным матрицам $\alpha_1\in M_{m,n_1},\ldots,\alpha_k\in M_{n,k_m}$ мы можем составить матрицу $[\alpha_1,\ldots,\alpha_k]\in M_{n,k_1+\ldots+k_m}$ выписав их все подряд слева направо.






















\subsection{Определения и примеры.} Для начала необходимо напомнить некоторые определения из \cite{LamOpFolgen}.

\begin{definition}[\cite{LamOpFolgen}, 1.1.7]\label{DefSQSpace} Пусть $E$ --- векторное пространство, и для каждого $n\in\mathbb{N}$ на векторном пространстве $E^n$ задана некоторая норма $\Vert \cdot \Vert_{\wideparen{n}}$. 
Будем говорить, что семейство $X = (E^n, (\Vert \cdot \Vert_{\wideparen{n}})_{n \in \mathbb{N}})$, задаёт на $E$ структуру \textit{секвенциального операторного} пространства, если выполнены следующие условия:

1) для всех $m, n \in \mathbb{N}$, $x \in E^{\wideparen{n}}$, $\alpha \in M_{m, n}$.
$$
\Vert \alpha x \Vert_{\wideparen{m}} \leq \Vert \alpha \Vert  \Vert x \Vert_{\wideparen{n}}
$$

2) для всех $m, n \in \mathbb{N}$, $x \in E^n$, $y \in E^m$

$$
\left\Vert \begin{pmatrix} x \\ y \end{pmatrix} \right\Vert^2_{\wideparen{n + m}} \leq   \Vert x \Vert_{\wideparen{n}}^2 + \Vert y \Vert_{\wideparen{m}}^2
$$

Пространство $E^n$ с нормой $\Vert \cdot \Vert_{\wideparen{n}}$ будем обозначать через $X^{\wideparen{n}}$.
\end{definition}

\begin{proposition}\label{PrRedundantAxiom} Пусть $X$ --- секвенциальное операторное пространство, $n\in\mathbb{N}$ и $x\in E^{\wideparen{n}}$, тогда
\newline
1) для любого $m\in\mathbb{N}$ выполнено $\Vert (x, 0)^{tr}\Vert_{\wideparen{n + m}}=\Vert x\Vert_{\wideparen{n}}$
\newline
2) для любой частчиной изометрии $s\in M_{n,n}$ выполнено $\Vert sx\Vert_{\wideparen{n}}=\Vert x\Vert_{\wideparen{n}}$. В частности, норма не меняется после перестановки координат.
\end{proposition}
\begin{proof} 1) Результат следует из неравенрств
$$
\Vert (x, 0)^{tr}\Vert_{\wideparen{n + m}}\leq \left(\Vert x\Vert_{\wideparen{n}}^2+\Vert 0\Vert_{\wideparen{m}}^2\right)^{1/2}=\Vert x\Vert_{\wideparen{n}}
$$
$$
\Vert x\Vert_{\wideparen{n}}=\Vert[\operatorname{diag}_n(1),0](x,0)^{tr}\Vert_{n}\leq\Vert[\operatorname{diag}_n(1),0]\Vert\Vert(x,0)^{tr}\Vert_{\wideparen{n+m}}=
\Vert(x,0)^{tr}\Vert_{\wideparen{n+m}}
$$
2) Так как $s$ частичная изометрия, то $s^*s=\operatorname{diag}_n(1)$, поэтому результат следует из неравенств
$$
\Vert sx\Vert_{\wideparen{n}}\leq\Vert s\Vert\Vert x\Vert_{\wideparen{n}}=\Vert x\Vert_{\wideparen{n}}=
\Vert s^*sx\Vert_{\wideparen{n}}\leq\Vert s^*\Vert\Vert sx\Vert_{\wideparen{n}}=\Vert sx\Vert_{\wideparen{n}}
$$
\end{proof}

\begin{proposition}\label{PrCHaveUniqueOSS} Гильбертово пространство $\mathbb{C}$ имеет единственную струтуру секвенциального операторного пространства задаваемую равенствами $\mathbb{C}^{\wideparen{n}}=l_2^n$.
\end{proposition}
\begin{proof} Пусть $\mathbb{C}$ наделено некоторой структурой секвенциального операторного пространства. Пусть $\xi\in\mathbb{C}^n$, тогда рассмотрим $\eta=(\Vert \xi\Vert_{l_2^n},0,\ldots,0)^{tr}\in \mathbb{C}^n$. Так как $\Vert\eta\Vert_{l_2^n}=\Vert\xi\Vert_{l_2^n}$, то существует унитарыная матрица $s\in M_{n,n}$ такая что $\eta=s\xi$. Следоватлеьно, $\Vert\eta\Vert_{\wideparen{n}}=\Vert s\xi\Vert_{\wideparen{n}}\leq\Vert s\Vert\Vert\xi\Vert_{\wideparen{n}}=\Vert\xi\Vert_{\wideparen{n}}$. По предложению \ref{PrRedundantAxiom} имеем $\Vert\eta\Vert_{\wideparen{n}}=\Vert \xi\Vert_{l_2^n}$, значит $\Vert\xi\Vert_{\wideparen{n}}\geq\Vert\xi\Vert_{l_2^n}$. С другой  стороны из второй аксиоммы секвенциальных операторных пространств следует, что $\Vert \xi\Vert_{\wideparen{n}}\leq\Vert\xi\Vert_{l_2^n}$, поэтому $\Vert \xi\Vert_{\wideparen{n}}=\Vert \xi\Vert_{l_2^n}$. Поскольку $n\in\mathbb{N}$ и $\xi\in \mathbb{C}^{\wideparen{n}}$ произвольны мы получаем $\mathbb{C}^{\wideparen{n}}=l_2^n$.
\end{proof}


\begin{proposition}\label{PrSQAxiomRed}
Пусть для семейства функций $(\Vert\cdot\Vert_{\wideparen{n}}:E^n\to\mathbb{R}_+)_{n\in\mathbb{N}}$ выполнены первая и вторая аксиомы секвенциального операторного пространства, и пусть $\Vert x\Vert_{\wideparen{1}}=0$ влечет $x=0$. 
Тогда $E$ --- секвенциальное операторное пространство.
\end{proposition}
\begin{proof}
Пусть $x\in E^n$ и $\lambda\in \mathbb{C}\setminus\{0\}$, тогда 
$$
\Vert\lambda x\Vert_{\wideparen{n}}
=\Vert\operatorname{diag}_n(\lambda)x\Vert_{\wideparen{n}}
\leq\Vert\operatorname{diag}_n(\lambda)\Vert\Vert x\Vert_{\wideparen{n}}
=|\lambda|\Vert x\Vert_{\wideparen{n}}
=|\lambda|\Vert\lambda^{-1}\lambda x\Vert_{\wideparen{n}}
\leq|\lambda||\lambda^{-1}|\Vert\lambda x\Vert_{\wideparen{n}}
=\Vert\lambda x\Vert_{\wideparen{n}}
$$
Следовательно $\Vert\lambda x\Vert_{\wideparen{n}}=|\lambda|\Vert x\Vert_{\wideparen{n}}$ при $\lambda\neq 0$. Для $\lambda=0$ равенство очевидно.
Пусть $x',x''\in E^n\setminus\{0\}$, тогда обозначим $\mu=(\Vert x'\Vert_{\wideparen{n}}^2+\Vert x''\Vert_{\wideparen{n}}^2)^{1/2}$, тогда
$$
\Vert x'+x''\Vert_{\wideparen{n}}^2
=\left\Vert\begin{pmatrix}\operatorname{diag}_n(\mu) & 0\\ 0 & \operatorname{diag}_n(\mu)\end{pmatrix}\begin{pmatrix}\mu^{-1}x'\\ \mu^{-1}x''\end{pmatrix}\right\Vert_{\wideparen{n}}^2
\leq\left\Vert\begin{pmatrix}\operatorname{diag}_n(\mu) & 0\\ 0 & \operatorname{diag}_n(\mu)\end{pmatrix}\right\Vert^2\left\Vert\begin{pmatrix}\mu^{-1}x'\\\mu^{-1}x''\end{pmatrix}\right\Vert_{\wideparen{n}}^2
$$
$$
\leq\mu^2(\mu^{-2}\Vert x'\Vert_{\wideparen{n}}^2+\mu^{-2}\Vert x''\Vert_{\wideparen{n}}^2)=\Vert x'\Vert_{\wideparen{n}}^2+\Vert x''\Vert_{\wideparen{n}}^2\leq (\Vert x'\Vert_{\wideparen{n}}+\Vert x''\Vert_{\wideparen{n}})^2
$$
Следовательно, для $x',x''\neq 0$ $\Vert x'+x''\Vert_{\wideparen{n}}\leq\Vert x'\Vert_{\wideparen{n}}+\Vert x''\Vert_{\wideparen{n''}}$. Для $x'=x''=0$ неравенство очевидно.
\end{proof}


\begin{proposition}[\cite{LamOpFolgen}, 1.1.4]\label{PrNormVsSQNorm} Пусть $X$ --- секвенциальное операторное пространство, $n\in\mathbb{N}$. Тогда для любого $x\in X^{\wideparen{n}}$ и $i\in\mathbb{N}_n$ выполнено
$$
\Vert x_i\Vert_{\wideparen{1}}\leq\Vert x\Vert_{\wideparen{n}}\leq\sum\limits_{k=1}^n\Vert x_k\Vert_{\wideparen{1}}\leq n\Vert x\Vert_{\wideparen{n}}
$$
\end{proposition}


Будем говорить, что семейство $X$ задает структуру \textit{секвенциального операторного пространства} для нормированного пространства $(E, \Vert \cdot \Vert_{\wideparen{1}}$). Легко заметить, что если $X$ --- секвенциальное операторное пространство, то каждое 
нормированное пространство $X^{\wideparen{n}}$ наделено естественной структурой секвенциального операторного пространства: достаточно отождествить $(X^{\wideparen{n}})^{\wideparen{k}}$ с $X^{\wideparen{nk}}$.

\begin{example}[\cite{LamOpFolgen}, 1.1.8]\label{ExHilSQ} Пусть $H$ --- гильбертово пространство, его максимальная структура секвенциального операторного пространства задается отождествлением $\max(H)^{\wideparen{n}}=\bigoplus{}_2\{H:\lambda\in\mathbb{N}_n\}$. Очевидно, что все пространства $\max(H)^{\wideparen{n}}$ гильбертовы. Мы будем называть эту структуру стандартной структурой гильбертового пространства $H$ и обычно будем обозначать $\max(H)$ через $H$.
\end{example}

\begin{definition}[\cite{LamOpFolgen}, 1.1.18]\label{ExT2nSQ} 
Пусть $H$ --- гильбертово пространство, его структура минимального секвенциального операторного пространства задается отождествлением $\min(H)^{\wideparen{n}} = \mathcal{B}(l_2^n, H)$. 
\end{definition}

Через $t_2^n$ будем обозначать $\min(l_2^n)$.

\begin{definition}\label{DefOpSubAlgSQ} Пусть $A$ --- подалгебра $\mathcal{B}(H)$ для некоторого гильбертова простарнства $H$, тогда мы определим её стандартную структуру секвенциального операторного пространства вложением $A^n\hookrightarrow \mathcal{B}(H,H^{\wideparen{n}})$.
\end{definition}

\begin{proposition}\label{PrCstarAlgSQ} Пусть $A$ --- $C^*$ алгебра, тогда её стандартная структура секвенциального операторного пространства не зависит от ее представления на гильбертовом прострастве и для любых $n\in\mathbb{N}$ и $a\in A^{\wideparen{n}}$ выполнено
$$
\Vert a\Vert_{\wideparen{n}}=\left\Vert\sum\limits_{i=1}^n a_i^*a_i\right\Vert^{1/2}
$$
В частности стандартные структуры секвенциального операторного пространства $\mathbb{C}$ рассмотренного как $C^*$ алгебра и как гильбертово пространство совпадают.
\end{proposition}
\begin{proof} Пусть $\pi:A\to\mathcal{B}(H)$ --- изометрическое ${}^*$-представление $A$ на гильбертовом пространстве $H$. Пусть $n\in\mathbb{N}$, тогда $a\in A^{\wideparen{n}}$ отождествляется с оператором $T:H\mapsto H^{\wideparen{n}}:\xi\mapsto \oplus{}_2\{\pi(a_i)(\xi):i\in\mathbb{N}_n\}$. Тогда 
$$
\Vert a\Vert_{\wideparen{n}}^2
=\Vert T\Vert^2
=\sup\{\Vert \oplus_2\{\pi(a_i)(\xi):i\in\mathbb{N}_n\}\Vert^2:\xi\in B_H\}=
$$
$$
=\sup\left\{ \sum\limits_{i=1}^n\langle \pi(a_i)(\xi),\pi(a_i)(\xi)\rangle:\xi\in B_H\right\}
=\sup\left\{ \left\langle \pi\left(\sum\limits_{i=1}^n a_i^*a_i\right)(\xi),\xi\right\rangle:\xi\in B_H\right\}
$$
Из предложений 2.2.4 и 2.2.5 \cite{MurphCstarOpTh} мы получаем, что $\sum_{i=1}^n a_i^* a_i\geq 0$ и $\pi(\sum_{i=1}^n a_i^* a_i)\geq 0$, поэтому из предложения 6.4.6 \cite{HelFA} получаем
$$
\Vert a\Vert_{\wideparen{n}}^2
=\sup\left\{ \left\langle \pi\left(\sum\limits_{i=1}^n a_i^*a_i\right)(\xi),\xi\right\rangle:\xi\in B_H\right\}
=\left\Vert \pi\left(\sum\limits_{i=1}^n a_i^*a_i\right)\right\Vert
=\left\Vert \sum\limits_{i=1}^n a_i^*a_i\right\Vert
$$
Если $A=\mathbb{C}$, то
$$
\Vert a\Vert_{\wideparen{n}}
=\left| \sum\limits_{i=1}^n \overline{a_i}a_i\right|^{1/2}
=\left(\sum\limits_{i=1}^n |a_i|^2\right)^{1/2}
=\Vert a\Vert_{l_2^n}
$$
значит оба определения дают одну и  ту же структуру секвенциального операторного пространства.
\end{proof}

\begin{proposition}\label{PrCommCstarSQ} Пусть $\Omega$ --- локально компактное топологическое пространство, тода для каждого $n\in\mathbb{N}$ имеет место изометрический изоморфизм
$$
i_C:C_0(\Omega)^{\wideparen{n}}\to C_0(\Omega,\mathbb{C}^n):f\mapsto (\omega\mapsto(f_i(\omega))_{i\in\mathbb{N}_n})
$$
\end{proposition} 
\begin{proof} Используя предложение \ref{PrCstarAlgSQ} для любого $f\in C_0(\Omega)^{\wideparen{n}}$ имеем
$$
\Vert f\Vert_{\wideparen{n}}
=\left\Vert \sum\limits_{i=1}^n f_i^* f_i\right\Vert^{1/2}
=\sup\left\{\left(\sum\limits_{i=1}^n |f_i(\omega)|^2\right)^{1/2}:\omega\in\Omega\right\}
=\sup\{\Vert i_C(f)(\omega)\Vert:\omega\in\Omega\}
=\Vert i_C(f)\Vert
$$
Значит $i_C$ --- изометрия. Для данного $g\in C_0(\Omega,\mathbb{C}^n)$ и каждого $i\in\mathbb{N}_n$ рассмотрим непрерывную функцию $f_i:\Omega\to\mathbb{C}:\omega\mapsto g(\omega)_i$ и определим $f=(f_1,\ldots,f_n)^{tr}\in C_0(\Omega)^{\wideparen{n}}$. Очевидно, $i_C(f)=g$, поэтому $i_C$ сюръективно. Таким образом, $i_C$ --- сюръективная изометрия, т. е. изометрический изоморфизм.
\end{proof}


































\subsection{Операторы между секвенциальными операторными пространствами}

\begin{definition}[\cite{LamOpFolgen}, 1.2.1]\label{DefSBOp}
Пусть $X$ и $Y$ ---- секвенциальные операторный проcтранства и $\varphi : X \to Y$ --- линейный оператор. Его \textit{размножением} называется семейство операторов $\varphi^{\wideparen{n}} : X^{\wideparen{n}} \to Y^{\wideparen{n}}$ 
где $n\in\mathbb{N}$, определённых равенством 
$$
\varphi^{\wideparen{n}}(x)=(\varphi(x_i))_{i\in\mathbb{N}_k}
$$
Будем называть оператор $\varphi$ \textit{секвенциально ограниченным}, если 
$$
\Vert \varphi \Vert_{sb} := \sup\{\Vert \varphi^{\wideparen{n}}\Vert_{\mathcal{B}(X^{\wideparen{n}},Y^{\wideparen{n}})}:n\in\mathbb{N}\}  < \infty
$$
\end{definition}

\begin{proposition}\label{PrSimplAmplProps}
Пусть $X$, $Y$, $Z$ --- секвенциальные операторные пространства, $\varphi:X\to Y$, $\psi:Y\to Z$ --- линейные операторы и $n,m\in\mathbb{N}$. Тогда
\newline
1) $\varphi$ инъективен (сюръективен) $\Longleftrightarrow$ $\varphi^{\wideparen{n}}$ инъективен (сюръективен).
\newline
2) имеет место неравенство $\Vert\varphi\Vert\leq\Vert\varphi\Vert_{sb}$ и, как следствие, $\mathcal{SB}(X,Y)\subset\mathcal{B}(X,Y)$.
\newline
3) имеет место равенство $(\psi\varphi)^{\wideparen{n}}=\psi^{\wideparen{n}}\varphi^{\wideparen{n}}$ и, как следствие,  $\Vert\psi\varphi\Vert_{sb}\leq\Vert\psi\Vert_{sb}\Vert\varphi\Vert_{sb}$
\newline
4) Для любых $\alpha\in M_{n,m}$, $x\in X^{\wideparen{m}}$ выполнено $\varphi^{\wideparen{n}}(\alpha x)=\alpha\varphi^{\wideparen{m}}(x)$
\end{proposition}
\begin{proof}
1)---2) Следует непосредственно из определения.
\newline
3) Для любого $x\in X^{\wideparen{n}}$ имеем
$$(\psi\varphi)^{\wideparen{n}}(x)
=((\psi\varphi)(x_i))_{i\in\mathbb{N}_n}
=((\psi(\varphi(x_i)))_{i\in\mathbb{N}_n}
=\psi^{\wideparen{n}}((\varphi(x_i))_{i\in\mathbb{N}_n})
=\psi^{\wideparen{n}}(\varphi^{\wideparen{n}}(x))
$$
откуда $(\psi\varphi)^{\wideparen{n}}=\psi^{\wideparen{n}}\varphi^{\wideparen{n}}$. Более того, 
$$
\Vert\psi\varphi\Vert_{sb}
=\sup\{\Vert\psi^{\wideparen{n}}\varphi^{\wideparen{n}}\Vert:n\in\mathbb{N}\}
\leq\sup\{\Vert\psi^{\wideparen{n}}\Vert\Vert\varphi^{\wideparen{n}}\Vert:n\in\mathbb{N}\}
\leq\Vert\psi\Vert_{sb}\Vert\varphi\Vert_{sb}
$$
4) Для любого $i\in\mathbb{N}_n$ имеем
$$
\varphi^{\wideparen{n}}(\alpha x)_i
=\varphi((\alpha x)_i)
=\varphi\left(\sum\limits_{j=1}^m \alpha_{ij }x_j\right)
=\sum\limits_{j=1}^m\alpha_{ij} \varphi(x_j)
=\sum\limits_{j=1}^m\alpha_{ij} \varphi^{\wideparen{m}}(x)_j=
(\alpha\varphi^{\wideparen{m}}(x))_i
$$
Сдедовательно $\varphi^{\wideparen{n}}(\alpha x)=\alpha\varphi^{\wideparen{m}}(x)$.
\end{proof}

\begin{definition}\label{DefSBOpType}
Пусть $\varphi:X\to Y$ --- секвенциально ограниченный оператор между секвенциальными операторными пространствами $X$, $Y$, тогда $\varphi$ называется:
\newline
1) \textit{секвенциально сжимающим}, если $\Vert \varphi\Vert_{sb}\leq 1$
\newline
2) \textit{секвенциально $c$-топологически инъективным}, если для всех $n \in \mathbb{N}$ оператор $\varphi^{\wideparen{n}}$ $c$-секвенциально топологически инъективный. Если упоминание константы $c$ не нужно то 
будем говорить, что $\varphi$ секвенциально топологически инъективный.
\newline
3) \textit{секвенциально (строго) $c$-топологически сюрьективным}, если для всех $n \in \mathbb{N}$ оператор $\varphi^{\wideparen{n}}$ строго $c$-секвенциально топологически сюрьективный. Если упоминание константы 
$c$ не нужно то будем говорить, что $\varphi$ (строго) секвенциально топологически сюръективный
\newline
4) \textit{секвенциально изометрическим}, если для каждого $n\in\mathbb{N}$ оператор $\varphi^{\wideparen{n}}$ изометричнен
\newline
5) \textit{секвенциально (строго) коизометрическим}, если каждого $n\in\mathbb{N}$ оператор $\varphi^{\wideparen{n}}$ (строго) коизометричен
\end{definition}

\begin{proposition}\label{PrComposeSQTopInjSur} Пусть $X$, $Y$, $Z$ --- секввенциальные операторные пространства и $\varphi_1\in\mathcal{SB}(X,Y)$, $\varphi_2\in\mathcal{SB}(Y,Z)$. Тогда
\newline
1) если $\varphi_i$ секвенциально $c_i$-тополоически инъективен для $i\in\mathbb{N}_2$, то $\varphi_2\varphi_1$ секвенциально $c_2c_1$-топологически инъективен.
\newline
2) если $\varphi_i$ (строго) секвенциально $c_i$-топологически сюръективен для $i\in\mathbb{N}_2$, то $\varphi_2\varphi_1$ (строго) секвенциально $c_2c_1$-топологически сюръективен.
\end{proposition}
\begin{proof}
1) Для каждого $n\in\mathbb{N}$ и $x\in X^{\wideparen{n}}$ имеем $\Vert(\varphi_2\varphi_1)^{\wideparen{n}}(x)\Vert_{\wideparen{n}}=\Vert\varphi_2^{\wideparen{n}}(\varphi_1^{\wideparen{n}}(x))\Vert_{\wideparen{n}}
\geq c_2^{-1}\Vert\varphi_1^{\wideparen{n}}(x)\Vert_{\wideparen{n}}\geq c_2^{-1}c_1^{-1}\Vert x\Vert_{\wideparen{n}}$, следовательно $\varphi_2\varphi_1$ секвенциально $c_2c_1$-топологически инъективен.

2) Допустим $\varphi_i$ секвенциально $c_i$-топологически сюръективен для $i\in\mathbb{N}_2$. Из предложения \ref{PrEquivDescOfIsomCoisomOp} для каждого $n\in\mathbb{N}$ имеем $(\varphi_2\varphi_1)^{\wideparen{n}}(B_{X^{\wideparen{n}}}^\circ)=\varphi_2^{\wideparen{n}}(\varphi_1^{\wideparen{n}}(B_{X^{\wideparen{n}}}^\circ))\supset\varphi_2^{\wideparen{n}}(c_1^{-1}B_{Y^{\wideparen{n}}}^\circ)=c_1^{-1}\varphi_2^{\wideparen{n}}(B_{Y^{\wideparen{n}}}^\circ)=c^{-1}c_2^{-1}B_{Z^{\wideparen{n}}}^\circ$. Снова из предложения \ref{PrEquivDescOfIsomCoisomOp} мы получаем, что $\varphi_2\varphi_1$ секвенциально $c_2c_1$-топологически сюръективен.
\end{proof}


\begin{proposition}[\cite{LamOpFolgen}, 1.2.14]\label{PrSmithsLemma}
Пусть $X$, $Y$ секвенциальные операторные пространства и $d=\operatorname{dim}(Y)<\infty$, тогда для любого $T\in\mathcal{B}(X,Y)$ выполнено
$$
\Vert T\Vert_{sb}=\Vert T^{\wideparen{d}}\Vert
$$
\end{proposition}

\begin{proposition}[\cite{LamOpFolgen}, 1.2.14]\label{PrSQSpaceIsSBFromT2n}
Пусть $X$ --- секвенциальное операторное пространство, $n\in\mathbb{N}$. Тогда имеет место изометрический изоморфизм 
$$
i_{t_2}:X^{\wideparen{n}}\to\mathcal{SB}(t^n_2, X):x\mapsto\left(\xi\mapsto\sum\limits_{i=1}^n\xi_ix_i\right)
$$
\end{proposition}

Пространство всех секвенциально ограниченных операторов между секвенциальными операторными пространствами $X$ и $Y$ будем обозначать $\mathcal{SB}(X, Y)$. Очевидно, это нормированное пространство; кроме того, на $\mathcal{SB}(X, Y)$ 
можно ввести структуру секвенциального операторного пространства, положив
$$
\mathcal{SB}(X, Y)^{\wideparen{n}} = \mathcal{SB}(X, Y^{\wideparen{n}})
$$
При этом каждому оператору $\varphi\in\mathcal{SB}(X,Y)^{\wideparen{n}}$ ставится в соответсвие оператор 
$$
A(\varphi):X\to Y^{\wideparen{n}}:x\mapsto(\varphi_i(x))_{i\in\mathbb{N}_n}
$$

\begin{definition}[\cite{LamOpFolgen}, 1.2.11]\label{DefSBbiOp}
Пусть $\mathcal{R}:X\times Y\to Z$ билинейный оператор между секвенциальными операторными пространствами $X$, $Y$, $Z$. Его размножением называется оператор
$$
\mathcal{R}^{\wideparen{n\times m}}:X^{\wideparen{n}}\times Y^{\wideparen{m}}\to Z^{\wideparen{nm}}:(x,y)\mapsto(\mathcal{R}(x_i,y_j))_{i\in\mathbb{N}_n,j\in\mathbb{N}_m}
$$
где $n,m\in\mathbb{N}$. Он называется секвенциально ограниченным если
$$
\Vert\mathcal{R}\Vert_{sb}:=\sup\{\Vert \mathcal{R}^{\wideparen{n\times m}}\Vert_{\mathcal{B}(X^{\wideparen{n}}\times Y^{\wideparen{m}}, Z^{\wideparen{nm}})}:n,m\in\mathbb{N}\}<\infty
$$
\end{definition}

\begin{definition}\label{DefSBBiOpType}
Пусть $\mathcal{R}:X\times Y\to Z$ ограниченный билинейный оператор между нормированными пространствами $X$, $Y$ и $Z$, тогда
\newline
1) если $Y$ и $Z$ ($X$ и $Z$) секвенциальные операторные пространства, то $\mathcal{R}$ секвенциально изометричным слева (справа) если оператор ${}^X\mathcal{R}:X\to\mathcal{SB}(Y,Z)$ ($\mathcal{R}^Y:Y\to\mathcal{SB}(X,Z)$) изометричен.
\newline
2) если $X$, $Y$, $Z$ секвенциальные операторные пространства, то $\mathcal{R}$ назыввается секвенциально сжимающим если $\Vert \mathcal{R}\Vert_{sb}\leq 1$.
\end{definition}

Для заданных секвенциальных операторных пространств $X$, $Y$, $Z$ через $\mathcal{SB}(X\times Y, Z)$ будем обозначть множество секвенциально ограниченных биоператоров из $X\times Y$ в $Z$. Очевидно, это нормированное пространство; 
кроме того, на $\mathcal{SB}(X\times Y, Z)$ можно ввести структуру секвенциального операторного пространства, положив
$$
\mathcal{SB}(X\times Y, Z)^{\wideparen{n}}=\mathcal{SB}(X\times Y,Z^{\wideparen{n}})
$$
где $n\in\mathbb{N}$. При этом каждому $\mathcal{R}\in\mathcal{SB}(X\times Y,Z)^{\wideparen{n}}$ ставится в соответсвие биоператор 
$$
A(\mathcal{R}):X\times Y\to Z^{\wideparen{n}}:(x,y)\mapsto(\mathcal{R}_i(x,y))_{i\in\mathbb{N}_n}
$$
При этом, легко видеть что для любых $x\in X^{\wideparen{n}}$, $y\in Y^{\wideparen{m}}$ и $\alpha\in M_{k,n}$  выполнено
$$
A((\mathcal{R}^Y)^{\wideparen{m}}(y))^{\wideparen{n}}(x)=A(({}^X\mathcal{R})^{\wideparen{n}}(x))^{\wideparen{m}}(y)=\mathcal{R}^{\wideparen{n\times m}}(x,y)
$$
$$
\mathcal{R}^{\wideparen{n\times k}}(x,\alpha y)
=[\alpha,\ldots,\alpha]\mathcal{R}^{\wideparen{n\times m}}(x,y)
$$


\begin{proposition}\label{PrScalMultSB}
Пусть $X$ суквенциальное операторное пространство, тогда билинейный оператор $\mathcal{M}:\mathbb{C}\times X\to X:(\alpha, x)\mapsto \alpha x$ секвенциально сжимающий.
\end{proposition}
\begin{proof}
Пусть $\alpha\in\mathbb{C}^{\wideparen{n}}$ и $x\in X^{\wideparen{m}}$. Рассмотрим матрицу $\beta=[\operatorname{diag}_m(\alpha_1),\ldots,\operatorname{diag}_m(\alpha_n)]^{tr}$, тогда легко проверить, что $\Vert\beta\Vert=\Vert\alpha\Vert_{\wideparen{n}}$. Теперь заметим, что $\Vert\mathcal{M}^{\wideparen{n\times m}}(\alpha, x)\Vert_{\wideparen{n\times m}}
=\Vert\beta x\Vert_{\wideparen{n\times m}}
\leq\Vert\alpha\Vert_{\wideparen{n}}\Vert x\Vert_{\wideparen{m}}$. Так как $m,n\in\mathbb{N}$ произовльны, то $\Vert\mathcal{M}\Vert_{sb}\leq 1$.
\end{proof}

\begin{proposition}\label{PrRestrOfSBBilOpIsSB}
Пусть $X$, $Y$ и $Z$ секвенциальные операторные пространства и $\mathcal{R}:X\times Y\to Z$ секвенциально ограниенный билинейный оператор, тогда для заданного $x\in X^{\wideparen{1}}$ ($y\in Y^{\wideparen{1}}$) линейный оператор ${}^X\mathcal{R}(x)$ ($\mathcal{R}^Y(y)$) секвенциально ограничен, причем $\Vert{}^X\mathcal{R}(x)\Vert_{sb}\leq\Vert\mathcal{R}\Vert_{sb}\Vert x\Vert_{\wideparen{1}}$ ($\Vert\mathcal{R}^Y(y)\Vert_{sb}\leq\Vert\mathcal{R}\Vert_{sb}\Vert y\Vert_{\wideparen{1}}$).
\end{proposition}
\begin{proof}
Пусть $n\in\mathbb{N}$ и $x\in X^{\wideparen{n}}$, тогда
$$
\Vert(\mathcal{R}^Y(y))^{\wideparen{n}}(x)\Vert_{\wideparen{n}}
=\Vert \mathcal{R}^{\wideparen{n\times 1}}(x,y)\Vert_{\wideparen{n\times 1}}
\leq\Vert \mathcal{R}\Vert_{sb}\Vert x\Vert_{\wideparen{n}}\Vert y\Vert_{\wideparen{1}}
$$
Следовательно $\Vert\mathcal{R}^Y(y)\Vert_{sb}\leq\Vert\mathcal{R}\Vert_{sb}\Vert y\Vert_{\wideparen{1}}$. Для другого случая доказательство аналогично. 
\end{proof}

\begin{proposition}\label{PrSQNormViaDuality}
Пусть $Z$ --- секвенциальное операторное пространство, $X$ ($Y$) --- секвенциальное операторное пространство, а $Y$ ($X$) --- нормированное пространство. Пусть $\mathcal{R}:X\times Y\to Z$ секвенциально изометричен справа (слева), тогда 
на $Y$ ($X$) существует структура секвенциального операторного пространства задаваемая системой норм
$$
\Vert y\Vert_{\wideparen{k}}^{\mathcal{R}}=\sup\{\Vert\mathcal{R}^{\wideparen{n\times k}}(x,y)\Vert_{\wideparen{n\times k}}:x\in B_{X^{\wideparen{n}}}, n\in\mathbb{N}\}
$$
$$
(\Vert x\Vert_{\wideparen{k}}^{\mathcal{R}}=\sup\{\Vert\mathcal{R}^{\wideparen{k\times n}}(x,y)\Vert_{\wideparen{k\times n}}:y\in B_{Y^{\wideparen{n}}}, n\in\mathbb{N}\})
$$
причем $\Vert \mathcal{R}\Vert_{sb}\leq 1$. Если помимо этого $d=\operatorname{dim}(Z)<\infty$, то
$$
\Vert y\Vert_{\wideparen{k}}^{\mathcal{R}}=\sup\{\Vert\mathcal{R}^{\wideparen{dk\times k}}(x,y)\Vert_{\wideparen{dk\times k}}:x\in B_{X^{\wideparen{dk}}}\}
$$
$$
(\Vert x\Vert_{\wideparen{k}}^{\mathcal{R}}=\sup\{\Vert\mathcal{R}^{\wideparen{k\times dk}}(x,y)\Vert_{\wideparen{k\times dk}}:y\in B_{Y^{\wideparen{dk}}}\})
$$
\end{proposition}
\begin{proof}
Мы рассмотрим только случай биоператора секвенциально изометричного справа, для второго случая доказательство аналогично. Пусть $y\in Y^{\wideparen{k}}$, где $k\in\mathbb{N}$. Покажем что $\Vert y\Vert_{\wideparen{k}}^{\mathcal{R}}$ 
корректно определено. Имеем
$$
\Vert\mathcal{R}^{\wideparen{n\times k}}(x,y)\Vert_{\wideparen{n\times k}}
=\left\Vert\sum\limits_{j=1}^k\mathcal{R}^{\wideparen{n\times k}}(x,(\delta_{ji}y_i)_{i\in\mathbb{N}_k})\right\Vert_{\wideparen{n\times k}}
\leq\sum\limits_{j=1}^k\Vert\mathcal{R}^{\wideparen{n\times n}}(x,(\delta_{ji}y_i)_{i\in\mathbb{N}_k})\Vert_{\wideparen{n\times k}}
$$
$$
=\sum\limits_{j=1}^k\Vert\mathcal{R}^{\wideparen{n\times 1}}(x,y_j)\Vert_{\wideparen{n\times 1}}
=\sum\limits_{j=1}^k\Vert A(\mathcal{R}^Y(y_j))^{\wideparen{n}}(x)\Vert_{\wideparen{n}}
$$
Тогда из секвенциальной изометричности справа следует
$$
\Vert y\Vert_{\wideparen{k}}^{\mathcal{R}}
\leq\sum\limits_{j=1}^k\sup\{\Vert A(\mathcal{R}^Y(y_j))^{\wideparen{n}}(x)\Vert_{\wideparen{n}}:x\in B_{X^{\wideparen{n}}}, n\in\mathbb{N}\}
=\sum\limits_{j=1}^k \Vert\mathcal{R}^Y(y_j)\Vert_{sb}
=\sum\limits_{j=1}^k \Vert y_j\Vert<+\infty
$$
Следовательно, рассматриваемая норма всюду конечна. Осталось проверить аксиомы секвенциальных операторных пространств. Пусть $\alpha\in M_{m,k}$ и $x\in X^{\wideparen{n}}$, тогда легко видеть, что
$$
\Vert\mathcal{R}^{\wideparen{n\times m}}(x,\alpha y)\Vert_{\wideparen{n\times m}}
=\Vert[\alpha,\ldots,\alpha]\mathcal{R}^{\wideparen{n\times k}}(x,y)\Vert_{\wideparen{n\times k}}
$$
$$
\leq\Vert[\alpha,\ldots,\alpha]\Vert_{M_{m,nk}}\Vert\mathcal{R}^{\wideparen{n\times k}}(x,y)\Vert_{\wideparen{n\times k}}
=\Vert\alpha\Vert\Vert\mathcal{R}^{\wideparen{n\times k}}(x,y)\Vert_{\wideparen{n\times k}}
$$
Откуда
$$
\Vert\alpha y\Vert_{\wideparen{m}}^{\mathcal{R}}
=\sup\{\Vert\mathcal{R}^{\wideparen{n\times m}}(x,\alpha y)\Vert_{\wideparen{n\times m}}:x\in B_{X^{\wideparen{n}}},n\in\mathbb{N}\}
\leq\sup\{\Vert\alpha\Vert\Vert\mathcal{R}^{\wideparen{n\times k}}(x,y)\Vert_{\wideparen{n\times k}}:x\in B_{X^{\wideparen{n}}},n\in\mathbb{N}\}
$$
$$
\leq\Vert\alpha\Vert\sup\{\Vert\mathcal{R}^{\wideparen{n\times k}}(x,y)\Vert_{\wideparen{n\times k}}:x\in B_{X^{\wideparen{n}}},n\in\mathbb{N}\}
=\Vert\alpha\Vert\Vert y\Vert_{\wideparen{k}}^{\mathcal{R}}
$$
Пусть $0<l<k$ и $y=(y', y'')^{tr}$, причем $y'\in Y^{\wideparen{l}}$, $y''\in Y^{\wideparen{k-l}}$, тогда
$$
\Vert \mathcal{R}^{\wideparen{n\times k}}(x,y)\Vert_{\wideparen{n\times k}}^2
=\left\Vert\begin{pmatrix}\mathcal{R}^{\wideparen{n\times l}}(x,y')\\ \mathcal{R}^{\wideparen{n\times (k-l)}}(x,y'')\end{pmatrix}\right\Vert_{n\times k}^2
\leq\Vert\mathcal{R}^{\wideparen{n\times l}}(x,y')\Vert_{\wideparen{n\times l}}^2
+\Vert\mathcal{R}^{\wideparen{n\times (k-l)}}(x,y'')\Vert_{\wideparen{n\times (k-l)}}^2
$$
Следовательно,
$$
\Vert y\Vert_{\wideparen{k}}^{\mathcal{R}}{}^2
\leq\sup\{\Vert\mathcal{R}^{\wideparen{n\times l}}(x,y')\Vert_{\wideparen{n\times l}}^2:x\in B_X^{\wideparen{n}},n\in\mathbb{N}\}
+\sup\{\Vert\mathcal{R}^{\wideparen{n\times (k-l)}}(x,y'')\Vert_{\wideparen{n\times (k-l)}}^2:x\in B_{X^{\wideparen{n}}},n\in\mathbb{N}\}
$$
$$
=\Vert y'\Vert_{\wideparen{l}}^{\mathcal{R}}{}^2
+\Vert y''\Vert_{\wideparen{k-l}}^{\mathcal{R}}{}^2
$$
Наконец, покажем что это семейство норм задает на $Y$ структуру секвенциального операторного пространства. Действительно, для любого $y\in Y^{\wideparen{1}}$ имеем
$$
\Vert y\Vert_{\wideparen{1}}^{\mathcal{R}}
=\sup\{\Vert\mathcal{R}^{\wideparen{n\times 1}}(x,y)\Vert_{\wideparen{n\times 1}}:x\in B_{X^{\wideparen{n}}},n\in\mathbb{N}\}
=\sup\{\Vert A(\mathcal{R}^Y(y))^{\wideparen{n}}(x)\Vert_{\wideparen{n}}:x\in B_{X^{\wideparen{n}}},n\in\mathbb{N}\}
$$
$$
=\Vert \mathcal{R}^Y(y)\Vert_{sb}=\Vert y\Vert
$$
Теперь из предложения \ref{PrSQAxiomRed} следует, что рассматриваемые функции задают на $Y$ структуру секвенциального операторного пространства. Из определения нормы $Y^{\wideparen{k}}$ следует что $\Vert\mathcal{R}^{\wideparen{n\times k}}\Vert\leq 1$ для 
всех $n\in\mathbb{N}$. Так как $n,k\in\mathbb{N}$ произвольны, то $\Vert\mathcal{R}\Vert_{sb}\leq 1$.

Если $Z$ конечномерно, то из предложения \ref{PrSmithsLemma} следует, что
$$
\Vert y\Vert_{\wideparen{k}}^{\mathcal{R}}
=\sup\{\Vert\mathcal{R}^{\wideparen{n\times k}}(x,y)\Vert_{\wideparen{n\times k}}:x\in B_{X^{\wideparen{n}}},n\in\mathbb{N}\}
=\sup\{\Vert A((\mathcal{R}^Y){}^{\wideparen{k}}(y))^{\wideparen{n}}(x)\Vert_{\wideparen{n\times k}}:x\in B_{X^{\wideparen{n}}},n\in\mathbb{N}\}
$$
$$
=\Vert A((\mathcal{R}^Y){}^{\wideparen{k}}(y))\Vert_{sb}
=\Vert A((\mathcal{R}^Y){}^{\wideparen{k}}(y))^{\wideparen{dk}}\Vert
=\sup\{\Vert A((\mathcal{R}^Y){}^{\wideparen{k}}(y))^{\wideparen{dk}}(x)\Vert_{\wideparen{k\times dk}}:x\in B_{X^{\wideparen{d}}}\}
$$
$$
=\sup\{\Vert\mathcal{R}^{\wideparen{dk\times k}}(x,y)\Vert_{\wideparen{dk\times k}}:x\in B_{X^{\wideparen{dk}}}\}
$$
\end{proof}

\begin{proposition}\label{PrFreezIsomSQIsom}
Пусть $Z$ --- секвенциальное операторное пространство, $X$ ($Y$) --- секвенциальное операторное пространство, а $Y$ ($X$) --- нормированное пространство. Пусть $\mathcal{R}:X\times Y\to Z$ секвенциально изометричен справа (слева). Наделим $Y$ ($X$) структурой секвенциального пространства, как это было сделано в предложении \ref{PrSQNormViaDuality}, тогда линейный 
оператор $\mathcal{R}^Y$ (${}^X\mathcal{R}$) секвенциально изометричен.
\end{proposition}
\begin{proof}
Мы рассмотрим только случай биоператора секвенциально изометричного справа, для второго случая доказательство аналогично. Пусть $k\in\mathbb{N}$, для любого $y\in Y^{\wideparen{k}}$ имеем
$$
\Vert y\Vert_{\wideparen{k}}
=\sup\{\Vert\mathcal{R}^{\wideparen{n\times k}}(x,y)\Vert_{\wideparen{n\times k}}:x\in B_{X^{\wideparen{n}}}, n\in\mathbb{N}\}
=\sup\{\Vert A((\mathcal{R}^Y)^{\wideparen{k}}(y))^{\wideparen{n}}(x)\Vert_{\wideparen{n\times k}}:x\in B_{X^{\wideparen{n}}}, n\in\mathbb{N}\}
$$
$$
=\Vert A((\mathcal{R}^Y)^{\wideparen{k}}(y))\Vert_{sb}
=\Vert (\mathcal{R}^Y)^{\wideparen{k}}(y)\Vert_{\wideparen{k}}
$$
Следовательно, $\mathcal{R}^Y$ секвенциально изометричен
\end{proof}

Если выполнены условия предыдущего предложения, то будем говорить что биоператор $\mathcal{R}$ индуцирует структуру секвенциального операторного пространства $Y$ ($X$).

\begin{proposition}\label{PrSQOpSqQuanIsEquivToStandard}
Пусть $X$, $Y$ секвенциальные операторные пространства, стандартная структура секвенциального операторного пространства на $\mathcal{SB}(X,Y)$ совпадает индуцированой биоператором
$$
\mathcal{E}:X\times\mathcal{SB}(X,Y)\to Y:(x,\varphi)\mapsto\varphi(x)
$$
\end{proposition}
\begin{proof}
В данном случае утверждение что $\mathcal{D}$ секвенциально изометричен слева тавтологично. Поэтому $\mathcal{E}$ индуцирует структуру секвенциального операторного пространства на $\mathcal{SB}(X,Y)$. Пусть $k\in\mathbb{N}$ и $\varphi\in\mathcal{SB}(X,Y)^{\wideparen{k}}$. 
Очевидно, что $\mathcal{\mathcal{E}}^{\mathcal{SB}(X,Y)}=1_{\mathcal{SB}(X,Y)}$, поэтому
$$
\Vert\varphi\Vert_{\wideparen{k}}^{\mathcal{E}}
=\sup\{\Vert\mathcal{E}^{\wideparen{n\times k}}(x,\varphi)\Vert_{\wideparen{n\times k}}:x\in B_{X^{\wideparen{n}}}, n\in\mathbb{N}\}
=\sup\{\Vert A((\mathcal{E}^{\mathcal{SB}(X,Y)})^{\wideparen{k}}(\varphi))^{\wideparen{n}}(x)\Vert_{\wideparen{n\times k}}:x\in B_{X^{\wideparen{n}}}, n\in\mathbb{N}\}
$$
$$
=\sup\{\Vert A((1_{\mathcal{SB}(X,Y)})^{\wideparen{k}}(\varphi))^{\wideparen{n}}\Vert: n\in\mathbb{N}\}
=\sup\{\Vert A(\varphi)^{\wideparen{n}}\Vert: n\in\mathbb{N}\}
=\Vert A(\varphi)\Vert_{sb}=\Vert\varphi\Vert_{\wideparen{k}}
$$
\end{proof}

























\subsection{Пополнение секвенциальных операторных пространств}

\begin{definition}\label{DefSQBanSpace}
Секвенциальное операторное пространство $X$ называется \textit{банаховым секвенциальным операторным пространством}, если $X^{\wideparen{1}}$ банахово пространство.
\end{definition}

\begin{proposition}\label{PrSQSpaceComplSoAmplISCompl}
Пусть $X$ --- секвенциальное операторное пространство, $n\in\mathbb{N}$. В этом случае $X^{\wideparen{1}}$ банахово тогда и только тогда когда $X^{\wideparen{n}}$ банахово.
\end{proposition}
\begin{proof}. Допустим, что $X^{\wideparen{1}}$ полно. Пусть $(x^{(k)})_{k\in\mathbb{N}}$ фундаментальная последовательность в $X^{\wideparen{n}}$. Фиксируем $\varepsilon>0$, тогда существует $N\in\mathbb{N}$ такое, что $k,m> N$ 
влечет $\Vert x^{(k)}-x^{(m)}\Vert_{X^{\wideparen{n}}}<\varepsilon$. Из предложения \ref{PrNormVsSQNorm} следует, что $\Vert x_i^{(k)}-x_i^{(m)}\Vert_{\wideparen{n}}<\varepsilon$ для $i\in\mathbb{N}_n$. Значит последовательности 
$(x_i^{(k)})_{k\in\mathbb{N}}$ где $i\in\mathbb{N}_n$ фундаментальны. Так как $X^{\wideparen{1}}$ полно, то существуют пределы $x_i=\lim\limits_{k\to\infty}x_i^{(k)}$. Рассмотрим столбец $x=(x_i)_{i\in\mathbb{N}_n}\in  X^{\wideparen{n}}$. 
Снова, из предложения \ref{PrNormVsSQNorm} следует, что
$$
\lim_{k\to\infty}\Vert x^{(k)}-x\Vert_{\wideparen{n}}\leq\sum\limits_{i=1}^n\lim\limits_{k\to\infty}\Vert x_i^{(k)}-x_i\Vert_{\wideparen{1}}=0
$$
Итак, произвольная фундаментальная последовательность $(x^{(k)})_{k\in\mathbb{N}}\subset X^{\wideparen{n}}$ сходится, значит $X^{\wideparen{n}}$ банахово. Обратно, пусть $X^{\wideparen{n}}$ банахово. Пусть $(x^{(k)})_{k\in\mathbb{N}}$ 
фундаментальная последовательность в $X^{\wideparen{1}}$. Фиксируем $\varepsilon>0$, тогда существует $N\in\mathbb{N}$, такое что $k,m> N$ влечет $\Vert x^{(k)}-x^{(m)}\Vert_{\wideparen{1}}<\varepsilon$.  Рассмотрим последовательность 
$(\tilde{x}^{(k)})_{k\in\mathbb{N}}$ в $X^{\wideparen{n}}$ такую что $\tilde{x}_i^{(k)}=x^{(k)}\delta_{1,i}$ для $i\in\mathbb{N}_n$. Тогда из первой аксиомы секвенциальных операторных пространств 
$\Vert \tilde{x}^{(k)}-\tilde{x}^{(m)}\Vert_{\wideparen{n}}=\Vert x^{(k)}-x^{(m)}\Vert_{\wideparen{1}}<\varepsilon$. Так как $X^{\wideparen{n}}$ полно, то существует предел $\tilde{x}\in X^{\wideparen{n}}$. Из предложения 
\ref{PrNormVsSQNorm} следует, что
$$
\lim\limits_{k\to\infty}\Vert x^{(k)}-\tilde{x}_1\Vert_{\wideparen{1}}
=\lim\limits_{k\to\infty}\Vert (\tilde{x}^{(k)}-\tilde{x})_1\Vert_{\wideparen{1}}
\leq\lim\limits_{k\to\infty}\Vert \tilde{x}^{(k)}-\tilde{x}\Vert_{\wideparen{n}}=0
$$
Итак произвольная фундаментальная последовательность $(x^{(k)})_{k\in\mathbb{N}}\subset X^{\wideparen{1}}$ сходится, значит $X^{\wideparen{1}}$ банахово. 
\end{proof}

\begin{theorem}\label{ThSQCompl}
Пусть $X$ --- секвенциальное операторное пространство, а $\overline{X}$ --- пополнение нормированного пространства $X^{\wideparen{1}}$, и $j_X:X\to \overline{X}$ --- изометрическое вложение с плотным образом. Тогда на $\overline{X}$ 
существует структура секвенциального операторного пространства такая, что оператор $j_X$ секвенциально изометричен. 
\end{theorem}
\begin{proof} Пусть $n\in\mathbb{N}$, $\overline{x}\in \overline{X}^{n}$. Тогда для каждого $i\in\mathbb{N}_n$ существует последовательность $(x_i^{(k)})_{k\in\mathbb{N}}$ такая что $\overline{x}_i=\lim\limits_{k\to\infty}j_X(x_i^{(k)})$. 
В частности, последовательности $(x_i^{(k)})_{k\in\mathbb{N}}$ фундаментальны в $X^{\wideparen{1}}$. Для каждого $k\in\mathbb{N}$ рассмотрим вектор $x^{(k)}=(x_i^{(k)})_{i\in\mathbb{N}_n}\in X^{\wideparen{n}}$. Положим по определению
$$
\Vert\overline{x}\Vert_{\wideparen{n}}=\lim\limits_{k\to\infty}\Vert x^{(k)}\Vert_{\wideparen{n}}
$$
Покажем, что определение корректно. Фиксируем $\varepsilon>0$, тогда т.к. последовательности $(x_i^{(k)})_{k\in\mathbb{N}}$ фундаментальны, то существуют $N_i\in\mathbb{N}$ где $i\in\mathbb{N}_n$ такие, что $k,m>N_i$ влечет 
$\Vert x_i^{(k)}-x_i^{(m)}\Vert_{\wideparen{1}}<\varepsilon$. Рассмотрим $N=\max\limits_{i\in\mathbb{N}_n}N_i$, тогда из предложения \ref{PrNormVsSQNorm} для $k,m>N$ получим
$$
\left|\Vert x^{(k)}\Vert_{\wideparen{n}}-\Vert x^{(m)}\Vert_{\wideparen{n}}\right|\leq\Vert x^{(k)}-x^{(m)}\Vert\leq\sum\limits_{i=1}^n\Vert x_i^{(k)}-x_i^{(m)}\Vert_{\wideparen{1}}<n\varepsilon
$$
Следовательно последовательность $(\Vert x^{(k)}\Vert)_{k\in\mathbb{N}}$ фундаментальна и предел в определении $\Vert \overline{x}\Vert_{\wideparen{n}}$ существует. Покажем что он не зависит от выбора последовательности. 
Пусть $(x''^{(k)})_{k\in\mathbb{N}}$, $(x'^{(k)})_{k\in\mathbb{N}}$ две такие последовательности в $X^{\wideparen{n}}$, что $\overline{x}_i=\lim\limits_{k\to\infty} j_X(x_i'^{(k)})=\lim\limits_{k\to\infty} j_X(x_i'^{(k)})$ для 
всех $i\in\mathbb{N}_n$. Тогда из предложения \ref{PrNormVsSQNorm}, следует, что 
$$
\left|\lim\limits_{k\to\infty}\Vert x''^{(k)}\Vert_{\wideparen{n}}-\lim\limits_{k\to\infty}\Vert x''^{(k)}\Vert_{\wideparen{n}}\right|\leq
\lim\limits_{k\to\infty}\Vert x''^{(k)}-x'^{(k)}\Vert_{\wideparen{n}}\leq
\sum\limits_{i=1}^n\lim\limits_{k\to\infty}\Vert x_i''^{(k)}-x_i'^{(k)}\Vert_{\wideparen{1}}\\
=\sum\limits_{i=1}^n0=0
$$
Следовательно пределы равны и $\Vert \overline{x}\Vert_{\wideparen{n}}$ определен корректно. Пусть $\overline{x}'\in X^{\wideparen{n}}$, $\overline{x}''\in X^{\wideparen{m}}$ и $\alpha\in M_{l,n}$, тогда
$$
\Vert\alpha\overline{x}'\Vert_{\wideparen{l}}
=\lim\limits_{k\to\infty}\Vert\alpha x'^{(k)}\Vert_{\wideparen{l}}
\leq\Vert\alpha\Vert\lim\limits_{k\to\infty}\Vert x'^{(k)}\Vert_{\wideparen{n}}
=\Vert\alpha\Vert\Vert\overline{x}'\Vert_{\wideparen{n}}
$$
$$
\left\Vert\begin{pmatrix} \overline{x}'\\ \overline{x}''\end{pmatrix}\right\Vert_{\wideparen{n+m}}^2
=\lim\limits_{k\to\infty}\left\Vert\begin{pmatrix} x'^{(k)}\\ x''^{(k)}\end{pmatrix}\right\Vert_{\wideparen{n+m}}^2
\leq\lim\limits_{k\to\infty}(\Vert x'^{(k)}\Vert_{\wideparen{n}}^2+\Vert x''^{(k)}\Vert_{\wideparen{m}}^2)
=\Vert\overline{x}'\Vert_{\wideparen{n}}^2+\Vert\overline{x}''\Vert_{\wideparen{m}}^2
$$
Из предложения \ref{PrSQAxiomRed} следует, что рассматриваемые функции действительно задают на $\overline{X}$ структуру секвенциального операторного пространства. Для любого $x\in X^{\wideparen{n}}$ рассмотрим стационарную последовательность 
$(j_X^{\wideparen{n}}(x))_{k\in\mathbb{N}}$, тогда
$$
\Vert j_X^{\wideparen{n}}(x)\Vert_{\wideparen{n}}
=\lim\limits_{k\to\infty}\Vert x^{(k)}\Vert_{\wideparen{n}}
=\Vert x\Vert_{\wideparen{n}}
$$
Следовательно $j_X$ секвенциально изометричен.
\end{proof}

\begin{proposition}\label{PrExtLinOpByCont} Пусть $X$ и $Y$ секвенциальные операторные пространства и $\varphi\in\mathcal{SB}(X,Y)$. Тогда существует единственный линейный оператор $\overline{\varphi}\in\mathcal{SB}(\overline{X},\overline{Y})$ продолжающий $\varphi$, при этом $\Vert \overline{\varphi}\Vert_{sb}=\Vert \varphi\Vert_{sb}$
\end{proposition}
\begin{proof}
Известно, что существует единственный линейный оператор $\overline{\varphi}\in\mathcal{B}(\overline{X},\overline{Y})$ продолжающий $\varphi$. Для заданного $x\in X^{\wideparen{n}}$ рассмотрим любую последоваательность $(x^{(k)})_{k\in\mathbb{N}}\subset X^{\wideparen{n}}$ такую, что $\overline{x}=\lim\limits_{k\to\infty} j_X(x^{(k)})$. Тогда
$$
\Vert\overline{\varphi}^{\wideparen{n}}(\overline{x})\Vert_{\wideparen{n}}
=\lim\limits_{k\to\infty}\Vert \varphi^{\wideparen{n}}(x^{(k)})\Vert_{\wideparen{n}}
\leq\Vert \varphi\Vert_{sb}\lim\limits_{k\to\infty}\Vert x^{(k)}\Vert_{\wideparen{n}}
=\Vert \varphi\Vert_{sb}\Vert \overline{x}\Vert_{\wideparen{n}}
$$
Аналогично, для любого $x\in X^{\wideparen{n}}$ имеем
$$
\Vert \varphi^{\wideparen{n}}(x)\Vert_{\wideparen{n}}
=\Vert\overline{\varphi}^{\wideparen{n}}(j_X^{\wideparen{n}}(x))\Vert_{\wideparen{n}}
=\Vert\overline{\varphi}\Vert_{sb}\Vert j_X^{\wideparen{n}}(x))\Vert_{\wideparen{n}}
=\Vert\overline{\varphi}\Vert_{sb}\Vert x\Vert_{\wideparen{n}}
$$
Так как $n\in\mathbb{N}$ произвольно, то $\Vert \overline{\varphi}\Vert_{sb}=\Vert \varphi\Vert_{sb}$ и в частности $\overline{\varphi}\in\mathcal{SB}(\overline{X},\overline{Y})$
\end{proof}


\begin{proposition}\label{PrExtBilOpByCont} Пусть $X$, $Y$ и $Z$ секвенциальные операторные пространства и $\mathcal{R}\in\mathcal{SB}(X\times Y,Z)$. Тогда существует единственный билинейный оператор $\overline{\mathcal{R}}\in\mathcal{SB}(\overline{X}\times\overline{Y},\overline{Z})$ продолжающий $\mathcal{R}$, при этом $\Vert\overline{\mathcal{R}}\Vert_{sb}=\Vert\mathcal{R}\Vert_{sb}$.
\end{proposition}
\begin{proof} Из предложения \cite{DefFloTensNorOpId} мы знаем, что существует единственный билинейный оператор $\overline{\mathcal{R}}\in\mathcal{B}(\overline{X}\times\overline{Y},\overline{Z})$ продолжающий $\mathcal{R}$. Для заданных $\overline{x}\in \overline{X}^{\wideparen{n}}$, $\overline{y}\in \overline{Y}^{\wideparen{m}}$ рассмотрим любые последовательности $(x^{(k)})_{k\in\mathbb{N}}\subset X^{\wideparen{n}}$ и $(y^{(k)})_{k\in\mathbb{N}}\subset Y^{\wideparen{m}}$ такие что $\overline{x}=\lim\limits_{k\to\infty} j_X^{\wideparen{n}}(x^{(k)})$ и $\overline{y}=\lim\limits_{k\to\infty} j_Y^{\wideparen{m}}(y^{(k)})$. Тогда $\overline{\mathcal{R}}^{\wideparen{n\times m}}(\overline{x},\overline{y})=\lim\limits_{k\to\infty}\mathcal{R}^{\wideparen{n\times m}}(x^{(k)}, y^{(k)})$ и
$$
\Vert\overline{\mathcal{R}}^{\wideparen{n\times m}}(\overline{x},\overline{y})\Vert_{\wideparen{n\times m}}
=\lim\limits_{k\to\infty}\Vert \mathcal{R}^{\wideparen{n\times m}}(x^{(k)}, y^{(k)})\Vert_{\wideparen{n\times m}}
\leq\Vert\mathcal{R}\Vert_{sb}\lim\limits_{k\to\infty}\Vert x^{(k)}\Vert_{\wideparen{n}}\Vert y^{(k)}\Vert_{\wideparen{m}}
=\Vert\mathcal{R}\Vert_{sb}\Vert\overline{x}\Vert_{\wideparen{n}}\Vert \overline{y}\Vert_{\wideparen{m}}
$$
Аналогично, для любого $x\in X^{\wideparen{n}}$, $y\in Y^{\wideparen{m}}$ мы имеем
$$
\Vert\mathcal{R}^{\wideparen{n\times m}}(x,y)\Vert_{\wideparen{n\times m}}
=\Vert\overline{\mathcal{R}}^{\wideparen{n\times m}}(j_X^{\wideparen{n}}(x),j_Y^{\wideparen{m}}(y))\Vert_{\wideparen{n\times m}}
\leq\Vert\overline{\mathcal{R}}\Vert_{sb}\Vert j_X^{\wideparen{n}}(x)\Vert_{\wideparen{n}}\Vert j_Y^{\wideparen{m}}(y)\Vert_{\wideparen{m}}
=\Vert\overline{\mathcal{R}}\Vert_{sb}\Vert x\Vert_{\wideparen{n}}\Vert y\Vert_{\wideparen{m}}
$$
Так как $n,m\in\mathbb{N}$ произвольны, то $\Vert\overline{\mathcal{R}}\Vert_{sb}=\Vert\mathcal{R}\Vert_{sb}$ and in particular $\overline{\mathcal{R}}\in\mathcal{SB}(\overline{X}\times\overline{Y},\overline{Z})$
\end{proof}

Теперь мы можем расширить список наших основных категории до $SQBan$ и $SQBan_1$. Их определения аналогичны определениям $SQNor$ и $SQNor_1$.












































\subsection{Двойственнность для секвенциальных операторных пространств}

\begin{definition}[\cite{LamOpFolgen}, 1.3.8]\label{DeffSQDual} 
Пусть $X$ --- секвенциальное операторное пространство, тогда его сопряженным секвенциальным операторным пространством называется: $X^\triangle := \mathcal{SB}(X, \mathbb{C})$, причем на $\mathbb{C}$ берётся стандартная структура секвенциального операторного пространства из примера \ref{ExHilSQ}. 
\end{definition}

\begin{proposition}[\cite{LamOpFolgen}, 1.3.9]\label{PrEveryLinFuncIsSQBounded}
Пусть $X$ -- секвенциальное операторное пространство, и $f\in X^\triangle$. Тогда для любого $n\in\mathbb{N}$ выполнено $\Vert f^{\wideparen{n}}\Vert=\Vert f\Vert$, и как следствие $\Vert f\Vert_{sb}=\Vert f\Vert$.
\end{proposition}

\begin{proposition}[\cite{LamOpFolgen}, 1.3.9]\label{PrSQNormsViaDuality}
Пусть $X$ --- секвенциальное операторное пространство, тогда $\mathcal{D}_{X,X^*}$ секвенциально изометрично слева и справа. Более того для всех $n\in\mathbb{N}$, $x\in X^{\wideparen{n}}$ и $f\in (X^\triangle)^{\wideparen{n}}$ выполнено
$$
\Vert x\Vert_{\wideparen{n}}
=\Vert x\Vert_{\wideparen{n}}^{\mathcal{D}_{X^*,X}}
\qquad\qquad
\Vert f\Vert_{\wideparen{n}}
=\Vert f\Vert_{\wideparen{n}}^{\mathcal{D}_{X,X^*}}
$$
Как следствие мы получаем, что естественное вложение во второе сопряженное пространство
$$
\iota_X:X\to X^{\triangle\triangle}
$$
секвенциально изомерично.
\end{proposition}
\begin{proof}
Так как стандартная скалярная двойственность изометрична слева и справа то из предложения \ref{PrEveryLinFuncIsSQBounded} мы получаем, что она также секвенциально изометрична слева и справа. Из предложения 1.3.12 \cite{LamOpFolgen} мы знаем, что
$$
\Vert x\Vert_{\wideparen{n}}
=\sup\{\Vert A(f)^{\wideparen{n}}(x)\Vert_{\wideparen{n\times n}}: f\in B_{(X^\triangle)^{\wideparen{n}}}\}
\qquad
\Vert f\Vert_{\wideparen{n}}
=\sup\{\Vert A(f)^{\wideparen{n}}(x)\Vert_{\wideparen{n\times n}}:x\in B_{X^{\wideparen{n}}}\}
$$
Теперь желаемые равенства следуют из тождества $\mathcal{D}_{X,X^*}^{\wideparen{n\times n}}(x,f)=A((\mathcal{D}_{X,X^*}^{X^*})^{\wideparen{n}}(f))^{\wideparen{n}}(x)
=A(f)^{\wideparen{n}}(x)$. Таким образом мы видим что исходная структура секвенциального операторного пространства в $X$ и $X^\triangle$ совпадает со структурой индуцированной билинейнымии операторами $\mathcal{D}_{X^*,X}$ и $\mathcal{D}_{X,X^*}$. Следовательно, используя тот факт, что стандартная скаляраня двойственность секвенциально изометрична справа, мы можем применить предложение \ref{PrFreezIsomSQIsom} и заключить, что оператор $\mathcal{D}_{X^*,X}^X$ секвенциально изометричен. Осталось заметить, что $\iota_X=\mathcal{D}_{X^*,X}^X$. 
\end{proof}

\begin{remark}\label{RemSqReflexiv} Будем говорить, что $X$ секвенциально рефлексивно, если $\iota_X$ --- секвенциально изометрический изоморфизм. По предложению \ref{PrSQNormsViaDuality} оператор $\iota_X$ всегда секвенциально изометричен, поэтому $\iota_X$ является секвенциально изометрическим изоморфизмом тогда и только тогда когда он сюръекивен, что эквивалентно обычной рефлексивности.
\end{remark}

\begin{proposition}\label{PrFreezDualityGetSQIsom}
Пусть $X$ ($Y$) --- секвенциальное операторное пространство, $Y$ ($X$) --- нормированное пространство. Пусть имеется скалярная двойственность $\mathcal{D}:X\times Y\to\mathbb{C}$ такая что $\mathcal{D}^F$ (${}^X\mathcal{D}$) 
изометрический изоморфизм, тогда если на $Y$ ($X$) рассмотреть индуцированную структуру секвенциального операторного пространства, то $\mathcal{D}^Y$ (${}^X\mathcal{D}$) будет секвенциально изометрическим изоморфизмом.
\end{proposition}
\begin{proof}
Мы рассмотрим только случай когда $\mathcal{D}^Y$ является изометрическим изоморфизмом, для второго случая доказательство аналогично. Пусть $n\in\mathbb{N}$. По предложению \ref{PrEveryLinFuncIsSQBounded} билинейный оператор $\mathcal{D}$ секвенциально изометричен справа. Тогда по предложению \ref{PrFreezIsomSQIsom} оператор $\mathcal{D}^Y$ секвенцильно изометричен, но он также и биективен т.к. биективен оператор 
$\mathcal{D}^Y$. Значит $\mathcal{D}^Y$ секвенциально изометрический изоморфизм. 
\end{proof}





























\subsection{Двойственность для секвенциально ограниченных операторов}

\begin{proposition}[\cite{LamOpFolgen}, 1.3.14]\label{PrDualSBOp}
Пусть $X$, $Y$ --- секвенциальные операторные пространства и $\varphi\in \mathcal{SB}(X,Y)$. Тогда $\varphi^\triangle \in\mathcal{SB}(Y^\triangle ,X^\triangle )$ и для любого $n\in\mathbb{N}$ выполнено 
$\Vert(\varphi^\triangle )^{\wideparen{n}}\Vert=\Vert\varphi^{\wideparen{n}}\Vert$. Как следствие, $\Vert\varphi^\triangle \Vert_{sb}=\Vert\varphi\Vert_{sb}$.
\end{proposition}

\begin{corollary}\label{CorDualFunc}
Из предложения \ref{PrDualSBOp} следует, что корректно определены 4 версии функтора ${}^\triangle$. Они имеют вид ${}^\triangle:\mathcal{K}\to\mathcal{K}$, где $\mathcal{K}\in\{SQNor,SQNor_1,SQBan,SQBan_1\}$. 
\end{corollary}

Далее мы докажем несколько технических предложений необходимых для описания двойственности между секвенциально ограниченными операторами.

\begin{definition}[\cite{LamOpFolgen}, 1.3.15]\label{DefT2n}
Пусть $X$ секвенциальное операторное пространство и $n\in\mathbb{N}$, тогда через $t_2^n(X)$, будем обозначть нормированное пространство $X^n$ c нормой
$$
\Vert x\Vert_{t_2^n(X)}:=\inf\left\{\Vert\tilde{\alpha}\Vert_{hs}\Vert \tilde{x}\Vert_{\wideparen{k}}:x=\tilde{\alpha} \tilde{x}\right\}
$$
где $\tilde{\alpha}\in M_{n,k}$, $x\in X^k$ и $k\in\mathbb{N}$. Если $Y$ --- секвенциальное операторное пространство, и $\varphi\in\mathcal{SB}(X,Y)$, то через $t_2^n(\varphi)$ будем обозначать линейный оператор
$$
t_2^n(\varphi): t_2^n(X)\to t_2^n(Y): x\mapsto \varphi^{\wideparen{n}}(x)
$$
\end{definition}

\begin{proposition}\label{PrT2nNormProperty}
Пусть $X$ --- секвенциальное операторное пространство и $n\in\mathbb{N}$, тогда
$$
\Vert x\Vert_{t_2^n(X)}=\inf\left\{\Vert\alpha'\Vert_{hs}\Vert x'\Vert_{\wideparen{k}}:x=\alpha'x'\right\}
$$
где $\alpha'\in M_{n,n}$ --- обратимая матрица, $x'\in X^{n}$.
\end{proposition}
\begin{proof}
Обозначим правую часть доказываемого равенства через $\Vert x\Vert_{t_2^n(X)}'$. Фиксируем $\varepsilon>0$, тогда существуют $\tilde{\alpha}\in M_{n,k}$ и $\tilde{x}\in X^{k}$, $k\in\mathbb{N}$ такие, что 
$x=\tilde{\alpha}\tilde{x}$ и $\Vert\tilde{\alpha}\Vert_{hs}\Vert\tilde{x}\Vert_{\wideparen{k}}<\Vert x\Vert_{t_2^n(X)}+\varepsilon$. Рассмотрим полярное разложение 
$\tilde{\alpha}=|\tilde{\alpha}^*| \rho$ матрицы $\tilde{\alpha}$. Пусть $p$ --- ортогональный проектор на $\operatorname{Im}(|\tilde{\alpha}^*|)^\perp$. Тогда для любого $\delta\in\mathbb{R}$ матрица 
$\alpha_\delta'=|\tilde{\alpha}^*|+\delta p$ обратима так как $\operatorname{Ker}(\alpha_\delta')=\{0\}$. Так как $\alpha'_0=|\tilde{\alpha}|$ и функция $\Vert\alpha_\delta'\Vert_{hs}$ непрерывна при 
$\delta\in\mathbb{R}$, то существует такое значение $\delta_0$, что 
$\Vert\alpha_{\delta_0}'\Vert_{hs}<\Vert|\tilde{\alpha}^*|\Vert_{hs}+\varepsilon\Vert \tilde{x}\Vert_{\wideparen{k}}^{-1}=\Vert\tilde{\alpha}\Vert_{hs}+\varepsilon\Vert \tilde{x}\Vert_{\wideparen{k}}^{-1}$. 
Обозначим $\alpha'=\alpha_{\delta_0}'\in M_{n,n}$ и $x'=\rho\tilde{x}\in Y^n$, тогда 
$$
\alpha'x'
=(|\tilde{\alpha}^*|+\delta_0 p)\rho \tilde{x}
=|\tilde{\alpha}^*|\rho \tilde{x}+\delta_0 p\rho \tilde{x}
=\tilde{\alpha}\tilde{x}
$$
По построению полярного разложения $\Vert \rho\Vert\leq 1$, поэтому с учетом определения $\Vert x\Vert_{t_2^n(X)}'$ получаем
$$
\Vert x\Vert_{t_2^n(X)}'\leq
\Vert\alpha'\Vert_{hs}\Vert x'\Vert_{\wideparen{n}}
\leq (\Vert\tilde{\alpha}\Vert_{hs}+\varepsilon\Vert \tilde{x}\Vert_{\wideparen{k}})\Vert \rho\Vert\Vert\tilde{x}\Vert_{\wideparen{n}}
\leq\Vert\tilde{\alpha}\Vert_{hs}\Vert\tilde{x}\Vert_{\wideparen{k}}+\varepsilon
\leq \Vert x\Vert_{t_2^n(X)}+2\varepsilon
$$
Так как $\varepsilon>0$ произвольно, то $\Vert x\Vert_{t_2^n(X)}'\leq\Vert x\Vert_{t_2^n(X)}$. Обратное неравенство очевидно, поэтому $\Vert x\Vert_{t_2^n(X)}=\Vert x\Vert_{t_2^n(X)}'$.
\end{proof}

\begin{proposition}\label{PrT2nOfOpIsWellDef}
Пусть $X$, $Y$ --- секвенциальные операторные пространства, $\varphi\in\mathcal{SB}(X,Y)$ и $n,k\in\mathbb{N}$. Тогда 
\newline
1) Для любых $\alpha\in M_{n,k}$ и $x\in t_2^k(X)$ выполнено $t_2^n(\varphi)(\alpha x)=\alpha t_2^k(\varphi)(x)$
\newline
2) $t_2^n(\varphi)\in\mathcal{B}(t_2^n(X),t_2^n(Y))$, причем $\Vert t_2^n(\varphi)\Vert\leq\Vert\varphi^{\wideparen{n}}\Vert$
\newline
3) если $\varphi^{\wideparen{n}}$ (строго) $c$-топологически сюръективно, то $t_2^n(\varphi)$ так же (строго) $c$-топологически сюръективно
\newline
4)  если $\varphi^{\wideparen{n}}$ $c$-топологически инъективно, то $t_2^n(\varphi)$ так же $c$-топологически инъективно
\end{proposition}
\begin{proof}
1) Так как $t_2^n(\varphi)=\varphi^{\wideparen{n}}$ как отображения линейных пространств, то результат следует из пункта 1 утверждения \ref{PrSimplAmplProps}. 
\newline
2) Пусть $x\in t_2^n(X)$ и $x=\alpha'x'$, где $\alpha\in M_{n,n}$ --- обратимая матрица и $x'\in X^{n}$, тогда $t_2^n(\varphi)(x)=\alpha't_2^n(\varphi)(x')=\alpha'\varphi^{\wideparen{n}}(x')$, поэтому из 
определения нормы в $t_2^n(Y)$ следует, что
$$
\Vert t_2^n(\varphi)(x)\Vert_{t_2^n(Y)}
\leq\Vert\alpha'\Vert_{hs}\Vert\varphi^{\wideparen{n}}(x')\Vert_{\wideparen{n}}
\leq\Vert\alpha'\Vert_{hs}\Vert\varphi^{\wideparen{n}}\Vert\Vert x'\vert_{\wideparen{n}}
$$
Теперь возьмем инфимум по всем представлениям $x$ описанным выше, тогда предложение \ref{PrT2nNormProperty} дает
$$
\Vert t_2^n(\varphi)(x)\Vert_{t_2^n(Y)}\leq\Vert\varphi^{\wideparen{n}}\Vert\Vert x\Vert_{t_2^n(X)}
$$
Следовательно $\Vert t_2^n(\varphi)\Vert\leq\Vert\varphi^{\wideparen{n}}\Vert$ и $t_2^n(\varphi)\in\mathcal{B}(t_2^n(X),t_2^n(Y))$.
\newline
3) Пусть $\varphi^{\wideparen{n}}$  $c$-топологически сюръективен. Пусть $y\in t_2^n(Y)$ и $y=\alpha' y'$, где $\alpha'\in M_{n,n}$ --- обратимая матрица, $y'\in Y^n$. Пусть $c<c''<c'$. Так как 
$\varphi^{\wideparen{n}}$ $c$-топологически сюръективно, то существует $x'\in X^n$ такое что $\varphi^{\wideparen{n}}(x')=y'$ и $\Vert x'\Vert_{\wideparen{n}}< c''\Vert y'\Vert_{\wideparen{n}}$. Рассмотрим 
$x:=\alpha'x'$, тогда $t_2^n(\varphi)(x)=\alpha't_2^n(\varphi)(x')=\alpha'\varphi^{\wideparen{n}}(x')=\alpha' y'=y$. Из определения нормы в $t_2^n(X)$ получаем
$$
\Vert x\Vert_{t_2^n(X)}
\leq\Vert\alpha'\Vert_{hs}\Vert x'\Vert_{\wideparen{n}}
\leq\Vert\alpha'\Vert_{hs} c''\Vert y'\Vert_{\wideparen{n}}
$$
Теперь возьмем инфимум по всем представлениям $y$ описанным выше, тогда предложение \ref{PrT2nNormProperty} дает $\Vert x\Vert_{t_2^n(X)}\leq c''\Vert y\Vert_{t_2^n(Y)}<c'\Vert y\Vert_{t_2^n(Y)}$
Таким образом, для любого $y\in t_2^n(Y)$ и любого $c'>c$ существует $x\in t_2^n(X)$ такой что $t_2^n(\varphi)(x)=y$ и $\Vert x\Vert_{t_2^n(X)}< c'\Vert y\Vert_{t_2^n(Y)}$. Следовательно $t_2^n(\varphi)$ 
$c$-топологически сюръективен.
\newline
Пусть $\varphi^{\wideparen{n}}$ строго $c$-топологически сюръективен. Пусть $y\in t_2^n(Y)$ и $y=\alpha' y'$, где $\alpha'\in M_{n,n}$ --- обратимая матрица, $y'\in Y^n$. Так как $\varphi^{\wideparen{n}}$ $c$-топологически 
сюръективно, то существует $x'\in X^n$ такое что $\varphi^{\wideparen{n}}(x')=y'$ и $\Vert x'\Vert_{\wideparen{n}}\leq c\Vert y'\Vert_{\wideparen{n}}$. Рассмотрим $x:=\alpha'x'$, тогда 
$t_2^n(\varphi)(x)=\alpha't_2^n(\varphi)(x')=\alpha'\varphi^{\wideparen{n}}(x')=\alpha' y'=y$. Из определения нормы в $t_2^n(X)$ получаем
$$
\Vert x\Vert_{t_2^n(X)}
\leq\Vert\alpha'\Vert_{hs}\Vert x'\Vert_{\wideparen{n}}
\leq\Vert\alpha'\Vert_{hs} c\Vert y'\Vert_{\wideparen{n}}
$$
Теперь возьмем инфимум по всем представлениям $y$ описанным выше, тогда предложение \ref{PrT2nNormProperty} дает $\Vert x\Vert_{t_2^n(X)}\leq c\Vert y\Vert_{t_2^n(Y)}$
Таким образом, для любого $y\in t_2^n(Y)$ существует $x\in t_2^n(X)$ такой что $t_2^n(\varphi)(x)=y$ и $\Vert x\Vert_{t_2^n(X)}\leq c\Vert y\Vert_{t_2^n(Y)}$. Следовательно $t_2^n(\varphi)$ строго $c$-топологически сюръективен.
\newline
4) Пусть $x\in t_2^n(X)$, обозначим $y:=t_2^n(\varphi)(x)$. Пусть имеется представление $y=\alpha' y'$, где $\alpha'\in M_{n,n}$ --- обратимая матрица, $y'\in Y^n$. Тогда 
$y'=(\alpha')^{-1}y=(\alpha')^{-1}t_2^n(\varphi)(x)=t_2^n(\varphi)((\alpha')^{-1}x)\in\operatorname{Im}(t_n^2(\varphi))
$. Так как $\varphi^{\wideparen{n}}$ $c$-топологически инъективен, то он инъективен, поэтому для $y'\in \operatorname{Im}(t_2^n(\varphi))$ существует $x'\in X^n$ такой что 
$y'=t_2^n(\varphi)(x')=\varphi^{\wideparen{n}}(x')$. Так как $\varphi^{\wideparen{n}}$ $c$-топологически инъективен, то $\Vert x'\Vert_{\wideparen{n}}\leq c\Vert y'\Vert$. Из определения нормы в $t_2^n(X)$ следует, что
$$
\Vert x\Vert_{t_2^n(X)}\leq\Vert\alpha'\Vert_{hs}\Vert x'\Vert_{\wideparen{n}}\leq c\Vert\alpha'\Vert_{hs}\Vert y'\Vert_{\wideparen{n}}
$$
Теперь возьмем инфимум по всем представлениям $y$ описанным выше, тогда предложение \ref{PrT2nNormProperty} дает $\Vert x\Vert_{t_2^n(X)}\leq c\Vert y\Vert_{t_2^n(Y)}=c\Vert t_2^n(\varphi)(x)\Vert_{t_2^n(Y)}$. 
Таким образом, для любого $x\in t_2^n(X)$ выполнено $\Vert t_2^n(\varphi)(x)\Vert_{t_2^n(Y)}\geq c^{-1}\Vert x\Vert_{t_2^n(X)}$. Следовательно, $t_2^n(\varphi)$ $c$-топологически инъективен.
\end{proof}

\begin{proposition}[\cite{LamOpFolgen}, 1.3.16]\label{PrT2nTraingDuality}
Пусть $X$ --- секвенциальное операторное пространство и $n\in\mathbb{N}$. Тогда имеют место изометрические изоморфизмы
$$
\alpha_X^n:t_2^n(X^\triangle)\to (X^{\wideparen{n}})^*: f\mapsto\left(x\mapsto\sum\limits_{i=1}^n f_i(x_i)\right)
\qquad
\beta_X^n:(X^\triangle)^{\wideparen{n}}\to t_2^n(X)^*:f\mapsto\left(x\mapsto\sum\limits_{i=1}^n f_i(x_i)\right)
$$
\end{proposition}

\begin{proposition}\label{PrTwoTypesDualOpEquiv}
Пусть $X$, $Y$ --- секвенциальные операторные пространства, $\varphi\in \mathcal{SB}(X,Y)$ и $n\in\mathbb{N}$, тогда 
\newline
1) $(\varphi^\triangle)^{\wideparen{n}}$ $c$-топологически инъективен (сюръективен) тогда и только тогда когда $t_2^n(\varphi)^*$ $c$-топологически инъективен (сюръективен)
\newline
2) $t_2^n(\varphi^\triangle)$ $c$-топологически инъективен (сюръективен) тогда и только тогда когда $(\varphi^{\wideparen{n}})^*$ $c$-топологически инъективен (сюръективен)
\newline
3) верны равенства $\Vert (\varphi^\triangle)^{\wideparen{n}}\Vert=\Vert t_2^n(\varphi)^*\Vert$ и $\Vert t_2^n(\varphi^\triangle)\Vert=\Vert (\varphi^{\wideparen{n}})^*\Vert$ и  $\Vert t_2^n(\varphi)\Vert=\Vert\varphi^{\wideparen{n}}\Vert$
\end{proposition}
\begin{proof}
Пусть $g\in (Y^\triangle)^{\wideparen{n}}$ и $x\in t_2^n(X)$, тогда
$$
(\alpha_X^n(\varphi^\triangle)^{\wideparen{n}})(g)(x)
=\alpha_X^n((\varphi^\triangle)^{\wideparen{n}}(g))(x)
=\sum\limits_{k=1}^n (\varphi^\triangle)^{\wideparen{n}}(g)_k(x_k)
=\sum\limits_{k=1}^n (\varphi^\triangle)(g_k)(x_k)
=\sum\limits_{k=1}^n g_k(\varphi(x_k))
$$
$$
(t_2^n(\varphi)^* \alpha_Y^n)(g)(x)
=t_2^n(\varphi)^*(\alpha_Y^n(g))(x)
=\alpha_Y^n(g)(t_2^n(\varphi)(x))
=\sum\limits_{k=1}^n g_k(t_2^n(\varphi)(x)_k)
=\sum\limits_{k=1}^n g_k(\varphi(x_k))
$$
Так как $g$ и $x$ произвольны, то $\alpha_X^n(\varphi^\triangle)^{\wideparen{n}}=t_2^n(\varphi)^* \alpha_Y^n$. Так как $\alpha_Y^n$ и $\alpha_X^n$ изометрические изоморфизмы, то мы получаем утверждение 1) и равенство $\Vert (\varphi^\triangle)^{\wideparen{n}}\Vert=\Vert t_2^n(\varphi)^*\Vert$.
Пусть $g\in t_2^n(Y^\triangle)$ и $x\in X^{\wideparen{n}}$, тогда
$$
(\beta_X^n t_2^n(\varphi^\triangle))(g)(x)
=\beta_X^n(t_2^n(\varphi^\triangle)(g))(x)
=\sum\limits_{k=1}^n t_2^n(\varphi^\triangle)(g)_k(x_k)
=\sum\limits_{k=1}^n (\varphi^\triangle)(g_k)(x_k)
=\sum\limits_{k=1}^n g_k(\varphi(x_k))
$$
$$
((\varphi^{\wideparen{n}})^*\beta_Y^n)(g)(x)
=(\varphi^{\wideparen{n}})^*(\beta_Y^n(g))(x)
=\beta_Y^n(g)(\varphi^{\wideparen{n}})(x))
=\sum\limits_{k=1}^n g_k(\varphi^{\wideparen{n}})(x)_k)
=\sum\limits_{k=1}^n g_k(\varphi(x_k))
$$
Так как $g$ и $x$ произвольны, то $\beta_X^n t_2^n(\varphi^\triangle)=(\varphi^{\wideparen{n}})^*\beta_Y^n$. Так как $\beta_Y^n$ и $\beta_X^n$ изометрические изоморфизмы, то мы получаем утверждение 2) и равенство $\Vert t_2^n(\varphi^\triangle)\Vert=\Vert (\varphi^{\wideparen{n}})^*\Vert$.

Наконец, из предложений \ref{PrT2nOfOpIsWellDef}, \ref{PrDualSBOp} следует что $\Vert t_2^n(\varphi)\Vert\leq\Vert\varphi^{\wideparen{n}}\Vert=\Vert(\varphi^\triangle)^{\wideparen{n}}\Vert=\Vert t_2^n(\varphi)^*\Vert=\Vert t_2^n(\varphi)\Vert$, т.е. $\Vert t_2^n(\varphi)\Vert=\Vert\varphi^{\wideparen{n}}\Vert$.
\end{proof}


\begin{theorem}\label{ThDualSQOps}
Пусть $X$, $Y$ --- секвенциальные операторные пространства и $\varphi\in\mathcal{SB}(X,Y)$, тогда
\newline
1) $\varphi$ (строго) секвенциально $c$-топологически сюръективен $\Longrightarrow$
$ \varphi^\triangle$ секвенциально $c$-топологически инъективен
\newline
2) $\varphi$ секвенциально $c$-топологически инъективен $\Longrightarrow$ строго
$ \varphi^\triangle$ строго секвенциально $c$-топологически сюръективен
\newline
3) $\varphi^\triangle$ (строго) секвенциально $c$-топологически сюръективен $\Longrightarrow$
$ \varphi$ секвенциально $c$-топологически инъективен
\newline
4) $\varphi^\triangle$ секвенциально $c$-топологически инъективен $\Longrightarrow$
$ \varphi$ строго секвенциально $c$-топологически сюръективен
\newline
5) $\varphi$ секвенциально коизометричен $\Longrightarrow$ 
$\varphi^\triangle$ секвенциально изометричен, если $X$ полно, то верно и обратное
\newline
6) $ \varphi$ секвенциально изометричен $\Longleftrightarrow$
$\varphi^\triangle$ секвенциально строго коизометричен
\end{theorem}
\begin{proof}
Для каждого $n\in\mathbb{N}$ мы имеем цепочку импликаций
\newline
\begin{tabular}{llllll}
$\varphi^{\wideparen{n}}$ & $c$-топологически инъективен & $\implies$ & $t_2^n(\varphi)$                    & $c$-топологически инъективен       &\ref{PrT2nOfOpIsWellDef}\\
                        &                              & $\implies$ & $t_2^n(\varphi)^*$                  & строго $c$-топологически сюръективен      &\ref{PrDualOps}\\
                        &                              & $\implies$ & $(\varphi^\triangle)^{\wideparen{n}}$ & строго $c$-топологически сюръективен &\ref{PrTwoTypesDualOpEquiv}\\
                        &                              & $\implies$ & $t_2^n(\varphi^\triangle)$          & строго $c$-топологически сюръективен &\ref{PrT2nOfOpIsWellDef}\\
                        &                              & $\implies$ & $(\varphi^{\wideparen{n}})^*$         & строго $c$-топологически сюръективен &\ref{PrTwoTypesDualOpEquiv}\\
                        &                              & $\implies$ & $\varphi^{\wideparen{n}}$             & $c$-топологически инъективен       &\ref{PrDualOps}\\
\end{tabular}
\newline
Откуда мы получаем 2) и 3). Снова для любого $n\in\mathbb{N}$ мы имеем цепочку ипликаций
\newline
\begin{tabular}{llclll}
$\varphi^{\wideparen{n}}$ & \begin{tabular}{c}(строго) $c$-топологически\\ сюръективен\end{tabular}  & $\implies$ & $t_2^n(\varphi)$                    & $c$-топологически сюръективен     &\ref{PrT2nOfOpIsWellDef}\\
                        &                               & $\implies$ & $t_2^n(\varphi)^*$                  & $c$-топологически инъективен      &\ref{PrDualOps}\\
                        &                               & $\implies$ & $(\varphi^\triangle)^{\wideparen{n}}$ & $c$-топологически инъективен &\ref{PrTwoTypesDualOpEquiv}\\
                        &                               & $\implies$ & $t_2^n(\varphi^\triangle)$          & $c$-топологически инъективен &\ref{PrT2nOfOpIsWellDef}\\
                        &                               & $\implies$ & $(\varphi^{\wideparen{n}})^*$         & $c$-топологически инъективен &\ref{PrTwoTypesDualOpEquiv}\\
                        &                               & $\overset{\mbox{$X$ полно}}{\implies}$ & $\varphi^{\wideparen{n}}$             & $c$-топологически сюръективен     &\ref{PrDualOps}\\
\end{tabular}
\newline
Откуда мы получаем 1) и 4). Пункты 5) и 6) являются прямым следствием 1)---4) при $c=1$ если учесть что $\varphi$ секвенциально сжимающий тогда и только тогда $\varphi^\triangle$ секвенцильно сжимающий (см. предложение \ref{PrDualSBOp}).
\end{proof}























\subsection{Слабые топологии для секвенциальных операторных пространств}

\begin{definition}\label{DefSQDconv} Пусть $\mathcal{D}:X\times Y\to Z$ --- векторная двойственность между секвенциальными операторными пространствами $X$, $Y$ и $Z$. Будем говорить, что направленность $(y_\nu)_{\nu\in N}\subset Y^{\wideparen{n}}$ секвенциально $\mathcal{D}$-сходится к $y\in Y^{\wideparen{n}}$ если она $\mathcal{D}^{\wideparen{m\times n}}$-сходится для каждого $m\in\mathbb{N}$.  Топологию задавааемую данным видом сходимости мы будем обозначать $\sigma_{\mathcal{D}}^{\widehat{n}}(Y,X)$.
\end{definition}

\begin{proposition}\label{PrDConvEquivCoordwsConv} Пусть $\mathcal{D}:X\times Y\to Z$ --- векторная двойственность между секвенциальными операторными пространствами $X$, $Y$ и $Z$, тогда следующие условия эквивалентны:
\newline
1) направленность $(y_\nu)_{\nu\in N}\subset Y^{\wideparen{n}}$ секвенциально $\mathcal{D}$-сходится к $y\in Y^{\wideparen{n}}$
\newline
2) для каждого $i\in\mathbb{N}_n$ направленность $((y_\nu)_i)_{\nu\in N}\subset Y^{\wideparen{1}}$ $\mathcal{D}$-сходится к $y_i\in Y^{\wideparen{1}}$.
\end{proposition}
\begin{proof}
1)$\implies$ 2) Заметим, что для всех $i\in\mathbb{N}_n$ и $x\in X^{\wideparen{1}}$ мы имеем $\mathcal{D}(x,(y_\nu)_i-y_i)=(\mathcal{D}^{\wideparen{1\times n}}(x,y_\nu-y))_i$. Используя предложение \ref{PrNormVsSQNorm}, мы получаем
$$
\lim\limits_{\nu}\Vert \mathcal{D}(x,(y_\nu)_i-y_i)\Vert_{\wideparen{1}}
\leq\lim\limits_{\nu}\Vert \mathcal{D}^{\wideparen{1\times n}}(x,y_\nu-y)\Vert_{\wideparen{1\times n}}=0
$$
поэтому $((y_\nu)_i)_{\nu\in N}$ $\mathcal{D}$-сходится к $y_i$.

2)$\implies$ 1) Снова из предложения \ref{PrNormVsSQNorm} для всех $m\in\mathbb{N}$ и $x\in X^{\wideparen{m}}$ выполнено
$$
\lim\limits_{\nu}\Vert\mathcal{D}^{\wideparen{m\times n}}(x,y_\nu-y)\Vert_{\wideparen{m\times n}}
\leq\lim\limits_{\nu}\sum\limits_{j=1}^{m}\sum\limits_{i=1}^n\Vert\mathcal{D}(x_j,(y_\nu)_i-y_i)\Vert_{\wideparen{1}}
=\sum\limits_{j=1}^{m}\sum\limits_{i=1}^n\lim\limits_{\nu}\Vert\mathcal{D}(x_j,(y_\nu)_i-y_i)\Vert_{\wideparen{1}}=0
$$
поэтому $(y_\nu)_{\nu\in N}$ секвенциально $\mathcal{D}$-сходится к $y$.
\end{proof}

\begin{proposition}\label{PrDContEquivCoordwsCont}
Пусть $\mathcal{D}_1:X_1\times Y_1\to Z_1$ и $\mathcal{D}_2:X_2\times Y_2\to Z_2$ --- векторные двойственности между  секвенциальными операторными пространствами и $\varphi:Y_1\to Y_2$ --- линейный оператор, тогда $\varphi$ $\sigma_{\mathcal{D}_1}(Y_1^{\wideparen{1}},X_1^{\wideparen{1}})$-$\sigma_{\mathcal{D}_2}(Y_2^{\wideparen{1}},X_2^{\wideparen{1}})$ непрерывен тогда и только тогда когда $\varphi^{\wideparen{n}}$ $\sigma_{\mathcal{D}_1}^{\wideparen{n}}(Y_1,X_1)$-$\sigma_{\mathcal{D}_2}^{\wideparen{n}}(Y_2,X_2)$ непрерывен.
\end{proposition}
\begin{proof}
1)$\implies$ 2) Допустим, направленность $(y_\nu)_{\nu\in N}\subset Y^{\wideparen{n}}$ секвенциально $\mathcal{D}_1$-сходится к $y\in Y^{\wideparen{n}}$. Тогда по предложению \ref{PrDConvEquivCoordwsConv} для каждого $i\in\mathbb{N}_n$ направленность $((y_\nu)_i)_{\nu\in N}$ $\mathcal{D}_1$-сходится к $y_i$. Из предположения на $\varphi$ мы получаем что направленность $(\varphi((y_\nu)_i))_{\nu\in N}$ $\mathcal{D}_2$-сходится к $\varphi(y_i)$ для каждого $i\in\mathbb{N}_n$. Последнее означает, что направленность $(\varphi^{\wideparen{n}}(y_\nu))_{\nu\in N}$ секвенциально $\mathcal{D}_2$-сходится к $\varphi^{\wideparen{n}}(y)$. Так как направленность $(y_\nu)_{\nu\in N}$ произвольна, то $\varphi^{\widehat{n}}$ $\sigma_{\mathcal{D}_1}^{\wideparen{n}}(Y_1,X_1)$-$\sigma_{\mathcal{D}_2}^{\wideparen{n}}(Y_2,X_2)$ непрерывен.

2)$\implies$ 1) Допустим, направленность $(y_\nu)_{\nu\in N}\subset Y^{\wideparen{1}}$ $\mathcal{D}_1$-сходится к $y\in Y^{\wideparen{1}}$. Рассмотрим $\widetilde{y}_\nu\in Y^{\wideparen{n}}$ такие, что $(\widetilde{y}_\nu)_1=y_\nu$ и $(\widetilde{y}_\nu)_i=0$ для $i\in\mathbb{N}_n\setminus\{1\}$. По предложению \ref{PrDConvEquivCoordwsConv} направленность $(\widetilde{y}_\nu)$ секвенциально $\mathcal{D}_1$-сходится к $y\in Y^{\wideparen{n}}$ такому, что $y_1=y$ and $y_i=0$ for $i\in\mathbb{N}_n\setminus\{1\}$. Из предположения на $\varphi^{\wideparen{n}}$ направленность $(\varphi^{\wideparen{n}}(\widetilde{y}_\nu))_{\nu\in N}$ секвенциально $\mathcal{D}_2$-сходится к $\varphi^{\wideparen{n}}(\widetilde{y})$. По предложению \ref{PrDConvEquivCoordwsConv} мы получаем, что направленность $(\varphi^{\wideparen{1}}((\widetilde{y}_\nu)_1))_{\nu\in N}=(\varphi(y_\nu)_{\nu\in N}$ $\mathcal{D}_2$-сходится к $\varphi^{\wideparen{1}}((\widetilde{y})_1)=\varphi(y)$. Так как направленность $(y_\nu)_{\nu\in N}$ произовльна, то $\varphi$ $\sigma_{\mathcal{D}_1}(Y_1,X_1)$-$\sigma_{\mathcal{D}_2}(Y_2,X_2)$ непрерывен.
\end{proof}

\begin{definition}\label{DefWeakConvForSQSp} Пусть $X$ секвенциальное операторное пространство, тогда мы определяем слабую топологию на $X^{\wideparen{n}}$ как $\sigma_{\mathcal{D}_{X^*,X}}^{\wideparen{n}}(X,X^*)$ топологию и слабую${}^*$ топологию на $(X^\triangle)^{\wideparen{n}}$ как $\sigma_{\mathcal{D}_{X,X^*}}^{\wideparen{n}}(X^*,X)$ топологию.
\end{definition}

В частности предложение \ref{PrDConvEquivCoordwsConv} показывает что слабая и слабая${}^*$ сходимость в размноженном пространстве эквивалентны слабой и слабой${}^*$ покоординатной сходимости соответсвенно. Из предложения \ref{PrDContEquivCoordwsCont} мы получаем, что непрерывность линейного оператора по отношению к различным слабым топологиям эквивалентна непрерыности того же типа размноженного оператора.


\begin{proposition}[\cite{LamOpFolgen}, 1.3.19]\label{PrDoubleDualIsom} Пусть $X$ --- секвенциальное операторное пространство, тогда существует изометрический изоморфизм
$$
\widetilde{\iota_X}^n:(X^{\triangle\triangle})^{\wideparen{n}}\to(X^{\wideparen{n}})^{**}:\psi\mapsto\left(f\mapsto\sum\limits_{i=1}^n \psi_i((\alpha_X^n)^{-1}(f)_i)\right)
$$
который также является слабо${}^*$-слабо${}^*$ гомеомрфизмом.
\end{proposition}
\begin{proof} Из предложения \ref{PrT2nTraingDuality}  следует, что искомый изометрический изоморфизм это $\widetilde{\iota_X}^n:=((\alpha_X^n)^*)^{-1}\beta_{X^\triangle}^n$. Его действие задается формулой $\widetilde{\iota_X}^n(\psi)(f)=\sum_{i=1}^n \psi_i((\alpha_X^n)^{-1}(f)_i)$ где $\psi\in (X^{\triangle\triangle})$ и $f\in (X^{\wideparen{n}})^*$. Допустим, направленность $(\psi_\nu)_{\nu\in N}\subset (X^{\triangle\triangle})^{\wideparen{n}}$ слабо${}^*$ сходится к $\psi\in (X^{\triangle\triangle})^{\wideparen{n}}$. По предложению \ref{PrDConvEquivCoordwsConv} это равносильно слабо${}^*$ сходимости $((\psi_\nu)_i)_{\nu\in N}\subset X^{\triangle\triangle}$ к $\psi_i\in X^{\triangle\triangle}$ для каждого $i\in\mathbb{N}_n$. Последнее эквивалентно сходимости направленности $(\psi_\nu)_i(g))_{\nu\in N}$ к $(\psi)_i(g)$ для всех $g\in X^\triangle$ и $i\in\mathbb{N}_n$. Легко видеть, что такая сходимость возможна только если направленность $(\sum_{i=1}^n(\psi_\nu)_i(g_i))_{\nu\in N}$ сходится к  $\sum_{i=1}^n(\psi_\nu)_i(g_i)$ для всех $g=(g_i)_{i\in\mathbb{N}_n}\in t_2^n(X^\triangle)$. Это равносильно сходимости направленности $(\widetilde{\iota_X}^n(\psi_\nu)(f))_{\nu\in N}$ к $\widetilde{\iota_X}^n(\psi)(f)$ для всех $f\in (X^{\wideparen{n}})^*$. Последнее означает, что направленность $(\widetilde{\iota_X}^n(\psi_\nu))_{\nu\in N}\subset (X^{\wideparen{n}})^{**}$ слабо${}^*$ сходится к $\widetilde{\iota_X}^n(\psi)$. так как $i\widetilde{\iota_X}^n$ биективно и все шаги доказательства были эквивалентностями, то $\widetilde{\iota_X}^n$ есть слабо${}^*$-слабо${}^*$ гомеоморфизм.
\end{proof}

Теперь мы можем доказать аналог теоремы Голдштайна.

\begin{proposition}\label{PrGoldsteinTh} Пусть $X$ --- секвенциальное операторное пространство, тогда $\iota_X^{\wideparen{n}}(B_{X^{\wideparen{n}}})$ слабо${}^*$ плотно в $B_{(X^{\triangle\triangle})^{\wideparen{n}}}$. Как следствие $\iota_X^{\wideparen{n}}(X^{\wideparen{n}})$ слабо${}^*$ плотно в $(X^{\triangle\triangle})^{\wideparen{n}}$.
\end{proposition} 
\begin{proof} Для всех $x\in X^{\wideparen{n}}$ и $f\in (X^*)^{\wideparen{n}}$ выполнено
$$
\widetilde{\iota_X}^n(\iota_X^{\wideparen{n}}(x))(f)
=\sum\limits_{i=1}^n\iota_X^{\wideparen{n}}(x)_i((\alpha_X^n)^{-1}(f)_i)
=\sum\limits_{i=1}^n\iota_X(x_i)((\alpha_X^n)^{-1}(f)_i)
=\sum\limits_{i=1}^n((\alpha_X^n)^{-1}(f)_i)(x_i)
$$
$$
=\alpha_X^n((\alpha_X^n)^{-1}(f))(x)=f(x)=\iota_{X^{\wideparen{n}}}(x)(f)
$$
поэтому $\widetilde{\iota_X}^n\iota_X^{\wideparen{n}}=\iota_{X^{\wideparen{n}}}$ и т.к. $\widetilde{\iota_X}^n$ --- изоморфизм, то $\iota_X^{\wideparen{n}}=(\widetilde{\iota_X}^n)^{-1}\iota_{X^{\wideparen{n}}}$. 
По теореме 3.96 \cite{FabZizBanSpTh} мы имеем, что $\iota_{X^{\wideparen{n}}}(B_{X^{\wideparen{n}}})$ слабо${}^*$ плотно в  $B_{(X^{\wideparen{n}})^{**}}$. Поскольку $\widetilde{\iota_X}$ -изометрический слабо${}^*$-слабо${}^*$ гомеоморфизм, то $\iota_X^{\wideparen{n}}(B_{X^{\wideparen{n}}})=(\widetilde{\iota_X})^{-1}\iota_{X^{\wideparen{n}}}(B_{X^{\wideparen{n}}})$ слабо${}^*$ плотен в $(\widetilde{\iota_X})^{-1}(B_{(X^{\wideparen{n}})^{**}})=B_{(X^{\triangle\triangle})^{\wideparen{n}}}$.
\end{proof}

\begin{proposition}\label{PrWStarContExtSBOp} Пусть $X$ и $Y$ --- секвенциальные операторные пространства и $\varphi\in \mathcal{SB}(X,Y^\triangle)$. Тогда существует единственный слабо${}^*$ непрерывный линейный оператор $\widetilde{\varphi}\in\mathcal{SB}(X^{\triangle\triangle},Y^\triangle)$ продолжающий $\varphi$, при этом $\Vert\widetilde{\varphi}\Vert_{sb}=\Vert\varphi\Vert$ 
\end{proposition}
\begin{proof} Обозначим $\widetilde{\varphi}=(\varphi^\triangle\iota_Y)^\triangle=\iota_{Y}^\triangle\varphi^{\triangle\triangle}$. Этот оператор слабо${}^*$ непрерывен как оператор сопряженный к ограниченному оператору. Легко проверить, что $\varphi^{\triangle\triangle}\iota_X=\iota_{Y^\triangle}\varphi$ и $\iota_Y^\triangle\iota_{Y^\triangle}=1_{Y^\triangle}$, поэтому $\widetilde{\varphi}\iota_X=\iota_{Y}^\triangle\varphi^{\triangle\triangle}\iota_X=\iota_{Y}^\triangle\iota_{Y^\triangle}\varphi=\varphi$ и мы получаем, что $\widetilde{\varphi}$ слабо${}^*$ непрерывное продолжение $\varphi$. По предложению \ref{PrGoldsteinTh} подпространство $\iota_X(X)$ слабо${}^*$ плотно в $X^{\triangle\triangle}$. Следовательно, $\widetilde{\varphi}$ --- единственное такое продолжение оператора $\varphi$. Из предложений \ref{PrSimplAmplProps}, \ref{PrSQNormsViaDuality} и \ref{PrDualSBOp} следует, что
$$
\Vert\varphi\Vert_{sb}
=\Vert\widetilde{\varphi}\iota_X\Vert_{sb}
\leq\Vert\widetilde{\varphi}\Vert_{sb}\Vert\iota_X\Vert_{sb}
=\Vert\varphi\Vert_{sb}
$$
$$
\Vert\widetilde{\varphi}\Vert_{sb}
=\Vert\iota_{Y}^\triangle\varphi^{\triangle\triangle}\Vert_{sb}
\leq\Vert\iota_{Y}^\triangle\Vert_{sb}\Vert\varphi^{\triangle\triangle}\Vert_{sb}
=\Vert\iota_{Y}\Vert_{sb}\Vert\varphi\Vert_{sb}
=\Vert\varphi\Vert_{sb}
$$
Таким образом, $\Vert\widetilde{\varphi}\Vert_{sb}=\Vert\widetilde{\varphi}\Vert_{sb}$.
\end{proof}


































\subsection{Подпространства и факторпространства секвенциальных операторных пространств}

\begin{definition}[\cite{LamOpFolgen}, 1.1.26]\label{DefSQSubSpace}
Пусть $X$ секвенциальное операторное пространство, а $X_0$ подпространство в $X$, тогда на $X_0$ можно задать стуктуру секвенциального операторного пространства если положить $X_0^{\wideparen{n}}=(X_0^n,\Vert\cdot\Vert_{\wideparen{n}})$.
\end{definition}

В этом случае естественное вложение $i_{X_0,X}:X_0\to X$ очевидно будет секвенциально изометрично.

\begin{definition}[\cite{LamOpFolgen}, 1.1.27]\label{DefSQFactorSpace}
Пусть $X$ секвенциальное операторное пространство, и $X_0$ его подпространство, тогда факторпространство $X / X_0$ также наделяется структурой секвенциального операторного пространства, если положить 
$(X / X_0)^{\wideparen{n}} = X^{\wideparen{n}} / X_0^{\wideparen{n}}$.
\end{definition}

\begin{proposition}\label{PrFactorSQOp} 
Пусть $\varphi:X\to Y$ --- секвенциально ограниченный оператор между секвенциальными операторными пространствами $E$ и $F$. Пусть $X_0$ и $Y_0$ --- замкнутые подпространства в $X$ и $Y$ соответсвенно, такие, что $\varphi(X_0)\subset Y_0$, тогда сущетсвует корректно определенный секвенциально ограниченный оператор $\widehat{\varphi}:X/X_0\to Y/Y_0:x+X_0\mapsto T(x)+Y_0$ такой что $\Vert\widehat{\varphi}^{\wideparen{n}}\Vert\leq\Vert \varphi^{\wideparen{n}}\Vert$ для всех $n\in\mathbb{N}$ поэтому $\Vert\widehat{\varphi}\Vert_{sb}\leq\Vert \varphi\Vert_{sb}$. Более того,
\newline
1) если $X_0\subset \operatorname{Ker}(\varphi)$, то  $\Vert\widehat{\varphi}\Vert_{sb}=\Vert \varphi\Vert_{sb}$
\newline
2) если $\operatorname{Ker}(\varphi)= X_0$ и $\varphi$ секвенциально $c$-топологически сюръективен, то $\widehat{\varphi}$ секвенциально $c$-топологически инъективный изоморфизм
\newline
3) если $\operatorname{Ker}(\varphi)= X_0$ и $\varphi$ секвенциально коизометричен, то $\widehat{\varphi}$ --- секвенциально изометрический изоморфизм
\end{proposition}
\begin{proof}
Так как для каждого $n\in\mathbb{N}$ выполнено $\varphi^{\widehat{n}}(X_0^{\wideparen{n}})\subset Y_0^{\wideparen{n}}$, то из предложения 1.5.2 \cite{HelFA} мы получаем, что $\Vert\widehat{\varphi}^{\wideparen{n}}\Vert\leq\Vert \varphi^{\wideparen{n}}\Vert$, поэтому $\Vert\wideparen{\varphi}\Vert_{sb}\leq\Vert\varphi\Vert_{sb}$. 1) Очевидно, $X_0^{\wideparen{n}}\subset\operatorname{Ker}(\varphi^{\wideparen{n}})$, поэтому из предложения 1.5.3 \cite{HelFA} мы получаем, что $\Vert\widehat{\varphi}^{\wideparen{n}}\Vert=\Vert \varphi^{\wideparen{n}}\Vert$ и $\Vert\wideparen{\varphi}\Vert_{sb}=\Vert\varphi\Vert_{sb}$. 2) Аналогично, $X_0^{\wideparen{n}}=\operatorname{Ker}(\varphi^{\wideparen{n}})$, поэтому из леммы A.2.1 \cite{EROpSp} мы получаем, что $\varphi^{\wideparen{n}}$ $c$-топологически инъективный изоморфизм. Следовательно $\varphi$ секвенциально $c$-топологический изоморфизм. 3) Из пункта 1) имеем $\Vert\widehat{\varphi}^{\wideparen{n}}\Vert\leq\Vert \varphi^{\wideparen{n}}\Vert\leq 1$. Из пункта 3) следует, что $\widehat{\varphi}$ $1$-топологический изоморфизм. Следовательно $\varphi^{\wideparen{n}}$ --- изометрический изоморфзим. Следовательно, $\varphi$ секвенциально изометрический изоморфизм.
\end{proof}

Применяя это предложение к оператору $\varphi=i_{X_0,X}$ мы видим, что естественная проекция $\pi_{X_0,X}$ секвенциально коизометрична.

\begin{proposition}[\cite{LamOpFolgen}, 1.4.13]\label{PrDualForQuotsAndSubsp} Пусть $X$ --- секвенциальное операторное пространство и $X_0$ его замкнутое подпространство, тогда существуют секвенциальные изометрические изоморфизмы
$$
(X/X_0)^\triangle= X_0^\perp\qquad X_0^\triangle=X^\triangle/X_0^\perp
$$
\end{proposition}
\begin{proof} По теореме \ref{ThDualSQOps} оператор $\pi_{X_0,X}^\triangle$ секвенциально изометричен. Заметим, что $\operatorname{Im}(\pi_{X_0,X}^\triangle)=\{f\circ\pi_{X_0,X}:f\in (X/X_0)^\triangle\}=\{g\in X^\triangle: g(X_0)=\{0\}\}=X_0^\perp$. Следоватлельно коограничение $\pi_{X_0,X}^\triangle|^{X_0^\perp}:(X/X_0)^\triangle\to X_0^\perp$ есть секвенциальный изометрический изоморфизм. Снова по теореме \ref{ThDualSQOps} оператор $i_{X_0,X}^\triangle$ секвенциально коизометричен. Заметим, что $\operatorname{Ker}(i_{X_0,X}^\triangle)=\{f\in X^\triangle:f\circ i=0\}=\{f\in X^\triangle: f(X_0)=\{0\}\}=X_0^\perp$. Следовательно, по предложению \ref{PrFactorSQOp} оператор $\widehat{i_{X_0,X}^\triangle}:X^\triangle/X_0^\perp\to X_0^\triangle$ секвенциально изометрический изоморфизм.
\end{proof}


\begin{proposition}\label{PrDualForWStarClQuotsAndSubsp} Пусть $X$ --- банахово секвенциальное операторное пространство и $W$ --- слабо${}^*$ замкнутое подпространство $X^*$, тогда существуют секвенциально изометрические изоморфизмы
$$
(X/W_\perp)^\triangle= W \qquad\qquad X^\triangle/W=W_\perp^\triangle
$$
который являются слабо${}^*$-слабо${}^*$ гомеоморфизмами.
\end{proposition}
\begin{proof} Так как $W$ слабо${}^*$ замкнуто, то по теореме 4.7 \cite{RudinFA} выполнено  $(W_\perp)^\perp=W$. Теперь применяя предложение \ref{PrDualForQuotsAndSubsp} к $X$ и $W_\perp$ мы получаем желаемые секвенциально изометрические  изоморфизмы, это $\pi_{W_\perp,X}^\triangle|^W$ и $\widehat{i_{W_\perp,X}^\triangle}$. Как дуальные поераторы $\pi_{W_\perp,X}^\triangle$ и $i_{W_\perp,X}^\triangle$ слабо${}^*$-слабо${}^*$ непрерывны. Очевидно, $\pi_{W_\perp,X}^\triangle|^W$ слабо${}^*$-слабо${}^*$ непрерывен как ограничение такого оператора на слабо${}^*$ замкнутое подпространство $W$. По лемме A.2.4 \cite{BleOpAlgAndMods} оператор $\widehat{i_{W_\perp,X}^\triangle}$ также слабо${}^*$-слабо${}^*$ непрерывен. Следовательно  $\pi_{W_\perp,X}^\triangle|^W$ и $\widehat{i_{W_\perp,X}^\triangle}$ слабо${}^*$-слабо${}^*$ непрерывные изометрии, поэтому по лемме A.2.5 \cite{BleOpAlgAndMods} они являются слабо${}^*$-слабо${}^*$ гомеоморфизмами.
\end{proof}






















\subsection{Прямые суммы секвенциальных операторных пространств}

\begin{definition}[\cite{LamOpFolgen}, 1.1.28]\label{DefSQProd}
Пусть $\{X_\lambda: \lambda \in \Lambda\}$ --- произвольное семейство секвенциальных операторных пространств. Их $\bigoplus_\infty$-суммой называется секвенциальное операторное пространство 
$\bigoplus_\infty\{X_\lambda^{\wideparen{1}}:\lambda\in \Lambda\}$, с семейством норм задаваемых отождеттсвленииями 
$$
\left(\bigoplus{}_\infty\{X_\lambda:\lambda \in \Lambda\}\right)^{\wideparen{n}}
=\bigoplus{}_\infty\{X_\lambda^{\wideparen{n}}:\lambda\in \Lambda\}
$$
Для заданного $x\in \left(\bigoplus{}_\infty\{X_\lambda:\lambda \in \Lambda\}\right)^{\wideparen{n}}$ через $x_\lambda$ мы будем обозначать элемент $X_\lambda^{\wideparen{n}}$ такой, что $(x_\lambda)_i=(x_i)_\lambda$ для всех $i\in\mathbb{N}_n$
\end{definition}

\begin{proposition}\label{PrVectDualProdComp} Пусть $\{X_\lambda:\lambda\in\Lambda\}$ и $\{Z_\lambda:\lambda\in\Lambda\}$ два семейства секвенциальных операторных пространств и $Y$ еще одно секвенциальное операторное пространство. Пусть $\mathcal{D}_\lambda: Y\times Z_\lambda\to X_\lambda$ для $\lambda\in\Lambda$ --- семейство векторных двойственностей, тогда рассмотрим векторную двойственность
$$
\mathcal{D}:Y\times\bigoplus{}_\infty\{Z_\lambda:\lambda\in\Lambda\}\to\bigoplus{}_\infty\{X_\lambda:\lambda\in\Lambda\}:(y,z)\mapsto\oplus_\infty\{\mathcal{D}_\lambda(y,z_\lambda):\lambda\in\Lambda\}
$$
Допустим, что $\mathcal{D}_\lambda^{Z_\lambda}$ секвенциально изометричен для каждого $\lambda\in\Lambda$, тогда таков же и $\mathcal{D}^{\bigoplus{}_\infty\{Z_\lambda:\lambda\in\Lambda\}}$. Если дополнительно $\mathcal{D}_\lambda^{Z_\lambda}$ сюръективен для каждого $\lambda\in\Lambda$, тогда $\mathcal{D}^{\bigoplus{}_\infty\{Z_\lambda:\lambda\in\Lambda\}}$ --- секвенциальный изометрический изоморфизм.
\end{proposition}
\begin{proof} Обозначим $Z=\bigoplus{}_\infty\{Z_\lambda:\lambda\in\Lambda\}$. Пусть $n\in\mathbb{N}$ и $z\in Z^{\wideparen{n}}$. Так как $\mathcal{D}_\lambda^{Z_\lambda}$ секвенциально изометричен, то
$$
\Vert z_\lambda\Vert_{\wideparen{n}}
=\Vert (\mathcal{D}_\lambda^{Z_\lambda})^{\wideparen{n}}(z_\lambda)\Vert_{\wideparen{n}}
=\sup\{\Vert \mathcal{D}_\lambda^{\wideparen{k\times n}}(y,z_\lambda)\Vert_{\wideparen{k\times n}}:k\in\mathbb{N},y\in B_{Y^{\wideparen{k}}}\}
$$
Теперь заметим, что
$$
\begin{aligned}
\Vert(\mathcal{D}^Z)^{\wideparen{n}}(z)\Vert_{\wideparen{n}}
&=\Vert A((\mathcal{D}^Z)^{\wideparen{n}}(z))\Vert_{sb}\\
&=\sup\{\Vert A((\mathcal{D}^Z)^{\wideparen{n}}(z))^{\wideparen{k}}(y)\Vert_{\wideparen{k\times n}}:k\in\mathbb{N},y\in B_{Y^{\wideparen{k}}}\}\\
&=\sup\{\Vert \mathcal{D}^{\wideparen{k\times n}}(y,z)\Vert_{\wideparen{k\times n}}:k\in\mathbb{N},y\in B_{Y^{\wideparen{k}}}\}\\
&=\sup\{\Vert \oplus_\infty\{\mathcal{D}_\lambda^{\wideparen{k\times n}}(y,z_\lambda):\lambda\in\Lambda\}\Vert_{\wideparen{k\times n}}:k\in\mathbb{N},y\in B_{Y^{\wideparen{k}}}\}\\
&=\sup\{\Vert \mathcal{D}_\lambda^{\wideparen{k\times n}}(y,z_\lambda)\Vert_{\wideparen{k\times n}}:k\in\mathbb{N},y\in B_{Y^{\wideparen{k}}},\lambda\in\Lambda\}\\
&=\sup\{\Vert z_\lambda\Vert_{\wideparen{n}}:\lambda\in\Lambda\}\\
&=\Vert z\Vert_{\wideparen{n}}
\end{aligned}
$$
Следовательно $\mathcal{D}^Z$ --- секвенциальная изометрия. Теперь перейдем ко второму предположению. Рассмотрим естественные проекции $p_\lambda:\bigoplus{}_\infty\{X_\lambda:\lambda\in\Lambda\}\to X_\lambda:x\mapsto x_\lambda$. Возьмем произвольный оператор $\varphi\in\mathcal{SB}(Y,X)$, и определим $\varphi_\lambda=p_\lambda\varphi$. Для каждого $\lambda\in\Lambda$ мы знаем, что $\mathcal{D}_\lambda^{Z_\lambda}$ сюръективен, поэтому существует $z_\lambda\in Z_\lambda$ такой что $\mathcal{D}_\lambda^{Z_\lambda}(z_\lambda)=\varphi_\lambda$. Так как $\mathcal{D}_\lambda^{Z_\lambda}$ изометричен, то $\Vert z_\lambda\Vert=\Vert p_\lambda\varphi\Vert\leq\Vert \varphi\Vert$ поэтому $\sup\{\Vert z_\lambda\Vert:\lambda\in\Lambda\}<\infty$. Следовательно, мы имеем корректно определенный $z\in\bigoplus{}_\infty\{Z_\lambda:\lambda\in\Lambda\}$. Заметим, что для всех $y\in Y$ выполнено
$$
\mathcal{D}^{Z}(z)(y)
=\oplus_\infty\{\mathcal{D}_\lambda^{Z_\lambda}(z_\lambda)(y):\lambda\in\Lambda\}
=\oplus_\infty\{\varphi_\lambda(y):\lambda\in\Lambda\}
=\oplus_\infty\{p_\lambda\varphi(y):\lambda\in\Lambda\}
=\varphi(y)
$$
следовательно $\mathcal{D}^Z(z)=\varphi$. Поскольку $\varphi$ произволен $\mathcal{D}^Z$ сюръективен, но он также инъективен как любая изометрия. Значит $\mathcal{D}^Z$ и вссе его размножения биективны, но они также и изометричны, значит $\mathcal{D}^Z$ секвенциальный изометрический изоморфизм.
\end{proof}

\begin{proposition}\label{PrSQProdUnivProp} Пусть $\{X_\lambda:\lambda\in \Lambda\}$ --- семейство секвенциальных операторных пространств, тогда
\newline
1) имеет место изометрический изоморфизм
$$
\mathcal{SB}\left(Y,\bigoplus{}_\infty\{X_\lambda:\lambda\in\Lambda\}\right)
=\bigoplus{}_\infty\{\mathcal{SB}(Y,X_\lambda):\lambda\in\Lambda\}
$$
\newline
2) секвенциальное операторное пространство $\bigoplus{}_\infty\{X_\lambda:\lambda\in\Lambda\}$ вместе с естественными проекциями $p_\lambda:\bigoplus{}_\infty\{X_\lambda:\lambda\in\Lambda\}\to X_\lambda$ является категорным произведением в $SQNor_1$.
\end{proposition}
\begin{proof} 1) По предложению \ref{PrSQOpSqQuanIsEquivToStandard} векторные двойственности $\mathcal{E}_\lambda:Y\times\mathcal{SB}(Y,X_\lambda)\to X_\lambda:(y,\varphi)\mapsto \varphi(y)$ удовлетворяют предположениям предложения \ref{PrVectDualProdComp}, следовательно $\mathcal{E}^{\bigoplus{}_\infty\{\mathcal{SB}(Y,X_\lambda):\lambda\in\Lambda\}}$ и есть искомый изометрический изоморфизм.
\newline
2) Для всех $n\in\mathbb{N}$ и $x\in \left(\bigoplus{}_\infty\{X_\lambda:\lambda\in\Lambda\}\right)^{\wideparen{n}}$ выполнено
$$
\Vert p_\lambda^{\wideparen{n}}(x)\Vert_{\wideparen{n}}
=\Vert (x_{i,\lambda})_{i\in\mathbb{N}_n}\Vert_{\wideparen{n}}
\leq\sup\{\Vert (x_{i,\lambda})_{i\in\mathbb{N}_n}\Vert_{\wideparen{n}}:\lambda\in \Lambda\}
=\Vert x\Vert_{\wideparen{n}}
$$
поэтому $p_\lambda$ секвенциально ограничен, и более того секвенциально сжимающий. Теперь рассмотрим произвольное семейство секвенциально сжимающих операторов $\{\varphi_\lambda\in\mathcal{SB}(Y,X_\lambda):\lambda\in\Lambda\}$. Из предыдущего пункта для $\varphi=\mathcal{E}^{\bigoplus{}_\infty\{\mathcal{SB}(Y,X_\lambda):\lambda\in\Lambda\}}(\oplus_\infty\{\varphi_\lambda:\lambda\in\Lambda\})$ выполнено $\Vert\varphi\Vert_{sb}=\sup\{\Vert\varphi_\lambda\Vert_{sb}:\lambda\in\Lambda\}\leq 1$. Более того, для всех $y\in Y$ имеет место равенство
$$
p_\lambda\varphi(y)
=p_\lambda\mathcal{E}^{\bigoplus{}_\infty\{\mathcal{SB}(Y,X_\lambda):\lambda\in\Lambda\}}(\oplus_\infty\{\varphi_\lambda:\lambda\in\Lambda\})(y)
=p_\lambda(\oplus_\infty\{\varphi_\lambda(y):\lambda\in\Lambda\})=\varphi_\lambda(y)
$$
то есть $p_\lambda\varphi=\varphi_\lambda$. Так как $Y$ и семмейство $\{\varphi_\lambda:\lambda\in\Lambda\}$ произвольны, то $\bigoplus{}_\infty\{X_\lambda:\lambda\in\Lambda\}$ действительно является произведением в $SQNor_1$.
\end{proof}

\begin{definition}\label{DefSQCoProd}
Пусть $\{X_\lambda: \lambda \in \Lambda\}$ --- произвольное семейство секвенциальных операторных пространств. Их $\bigoplus_1^0$-суммой называется секвенциальное операторное пространство  
$\bigoplus_1^0\{X_\lambda^{\wideparen{1}}:\lambda\in \Lambda\}$, с нормами индуцированными вложением
$$
\bigoplus{}_1^0\{X_\lambda:\lambda \in \Lambda\}\hookrightarrow
\left(\bigoplus{}_\infty\{X_\lambda^\triangle:\lambda\in \Lambda\}\right)^\triangle
$$
\end{definition}

\begin{proposition}\label{PrVectDualCoProdComp} Пусть $\{X_\lambda:\lambda\in\Lambda\}$ и $\{Z_\lambda:\lambda\in\Lambda\}$ два семейства секвенциальных опператорных пространств и $Y$ еще одно секвенциальное операторное пространствоb. Пусть $\mathcal{D}_\lambda: X_\lambda\times Z_\lambda\to Y$ для $\lambda\in\Lambda$ --- семейство векторных двойственностей,тогда рассмотрим векторную двойственность
$$
\mathcal{D}:\bigoplus{}_1^0\{X_\lambda:\lambda\in\Lambda\}\times\bigoplus{}_\infty\{Z_\lambda:\lambda\in\Lambda\}\to Y:(x,z)\mapsto\sum\limits_{\lambda\in\Lambda}\mathcal{D}_\lambda(x_\lambda,z_\lambda)
$$
Допустим, что $\mathcal{D}_\lambda^{Z_\lambda}$ секвенциально изометричен для каждого $\lambda\in\Lambda$, тогда таков же и $\mathcal{D}^{\bigoplus{}_\infty\{Z_\lambda:\lambda\in\Lambda\}}$. Если дополнительно $\mathcal{D}_\lambda^{Z_\lambda}$ сюръективен для каждого $\lambda\in\Lambda$, то $\mathcal{D}^{\bigoplus{}_\infty\{Z_\lambda:\lambda\in\Lambda\}}$ секвенциальный изометрический изоморфизм.
\end{proposition}
\begin{proof} Обозначим $Z=\bigoplus{}_\infty\{Z_\lambda:\lambda\in\Lambda\}$ и $X=\bigoplus{}_1^0\{X_\lambda:\lambda\in\Lambda\}$. Пусть $n\in\mathbb{N}$ и $z\in Z^{\wideparen{n}}$. Так как $\mathcal{D}_\lambda^{Z_\lambda}$ секвенциально изометричен, то
$$
\Vert z_\lambda\Vert_{\wideparen{n}}
=\Vert (\mathcal{D}_\lambda^{Z_\lambda})^{\wideparen{n}}(z_\lambda)\Vert_{\wideparen{n}}
=\sup\{\Vert \mathcal{D}_\lambda^{\wideparen{k\times n}}(x_\lambda,z_\lambda)\Vert_{\wideparen{k\times n}}:k\in\mathbb{N},x_\lambda\in B_{X_\lambda^{\wideparen{k}}}\}
$$
Теперь заметим, что
$$
\begin{aligned}
\Vert(\mathcal{D}^Z)^{\wideparen{n}}(z)\Vert_{\wideparen{n}}
&=\Vert A((\mathcal{D}^Z)^{\wideparen{n}}(z))\Vert_{sb}\\
&=\sup\{\Vert A((\mathcal{D}^Z)^{\wideparen{n}}(z))^{\wideparen{k}}(x)\Vert_{\wideparen{k\times n}}:k\in\mathbb{N},x\in B_{X^{\wideparen{k}}}\}\\
&=\sup\{\Vert \mathcal{D}_{Y,Y^*}^{\wideparen{kn\times m}}(A((\mathcal{D}^Z)^{\wideparen{n}}(z))^{\wideparen{k}}(x),f)\Vert_{\wideparen{kn\times m}}:k\in\mathbb{N},x\in B_{X^{\wideparen{k}}},m\in\mathbb{N},f\in B_{(Y^\triangle)^{\wideparen{m}}}\}\\
\end{aligned}
$$
Легко проверить, что $\mathcal{D}_{Y,Y^*}(\mathcal{D}^Z(z)(x),f)=\mathcal{D}_{\bigoplus{}_1^0\{X_\lambda:\lambda\in\Lambda\},\bigoplus_\infty\{X_\lambda:\lambda\in\Lambda\}}(x,\oplus_\infty\{((\mathcal{D}_\lambda^{Z_\lambda})(z_\lambda))^*(f):\lambda\in\Lambda\})$, поэтому применяя предложение \ref{PrSQNormsViaDuality} мы получим
$$
\begin{aligned}
\Vert(\mathcal{D}^Z)^{\wideparen{n}}(z)\Vert_{\wideparen{n}}
&=\sup\{\Vert \mathcal{D}_{Y,Y^*}^{\wideparen{kn\times m}}(A((\mathcal{D}^Z)^{\wideparen{n}}(z))^{\wideparen{k}}(x),f)\Vert_{\wideparen{kn\times m}}:k\in\mathbb{N},x\in B_{X^{\wideparen{k}}},m\in\mathbb{N},f\in B_{(Y^\triangle)^{\wideparen{m}}}\}\\
&=\sup\{\Vert \mathcal{D}_{\bigoplus{}_1^0\{X_\lambda:\lambda\in\Lambda\},\bigoplus_\infty\{X_\lambda^*:\lambda\in\Lambda\}}^{\wideparen{k\times nm}}(x,\oplus_\infty\{A((({}^\triangle\cdot\mathcal{D}_\lambda^{Z_\lambda})^{\wideparen{n}}(z_\lambda))^{\wideparen{m}}(f):\lambda\in\Lambda\})\Vert_{\wideparen{k\times nm}}: \\
&\qquad\qquad k\in\mathbb{N},x\in B_{X^{\wideparen{k}}},m\in\mathbb{N},f\in B_{(Y^\triangle)^{\wideparen{m}}}\}\\
&=\sup\{\Vert \oplus_\infty\{A((({}^\triangle\cdot\mathcal{D}_\lambda^{Z_\lambda})^{\wideparen{n}}(z_\lambda))^{\wideparen{m}}(f):\lambda\in\Lambda\}\Vert_{\wideparen{m\times n}}: m\in\mathbb{N},f\in B_{(Y^\triangle)^{\wideparen{m}}}\}\\
&=\sup\{\Vert A((({}^\triangle\cdot\mathcal{D}_\lambda^{Z_\lambda})^{\wideparen{n}}(z_\lambda))^{\wideparen{m}}(f)\Vert_{\wideparen{m\times n}}: m\in\mathbb{N},f\in B_{(Y^\triangle)^{\wideparen{m}}},\lambda\in\Lambda\}\\
\end{aligned}
$$
Применим предложение \ref{PrSQNormsViaDuality} еще раз
$$
\begin{aligned}
\Vert(\mathcal{D}^Z)^{\wideparen{n}}(z)\Vert_{\wideparen{n}}
&=\sup\{\Vert A((({}^\triangle\cdot\mathcal{D}_\lambda^{Z_\lambda})^{\wideparen{n}}(z_\lambda))^{\wideparen{m}}(f)\Vert_{\wideparen{m\times n}}: m\in\mathbb{N},f\in B_{(Y^\triangle)^{\wideparen{m}}},\lambda\in\Lambda\}\\
&=\sup\{\Vert\mathcal{D}_{X_\lambda,X_\lambda^*}^{\wideparen{ln\times m}}(A(((\mathcal{D}_\lambda^{Z_\lambda})^{\wideparen{n}}(z_\lambda))^{\wideparen{l}}(x_\lambda),f)\Vert_{\wideparen{ln\times m}}: l\in\mathbb{N},x_\lambda\in B_{X_\lambda^{\wideparen{l}}},m\in\mathbb{N},\\
&\qquad\qquad f\in B_{(Y^\triangle)^{\wideparen{m}}},\lambda\in\Lambda\}\\
&=\sup\{\Vert A(((\mathcal{D}_\lambda^{Z_\lambda})^{\wideparen{n}}(z_\lambda))^{\wideparen{l}}(x_\lambda)\Vert_{\wideparen{l\times n}}: l\in\mathbb{N},x_\lambda\in B_{X_\lambda^{\wideparen{l}}},\lambda\in\Lambda\}\\
&=\sup\{\Vert (\mathcal{D}_\lambda^{Z_\lambda})^{\wideparen{n}}(z_\lambda)\Vert_{\wideparen{n}}: \lambda\in\Lambda\}\\
&=\sup\{\Vert z_\lambda \Vert_{\wideparen{n}}: \lambda\in\Lambda\}\\
&=\Vert z \Vert_{\wideparen{n}}
\end{aligned}
$$
Следовательно $\mathcal{D}^Z$ --- секвенциальная изометрия. Теперь перейдем ко второму предположению. Рассмотрим естественные вложения $i_\lambda:X_\lambda\to\bigoplus{}_1^0\{X_\lambda:\lambda\in\Lambda\}:x_\lambda\mapsto (\ldots,0,x_\lambda,0,\ldots)$. Возьмем произвольный оператор $\varphi\in\mathcal{SB}(\bigoplus{}_1^0\{X_\lambda:\lambda\in\Lambda\})$, и определим $\varphi_\lambda=\varphi i_\lambda$. Для каждого $\lambda\in\Lambda$ мы знаем, что  $\mathcal{D}_\lambda^{Z_\lambda}$ сюръективен, поэтому существует $z_\lambda\in Z_\lambda$ такой что $\mathcal{D}_\lambda^{Z_\lambda}(z_\lambda)=\varphi_\lambda$. Так как $\mathcal{D}_\lambda^{Z_\lambda}$ изометричен, то $\Vert z_\lambda\Vert=\Vert p_\lambda\varphi\Vert\leq\Vert \varphi\Vert$, поэтому $\sup\{\Vert z_\lambda\Vert:\lambda\in\Lambda\}<\infty$. Следовательно мы имеем корректно определенный $z\in\bigoplus{}_\infty\{Z_\lambda:\lambda\in\Lambda\}$. Заметим, что для всех $x\in \bigoplus{}_1^0\{X_\lambda:\lambda\in\Lambda\}$ выполнено
$$
\mathcal{D}^{Z}(z)(x)
=\sum_{\lambda\in\Lambda}\mathcal{D}_\lambda^{Z_\lambda}(z_\lambda)(x_\lambda)
=\sum_{\lambda\in\Lambda}\varphi_\lambda(x_\lambda)
=\sum_{\lambda\in\Lambda}\varphi i_\lambda(x_\lambda)
=\varphi\left(\sum_{\lambda\in\Lambda} i_\lambda(x_\lambda)\right)
=\varphi(x)
$$
следовательно $\mathcal{D}^Z(z)=\varphi$. Поскольку $\varphi$ произволен $\mathcal{D}^Z$ сюръективен, но он также инъективен как любая изометрия. Значит $\mathcal{D}^Z$ и все его размножения биективны, но они также и изометричны, следовательно $\mathcal{D}^Z$ секвенциальный изометрический изоморфизм.
\end{proof}

\begin{proposition}\label{PrSQCoProdUnivProp} Пусть $\{X_\lambda:\lambda\in \Lambda\}$ ---  семейство секвенциальных операторных пространств, тогда
\newline
1) имеет место секвенциальный изометрический изоморфизм
$$
\mathcal{SB}\left(\bigoplus{}_1^0\{X_\lambda:\lambda\in\Lambda\},Y\right)
=\bigoplus{}_\infty\{\mathcal{SB}(X_\lambda,Y):\lambda\in\Lambda\}
$$
\newline
2) секвенциальное операторное пространство $\bigoplus{}_1^0\{X_\lambda:\lambda\in\Lambda\}$ вместе с естественными вложениями $i_\lambda:X_\lambda\to\bigoplus{}_\infty\{X_\lambda:\lambda\in\Lambda\}$ является категорным копроизведением в $SQNor_1$.
\end{proposition}
\begin{proof} 1) По предложению \ref{PrSQOpSqQuanIsEquivToStandard} векторные двойственности $\mathcal{E}_\lambda:X_\lambda\times\mathcal{SB}(X_\lambda,Y)\to Y:(x_\lambda,\varphi)\mapsto \varphi(x_\lambda)$ удовлетворяют предположениям предложения \ref{PrVectDualProdComp}, следовательно $\mathcal{E}^{\bigoplus{}_\infty\{\mathcal{SB}(X_\lambda,Y):\lambda\in\Lambda\}}$ и есть искомый изометрический изоморфизм.
\newline
2) Для всех $n\in\mathbb{N}$ и $x\in \left(\bigoplus{}_1^0\{X_\lambda:\lambda\in\Lambda\}\right)^{\wideparen{n}}$ выполнено
$$
\begin{aligned}
\Vert i_\lambda^{\wideparen{n}}(x)\Vert_{\wideparen{n}}
&=\sup\{\Vert\mathcal{D}_{\bigoplus{}_1^0\{X_\lambda:\lambda\in \Lambda\},\bigoplus{}_\infty\{X_\lambda^*:\lambda\in \Lambda\}}^{\wideparen{n\times n}}(i_\lambda^{\wideparen{n}}(x),f)\Vert_{\wideparen{n\times n}}: f\in B_{(\bigoplus{}_\infty\{X_\lambda^\triangle:\lambda\in \Lambda\})^{\wideparen{n}}}\}\\
&=\sup\{\Vert\mathcal{D}_{X_\lambda,X_\lambda^*}^{\wideparen{n\times n}}(\tilde{p}_\lambda^{\wideparen{n}}(f),x)\Vert_{\wideparen{n\times n}}: f\in B_{(\bigoplus{}_\infty\{X_\lambda^\triangle:\lambda\in \Lambda\})^{\wideparen{n}}}\}\\
&=\sup\{\Vert\mathcal{D}_{X_\lambda,X_\lambda^*}^{\wideparen{n\times n}}(f,x)\Vert_{\wideparen{n\times n}}: f\in B_{(X_\lambda^\triangle)^{\wideparen{n}}}\}\\
&=\Vert x\Vert_{\wideparen{n}}
\end{aligned}
$$ 
поэтому $i_\lambda$ секвенциально ограничен, и более того секвенциально сжимающий. Теперь рассмотрим произвольное семейство секвенциально сжимающих операторов $\{\varphi_\lambda\in\mathcal{SB}(X_\lambda,Y):\lambda\in\Lambda\}$. Из предыдущего пункта для  $\varphi=\mathcal{E}^{\bigoplus{}_\infty\{\mathcal{SB}(X_\lambda,Y):\lambda\in\Lambda\}}(\oplus_\infty\{\varphi_\lambda:\lambda\in\Lambda\})$ выполнено $\Vert\varphi\Vert_{sb}=\sup\{\Vert\varphi_\lambda\Vert_{sb}:\lambda\in\Lambda\}\leq 1$. Более того, для всех $y\in Y$ имеет место равенство
$$
\varphi i_\lambda(x_\lambda)
=\mathcal{E}^{\bigoplus{}_\infty\{\mathcal{SB}(Y,X_\lambda):\lambda\in\Lambda\}}(\oplus_\infty\{\varphi_\lambda:\lambda\in\Lambda\})(i_\lambda(x_\lambda))
=\sum\limits_{\lambda'\in\Lambda}\varphi_{\lambda'}(i_\lambda(x_\lambda))
=\varphi_\lambda(x_\lambda)
$$
то есть $\varphi i_\lambda=\varphi_\lambda$. Так как $Y$ и семейство $\{\varphi_\lambda:\lambda\in\Lambda\}$ произвольны, то $\bigoplus{}_1^0\{X_\lambda:\lambda\in\Lambda\}$ действительно является копроизведением в $SQNor_1$.
\end{proof}

\begin{proposition}\label{PrDualOfCoprodIsProd}
Пусть $\{X_\lambda:\lambda\in \Lambda\}$ --- семейство секвенциальных операторных пространств, тогда существует секвенциально изометрический изоморфизм
$$
\left(\bigoplus{}_1^0\{X_\lambda:\lambda\in \Lambda\}\right)^\triangle
=\bigoplus{}_\infty\{X_\lambda^\triangle:\lambda\in \Lambda\}
$$
\end{proposition}
\begin{proof}
Результат следует из предложения \ref{PrSQCoProdUnivProp} где $Y=\mathbb{C}$.
\end{proof}

\begin{definition}\label{DefSQc0Sum}
Пусть $\{X_\lambda: \lambda \in \Lambda\}$ --- семейство секвенциальных операторных пространств. По определению их $\bigoplus_0^0$-суммой называется структура секвенциального операторного пространства $\bigoplus_0^0\{X_\lambda^{\wideparen{1}}:\lambda\in \Lambda\}$, рассмотренного как подпространство в секвенциальном операторном пространстве $\bigoplus_\infty\{X_\lambda:\lambda\in \Lambda\}$.
\end{definition}

\begin{proposition}\label{PrDensSubsetOfSumOfDoubleDuals} Пусть $\{X_\lambda:\lambda\in\Lambda\}$ --- семейство секвенциальных операторных пространств, тогда множество $\{\bigoplus{}_\infty\{\iota_{X_\lambda}^{\wideparen{n}}(x_\lambda):\lambda\in \Lambda\}:x\in B_{(\bigoplus{}_0^0\{X_\lambda:\lambda\in \Lambda\})^{\wideparen{n}}}\}$ слабо${}^*$ плотно в  $B_{(\bigoplus{}_\infty\{X_\lambda^{\triangle\triangle}:\lambda\in \Lambda\})^{\wideparen{n}}}$
\end{proposition}
\begin{proof}
Пусть $\psi\in (\bigoplus{}_\infty\{X_\lambda^{**}:\lambda\in \Lambda\})^{\wideparen{m}}$ с нормой $\Vert\psi\Vert_{\wideparen{m}}\leq 1$. В частности $\Vert\psi_{i,\lambda}\Vert\leq 1$ для всех $i\in\mathbb{N}_m$ и $\lambda\in\Lambda$. Для всех $\lambda\in\Lambda$ по теореме 3.96 \cite{FabZizBanSpTh} мы имеем, что $\iota(B_{X_\lambda})$ слабо${}^*$ плотно в $X_\lambda^{**}$ поэтому для всех $i\in\mathbb{N}_m$ существует направленность $(x_{\nu,i,\lambda}'':\nu\in N_{i,\lambda})\subset B_{X_\lambda}$ слабо${}^*$ сходящаяся к $\psi_{i,\lambda}$. Для каждого $i\in\mathbb{N}_m$ рассмотрим частично упорядоченное множество $N_i=\prod_{\lambda\in\Lambda}N_{i,\lambda}$со стандартным произведением порядков, естественные проекции $\pi_{i,\lambda}:N_i\to N_{i,\lambda}$ и определим поднаправленность $x_{\nu,i,\lambda}'=x_{\pi_{i,\lambda}(\nu),i,\lambda}''$ для всех $\nu\in N_i$. Таким образом мы получили направленность $(x_{\nu,i,\lambda}':\nu\in N_i)$ слабо${}^*$ сходящуюся к $\psi_{i,\lambda}$. Последнее означает слабую${}^*$ сходимость направленности $(\bigoplus_\infty\{\iota_{X_\lambda}(x_{\nu,i,\lambda}'):\lambda\in\Lambda\}:\nu\in N_i)\subset B_{\bigoplus{}_\infty\{X_\lambda^{**}:\lambda\in \Lambda\}}$ к $\psi_i$. Снова, рассмотрим частично упорядоченное множество $N=\prod_{i=1}^m N_i$ со стандартным произведением порядков, естественные проекции $\pi_i:N\to N_i$ и определим поднаправленность $x_{\nu,i,\lambda}=x_{\pi_i(\nu),i,\lambda}'$ для всех $\nu\in N$. Тогда мы получим направленность $(\bigoplus_\infty\{\iota_{X_\lambda}(x_{\nu,i,\lambda}):\lambda\in\Lambda\}:\nu\in N)$ слабо${}^*$ сходящуюся к $\psi_i$. По предложению \ref{PrDConvEquivCoordwsConv} мы получаем что направленность $(\bigoplus_\infty\{\iota_{X_\lambda}(x_{\nu,\lambda}):\lambda\in\Lambda\}:\nu\in N)$ слабо${}^*$ сходится к $\psi$ и благодаря определению нормы в $\bigoplus_\infty$-sum эта направленность содержится в единичном шаре $(\bigoplus{}_\infty\{X_\lambda^{\triangle\triangle}:\lambda\in \Lambda\})^{\wideparen{m}}$. 
\end{proof}

\begin{proposition}\label{PrDualOfc0SumIsCoProd}
Пусть $\{X_\lambda:\lambda\in \Lambda\}$ --- семейство секвенциальных операторных пространств, тогда существует секвенциально изометрический изоморфизм
$$
\left(\bigoplus{}_0^0\{X_\lambda:\lambda\in \Lambda\}\right)^\triangle
=\bigoplus{}_1\{X_\lambda^\triangle:\lambda\in \Lambda\}
$$
\end{proposition}
\begin{proof}
Для каждого $n\in\mathbb{N}$ и $f\in \left(\bigoplus{}_0^0\{X_\lambda^\triangle:\lambda\in \Lambda\}\right)^{\wideparen{n}}$ имеем 
$$
\begin{aligned}
\Vert(\mathcal{D}&_{\bigoplus{}_0^0\{X_\lambda:\lambda\in \Lambda\},\bigoplus{}_1\{X_\lambda^*:\lambda\in \Lambda\}}^{\bigoplus{}_1\{X_\lambda^*:\lambda\in \Lambda\}})^{\wideparen{n}}(f)\Vert_{\wideparen{n}}=\\
&=\sup\{\Vert\mathcal{D}_{\bigoplus{}_0^0\{X_\lambda:\lambda\in \Lambda\},\bigoplus{}_1\{X_\lambda^*:\lambda\in \Lambda\}}^{\wideparen{m\times n}}(x,f)\Vert_{\wideparen{m\times n}}:m\in\mathbb{N}, x\in B_{\bigoplus{}_0^0\{X_\lambda:\lambda\in \Lambda\}}\}\\
&=\sup\{\Vert\mathcal{D}_{\bigoplus{}_1\{X_\lambda^*:\lambda\in \Lambda\},\bigoplus{}_\infty\{X_\lambda^{**}:\lambda\in \Lambda\}}^{\wideparen{m\times n}}(f,\oplus_\infty\{\iota_{X_\lambda}(x_\lambda):\lambda\in\Lambda\})\Vert_{\wideparen{m\times n}}:m\in\mathbb{N}, x\in B_{\bigoplus{}_0^0\{X_\lambda:\lambda\in \Lambda\}}\}\\
\end{aligned}
$$
Так как, очевидно, $\mathcal{D}$ слабо${}^*$ непрерывен по второй переменной, то из предложения \ref{PrDensSubsetOfSumOfDoubleDuals} мы получаем
$$
\begin{aligned}
\Vert(&\mathcal{D}_{\bigoplus{}_0^0\{X_\lambda:\lambda\in \Lambda\},\bigoplus{}_1\{X_\lambda^*:\lambda\in \Lambda\}}^{\bigoplus{}_1\{X_\lambda^*:\lambda\in \Lambda\}})^{\wideparen{n}}(f)\Vert_{\wideparen{n}}=\\
&=\sup\{\Vert\mathcal{D}_{\bigoplus{}_1\{X_\lambda^*:\lambda\in \Lambda\},\bigoplus{}_\infty\{X_\lambda^{**}:\lambda\in \Lambda\}}^{\wideparen{m\times n}}(f,\psi)\Vert_{\wideparen{m\times n}}:m\in\mathbb{N}, x\in B_{\bigoplus{}_\infty\{X_\lambda^{\triangle\triangle}:\lambda\in \Lambda\}}\}\\
&=\Vert f\Vert_{\wideparen{n}}
\end{aligned}
$$
Следовательно, $\mathcal{D}_{\bigoplus{}_0^0\{X_\lambda:\lambda\in \Lambda\},\bigoplus{}_1\{X_\lambda^*:\lambda\in \Lambda\}}^{\bigoplus{}_1\{X_\lambda^*:\lambda\in \Lambda\}}$ секвенциально изометричен, но по предложению \ref{PrSumDuality} он также биективен, следовательно он и есть искомый изометрический изоморфизм.
\end{proof}

Аналогичные утверждения справедливы и в банаховом случае (с заменой $\bigoplus{}_1^0$-сумм и $\bigoplus{}_0^0$-сумм на $\bigoplus{}_1$-суммы и $\bigoplus{}_0$-суммы).

Следующее предложение активно использует терминологию и результаты работы \cite{BrownItoUniquePredual}.

\begin{proposition}\label{PrUniquePredualForCoproduct}
Пусть $\{X_\lambda:\lambda\in\Lambda\}$ --- семейство рефлексивных секвенциальных операторных пространств, тогда $\bigoplus{}_\infty\{X_\lambda:\lambda\in\Lambda\}$ имеет единственный (с точностью до секвенциального изометрического изоморфизма) банахов секвенциальный предуал $\bigoplus{}_1\{X_\lambda^\triangle:\lambda\in\Lambda\}$
\end{proposition} 
\begin{proof} Для каждого $\lambda\in\Lambda$ пространство $X_\lambda$ рефлексивно, поэтому оно принадлежит классу $(L_0)$ и по теореме 1 \cite{BrownItoUniquePredual} пространство $\bigoplus_0\{X_\lambda:\lambda\in\Lambda\}$ также принадлежит классу $(L_0)$. Из замечания следовавшего после предложения 4 \cite{BrownItoUniquePredual} и предложений \ref{PrDualOfCoprodIsProd}, \ref{PrDualOfc0SumIsCoProd} мы получаем, что $\bigoplus_0\{X_\lambda:\lambda\in\Lambda\}^{**}=\bigoplus_\infty\{X_\lambda^{**}:\lambda\in\Lambda\}=\bigoplus_\infty\{X_\lambda:\lambda\in\Lambda\}$ имеет как банахово пространство единственный с точностью до изометрического изоморфизма предуал $(\bigoplus_0\{X_\lambda:\lambda\in\Lambda\})^{*}=\bigoplus_1\{X_\lambda^{*}:\lambda\in\Lambda\}$. Так как свойство быть секвенциальным предуалом сильнее чем свойство быть предуалом, то единственный кандидат на роль секвенциального предуала $\bigoplus{}_\infty\{X_\lambda:\lambda\in\Lambda\}$ это $\bigoplus{}_1\{X_\lambda^\triangle:\lambda\in\Lambda\}$. Из замечания \ref{RemSqReflexiv} следует, что пространство $X_\lambda$ секвенциально рефлексивно для всех $\lambda\in\Lambda$ и по предложению \ref{PrDualOfCoprodIsProd} имеем
$$
\left(\bigoplus{}_1\{X_\lambda^\triangle:\lambda\in\Lambda\}\right)^\triangle
=\bigoplus{}_\infty\{X_\lambda^{\triangle\triangle}:\lambda\in\Lambda\}
=\bigoplus{}_\infty\{X_\lambda:\lambda\in\Lambda\}
$$
\end{proof}











































\subsection{Минимальная и максимальная структура секвенциального операторного операторного пространства}

\begin{definition}[\cite{LamOpFolgen}, 2.1.1]\label{DefSQMin} Минимальная структура секвенциального операторного пространства $\min(E)$ для нормированного пространства $E$ задаётся равенством $\min(E)^{\wideparen{n}} = \mathcal{B}(l_2^n, E)$. 
При этом для каждого $x \in E^n$ выполнено
$$
\Vert x\Vert_{\wideparen{n}}=\sup\left\{\left\Vert\sum\limits_{i=1}^n \xi_i x_i\right\Vert:\xi\in B_{l_2^n}\right\}
$$
\end{definition}

\begin{proposition}[\cite{LamOpFolgen}, 2.1.4]\label{PrCharMinSQ}
Пусть $X$ --- секвенциальное операторное пространство, тогда следующие условия эквивалентны
\newline
1) $X=\min(X^{\wideparen{1}})$
\newline
2) для каждого секвенциального операторного пространства $Y$ каждый ограниченный оператор $\varphi:Y\to X$ секвенциально  ограничен и $\Vert\varphi\Vert_{sb}=\Vert\varphi\Vert$
\newline
3) для каждого секвенциального операторного пространства $Y$ имеет место изометрический изоморфизм $\mathcal{SB}(Y,X)^{\wideparen{1}}=\mathcal{B}(Y^{\wideparen{1}},X^{\wideparen{1}})$
\end{proposition}

\begin{proposition}[\cite{LamOpFolgen}, 1.1.11, 2.1.5]\label{PrMinFucntor}
Отображение
$$
\begin{aligned}
\min : Nor_1 \to SQNor_1 : X&\mapsto \min(X)\\
\varphi&\mapsto\varphi
\end{aligned}
$$
задает ковариантный функтор из категории номированных пространств в категорию секвенциальных  операторных пространств.
\end{proposition}

Очевидно, что следующее определение есть обобщение примера \ref{ExT2nSQ}.

\begin{definition}[\cite{LamOpFolgen}, 2.1.7]\label{DefSQMax} Максимальная структура секвенциального операторного пространства $\max(E)$ для нормированного пространства $E$ задаётся семейством норм
$$
\Vert x\Vert_{\wideparen{n}}=\inf\left\{\Vert\alpha\Vert_{M_{n,k}}\left(\sum\limits_{i=1}^k\Vert\tilde  x_i\Vert^2\right)^{1/2}:x=\alpha\tilde x\right\}
$$
где $x\in E^{\wideparen{n}}$, $\alpha\in M_{n,k}$, $\tilde{x}\in E^k$.
\end{definition}

\begin{proposition}[\cite{LamOpFolgen}, 2.1.9]\label{PrCharMaxSQ}
Пусть $X$ --- секвенциальное операторное пространство, тогда следующие условия эквивалентны
\newline
1) $X=\max(X^{\wideparen{1}})$
\newline
2) для каждого секвенциального операторного пространства $Y$ каждый ограниченный оператор $\varphi:X\to Y$ секвенциально  ограничен и $\Vert\varphi\Vert_{sb}=\Vert\varphi\Vert$
\newline
3) для каждого секвенциального операторного пространства $Y$ имеет место изометрический изоморфизм $\mathcal{SB}(X,Y)^{\wideparen{1}}=\mathcal{B}(X^{\wideparen{1}},Y^{\wideparen{1}})$
\end{proposition}

\begin{proposition}[\cite{LamOpFolgen}, 1.1.11,2 2.1.10]\label{PrMaxFucntor}
Отображение
$$
\begin{aligned}
\max : Nor_1 \to SQNor_1 : X&\mapsto \max(X)\\
\varphi&\mapsto\varphi
\end{aligned}
$$
задает ковариантный функтор из категории номированных пространств в категорию секвенциальных операторных пространств.
\end{proposition}

\begin{proposition}\label{PrMinPreserveEmbedings} Пусть $\varphi:E\to F$ --- ограниченный линейный оператор между нормированными пространствами $E$ и $F$, тогда 
\newline
1) если $\varphi$ $c$-топологически инъективен, то $\min(\varphi)$ секвенциально $c$-топологически инъективен
\newline
2) если $\varphi$ изометричен, то $\min(\varphi)$ секвенциально изометричен
\end{proposition}
\begin{proof} 1) Для каждого $n\in\mathbb{N}$ и $x\in \min(E)^{\wideparen{n}}$ выполнено
$$
\Vert \min(\varphi)^{\wideparen{n}}(x)\Vert_{\wideparen{n}}
=\sup\left\{\left\Vert\sum\limits_{i=1}^n\xi_i \varphi^{\wideparen{n}}(x)_i\right\Vert:\xi\in B_{l_2^n}\right\}
=\sup\left\{\left\Vert\sum\limits_{i=1}^n\xi_i \varphi(x_i)\right\Vert:\xi\in B_{l_2^n}\right\}
$$
$$
=\sup\left\{\left\Vert\varphi\left(\sum\limits_{i=1}^n\xi_i x_i\right)\right\Vert:\xi\in B_{l_2^n}\right\}
\geq c^{-1}\sup\left\{\left\Vert\sum\limits_{i=1}^n\xi_i x_i\right\Vert:\xi\in B_{l_2^n}\right\}
=c^{-1}\Vert x\Vert_{\wideparen{n}}
$$
Следовательно, $\min(\varphi)$ секвенциально $c$-топологически инъективен.
\newline
2) Из предыдущего пункта $\min(\varphi)$ $1$-топологически инъективен. С другой стороны, по предложению \ref{PrCharMinSQ} имеем равенство $\Vert\min(\varphi)\Vert_{sb}=\Vert\varphi\Vert=1$. Следовательно $\min(\varphi)$ секвенциально изометричен.
\end{proof}

\begin{proposition}\label{PrMinCommuteWithProd} Пусть $\{X_\lambda:\lambda\in\Lambda\}$ семейство минимальных секвенциальных операторных пространств, тогда $\bigoplus{}_\infty\{X_\lambda:\lambda\in\Lambda\}$ также минимально.
\end{proposition} 
\begin{proof}
Пусть $Y$ --- произовльное секвенциальное операторное пространство, тогда из предложений \ref{PrSQProdUnivProp}, \ref{PrCharMinSQ} и \ref{PrCharMaxSQ} мы имеем изометрические изоморфизмы
$$
\mathcal{SB}\left(Y,\bigoplus{}_\infty\{X_\lambda:\lambda\in\Lambda\}\right)^{\wideparen{1}}
=\bigoplus{}_\infty\{\mathcal{SB}(Y,X_\lambda)^{\wideparen{1}}:\lambda\in\Lambda\}
=\bigoplus{}_\infty\{\mathcal{B}(Y^{\wideparen{1}},X_\lambda^{\wideparen{1}}):\lambda\in\Lambda\}
$$
$$
=\bigoplus{}_\infty\{\mathcal{SB}(\max(Y^{\wideparen{1}}),X_\lambda)^{\wideparen{1}}:\lambda\in\Lambda\}
=\mathcal{SB}\left(\max(Y^{\wideparen{1}}),\bigoplus{}_\infty\{X_\lambda:\lambda\in\Lambda\}\right)^{\wideparen{1}}
$$
$$
=\mathcal{B}\left(Y^{\wideparen{1}},\left(\bigoplus{}_\infty\{X_\lambda:\lambda\in\Lambda\}\right)^{\wideparen{1}}\right)
$$
Поскольку $Y$ произвольно, из предложения \ref{PrCharMinSQ} мы получаем, что $\bigoplus{}_\infty\{X_\lambda:\lambda\in\Lambda\}$ имеет минимальную структуру секвенциального операторного пространства.
\end{proof}

\begin{proposition}\label{PrCommCstarAlgIsMin} Пусть $A$ --- коммутативная $C^*$ алгебра и $X$ --- секвенциальное операторное пространство, тогда каждый ограниченный линейный оператор $\varphi:X\to A$ секвенциально ограничен и $\Vert\varphi\Vert_{sb}=\Vert\varphi\Vert$. Как следствие стандартная структура секвенциального операторного пространства $A$ минимальна	.
\end{proposition}
\begin{proof} Так как $A$ --- коммутативная $C^*$ алгебра, то по теореме Гельфанда-Наймарка 2.1.10 \cite{MurphCstarOpTh} мы можем предполаать, что $A=C_0(\Omega)$. Используя предложение \ref{PrCstarAlgSQ} для любых $n\in\mathbb{N}$ и $x\in X^{\wideparen{n}}$ имеем 
$$
\Vert\varphi^{\wideparen{n}}(x)\Vert_{\wideparen{n}}
=\Vert i_C(\varphi^{\wideparen{n}}(x))\Vert
=\sup\{\Vert i_C(\varphi^{\wideparen{n}}(x))(\omega)\Vert:\omega\in\Omega\}
=\sup\{\langle i_C(\varphi^{\wideparen{n}}(x))(\omega),\xi\rangle:\omega\in\Omega,\xi\in B_{\mathbb{C}^n}\}
$$
$$
=\sup\left\{\left|\sum_{i=1}^n \varphi(x_i)(\omega)\overline{\xi_i}\right|:\omega\in\Omega,\xi\in B_{\mathbb{C}^n}\right\}
=\sup\left\{\left| \varphi\left(\sum_{i=1}^n \overline{\xi_i} x_i\right)(\omega)\right|:\omega\in\Omega,\xi\in B_{\mathbb{C}^n}\right\}
$$
$$
=\sup\left\{\left\Vert \varphi\left(\sum_{i=1}^n \overline{\xi_i} x_i\right)\right\Vert:\xi\in B_{\mathbb{C}^n}\right\}
\leq\Vert\varphi\Vert\sup\left\{\left\Vert \sum_{i=1}^n \overline{\xi_i} x_i\right\Vert_{\wideparen{n}}:\xi\in B_{\mathbb{C}^n}\right\}
$$
$$
\leq\Vert\varphi\Vert\Vert x\Vert_{\wideparen{n}}\sup\{\Vert\operatorname{diag}_n(\overline{\xi_1},\ldots,\overline{\xi_n})\Vert:\xi\in B_{\mathbb{C}^n}\}
=\Vert\varphi\Vert\Vert x\Vert_{\wideparen{n}}\sup\left\{\max_{i\in\mathbb{N}_n}|\overline{\xi_i}|:\xi\in B_{\mathbb{C}^n}\right\}
\leq\Vert\varphi\Vert\Vert x\Vert_{\wideparen{n}}
$$
Следовательно, $\Vert\varphi\Vert_{sb}\leq\Vert\varphi\Vert$. Так как всегда выполнено $\Vert\varphi\Vert\leq\Vert\varphi\Vert_{sb}$, то мы получаем желаемое равенство. Так как еквенциальное операторное пространство $X$ произвольно, то из предложения \ref{PrCharMinSQ} мы видим, что $A$ имеет минимальную структуру.
\end{proof}

\begin{proposition}\label{PrMinIsSubspOfCommCstarAlg} Пусть $X$ --- секвенциальное операторное пространство, тогда $X$ минимально тогда и только тогда когда существует секвенциальная изометрия из $X$ в $C(\Omega)$ для некоторого компактного топологического пространства $\Omega$.
\end{proposition}
\begin{proof} 
Предположим, что $X$ имеет минимальную структуру. Рассмотрим естественную изометрию $i:X\to C(B_{X^*})$ (см. A1 \cite{DefFloTensNorOpId}). По предложению \ref{PrMinPreserveEmbedings} мы знаем, что $\min(i):\min(X^{\wideparen{1}})\to\min(C(B_{X^*})^{\wideparen{1}})$ секвенциально изометричен. Из предложения \ref{PrCommCstarAlgIsMin} известно, что $\min(C(B_{X^*})^{\wideparen{1}})=C(B_{X^*})$ и по предположению $\min(X^{\wideparen{1}})=X$, откуда мы получаем искоммую секвенциальную изометрию $\min(i):X\to C(B_{X^*})$.

Обратно, предположим, что нам дана секвенциальная изометрия $i:X\to C(\Omega)$. Так как $i^{\wideparen{1}}:X^{\wideparen{1}}\to C(\Omega)^{\wideparen{1}}$ --- изометрия, то по предложению  \ref{PrMinPreserveEmbedings} мы получаем секвенциальную изометрию $\min(i):\min(X^{\wideparen{1}})\to\min(C(\Omega)^{\wideparen{1}})$. Из предложения \ref{PrCommCstarAlgIsMin} следует, что $\min(C(\Omega)^{\wideparen{1}})=C(\Omega)$, поэтому у нас есть еще одна секвенциальная изометрия $\min(i):\min(X^{\wideparen{1}})\to C(\Omega)$. Так как $i=\min(i)$ как линейные операторы, то $X=\min(X^{\wideparen{1}})$ 
\end{proof}

\begin{proposition}\label{PrMaxPreserveQuotients} Пусть $\varphi:E\to F$ --- ограниченный линейный оператор между нормированными пространствами $E$ и $F$, тогда
\newline
1) если $\varphi$ $c$-топологически сюръективен, то $\max(\varphi)$ секвенциально $c$-топологически сюръективен
\newline
2) если $\varphi$ коизометричен, то $\max(\varphi)$ секвенциально коизометричен
\end{proposition}
\begin{proof} 1) Из леммы A.2.1 \cite{EROpSp} следует, что $\widehat{\varphi}:E/\operatorname{Ker}(\varphi)\to F$ является $c^{-1}$-топологически инъективным изоморфизмом нормированных пространств. Следовательно, он имеет ограниченный обратный линейный оператор $\psi:F\to E/\operatorname{Ker}(\varphi)$, причем $\Vert\psi\Vert\leq c$. По предложению \ref{PrCharMaxSQ} корректно определени секвенциально ограниченный оператор $\psi':\max(F)\to\max(E)/\operatorname{Ker}(\varphi):x\mapsto \psi(x)$, причем $\Vert\psi'\Vert_{sb}=\Vert\psi\Vert\leq c$. Из предложения \ref{PrFactorSQOp} мы имеем представление  $\max(\varphi)=\widehat{\max(\varphi)}\pi_{\operatorname{Ker}(\varphi),E}$, в котором $\widehat{\max(\varphi)}:E/\operatorname{Ker}(\varphi)\to F$ --- секвенциально ограниченный линейный оператор. Очевидно, $\widehat{\max(\varphi)}=\widehat{\varphi}$ и $\psi=\psi'$ как линейные опреаторы, поэтому $\widehat{\max(\varphi)}$ и $\psi'$ --- секвенциально ограниченные операторы обратные друг к другу. Теперь, для любого $n\in\mathbb{N}$ и  $y\in\max(F)^{\wideparen{n}}$ рассмотрим $x=(\psi')^{\wideparen{n}}(y)$, тогда $(\widehat{\max(\varphi)})^{\wideparen{n}}(x)=y$ и $\Vert x\Vert_{\wideparen{n}}=\Vert(\psi')^{\wideparen{n}}(y)\Vert_{\wideparen{n}}\leq\Vert(\psi')^{\wideparen{n}}\Vert\Vert y\Vert_{\wideparen{n}}\leq\Vert \psi'\Vert_{sb}\Vert y\Vert_{\wideparen{n}}\leq c\Vert y\Vert_{\wideparen{n}}$. Так как $n\in\mathbb{N}$ и $y\in \max(F)^{\wideparen{n}}$  произвольны, то $\widehat{\max(\varphi)}$ секвенциально $c$-топологически сюръективен. Поскольку $\pi_{\operatorname{Ker}(\varphi),E}$ секвенциально $1$-топологически сюръективен, то по предложению \ref{PrComposeSQTopInjSur} оператор $\max(\varphi)=\widehat{\max(\varphi)}\pi_{\operatorname{Ker}(\varphi),E}$ будет $c$-топологически сюръективным.

2) Из предыдущего пункта $\max(\varphi)$ $1$-топологически сюръективен. С другой стороны, по предложению \ref{PrCharMaxSQ} мы имеем $\Vert\max(\varphi)\Vert_{sb}=\Vert\varphi\Vert=1$. Следовательно, $\max(\varphi)$ секвенциально коизометричен.
\end{proof}

\begin{proposition}\label{PrMaxCommuteWithCoprod} Пусть $\{X_\lambda:\lambda\in\Lambda\}$ --- семейство максимальных секвенциальных операторных пространств, тогда $\bigoplus{}_1\{X_\lambda:\lambda\in\Lambda\}$ также максимально.
\end{proposition} 
\begin{proof}
Пусть $Y$ произвольное секвенциальное операторное пространство, тогда из предложений \ref{PrSQCoProdUnivProp}, \ref{PrCharMinSQ} и \ref{PrCharMaxSQ} мы имеем изометрические изоморфизмы
$$
\mathcal{SB}\left(\bigoplus{}_1^0\{X_\lambda:\lambda\in\Lambda\},Y\right)^{\wideparen{1}}
=\bigoplus{}_\infty\{\mathcal{SB}(X_\lambda,Y)^{\wideparen{1}}:\lambda\in\Lambda\}
=\bigoplus{}_\infty\{\mathcal{B}(X_\lambda^{\wideparen{1}},Y^{\wideparen{1}}):\lambda\in\Lambda\}
$$
$$
=\bigoplus{}_\infty\{\mathcal{SB}(X_\lambda,\min(Y^{\wideparen{1}}))^{\wideparen{1}}:\lambda\in\Lambda\}
=\mathcal{SB}\left(\bigoplus{}_1^0\{X_\lambda:\lambda\in\Lambda\},\min(Y^{\wideparen{1}})\right)^{\wideparen{1}}
$$
$$
=\mathcal{B}\left(\left(\bigoplus{}_1^0\{X_\lambda:\lambda\in\Lambda\}\right)^{\wideparen{1}},Y^{\wideparen{1}}\right)
$$
Поскольку $Y$ произвольно, из предложения \ref{PrCharMaxSQ} мы получаем, что $\bigoplus{}_1^0\{X_\lambda:\lambda\in\Lambda\}$ имеет максимальную структуру секвенциального операторного пространства.
\end{proof}

\begin{proposition}\label{Prl1IsMax} Пусть $\Lambda$ --- произвольное множество, тогда $l_1^0(\Lambda):=\bigoplus_1\{\mathbb{C}:\lambda\in\Lambda\}$ имеет структуру максимального операторного пространства.
\end{proposition}
\begin{proof} По предложению \ref{PrCHaveUniqueOSS} структура секвенциального операторного пространства $\mathbb{C}$ единственна и в частности максимальна. Теперь желаемый результат следует из предложения \ref{PrMaxCommuteWithCoprod}.
\end{proof}

\begin{proposition}\label{PrMaxIsQuotientOfl1} Пусть $X$ ---  секвенциальное операторное пространство, тогда $X$ максимально тогда и только тогда когда существует секвенциальная коизометрия из $l_1(\Lambda)$ на $X$ для некоторого множества  $\Lambda$.
\end{proposition}
\begin{proof} 
Предположим, что $X$ имеет максимальную структуру. Рассмотрим естественную коизометрию $\pi:l_1(B_X)\to X$ (см. A1 \cite{DefFloTensNorOpId}). По предложению \ref{PrMaxPreserveQuotients} мы знаем, что $\max(\pi):\max(l_1(B_X)^{\wideparen{1}})\to\max(X^{\wideparen{1}})$ секвенциально коизометричен. Из предложения \ref{Prl1IsMax} мы имеем $\max(l_1(B_X)^{\wideparen{1}})=l_1(B_X)$ и по предположению $\max(X^{\wideparen{1}})=X$, поэтому $\max(\pi):l_1(B_X)\to X$ и есть искомая секвенциальная коизометрия.

Обратно, предположим, что нам дана секвенциальная коизометрия $\pi:l_1(\Lambda)\to X$, тогда из предложению \ref{PrFactorSQOp} следует, что $X$ и $l_1(\Lambda)/\operatorname{Ker}(\pi)$ секвенциально изометрически изоморфны посредством $\widehat{\pi}$. Так как $\pi^{\wideparen{1}}:l_1(\Lambda)^{\wideparen{1}}\to X^{\wideparen{1}}$ также коизометричен, то по предложению \ref{PrMaxPreserveQuotients} мы имеем секвенциальную коизометрию $\max(\pi):\max(l_1(\Lambda)^{\wideparen{1}})\to \max(X^{\wideparen{1}})$. Из предложения \ref{Prl1IsMax} известно, что $\max(l_1(\Lambda)^{\wideparen{1}})=l_1(\Lambda)$, поэтому мы иеем еще одну секвенциальную коизометрию $\max(\pi):l_1(\Lambda)\to\max(X^{\wideparen{1}})$. Снова из предложения \ref{PrFactorSQOp} мы получаем, что  $\max(X^{\wideparen{1}})$ и $l_1(\Lambda)/\operatorname{Ker}(\pi)$ секвенциально изометрически изоморфны посредством $\widehat{\max(\pi)}$. Следовательно, $X=l_1(\Lambda)/\operatorname{Ker}(\pi)=\max(X^{\wideparen{1}})$.
\end{proof}

\begin{proposition}[\cite{LamOpFolgen}, 2.1.11]\label{PrDualityAndMinMax}
Пусть $E$ --- нормированное пространство, тогда тождественный оператор осущетсвляет изометрические изоморфизмы
$$
\max(E^*)=\min(E)^\triangle
\qquad
\min(E^*)=\max(E)^\triangle
$$
\end{proposition}

Аналогичные результаты верны и для категории $SQNor$, $SQBan$ и $SQBan_1$.







































\subsection{Тензорные произведения секвенциальных операторных пространств}

Естественно ожидать существования некоторого тензорного произведения линеаризующего секвенциально ограниченные биоператоры. 
\begin{definition}[\cite{LamOpFolgen}, 3.1.1]\label{DefSQMaxTenProd}
Пусть $X$ и $Y$ секвенциальные операторные пространства, тогда их максимальным тензорным произведением будем называть секвенциальное операторное пространство $X\otimes_{\mathrm{Max}}Y$ c системой норм 
$(\Vert\cdot\Vert_{(X\otimes_{\mathrm{Max}}Y)^{\wideparen{n}}})_{n\in\mathbb{N}}$ заданных равенствами
$$
\Vert u\Vert_{(X\otimes_{\mathrm{Max}}Y)^{\wideparen{n}}}
=\inf\left\{\Vert[\alpha_1,\ldots,\alpha_k]\Vert_{M_{n,kl^2}}\left(\sum\limits_{i=1}^k\Vert x_i\Vert_{X^{\wideparen{l}}}^2\Vert y_i\Vert_{Y^{\wideparen{l}}}^2 \right)^{1/2}:u=\sum\limits_{i=1}^k\alpha_i(x_i\otimes y_i)\right\}
$$
где $u\in (X\otimes_{\mathrm{Max}}Y)^{\wideparen{n}}$, $\alpha_1,\ldots,\alpha_k\in M_{n,l^2}$ и $x\in X^{\wideparen{l}}$, $y\in Y^{\wideparen{l}}$. Используя стандартную конструкцию пополнения 
для секвенциальных операторных пространств, мы получим пополненую версию данного тензорного произведения, которую будем обозначать $X\otimes^{\mathrm{Max}} Y$.
\end{definition}

В [\cite{LamOpFolgen} 3.1.2] доказано, что определенная выше норма это максимальная из кросснорм задающих на $X\otimes Y$ структуру секвенциального операторного пространства. Определённое таким образом 
тензорное произведение называется \textit{максимальным} и обозначается $X \otimes_{\mathrm{Max}} Y$. Максимальное тензорное произведение обладает свойством универсальности относительно класса 
секвенциально ограниченных биоператоров.

\begin{proposition}[\cite{LamOpFolgen}, 3.1.3, 3.1.4]\label{PrSQUnivPropMaxTenProd}
Пусть $X$, $Y$ и $Z$ --- секвенциальные операторные пространства, тогда 	имеют место секвенциально изометрические изоморфизмы
$$
\mathcal{SB}(X\otimes_{\mathrm{Max}}Y, Z)
=\mathcal{SB}(X\otimes^{\mathrm{Max}}Y, Z)
=\mathcal{SB}(X\times Y, Z)
=\mathcal{SB}(X,\mathcal{SB}(Y,Z))
=\mathcal{SB}(Y,\mathcal{SB}(X,Z))
$$
естественные по $X$, $Y$ и $Z$.
\end{proposition}

\begin{corollary}\label{CorSQUnivPropMaxTenProd}
Пусть $X$, $Y$ --- секвенциальные операторные пространства, тогда имеют место секвенциально изометрические изоморфизмы
$$
\mathcal{SB}(X^\triangle, Y)=\mathcal{SB}(X,Y^\triangle)=(X\otimes_{\operatorname{Max}} Y)^\triangle
$$
естественные по $X$ и $Y$. 
\end{corollary}







































\section{Общекатегорные понятия}

\subsection{Проективность и инъективность. Свобода и косвобода}

Теперь нам пригодятся некоторые определения из \cite{HelMetrFrQmod}. Пусть $\mathcal{K}$ - произвольная категория.
\begin{definition}[\cite{HelMetrFrQmod}, 2.1]\label{DefRigCat}
Пара ($\mathcal{K}, \square:\mathcal{K}\to\mathcal{L}$), где $\square$ --- верный ковариантный функтор, называется оснащенной категорией. Дуальной к оснащенной категории $(\mathcal{K}, \square)$ 
будем называть оснащенную категорию $(\mathcal{K}^{o},\square^{o}:\mathcal{K}^{o}\to\mathcal{L}^{o})$. 
\end{definition}
\begin{definition}[\cite{HelMetrFrQmod}, 2.1]\label{DefAdmMorph}
Морфизм $\tau$ из $\mathcal{K}$ называется $\square$-\textit{допустимым} эпиморфизмом (мономорфизмом), если $\square (\tau)$ - ретракция (коретракция) в $\mathcal{L}$.
\end{definition}
\begin{definition}[\cite{HelMetrFrQmod}, 2.2]\label{DefProjInj}
Объект $P \in \mathcal{K}$ ($I \in \mathcal{K}$) называется $\square$-\textit{проективным} ($\square$-\textit{инъективным}) относительно оснащённой категории $(\mathcal{K}, \square)$, если для 
всякой пары объектов $X,Y\in\mathcal{K}$, морфизма $\varphi : P \to X$ ($\varphi : X \to I$) и $\square$-допустимого эпиморфизма $\tau : Y \to X$ (мономорфизма $\tau : X \to Y$)  существует 
морфизм $\psi : P \to Y$ ($\psi : Y \to I$), делающий коммутативной диаграмму

$$
\xymatrix{
& {Y} \ar[d]_{\tau}\\
{P} \ar@{-->}[ur]^{\psi} \ar[r]^{\varphi} &{X}}
\qquad\qquad\quad
\xymatrix{
& {Y} \ar@{-->}[dl]_{\psi} \\
{I} &{X} \ar[l]_{\varphi} \ar[u]_{\tau}}
$$
\end{definition}
\begin{definition}[\cite{HelMetrFrQmod}, 2.10]\label{DefFrAndCoFr}
Объект $F \in \mathcal{K}$ называется $\square$-\textit{свободным} ($\square$-\textit{косвободным}) с \textit{базой} $M \in \mathcal{L}$, если существует морфизм $j : M \to \square(F)$ ($j : \square(F) \to M$), 
такой что для всякого объекта $X \in \mathcal{K}$ и морфизма $\varphi : M \to \square(X)$ ($\varphi : \square(X) \to M$) существует единственный морфизм $\psi : F \to X$ ($\psi : X \to F$), 
делающий коммутативной диаграмму
$$
\xymatrix{
{\square (F)} \ar@{-->}[dr]^{\square (\psi)} & \\
{M} \ar[u]^{j} \ar[r]^{\varphi} &{\square (X)}}
\qquad\qquad\quad
\xymatrix{
{\square (F)} \ar[d]_{j} & \\
{M} &{\square (X)} \ar[l]_\varphi \ar@{-->}[ul]_{\square(\psi)}
}
$$
\end{definition}
\begin{definition}\label{DefFrAndCoFrLove}
Оснащённая категория $(\mathcal{K},\square)$ называется \textit{свободолюбивой} (\textit{косвободолюбивой}), если всякий объект в $\mathcal{L}$ является базой некоторого $\square$-свободного ($\square$-косвободного) объекта.
\end{definition}

Следующие результаты не раз пригодятся нам в дальнейшем.

\begin{proposition}\label{PrUniqFr}
Пусть $M\in\mathcal{L}$ --- база $\square$-свободных ($\square$-косвободных) объектов $F_1$, $F_2$ в оснащенной категории $(\mathcal{K},\square)$, тогда $F_1$ и $F_2$ изоморфны.  
\end{proposition}
\begin{proof}
Рассмотрим категорию $\mathcal{K}_M$, в которой объекты --- это пары вида $(X,\varphi:M\to\square X)$, морфизмы из $(X_1,\varphi_1)$ в $(X_2,\varphi_2)$ --- это такие морфизмы $\psi$ в 
$\mathcal{K}$, что $\varphi_2=\square(\psi)\varphi_1$. Композиция морфизмов такая же, как и в $\mathcal{K}$. Очевидно, что что объект $F$ $\square$-свободен в $\mathcal{K}$ с базой $M$ и 
универсальной стрелкой $j$ тогда и только тогда, когда $(F,j)$ является инициальным объектом в $\mathcal{K}_M$. Вспомним, что в любой категории финальный объект единственен с точностью до 
изоморфизма. Осталось заметить, что изоморфизмы в $\mathcal{K}$ и $\mathcal{K}_M$ одни и те же. Доказательство для $\square$-косвободных объектов аналогично.
\end{proof}

\begin{proposition}\label{PrCompOfFrIsFr} 
Пусть $\square_{12}:\mathcal{K}_1\to\mathcal{K}_2$, $\square_{23}:\mathcal{K}_2\to\mathcal{K}_3$ --- верные функторы. Обозначим $\square_{13}=\square_{23}\square_{12}$. Пусть 
$F_1$ $\square_{12}$-свободен ($\square_{12}$-косвободен) с базой $F_2$ и универсальной стрелкой $j_{12}$ в оснащенной категории $(\mathcal{K}_1,\square_{12})$. Пусть $F_2$ $\square_{23}$-свободен 
($\square_{23}$-косвободен) с базой $F_3$ и универсальной стрелкой $j_{23}$ в оснащенной категории $(\mathcal{K}_2,\square_{23})$. Тогда $F_1$ является $\square_{13}$-свободным ($\square_{13}$-косвободным) 
объектом с базой $F_3$ и универсальной стрелкой $\square_{23}(j_{23})j_{12}$ в оснащенной категории $(\mathcal{K}_1,\square_{13})$. 
$$
\xymatrix{
& { \mathcal{K}_2} \ar[dr]^{\square_{23}} & \\
{\mathcal{K}_1} \ar[ur]^{\square_{12}} \ar[rr]_{\square_{13}} & & {\mathcal{K}_3}}
$$
Как следствие, если категории $(\mathcal{K}_1,\square_{12})$, $(\mathcal{K}_1,\square_{12})$ свободолюбивы (косвободолюбивы), то такова же и категория $(\mathcal{K}_1,\square_{13})$ 
\end{proposition}
\begin{proof}
Рассмотрим произвольный объект $X\in\mathcal{K}_1$ и морфизм $\varphi:F_3\to \square_{13}(X)$. Так как $F_2$ $\square_{23}$-свободен, то существует единственный морфизм $\psi:F_2\to \square_{12}(X)$, 
такой что $\varphi=\square_{23}(\psi)j_{23}$. Так как $F_1$ $\square_{12}$-свободен, то существует единственный $\chi:F_1\to X$, такой что $\psi=\square_{12}(\chi)j_{12}$.
$$
\xymatrix
{
{\square_{23}(\square_{12}(F_1)}) \ar[rr]^{\square_{23}(\square_{12}(\chi))}& & {\square_{23}(\square_{12}(X)}\\
{\square_{23}(F_2)} \ar[u]^{\square_{23}(j_{12})} \ar[urr]^{\square_{23}(\psi)} & &\\
{F_3} \ar[u]^{j_{23}} \ar[uurr]^{\varphi} & \\
} 
$$
Таким образом, $\varphi=\square_{23}(\psi)j_{23}=\square_{23}(\square_{12}(\chi))\square_{23}(j_{23})j_{12}=\square_{13}(\chi) j_{13}$, где $j_{13}=\square_{23}(j_{23})j_{12}$ --- универсальная стрелка. 
Так как $X$ и $\varphi$ произвольны, то $F_1$ является $\square_{13}$-свободным объектом с базой $F_3$. Для косвободных объектов доказательство аналогично.
\end{proof}

\begin{proposition}[\cite{HelMetrFrQmod}, 2.3]\label{PrRetractsProjInj} Пусть $(\mathcal{K},\square)$ --- оснащенная категория, и $P\in\mathcal{K}$ ($I\in\mathcal{K}$) является 
$\square$-проективным ($\square$-инъективным) объектом, тогда
\newline
1) если $\sigma:P\to Q$ ($\sigma:I\to J$) ретракция, то $Q$ $\square$-проективен ($J$ $\square$-инъективен)
\newline
2) если $\sigma:X\to P$ ($\sigma:X\to I$) является $\square$-допустимым эпиморфизмом (мономорфизмом), то $\sigma$ --- ретракция (коретракция)
\end{proposition}

\begin{proposition}[\cite{HelMetrFrQmod}, 2.11]\label{PrFrCoFrProjInjObjProp} Пусть $(\mathcal{K},\square)$ --- оснащенная категория, тогда
\newline
1) если $F\in\mathcal{K}$ является $\square$-свободным ($\square$-косвободным), то он $\square$-проективен ($\square$-инъективен)
\newline
2) если $X$ такой объект в $\mathcal{K}$, что $\square(X)$ является базой $\square$-свободного ($\square$-косвободного) объекта $F$, то существует $\square$-допустимый эпиморфизм (мономорфизм) из $F$ в $X$ (из $X$ в $F$).
\newline
3) если категория $\mathcal{K}$ свободолюбива (косвободолюбива), то объект $P\in\mathcal{K}$ ($I\in\mathcal{K}$) $\square$-проективен ($\square$-инъективен) тогда и только тогда когда он является ретрактом $\square$-свободного ($\square$-косвободного) объекта.
\end{proposition}

\begin{proposition}[\cite{HelMetrFrQmod}, 2.13]\label{PrCoprodFrIsFr} Пусть $(\mathcal{K},\square)$ --- оснащенная категория, и $\Lambda$ --- некоторое множество. Пусть для 
каждого $\lambda \in \Lambda$ объект $F_{\lambda} \in \mathcal{K}$ является $\square$-свободным ($\square$-косвободным) с базой $M_{\lambda} \in \mathcal{L}$. Пусть кроме того 
семейство $\{ F_\lambda:\lambda \in \Lambda\}$ обладает копроизведением (произведением) $F$, а семейство $\{ M_\lambda:\lambda \in \Lambda\}$ --- копроизведением (произведением) $M$. 
Тогда объект $F$ является $\square$-свободным ($\square$-косвободным) с базой $M$.
\end{proposition}

\begin{proposition}[\cite{HelMetrFrQmod}, 4.5]\label{PrFunctorMapFrToFr} Пусть даны оснащённые категории $(\mathcal{K}_1, \square_1: \mathcal{K}_1 \to \mathcal{L}_1)$ и $(\mathcal{K}_2, \square_2 : \mathcal{K}_2 \to \mathcal{L}_2)$,
а ковариантные функторы $\Phi : \mathcal{K}_1 \to \mathcal{K}_2$ и $\Psi : \mathcal{L}_1 \to \mathcal{L}_2$ таковы, что диаграмма
$$
\xymatrix{
{\mathcal{K}_1}\ar[d]_{\Phi}
\ar[rr]^{\square_1} & & {\mathcal{L}_1}\ar[d]^{\Psi}\\
{\mathcal{K}_2}\ar[rr]^{\square_2} & & {\mathcal{L}_2}}
$$
коммутативна. Предположим, что у $\Phi$ и $\Psi$ существуют левые (правые) сопряжённые функторы $\Phi^*$ и $\Psi^*$ (${}^*\Phi$ и ${}^*\Psi$) соответственно, а $F\in\mathcal{K}_2$ является 
$\square_2$-свободным ($\square_2$-косвободным) объектом с базой $M\in\mathcal{L}_2$.
Тогда $\Phi^*(F)\in\mathcal{K}_1$ (${}^*\Phi(F)\in\mathcal{K}_1$) является $\square_1$-свободным ($\square_1$-косвободным) объектом с базой $\Psi^*(M)\in\mathcal{L}_1$ (${}^*\Psi(M)\in\mathcal{L}_1$). 
\end{proposition}
























\subsection{Нормированные полулинейные пространства}

В дальнейшем нам понадобится следующее понятие.

\begin{definition}\label{DefSemiLinSp} Полулинейное пространство $V$ над полем $K$ - это упорядоченная тройка $(V, K, \cdot)$, где $V$ --- непустое множество, элементы которого называются векторами, $K$ --- поле, элементы которого называются скалярами, $\cdot : K \times V \to V$ --- операция умножения векторов на скаляры, удовлетворяющая следующим аксиомам:

1) для всех $x\in V$, $\alpha,\beta\in K$ выполнено $\alpha \cdot (\beta \cdot x) = (\alpha \beta) \cdot x $

2) для всех $x\in V$ выполнено $1_K \cdot x = x$

3) существует такой элемент $0 \in V$, что $0_K \cdot x = 0$ для всех $x\in V$.

Элемент $0\in V$ называется нулевым.
\end{definition}

Легко проверить, что элемент $0\in V$, единственен и что $\alpha \cdot 0 = 0$ для всех $\alpha\in K$.

\begin{example}\label{ExSemiLinModelSp}
Рассмотрим букет пространств $\bigvee\{K: \lambda \in\Lambda\}$, где все экземпляры поля $K$, пересекаются друг с другом по нулю, $\Lambda$ --- непустое множество, а умножение наследуется из поля. Легко проверить, что это полулинейное пространство, которое мы будем обозначать $K^{\Lambda}$. Под $K^{\varnothing}$ будем понимать полулинейное пространство, состоящее из единственного элемента $0$. 
\end{example}

\begin{definition}\label{DefSemiLinOp} Отображение $\varphi : V \to W$ между полулинейными пространствами $V$ и $W$ называется полулинейным оператором, если $\varphi(\alpha \cdot x) = \alpha \cdot \varphi(x)$ для всех $\alpha \in K$ и $x \in V$.
\end{definition}

Рассмотрим категорию $Lin_{0}^{K}$, объектами которой являются полулинейные пространства над полем $K$, а морфизмами --- полулинейные операторы. Нетрудно получить полную классификацию объектов этой категории.

\begin{proposition}\label{PrSemiLinSpDesc}
Всякое полулинейное пространство изоморфно в $Lin_{0}^{K}$ пространству $K^{\Lambda}$ для некоторого множества $\Lambda$.
\end{proposition}
\begin{proof} Будем говорить что два вектора $x,y\in V$ эквивалентны если $x=\alpha y$ для некторого $\alpha\in K\setminus\{0\}$. Это отношение $\sim$ есть отношение эквивалентности. Пусть $\{x_\lambda:\lambda\in \Lambda\}$ есть семейство представителей каждого класса эквивалентности кроме класса эквивалентности нуля. Тогда полулинейный оператор $\varphi: K^\Lambda\to V: z_\lambda\mapsto z_\lambda x_\lambda$ есть изоморфизм в $Lin_0^K$
\end{proof}

\begin{definition}\label{DefSemiLinNorSp} Полулинейным нормированным пространством над нормированным полем $K$ называется пара $(E, \Vert \cdot \Vert)$, где $E$ --- полулинейное пространство на полем $K$ и $ \Vert \cdot \Vert : E \to \mathbb{R}_+$ --- отображение, которое мы будем называть нормой, удовлетворяющее следующим аксиомам:

1) если $x\in E$ и $\Vert x \Vert = 0$, то $x = 0$;

2) для всех $x\in E$ и $\alpha\in K$ выполнено $\Vert \alpha \cdot x \Vert = | \alpha| \Vert x \Vert$.
\end{definition}

\begin{example}\label{ExSemiLinNorModelSp}
Для нормированного поля $K$ введём норму на $K^{\Lambda}$, положив по определению $\Vert z_\lambda\Vert:=|z_\lambda|_K$ для каждого $z_\lambda\in K^\Lambda$. 
\end{example}

\begin{definition}\label{DefSemiLinBndOp} Полулинейный оператор $\varphi : E \to F$ между полулинейными нормированными пространствами $E$ и $F$ называется ограниченным, если $\Vert \varphi (x)\Vert \leq C \Vert x \Vert$ для некторого $C\in\mathbb{R}_+$. Инфимум таких чисел мы будем называть нормой полулинейного оператора $\varphi$ и будем обозначать $\Vert\varphi\Vert$.
\end{definition}

Рассмотрим теперь категорию $Nor_0^K$, объектами которой являются полулинейные  нормированные пространства, а морфизмами --- ограниченные полулинейные операторы. Нетрудно получить полную классификацию объектов и этой категории.

\begin{proposition}\label{PrSemiLinNorSpDesc} Всякое полулинейное нормированное пространство в $Nor_0^K$ изоморфно пространству $K^{\Lambda}$ для некоторого множества $\Lambda$.
\end{proposition}

\begin{proof} Аналогично предложению \ref{PrSemiLinSpDesc} построим отношение эквивалентности $\sim$ и рассмотрим множество $\{x_\lambda:\lambda\in\Lambda\}$ представителей классов эквивалентности, кроме класса эквивалентности нуля. Зафиксируем некоторое $\alpha\in K$ такое, что $0<|\alpha|<1$. Для каждого $\lambda\in\Lambda$ мы можем найти $m_\lambda\in\mathbb{Z}$ такое, что $|\alpha|^{-m_\lambda}\leq\Vert x_\lambda\Vert< |\alpha|^{-m_\lambda+1}$. Рассмотрим $y_\lambda=\alpha^{m_\lambda} x_\lambda$, тогда $1\leq \Vert y_\lambda\Vert<|\alpha|$. Теперь легко видеть, что полулинейный оператор $\varphi: K^\Lambda\to E: z_\lambda\mapsto z_\lambda y_\lambda$ есть изоморфизм в $Nor_0^K$
\end{proof}

Далее, через $Nor_0$ мы будем обозначать категорию $Nor_0^\mathbb{C}$.






































\subsection{Примеры оснащенных категорий}

Рассмотрим несколько примеров. Для простоты будем иметь дело с нормированными пространствами, на нормированные модули все перечисленные ниже результаты легко переносятся.

\begin{example}[Метрическая свобода, \cite{HelMetrFrQmod}]\label{ExMetrFr}
Пусть $\mathcal{K} = Nor_1$, $\mathcal{L} = Set$. Пусть функтор $\square$ отправляет нормированное пространство в свой замкнутый единичный шар, а морфизмы как как отображения множеств оставляет 
без изменения. В этом случае $\square$-допустимые эпиморфизмы --- строгие коизометрии, $\square$-cвободный объект с одноточечной базой --- это $\mathbb{C}$. Поэтому из предложения \ref{PrCoprodFrIsFr} 
мгновенно получаем, что $\square$-свободный объект с базой $\Lambda$ это в точности $l_1^0(\Lambda)$.
\end{example}

\begin{example}[Топологическая свобода]\label{ExTopFr}
Пусть $\mathcal{K} = Nor$, $\mathcal{L} = Nor_0$. Пусть функтор $\square$ отправляет нормированное пространство в полулинейное нормированное пространство с той же нормой, а морфизмы как отображения множеств оставляет без изменения. В этом случае $\square$-допустимые эпиморфизмы --- топологически сюрьективные операторы, $\square$-cвободные объекты с базой $\mathbb{C}^\Lambda$ это в точности $l_1^0(\Lambda)$.
\end{example}

\begin{example}[Метрическая косвобода, \cite{HelMetrFrQmod}]\label{ExMetrCoFr}
Пусть $\mathcal{K} = Nor_1$, $\mathcal{L} = Set^0$. Пусть функтор $\square$ отправляет нормированное пространство $X$ в единичный шар пространства $X^*$, а морфизмы переводит в биограничения 
сопряженного оператора на единичные шары области и кообласти. В этом случае $\square$-допустимые эпиморфизмы --- изометрии, $\square$-косвободные объекты с базой $\Lambda$ легко получаются из примера \ref{ExMetrFr} 
и предложения \ref{PrCoprodFrIsFr}, это в точности $l_\infty(\Lambda)$.
\end{example}

\begin{example}[Топологическая косвобода, \cite{ShtTopFr}]\label{ExTopCoFr}
Пусть $\mathcal{K} = Nor$, $\mathcal{L} = Nor_0^o$. Пусть функтор $\square$ отправляет нормированное пространство в полулинейное нормированное пространство своего сопряженного пространства с той же самой нормой, а морфизму сопоставляет сопряженный оператор. В этом случае $\square$-допустимые 
эпиморфизмы --- топологически инъективные операторы, $\square$-косвободные объекты с базой $\mathbb{C}^\Lambda$ легко получаются из примера \ref{ExTopFr} и предложения \ref{PrCoprodFrIsFr}, это в точности $l_\infty(\Lambda)$. 
\end{example}

Все приведённые выше примеры имеют свои очевидные банаховы аналоги, получающиеся пополнением указанных свободных и косвободных объектов. Кроме того, все они имеют и квантовые версии: в роли свободного 
объекта для одноточечного множества здесь вместо $\mathbb{C}$ выступает пространство $\mathcal{N}_{\infty} =  \bigoplus{}_1^0\{\mathcal{N}(\mathbb{C}^n):n\in\mathbb{N}\}$([\cite{HelMetrFrQmod}, 5.9], 
см. также \cite{ShtTopFr}). Нашей ближайшей целью будет показать, что в случае секвенциальных операторных пространств эту роль выполняет пространство $t_2^{\infty} :=  \bigoplus_1^0\{t_2^n:n\in\mathbb{N}\}$.  






































\section{Свободные секвенциальные операторные пространства}

\subsection{Метрическая свобода}

Начнём с рассмотрения метрической версии свободы для секвенциальных операторных пространств. Рассмотрим функтор 
$$
\begin{aligned}
\square_{sqMet} : SQNor_1 \to Set : X&\mapsto\prod\left\{ B_{X^{\wideparen{n}}}:n \in \mathbb{N}\right\}\\
\varphi&\mapsto \prod\left\{\varphi^{\wideparen{n}}|_{B_{X^{\wideparen{n}}}}^{B_{Y^{\wideparen{n}}}}:n\in\mathbb{N}\right\}
\end{aligned}
$$
отправляющий  секвенциальное операторное пространство $X$ в декартово произведение единичных шаров каждого из пространств $X^{\wideparen{n}}$. 

\begin{proposition}\label{PrDecsMetrAdmEpiMorph}
$\square_{sqMet}$-допустимыми эпиморфизмами являются в точности секвенциально строго коизометрические операторы.
\end{proposition}
\begin{proof}
Морфизм $\varphi$ является $\square_{sqMet}$-допустмым эпиморфизмом если $\square_{sqMet}(\varphi)$ обратим справа как морфизм в $Set$. Это равносильно тому что $\square_{sqMet}(\varphi)$ сюръективно, 
что эквивалентно  сюръективности $\varphi^{\wideparen{n}}|_{B_{X^{\wideparen{n}}}}^{B_{Y^{\wideparen{n}}}}$ для всех $n\in\mathbb{N}$. Последнее означает что $\varphi^{\wideparen{n}}$ строго коизометрично 
для каждого $n\in\mathbb{N}$. Это означает что $\varphi^{\wideparen{n}}$ секвенциально строго коизометричен.
\end{proof}


Обозначим через $I_n$ элемент из $(t_2^n)^{\wideparen{n}} = \mathcal{B}(l_2^n, l_2^n)$, соответствующий тождественному оператору.

\begin{proposition}\label{PrMetrFrLem} Пусть $X$ - произвольное секвенциальное операторное пространство и $x \in B_{X^{\wideparen{n}}}$. Тогда существует единственный секвенциально сжимающий оператор 
$\psi_n \in \mathcal{SB}(t_2^n, X)$, такой что $\psi_n^{\wideparen{n}}(I_n) = x$.
\end{proposition}
\begin{proof}
Итак, $I_n = (e_i)_{i\in\mathbb{N}_n}$, где $e_i$ - $i$-й орт подлежащего пространства $t_2^n$. Ясно, что есть только один линейный оператор $\psi_n$, удовлетворяющий условиям $\psi_n(e_i) = x_i$, $i\in\mathbb{N}_n$. 
Осталось проверить, что $\psi_n$ является секвенциально сжимающим. Итак, пусть $k \in \mathbb{N}$ и $y \in B_{(t_2^n)^{\wideparen{k}}} $, тогда $y_i = \sum\limits_{j = 1}^n \alpha_{ij}e_j$, $i\in\mathbb{N}_k$ 
для некоторой матрицы $\alpha\in M_{k,n}$. Тогда 
$$
\Vert\psi_n^{\wideparen{k}}(y)\Vert_{\wideparen{k}}
=\left\Vert\left(\psi_n(y_i)\right)_{i\in\mathbb{N}_k}\right\Vert_{\wideparen{k}}
=\left\Vert\left(\sum\limits_{j=1}^n\alpha_{ij}\psi_n(e_j)\right)_{i\in\mathbb{N}_k}\right\Vert_{\wideparen{k}}
=\left\Vert\left(\sum\limits_{j=1}^n\alpha_{ij}x_j\right)_{i\in\mathbb{N}_k}\right\Vert_{\wideparen{k}}
$$
$$
=\Vert\alpha x\Vert_{\wideparen{k}}
\leq\Vert\alpha\Vert\Vert x\Vert_{\wideparen{n}}
=\Vert y\Vert_{(t_2^n)^{\wideparen{k}}}\Vert x\Vert_{\wideparen{n}}\leq 1
$$
Предложение доказано.
\end{proof}

\begin{proposition}\label{PrOnePtMetrFr} Метрически свободным секвенциальным операторным пространством с базой из одноточечного множества является пространство $t_2^{\infty} := \bigoplus_1^0 \{t_2^n: n \in \mathbb{N}\}$.
\end{proposition}
\begin{proof}
Универсальную стрелку определим следующим образом $j:\{\lambda\}\to t_2^\infty:\lambda\mapsto(I_1,I_2,\ldots,I_n,\ldots)$. Пусть $X$ - произвольное секвенциальное операторное пространство, и 
$\varphi:\{\lambda\}\to \prod_{n \in \mathbb{N}} B_{X^{\wideparen{n}}}$. Обозначим $x=\varphi(\lambda)$. Тогда из предложения \ref{PrMetrFrLem} и свойств копроизведения ясно, что существует 
единственный секвенциально сжимающий морфизм $\psi=\bigoplus_1^0\{\psi_n:n\in\mathbb{N}\}\in$ \\$  \mathcal{SB}\left(\bigoplus_1^0\{ t_2^n:n\in\mathbb{N}\}, X\right)$, такой что $\psi^{\wideparen{n}}(i_n(I_n)) = x$, 
для всех $n \in \mathbb{N}$. Здесь $i_n:t_2^n\to t_2^\infty$ --- стандартное вложение.
$$
\xymatrix{
{\square_{sqMet} (t_2^\infty)} \ar@{-->}[dr]^{\square_{sqMet} (\psi)} & \\
{\{\lambda\}} \ar[u]^{j} \ar[r]^{\varphi} &{\square_{sqMet} (X)}}
$$
В этом случае $\varphi=\square_{sqMet}(\psi) j$. Так как $X$ и $\varphi$ произвольны то $t_2^\infty$ метрически свободен и имеет одноточечную базу. 
\end{proof}

Итак, теперь мы готовы сформулировать итоговый результат.

\begin{theorem}\label{ThMetrFrDesc} Метрически свободным секвенциальным операторным пространством с базой $\Lambda$ является, с точностью до секвенциального изометрического изоморфизма, 
$\bigoplus{}_1^0$-сумма копий пространства $t_2^{\infty}$, заиндексированных элементами множества $\Lambda$. 
\end{theorem}
\begin{proof}
Результат следует из предложений \ref{PrCoprodFrIsFr} и \ref{PrOnePtMetrFr}
\end{proof}

\begin{corollary}\label{CorSQSpaceIsImgMetrAdmEpiMorph}
Всякое секвенциальное операторное пространство является образом секвенциально строго коизометрического оператора из пространства $\bigoplus_1^0\{t_2^\infty:\lambda\in\Lambda\}$ для некоторого множества $\Lambda$.
\end{corollary}
\begin{proof}
Из теоремы \ref{ThMetrFrDesc} следует что оснащенная категория $(SQNor_1,\square_{sqMet})$ свободолюбива. Теперь желаемый результат следует из предложений \ref{PrFrCoFrProjInjObjProp} и \ref{PrDecsMetrAdmEpiMorph}
\end{proof}

Аналогичные утверждения справедливы и в банаховом случае (с заменой $\bigoplus{}_1^0$ суммы на $\bigoplus{}_1$ сумму).








































\subsection{Топологическая свобода}

Перейдём теперь к рассмотрению секвенциальной операторной версии  топологической свободы. Рассмотрим функтор 
$$
\begin{aligned}
\square_{sqTop} : SQNor \to Nor_0: X &\mapsto \bigoplus{}_\infty \{X^{\wideparen{n}} : n \in \mathbb{N}\}\\
\varphi&\mapsto\bigoplus{}_\infty\{\varphi^{\wideparen{n}}:n\in\mathbb{N}\},\\
\end{aligned}
$$ 
то есть секвенциальное операторное пространство $X$ отображается в $\bigoplus{}_\infty$-сумму своих размножений без аддитивной структуры.

\begin{proposition}\label{PrCTopSurIsRetrInNor0} Пусть $\varphi:X\to Y$ --- ограниченный оператор между нормиированными пространствами $X$ и $Y$, тогда он $c$-топологически сюръективен тогда и только тогда когда существует ограниченный полулинейный оператор $\rho:Y\to X$ такой что $\Vert\rho\Vert\leq c$ и $\varphi\rho=1_Y$.
\end{proposition}
\begin{proof} Допустим, что $\varphi$ $c$-топологически сюръективен. Расссмотрим отношение $\sim$ на $S_Y$ определенное следующим образом: $e_1\sim e_2$ тогда и только тогда когда существует $\alpha\in\mathbb{T}$ такое, что $e_1=\alpha e_2$. Очевидно, $\sim$ есть отношение эквивалентности, поэтому рассмотрим множество ненулевых представителей классов эквивалентностей, которое обозначим $\{r_\lambda:\lambda\in\Lambda\}$. По построению, для каждого $e\in S_Y$ сущетсвует единственные $\alpha(e)\in\mathbb{T}$ и $\lambda(e)\in\Lambda$ такие, что $e=\alpha(e)r_{\lambda(e)}$. Ясно, что для любых $z\in\mathbb{T}$ и $e\in S_Y$ выполнено $\alpha(ze)=z\alpha(e)$ и $\lambda(ze)=\lambda(e)$. Так как $\varphi$ $c$-топологически сюръективен, то, в частности, для каждого $\lambda\in\Lambda$ существует $x(\lambda)\in X$ такой что $\Vert x(\lambda)\Vert\leq c\Vert r_\lambda\Vert$ и $\varphi(x(\lambda))=r_\lambda$. Рассмотрим, отображение $\tilde{\rho}:S_Y\to X:e\mapsto \alpha(e)x(\lambda(e))$. Легко видеть, что для всех $z\in\mathbb{T}$ и $e\in S_Y$ выполнено $\tilde{\rho}(z e)=z\tilde{\rho}(e)$, $\Vert\tilde{\rho}(e)\Vert\leq C$ и $\varphi(\tilde{\rho}(e))=e$. Теперь рассмотрим отображение $\rho:Y\to X: y\mapsto \Vert y\Vert\tilde{\rho}(\Vert y\Vert^{-1} y)$ и $\rho(0)=0$. Используя свойства $\tilde{\rho}$ легко проверить, что $\rho$ --- полулинейный оператор такой, что $\Vert\rho\Vert\leq C$ и $\varphi\rho=1_Y$.

Обратно, допустим, что существует ограниченный полулинейный оператор $\rho:Y\to X$ такой, что $\Vert\rho\Vert\leq c$ и $\varphi\rho=1_Y$. Возьмем произвольный $y\in Y$ и рассмотрим $x=\rho(y)$, тогда $\Vert x\Vert\leq C\Vert y\Vert$ и $\varphi(x)=y$. Следовательно $\varphi$ $c$-топологически сюръективен.
\end{proof}

\begin{proposition}\label{PrDecsTopAdmEpiMorph} $\square_{sqTop}$-допустимыми эпиморфизмами являются в точности секвенциальные топологически сюрьективные операторы.
\end{proposition}
\begin{proof}
Для произвольного секвенциального операторного пространства $Z$ через $i_n^Z:Z^{\wideparen{n}}\to\square_{sqTop}(Z)$ обозначим стандартное вложение, а через $p_n^Z:\square_{sqTop}(Z)\to Z^{\wideparen{n}}$ обозначим 
стандартную проекцию. Допустим что $\varphi:X\to Y$ $c$-секвенциально топологически сюръективен. Фиксируем $n\in\mathbb{N}$, тогда по предложению \ref{PrCTopSurIsRetrInNor0} существует ограниченный полулинейный оператор $\rho^n$ такой, что $\varphi^{\wideparen{n}}\rho^n=1_{Y^{\wideparen{n}}}$ и $\Vert\rho^n\Vert\leq c$. Рассмотрим отображение 
$ \rho=\bigoplus{}_\infty\{\rho^n:n\in\mathbb{N}\}$. Для любого $y\in \square_{sqTop}(Y)$ имеем 
$$
\Vert \rho(y)\Vert=\sup\{\Vert\rho^n(p_n^Y(y))\Vert_{\wideparen{n}}: n\in\mathbb{N}\}\leq
c\sup\{\Vert p_n^Y(y)\Vert_{\wideparen{n}}: n\in\mathbb{N}\}=c\Vert y\Vert
$$
следовательно $\rho$ --- полулинейный ограниченый оператор. Более того, $\square_{sqTop}(\varphi)\rho=1_{\square_{sqTop}(Y)}$, значит $\varphi$ $\square_{sqTop}$-допустимый эпиморфизм. Обратно, если 
$\varphi$ $\square_{sqTop}$-допустимый эпиморфизм, то существует ограниченный правый обратный полулинейный оператор  $\rho$ к $\square_{sqTop}(\varphi)$. Тогда для 
любого $y\in Y^{\wideparen{n}}$ выполнено $\square_{sqTop}(\varphi)\rho(i_n^Y(y))=i_n^Y(y)$. В частности $\varphi^{\wideparen{n}}(p_n^X(\rho(i_n^Y(y))))=y$. Положим $x=p_n^X(\rho(i_n^Y(y)))$ и 
$c=\Vert\rho\Vert$, тогда $\varphi^{\wideparen{n}}(x)=y$ и $\Vert x\Vert_{\wideparen{n}}\leq\Vert\rho(i_n^Y(y))\Vert\leq c\Vert i_n^Y(y)\Vert=c\Vert y\Vert_{\wideparen{n}}$. Следовательно, 
$\varphi$  секвенциально топологически сюръективен.
\end{proof}

Сформулируем и докажем основное утверждение раздела.

\begin{proposition}\label{PrMetrFrIsTopFr} Пусть $F$ -  секвенциальное метрически свободное пространство с базой $\Lambda$. Тогда $F$ является секвенциальным  операторным топологически свободным с базой $\mathbb{C}^{\Lambda}$.
\end{proposition}
\begin{proof} Пусть $j':\Lambda\to \square_{sqMet}(F)$ --- универсальная стрелка в диаграмме для секвенциальной метрической свободы. Определим полулинейный ограниченный оператор $j: \mathbb{C}^{\Lambda} \to \square_{sqTop}(F): z_\lambda\mapsto z_\lambda j(\lambda)$. Рассмотрим произвольный ораниченный полулинейный оператор $\varphi : \mathbb{C}^{\Lambda} \to \square_{sqTop}(X)$, где $X$ --- произвольное секвенциальное операторное пространство.  Тогда для $\varphi':=\Vert \varphi \Vert_{sb}^{-1}\varphi $ существует единственный морфизм $\psi^{'}$, такой что $\varphi'=\square_{sqMet}(\psi')j$. Теперь, легко видеть что для морфизма $\psi:=\Vert \varphi \Vert_{sb} \psi^{'}$ диаграмма
$$
\xymatrix{
{\square_{sqTop}(F)}\ar@{-->}[dr]^{\square_{sqTop}(\psi)} & \\
{\mathbb{C}^{\Lambda}}\ar[u]_{j}\ar[r]_{\varphi}  &{\square_{sqTop}(X)} }
$$
коммутативна.
	
Единственность $\psi$ доказывается следующим образом. Пусть для диаграммы выше есть два различных подходящих морфизма $\psi_1$ и $\psi_2$. Обозначим $C=\max( \Vert \varphi\Vert_{sb}, \Vert \psi_1 \Vert_{sb}, \Vert \psi_2\Vert_{sb})$, тогда ясно что морфизмы $C^{-1}\psi_1$ и $C^{-1}\psi_2$ подходят для следующей диаграммы, соответствующей секвенциальной метрической проективности:
$$
\xymatrix{
{\square_{sqMet}(F)}\ar@{-->}[dr]^{?} & \\
{\mathbb{C}^\Lambda}\ar[u]_{j'}\ar[r]_{ C^{-1}\varphi'}  &{\square_{sqMet}(X)} }
$$
Это противоречит единственности морфизма $\psi'$, значит $\psi$ единственен.
\end{proof}

Как следствие мы получаем описание топологически свободных секвенциальных операторных пространств.

\begin{theorem}\label{ThTopFrDesc} 
Секвенциальное операторное пространство является топологически свободным тогда и только тогда, когда оно секвенциально топологически изоморфно $\bigoplus{}_1^0$-сумме пространств $t_2^\infty$, заиндексированных 
элементами некоторого множества $\Lambda$.
\end{theorem}

\begin{corollary}\label{CorSQSpaceIsImgTopAdmEpiMorph}
Всякое секвенциальное операторное пространство является образом секвенциально топологически сюръективного оператора из пространства $\bigoplus_1^0\{t_2^\infty:\lambda\in\Lambda\}$ для некоторого множества $\Lambda$.
\end{corollary}
\begin{proof}
Из теоремы \ref{ThMetrFrDesc} следует что оснащенная категория $(SQNor,\square_{sqTop})$ свободолюбива. Теперь желаемый результат следует из предложений \ref{PrFrCoFrProjInjObjProp} и \ref{PrDecsTopAdmEpiMorph}
\end{proof}


Аналогичное утверждение справедливо и в банаховом случае с заменой $\bigoplus{}_1^0$ суммы на $\bigoplus{}_1$ сумму.







































\subsection{Псевдотопологическая свобода и проективность}

Может возникнуть вопрос, насколько естественным в определении секвенциально топологически сюрьективных операторов является требование существования общей константы? Действительно, более 
естественным было бы определение, в котором бы требовалась лишь топологическая сюръективность размножений оператора. Далее мы увидим, что такой класс допустимых морфизмов не дает дает богатой 
гомологической теории.

Тип проективности, описываемый такими секвенциально ограниченными операторами, назовем псевдотопологической. Рассмотрим функтор
$$
\begin{aligned}
\square_{sqpTop} : SQNor \to Nor_0: X &\mapsto X^{\wideparen{1}}\\
\varphi&\mapsto\varphi\\
\end{aligned}
$$
который секвенциальному операторному пространству сопоставляет полулинейное нормированное пространство, построенное по $X^{\wideparen{1}}$, а каждому морфизму сопоставляет такое же теоретико- множественное отображение.

\begin{definition}\label{DefPsSQTopSurjOp} Секвециально ограниченный оператор $\varphi:X\to Y$ называется псевдосеквенциально топологически сюръективным, только если для любого номера $n\in\mathbb{N}$ 
существует $c_n>0$ и для любого $y\in Y^{\wideparen{n}}$ существует $x\in X^{\wideparen{n}}$ такой что $\varphi^{\wideparen{n}}(x)=y$ и $\Vert x\Vert_{\wideparen{n}}\leq c_n\Vert y\Vert_{\wideparen{n}}$ 
\end{definition}

\begin{proposition}\label{PrDecsPsTopAdmEpiMorph} Пусть $\varphi:X\to Y$ секвенциально ограниченный оператор между секвенциальными операторными пространствами, тогда следующие условия эквивалентны
\newline
1) $\varphi$ $\square_{sqpTop}$-допустимый эпиморфизм
\newline
2) $\varphi$ псевдосеквенциально топологически сюръективен
\newline
3) $\varphi^{\wideparen{1}}$ топологически сюръективен 
\end{proposition}
\begin{proof}
$1)\implies 2)$ Пусть $\varphi$ является $\square_{sqpTop}$-допустимым эпиморфизмом, тогда для некоторого $c_1>0$ и любого $y\in Y$ существует $x\in X$, такой что $\varphi(x)=y$ и $\Vert x\Vert\leq c_1\Vert y\Vert$. Пусть $n\in\mathbb{N}$ и  $y\in Y^{\wideparen{n}}$, тогда рассмотрим $x\in X^{\wideparen{n}}$, такой 
что $\varphi(x_i)=y_i$ и $\Vert x_i\Vert\leq c_1\Vert y_i\Vert$ для всех $i\in\mathbb{N}_n$. Пусть $e_i\in M_{1,n}$ -  матрица-строка из нулей и единицы на $i$-ом месте, тогда 
$$
\Vert x\Vert_{\wideparen{n}}
\leq  \left(\sum\limits_{i=1}^n\Vert x_i\Vert_{\wideparen{1}}^2\right)^{1/2}
\leq  \left(\sum\limits_{i=1}^n c_1^2\Vert y_i\Vert_{\wideparen{1}}^2\right)^{1/2}
\leq c_1\left(\sum\limits_{i=1}^n\Vert e_i y\Vert_{\wideparen{n}}^2\right)^{1/2}
$$
$$
\leq c_1\left(\sum\limits_{i=1}^n\Vert e_i\Vert^2 \Vert y\Vert_{\wideparen{n}}^2\right)^{1/2}
=c_1n^{1/2}\Vert y\Vert_{\wideparen{n}}
$$
Очевидно также, что $\varphi(x)=y$. Следовательно, $\varphi$ псевдотопологически сюръективен.
\newline
$2)\implies 3)$ Очевидно.
\newline
$3)\implies 1)$ По предложению \ref{PrCTopSurIsRetrInNor0} существует ограниченный полулинейный оператор $\rho$ такой, что $\varphi\rho=1_Y$. Это значит, что $\square_{sqpTop}(\varphi)$ имеет правое обратное ограниченное отображение, т.е. $\varphi$ является $\square_{sqpTop}$-допустимым эпиморфизмом. 
\end{proof}

Рассмотрим функторы
$$
\begin{aligned}
\square_{sqRel} : SQNor \to Nor: X &\mapsto X^{\wideparen{1}}\\
\varphi&\mapsto\varphi\\
\end{aligned}
\qquad\qquad
\begin{aligned}
\square_{norTop} : Nor \to Nor_0: X &\mapsto X\\
\varphi&\mapsto\varphi\\
\end{aligned}
$$
Отметим очевидный факт $\square_{sqpTop}=\square_{norTop}\square_{sqRel}$.

\begin{proposition}\label{PrSQRelChar}
В оснащенной категории $(SQNor,\square_{sqRel})$
\newline
1) $\square_{sqRel}$-свободными объектами являются секвенциальные операторные пространства, секвенциально топологически изоморфные $\max(E)$ для некоторого нормированного пространства $E$. Данная категория свободолюбива. 
\newline
2) всякий ретракт $\square_{sqRel}$-свободного объекта имеет максимальную структуру секвенциального операторного пространства.
\newline
3) все $\square_{sqRel}$-проективные объекты $\square_{sqRel}$-свободны.
\end{proposition}
\begin{proof}
1) Пусть $E\in Nor$. Покажем что $\max(E)$ --- $\square_{sqRel}$-свободный объект с базой $E$. Универсальную стрелку определим следующим образом $j:E\to\square_{sqRel}(\max(E)):x\mapsto x$. Пусть $X$ 
произвольное секвенциальное операторное пространство и $\varphi:E\to\square_{sqRel}(X)$ произвольный ограниченный оператор. Рассмотрим линейный оператор $\psi: \max(E)\to X:x\mapsto\varphi(x)$. Из предложения \ref{PrCharMaxSQ} 
следует, что $\psi$ секвенциально ограничен. Легко видеть, что $\varphi=\square_{sqRel}(\psi)j$. Так как $X$ и $\varphi$ произвольны то $\max(E)$ $\square_{sqRel}$-свободный объект. Из предложения \ref{PrUniqFr} следует, что все $\square_{sqRel}$-свободные объекты секвенциально топологически изоморфны 
$\max(E)$. Так как $E$ --- произвольное нормированное пространство, то оснащенная категория $(SQNor,\square_{sqRel})$ свободолюбива.
\newline
2) Пусть $\sigma:\max(E)\to X$ ретракция в $SQNor$. Тода $\sigma$ топологически сюръективен и тогда по предложению \ref{PrMaxPreserveQuotients} пространство $X$ имеет максимальную структуру.
\newline
3) Пусть $P$ $\square_{sqRel}$-проективен, тогда из предложения \ref{PrFrCoFrProjInjObjProp} следует, что он является ретрактом $\square_{sqRel}$-свободного обекта, а из пункта 2) --- что $P$ имеет 
максимальную структуру секвенциального операторного пространства, т. е. $P=\max(\square_{sqRel}(P))$. По пункту 1) получаем, что $P$ $\square_{sqRel}$-свободен.  
\end{proof}

\begin{proposition}\label{PrNorTopChar}
В оснащенной категории $(Nor, \square_{norTop})$
\newline
1) $\square_{norTop}$-допустимыми эпиморфизмами являются в точности топологически сюръективные операторы
\newline
2) $\square_{norTop}$-свободными являются пространства, топологически изоморфные $l_1^0(\Lambda)$ с базой $\mathbb{C}^\Lambda$.
\newline
3) $\square_{norTop}$-проективными являются пространства, топологически изоморфные $l_1^0(\Lambda)$ для некоторого множества $\Lambda$. 
\end{proposition}
\begin{proof}
1) Следует из предложения \ref{PrCTopSurIsRetrInNor0}.

2) Рассмотрим отображение $j:\mathbb{C}^\Lambda\to l_1^0(\Lambda):z_\lambda\to z_\lambda\delta_\lambda$. Для заданного ограниченного полулинейного оператора $\varphi:\mathbb{C}^\Lambda\to\square_{norTop}(X)$, где $X$ - произвольное нормированное пространство рассмотрим линейный оператор $\psi:l_1^0(\Lambda)\to X:f\mapsto\sum_{\lambda\in\Lambda}f(\lambda)\varphi(1_\lambda)$. Так как $\Vert\psi(f)\Vert\leq\Vert\varphi\Vert\Vert f\Vert$, то $\psi$ ограничен. Более того непосредственно проверяется, что $\square_{norTop}(\psi)j=\varphi$. Единственность $\psi$ следует из цепочки равенств 
$$
\psi(f)=\sum_{\lambda\in\Lambda} f(\lambda)\psi(\delta_\lambda)=\sum_{\lambda\in\Lambda} f(\lambda)\square_{norTop}(\psi)(j(1_\lambda))=\sum_{\lambda\in\Lambda} f(\lambda)\varphi(1_\lambda)
$$

3) См. \cite{GroTopNorPr} теорема 0.12
\end{proof}

\begin{theorem}\label{ThPsTopFrDesc} Секвенциальное операторное пространство является псевдотопологически свободным тогда и только тогда, когда оно секвенциально топологически изморфно $\max(l_1^0(\Lambda))$ для некоторого множества $\Lambda$.
\end{theorem}
\begin{proof}
Из предложения \ref{PrNorTopChar} следует, что $l_1^0(\Lambda)$ $\square_{norTop}$-свободно с базой $\mathbb{C}^\Lambda$. Из предложения \ref{PrSQRelChar} получаем, что $\max(l_1^0(\Lambda))$ $\square_{sqRel}$-свободно с базой 
$l_1^0(\Lambda)$. Тогда из предложения \ref{PrCompOfFrIsFr} следует, что $\max(l_1^0(\Lambda))$ $\square_{sqpTop}$-свободно с базой $\mathbb{C}^\Lambda$. Из предложения \ref{PrUniqFr} следует, что все псевдотопологически 
свободные объекты имеют вид $\max(l_1^0(\Lambda))$ для некоторого множества $\Lambda$.
\end{proof}

\begin{corollary}\label{CorSQSpaceIsImgPsTopAdmEpiMorph}
Всякое секвенциальное операторное пространство является образом топологически сюръективного оператора из пространства $\max(l_1^0(\Lambda))$ для некоторого множества $\Lambda$.
\end{corollary}
\begin{proof}	
Из теоремы \ref{ThPsTopFrDesc} следует что оснащенная категория $(SQNor,\square_{sqpTop})$ свободолюбива. Теперь желаемый результат следует из предложений \ref{PrFrCoFrProjInjObjProp} и \ref{PrDecsPsTopAdmEpiMorph}
\end{proof}

\begin{theorem}\label{ThPsTopProjDesc}
Всякое псевдотопологически проективное секвенциальное операторное пространство секвенциально топологически изоморфно $\max(l_1^0(\Lambda))$ для некоторого множества $\Lambda$.
\end{theorem}
\begin{proof}
Пусть $P$ псевдотопологически проективное секвенциальное операторное пространство. 
Из предложений \ref{ThPsTopFrDesc}, \ref{CorSQSpaceIsImgPsTopAdmEpiMorph} видим, что существует $\square_{sqpTop}$ - допустимый эпиморфизм $\sigma:\max(l_1^0(\Lambda))\to P$ для некоторого множества $\Lambda$. Так как $\max(l_1^0(\Lambda))$ $\square_{sqpTop}$-свободный объект, то из предложения \ref{PrRetractsProjInj} следует, что $\sigma$ ретракция в $SQNor$. Из пункта 2) предложения \ref{PrSQRelChar} получаем, 
что структура секвенциального операторного пространства $P$ максимальна, т.е. $P=\max(\square_{sqRel}(P))$. Так как $\sigma$ есть ретракция в $SQNor$, то это также и ретракция в $Nor$ из пространства $l_1^0(\Lambda)$. По предложению \ref{PrNorTopChar} пространство $l_1^0(\Lambda)$ $\square_{sqRel}$-свободно. Так как $\square_{sqRel}(P)$ --- его ретракт в $Nor$, то из предложения 
\ref{PrFrCoFrProjInjObjProp} следует, что $\square_{sqRel}(P)$ $\square_{norTop}$-проективен. В этом случае снова из предложения \ref{PrNorTopChar} следует, что $\square_{sqRel}(P)$ топологически изоморфно 
$l_1^0(\Lambda')$ для некоторого множества $\Lambda'$. Применяя к этому изоморфизму функтор $\max$, получаем секвенциальный топологический изоморфизм между $P=\max(\square_{sqRel}(P))$ и $\max(l_1^0(\Lambda'))$.
\end{proof}

Аналогичные утверждения справедливы и в банаховом случае с заменой $l_1^0$  на $l_1$ пространства.






























\section{Косвободные секвенциальные операторные пространства}


В дальнейшем нам понадобится следующee простое наблюдение.

Из предложений \ref{PrDualOfCoprodIsProd} и \ref{PrSQSpaceIsSBFromT2n} следует, что имеют место секвенциальные изометрические изоморфизмы
$$
(t_2^\infty)^\triangle
=\bigoplus{}_\infty\{(t_2^n)^\triangle:n\in\mathbb{N}\}
=\bigoplus{}_\infty\{l_2^n:n\in\mathbb{N}\}
=l_2^\infty
$$
Поэтому снова применяя предложение \ref{PrDualOfCoprodIsProd} получим секвенциальный изометрический изоморфизм:
$$
\left(\bigoplus{}_1^0\{t_2^\infty:\lambda\in\Lambda\}\right)^\triangle
=\bigoplus{}_\infty\{l_2^\infty:\lambda\in\Lambda\}
$$
 
\subsection{Метрическая косвобода}

Рассмотрим функтор 
$$
\begin{aligned}
\square_{sqMet}^d : SQNor_1 \to Set^o: X &\mapsto \prod \left\{B_{(X^\triangle )^{\wideparen{n}}}:n\in\mathbb{N}\right\}\\
\varphi&\mapsto\prod\left\{ (\varphi^\triangle )^{\wideparen{n}}|_{B_{(Y^\triangle )^{\wideparen{n}}}}^{B_{(X^\triangle )^{\wideparen{n}}}}:n\in\mathbb{N}\right\}\\
\end{aligned}
$$
\begin{proposition}\label{PrDecsMetrAdmMonoMorph}
$\square_{sqMet}^d$-допустимыми мономорфизмами являются в точности секвенциально  изометрические операторы.
\end{proposition}
\begin{proof}
Морфизм $\varphi$ является $\square_{sqMet}^d$-допустмым мономорфизмом только если $\square_{sqMet}^d(\varphi)$ обратим слева как морфизм в $Set^o$. Это равносильно тому что 
$\square_{sqMet}^d(\varphi^\triangle)$ сюръективно. Последнее эквивалентно  сюръективности $(\varphi^\triangle)^{\wideparen{n}}|_{B_{X^{\wideparen{n}}}}^{B_{Y^{\wideparen{n}}}}$ 
для всех $n\in\mathbb{N}$. Это означает, что $(\varphi^\triangle)^{\wideparen{n}}$ строго коизометрично для каждого $n\in\mathbb{N}$, т.е. $\varphi^\triangle$ секвенциально строго коизометричен. 
По теореме \ref{ThDualSQOps} это равносильно тому, что $\varphi$ секвенциально изометричен.
\end{proof}

\begin{theorem}\label{ThMetCoFrDesc}
Метрически косвободным секвенциальным операторным пространством с базой $\Lambda$ является, с точностью до секвенциального изометрического изоморфизма, 
$\bigoplus{}_\infty$-сумма копий пространства $l_2^\infty:=\bigoplus{}_\infty\{l_2^n:n\in\mathbb{N}\}$, заиндексированных элементами множества $\Lambda$.
\end{theorem}
\begin{proof}
Пусть $\Lambda$ --- произвольное множество. Рассмотрим коммутативную диаграмму
$$
\xymatrix{
SQNor_1^o \ar[d]_{\nabla } \ar[rr]^{(\square_{sqMet}^{d})^o} & & {Set}\ar[d]^{1_{Set}}\\
SQNor_1\ar[rr]^{\square_{sqMet}}&  &{Set}}
$$
Здесь ${}^\nabla$ есть ковариантная версия функтора ${}^\triangle$.
Эта диаграмма коммутативна так как для произвольных секвенциальных операторных пространств $X$, $Y$ и любого $\varphi\in\mathcal{SB}(X,Y)$ выполнено
$$
1_{Set}((\square_{sqMet}^d)^o(\varphi))
=\prod\limits_{n\in\mathbb{N}} (\varphi^\triangle )^{\wideparen{n}}|_{B_{(Y^\triangle )^{\wideparen{n}}}}^{B_{(X^\triangle )^{\wideparen{n}}}}
=\square_{sqMet}({}^\nabla(\varphi))
$$
Из замечания \ref{CorSQUnivPropMaxTenProd} следует что функтор ${}^\nabla$ имеет левый сопряженный функтор, а именно ${}^\triangle$. Аналогично $1_{Set}$ сопряжен слева к самому себе. 
По теореме \ref{ThMetrFrDesc} объект $\bigoplus{}_1^0\{t_2^\infty:\lambda\in\Lambda\}$ $\square_{sqMet}$-свободен, поэтому по предложению \ref{PrFunctorMapFrToFr} объект 
$(\bigoplus{}_1^0\{t_2^\infty:\lambda\in\Lambda\})^\triangle=\bigoplus{}_\infty\{l_2^\infty:\lambda\in\Lambda\}$ является $(\square_{sqMet}^d)^o$ свободным, или что то же самое $\square_{sqMet}^d$-косвободным. 
Так как множество $\Lambda$ произвольно, то с учетом предложения \ref{PrUniqFr} получаем что все $\square_{sqMet}$-косвободные объекты с базой 
$\Lambda$ секвенциально изометрически изоморфны пространствам указанного вида.
\end{proof}

\begin{corollary}\label{CorSQSpaceIsFromMetrAdmMonoMorph}
Из всякого секвенциального операторного пространства существует секвенциально изометрический оператор в $\bigoplus_\infty\{l_2^\infty:\lambda\in\Lambda\}$ для некторого множества $\Lambda$.
\end{corollary}
\begin{proof}
Из теоремы \ref{ThMetrFrDesc} следует что оснащенная категория $(SQNor_1,\square_{sqMet}^d)$ косвободолюбива. Теперь желаемый результат следует из предложений \ref{PrFrCoFrProjInjObjProp} и \ref{PrDecsMetrAdmMonoMorph}.
\end{proof}

\begin{proposition}\label{PrCharacDualSQSp} Секвенциальное операторное пространство $X$ является дуальным секвенциальным операторным пространством тогда и только тогда когда существует секвенциально изометрический слабо${}^*$-слабо${}^*$ гомеоморфизм на слабо${}^*$ замкнутое подпространство в $\bigoplus_\infty\{l_2^\infty:\lambda\in\Lambda\}$  для некторого множества $\Lambda$.
\end{proposition}
\begin{proof}
Допустим, $X$ --- дуальное секвенциальное операторное пространство с секвенциальным предуалом $X_\triangle$. По предложению \ref{CorSQSpaceIsImgMetrAdmEpiMorph} для некоторого множества $\Lambda$ сущетсвует секвенциально коизометрический оператор $\pi:\bigoplus_1^0\{t_2^\infty:\lambda\in\Lambda\}\to X_\triangle$. По теореме \ref{ThDualSQOps} оператор $\pi^\triangle$ секвенциальная изометрия из $X_\triangle^\triangle=X$ в $(\bigoplus_1^0\{t_2^\infty:\lambda\in\Lambda\})^\triangle=\bigoplus_\infty\{l_2^\infty:\lambda\in\Lambda\}$. По леммме A.2.5 \cite{BleOpAlgAndMods} оператор $\pi^\triangle$ является слабо${}^*$-слабо${}^*$ гомеоморфизмом на свой слабо${}^*$ замкнутый образ.

Обратно, если $X$ слабо${}^*$ замкнутое подпространство в $Y:=\bigoplus_\infty\{l_2^\infty:\lambda\in\Lambda\}$ для некоторого множества $\Lambda$, то из предложению \ref{PrDualForWStarClQuotsAndSubsp} мы получаем, что $X=(Y/X_\perp)^\triangle$. Следовательно, $X$ дуальное секвенциальное операторное пространство с секвенциальным предуалом $X_\triangle:=Y/X_\perp$.
\end{proof}

Аналогичные утверждения справедливы и в банаховом случае.






























\subsection{Топологическая косвобода}

Рассмотрим 
функтор 
$$
\begin{aligned}
\square_{sqTop}^d : SQNor \to Nor_0^o, X &\mapsto \bigoplus{}_\infty \{(X^{\triangle })^{\wideparen{n}} : n \in \mathbb{N}\}\\
\varphi&\mapsto\bigoplus{}_\infty \{(\varphi^\triangle )^{\wideparen{n}} : n \in \mathbb{N}\}
\end{aligned}
$$

\begin{proposition}\label{PrDecsTopAdmMonoMorph}
$\square_{sqTop}^d$-допустимыми мономорфизмами являются в точности секвенциально топологически инъективные операторы.
\end{proposition}
\begin{proof}
Морфизм $\varphi$ является $\square_{sqTop}^d$-допустмым мономорфизмом только если $\square_{sqTop}^d(\varphi)$ обратим слева как морфизм в $Nor_0^o$. Это равносильно тому что 
$\square_{sqTop}^d(\varphi)=\square_{sqTop}(\varphi^\triangle)$ обратим справа в как морфизм в $Nor_0$. По предложению \ref{PrDecsTopAdmEpiMorph} это эквивалентно секвенциальной топологической сюръективности $\varphi^\triangle$. 
По теореме \ref{ThDualSQOps} это равносильно тому, что $\varphi$ секвенциально топологически инъективен.
\end{proof}
 
\begin{theorem}\label{ThTopCoFrDesc}
Секвенциальное операторное пространство является топологически косвободным тогда и только тогда, когда оно секвенциально топологически изоморфно $\bigoplus{}_\infty$ сумме пространств $l_2^\infty$ заиндексированных элементами множества $\Lambda$.
\end{theorem}
\begin{proof}
Пусть $\Lambda$ произвольное множество.Рассмотрим коммутативную диаграмму
$$
\xymatrix{
SQNor^o \ar[d]_{\nabla } \ar[rr]^{(\square_{sqTop}^d)^o} & & {Nor_0} \ar[d]^{1_{Nor_0}}\\
SQNor\ar[rr]^{\square_{sqTop}} & & {Nor_0}}
$$
Здесь $\nabla$ есть ковариантная версия функтора $\triangle$.
Эта диаграмма коммутативна так как для произвольных секвенциальных операторных пространств $X$, $Y$ и любого $\varphi\in\mathcal{SB}(X,Y)$ выполнено
$$
1_{Nor_0}((\square_{sqTop}^d)^o(\varphi))
=\bigoplus{}_\infty \{(\varphi^\triangle )^{\wideparen{n}} : n \in \mathbb{N}\}
=\square_{sqTop}({}^\nabla(\varphi))
$$
Из замечания \ref{CorSQUnivPropMaxTenProd} следует что функтор ${}^\nabla$ имеет левый сопряженный функтор, а именно ${}^\triangle$. Аналогично $1_{Nor_0}$ сопряжен слева к самому себе. 
По теореме \ref{ThTopFrDesc} объект $\bigoplus{}_1^0\{t_2^\infty:\lambda\in\Lambda\}$ $\square_{sqTop}$-свободен, поэтому по предложению \ref{PrFunctorMapFrToFr} объект 
$(\bigoplus{}_1^0\{t_2^\infty:\lambda\in\Lambda\})^\triangle=\bigoplus{}_\infty\{l_2^\infty:\lambda\in\Lambda\}$ является $(\square_{sqTop}^d)^o$-свободным, или что то же самое $\square_{sqTop}^d$-косвободным. С учетом предложения \ref{PrUniqFr} получаем, что все $\square_{sqTop}$-косвободные объекты с базой 
$\mathbb{C}^\Lambda$ секвенциально топологически изоморфны пространствам указанного вида.
\end{proof}

\begin{corollary}\label{PrMetrCoFrIsTopFr}
Всякое метрически косвободное секвенциальное операторное пространство топологически косвободно
\end{corollary}


\begin{corollary}\label{CorSQSpaceIsFromTopAdmMonoMorph} 
Из всякого секвенциального операторного пространства существует секвенциально топологически инъективный оператор в $\bigoplus_\infty\{l_2^\infty:\lambda\in\Lambda\}$ для некторого множества $\Lambda$.
\end{corollary}
\begin{proof} Из теоремы \ref{ThTopCoFrDesc} следует что оснащенная категория $(SQNor,\square_{sqTop}^d)$ косвободолюбива. Теперь желаемый результат следует из предложений \ref{PrFrCoFrProjInjObjProp} и \ref{PrDecsTopAdmMonoMorph}.
\end{proof} 

Аналогичные утверждения справедливы и в банаховом случае.





























\subsection{Псевдотопологическая косвобода и инъективность}

Рассмотрим функтор 
$$
\begin{aligned}
\square_{sqpTop}^d : SQNor \rightarrow Nor_0^o, X &\mapsto (X^\triangle)^{\wideparen{1}}\\
\varphi&\mapsto\varphi^\triangle
\end{aligned}
$$
который секвенциальному операторному пространству сопоставляет полулинейное нормированное пространство, построенное по $(X^\triangle)^{\wideparen{1}}$, а каждому морфизму сопоставляет $\varphi^\triangle$ рассматриваемый как ограниченный полулинейный оператор.

\begin{definition}\label{DefPsSQTopInjOp} Секвециально ограниченный оператор $\varphi:X\to Y$ называется псевдосеквенциально топологически инъективным, если для любого номера $n\in\mathbb{N}$ 
существует $c_n>0$ такое что для любого $x\in X^{\wideparen{n}}$ выполнено $c_n\Vert\varphi^{\wideparen{n}}(x)\Vert_{\wideparen{n}}\geq \Vert x\Vert_{\wideparen{n}}$
\end{definition}

\begin{proposition}\label{PrDecsPsTopAdmMonoMorph}
Пусть $\varphi:X\to Y$ секвенциально ограниченный оператор между секвенциальными операторными пространствами, тогда следующие условия эквивалентны
\newline
1) $\varphi$ $\square_{sqpTop}$-допустимый мономорфизм
\newline
2) $\varphi$ псевдосеквенциально топологически инъективен
\newline
3) $\varphi^{\wideparen{1}}$ топологически инъективен 
\end{proposition}
\begin{proof}
$1)\implies 2)$ Пусть $\varphi$ является $\square_{sqpTop}^d$-допустимым мономорфизмом. Тогда $\square_{sqpTop}^d(\varphi)$ обратим слева как морфизм в $Nor_0^o$. Это равносильно тому что 
$\square_{sqpTop}^d(\varphi)=\square_{sqpTop}(\varphi^\triangle)$ обратим справа в как морфизм в $Nor_0$. По предложению \ref{PrDecsPsTopAdmEpiMorph} это эквивалентно псевдосеквенциальной топологической сюръективности $\varphi^\triangle$. 
По предложению \ref{PrDualOps} это равносильно тому, что $\varphi$ псевдосеквенциально топологически инъективен.
\newline
$2)\implies 3)$ Очевидно.
\newline
$3)\implies 1)$ Пусть $\varphi^{\wideparen{1}}$ топологически инъективен, тогда по предложению \ref{PrDualOps} $(\varphi^\triangle)^{\wideparen{1}}$ топологически сюръективен. По предложению \ref{PrDecsPsTopAdmEpiMorph} $\varphi^\triangle$ $\square_{sqpTop}$-допустимый эпиморфизм, т.е. $\square_{sqpTop}(\varphi^\triangle)$ обратим справа как морфизм в $Nor_0$. Значит $\square_{sqpTop}^d(\varphi)=\square_{sqpTop}(\varphi^\triangle)$ обратим слева как морфизм в $Nor_0^o$. Значит $\varphi$ $\square_{sqpTop}^d$-допустимый мономорфизм. 
\end{proof}

Рассмотрим функторы
$$
\begin{aligned}
\square_{sqRel}^d : SQNor \to Nor^o: X &\mapsto (X^\triangle)^{\wideparen{1}}\\
\varphi&\mapsto\varphi\\
\end{aligned}
\qquad\qquad
\begin{aligned}
\square_{norTop}^d : Nor^o \to Nor_0^o: X &\mapsto X\\
\varphi&\mapsto\varphi\\
\end{aligned}
$$
Отметим очевидный факт $\square_{sqpTop}^d=\square_{norTop}^d\square_{sqRel}^d$.

\begin{proposition}\label{PrSQReldChar}
В оснащенной категории $(SQNor,\square_{sqRel}^d)$
\newline
1) $\square_{sqRel}^d$-косвободными объектами являются секвенциальные операторные пространства, секвенциально топологически изоморфные $\min(E^*)$ для некоторого нормированного пространства $E$. Данная категория косвободолюбива. 
\newline
2) всякий ретракт $\square_{sqRel}^d$-косвободного объекта имеет структуру минимального секвенциального операторного пространства.
\end{proposition}
\begin{proof}
1) Пусть $E\in Nor$. Рассмотрим коммутативную диаграмму
$$
\xymatrix{
SQNor^o \ar[d]_{\nabla } \ar[rr]^{(\square_{sqRel}^{d})^o} & & {Nor}\ar[d]^{1_{Nor}}\\
SQNor\ar[rr]^{\square_{sqRel}}&  &{Nor}}
$$
Здесь ${}^\nabla$ есть ковариантная версия функтора ${}^\triangle$.
Эта диаграмма коммутативна так как для произвольных секвенциальных операторных пространств $X$, $Y$ и любого $\varphi\in\mathcal{SB}(X,Y)$ выполнено
$$
1_{Nor}((\square_{sqRel}^d)^o(\varphi))
=\varphi^\triangle
=\square_{sqRel}({}^\nabla(\varphi))
$$
Из замечания \ref{CorSQUnivPropMaxTenProd} следует что функтор ${}^\nabla$ имеет левый сопряженный функтор, а именно ${}^\triangle$. Аналогично $1_{Nor}$ сопряжен слева к самому себе. 
По предложению \ref{PrSQRelChar} объект $\max(E)$ $\square_{sqRel}$-свободен, поэтому из предложений \ref{PrFunctorMapFrToFr}, \ref{PrDualityAndMinMax} объект 
$(\max(E))^\triangle=\min(E^*)$ является $(\square_{sqRel}^d)^o$-свободным, или что то же самое $\square_{sqRel}^d$-косвободным. 
Так как пространство $E$ произвольно, то с учетом предложения \ref{PrUniqFr} получаем что все $\square_{sqRel}^d$-косвободные объекты с базой 
$E$ секвенциально топологически изоморфны пространствам указанного вида. Как следствие оснащенная категория $(SQNor,\square_{sqRel}^d)$ косвободолюбива.
\newline
2) Пусть $\sigma:\min(E^*)\to X$ ретракция в $SQNor$. Правый обратный к $\sigma$ обозначим через $\rho$. Так как $\rho$ топологически инъекивен, то по предложению \ref{PrMinPreserveEmbedings} пространство $X$ имеет минимальную структуру.
\end{proof}

\begin{proposition}\label{PrNorTopdChar}
В оснащенной категории $(Nor^o, \square_{norTop}^d)$
\newline
1) $\square_{norTop}^d$-допустимыми мономорфизмами являются топологически сюръективные операторы
\newline
2) $\square_{norTop}^d$-косвободными являются пространства, топологически изоморфные $l_1^0(\Lambda)$ для некоторого $\mathbb{C}^\Lambda$.
\newline
3) $\square_{norTop}^d$-инъективными являются пространства, топологически изоморфные $l_1^0(\Lambda)$ для некоторого множества $\Lambda$. 
\end{proposition}
\begin{proof}
Все три утверждения следуют из предложения \ref{PrNorTopdChar} если учесть, что $\square_{norTop}^d=\square_{norTop}^o$.
\end{proof}

\begin{theorem}\label{ThPsTopCoFrDesc} 
Секвенциальное операторное пространство является псевдотопологически косвободным тогда и только тогда, когда оно секвенциально топологически изморфно $\min(l_\infty(\Lambda))$ для некоторого множества $\Lambda$.
\end{theorem}
\begin{proof}
Из предложения \ref{PrNorTopdChar} следует, что $l_1^0(\Lambda)$ $\square_{norTop}^d$-косвободно с базой $\mathbb{C}^\Lambda$. Из предложения \ref{PrSQReldChar} получаем, что $\min((l_1^0(\Lambda))^*)=\min(l_\infty(\Lambda))$ $\square_{sqRel}^d$-косвободно с базой 
$l_1^0(\Lambda)$. Тогда из предложения \ref{PrCompOfFrIsFr} следует, что $\min(l_\infty(\Lambda))$ $\square_{sqpTop}^d$-косвободно с базой $\mathbb{C}^\Lambda$. Из предложения \ref{PrUniqFr} следует, что все псевдотопологически 
косвободные объекты имеют вид $\min(l_\infty(\Lambda))$ для некоторого множества $\Lambda$. 
\end{proof}

\begin{corollary}\label{CorSQSpaceIsFromPsTopAdmMonoMorph}
Из всякого секвенциального операторного пространства существует топологически инъективный оператор в $\min(l_\infty(\Lambda))$ для некторого множества $\Lambda$.
\end{corollary}
\begin{proof}
Из теоремы \ref{ThPsTopCoFrDesc} следует что оснащенная категория $(SQNor,\square_{sqTop}^d)$ косвободолюбива. Теперь желаемый результат следует из предложений \ref{PrFrCoFrProjInjObjProp} и \ref{PrDecsPsTopAdmMonoMorph}.
\end{proof}

\begin{theorem}\label{ThPsTopInjDesc} 
Всякое псевдотопологически инъекивное секвенциальное операторное пространство секвенциально топологически изоморфно $\min(F)$, где $F$ ретракт в $Nor$ пространства $l_\infty(\Lambda)$ для некоторого множества $\Lambda$.
\end{theorem}
\begin{proof}
Пусть $I$ псевдотопологически инъективное секвенциальное операторное пространство. 
Из предложений \ref{ThPsTopCoFrDesc}, \ref{CorSQSpaceIsFromPsTopAdmMonoMorph} видим, что существует $\square_{sqpTop}^d$ - допустимый мономорфизм $\sigma:I\to\min(l_\infty(\Lambda))$ для некоторого множества $\Lambda$. Так как $\min(l_\infty(\Lambda))$ $\square_{sqpTop}^d$-косвободный объект, то из предложения \ref{PrRetractsProjInj} следует, что $\sigma$ коретракция в $SQNor$. Пусть $\rho$ правый обратный к $\sigma$ морфизм в $SQNor$. Он является ретракцией в $SQNor$, тогда из пункта 2) предложения \ref{PrSQReldChar} получаем, что структура секвенциального операторного пространства $I$ минимальна, т.е. $I=\min(\square_{sqRel}(I))$. Так как $\rho$ есть ретракция в $SQNor$, то это также и ретракция в $Nor$ из пространства $l_\infty(\Lambda)$ в $F:=\square_{sqRel}(I)$.
\end{proof}

Аналогичные утверждения справедливы и в банаховом случае.

\begin{remark} К сожалению, в теореме \ref{ThPsTopInjDesc} по аналогии с теоремой \ref{ThPsTopProjDesc} нельзя утверждать, что ретракты пространства $l_\infty(\Lambda)$ имеют вид $l_\infty(\Lambda')$ для некоторого множества $\Lambda'$. Действительно, в \cite{RosInjLmuSp} следствие 4.4 доказано существование топологически инъективного пространства $F$ которое не может быть топологически изоморфно ни одному дуальному пространству и в частности $l_\infty(\Lambda')$ для какого-нибудь множетсва $\Lambda'$. С другой стороны для некоторого множества $\Lambda$ существует изометрическое вложение $i:F\to l_\infty(\Lambda)$. Так как $F$ топологически инъективно, то по предложению \ref{PrRetractsProjInj} это вложение коретракция. Таким образом, $F$ ретракт $l_\infty(\Lambda)$, который не топологически изоморфен $l_\infty(\Lambda')$ ни для какого множества $\Lambda'$.
\end{remark}


\newpage
\begin{thebibliography}{999}
\bibitem{HelQFA}
\textit{Хелемский А. Я.} Квантовый функциональный анализ в бескоординатном изложении. — М.:МЦНМО, 2009.
\bibitem{HelFA}
\textit{Хелемский А. Я.} Лекции по функциональному анализу. — М.:МЦНМО, 2004.
\bibitem{EROpSp}
\textit{Ruan Z.-J., Effros E.} Operator Spaces, London Math. Soc. Monographs, New Series 23, Oxford University Press, New York 2000. 
\bibitem{MurphCstarOpTh}
\textit{Murphy G. J.} $C^*$-Algebras and Operator Theory. Academic Press, 1 edition (September 11, 1990)
\bibitem{RudinFA}
\textit{Рудин У.} Функциональный анализ. — М.:Мир, 1975.
\bibitem{LamOpFolgen}\textit{Lambert A.} Operatorfolgenr\"{a}ume. Eine Kategorie auf dem Weg von den Banach-R\"{a}umen zu den Operatorr\"{a}umen. Dissertation zur Erlangung des Grades Doktor der Naturwissenschaften der Technisch-Naturwissenschaftlichen Fakult\"{a}t I der Universit\"{a}t des Saarlandes. Saarbr\"{u}cken, 2002.
\bibitem{DefFloTensNorOpId}
\textit{Defant A., Floret K.} Tensor norms and operator ideals, North-Holland Amsterdam,
London, New York, Tokio 1993.
\bibitem{BleOpAlgAndMods}
\textit{Blecher P., Le Merdy C.} Operator Algebras and Their Modules: An Operator Space Approach, Oxford University Press, USA (January 13, 2005)
\bibitem{FabZizBanSpTh} 
\textit{M. Fabian, P. Habala, V. Montesinos, V. Zizler} Banach Space Theory: The Basis for Linear and Nonlinear Analysis, Springer; 2011 edition (December 15, 2010)
\bibitem{BrownItoUniquePredual}
\textit{L. Brown, T. Itô} Classes of Banach spaces with unique isometric preduals. Pacific J. Math. Volume 90, Number 2 (1980), 249-492
\bibitem{HelMetrFrQmod}
\textit{Хелемский А. Я.} Метрическая свобода и проективность для классических и квантовых нормированных модулей,, Матем. сб., 204:7 (2013), 127–158 
\bibitem{ShtTopFr}
\textit{Штейнер С. М.} Топологическая свобода для классических и квантовых модулей. Препринт.
\bibitem{GroTopNorPr}
\textit{N. Groenbaek.} Lifting problems for normed spaces. Preprint.
\bibitem{RosInjLmuSp}
\textit{Haskell P. Rosenthal} On injective Banach spaces and the spaces $L_\infty(\mu)$ for finite measures $\mu$. Acta Mathematica, July 1970, Volume 124, Issue 1, pp 205-248
\end{thebibliography}
\end{document}