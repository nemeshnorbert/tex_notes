\documentclass[12pt]{article}
\usepackage[left=2cm,right=2cm,
top=2cm,bottom=2cm,bindingoffset=0cm]{geometry}
\usepackage{amssymb,amsmath}
\usepackage[utf8]{inputenc} 
\usepackage[matrix,arrow,curve]{xy}
\usepackage[final]{graphicx} 
\usepackage{mathrsfs}
\usepackage[
    colorlinks=true, 
    urlcolor=blue, 
    linkcolor=blue, 
    citecolor=blue, 
    pdfborder={0 0 0}]{hyperref}
\usepackage{yhmath}
\usepackage{enumitem}

\hypersetup{frenchlinks=true}

\newtheorem{theorem}{Theorem}[subsection]
\newtheorem{lemma}[theorem]{Lemma}
\newtheorem{proposition}[theorem]{Proposition}
\newtheorem{remark}[theorem]{Remark}
\newtheorem{corollary}[theorem]{Corollary}
\newtheorem{definition}[theorem]{Definition}
\newtheorem{example}[theorem]{Example}

\newenvironment{proof}{\par $\triangleleft$}{$\triangleright$}

\pagestyle{plain}

\begin{document}

\begin{center}

\Large \textbf{
    Metric and topological freedom for operator sequence spaces}\\[0.5cm]
\small {Norbert Nemesh, Sergei Shteiner}\\[0.5cm]

\end{center}
\thispagestyle{empty}

\begin{abstract} 
In this paper we give description of free and cofree objects in the  category of
operator sequence spaces. First we show that this category possess the same
duality theory as category of normed spaces, then with the aid of these results
we give complete description of metrically and topologically free and cofree
objects.
\end{abstract}

\section{Preliminaries}

\subsection{Duality theory for normed spaces}
\begin{definition}[\cite{HelQFA}, 0.0.1, 4.4.1]\label{DefDuality} Let $E$, $F$
and $G$ be normed spaces and $\mathcal{D}:E\times F\to G$ be a bounded bilinear
operator, then 
\begin{enumerate}[label = (\roman*)]
    \item $\mathcal{D}$ is called non-degenerate from the left 
    (right) if the operator 
    $$
    {}^E\mathcal{D}
    :E\mapsto\mathcal{B}(F,G)
    :x\mapsto(y\mapsto\mathcal{D}(x,y))\qquad
    (\mathcal{D}^F
    :F\mapsto\mathcal{B}(E,G)
    :y\mapsto(x\mapsto\mathcal{D}(x,y)))
    $$ 
    is injective;
    
    \item $\mathcal{D}$ is called isometric from the left 
    (right) if ${}^E\mathcal{D}$ ($\mathcal{D}^F$) is isometric;
    
    \item $\mathcal{D}$ is called a vector duality if it is 
    non degenerate from the left and from the right;
    
    \item if $G=\mathbb{C}$ then vector duality $\mathcal{D}$ 
    is called scalar duality; 
\end{enumerate}
\end{definition}
Bilinear functionals of the form
$$
\mathcal{D}_{E,E^*}:E\times E^*\to\mathbb{C}:(x,f)\mapsto f(x)
\qquad
\mathcal{D}_{E^*,E}:E^*\times E\to\mathbb{C}:(f,x)\mapsto f(x)
$$
are called the standard scalar dualities. For all $x\in E$ and $f\in E^*$ we
have
$$
\Vert x\Vert=\sup \{|\mathcal{D}_{E,E^*}(x,f)|:f\in B_{E^*} \}
\qquad
\Vert f\Vert=\sup \{|\mathcal{D}_{E,E^*}(x,f)|:x\in  B_E \} 
$$
The first equality is a consequence of Hahn-Banach theorem, the second one is
the usual definition of operator norm. Note that $\mathcal{D}_{E,E^*}^E$ is the
natural embedding $\iota_E$ into the second dual space. For a given $T\in
\mathcal{B}(E,F)$, we have
$\mathcal{D}_{F,F^*}(T(x),g)=\mathcal{D}_{E,E^*}(x,T^*(g))$ where $x\in E$ and
$g\in F^*$. This is nothing more than the usual definition of adjoint operator.

\begin{definition}\label{DefDConv} Let $\mathcal{D}:E\times F\to G$ be a vector
duality between normed spaces $E$, $F$ and $G$. We say that a net
${(y_\nu)}_{\nu\in N}\subset F$ $\mathcal{D}$-conerges to $y\in F$ if for all
$x\in E$ a net ${\mathcal{D}(x,y_\nu-y)}_{\nu\in N}$ converges to $0$. Topology
generated by this type of convergence we will denote by
$\sigma_\mathcal{D}(F,E)$.
\end{definition}

Many types of convergence in functional analysis may be formulated in terms of
$\mathcal{D}$-convergence, for example usual weak convergence is nothing more
than $\mathcal{D}_{X^*,X}$-convergence. 

For a given $p\in[1,+\infty]\cup \{0 \}$ by $p'$ we denote conjugate exponent,
i.e. $p'=p/(p-1)$ for $p\in(1,+\infty)$ while $1'=\infty$ and $0'=\infty'=1$.
Recall the following standard fact.

\begin{proposition}\label{PrSumDuality} Let $ \{E_\lambda:\lambda\in \Lambda \}$
be a family of normed spaces and $p\in[1,+\infty]\cup \{0 \}$, then for the 
scalar duality
$$
\mathcal{D}:
    \bigoplus{}_p^0 \{E_\lambda:\lambda\in \Lambda \}\times 
    \bigoplus{}_{p'} \{E_\lambda^*:\lambda\in \Lambda \}\to \mathbb{C}: 
    (x,f)\mapsto\sum\limits_{\lambda\in \Lambda} f_\lambda(x_\lambda)
$$
the linear operator 
$\mathcal{D}^{\bigoplus{}_{p'} \{E_\lambda^*:\lambda\in \Lambda \}}$ 
is isometric. If $p\neq\infty$, then it is an isometric isomorphism.
\end{proposition}

Similar result holds for $\bigoplus{}_p$-sums.
























\subsection{Operators between normed spaces}

\begin{definition}\label{DefNorOpType} Let $ T:E\to F$ be a bounded linear
operator between normed spaces $E$ and $F$, then $ T$ is called
\begin{enumerate}[label = (\roman*)]
    \item contractive, if $\Vert T\Vert\leq 1$;

    \item \textit{$c$-topologically injective}, if there exist $c > 0$ 
    such that for all $x \in E$ holds $\Vert x\Vert\leq c\Vert  T(x)\Vert$. 
    If mentioning of constant $c$ will be irrelevant we will simply say 
    that $ T$ is topologically injective.

    \item \textit{(strictly) $c$-topologically surjective}, if for 
    all $c'>c$ and $y\in F$ there exist $x \in E$ such that 
    $ T(x) = y$ and $\Vert x \Vert < c' \Vert y \Vert$ 
    ($\Vert x \Vert \leq c \Vert y \Vert$). If mentioning of constant $c$
    will be irrelevant we will simply say that $ T$ is (strictly) topologically
    injective.

    \item isometric, if it is contractive and $1$-topologically injective

    \item (strictly) coisometric, if it is contractive and (strictly) 
    $1$-topologically surjective.
\end{enumerate}
\end{definition}

Clearly, our definition of isometric operator is equivalent to the usual one.

\begin{proposition}\label{PrEquivDescOfIsomCoisomOp} Let $E$, $F$ be normed
spaces and $T:E\to F$ be bounded linear operator. Then,
\begin{enumerate}[label = (\roman*)]
    \item $T$ (strictly) $c$-topologically surjective $\Longleftrightarrow$
    $T(B_E^\circ)\supset c^{-1}B_F^\circ$ ($T(B_E)\supset c^{-1}B_F$) 
    \item $T$ (strictly) coisometric 
    $\Longleftrightarrow$ $T(B_E^\circ)=B_F^\circ$ ($T(B_E)=B_F$)
\end{enumerate}
\end{proposition}
\begin{proof}
$(i)$ Assume $T$ is $c$-topologically surjective. Let $y\in c^{-1}B_F^\circ$, 
then there exist $k'>1$ such that $k'y\in c^{-1}B_F^\circ$. Define $c'=k'c>c$. 
By assumption there exist $x\in E$ such that $T(x)=y$ and 
$\Vert x\Vert< c'\Vert y\Vert=c\Vert k'y\Vert<1$. 
Since $y\in c^{-1}B_F^\circ$ is arbitrary, then
$T(B_E^\circ)\supset c^{-1}B_F^\circ$. Conversely, assume that
$T(B_E^\circ)\supset c^{-1}B_F^\circ$. Let $y\in F$ and $c'>c$, then
$\tilde{y}={(c')}^{-1}\Vert y\Vert^{-1}y\in c^{-1}B_F^\circ$. 
By assumption there
exist $\tilde{x}\in B_E^\circ$ such that $T(\tilde{x})=\tilde{y}$. In this case
for $x:=c'\Vert y\Vert\tilde{x}$ we have $\Vert x \Vert=c'\Vert
y\Vert\Vert\tilde{x}\Vert< c'\Vert y\Vert$ and $T(x)=c'\Vert\ y\Vert
T(\tilde{x})=c'\Vert y\Vert\tilde{y}=y$. Since $y\in F$ and $c'>c$ are
arbitrary, we conclude that $T$ is $c$-topologically surjective.

Assume $T$ is strictly $c$-topologically surjective. Let $y\in c^{-1}B_F$, then
by assumption there exist $x\in E$ such that $T(x)=y$ and 
$\Vert x\Vert\leq c\Vert y\Vert=1$. Since $y\in c^{-1}B_F^\circ$ 
is arbitrary, we have $T(B_E)\supset c^{-1}B_F$. 
Conversely assume that $T(B_E)\supset c^{-1}B_F$. Let
$y\in F$, then $\tilde{y}=c^{-1}\Vert y\Vert^{-1}y\in c^{-1}B_F$. By assumption
there exist $\tilde{x}\in B_E$ such that $T(\tilde{x})=\tilde{y}$. In this case
for $x:=c\Vert y\Vert\tilde{x}$ we have $\Vert x \Vert=c'\Vert
y\Vert\Vert\tilde{x}\Vert\leq c\Vert y\Vert$ and $T(x)=c'\Vert\ y\Vert
T(\tilde{x})=c'\Vert y\Vert\tilde{y}=y$. Since $y\in F$ is arbitrary, then $T$
is strictly $c$-topologically surjective.

$(ii)$ Assume $T$ is coisometric. Then $\Vert T\Vert\leq 1$ 
and as the consequence $T(B_E^\circ)\subset B_F^\circ$. From paragraph $(i)$ 
it follows that $T(B_E^\circ)\supset B_F^\circ$. 
Taking into account the reverse inclusion we
can say $T(B_E^\circ)=B_F^\circ$. Conversely, assume that
$T(B_E^\circ)=B_F^\circ$. In particular $\Vert T\Vert\leq 1$ and
$T(B_E^\circ)\supset B_F^\circ$. From paragraph $(ii)$ it follows that $T$ is
$1$-topologically surjective. Hence, $T$ is coisometric. Similar arguments
applies for strictly coisometric operators.
\end{proof}

\begin{proposition}\label{PrDualOps} Let $ T:E\to F$ be bounded linear operator
between normed spaces and $c>0$, then
\begin{enumerate}[label = (\roman*)]
    \item if $ T$ is (strictly) $c$-topologically surjective, then $ T^*$ is
    $c$-topologically injective

    \item if $ T$ $c$-topologically injective, then $ T^*$ is strictly
    $c$-topologically surjective

    \item if $ T^*$ (strictly) $c$-topologically surjective, then $ T$ is
    $c$-topologically injective

    \item if $ T^*$ $c$-topologically injective and $E$ is complete, then $ T$ 
    is $c$-topologically surjective
\end{enumerate}
\end{proposition}
\begin{proof}
$(i)$ Since $T$ is $c$-topologically surjective we have $c^{-1}B_F^\circ\subset
T(B_E^\circ)$, hence for all $g\in F^*$ we have
$$
\Vert  T^*(g)\Vert
=\sup \{|g( T(x))|:x\in B_E^\circ \}
=\sup \{|g(y)|: y\in T(B_E^\circ) \}
\geq\sup \{|g(y)|: y\in c^{-1}B_F^\circ \}
$$
$$
=\sup \{|g(c^{-1}y)|: y\in B_F^\circ \}
=c^{-1}\sup \{|g(y)|: y\in B_F^\circ \}
=c^{-1}\Vert g\Vert
$$
Since $g\in F^*$ is arbitrary $ T^*$ is $c$-topologically injective. Similar
argument applies for strictly $c$-topologically surjective operator.

$(ii)$ Let $g\in E^*$. Since $ T$ is $c$-topologically injective, then 
$\tilde{ T}:=T|^{\operatorname{Im}( T)}$ topological linear isomorphism. 
Denote by $i:\operatorname{Im}( T)\to F$ the natural embedding 
of $\operatorname{Im}( T)$ into $F$, then $ T=i\tilde{ T}$. 
Now consider bounded linear functional
$f_0:=g\tilde{ T}^{-1}\in F^*$. By Hahn-Banach theorem there exist bounded
linear functional $f\in F^*$ such that $\Vert f\Vert=\Vert f_0\Vert$ and
$f_0=fi$. In this case $g=f_0\tilde{ T}=f_0 i\tilde{ T}=f T= T^*(f)$. Since $ T$
is $c$-topologically injective, then for all $x\in F$ we have
$$
|f(x)|=|g(\tilde{ T}^{-1}(x))|
\leq\Vert g\Vert\Vert \tilde{ T}^{-1}(x)\Vert
\leq\Vert g\Vert c\Vert  T(\tilde{ T}^{-1}(x))\Vert
\leq c\Vert g\Vert\Vert x\Vert
$$
Hence $\Vert f\Vert\leq c\Vert g\Vert$. Since $g\in E^*$ is arbitrary, then $
T^*$ is strictly $c$-topologically surjective.

$(iii)$ From paragraph $(i)$ it follows that $ T^{**}$ is $c$-topologically 
injective. Note that natural embedding into the second dual is isometric 
and also that $\iota_F  T = T^{**}\iota_E$. Then for all $x\in E$ we get
$$
\Vert T(x)\Vert
=\Vert \iota_F( T(x))\Vert
=\Vert T^{**}(\iota_E(x))\Vert
\geq c^{-1}\Vert \iota_E(x)\Vert
=c^{-1}\Vert x\Vert
$$
Since $x\in E$ is arbitrary then $ T$ is $c$-topologically injective.

$(iv)$ Assume that 
$c^{-1}B_F^\circ\not\subset\operatorname{cl}_F( T(B_E^\circ))$, then there exist
$y_0\in c^{-1}B_F\setminus\operatorname{cl}_F( T(B_E^\circ))$.
In particular, $\Vert y_0\Vert<c^{-1}$. Consider sets $A= \{y_0 \}$ and
$B=\operatorname{cl}_F( T(B_E^\circ))$. Obviously, $A$ is compact and convex.
Since  $B_E^\circ$ is convex, and $ T$ is linear, then $ T(B_E^\circ)$ is also
convex. As the consequence $B$ is closed and convex. 
By theorem 3.4~\cite{RudinFA}  
there exist $g\in F^*$ and $\gamma_1,\gamma_2\in\mathbb{R}$ such that 
for all $y\in B$ holds 
$\operatorname{Re}(g(y_0))>\gamma_2>\gamma_1>\operatorname{Re}(g(y))$. Without
loss of generality we may assume that $\gamma_1>\gamma_2=1$. So all $x\in
B_E^\circ$ we get $\operatorname{Re}(g(y_0))>1>\operatorname{Re}(g( T(x)))$.
Note that for all $x\in B_E^\circ$ there exist $\alpha\in\mathbb{C}$ such that
$|\alpha|<1$ and $|g( T(x))|=\operatorname{Re}(g(T(\alpha x)))$. Since
$|\alpha|\leq 1$, we see that $\alpha x\in B_E^\circ$ and $| T^*(g)(x)|=|
T(g(x))|=\operatorname{Re}(g( T(\alpha x)))<1$. Since $x\in B_E^\circ$ is
arbitrary, then $\Vert T^*(g)\Vert\leq 1$. Further $\Vert g\Vert>|g(y_0)|/\Vert
y_0\Vert>c\operatorname{Re}(g(y_0))>c$, but $ T^*$ is $c$-topologically
injective. Hence, $\Vert g\Vert\leq c\Vert T^*(g)\Vert\leq c$. Contradiction, so
$c^{-1}B_F^\circ\subset \operatorname{cl}_F(T(B_E^\circ))$. As $E$ is complete,
by proposition 4.4.1~\cite{HelFA} we get $c^{-1}B_F^\circ\subset T(B_E^\circ)$.
This implies that $T$ is $c$-topologically surjective.
\end{proof}





























\section{Operator sequence spaces}

\subsection{Matrix notation}

\begin{definition}\label{DefMatrNot} Let $n,k\in\mathbb{N}$, then by $M_{n,k}$
we denote a linear space of complex valued matrices of the size $n\times k$. If
$E$ is a linear space, then by $E^k$ we denote linear space of columns of the
height $k$ with entries in $E$.
\end{definition}

For a given $\alpha\in M_{n,k}$ and $x\in E^k$ by $\alpha x$ we denote column in
$E^n$ such that
$$
{(\alpha x)}_i=\sum\limits_{j=1}^n \alpha_{ij} x_j
$$
This formula is a natural generalization of matrix multiplication.

By default, the linear space $M_{n,k}$, endowed with operator norm
$\Vert\cdot\Vert$, but sometimes we will need so called Hilbert-Schmidt norm. It
is defined as follows. Let $\alpha\in M_{n,k}$, then its Hilbert-Schmidt norms
is defined as
$$
\Vert\alpha\Vert_{hs}=\operatorname{trace}{(|\alpha|^2)}^{1/2}
$$
where $|\alpha|={(\alpha^*\alpha)}^{1/2}$. Note that
$\Vert\alpha\Vert\leq\Vert\alpha\Vert_{hs}$ and
$\Vert|\alpha|\Vert_{hs}=\Vert|\alpha^*|\Vert=\Vert\alpha\Vert_{hs}$
(\cite{EROpSp}, 1.2). By $\operatorname{diag}_n(\lambda_1,\ldots,\lambda_n)$  we
will denote diagonal matrix of the size $n\times n$ with
$\lambda_1,\ldots,\lambda_n$ on the main diagonal. We also use the notation
$\operatorname{diag}_n(\lambda):=\operatorname{diag}_n(\lambda,\ldots,\lambda)$.
Given matrices $\alpha_1\in M_{m,n_1},\ldots,\alpha_k\in M_{n,k_m}$ we can glue
them together from the right to get the matrix 
$[\alpha_1,\ldots,\alpha_k]\in M_{n,k_1+\cdots+k_m}$.





























\subsection{Examples and definitions} 
For the beginning we need to recall standard definitions 
from~\cite{LamOpFolgen}.

\begin{definition}[\cite{LamOpFolgen}, 1.1.7]\label{DefSQSpace} Let $E$ be a
linear space, and for each $n\in\mathbb{N}$ we have a norm on $\Vert \cdot
\Vert_{\wideparen{n}}:E^n\to\mathbb{R}_+$. We say that the pair 
$X = (E^n, {(\Vert \cdot \Vert_{\wideparen{n}})}_{n \in \mathbb{N}})$, 
defines the structure of \textit{operator sequence} space on $E$, 
if the following conditions are satisfied:
\begin{enumerate}[label = (\roman*)]
    \item for all $m, n \in \mathbb{N}$, 
    $x \in E^{\wideparen{n}}$, $\alpha \in M_{m, n}$ holds
    $$
    \Vert 
        \alpha x 
    \Vert_{\wideparen{m}} 
    \leq \Vert \alpha \Vert  \Vert x \Vert_{\wideparen{n}}
    $$

    \item for all $m, n \in \mathbb{N}$, $x \in E^n$, $y \in E^m$ holds
    $$
    \left\Vert 
    \begin{pmatrix} x \\ y \end{pmatrix} 
    \right\Vert^2_{\wideparen{n + m}} 
    \leq   
    \Vert x \Vert_{\wideparen{n}}^2 + \Vert y \Vert_{\wideparen{m}}^2
    $$
\end{enumerate} 

By $X^{\wideparen{n}}$ we will denote the normed space $(E^n,\Vert \cdot
\Vert_{\wideparen{n}})$, we will call it $n$-th amplification of $X$.
\end{definition}

\begin{proposition}\label{PrRedundantAxiom} Let $X$ be an operator sequence
space, $n\in\mathbb{N}$ and $x\in E^{\wideparen{n}}$, then

$(i)$ for all $m\in\mathbb{N}$ holds 
$\Vert {(x, 0)}^{tr}\Vert_{\wideparen{n + m}}=\Vert x\Vert_{\wideparen{n}}$

$(ii)$ for any partial isometry $s\in M_{n,n}$ holds $\Vert
sx\Vert_{\wideparen{n}}=\Vert x\Vert_{\wideparen{n}}$. In particluar norm
doesn't change after permuation of coordinates
\end{proposition}
\begin{proof} $(i)$ Result follows from inequalities
$$
\Vert {(x, 0)}^{tr}\Vert_{\wideparen{n + m}}
\leq 
{\left(
    \Vert x\Vert_{\wideparen{n}}^2+\Vert 0\Vert_{\wideparen{m}}^2
\right)}^{1/2}=\Vert x\Vert_{\wideparen{n}}
$$
$$
\Vert x\Vert_{\wideparen{n}}
=\Vert[\operatorname{diag}_n(1),0]{(x,0)}^{tr}\Vert_{n}
\leq
\Vert[\operatorname{diag}_n(1),0]\Vert\Vert{(x,0)}^{tr}\Vert_{\wideparen{n+m}}
=\Vert{(x,0)}^{tr}\Vert_{\wideparen{n+m}}
$$
$(ii)$ Since $s$ is partial isometry, then $s^*s=\operatorname{diag}_n(1)$, so
result follows from inequalities
$$
\Vert sx\Vert_{\wideparen{n}}
\leq\Vert s\Vert\Vert x\Vert_{\wideparen{n}}
=\Vert x\Vert_{\wideparen{n}}=
\Vert s^*sx\Vert_{\wideparen{n}}
\leq\Vert s^*\Vert\Vert sx\Vert_{\wideparen{n}}
=\Vert sx\Vert_{\wideparen{n}}
$$
\end{proof}

\begin{proposition}\label{PrCHaveUniqueOSS} The Hilbert space $\mathbb{C}$ have
unique operator sequence space structure given by identifications
$\mathbb{C}^{\wideparen{n}}=l_2^n$.
\end{proposition}
\begin{proof} Let $\mathbb{C}$ endowed with some operator sequence space
structure. Fix $\xi\in\mathbb{C}^n$, then consider 
$\eta={(\Vert \xi\Vert_{l_2^n},0,\ldots,0)}^{tr}\in \mathbb{C}^n$. Since
$\Vert\eta\Vert_{l_2^n}=\Vert\xi\Vert_{l_2^n}$  there exist unitary matrix 
$s\in M_{n,n}$ such that $\eta=s\xi$. Therefore 
$\Vert\eta\Vert_{\wideparen{n}}
    =\Vert s\xi\Vert_{\wideparen{n}}
    \leq\Vert s\Vert\Vert\xi\Vert_{\wideparen{n}}
    =\Vert\xi\Vert_{\wideparen{n}}$. 
By proposition~\ref{PrRedundantAxiom} we get that
$\Vert\eta\Vert_{\wideparen{n}}=\Vert \xi\Vert_{l_2^n}$, hence
$\Vert\xi\Vert_{\wideparen{n}}\geq\Vert\xi\Vert_{l_2^n}$. On the other hand from
second axiom of operator sequence spaces we have $\Vert
\xi\Vert_{\wideparen{n}}\leq\Vert\xi\Vert_{l_2^n}$, therefore $\Vert
\xi\Vert_{\wideparen{n}}=\Vert \xi\Vert_{l_2^n}$. Since $n\in\mathbb{N}$ and
$\xi\in \mathbb{C}^{\wideparen{n}}$ are arbitrary we conclude
$\mathbb{C}^{\wideparen{n}}=l_2^n$.
\end{proof}

\begin{proposition}\label{PrSQAxiomRed} Let
${(\Vert\cdot\Vert_{\wideparen{n}}:E^n\to\mathbb{R}_+)}_{n\in\mathbb{N}}$ be a
family of functions satisfying axioms of operator sequence spaces, and assume
that equality $\Vert x\Vert_{\wideparen{1}}=0$ implies $x=0$. Then $E$ is a
operator sequence space.
\end{proposition}
\begin{proof}
Let $x\in E^n$ and $\lambda\in \mathbb{C}\setminus \{0 \}$, then 
$$
\Vert\lambda x\Vert_{\wideparen{n}}
=\Vert\operatorname{diag}_n(\lambda)x\Vert_{\wideparen{n}}
\leq\Vert\operatorname{diag}_n(\lambda)\Vert\Vert x\Vert_{\wideparen{n}}
=|\lambda|\Vert x\Vert_{\wideparen{n}}
=|\lambda|\Vert\lambda^{-1}\lambda x\Vert_{\wideparen{n}}
\leq|\lambda||\lambda^{-1}|\Vert\lambda x\Vert_{\wideparen{n}}
=\Vert\lambda x\Vert_{\wideparen{n}}
$$
Consequently $\Vert\lambda x\Vert_{\wideparen{n}}=|\lambda|\Vert
x\Vert_{\wideparen{n}}$ for all $\lambda\neq 0$. For $\lambda=0$ equality is
obvious. Let $x',x''\in E^n\setminus \{0 \}$, then denote 
$\mu={(
    \Vert x'\Vert_{\wideparen{n}}^2+
    \Vert x''\Vert_{\wideparen{n}}^2)}^{1/2}$. 
In this case
$$
\Vert x'+x''\Vert_{\wideparen{n}}^2
=\left\Vert
    \begin{pmatrix}\operatorname{diag}_n(\mu) & 0\\ 
        0 & \operatorname{diag}_n(\mu)
    \end{pmatrix}
    \begin{pmatrix}
        \mu^{-1}x'\\ 
        \mu^{-1}x''
    \end{pmatrix}
\right\Vert_{\wideparen{n}}^2
\leq\left\Vert
    \begin{pmatrix}
        \operatorname{diag}_n(\mu) & 0\\ 0
         & \operatorname{diag}_n(\mu)
    \end{pmatrix}
\right\Vert^2
\left\Vert
    \begin{pmatrix}\mu^{-1}x'\\
        \mu^{-1}x''
    \end{pmatrix}
\right\Vert_{\wideparen{n}}^2
$$
$$
\leq\mu^2(\mu^{-2}\Vert x'\Vert_{\wideparen{n}}^2
+\mu^{-2}\Vert x''\Vert_{\wideparen{n}}^2)
=\Vert x'\Vert_{\wideparen{n}}^2+\Vert x''\Vert_{\wideparen{n}}^2
\leq {(\Vert x'\Vert_{\wideparen{n}}+\Vert x''\Vert_{\wideparen{n}})}^2
$$
Hence, for $x',x''\neq 0$ we have $\Vert x'+x''\Vert_{\wideparen{n}}\leq\Vert
x'\Vert_{\wideparen{n}}+\Vert x''\Vert_{\wideparen{n''}}$. For $x'=x''=0$ the
equality is obvious.
\end{proof}


\begin{proposition}[\cite{LamOpFolgen}, 1.1.4]\label{PrNormVsSQNorm} Let $X$ be
operator sequence space, $n\in\mathbb{N}$. Then for all $x\in X^{\wideparen{n}}$
and $i=1,n$ holds
$$
\Vert x_i\Vert_{\wideparen{1}}
\leq\Vert x\Vert_{\wideparen{n}}
\leq\sum\limits_{k=1}^n\Vert x_k\Vert_{\wideparen{1}}
\leq n\Vert x\Vert_{\wideparen{n}}
$$
\end{proposition}


We say that $X$ is a \textit{operator sequence} space of normed space $(E, \Vert
\cdot \Vert_{\wideparen{1}}$). It is easy to see for a given operator sequence
space $X$ the normed space $X^{\wideparen{n}}$ have its own natural structure of
operator sequence space: it is enough to identify
${(X^{\wideparen{n}})}^{\wideparen{k}}$ with $X^{\wideparen{nk}}$.

\begin{example}[\cite{LamOpFolgen}, 1.1.8]\label{ExHilSQ} Let $H$ be a Hilbert
space, then its maximal operator sequence space structure is given by
identifications
${\max(H)}^{\wideparen{n}}=\bigoplus{}_2 \{H:\lambda\in\mathbb{N}_n \}$. 
Obviously
${\max(H)}^{\wideparen{n}}$ is a Hilbert space for every $n\in\mathbb{N}$. 
We will call this structure the standard operator sequence space structure 
of $H$ and usually denote $\max(H)$ as $H$.
\end{example}

\begin{definition}[\cite{LamOpFolgen}, 1.1.18]\label{ExT2nSQ} Let $H$ be a
Hilbert space, then its minimal operator sequence space structure is given by
identifications ${\min(H)}^{\wideparen{n}} = \mathcal{B}(l_2^n,H)$. 
\end{definition}

By $t_2^n$ we denote $\min(l_2^n)$.

\begin{definition}\label{DefOpSubAlgSQ} Let $A$ be a subalgebra of
$\mathcal{B}(H)$ for some Hilbert space $H$, then we define its standard
operator sequence space structure by embedding $A^n\hookrightarrow
\mathcal{B}(H,H^{\wideparen{n}})$.
\end{definition}

\begin{proposition}\label{PrCstarAlgSQ} Let $A$ be a $C^*$ algebra, then its
standard operator sequence space structure doesn't depend on its representation
on Hilbert space and for any $n\in\mathbb{N}$ and $a\in A^{\wideparen{n}}$ we
have
$$
\Vert a\Vert_{\wideparen{n}}
=\left\Vert\sum\limits_{i=1}^n a_i^*a_i\right\Vert^{1/2}
$$
In particular standard operator sequence space structures of $\mathbb{C}$
regarded as $C^*$ algebra and as Hilbert space are the same.
\end{proposition}
\begin{proof} Let $\pi:A\to\mathcal{B}(H)$ be any isometric
${}^*$-representation of $A$ on the Hilbert space $H$. Fix $n\in\mathbb{N}$ then
$a\in A^{\wideparen{n}}$ is identified with operator $T:H\mapsto
H^{\wideparen{n}}:\xi\mapsto \oplus{}_2 \{\pi(a_i)(\xi):i\in\mathbb{N}_n \}$. 
Then
$$
\Vert a\Vert_{\wideparen{n}}^2
=\Vert T\Vert^2
=\sup \{\Vert \oplus_2 \{\pi(a_i)(\xi):i\in\mathbb{N}_n \}\Vert^2:\xi\in B_H \}=
$$
$$
=\sup\left \{ 
    \sum\limits_{i=1}^n\langle \pi(a_i)(\xi),\pi(a_i)(\xi)\rangle:\xi\in B_H
\right \}
=\sup\left \{ 
    \left\langle \pi\left(
        \sum\limits_{i=1}^n a_i^*a_i
    \right)(\xi),\xi\right\rangle:\xi\in B_H
\right \}
$$
From proposition 2.2.4 and 2.2.5~\cite{MurphCstarOpTh} we get that $\sum_{i=1}^n
a_i^* a_i\geq 0$, so $\pi(\sum_{i=1}^n a_i^* a_i)\geq 0$ and by proposition
6.4.6~\cite{HelFA} we get that
$$
\Vert a\Vert_{\wideparen{n}}^2
=\sup\left \{ 
    \left\langle \pi\left(
        \sum\limits_{i=1}^n a_i^*a_i
    \right)(\xi),\xi\right\rangle:\xi\in B_H
\right \}
=\left\Vert \pi\left(\sum\limits_{i=1}^n a_i^*a_i\right)\right\Vert
=\left\Vert \sum\limits_{i=1}^n a_i^*a_i\right\Vert
$$
If $A=\mathbb{C}$ we get
$$
\Vert a\Vert_{\wideparen{n}}
=\left| \sum\limits_{i=1}^n \overline{a_i}a_i\right|^{1/2}
={\left(\sum\limits_{i=1}^n |a_i|^2\right)}^{1/2}
=\Vert a\Vert_{l_2^n}
$$
so both definitions give the same operator sequence space structure.
\end{proof}

\begin{proposition}\label{PrCommCstarSQ} Let $\Omega$ be a locally compact
topological space, then for any $n\in\mathbb{N}$ we have an isometric
isomorphism
$$
i_C:{C_0(\Omega)}^{\wideparen{n}}\to C_0(\Omega,\mathbb{C}^n):
f\mapsto (\omega\mapsto{(f_i(\omega))}_{i\in\mathbb{N}_n})
$$
\end{proposition} 
\begin{proof} Using proposition~\ref{PrCstarAlgSQ} for any 
$f\in {C_0(\Omega)}^{\wideparen{n}}$ we get
$$
\Vert f\Vert_{\wideparen{n}}
=\left\Vert \sum\limits_{i=1}^n f_i^* f_i\right\Vert^{1/2}
=\sup\left \{
    {\left(\sum\limits_{i=1}^n |f_i(\omega)|^2\right)}^{1/2}:\omega\in\Omega
\right \}
=\sup \{\Vert i_C(f)(\omega)\Vert:\omega\in\Omega \}
=\Vert i_C(f)\Vert
$$
Thus $i_C$ is an isometry. For a given $g\in C_0(\Omega,\mathbb{C}^n)$ and each
$i\in\mathbb{N}_n$ consider continuous function
$f_i:\Omega\to\mathbb{C}:\omega\mapsto {g(\omega)}_i$ and define
$f={(f_1,\ldots,f_n)}^{tr}\in {C_0(\Omega)}^{\wideparen{n}}$. 
Clearly, $i_C(f)=g$, so $i_C$ is surjective. 
Therefore $i_C$ is a surjective isometry, hence isometric isomorphism.
\end{proof}




























\subsection{Operators between operator sequence spaces}

\begin{definition}[\cite{LamOpFolgen}, 1.2.1]\label{DefSBOp} Let $X$ and $Y$ be
operator sequence spaces and $\varphi : X \to Y$ be a linear operator. For a
given $n\in\mathbb{N}$ its $n$-th \textit{amplification} is called a linear
operator $\varphi^{\wideparen{n}} : X^{\wideparen{n}} \to Y^{\wideparen{n}}$
defined by 
$$
\varphi^{\wideparen{n}}(x)={(\varphi(x_i))}_{i=1,n}
$$
We say that $\varphi$ \textit{sequentially bounded}, if 
$$
\Vert \varphi \Vert_{sb} 
:= \sup \{
    \Vert 
        \varphi^{\wideparen{n}}
    \Vert_{\mathcal{B}(X^{\wideparen{n}},Y^{\wideparen{n}})}
    :n\in\mathbb{N}
 \}  < \infty
$$
\end{definition}

\begin{proposition}\label{PrSimplAmplProps} Let $X$, $Y$, $Z$ be operator
sequence spaces, $\varphi:X\to Y$, $\psi:Y\to Z$ be linear operators and
$n,m\in\mathbb{N}$. Then 
\begin{enumerate}[label = (\roman*)]
    \item $\varphi$ injective (surjective) 
    if and only if  $\varphi^{\wideparen{n}}$ injective (surjective).
    
    \item $\Vert\varphi\Vert_{\wideparen{n}}
    \leq\Vert\varphi\Vert_{\wideparen{n+1}}$
    and as the consequence $\mathcal{SB}(X,Y)\subset\mathcal{B}(X,Y)$.
    
    \item ${(\psi\varphi)}^{\wideparen{n}}
    =\psi^{\wideparen{n}}\varphi^{\wideparen{n}}$
    and as the consequence
    $\Vert\psi\varphi\Vert_{sb}
    \leq\Vert\psi\Vert_{sb}\Vert\varphi\Vert_{sb}$
    
    \item For all $\alpha\in M_{n,m}$, $x\in X^{\wideparen{m}}$ holds
    $\varphi^{\wideparen{n}}(\alpha x)=\alpha\varphi^{\wideparen{m}}(x)$
\end{enumerate}
\end{proposition}
\begin{proof}
$(i)$ Directly follows from definition

$(ii)$ From proposition~\ref{PrRedundantAxiom} we get
$$
\Vert\varphi\Vert_{\wideparen{n}}
=\sup \{
    \Vert
        \varphi^{\wideparen{n}}(x)
    \Vert_{\wideparen{n}}:x\in B_{X^{\wideparen{n}}} \}
=\sup \{
    \Vert
        \varphi^{\wideparen{n+1}}({(x,0)}^{tr})
    \Vert_{\wideparen{n}}:{(x,0)}^{tr}\in B_{X^{\wideparen{n+1}}} \}
$$
$$
=\sup \{
    \Vert
        \varphi^{\wideparen{n+1}}(x)
    \Vert_{\wideparen{n}}:x\in B_{X^{\wideparen{n+1}}} \}
=\Vert\varphi\Vert_{\wideparen{n+1}}
$$	

$(ii)$ For all $x\in X^{\wideparen{n}}$ we have
$$
{(\psi\varphi)}^{\wideparen{n}}(x)
={((\psi\varphi)(x_i))}_{i\in\mathbb{N}_n}
={(\psi(\varphi(x_i)))}_{i\in\mathbb{N}_n}
={\psi}^{\wideparen{n}}({(\varphi(x_i))}_{i\in\mathbb{N}_n})
=\psi^{\wideparen{n}}(\varphi^{\wideparen{n}}(x))
$$
so 
${(\psi\varphi)}^{\wideparen{n}}=\psi^{\wideparen{n}}\varphi^{\wideparen{n}}$. 
And what is more, 
$$
\Vert\psi\varphi\Vert_{sb}
=\sup \{\Vert\psi^{\wideparen{n}}\varphi^{\wideparen{n}}\Vert:n\in\mathbb{N} \}
\leq\sup \{
    \Vert
        \psi^{\wideparen{n}}
    \Vert\Vert\varphi^{\wideparen{n}}\Vert:n\in\mathbb{N} \}
\leq\Vert\psi\Vert_{sb}\Vert\varphi\Vert_{sb}
$$
$(iv)$ For each $i\in\mathbb{N}_n$ holds
$$
\varphi^{\wideparen{n}}{(\alpha x)}_i 
=\varphi({(\alpha x)}_i)
=\varphi\left(\sum\limits_{j=1}^m \alpha_{ij }x_j\right)
=\sum\limits_{j=1}^m\alpha_{ij} \varphi(x_j) 
=\sum\limits_{j=1}^m\alpha_{ij}{\varphi^{\wideparen{m}}(x)}_j
={(\alpha\varphi^{\wideparen{m}}(x))}_i
$$
So $\varphi^{\wideparen{n}}(\alpha x)=\alpha\varphi^{\wideparen{m}}(x)$.
\end{proof}

\begin{definition}\label{DefSBOpType}
Let $\varphi:X\to Y$ be sequentially bounded operator between operator 
sequence spaces $X$ and $Y$, then $\varphi$ is called:
\begin{enumerate}[label = (\roman*)]
    \item \textit{sequentially contractive}, if $\Vert \varphi\Vert_{sb}\leq 1$

    \item \textit{sequentially $c$-topologically injective}, if for all 
    $n \in \mathbb{N}$ the linear operator $\varphi^{\wideparen{n}}$ is 
    $c$-topologically injective. If mentioning of constant $c$ will be 
    irrelevant we will simply say that $\varphi$ sequentially 
    topologically injective.
    
    \item \textit{sequentially (strictly) $c$-topologically surjective}, 
    if for all $n \in \mathbb{N}$ the linear operator 
    $\varphi^{\wideparen{n}}$ is (strictly) $c$-topologically surjective. 
    If mentioning of constant $c$ will be irrelevant we will simply say 
    that $\varphi$ sequentially topologically surjective.

    \item \textit{sequentially isometric}, if for all $n\in\mathbb{N}$ 
    the linear operator $\varphi^{\wideparen{n}}$ is isometric

    \item \textit{sequentially (strictly) coisometric}, if for all 
    $n\in\mathbb{N}$ the linear operator $\varphi^{\wideparen{n}}$ is 
    (strictly) coisometric
\end{enumerate}
\end{definition}
\begin{proposition}\label{PrComposeSQTopInjSur} 
Let $X$, $Y$, $Z$ be operator sequence spaces and 
$\varphi_1\in\mathcal{SB}(X,Y)$, $\varphi_2\in\mathcal{SB}(Y,Z)$. Then

$(i)$ if $\varphi_i$ is sequentially $c_i$-topologically injective for 
$i\in\mathbb{N}_2$, then $\varphi_2\varphi_1$ is sequentially 
$c_2c_1$-topologically injective.

$(ii)$ if $\varphi_i$ is (strictly) sequentially $c_i$-topologically surjective 
for $i\in\mathbb{N}_2$, then $\varphi_2\varphi_1$ is (strictly) 
sequentially $c_2c_1$-topologically surjective.
\end{proposition}
\begin{proof}
$(i)$ For each $n\in\mathbb{N}$ and $x\in X^{\wideparen{n}}$ we have 
$\Vert{(\varphi_2\varphi_1)}^{\wideparen{n}}(x)\Vert_{\wideparen{n}}
=\Vert
    \varphi_2^{\wideparen{n}}(\varphi_1^{\wideparen{n}}(x))
\Vert_{\wideparen{n}}
\geq c_2^{-1}\Vert\varphi_1^{\wideparen{n}}(x)\Vert_{\wideparen{n}}
\geq c_2^{-1}c_1^{-1}\Vert x\Vert_{\wideparen{n}}$, hence 
$\varphi_2\varphi_1$ is sequentially $c_2c_1$-topologically injective.

 Assume $\varphi_i$ is sequentially $c_i$-topologically surjective for 
$i\in\mathbb{N}_2$. From proposition~\ref{PrEquivDescOfIsomCoisomOp} for each 
$n\in\mathbb{N}$ we have 
${(\varphi_2\varphi_1)}^{\wideparen{n}}(B_{X^{\wideparen{n}}}^\circ)
=\varphi_2^{\wideparen{n}}(\varphi_1^{\wideparen{n}}(
    B_{X^{\wideparen{n}}}^\circ
))
\supset\varphi_2^{\wideparen{n}}(c_1^{-1}B_{Y^{\wideparen{n}}}^\circ)
=c_1^{-1}\varphi_2^{\wideparen{n}}(B_{Y^{\wideparen{n}}}^\circ)
=c^{-1}c_2^{-1}B_{Z^{\wideparen{n}}}^\circ$. 
Again from proposition~\ref{PrEquivDescOfIsomCoisomOp} 
we get that $\varphi_2\varphi_1$ is sequentially 
$c_2c_1$-topologically surjective.
\end{proof}


Now we can define two main categories in question. These are $SQNor$ and 
$SQNor_1$. Objects in $SQNor$ are operator sequence spaces, morphisms are 
sequentially bounded operators. Objects of $SQNor_1$ are operator sequence 
spaces, morphisms are sequentially contractive operators. 

\begin{proposition}[\cite{LamOpFolgen}, 1.2.14]\label{PrSmithsLemma}
Let $X$, $Y$ be operator sequence spaces and $d=\dim(Y)<\infty$, 
then for all $T\in\mathcal{B}(X,Y)$ holds
$$
\Vert T\Vert_{sb}=\Vert T^{\wideparen{d}}\Vert
$$
\end{proposition}

\begin{proposition}[\cite{LamOpFolgen}, 1.2.14]\label{PrSQSpaceIsSBFromT2n}
Let $X$ be operator sequence space, $n\in\mathbb{N}$. Then the linear map 
$$
i_{t_2}:X^{\wideparen{n}}\to\mathcal{SB}(t^n_2,X)
:x\mapsto\left(\xi\mapsto\sum\limits_{i=1}^n\xi_i x_i\right)
$$
is an isometric isomorphism.
\end{proposition}

The space of sequentially bounded operators between operator sequence 
spaces $X$ and $Y$ will be denoted by $\mathcal{SB}(X, Y)$. Obviously, 
this is normed space, and what is more we can define operator sequence space 
structure on $\mathcal{SB}(X, Y)$ via identification
$$
{\mathcal{SB}(X, Y)}^{\wideparen{n}} = \mathcal{SB}(X, Y^{\wideparen{n}})
$$
In this identification every $\varphi\in{\mathcal{SB}(X,Y)}^{\wideparen{n}}$ is 
mapped to the linear operator
$$
A(\varphi):X\to Y^{\wideparen{n}}:x\mapsto{(\varphi_i(x))}_{i\in\mathbb{N}_n}
$$

\begin{definition}[\cite{LamOpFolgen}, 1.2.11]\label{DefSBbiOp}
Let $\mathcal{R}:X\times Y\to Z$ be bilinear operator between operator sequence 
spaces $X$, $Y$, $Z$. For a given $n,m\in\mathbb{N}$ its $n\times m$-th 
amplification is a linear operator
$$
\mathcal{R}^{\wideparen{n\times m}}
:X^{\wideparen{n}}\times Y^{\wideparen{m}}\to Z^{\wideparen{nm}}
:(x,y)\mapsto{(\mathcal{R}(x_i,y_j))}_{i\in\mathbb{N}_n,j\in\mathbb{N}_m}
$$
Bilinear operator $\mathcal{R}$ is called sequentially bounded if
$$
\Vert\mathcal{R}\Vert_{sb}:=\sup \{\Vert \mathcal{R}^{\wideparen{n\times
m}}\Vert_{\mathcal{B}(X^{\wideparen{n}}\times Y^{\wideparen{m}},
Z^{\wideparen{nm}})}:n,m\in\mathbb{N} \}<\infty
$$
\end{definition}

\begin{definition}\label{DefSBBiOpType}
Let $\mathcal{R}:X\times Y\to Z$ be bounded bilinear operator between normed 
spaces $X$, $Y$ and $Z$, then
\begin{enumerate}[label = (\roman*)]
    \item if $Y$ and $Z$ ($X$ and $Z$) are operator sequence spaces, 
    then $\mathcal{R}$ is called sequentially isometric from the left (right) if 
    the operator 
    ${}^X\mathcal{R}
    :X\to\mathcal{SB}(Y,Z)$ ($\mathcal{R}^Y
    :Y\to\mathcal{SB}(X,Z)$) 
    is isometric.

    \item if $X$, $Y$, $Z$ are operator sequence spaces, then $\mathcal{R}$ is 
    called sequentially contractive if $\Vert \mathcal{R}\Vert_{sb}\leq 1$.
\end{enumerate}
\end{definition}

For a given operator sequence spaces $X$, $Y$, $Z$ by 
$\mathcal{SB}(X\times Y, Z)$ we will denote the space of sequentially bounded 
bilinear operators from $X\times Y$ to $Z$. Obviously, this is a normed space, 
and what is more we can define operator sequence space structure on 
$\mathcal{SB}(X\times Y, Z)$ via identification
$$
{\mathcal{SB}(X\times Y, Z)}^{\wideparen{n}}=\mathcal{SB}(X\times
Y,Z^{\wideparen{n}})
$$
where $n\in\mathbb{N}$. In this identification every 
$\mathcal{R}\in{\mathcal{SB}(X\times Y,Z)}^{\wideparen{n}}$ is mapped to the 
bilinear operator 
$$
A(\mathcal{R}):X\times Y\to
Z^{\wideparen{n}}:(x,y)\mapsto{(\mathcal{R}_i(x,y))}_{i\in\mathbb{N}_n}
$$
It is easy to check that for all $x\in X^{\wideparen{n}}$, 
$y\in Y^{\wideparen{m}}$ and $\alpha\in M_{k,n}$  holds
$$
{A({(\mathcal{R}^Y)}^{\wideparen{m}}(y))}^{\wideparen{n}}(x)
={A({({}^X\mathcal{R})}^{\wideparen{n}}(x))}^{\wideparen{m}}(y)
=\mathcal{R}^{\wideparen{n\times m}}(x,y)
$$
$$
\mathcal{R}^{\wideparen{n\times k}}(x,\alpha y)
=[\alpha,\ldots,\alpha]\mathcal{R}^{\wideparen{n\times m}}(x,y)
$$


\begin{proposition}\label{PrScalMultSB}
Let $X$ be a operator sequence space, then the bilinear operator 
$\mathcal{M}:\mathbb{C}\times X\to X:(\alpha, x)\mapsto \alpha x$ is 
sequentially contractive.
\end{proposition}
\begin{proof}
Let $\alpha\in\mathbb{C}^{\wideparen{n}}$ and $x\in X^{\wideparen{m}}$. 
Consider matrix 
$\beta
={
    [\operatorname{diag}_m(\alpha_1),\ldots,\operatorname{diag}_m(\alpha_n)]
}^{tr}$, 
then one can easily check that 
$\Vert\beta\Vert=\Vert\alpha\Vert_{\wideparen{n}}$. Now note that 
$\Vert
    \mathcal{M}^{\wideparen{n\times m}}(\alpha, x)
\Vert_{\wideparen{n\times m}}
=\Vert\beta x\Vert_{\wideparen{n\times m}}
\leq\Vert\alpha\Vert_{\wideparen{n}}\Vert x\Vert_{\wideparen{m}}$. 
Since $m,n\in\mathbb{N}$ are arbitrary $\Vert\mathcal{M}\Vert_{sb}\leq 1$.
\end{proof}


\begin{proposition}\label{PrRestrOfSBBilOpIsSB}
Let $X$, $Y$ and $Z$ be operator sequence spaces and 
$\mathcal{R}:X\times Y\to Z$ be a sequentially bounded bilinear operator, then 
for a fixed $x\in X^{\wideparen{1}}$ ($y\in Y^{\wideparen{1}}$) the linear 
operator ${}^X\mathcal{R}(x)$ ($\mathcal{R}^Y(y)$) is sequentially bounded with 
$\Vert
    {}^X\mathcal{R}(x)
\Vert_{sb}
\leq\Vert\mathcal{R}\Vert_{sb}
\Vert x\Vert_{\wideparen{1}}$ 
($\Vert
    \mathcal{R}^Y(y)
\Vert_{sb}\leq\Vert\mathcal{R}\Vert_{sb}\Vert y\Vert_{\wideparen{1}}$).
\end{proposition}
\begin{proof}
Let $n\in\mathbb{N}$ and $x\in X^{\wideparen{n}}$, then
$$
\Vert{(\mathcal{R}^Y(y))}^{\wideparen{n}}(x)\Vert_{\wideparen{n}} =\Vert
\mathcal{R}^{\wideparen{n\times 1}}(x,y)\Vert_{\wideparen{n\times 1}} \leq\Vert
\mathcal{R}\Vert_{sb}\Vert x\Vert_{\wideparen{n}}\Vert y\Vert_{\wideparen{1}}
$$
Hence 
$\Vert
    \mathcal{R}^Y(y)
\Vert_{sb}\leq\Vert\mathcal{R}\Vert_{sb}\Vert y\Vert_{\wideparen{1}}$.
For the remaining case the proof is the same.
\end{proof}

\begin{proposition}\label{PrSQNormViaDuality}
Let $Z$ be operator sequence space, $X$ ($Y$) be operator sequence space and 
$Y$ ($X$) be a normed space. Assume $\mathcal{R}:X\times Y\to Z$ is sequentially 
isometric from the right (from the left), then 
there is operator sequence space structure on $Y$ ($X$) given by family of norms
$$
\Vert
y\Vert_{\wideparen{k}}^{\mathcal{R}}=\sup \{\Vert\mathcal{R}^{\wideparen{n\times
k}}(x,y)\Vert_{\wideparen{n\times k}}:x\in B_{X^{\wideparen{n}}},
n\in\mathbb{N} \}
$$
$$
(\Vert
x\Vert_{\wideparen{k}}^{\mathcal{R}}=\sup \{\Vert\mathcal{R}^{\wideparen{k\times
n}}(x,y)\Vert_{\wideparen{k\times n}}:y\in B_{Y^{\wideparen{n}}},
n\in\mathbb{N} \})
$$
meanwhile $\Vert \mathcal{R}\Vert_{sb}\leq 1$. 
If additionally $d=\dim(Z)<\infty$, then
$$
\Vert
y\Vert_{\wideparen{k}}^{\mathcal{R}}=
\sup \{
    \Vert
        \mathcal{R}^{\wideparen{dk\times k}}(x,y)
    \Vert_{\wideparen{dk\times k}}
    :x\in B_{X^{\wideparen{dk}}} 
 \}
$$
$$
(\Vert x\Vert_{\wideparen{k}}^{\mathcal{R}}
=\sup \{
    \Vert
        \mathcal{R}^{\wideparen{k\times dk}}(x,y)
    \Vert_{\wideparen{k\times dk}}
    :y\in B_{Y^{\wideparen{dk}}} 
 \})
$$
\end{proposition}
\begin{proof}
We will consider only the case of bilinear operator sequentially isometric 
from the right. For the remaining case all arguments are the same. 
Let $y\in Y^{\wideparen{k}}$, where $k\in\mathbb{N}$. We will show that 
$\Vert y\Vert_{\wideparen{k}}$ is well defined. Indeed
$$
\Vert\mathcal{R}^{\wideparen{n\times k}}(x,y)\Vert_{\wideparen{n\times k}}
=\left\Vert\sum\limits_{j=1}^k\mathcal{R}^{\wideparen{n\times
k}}(x,{(\delta_{ji}y_i)}_{i\in\mathbb{N}_k})\right\Vert_{\wideparen{n\times k}}
\leq\sum\limits_{j=1}^k\Vert\mathcal{R}^{\wideparen{n\times
k}}(x,{(\delta_{ji}y_i)}_{i\in\mathbb{N}_k})\Vert_{\wideparen{n\times k}}
$$
$$
=\sum\limits_{j=1}^k\Vert
    \mathcal{R}^{\wideparen{n\times 1}}(x,y_j)
\Vert_{\wideparen{n\times 1}}
=\sum\limits_{j=1}^k\Vert 
    {A(\mathcal{R}^Y(y_j))}^{\wideparen{n}}(x)
\Vert_{\wideparen{n}}
$$
Now recall that $\mathcal{R}$ is sequentially isometric from the right
$$
\Vert y\Vert_{\wideparen{k}}^{\mathcal{R}} 
\leq\sum\limits_{j=1}^k\sup \{
    \Vert
        {A(\mathcal{R}^Y(y_j))}^{\wideparen{n}}(x)
    \Vert_{\wideparen{n}}
    :x\in B_{X^{\wideparen{n}}}, n\in\mathbb{N} 
 \} 
=\sum\limits_{j=1}^k
\Vert\mathcal{R}^Y(y_j)\Vert_{sb} =\sum\limits_{j=1}^k \Vert y_j\Vert<+\infty
$$
Hence, the function 
$\Vert\cdot\Vert_{\wideparen{k}}:Y^{\wideparen{k}}\to\mathbb{R}_+$ is well 
defined. It is remains to check axioms of operator sequence spaces. 
Let $\alpha\in M_{m,k}$ and $x\in X^{\wideparen{n}}$, 
then it is easy to see that
$$
\Vert\mathcal{R}^{\wideparen{n\times m}}(x,\alpha y)\Vert_{\wideparen{n\times
m}} =\Vert[\alpha,\ldots,\alpha]\mathcal{R}^{\wideparen{n\times
k}}(x,y)\Vert_{\wideparen{n\times k}}
$$
$$
\leq
\Vert
    [\alpha,\ldots,\alpha]
\Vert_{M_{m,nk}}
\Vert
    \mathcal{R}^{\wideparen{n\times k}}(x,y)
\Vert_{\wideparen{n\times k}}
=\Vert
    \alpha\Vert\Vert\mathcal{R}^{\wideparen{n\times k}}(x,y)
\Vert_{\wideparen{n\times k}}
$$
Hence
$$
\Vert\alpha y\Vert_{\wideparen{m}}^{\mathcal{R}}
=\sup \{
    \Vert
        \mathcal{R}^{\wideparen{n\times m}}(x,\alpha y)
    \Vert_{\wideparen{n\times m}}
    :x\in B_{X^{\wideparen{n}}},n\in\mathbb{N} 
 \}
\leq\sup \{
    \Vert
        \alpha
    \Vert
    \Vert
        \mathcal{R}^{\wideparen{n\times k}}(x,y)
    \Vert_{\wideparen{n\times k}}
    :x\in B_{X^{\wideparen{n}}},n\in\mathbb{N} 
 \}
$$
$$
\leq\Vert\alpha\Vert\sup \{\Vert\mathcal{R}^{\wideparen{n\times
k}}(x,y)\Vert_{\wideparen{n\times k}}:x\in
B_{X^{\wideparen{n}}},n\in\mathbb{N} \} =\Vert\alpha\Vert\Vert
y\Vert_{\wideparen{k}}^{\mathcal{R}}
$$
Let $0<l<k$ and $y={(y', y'')}^{tr}$, where $y'\in Y^{\wideparen{l}}$, 
$y''\in Y^{\wideparen{k-l}}$, then
$$
\Vert \mathcal{R}^{\wideparen{n\times k}}(x,y)\Vert_{\wideparen{n\times k}}^2
=\left\Vert
\begin{pmatrix}
    \mathcal{R}^{\wideparen{n\times l}}(x,y')\\
    \mathcal{R}^{\wideparen{n\times (k-l)}}(x,y'')
\end{pmatrix}
\right\Vert_{n\times k}^2 
\leq
\Vert
    \mathcal{R}^{\wideparen{n\times l}}(x,y')
\Vert_{\wideparen{n\times l}}^2 
+\Vert
    \mathcal{R}^{\wideparen{n\times (k-l)}}(x,y'')
\Vert_{\wideparen{n\times (k-l)}}^2
$$
Consequently,
$$
\Vert y\Vert_{\wideparen{k}}^{\mathcal{R}}{}^2
\leq\sup \{
    \Vert
        \mathcal{R}^{\wideparen{n\times l}}(x,y')
    \Vert_{\wideparen{n\times l}}^2
    :x\in B_X^{\wideparen{n}},n\in\mathbb{N} 
\} 
+\sup \{
    \Vert
        \mathcal{R}^{\wideparen{n\times (k-l)}}(x,y'')
    \Vert_{\wideparen{n\times (k-l)}}^2
    :x\in B_{X^{\wideparen{n}}},n\in\mathbb{N} 
 \}
$$
$$
=\Vert y'\Vert_{\wideparen{l}}^{\mathcal{R}}{}^2 +\Vert
y''\Vert_{\wideparen{k-l}}^{\mathcal{R}}{}^2
$$
Finally, for all $y\in Y^{\wideparen{1}}$ we have
$$
\Vert y\Vert_{\wideparen{1}}^{\mathcal{R}}
=\sup \{\Vert
    \mathcal{R}^{\wideparen{n\times 1}}(x,y)\Vert_{\wideparen{n\times 1}}
    :x\in B_{X^{\wideparen{n}}},n\in\mathbb{N} 
 \} 
=\sup \{\Vert
    {A(\mathcal{R}^Y(y))}^{\wideparen{n}}(x)\Vert_{\wideparen{n}}
    :x\in B_{X^{\wideparen{n}}},n\in\mathbb{N} 
 \}
$$
$$
=\Vert \mathcal{R}^Y(y)\Vert_{sb}=\Vert y\Vert
$$
Now from proposition~\ref{PrSQAxiomRed} it follows that, family of functions 
${(\Vert\cdot\Vert_{\wideparen{k}})}_{k\in\mathbb{N}}$ defines a 
operator sequence space structure on the normed space $Y$. 
From definition of norm on 
$Y^{\wideparen{k}}$ it follows that 
$\Vert\mathcal{R}^{\wideparen{n\times k}}\Vert\leq 1$ for all $n\in\mathbb{N}$. 
Since $n,k\in\mathbb{N}$ are arbitrary, then $\Vert\mathcal{R}\Vert_{sb}\leq 1$.

If $Z$ is finite dimensional, then from proposition~\ref{PrSmithsLemma} we get
$$
\Vert y\Vert_{\wideparen{k}}^{\mathcal{R}}
=\sup \{
    \Vert\mathcal{R}^{\wideparen{n\times k}}(x,y)
\Vert_{\wideparen{n\times k}}:x\in B_{X^{\wideparen{n}}},n\in\mathbb{N} \} 
=\sup \{
    \Vert 
        {A((\mathcal{R}^Y){}^{\wideparen{k}}(y))}^{\wideparen{n}}(x)
    \Vert_{\wideparen{n\times k}}
    :x\in B_{X^{\wideparen{n}}},n\in\mathbb{N}
 \}
$$
$$
=\Vert A((\mathcal{R}^Y){}^{\wideparen{k}}(y))\Vert_{sb} 
=\Vert  
    {A((\mathcal{R}^Y){}^{\wideparen{k}}(y))}^{\wideparen{dk}}
\Vert
=\sup \{
    \Vert
        {A((\mathcal{R}^Y){}^{\wideparen{k}}(y))}^{\wideparen{dk}}(x)
    \Vert_{\wideparen{k\times dk}}
    :x\in B_{X^{\wideparen{d}}}
 \}
$$
$$
=\sup \{
    \Vert
        \mathcal{R}^{\wideparen{dk\times k}}(x,y)
    \Vert_{\wideparen{dk\times k}}
    :x\in B_{X^{\wideparen{dk}}}
 \}
$$
\end{proof}

\begin{proposition}\label{PrFreezIsomSQIsom}
Let $Z$ be operator sequence space, $X$ ($Y$) be operator sequence space 
and $Y$ ($X$) be a normed space. Assume $\mathcal{R}:X\times Y\to Z$ is 
sequentially isometric from the right (from the left). Endow $Y(X)$ with the 
structure of operator sequence space as it was done in~\ref{PrSQNormViaDuality}, 
then the linear operator $\mathcal{R}^Y$ (${}^X\mathcal{R}$) is sequentially 
isometric.
\end{proposition}
\begin{proof}
We will consider only the case of bilinear operator sequentially isometric from 
the right. For the remaining case all arguments are the same. Let 
$k\in\mathbb{N}$. For all $y\in Y^{\wideparen{k}}$ we have
$$
\Vert y\Vert_{\wideparen{k}}
=\sup \{
    \Vert
        \mathcal{R}^{\wideparen{n\times k}}(x,y)
    \Vert_{\wideparen{n\times k}}
    :x\in B_{X^{\wideparen{n}}}
    :n\in\mathbb{N} \}
=\sup \{
    \Vert
        {A({(\mathcal{R}^Y)}^{\wideparen{k}}(y))}^{\wideparen{n}}(x)
    \Vert_{\wideparen{n\times k}}
    :x\in B_{X^{\wideparen{n}}}, n\in\mathbb{N}
 \}
$$
$$
=\Vert A({(\mathcal{R}^Y)}^{\wideparen{k}}(y))\Vert_{sb} 
=\Vert{(\mathcal{R}^Y)}^{\wideparen{k}}(y)\Vert_{\wideparen{k}}
$$
So, $\mathcal{R}^Y$ is sequentially isometric
\end{proof}

If conditions of previous proposition are satisfied we say that bilinear 
operator $\mathcal{R}$ induces operator sequence space structure on $Y$ ($X$).

\begin{proposition}\label{PrSQOpSqQuanIsEquivToStandard}
Let $X$, $Y$ be operator sequence spaces, then the standard operator sequence 
space structure of $\mathcal{SB}(X,Y)$ coincides with operator sequence space 
structure of induced by bilinear operator
$$
\mathcal{E}:X\times\mathcal{SB}(X,Y)\to Y:(x,\varphi)\mapsto\varphi(x)
$$
\end{proposition}
\begin{proof}
In this particular case statement that $\mathcal{E}$ is sequentially isometric 
from the right is tautological. Hence $\mathcal{E}$ induces operator sequence 
space structure on $\mathcal{SB}(X,Y)$. Let $k\in\mathbb{N}$ and 
$\varphi\in{\mathcal{SB}(X,Y)}^{\wideparen{k}}$. 
Obviously $\mathcal{\mathcal{E}}^{\mathcal{SB}(X,Y)}=1_{\mathcal{SB}(X,Y)}$, so
$$
\Vert
    \varphi
\Vert_{\wideparen{k}}^{\mathcal{E}}
=\sup \{
    \Vert
        \mathcal{E}^{\wideparen{n\times k}}(x,\varphi)
    \Vert_{\wideparen{n\times k}}
    :x\in B_{X^{\wideparen{n}}},n\in\mathbb{N} \}
=\sup \{
    \Vert
        {A(
            {(
                \mathcal{E}^{\mathcal{SB}(X,Y)}
            )}^{\wideparen{k}}(\varphi)
        )}^{\wideparen{n}}(x)
    \Vert_{\wideparen{n\times k}}
    :x\in B_{X^{\wideparen{n}}}, n\in\mathbb{N}
 \}
$$
$$
=\sup \{
    \Vert
        {A({(1_{\mathcal{SB}(X,Y)})}^{\wideparen{k}}(\varphi))}^{\wideparen{n}}
    \Vert
    :n\in\mathbb{N} \}
=\sup \{
    \Vert 
        {A(\varphi)}^{\wideparen{n}}
    \Vert
        n\in\mathbb{N}
 \}
=\Vert A(\varphi)\Vert_{sb}=\Vert\varphi\Vert_{\wideparen{k}}
$$
\end{proof}

























\subsection{Completion of operator sequence spaces}

\begin{definition}\label{DefSQBanSpace}
Sequential operator space $X$ is called \textit{Banach operator sequence space}, 
if $X^{\wideparen{1}}$ is a Banach space.
\end{definition}

\begin{proposition}\label{PrSQSpaceComplSoAmplISCompl}
Let $X$ be operator sequence space, $n\in\mathbb{N}$. Then $X^{\wideparen{1}}$ 
is a Banach space  if and only if  $X^{\wideparen{n}}$ is a Banach space.
\end{proposition}
\begin{proof}. Assume $X^{\wideparen{1}}$ is complete. Let 
${(x^{(k)})}_{k\in\mathbb{N}}$ be a Cauchy sequence in $X^{\wideparen{n}}$. 
Fix $\varepsilon>0$, then there exist $N\in\mathbb{N}$ such that $k,m> N$ 
implies $\Vert x^{(k)}-x^{(m)}\Vert_{X^{\wideparen{n}}}<\varepsilon$. From 
proposition~\ref{PrNormVsSQNorm} it follows that 
$\Vert x_i^{(k)}-x_i^{(m)}\Vert_{\wideparen{n}}<\varepsilon$ for 
$i\in\mathbb{N}_n$. Hence the sequences 
${(x_i^{(k)})}_{k\in\mathbb{N}}$ for $i\in\mathbb{N}_n$ are Cauchy sequences. 
Since $X^{\wideparen{1}}$ is complete, then there exist limits 
$x_i=\lim\limits_{k\to\infty}x_i^{(k)}$. Consider column 
$x={(x_i)}_{i\in\mathbb{N}_n}\in  X^{\wideparen{n}}$. 
Again from proposition~\ref{PrNormVsSQNorm} we have
$$
\lim_{k\to\infty}\Vert
x^{(k)}-x\Vert_{\wideparen{n}}
\leq\sum\limits_{i=1}^n\lim\limits_{k\to\infty}\Vert 
    x_i^{(k)}-x_i\Vert_{\wideparen{1}}
=0
$$
Thus, any Cauchy sequence 
${(x^{(k)})}_{k\in\mathbb{N}}\subset X^{\wideparen{n}}$ 
is convergent, hence $X^{\wideparen{n}}$ is a Banach space. Conversely, 
assume $X^{\wideparen{n}}$ is a Banach space. Let ${(x^{(k)})}_{k\in\mathbb{N}}$ 
be a Cauchy sequence in $X^{\wideparen{1}}$. Fix $\varepsilon>0$, then there 
exist $N\in\mathbb{N}$, such that $k,m> N$ implies 
$\Vert x^{(k)}-x^{(m)}\Vert_{\wideparen{1}}<\varepsilon$.  Consider sequence 
${(\tilde{x}^{(k)})}_{k\in\mathbb{N}}$ in $X^{\wideparen{n}}$ such that 
$\tilde{x}_i^{(k)}=x^{(k)}\delta_{1,i}$ for $i\in\mathbb{N}_n$. Then from 
proposition~\ref{PrRedundantAxiom} we see that 
$\Vert \tilde{x}^{(k)}-\tilde{x}^{(m)}\Vert_{\wideparen{n}}
=\Vert x^{(k)}-x^{(m)}\Vert_{\wideparen{1}}<\varepsilon$. Since 
$X^{\wideparen{n}}$ is complete, there exist the limit 
$\tilde{x}\in X^{\wideparen{n}}$. From proposition 
~\ref{PrNormVsSQNorm} it follows that
$$
\lim\limits_{k\to\infty}\Vert x^{(k)}-\tilde{x}_1\Vert_{\wideparen{1}}
=\lim\limits_{k\to\infty}\Vert
    {(\tilde{x}^{(k)}-\tilde{x})}_1
\Vert_{\wideparen{1}}
\leq\lim\limits_{k\to\infty}\Vert
\tilde{x}^{(k)}-\tilde{x}\Vert_{\wideparen{n}}=0
$$
Thus any Cauchy sequence ${(x^{(k)})}_{k\in\mathbb{N}}\subset X^{\wideparen{1}}$ 
is convergent, hence $X^{\wideparen{1}}$ is a Banach space. 
\end{proof}

\begin{theorem}\label{ThSQCompl}
Let $X$ be operator sequence space, $\overline{X}$ be completion of 
$X^{\wideparen{1}}$, and $j_X:X\to \overline{X}$ be isometric inclusion with 
dense image. Then there is operator sequence space structure on $\overline{X}$, 
such that $j_X$ is sequentially isometric.
\end{theorem}
\begin{proof} Let $n\in\mathbb{N}$, $\overline{x}\in \overline{X}^{n}$. Then 
for each $i\in\mathbb{N}_n$ there exist a sequence 
${(x_i^{(k)})}_{k\in\mathbb{N}}$ such that 
$\overline{x}_i=\lim\limits_{k\to\infty}j_X(x_i^{(k)})$. 
In particular, sequences ${(x_i^{(k)})}_{k\in\mathbb{N}}$ are Cauchy 
sequences in $X^{\wideparen{1}}$. For each $k\in\mathbb{N}$ consider 
$x^{(k)}={(x_i^{(k)})}_{i\in\mathbb{N}_n}\in X^{\wideparen{n}}$. By definition 
we put
$$
\Vert\overline{x}\Vert_{\wideparen{n}}=\lim\limits_{k\to\infty}\Vert
x^{(k)}\Vert_{\wideparen{n}}
$$
We will show, that this is well defined norm on $X^{\wideparen{n}}$ and what is 
more this family of norms defines operator sequence space structure on 
$\overline{X}$. Fix $\varepsilon>0$, since ${(x_i^{(k)})}_{k\in\mathbb{N}}$ are 
Cauchy sequences, then there exist $N_i\in\mathbb{N}$ for $i\in\mathbb{N}_n$ 
such that $k,m>N_i$ implies 
$\Vert x_i^{(k)}-x_i^{(m)}\Vert_{\wideparen{1}}<\varepsilon$. Consider 
$N=\max\limits_{i\in\mathbb{N}_n}N_i$, then from proposition 
~\ref{PrNormVsSQNorm} for $k,m>N$ we get
$$
\left|\Vert x^{(k)}\Vert_{\wideparen{n}}-\Vert
x^{(m)}\Vert_{\wideparen{n}}\right|\leq\Vert
x^{(k)}-x^{(m)}\Vert\leq\sum\limits_{i=1}^n\Vert
x_i^{(k)}-x_i^{(m)}\Vert_{\wideparen{1}}<n\varepsilon
$$
Thus the sequence ${(\Vert x^{(k)}\Vert)}_{k\in\mathbb{N}}$ is a Cauchy sequence 
and its limit in definition of $\Vert \overline{x}\Vert_{\wideparen{n}}$ does 
exists. Now we will show this limit does not depend on the choice of the 
sequence. Let ${(x''^{(k)})}_{k\in\mathbb{N}}$, ${(x'^{(k)})}_{k\in\mathbb{N}}$ 
be two such sequences in $X^{\wideparen{n}}$, such that 
$\overline{x}_i
=\lim\limits_{k\to\infty} j_X(x_i'^{(k)})
=\lim\limits_{k\to\infty} j_X(x_i'^{(k)})$ for all $i\in\mathbb{N}_n$. Then 
from proposition~\ref{PrNormVsSQNorm}, we have 

$$
\left|\lim\limits_{k\to\infty}\Vert
x''^{(k)}\Vert_{\wideparen{n}}-\lim\limits_{k\to\infty}\Vert
x''^{(k)}\Vert_{\wideparen{n}}\right|\leq \lim\limits_{k\to\infty}\Vert
x''^{(k)}-x'^{(k)}\Vert_{\wideparen{n}}\leq
\sum\limits_{i=1}^n\lim\limits_{k\to\infty}\Vert
x_i''^{(k)}-x_i'^{(k)}\Vert_{\wideparen{1}}\\
=\sum\limits_{i=1}^n0=0
$$
Hence this limits are equal and $\Vert \overline{x}\Vert_{\wideparen{n}}$ is 
well defined. Let $\overline{x}'\in X^{\wideparen{n}}$, 
$\overline{x}''\in X^{\wideparen{m}}$ and $\alpha\in M_{l,n}$, then
$$
\Vert\alpha\overline{x}'\Vert_{\wideparen{l}}
=\lim\limits_{k\to\infty}\Vert\alpha x'^{(k)}\Vert_{\wideparen{l}}
\leq\Vert\alpha\Vert\lim\limits_{k\to\infty}\Vert x'^{(k)}\Vert_{\wideparen{n}}
=\Vert\alpha\Vert\Vert\overline{x}'\Vert_{\wideparen{n}}
$$
$$
\left\Vert\begin{pmatrix} \overline{x}'\\
\overline{x}''\end{pmatrix}\right\Vert_{\wideparen{n+m}}^2
=\lim\limits_{k\to\infty}\left\Vert\begin{pmatrix} x'^{(k)}\\
x''^{(k)}\end{pmatrix}\right\Vert_{\wideparen{n+m}}^2
\leq\lim\limits_{k\to\infty}(\Vert x'^{(k)}\Vert_{\wideparen{n}}^2+\Vert
x''^{(k)}\Vert_{\wideparen{m}}^2)
=\Vert\overline{x}'\Vert_{\wideparen{n}}^2
+\Vert\overline{x}''\Vert_{\wideparen{m}}^2
$$
From proposition~\ref{PrSQAxiomRed} we see that functions in question defines 
operator sequence space structure on $\overline{X}$. For all 
$x\in X^{\wideparen{n}}$ consider stationary sequence 
${({j_X}^{\wideparen{n}}(x))}_{k\in\mathbb{N}}$, then
$$
\Vert j_X^{\wideparen{n}}(x)\Vert_{\wideparen{n}} =\lim\limits_{k\to\infty}\Vert
x^{(k)}\Vert_{\wideparen{n}} =\Vert x\Vert_{\wideparen{n}}
$$
So $j_X$ is sequentially isometric.
\end{proof}

\begin{proposition}\label{PrExtLinOpByCont} Let $X$ and $Y$ be operator sequence 
spaces and $\varphi\in\mathcal{SB}(X,Y)$. Then there exist unique 
$\overline{\varphi}\in\mathcal{SB}(\overline{X},\overline{Y})$ extending 
$\varphi$ and what is more 
$\Vert \overline{\varphi}\Vert_{sb}=\Vert \varphi\Vert_{sb}$
\end{proposition}
\begin{proof}
It is well known that there exist unique extension 
$\overline{\varphi}\in\mathcal{B}(\overline{X},\overline{Y})$. For a given 
$x\in X^{\wideparen{n}}$ choose any sequence 
${(x^{(k)})}_{k\in\mathbb{N}}\subset X^{\wideparen{n}}$ such that 
$\overline{x}=\lim\limits_{k\to\infty} j_X(x^{(k)})$. Then
$$
\Vert\overline{\varphi}^{\wideparen{n}}(\overline{x})\Vert_{\wideparen{n}}
=\lim\limits_{k\to\infty}\Vert
\varphi^{\wideparen{n}}(x^{(k)})\Vert_{\wideparen{n}} \leq\Vert
\varphi\Vert_{sb}\lim\limits_{k\to\infty}\Vert x^{(k)}\Vert_{\wideparen{n}}
=\Vert \varphi\Vert_{sb}\Vert \overline{x}\Vert_{\wideparen{n}}
$$
Similarly, for any $x\in X^{\wideparen{n}}$ we have
$$
\Vert \varphi^{\wideparen{n}}(x)\Vert_{\wideparen{n}}
=\Vert
    \overline{\varphi}^{\wideparen{n}}(j_X^{\wideparen{n}}(x))
\Vert_{\wideparen{n}}
=\Vert\overline{\varphi}\Vert_{sb}
\Vert
    {j_X}^{\wideparen{n}}(x)
\Vert_{\wideparen{n}}
=\Vert\overline{\varphi}\Vert_{sb}\Vert x\Vert_{\wideparen{n}}
$$
Since $n\in\mathbb{N}$ is arbitrary, then 
$\Vert \overline{\varphi}\Vert_{sb}=\Vert \varphi\Vert_{sb}$ and in particular 
$\overline{\varphi}\in\mathcal{SB}(\overline{X},\overline{Y})$
\end{proof}


\begin{proposition}\label{PrExtBilOpByCont} Let $X$, $Y$ and $Z$ be operator 
sequence spaces and $\mathcal{R}\in\mathcal{SB}(X\times Y,Z)$. Then there exist 
unique 
$\overline{\mathcal{R}}\in
\mathcal{SB}(\overline{X}\times\overline{Y},\overline{Z})$ extending 
$\mathcal{R}$ and what is more 
$\Vert\overline{\mathcal{R}}\Vert_{sb}=\Vert\mathcal{R}\Vert_{sb}$.
\end{proposition}
\begin{proof} From proposition 1.9~\cite{DefFloTensNorOpId} we know that there 
exist unique bounded bilinear extension 
$\overline{\mathcal{R}}\in
\mathcal{B}(\overline{X}\times\overline{Y},\overline{Z})$. For a given 
$\overline{x}\in \overline{X}^{\wideparen{n}}$, 
$\overline{y}\in \overline{Y}^{\wideparen{m}}$ choose any sequences 
${(x^{(k)})}_{k\in\mathbb{N}}\subset X^{\wideparen{n}}$ and 
${(y^{(k)})}_{k\in\mathbb{N}}\subset Y^{\wideparen{m}}$ such that 
$\overline{x}=\lim\limits_{k\to\infty} j_X^{\wideparen{n}}(x^{(k)})$ and 
$\overline{y}=\lim\limits_{k\to\infty} j_Y^{\wideparen{m}}(y^{(k)})$. Then 
$\overline{\mathcal{R}}^{\wideparen{n\times m}}(\overline{x},\overline{y})
=\lim\limits_{k\to\infty}\mathcal{R}^{\wideparen{n\times m}}(x^{(k)}, y^{(k)})$ 
and
$$
\Vert\overline{\mathcal{R}}^{\wideparen{n\times
m}}(\overline{x},\overline{y})\Vert_{\wideparen{n\times m}}
=\lim\limits_{k\to\infty}\Vert \mathcal{R}^{\wideparen{n\times m}}(x^{(k)},
y^{(k)})\Vert_{\wideparen{n\times m}}
\leq\Vert\mathcal{R}\Vert_{sb}\lim\limits_{k\to\infty}\Vert
x^{(k)}\Vert_{\wideparen{n}}\Vert y^{(k)}\Vert_{\wideparen{m}}
=\Vert\mathcal{R}\Vert_{sb}\Vert\overline{x}\Vert_{\wideparen{n}}\Vert
\overline{y}\Vert_{\wideparen{m}}
$$
Similarly for any $x\in X^{\wideparen{n}}$, $y\in Y^{\wideparen{m}}$ we have
$$
\Vert\mathcal{R}^{\wideparen{n\times m}}(x,y)\Vert_{\wideparen{n\times m}}
=\Vert\overline{\mathcal{R}}^{\wideparen{n\times
m}}(j_X^{\wideparen{n}}(x),j_Y^{\wideparen{m}}(y))\Vert_{\wideparen{n\times m}}
\leq\Vert\overline{\mathcal{R}}\Vert_{sb}\Vert
j_X^{\wideparen{n}}(x)\Vert_{\wideparen{n}}\Vert
j_Y^{\wideparen{m}}(y)\Vert_{\wideparen{m}}
=\Vert\overline{\mathcal{R}}\Vert_{sb}\Vert x\Vert_{\wideparen{n}}\Vert
y\Vert_{\wideparen{m}}
$$
Since $n,m\in\mathbb{N}$ are arbitrary, then 
$\Vert\overline{\mathcal{R}}\Vert_{sb}=\Vert\mathcal{R}\Vert_{sb}$ and in 
particular 
$\overline{\mathcal{R}}
\in\mathcal{SB}(\overline{X}\times\overline{Y},\overline{Z})$
\end{proof}

Now we can enlarge the list of our main categories with $SQBan$ and $SQBan_1$. 
Their definitions are similar to definitions of $SQNor$ and $SQNor_1$.










































\subsection{Duality theory for operator sequence spaces}

\begin{definition}[\cite{LamOpFolgen}, 1.3.8]\label{DeffSQDual} 
Let $X$ be operator sequence space, then by definition its sequential dual space 
is the space $X^\triangle := \mathcal{SB}(X, \mathbb{C})$. Note that here we 
consider $\mathbb{C}$ with standard operator sequence space structure from 
example~\ref{ExHilSQ}. 
\end{definition}

\begin{proposition}[\cite{LamOpFolgen}, 1.3.9]\label{PrEveryLinFuncIsSQBounded}
Let $X$ be operator sequence space, and $f\in X^\triangle$. Then for all 
$n\in\mathbb{N}$ holds $\Vert f^{\wideparen{n}}\Vert=\Vert f\Vert$, and as the 
consequence $\Vert f\Vert_{sb}=\Vert f\Vert$.
\end{proposition}

\begin{proposition}[\cite{LamOpFolgen}, 1.3.9]\label{PrSQNormsViaDuality}
Let $X$ be operator sequence space, then $\mathcal{D}_{X,X^*}$ is sequentially 
isometric from the left and from the right. What is more for all 
$n\in\mathbb{N}$, $x\in X^{\wideparen{n}}$ and 
$f\in {(X^\triangle)}^{\wideparen{n}}$ we have
$$
\Vert x\Vert_{\wideparen{n}} =\Vert x\Vert_{\wideparen{n}}^{\mathcal{D}_{X^*,X}}
\qquad\qquad \Vert f\Vert_{\wideparen{n}} =\Vert
f\Vert_{\wideparen{n}}^{\mathcal{D}_{X,X^*}}
$$
As the consequence we get that natural embedding into the second dual
$$
\iota_X:X\to X^{\triangle\triangle}
$$
is sequentially isometric.
\end{proposition}
\begin{proof}
Since standard scalar duality is isometric from the left and from the right then 
using proposition~\ref{PrEveryLinFuncIsSQBounded} we conclude that it is also 
sequentially isometric from the left and from the right. From proposition 1.3.12 
~\cite{LamOpFolgen} we know that
$$
\Vert x\Vert_{\wideparen{n}} =\sup \{\Vert
{A(f)}^{\wideparen{n}}(x)\Vert_{\wideparen{n\times n}}: f\in
B_{{(X^\triangle)}^{\wideparen{n}}} \}
\qquad
\Vert f\Vert_{\wideparen{n}} =\sup \{\Vert
{A(f)}^{\wideparen{n}}(x)\Vert_{\wideparen{n\times n}}:x\in
B_{X^{\wideparen{n}}} \}
$$
Now the desired equalities follow from identity 
$\mathcal{D}_{X,X^*}^{\wideparen{n\times n}}(x,f)
={A({(\mathcal{D}_{X,X^*}^{X^*})}^{\wideparen{n}}(f))}^{\wideparen{n}}(x)
={A(f)}^{\wideparen{n}}(x)$.
Thus we see that original operator sequence space structures of $X$ and 
$X^\triangle$ coincide with the ones induced by bilinear operators 
$\mathcal{D}_{X^*,X}$ and $\mathcal{D}_{X,X^*}$. Hence, using that standard 
scalar duality is sequentially isometric from the right, we can apply 
proposition~\ref{PrFreezIsomSQIsom} to get that operator 
$\mathcal{D}_{X^*,X}^X$ is a sequential isometry. It is remains to note that 
$\iota_X=\mathcal{D}_{X^*,X}^X$. 
\end{proof}

\begin{remark}\label{RemSqReflexiv} We will say that $X$ is sequentially 
reflexive if $\iota_X$ is sequential isometric isomorphism. By proposition 
~\ref{PrSQNormsViaDuality} operator $\iota_X$ is always sequentially 
isometric, so $\iota_X$ is a sequential isometric isomorphism if and only if 
it is surjective, which is equivalent to the usual reflexivity.
\end{remark}

\begin{proposition}\label{PrFreezDualityGetSQIsom}
Let $X$ ($Y$) be operator sequence space and $Y$ ($X$) be a normed space. 
Assume we are given a scalar duality $\mathcal{D}:X\times Y\to\mathbb{C}$ such 
that $\mathcal{D}^Y$ (${}^X\mathcal{D}$) 
are isometric isomorphisms, then if we consider $Y$ ($X$) with induced operator 
sequence space structure, then $\mathcal{D}^Y$ (${}^X\mathcal{D}$) would become 
sequentially isometric isomorphism.
\end{proposition}
\begin{proof}
We will consider the case when $\mathcal{D}^Y$ is an isometric isomorphism, for 
the remaining case all arguments are the same. Let $n\in\mathbb{N}$. By 
proposition~\ref{PrEveryLinFuncIsSQBounded} bilinear operator $\mathcal{D}$ is 
sequentially isometric from the right. Then by proposition 
~\ref{PrFreezIsomSQIsom} the linear operator $\mathcal{D}^Y$ is sequentially 
isometric, but it is also bijective, because $\mathcal{D}^Y$ is bijective. 
Therefore $\mathcal{D}^Y$ is sequentially isometric isomorphism. 
\end{proof}































\subsection{Duality theory for operators between operator sequence spaces}

\begin{proposition}[\cite{LamOpFolgen}, 1.3.14]\label{PrDualSBOp}
Let $X$, $Y$ be operator sequence spaces and $\varphi\in \mathcal{SB}(X,Y)$. 
Then $\varphi^\triangle \in\mathcal{SB}(Y^\triangle ,X^\triangle )$ and for all 
$n\in\mathbb{N}$ holds 
$\Vert{(\varphi^\triangle )}^{\wideparen{n}}\Vert
=\Vert\varphi^{\wideparen{n}}\Vert$. As the consequence, 
$\Vert\varphi^\triangle \Vert_{sb}=\Vert\varphi\Vert_{sb}$.
\end{proposition}

\begin{corollary}\label{CorDualFunc}
From proposition~\ref{PrDualSBOp} it follows that we have four well defined 
versions of functor ${}^\triangle$. They are of the form 
${}^\triangle:\mathcal{K}\to\mathcal{K}$, where 
$\mathcal{K}\in \{SQNor,SQNor_1,SQBan,SQBan_1 \}$. 
\end{corollary}

Further we will prove several technical lemmas necessary for description of 
duality for sequentially bounded operators.

\begin{definition}[\cite{LamOpFolgen}, 1.3.15]\label{DefT2n}
Let $X$ be operator sequence space and $n\in\mathbb{N}$, then by $t_2^n(X)$, we 
denote the normed space $X^n$ with the norm
$$
\Vert x\Vert_{t_2^n(X)}:=\inf\left \{\Vert\tilde{\alpha}\Vert_{hs}\Vert
\tilde{x}\Vert_{\wideparen{k}}:x=\tilde{\alpha} \tilde{x}\right \}
$$
where $\tilde{\alpha}\in M_{n,k}$, $x\in X^k$ and $k\in\mathbb{N}$. If $Y$ is an 
operator sequence space and $\varphi\in\mathcal{SB}(X,Y)$, then by 
$t_2^n(\varphi)$ we will denote the linear operator
$$
t_2^n(\varphi): t_2^n(X)\to t_2^n(Y): x\mapsto \varphi^{\wideparen{n}}(x)
$$
\end{definition}

\begin{proposition}\label{PrT2nNormProperty}
Let $X$ be a operator sequence space $n\in\mathbb{N}$, then
$$
\Vert x\Vert_{t_2^n(X)}=\inf\left \{\Vert\alpha'\Vert_{hs}\Vert
x'\Vert_{\wideparen{k}}:x=\alpha'x'\right \}
$$
where $\alpha'\in M_{n,n}$ is an invertible matrix and $x'\in X^{n}$.
\end{proposition}
\begin{proof}
Define right hand side of the equality to be proved by 
$\Vert x\Vert_{t_2^n(X)}'$. Fix $\varepsilon>0$, then there exist 
$\tilde{\alpha}\in M_{n,k}$ and $\tilde{x}\in X^{k}$, $k\in\mathbb{N}$ such that 
$x=\tilde{\alpha}\tilde{x}$ and 
$\Vert\tilde{\alpha}\Vert_{hs}\Vert\tilde{x}\Vert_{\wideparen{k}}
<\Vert x\Vert_{t_2^n(X)}+\varepsilon$. Consider polar decomposition 
$\tilde{\alpha}=|\tilde{\alpha}^*| \rho$ of matrix $\tilde{\alpha}$. 
Let $p$ be orthogonal projection on 
$\operatorname{Im}{(|\tilde{\alpha}^*|)}^\perp$. Then for all 
$\delta\in\mathbb{R}$ the matrix 
$\alpha_\delta'=|\tilde{\alpha}^*|+\delta p$ is invertible because 
$\operatorname{Ker}(\alpha_\delta')= \{0 \}$. Since 
$\alpha'_0=|\tilde{\alpha}|$ and the function $\Vert\alpha_\delta'\Vert_{hs}$ is 
continuous for $\delta\in\mathbb{R}$, then there exist such $\delta_0$ that 
$\Vert\alpha_{\delta_0}'\Vert_{hs}
<\Vert|\tilde{\alpha}^*|\Vert_{hs}
+\varepsilon\Vert \tilde{x}\Vert_{\wideparen{k}}^{-1}
=\Vert\tilde{\alpha}\Vert_{hs}
+\varepsilon\Vert \tilde{x}\Vert_{\wideparen{k}}^{-1}$. 
Denote $\alpha'=\alpha_{\delta_0}'\in M_{n,n}$ and $x'=\rho\tilde{x}\in Y^n$, 
then 
$$
\alpha'x' =(|\tilde{\alpha}^*|+\delta_0 p)\rho \tilde{x} 
=|\tilde{\alpha}^*|\rho \tilde{x}
+\delta_0 p\rho \tilde{x} 
=\tilde{\alpha}\tilde{x}
$$
By construction of polar decomposition $\Vert \rho\Vert\leq 1$ hence using 
definition of $\Vert x\Vert_{t_2^n(X)}'$ we get
$$
\Vert x\Vert_{t_2^n(X)}'
\leq \Vert\alpha'\Vert_{hs}\Vert x'\Vert_{\wideparen{n}}
\leq (\Vert\tilde{\alpha}\Vert_{hs}+\varepsilon\Vert
\tilde{x}\Vert_{\wideparen{k}})\Vert
\rho\Vert\Vert\tilde{x}\Vert_{\wideparen{n}}
\leq\Vert\tilde{\alpha}\Vert_{hs}\Vert\tilde{x}\Vert_{\wideparen{k}}+\varepsilon
\leq \Vert x\Vert_{t_2^n(X)}+2\varepsilon
$$
Since $\varepsilon>0$ is arbitrary, then 
$\Vert x\Vert_{t_2^n(X)}'\leq\Vert x\Vert_{t_2^n(X)}$. The reverse inequality 
is obvious, so $\Vert x\Vert_{t_2^n(X)}=\Vert x\Vert_{t_2^n(X)}'$.
\end{proof}

\begin{proposition}\label{PrT2nOfOpIsWellDef}
Let $X$, $Y$ be operator sequence spaces, 
$\varphi\in\mathcal{SB}(X,Y)$ and $n,k\in\mathbb{N}$. Then 
\begin{enumerate}[label = (\roman*)]
    \item for all $\alpha\in M_{n,k}$ and $x\in t_2^k(X)$ holds 
    $t_2^n(\varphi)(\alpha x)=\alpha t_2^k(\varphi)(x)$
 

    \item $t_2^n(\varphi)\in\mathcal{B}(t_2^n(X),t_2^n(Y))$, and 
    $\Vert t_2^n(\varphi)\Vert\leq\Vert\varphi^{\wideparen{n}}\Vert$

    \item if $\varphi^{\wideparen{n}}$ (strictly) $c$-topologically surjective, 
    then $t_2^n(\varphi)$ is also (strictly) $c$-topologically surjective

    \item if $\varphi^{\wideparen{n}}$ $c$-topologically injective, then 
    $t_2^n(\varphi)$ is also $c$-topologically injective
\end{enumerate}
\end{proposition}
\begin{proof}
$(i)$ Since $t_2^n(\varphi)=\varphi^{\wideparen{n}}$ as linear maps, then the 
result follows from paragraph 4 of proposition~\ref{PrSimplAmplProps}. 

$(ii)$ Let $x\in t_2^n(X)$ and $x=\alpha'x'$, where $\alpha\in M_{n,n}$ is an 
invertible matrix and $x'\in X^{n}$, then 
$t_2^n(\varphi)(x)=\alpha't_2^n(\varphi)(x')
=\alpha'\varphi^{\wideparen{n}}(x')$. Hence from the definition of the norm 
on $t_2^n(Y)$ it follows
$$
\Vert t_2^n(\varphi)(x)\Vert_{t_2^n(Y)}
\leq\Vert\alpha'\Vert_{hs}\Vert\varphi^{\wideparen{n}}(x')\Vert_{\wideparen{n}}
\leq\Vert\alpha'\Vert_{hs}\Vert\varphi^{\wideparen{n}}\Vert\Vert
x'\vert_{\wideparen{n}}
$$
Now take infimum over all representations of $x$ described above, then by 
proposition~\ref{PrT2nNormProperty} we have
$$
\Vert
t_2^n(\varphi)(x)\Vert_{t_2^n(Y)}\leq\Vert\varphi^{\wideparen{n}}\Vert\Vert
x\Vert_{t_2^n(X)}
$$
Therefore $\Vert t_2^n(\varphi)\Vert\leq\Vert\varphi^{\wideparen{n}}\Vert$ and 
$t_2^n(\varphi)\in\mathcal{B}(t_2^n(X),t_2^n(Y))$.

$(iii)$ Assume $\varphi^{\wideparen{n}}$ is $c$-topologically surjective. 
Let $y\in t_2^n(Y)$ and $y=\alpha' y'$, where $\alpha'\in M_{n,n}$ is an 
invertible matrix, $y'\in Y^n$. Let $c<c''<c'$. Since $\varphi^{\wideparen{n}}$ 
is $c$-topologically surjective, then there exist $x'\in X^n$ such that 
$\varphi^{\wideparen{n}}(x')=y'$ and $\Vert x'\Vert_{\wideparen{n}}
< c''\Vert y'\Vert_{\wideparen{n}}$. Consider 
$x:=\alpha'x'$, then $t_2^n(\varphi)(x)=\alpha't_2^n(\varphi)(x')
=\alpha'\varphi^{\wideparen{n}}(x')=\alpha' y'=y$. From definition of the norm 
on $t_2^n(X)$ we have
$$
\Vert x\Vert_{t_2^n(X)} \leq\Vert\alpha'\Vert_{hs}\Vert x'\Vert_{\wideparen{n}}
\leq\Vert\alpha'\Vert_{hs} c''\Vert y'\Vert_{\wideparen{n}}
$$
Now take infimum over all representation of $y$ described above, then 
proposition~\ref{PrT2nNormProperty} gives 
$\Vert x\Vert_{t_2^n(X)}\leq c''\Vert y\Vert_{t_2^n(Y)}
<c'\Vert y\Vert_{t_2^n(Y)}$
Thus, for all $y\in t_2^n(Y)$ and $c'>c$ there exist $x\in t_2^n(X)$ such that 
$t_2^n(\varphi)(x)=y$ and $\Vert x\Vert_{t_2^n(X)}< c'\Vert y\Vert_{t_2^n(Y)}$. 
Therefore $t_2^n(\varphi)$ is 
$c$-topologically surjective.

Assume $\varphi^{\wideparen{n}}$ is strictly $c$-topologically surjective. 
Let $y\in t_2^n(Y)$ and $y=\alpha' y'$, where $\alpha'\in M_{n,n}$ is an 
invertible matrix, $y'\in Y^n$. Since $\varphi^{\wideparen{n}}$ is  
$c$-topologically surjective, then there exist $x'\in X^n$ such that 
$\varphi^{\wideparen{n}}(x')=y'$ and 
$\Vert x'\Vert_{\wideparen{n}}\leq c\Vert y'\Vert_{\wideparen{n}}$. Consider 
$x:=\alpha'x'$, then 
$t_2^n(\varphi)(x)=\alpha't_2^n(\varphi)(x')
=\alpha'\varphi^{\wideparen{n}}(x')=\alpha' y'=y$. From the definition of the 
norm on $t_2^n(X)$ we have
$$
\Vert x\Vert_{t_2^n(X)} \leq\Vert\alpha'\Vert_{hs}\Vert x'\Vert_{\wideparen{n}}
\leq\Vert\alpha'\Vert_{hs} c\Vert y'\Vert_{\wideparen{n}}
$$
Now take infimum over all representations of $y$ described above, then 
proposition~\ref{PrT2nNormProperty} gives 
$\Vert x\Vert_{t_2^n(X)}\leq c\Vert y\Vert_{t_2^n(Y)}$
Thus, for all $y\in t_2^n(Y)$ there exist $x\in t_2^n(X)$ such that 
$t_2^n(\varphi)(x)=y$ and 
$\Vert x\Vert_{t_2^n(X)}\leq c\Vert y\Vert_{t_2^n(Y)}$. Therefore 
$t_2^n(\varphi)$ strictly $c$-topologically surjective.

$(iv)$ Assume $x\in t_2^n(X)$, then denote $y:=t_2^n(\varphi)(x)$. Consider 
representation $y=\alpha' y'$, where $\alpha'\in M_{n,n}$ is an invertible 
matrix and $y'\in Y^n$. Then 
$y'={(\alpha')}^{-1}y={(\alpha')}^{-1}t_2^n(\varphi)(x)
=t_2^n(\varphi)({(\alpha')}^{-1}x)\in\operatorname{Im}(t_n^2(\varphi))$. Since
$\varphi^{\wideparen{n}}$ is $c$-topologically injective, then it is injective, 
so for $y'\in \operatorname{Im}(t_2^n(\varphi))$ there exist $x'\in X^n$ such 
that $y'=t_2^n(\varphi)(x')=\varphi^{\wideparen{n}}(x')$. Since 
$\varphi^{\wideparen{n}}$ is $c$-topologically injective, then 
$\Vert x'\Vert_{\wideparen{n}}\leq c\Vert y'\Vert$. From the definition of the 
norm on $t_2^n(X)$ we have
$$
\Vert x\Vert_{t_2^n(X)}\leq\Vert\alpha'\Vert_{hs}\Vert
x'\Vert_{\wideparen{n}}\leq c\Vert\alpha'\Vert_{hs}\Vert y'\Vert_{\wideparen{n}}
$$
Now take infimum over all representations of $y$ described above, then 
proposition~\ref{PrT2nNormProperty} gives 
$\Vert x\Vert_{t_2^n(X)}\leq c
\Vert y\Vert_{t_2^n(Y)}=c\Vert t_2^n(\varphi)(x)\Vert_{t_2^n(Y)}$. 
Thus for all $x\in t_2^n(X)$ holds 
$\Vert t_2^n(\varphi)(x)\Vert_{t_2^n(Y)}\geq c^{-1}\Vert x\Vert_{t_2^n(X)}$. 
Therefore $t_2^n(\varphi)$ is $c$-topologically injective.
\end{proof}

\begin{proposition}[\cite{LamOpFolgen}, 1.3.16]\label{PrT2nTraingDuality}
Let $X$ be operator sequence space and $n\in\mathbb{N}$. Then we have isometric 
isomorphisms
$$
\alpha_X^n:t_2^n(X^\triangle)\to {(X^{\wideparen{n}})}^*:
f\mapsto\left(x\mapsto\sum\limits_{i=1}^n f_i(x_i)\right)
\qquad
\beta_X^n
:{(X^\triangle)}^{\wideparen{n}}\to {t_2^n(X)}^*
:f\mapsto\left(x\mapsto\sum\limits_{i=1}^n f_i(x_i)\right)
$$
\end{proposition}

\begin{proposition}\label{PrTwoTypesDualOpEquiv}
Let $X$, $Y$ be operator sequence spaces, $\varphi\in \mathcal{SB}(X,Y)$ and 
$n\in\mathbb{N}$, then 

$(i)$ ${(\varphi^\triangle)}^{\wideparen{n}}$ is $c$-topologically (surjective) 
injective $\Longleftrightarrow$ ${t_2^n(\varphi)}^*$ is $c$-topologically 
(surjective) injective

$(ii)$ $t_2^n(\varphi^\triangle)$ is $c$-topologically (surjective) 
injective $\Longleftrightarrow$ ${(\varphi^{\wideparen{n}})}^*$ is 
$c$-topologically (surjective) injective

$(iii)$ $\Vert {(\varphi^\triangle)}^{\wideparen{n}}\Vert
=\Vert {t_2^n(\varphi)}^*\Vert$ and 
$\Vert t_2^n(\varphi^\triangle)\Vert=\Vert {(\varphi^{\wideparen{n}})}^*\Vert$ 
and 
$\Vert t_2^n(\varphi)\Vert=\Vert\varphi^{\wideparen{n}}\Vert$
\end{proposition}
\begin{proof}
Let $g\in {(Y^\triangle)}^{\wideparen{n}}$ and $x\in t_2^n(X)$, then
$$
(\alpha_X^n{(\varphi^\triangle)}^{\wideparen{n}})(g)(x)
=\alpha_X^n({(\varphi^\triangle)}^{\wideparen{n}}(g))(x) =\sum\limits_{k=1}^n
{{(\varphi^\triangle)}^{\wideparen{n}}(g)}_k(x_k) =\sum\limits_{k=1}^n
(\varphi^\triangle)(g_k)(x_k) =\sum\limits_{k=1}^n g_k(\varphi(x_k))
$$
$$
({t_2^n(\varphi)}^* \alpha_Y^n)(g)(x) ={t_2^n(\varphi)}^*(\alpha_Y^n(g))(x)
=\alpha_Y^n(g)(t_2^n(\varphi)(x)) 
=\sum\limits_{k=1}^n g_k({t_2^n(\varphi)(x)}_k)
=\sum\limits_{k=1}^n g_k(\varphi(x_k))
$$
Since $g$ and $x$ are arbitrary, then 
$\alpha_X^n{(\varphi^\triangle)}^{\wideparen{n}}={t_2^n(\varphi)}^* \alpha_Y^n$. 
As $\alpha_Y^n$ and $\alpha_X^n$ are isometric isomorphisms we get that $(i)$ 
holds and 
$\Vert {(\varphi^\triangle)}^{\wideparen{n}}\Vert
=\Vert {t_2^n(\varphi)}^*\Vert$.
Let $g\in t_2^n(Y^\triangle)$ and $x\in X^{\wideparen{n}}$, then
$$
(\beta_X^n t_2^n(\varphi^\triangle))(g)(x)
=\beta_X^n(t_2^n(\varphi^\triangle)(g))(x) =\sum\limits_{k=1}^n
{t_2^n(\varphi^\triangle)(g)}_k(x_k)
=\sum\limits_{k=1}^n(\varphi^\triangle)(g_k)(x_k)
=\sum\limits_{k=1}^n g_k(\varphi(x_k))
$$
$$
({(\varphi^{\wideparen{n}})}^*\beta_Y^n)(g)(x)
={(\varphi^{\wideparen{n}})}^*(\beta_Y^n(g))(x)
=\beta_Y^n(g)(\varphi^{\wideparen{n}})(x) 
=\sum\limits_{k=1}^n {g_k(\varphi^{\wideparen{n}})(x)}_k 
=\sum\limits_{k=1}^n g_k(\varphi(x_k))
$$
Since $g$ and $x$ are arbitrary, then 
$\beta_X^n t_2^n(\varphi^\triangle)={(\varphi^{\wideparen{n}})}^*\beta_Y^n$. 
As $\beta_Y^n$ and $\beta_X^n$ are isometric isomorphisms we get that $(ii)$ 
holds and $\Vert t_2^n(\varphi^\triangle)\Vert
=\Vert {(\varphi^{\wideparen{n}})}^*\Vert$.

Finally, from propositions~\ref{PrT2nOfOpIsWellDef},~\ref{PrDualSBOp} we 
have inequalities 
$\Vert t_2^n(\varphi)\Vert
\leq\Vert\varphi^{\wideparen{n}}\Vert
=\Vert{(\varphi^\triangle)}^{\wideparen{n}}\Vert
=\Vert {t_2^n(\varphi)}^*\Vert=\Vert t_2^n(\varphi)\Vert$, so 
$\Vert t_2^n(\varphi)\Vert=\Vert\varphi^{\wideparen{n}}\Vert$.
\end{proof}

\begin{theorem}\label{ThDualSQOps}
Let $X$, $Y$ be operator sequence spaces and $\varphi\in\mathcal{SB}(X,Y)$, then
\begin{enumerate}
    \item $\varphi$ (strictly) sequentially $c$-topologically surjective 
    $\Longrightarrow$
    $ \varphi^\triangle$ sequentially $c$-topologically injective

    \item $ \varphi$ sequentially $c$-topologically injective $\Longrightarrow$
    $ \varphi^\triangle$ strictly sequentially $c$-topologically surjective

    \item $\varphi^\triangle$ (strictly) sequentially $c$-topologically 
    surjective $\Longrightarrow$
    $ \varphi$ sequentially $c$-topologically injective

    \item $ \varphi^\triangle$ sequentially $c$-topologically injective and 
    $X$ is complete $\Longrightarrow$
    $ \varphi$ sequentially $c$-topologically surjective

    \item $\varphi$ sequentially coisometric $\Longrightarrow$ 
    $\varphi^\triangle$ sequentially isometric, if $X$ is complete, 
    then the reverse implication is also true.

    \item $ \varphi$ sequentially isometric $\Longleftrightarrow$
    $\varphi^\triangle$ sequentially strictly coisometric
\end{enumerate}
\end{theorem}
\begin{proof}
For each $n\in\mathbb{N}$ we have the following chain of implications

\begin{tabular}{llllll}
$\varphi^{\wideparen{n}}$               & 
$c$-topologically injective             & 
$\implies$                              & 
$t_2^n(\varphi)$                        & 
$c$-topologically injective             &
~\ref{PrT2nOfOpIsWellDef}               \\
                                        &
                                        & 
$\implies$                              & 
${t_2^n(\varphi)}^*$                    & 
strictly $c$-topologically surjective   &
~\ref{PrDualOps}                        \\
                                        &
                                        & 
$\implies$                              &
${(\varphi^\triangle)}^{\wideparen{n}}$ & 
strictly $c$-topologically surjective   &
~\ref{PrTwoTypesDualOpEquiv}            \\
                                        &
                                        & 
$\implies$                              & 
$t_2^n(\varphi^\triangle)$              & 
strictly $c$-topologically surjective   &
~\ref{PrT2nOfOpIsWellDef}               \\
                                        &
                                        & 
$\implies$                              & 
${(\varphi^{\wideparen{n}})}^*$         & 
strictly $c$-topologically surjective   &
~\ref{PrTwoTypesDualOpEquiv}            \\
                                        &
                                        & 
$\implies$                              & 
$\varphi^{\wideparen{n}}$               & 
$c$-topologically injective             &
~\ref{PrDualOps}                        \\
\end{tabular}

So we get $(ii)$ and $(iii)$. Again for each $n\in\mathbb{N}$ we 
have the following chain of implications

\begin{tabular}{llclll}
$\varphi^{\wideparen{n}}$                           & 
(strictly) $c$-topologically surjective             & 
$\implies$                                          & 
$t_2^n(\varphi)$                                    & 
$c$-topologically surjective                        &
~\ref{PrT2nOfOpIsWellDef}                           \\
                                                    &
                                                    & 
$\implies$                                          &
${t_2^n(\varphi)}^*$                                & 
$c$-topologically injective                         &
~\ref{PrDualOps}                                    \\
                                                    &
                                                    & 
$\implies$                                          & 
${(\varphi^\triangle)}^{\wideparen{n}}$             & 
$c$-topologically injective                         &
~\ref{PrTwoTypesDualOpEquiv}                        \\
                                                    &
                                                    &
$\implies$                                          & 
$t_2^n(\varphi^\triangle)$                          & 
$c$-topologically injective                         &
~\ref{PrT2nOfOpIsWellDef}                           \\
                                                    &
                                                    & 
$\implies$                                          & 
${(\varphi^{\wideparen{n}})}^*$                     & 
$c$-topologically injective                         &
~\ref{PrTwoTypesDualOpEquiv}                        \\
                                                    &
                                                    & 
$\overset{\mbox{\tiny{$X$ complete}}}{\implies}$ & 
$\varphi^{\wideparen{n}}$                           & 
$c$-topologically surjective                        &
~\ref{PrDualOps}                                    \\
\end{tabular}

So we get $(i)$ and $(iv)$. Paragraphs $(iv)$--$(vi)$ 
are a direct consequences of $(i)$--$(iv)$ with 
$c=1$ if one takes into account that $\varphi$ is sequentially contractive  if 
and only if  $\varphi^\triangle$ is sequentially contractive 
(see proposition~\ref{PrDualSBOp}).
\end{proof}
























\subsection{Weak topologies for operator sequence spaces}


\begin{definition}\label{DefSQDconv} Let $\mathcal{D}:X\times Y\to Z$ be a 
vector duality between operator sequence spaces $X$, $Y$ and $Z$. We say that a 
net ${(y_\nu)}_{\nu\in N}\subset Y^{\wideparen{n}}$ sequentially 
$\mathcal{D}$-converges to $y\in Y^{\wideparen{n}}$ if it 
$\mathcal{D}^{\wideparen{m\times n}}$-converges for each $m\in\mathbb{N}$. 
Topology generated by this type of convergence we will denote 
by $\sigma_{\mathcal{D}}^{\widehat{n}}(Y,X)$.
\end{definition}

\begin{proposition}\label{PrDConvEquivCoordwsConv} Let 
$\mathcal{D}:X\times Y\to Z$ be a vector duality between operator sequence 
spaces $X$, $Y$ and $Z$, then the following are equivlent

$(i)$ net ${(y_\nu)}_{\nu\in N}\subset Y^{\wideparen{n}}$ sequentially 
$\mathcal{D}$-converges to $y\in Y^{\wideparen{n}}$

$(ii)$ for each $i\in\mathbb{N}_n$ the net 
${( {(y_\nu)}_i)}_{\nu\in N}\subset Y^{\wideparen{1}}$ $\mathcal{D}$-converges 
to $y_i\in Y^{\wideparen{1}}$.
\end{proposition}
\begin{proof}
$(i)\implies (ii)$ Note that for all $i\in\mathbb{N}_n$ and 
$x\in X^{\wideparen{1}}$ we have 
$\mathcal{D}(x,{(y_\nu)}_i-y_i)
={(\mathcal{D}^{\wideparen{1\times n}}(x,y_\nu-y))}_i$. 
Using proposition~\ref{PrNormVsSQNorm}, we get
$$
\lim\limits_{\nu}\Vert \mathcal{D}(x,{(y_\nu)}_i-y_i)\Vert_{\wideparen{1}}
\leq\lim\limits_{\nu}\Vert \mathcal{D}^{\wideparen{1\times
n}}(x,y_\nu-y)\Vert_{\wideparen{1\times n}}=0
$$
so ${({(y_\nu)}_i)}_{\nu\in N}$ $\mathcal{D}$-converges to $y_i$.

$(ii) \implies (i)$ Again from proposition~\ref{PrNormVsSQNorm} for all 
$m\in\mathbb{N}$ and $x\in X^{\wideparen{m}}$ we get
$$
\lim\limits_{\nu}\Vert
    \mathcal{D}^{\wideparen{m\times n}}(x,y_\nu-y)
\Vert_{\wideparen{m\times n}}
\leq\lim\limits_{\nu}\sum\limits_{j=1}^{m}\sum\limits_{i=1}^n\Vert
    \mathcal{D}{(x_j,{(y_\nu)}_i-y_i)}
\Vert_{\wideparen{1}}
=\sum\limits_{j=1}^{m}\sum\limits_{i=1}^n\lim\limits_{\nu}\Vert
    \mathcal{D}{(x_j,{(y_\nu)}_i-y_i)}
\Vert_{\wideparen{1}}=0
$$
so ${(y_\nu)}_{\nu\in N}$ sequentially $\mathcal{D}$-converges to $y$.
\end{proof}

\begin{proposition}\label{PrDContEquivCoordwsCont}
Let $\mathcal{D}_1:X_1\times Y_1\to Z_1$ and 
$\mathcal{D}_2:X_2\times Y_2\to Z_2$ be vector dualities between operator 
sequence spaces and $\varphi:Y_1\to Y_2$ be a linear operator, then $\varphi$ is 
$\sigma_{\mathcal{D}_1}(Y_1^{\wideparen{1}},X_1^{\wideparen{1}})$-
$\sigma_{\mathcal{D}_2}(Y_2^{\wideparen{1}},X_2^{\wideparen{1}})$ continuous if 
and only if $\varphi^{\wideparen{n}}$ is 
$\sigma_{\mathcal{D}_1}^{\wideparen{n}}(Y_1,X_1)$-
$\sigma_{\mathcal{D}_2}^{\wideparen{n}}(Y_2,X_2)$ continuous.
\end{proposition}
\begin{proof}
$(i)\implies (ii)$ Assume the net 
${(y_\nu)}_{\nu\in N}\subset Y^{\wideparen{n}}$ 
sequentially $\mathcal{D}_1$-converges to $y\in Y^{\wideparen{n}}$. Then by 
proposition~\ref{PrDConvEquivCoordwsConv} for each $i\in\mathbb{N}_n$ the net 
${({(y_\nu)}_i)}_{\nu\in N}$ $\mathcal{D}_1$-converges to $y_i$. 
From assumption on $\varphi$ we get that the 
net ${(\varphi({(y_\nu)}_i))}_{\nu\in N}$ 
$\mathcal{D}_2$-converges to $\varphi(y_i)$ for each $i\in\mathbb{N}_n$. Again 
by the same proposition this means that the net 
${(\varphi^{\wideparen{n}}(y_\nu))}_{\nu\in N}$ sequentially 
$\mathcal{D}_2$-converges to $\varphi^{\wideparen{n}}(y)$. Since the net 
${(y_\nu)}_{\nu\in N}$ is arbitrary, then $\varphi^{\widehat{n}}$ is 
$\sigma_{\mathcal{D}_1}^{\wideparen{n}}(Y_1,X_1)$-
$\sigma_{\mathcal{D}_2}^{\wideparen{n}}(Y_2,X_2)$ continuous.

$(ii)\implies (i)$ Assume the net 
${(y_\nu)}_{\nu\in N}\subset Y^{\wideparen{1}}$ $\mathcal{D}_1$-converges to 
$y\in Y^{\wideparen{1}}$. Define $\widetilde{y}_\nu\in Y^{\wideparen{n}}$ such 
that ${(\widetilde{y}_\nu)}_1=y_\nu$ and ${(\widetilde{y}_\nu)}_i=0$ for 
$i\in\mathbb{N}_n\setminus \{1 \}$. By proposition~\ref{PrDConvEquivCoordwsConv} 
the net $(\widetilde{y}_\nu)$ sequentially $\mathcal{D}_1$-converges to 
$y\in Y^{\wideparen{n}}$ such that $y_1=y$ and $y_i=0$ for 
$i\in\mathbb{N}_n\setminus \{1 \}$. From assumption on $\varphi^{\wideparen{n}}$ 
the net ${(\varphi^{\wideparen{n}}(\widetilde{y}_\nu))}_{\nu\in N}$ sequentially  
$\mathcal{D}_2$-converges to $\varphi^{\wideparen{n}}(\widetilde{y})$. By 
proposition~\ref{PrDConvEquivCoordwsConv} we get that the net 
${(\varphi^{\wideparen{1}}({(\widetilde{y}_\nu)}_1))}_{\nu\in N}
={(\varphi(y_\nu))}_{\nu\in N}$ $\mathcal{D}_2$-converges to 
$\varphi^{\wideparen{1}}({(\widetilde{y})}_1)=\varphi(y)$. Since the net 
${(y_\nu)}_{\nu\in N}$ is arbitrary, then $\varphi$ is 
$\sigma_{\mathcal{D}_1}(Y_1,X_1)$-$\sigma_{\mathcal{D}_2}(Y_2,X_2)$ continuous.
\end{proof}

\begin{definition}\label{DefWeakConvForSQSp} Let $X$ be an operator sequence 
space, then we define weak topology on $X^{\wideparen{n}}$ as 
$\sigma_{\mathcal{D}_{X^*,X}}^{\wideparen{n}}(X,X^*)$ topology and weak${}^*$ 
topology on ${(X^\triangle)}^{\wideparen{n}}$ as 
$\sigma_{\mathcal{D}_{X,X^*}}^{\wideparen{n}}(X^*,X)$ topology.
\end{definition}

In particular proposition~\ref{PrDConvEquivCoordwsConv} tells us that weak and 
weak${}^*$ convergence are equivalent to weak and weak${}^*$ coordinatewise 
convergence respectively. From proposition~\ref{PrDContEquivCoordwsCont} we get 
that continuity of the linear operator with respect to differerent 
weak topologies is equivalent to the continuity of the same type 
of amlified operator.

\begin{proposition}[\cite{LamOpFolgen}, 1.3.19]\label{PrDoubleDualIsom} 
Let $X$ be an operator sequence space, then there exist isometric isomorphism
$$
\widetilde{\iota_X}^n
:{(X^{\triangle\triangle})}^{\wideparen{n}}\to{(X^{\wideparen{n}})}^{**}
:\psi\mapsto\left(f\mapsto\sum\limits_{i=1}^n
\psi_i({(\alpha_X^n)}^{-1}{(f)}_i)\right)
$$
which is also a weak${}^*$-weak${}^*$ homeomorphism.
\end{proposition}
\begin{proof} From proposition~\ref{PrT2nTraingDuality} it follows that the 
desired isometric isomorphism is 
$$
\widetilde{\iota_X}^n:={({(\alpha_X^n)}^*)}^{-1}\beta_{X^\triangle}^n.
$$ 
Its action is given by the formula 
$\widetilde{\iota_X}^n(\psi)(f)=\sum_{i=1}^n \psi_i({(\alpha_X^n)}^{-1}{(f)}_i)$ 
where $\psi\in (X^{\triangle\triangle})$ and $f\in {(X^{\wideparen{n}})}^*$. 
Assume a net 
${(\psi_\nu)}_{\nu\in N}\subset {(X^{\triangle\triangle})}^{\wideparen{n}}$ 
weak${}^*$ converges to $\psi\in {(X^{\triangle\triangle})}^{\wideparen{n}}$. 
By proposition~\ref{PrDConvEquivCoordwsConv} this is equivalent to weak${}^*$ 
convergence of ${({(\psi_\nu)}_i)}_{\nu\in N}\subset X^{\triangle\triangle}$ to 
$\psi_i\in X^{\triangle\triangle}$ for each $i\in\mathbb{N}_n$. The latter is 
equivalent to convergence of the net ${({(\psi_\nu)}_i(g))}_{\nu\in N}$ to 
${(\psi)}_i(g)$ for all $g\in X^\triangle$ and $i\in\mathbb{N}_n$. 
One can easily see such converrgence is possible  if and only if  the net 
${(\sum_{i=1}^n{(\psi_\nu)}_i(g_i))}_{\nu\in N}$ converges to 
$\sum_{i=1}^n{(\psi_\nu)}_i(g_i)$ for all 
$g={(g_i)}_{i\in\mathbb{N}_n}\in t_2^n(X^\triangle)$. This is equivalent to 
convergence of the net ${(\widetilde{\iota_X}(\psi_\nu)(f))}_{\nu\in N}$ to 
$\widetilde{\iota_X}(\psi)(f)$ for all $f\in {(X^{\wideparen{n}})}^*$. 
This means that the net 
${(\widetilde{\iota_X}^n(\psi_\nu))}_{\nu\in N}
\subset {(X^{\wideparen{n}})}^{**}$ 
weak${}^*$ converges to $\widetilde{\iota_X}^n(\psi)$. Since 
$\widetilde{\iota_X}^n$ is a bijections and all steps in the proof where 
equivalences then $\widetilde{\iota_X}^n$ is a weak${}^*$-weak${}^*$ 
homeomorphism.
\end{proof}

Now we are able to proof an operator sequence space analogue of 
Goldstine theorem.

\begin{proposition}\label{PrGoldsteinTh} Let $X$ be an operator sequence space, 
then $\iota_X^{\wideparen{n}}(B_{X^{\wideparen{n}}})$ is weak${}^*$ dense in 
$B_{{(X^{\triangle\triangle})}^{\wideparen{n}}}$. As the consequence 
$\iota_X^{\wideparen{n}}(X^{\wideparen{n}})$ is weak${}^*$ dense in 
${(X^{\triangle\triangle})}^{\wideparen{n}}$.
\end{proposition} 
\begin{proof} For all $x\in X^{\wideparen{n}}$ and 
$f\in {(X^*)}^{\wideparen{n}}$ we have
$$
\widetilde{\iota_X}^n(\iota_X^{\wideparen{n}}(x))(f)
=\sum\limits_{i=1}^n{\iota_X^{\wideparen{n}}(x)}_i({(\alpha_X^n)}^{-1}{(f)}_i)
=\sum\limits_{i=1}^n\iota_X(x_i)({(\alpha_X^n)}^{-1}{(f)}_i)
=\sum\limits_{i=1}^n({(\alpha_X^n)}^{-1}{(f)}_i)(x_i)
$$
$$
=\alpha_X^n({(\alpha_X^n)}^{-1}(f))(x)=f(x)=\iota_{X^{\wideparen{n}}}(x)(f)
$$
so $\widetilde{\iota_X}^n\iota_X^{\wideparen{n}}=\iota_{X^{\wideparen{n}}}$ 
and since $\widetilde{\iota_X}^n$ is an isomorphism 
$\iota_X^{\wideparen{n}}
={(\widetilde{\iota_X}^n)}^{-1}\iota_{X^{\wideparen{n}}}$. 
By theorem 3.96~\cite{FabZizBanSpTh} we have that 
$\iota_{X^{\wideparen{n}}}(B_{X^{\wideparen{n}}})$ is weak${}^*$ dense in 
$B_{{(X^{\wideparen{n}})}^{**}}$. Since $\widetilde{\iota_X}$ is an isometric 
weak${}^*$-weak${}^*$ homeomorphism, then 
$\iota_X^{\wideparen{n}}(B_{X^{\wideparen{n}}})
={(\widetilde{\iota_X})}^{-1}\iota_{X^{\wideparen{n}}}(B_{X^{\wideparen{n}}})$ 
is weak${}^*$ dense in 
${(\widetilde{\iota_X})}^{-1}(B_{{(X^{\wideparen{n}})}^{**}})
=B_{{(X^{\triangle\triangle})}^{\wideparen{n}}}$.
\end{proof}

\begin{proposition}\label{PrWStarContExtSBOp} Let $X$ and $Y$ be two operator 
sequence spaces and $\varphi\in \mathcal{SB}(X,Y^\triangle)$. Then there 
exist unique weak${}^*$ continuous 
$\widetilde{\varphi}\in\mathcal{SB}(X^{\triangle\triangle},Y^\triangle)$ 
extending $\varphi$ and what is more 
$\Vert\widetilde{\varphi}\Vert_{sb}=\Vert\varphi\Vert$ 
\end{proposition}
\begin{proof} Denote 
$\widetilde{\varphi}
={(\varphi^\triangle\iota_Y)}^\triangle
=\iota_{Y}^\triangle\varphi^{\triangle\triangle}$. It is weak${}^*$ continuous 
as a dual of bounded operator. One can easily check that 
$\varphi^{\triangle\triangle}\iota_X=\iota_{Y^\triangle}\varphi$ and 
$\iota_Y^\triangle\iota_{Y^\triangle}=1_{Y^\triangle}$, so 
$\widetilde{\varphi}\iota_X
=\iota_{Y}^\triangle\varphi^{\triangle\triangle}\iota_X
=\iota_{Y}^\triangle\iota_{Y^\triangle}\varphi
=\varphi$ and we get that $\widetilde{\varphi}$ is weak${}^*$-continuous 
extension of $\varphi$. By proposition~\ref{PrGoldsteinTh} we have that 
$\iota_X(X)$ is weak${}^*$ dense in $X^{\triangle\triangle}$. Hence 
$\widetilde{\varphi}$ is the unique extension of  $\varphi$. From propositions 
~\ref{PrSimplAmplProps},~\ref{PrSQNormsViaDuality} and~\ref{PrDualSBOp} we have
$$
\Vert\varphi\Vert_{sb} =\Vert\widetilde{\varphi}\iota_X\Vert_{sb}
\leq\Vert\widetilde{\varphi}\Vert_{sb}\Vert\iota_X\Vert_{sb}
=\Vert\varphi\Vert_{sb}
$$
$$
\Vert\widetilde{\varphi}\Vert_{sb}
=\Vert\iota_{Y}^\triangle\varphi^{\triangle\triangle}\Vert_{sb}
\leq\Vert\iota_{Y}^\triangle\Vert_{sb}
\Vert\varphi^{\triangle\triangle}\Vert_{sb}
=\Vert\iota_{Y}\Vert_{sb}\Vert\varphi\Vert_{sb} =\Vert\varphi\Vert_{sb}
$$
So, $\Vert\widetilde{\varphi}\Vert_{sb}=\Vert\widetilde{\varphi}\Vert_{sb}$.
\end{proof}


































\subsection{Subspaces and quotients of operator sequence spaces}

\begin{definition}[\cite{LamOpFolgen}, 1.1.26]\label{DefSQSubSpace}
Let $X$ be operator sequence space, $X_0$ subspace of $X$, then there is natural 
operator sequence space structure on $X_0$ defined by 
$X_0^{\wideparen{n}}=(X_0^n,\Vert\cdot\Vert_{\wideparen{n}})$.
\end{definition}

In this case the natural inclusion $i_{X_0,X}:X_0\to X$ obviously is 
sequentially isometric.

\begin{definition}[\cite{LamOpFolgen}, 1.1.27]\label{DefSQFactorSpace}
Let $X$ be operator sequence space, and $X_0$ subspace of $X$, then there is 
natural operator sequence space structure on $X / X_0$ defined by 
identifications 
${(X / X_0)}^{\wideparen{n}} = X^{\wideparen{n}} / X_0^{\wideparen{n}}$, 
where $n\in\mathbb{N}$. 
\end{definition}

\begin{proposition}\label{PrFactorSQOp} 
Let $\varphi:X\to Y$ be a sequentially bounded operator between operator 
sequence spaces $E$ and $F$. Let $X_0$ and $Y_0$ be closed subspaces of $X$ and 
$Y$ respectively, such that $\varphi(X_0)\subset Y_0$, then there exist well 
defined sequentially bounded linear operator 
$\widehat{\varphi}:X/X_0\to Y/Y_0:x+X_0\mapsto T(x)+Y_0$ such that 
$\Vert\widehat{\varphi}^{\wideparen{n}}\Vert
\leq\Vert \varphi^{\wideparen{n}}\Vert$ for all $n\in\mathbb{N}$ so 
$\Vert\widehat{\varphi}\Vert_{sb}\leq\Vert \varphi\Vert_{sb}$. Moreover,
\begin{enumerate}
    \item if $X_0\subset \operatorname{Ker}(\varphi)\subset X_0$, then 
    $\Vert\widehat{\varphi}\Vert_{sb}=\Vert \varphi\Vert_{sb}$

    \item if $\operatorname{Ker}(\varphi)= X_0$ and $\varphi$ is sequentially 
    $c$-topologically surjective, then $\widehat{\varphi}$ is sequentially 
    $c$-topologicaly injective isomorphism

    \item if $\operatorname{Ker}(\varphi)= X_0$ and $\varphi$ is sequentially 
    coisometric, then $\widehat{\varphi}$ is a sequential isometric isomorphism
\end{enumerate}
\end{proposition}
\begin{proof}
Since for each $n\in\mathbb{N}$ we have 
$\varphi^{\widehat{n}}(X_0^{\wideparen{n}})\subset Y_0^{\wideparen{n}}$, then 
from proposition 1.5.2~\cite{HelFA} we get that 
$\Vert\widehat{\varphi}^{\wideparen{n}}\Vert
\leq\Vert \varphi^{\wideparen{n}}\Vert$, so 
$\Vert\wideparen{\varphi}\Vert_{sb}\leq\Vert\varphi\Vert_{sb}$. $(i)$ Clearly, 
$X_0^{\wideparen{n}}\subset\operatorname{Ker}(\varphi^{\wideparen{n}})$, so from 
proposition 1.5.3~\cite{HelFA} we get that 
$\Vert\widehat{\varphi}^{\wideparen{n}}\Vert
=\Vert \varphi^{\wideparen{n}}\Vert$, so 
$\Vert\wideparen{\varphi}\Vert_{sb}=\Vert\varphi\Vert_{sb}$. 
$(ii)$ Similarly, 
$X_0^{\wideparen{n}}=\operatorname{Ker}(\varphi^{\wideparen{n}})$, 
so again from lemma A.2.1~\cite{EROpSp} we get that $\varphi^{\wideparen{n}}$ is 
$c$-topologically injective isomorphism. Hence $\varphi$ is sequentially $c$ 
topologically injective ismorphism. $(iii)$ By paragraph $(i)$ we have 
$\Vert\widehat{\varphi}^{\wideparen{n}}\Vert
\leq\Vert \varphi^{\wideparen{n}}\Vert\leq 1$. By paragraph $(iii)$ we get that 
$\widehat{\varphi}$ is a $1$-topologically injective isommorphism. Therefore 
$\varphi^{\wideparen{n}}$ is an isometric isomorphism for each $n\in\mathbb{N}$. 
Hence $\varphi$ is a sequentially isometric isomorphism.
\end{proof}

Applying this propostion to $\varphi=i_{X_0,X}$ we see that the natural quotient 
mapping $\pi_{X_0,X}$ is sequentially coisometric.

\begin{proposition}[\cite{LamOpFolgen}, 1.4.13]\label{PrDualForQuotsAndSubsp} 
Let $X$ be an operator sequence space and $X_0$ its closed subspace, then 
there exist sequentially isometric ismorphisms
$$
{(X/X_0)}^\triangle= X_0^\perp\qquad\qquad X^\triangle/X_0^\perp=X_0^\triangle
$$
\end{proposition}
\begin{proof} From theorem~\ref{ThDualSQOps} operator $\pi_{X_0,X}^\triangle$ is 
sequentially isometric. Note that 
$\operatorname{Im}(\pi_{X_0,X}^\triangle)
= \{f\circ\pi_{X_0,X}:f\in {(X/X_0)}^\triangle \}
= \{g\in X^\triangle: g(X_0)= \{0 \} \}=X_0^\perp$. Hence corestriction of 
$\pi_{X_0,X}^\triangle|^{X_0^\perp}:{(X/X_0)}^\triangle\to X_0^\perp$ is a 
sequentially isometric isomorphism. Again from theorem~\ref{ThDualSQOps} 
operator $i_{X_0,X}^\triangle$ is sequentially coisometric. Note that 
$\operatorname{Ker}(i_{X_0,X}^\triangle)= \{f\in X^\triangle:f\circ i=0 \}
= \{f\in X^\triangle: f(X_0)= \{0 \} \}=X_0^\perp$. Hence by proposition 
~\ref{PrFactorSQOp} operator 
$\widehat{i_{X_0,X}^\triangle}:X^\triangle/X_0^\perp\to X_0^\triangle$ is a 
sequentially isometric isomorphism.
\end{proof}

\begin{proposition}\label{PrDualForWStarClQuotsAndSubsp} Let $X$ be a Banach 
operator sequence space and $W$ be weak${}^*$ closed subspace of $X^*$, 
then there exist sequentially isometric ismorphisms
$$
{(X/W_\perp)}^\triangle= W \qquad\qquad X^\triangle/W=W_\perp^\triangle
$$
which are weak${}^*$-weak${}^*$ homeomorphisms.
\end{proposition}
\begin{proof} Since $W$ is weak${}^*$ closed, then by theorem 4.7~\cite{RudinFA}
we have  ${(W_\perp)}^\perp=W$. Now applying proposition 
~\ref{PrDualForQuotsAndSubsp} to $X$ and $W_\perp$ we get the desired sequential 
isometric isomorphisms, they are $\pi_{W_\perp,X}^\triangle|^W$ and 
$\widehat{i_{W_\perp,X}^\triangle}$. As dual operators 
$\pi_{W_\perp,X}^\triangle$ and $i_{W_\perp,X}^\triangle$ are 
weak${}^*$-weak${}^*$ continuous. Clearly, $\pi_{W_\perp,X}^\triangle|^W$ 
is weak${}^*$-weak${}^*$ continuous as corestriction of such operator to the 
weak${}^*$ closed subspace $W$. By lemma A.2.4~\cite{BleOpAlgAndMods} operator 
$\widehat{i_{W_\perp,X}^\triangle}$ is also weak${}^*$-weak${}^*$ continuous. 
Thus  $\pi_{W_\perp,X}^\triangle|^W$ and $\widehat{i_{W_\perp,X}^\triangle}$ 
are weak${}^*$-weak${}^*$ continuous isometries, so by lemma A.2.5 
~\cite{BleOpAlgAndMods} they are weak${}^*$-weak${}^*$ homeomorphisms.
\end{proof}

























\subsection{Direct sums of operator sequence spaces}

\begin{definition}[\cite{LamOpFolgen}, 1.1.28]\label{DefSQProd}
Let $ \{X_\lambda: \lambda \in \Lambda \}$ be a family of operator sequence 
spaces. By definition their $\bigoplus_\infty$-sum is a operator sequence 
space structure on 
$\bigoplus_\infty \{X_\lambda^{\wideparen{1}}:\lambda\in \Lambda \}$, defined 
by identification
$$
{\left(
    \bigoplus{}_\infty \{X_\lambda:\lambda \in \Lambda \}
\right)}^{\wideparen{n}}
=\bigoplus{}_\infty \{X_\lambda^{\wideparen{n}}:\lambda\in \Lambda \}
$$
Also, for a given 
$x\in {\left(
    \bigoplus{}_\infty \{X_\lambda:\lambda \in \Lambda \}
\right)}^{\wideparen{n}}$ 
by $x_\lambda$ we denote element of 
$X_\lambda^{\wideparen{n}}$ such that 
${(x_\lambda)}_i={(x_i)}_\lambda$ for all $i\in\mathbb{N}_n$.
\end{definition}

\begin{proposition}\label{PrVectDualProdComp} Let 
$ \{X_\lambda:\lambda\in\Lambda \}$ and 
$ \{Z_\lambda:\lambda\in\Lambda \}$ be two families of operator sequence spaces 
and $Y$ be a operator sequence space. Let 
$\mathcal{D}_\lambda: Y\times Z_\lambda\to X_\lambda$ where 
$\lambda\in\Lambda$ is a family of vector dualities, then define vector duality
$$
\mathcal{D}
:Y\times\bigoplus{}_\infty \{Z_\lambda:\lambda\in\Lambda \}
    \to
\bigoplus{}_\infty \{X_\lambda:\lambda\in\Lambda \}
:(y,z)\mapsto
\oplus_\infty \{\mathcal{D}_\lambda(y,z_\lambda):\lambda\in\Lambda \}
$$
Assume $\mathcal{D}_\lambda^{Z_\lambda}$ is sequentially isometric for each 
$\lambda\in\Lambda$, then so does 
$\mathcal{D}^{\bigoplus{}_\infty \{Z_\lambda:\lambda\in\Lambda \}}$. 
If additionally $\mathcal{D}_\lambda^{Z_\lambda}$ is surjective for each 
$\lambda\in\Lambda$, then 
$\mathcal{D}^{\bigoplus{}_\infty \{Z_\lambda:\lambda\in\Lambda \}}$ is a 
sequential isometric isomorphism.
\end{proposition}
\begin{proof} Denote 
$Z=\bigoplus{}_\infty \{Z_\lambda:\lambda\in\Lambda \}$. 
Let $n\in\mathbb{N}$ and $z\in Z^{\wideparen{n}}$. Since 
$\mathcal{D}_\lambda^{Z_\lambda}$ is sequentially isometric, then
$$
\Vert z_\lambda\Vert_{\wideparen{n}}
=\Vert
    {(\mathcal{D}_\lambda^{Z_\lambda})}^{\wideparen{n}}(z_\lambda)
\Vert_{\wideparen{n}}
=\sup \{
    \Vert
        \mathcal{D}_\lambda^{\wideparen{k\times n}}(y,z_\lambda)
    \Vert_{\wideparen{k\times n}}
    :k\in\mathbb{N},y\in B_{Y^{\wideparen{k}}}
 \}
$$
Now note that,
$$
\begin{aligned}
\Vert{(\mathcal{D}^Z)}^{\wideparen{n}}(z)\Vert_{\wideparen{n}} &
=\Vert
    A({(\mathcal{D}^Z)}^{\wideparen{n}}(z))
\Vert_{sb}\\
&=\sup \{
    \Vert
        {A({(\mathcal{D}^Z)}^{\wideparen{n}}(z))}^{\wideparen{k}}(y)
    \Vert_{\wideparen{k\times n}}
    :k\in\mathbb{N},y\in B_{Y^{\wideparen{k}}}
 \} \\
&=\sup \{
    \Vert 
        \mathcal{D}^{\wideparen{k\times n}}(y,z)
    \Vert_{\wideparen{k\times n}}
    :k\in\mathbb{N},y\in B_{Y^{\wideparen{k}}}
 \} \\
&=\sup \{
    \Vert 
        \oplus_\infty
         \{
            \mathcal{D}_\lambda^{\wideparen{k\times n}}(y,z_\lambda)
            :\lambda\in\Lambda
         \}
    \Vert_{\wideparen{k\times n}}
    :k\in\mathbb{N},y\in B_{Y^{\wideparen{k}}}
 \} \\
&=\sup \{
    \Vert 
        \mathcal{D}_\lambda^{\wideparen{k\times n}}(y,z_\lambda)
    \Vert_{\wideparen{k\times n}}
    :k\in\mathbb{N},y\in B_{Y^{\wideparen{k}}},\lambda\in\Lambda
 \} \\
&=\sup \{
    \Vert z_\lambda\Vert_{\wideparen{n}}:\lambda\in\Lambda
 \} \\
&=\Vert z\Vert_{\wideparen{n}}
\end{aligned}
$$
Hence $\mathcal{D}^Z$ is a sequential isometry. Now consider second assumption. 
Define natural projections 
$p_\lambda
:\bigoplus{}_\infty \{X_\lambda:\lambda\in\Lambda \}
    \to 
X_\lambda:x\mapsto x_\lambda$. Take any $\varphi\in\mathcal{SB}(Y,X)$, and 
define $\varphi_\lambda=p_\lambda\varphi$. For each $\lambda\in\Lambda$ we 
know that $\mathcal{D}_\lambda^{Z_\lambda}$ is surjective, so there is 
$z_\lambda\in Z_\lambda$ such taht 
$\mathcal{D}_\lambda^{Z_\lambda}(z_\lambda)=\varphi_\lambda$. Since 
$\mathcal{D}_\lambda^{Z_\lambda}$ is isometric, then 
$\Vert z_\lambda\Vert=\Vert p_\lambda\varphi\Vert\leq\Vert \varphi\Vert$ so 
$\sup \{\Vert z_\lambda\Vert:\lambda\in\Lambda \}<\infty$. Then we have well 
defined $z\in\bigoplus{}_\infty \{Z_\lambda:\lambda\in\Lambda \}$. Note that for 
all $y\in Y$ we have
$$
\mathcal{D}^{Z}(z)(y)
=\oplus_\infty \{
    \mathcal{D}_\lambda^{Z_\lambda}(z_\lambda)(y):\lambda\in\Lambda
 \}
=\oplus_\infty \{\varphi_\lambda(y):\lambda\in\Lambda \}
=\oplus_\infty \{p_\lambda\varphi(y):\lambda\in\Lambda \} =\varphi(y)
$$
hence $\mathcal{D}^Z(z)=\varphi$. Since $\varphi$ is arbitrary, then 
$\mathcal{D}^Z$ is surjective, but it is also injective as any isometry. 
Hence $\mathcal{D}^Z$ and all its amplifications are bijective, but they are 
all isometric, therefore $\mathcal{D}^Z$ is a sequential isometric isomorphism.
\end{proof}

\begin{proposition}\label{PrSQProdUnivProp} Let 
$ \{X_\lambda:\lambda\in \Lambda \}$ be a family of operator sequence spaces, 
then
\begin{enumerate}[label = (\roman*)]
    \item there is a sequential isometric isomorphism
    $$
    \mathcal{SB}\left(
        Y,\bigoplus{}_\infty \{X_\lambda:\lambda\in\Lambda \}
    \right)
    =\bigoplus{}_\infty \{\mathcal{SB}(Y,X_\lambda):\lambda\in\Lambda \}
    $$

    \item the operator sequence space 
    $\bigoplus{}_\infty \{X_\lambda:\lambda\in\Lambda \}$ 
    with natural projections 
    $$
        p_\lambda
        :\bigoplus{}_\infty \{X_\lambda
        :\lambda\in\Lambda \}\to X_\lambda
    $$
    is a categorical product in $SQNor_1$.
\end{enumerate}
\end{proposition}
\begin{proof} $(i)$ By proposition~\ref{PrSQOpSqQuanIsEquivToStandard} vector 
dualities 
$\mathcal{E}_\lambda
:Y\times\mathcal{SB}(Y,X_\lambda)\to X_\lambda:(y,\varphi)\mapsto \varphi(y)$ 
satisfy both assumptions of proposition~\ref{PrVectDualProdComp}, hence 
$\mathcal{E}^{\bigoplus{}_\infty \{\mathcal{SB}(Y,X_\lambda)
:\lambda\in\Lambda \}}$ 
is a desired isometric isomorphism.

$(ii)$ For all $n\in\mathbb{N}$ and 
$x\in {\left(
    \bigoplus{}_\infty \{X_\lambda:\lambda\in\Lambda \}
\right)}^{\wideparen{n}}$ we have
$$
\Vert p_\lambda^{\wideparen{n}}(x)\Vert_{\wideparen{n}} 
=\Vert
    {(x_{i,\lambda})}_{i\in\mathbb{N}_n}
\Vert_{\wideparen{n}} 
\leq\sup \{
    \Vert
        {(x_{i,\lambda})}_{i\in\mathbb{N}_n}
    \Vert_{\wideparen{n}}
    :\lambda\in \Lambda 
 \}
=\Vert x\Vert_{\wideparen{n}}
$$
so $p_\lambda$ is sequentially bounded, and even sequentially contractive. 
Now consider any family of sequentially contractie operators 
$ \{\varphi_\lambda\in\mathcal{SB}(Y,X_\lambda):\lambda\in\Lambda \}$. 
By previous paragraph for  
$\varphi
=\mathcal{E}^{\bigoplus{}_\infty \{
    \mathcal{SB}(Y,X_\lambda):\lambda\in\Lambda 
 \}}
(\oplus_\infty \{\varphi_\lambda:\lambda\in\Lambda \})$ we have 
$\Vert\varphi\Vert_{sb}
=\sup \{\Vert\varphi_\lambda\Vert_{sb}:\lambda\in\Lambda \}\leq 1$. 
Moreover, for all $y\in Y$ we have
$$
p_\lambda\varphi(y)
=p_\lambda\mathcal{E}^{\bigoplus{}_\infty \{
    \mathcal{SB}(Y,X_\lambda):\lambda\in\Lambda
 \}}(\oplus_\infty \{\varphi_\lambda:\lambda\in\Lambda \})(y)
=p_\lambda(\oplus_\infty \{\varphi_\lambda(y):\lambda\in\Lambda \})
=\varphi_\lambda(y)
$$
i.e. $p_\lambda\varphi=\varphi_\lambda$. Since $Y$ and the family 
$ \{\varphi_\lambda:\lambda\in\Lambda \}$ are arbitrary, then 
$\bigoplus{}_\infty \{X_\lambda:\lambda\in\Lambda \}$ is indeed a 
product in $SQNor_1$.
\end{proof}

\begin{definition}\label{DefSQCoProd}
Let $ \{X_\lambda: \lambda \in \Lambda \}$ be a family of operator sequence 
spaces. By definition their $\bigoplus_1^0$-sum is a operator sequence space 
structure on  
$\bigoplus_1^0 \{X_\lambda^{\wideparen{1}}:\lambda\in \Lambda \}$, defined 
by embedding
$$
\bigoplus{}_1^0 \{X_\lambda:\lambda \in \Lambda \}\hookrightarrow
{\left(\bigoplus{}_\infty \{X_\lambda^\triangle:\lambda\in
\Lambda \}\right)}^\triangle
$$
\end{definition}

\begin{proposition}\label{PrVectDualCoProdComp} Let 
$ \{X_\lambda:\lambda\in\Lambda \}$ and $ \{Z_\lambda:\lambda\in\Lambda \}$ 
be two families of operator sequence spaces and $Y$ be a 
operator sequence space. Let 
$\mathcal{D}_\lambda: X_\lambda\times Z_\lambda\to Y$ where 
$\lambda\in\Lambda$ is a family of vector dualities, then define vector duality
$$
\mathcal{D}
:\bigoplus{}_1^0 \{X_\lambda:\lambda\in\Lambda \}
    \times
\bigoplus{}_\infty \{Z_\lambda:\lambda\in\Lambda \}
    \to
Y
:(x,z)
    \mapsto
\sum\limits_{\lambda\in\Lambda}\mathcal{D}_\lambda(x_\lambda,z_\lambda)
$$
Assume $\mathcal{D}_\lambda^{Z_\lambda}$ is sequentially isometric for each 
$\lambda\in\Lambda$, then so does 
$\mathcal{D}^{\bigoplus{}_\infty \{Z_\lambda:\lambda\in\Lambda \}}$. If 
additionally $\mathcal{D}_\lambda^{Z_\lambda}$ is surjective for each 
$\lambda\in\Lambda$, then 
$\mathcal{D}^{\bigoplus{}_\infty \{Z_\lambda:\lambda\in\Lambda \}}$ is a 
sequential isometric isomorphism.
\end{proposition}
\begin{proof} Denote $Z=\bigoplus{}_\infty \{Z_\lambda:\lambda\in\Lambda \}$ and 
$X=\bigoplus{}_1^0 \{X_\lambda:\lambda\in\Lambda \}$. Let $n\in\mathbb{N}$ and 
$z\in Z^{\wideparen{n}}$. Since $\mathcal{D}_\lambda^{Z_\lambda}$ is 
sequentially isometric, then
$$
\Vert z_\lambda\Vert_{\wideparen{n}}=
\Vert
    {(\mathcal{D}_\lambda^{Z_\lambda})}^{\wideparen{n}}(z_\lambda)
\Vert_{\wideparen{n}}
=\sup \{
    \Vert 
        \mathcal{D}_\lambda^{\wideparen{k\times n}}(x_\lambda,z_\lambda)
    \Vert_{\wideparen{k\times n}}
    :k\in\mathbb{N},x_\lambda\in B_{X_\lambda^{\wideparen{k}}}
 \}
$$
Now note that,
$$
\begin{aligned}
\Vert{(\mathcal{D}^Z)}^{\wideparen{n}}(z)\Vert_{\wideparen{n}} &
=\Vert  
    A({(\mathcal{D}^Z)}^{\wideparen{n}}(z))
\Vert_{sb}\\
&=\sup \{
    \Vert
        {A({(\mathcal{D}^Z)}^{\wideparen{n}}(z))}^{\wideparen{k}}(x)
    \Vert_{\wideparen{k\times n}}
    :k\in\mathbb{N},x\in B_{X^{\wideparen{k}}}
 \} \\
&=\sup \{
    \Vert 
        \mathcal{D}_{Y,Y^*}^{\wideparen{kn\times m}}(
            {A({(\mathcal{D}^Z)}^{\wideparen{n}}(z))}^{\wideparen{k}}(x),f
        )
    \Vert_{\wideparen{kn\times m}}
    :k\in\mathbb{N},x\in B_{X^{\wideparen{k}}},m\in\mathbb{N},
    f\in B_{{(Y^\triangle)}^{\wideparen{m}}}
 \} \\
\end{aligned}
$$
One can check that 
$\mathcal{D}_{Y,Y^*}(\mathcal{D}^Z(z)(x),f)
=\mathcal{D}_{
    \bigoplus{}_1^0 \{X_\lambda:\lambda\in\Lambda \},
    \bigoplus_\infty \{X_\lambda:\lambda\in\Lambda \}
}(
    x,\oplus_\infty \{
        {((\mathcal{D}_\lambda^{Z_\lambda})(z_\lambda))}^*(f)
        :\lambda\in\Lambda
     \}
)$, 
so applying proposition~\ref{PrSQNormsViaDuality} we get
$$
\begin{aligned}
\Vert{(\mathcal{D}^Z)}^{\wideparen{n}}(z)\Vert_{\wideparen{n}} &
=\sup \{
    \Vert
        \mathcal{D}_{Y,Y^*}^{\wideparen{kn\times m}}(
            {A({(\mathcal{D}^Z)}^{\wideparen{n}}(z))}^{\wideparen{k}}(x),
        f)\Vert_{\wideparen{kn\times m}}
        :k\in\mathbb{N},x\in B_{X^{\wideparen{k}}},m\in\mathbb{N},
        f\in B_{{(Y^\triangle)}^{\wideparen{m}}}
 \} \\
&=\sup \{
    \Vert
        \mathcal{D}_{
            \bigoplus{}_1^0 \{
                X_\lambda:\lambda\in\Lambda
             \},
            \bigoplus_\infty \{
                X_\lambda^*:\lambda\in\Lambda
             \}}^{\wideparen{k\times nm}}
            (
                x,
                \oplus_\infty \{
                    A(
                        {(
                            {(
                                {}^\triangle
                                \cdot
                                \mathcal{D}_\lambda^{Z_\lambda}
                            )}^{\wideparen{n}}(z_\lambda)
                        )}^{\wideparen{m}}(f)
                    )
                    :\lambda\in\Lambda
                 \}
            )
    \Vert_{\wideparen{k\times nm}}: \\
    &\qquad\qquad 
    k\in\mathbb{N},x\in B_{X^{\wideparen{k}}},m\in\mathbb{N},
    f\in B_{{(Y^\triangle)}^{\wideparen{m}}}
 \} \\
&=\sup \{
    \Vert
        \oplus_\infty \{
            A(
                {(
                    {(
                        {}^\triangle
                        \cdot
                        \mathcal{D}_\lambda^{Z_\lambda}
                    )}^{\wideparen{n}}(z_\lambda)
                )}^{\wideparen{m}}(f)
            )
            :\lambda\in\Lambda
         \}
    \Vert_{\wideparen{m\times n}}
    : m\in\mathbb{N},f\in B_{{(Y^\triangle)}^{\wideparen{m}}}
 \} \\
&=\sup \{
    \Vert
        A(
            {(
                {(
                    {}^\triangle
                    \cdot
                    \mathcal{D}_\lambda^{Z_\lambda}
                )}^{\wideparen{n}}(z_\lambda)
            )}^{\wideparen{m}}(f)
        )
    \Vert_{\wideparen{m\times n}}
    : m\in\mathbb{N},f\in B_{{(Y^\triangle)}^{\wideparen{m}}},\lambda\in\Lambda
 \} \\
\end{aligned}
$$
Apply proposition~\ref{PrSQNormsViaDuality} once again
$$
\begin{aligned}
\Vert
    {(\mathcal{D}^Z)}^{\wideparen{n}}(z)
\Vert_{\wideparen{n}} &
=\sup \{
    \Vert
        A
            {(
                {(
                    {}^\triangle
                    \cdot
                    \mathcal{D}_\lambda^{Z_\lambda}
                )}^{\wideparen{n}}(z_\lambda)
            )}^{\wideparen{m}}(f)
    \Vert_{\wideparen{m\times n}}
    : m\in\mathbb{N},f\in B_{{(Y^\triangle)}^{\wideparen{m}}},\lambda\in\Lambda
 \} \\
&=\sup \{
    \Vert
        \mathcal{D}_{X_\lambda,X_\lambda^*}^{\wideparen{l n\times m}}(
            A(
                {(
                    {(
                        \mathcal{D}_\lambda^{Z_\lambda}
                    )}^{\wideparen{n}}(z_\lambda)
                )}^{\wideparen{l}}(x_\lambda),
                f
            )
        )
    \Vert_{\wideparen{l n\times m}}
    :l\in\mathbb{N},x_\lambda\in B_{X_\lambda^{\wideparen{l}}},m\in\mathbb{N},\\
    &\qquad\qquad f\in B_{{(Y^\triangle)}^{\wideparen{m}}},\lambda\in\Lambda
 \} \\
&=\sup \{
    \Vert
        A(
            {(
                {(
                    \mathcal{D}_\lambda^{Z_\lambda}
                )}^{\wideparen{n}}(z_\lambda)
            )}^{\wideparen{l}}(x_\lambda)
    \Vert_{\wideparen{l\times n}}
    : l\in\mathbb{N},x_\lambda\in B_{X_\lambda^{\wideparen{l}}},
    \lambda\in\Lambda
 \} \\
&=\sup \{
    \Vert
        {(\mathcal{D}_\lambda^{Z_\lambda})}^{\wideparen{n}}(z_\lambda)
    \Vert_{\wideparen{n}}
    :\lambda\in\Lambda
 \} \\
&=\sup \{
    \Vert z_\lambda \Vert_{\wideparen{n}}
    :\lambda\in\Lambda
 \} \\
&=\Vert z \Vert_{\wideparen{n}}
\end{aligned}
$$
Hence $\mathcal{D}^Z$ is a sequential isometry. Now consider second assumption. 
Define natural injections 
$i_\lambda
:X_\lambda\to\bigoplus{}_1^0 \{X_\lambda:\lambda\in\Lambda \}
:x_\lambda\mapsto (\ldots,0,x_\lambda,0,\ldots)$. 
Take any 
$\varphi\in\mathcal{SB}(\bigoplus{}_1^0 \{X_\lambda:\lambda\in\Lambda \})$, and 
define $\varphi_\lambda=\varphi i_\lambda$. For each 
$\lambda\in\Lambda$ we know that $\mathcal{D}_\lambda^{Z_\lambda}$ is 
surjective, so there is $z_\lambda\in Z_\lambda$ such 
that $\mathcal{D}_\lambda^{Z_\lambda}(z_\lambda)=\varphi_\lambda$. 
Since $\mathcal{D}_\lambda^{Z_\lambda}$ is isometric, 
then $\Vert z_\lambda\Vert=\Vert p_\lambda\varphi\Vert\leq\Vert \varphi\Vert$ 
so $\sup \{\Vert z_\lambda\Vert:\lambda\in\Lambda \}<\infty$. Then we have well 
defined $z\in\bigoplus{}_\infty \{Z_\lambda:\lambda\in\Lambda \}$. Note that for 
all $x\in \bigoplus{}_1^0 \{X_\lambda:\lambda\in\Lambda \}$ we have
$$
\mathcal{D}^{Z}(z)(x)
=\sum_{\lambda\in\Lambda}\mathcal{D}_\lambda^{Z_\lambda}(z_\lambda)(x_\lambda)
=\sum_{\lambda\in\Lambda}\varphi_\lambda(x_\lambda)
=\sum_{\lambda\in\Lambda}\varphi i_\lambda(x_\lambda)
=\varphi\left(\sum_{\lambda\in\Lambda} i_\lambda(x_\lambda)\right) =\varphi(x)
$$
hence $\mathcal{D}^Z(z)=\varphi$. Since $\varphi$ is arbitrary, 
then $\mathcal{D}^Z$ is surjective, but it is also injective as any isometry. 
Hence $\mathcal{D}^Z$ and all its amplifications are bijective, but they are 
all isometric, therefore $\mathcal{D}^Z$ is a sequential isometric isomorphism.
\end{proof}

\begin{proposition}\label{PrSQCoProdUnivProp} Let 
$ \{X_\lambda:\lambda\in \Lambda \}$ be a family of operator sequence spaces, 
then
\begin{enumerate}[label = (\roman*)]
    \item there is a sequential isometric isomorphism
    $$
    \mathcal{SB}\left(\bigoplus{}_1^0 \{X_\lambda:\lambda\in\Lambda \},Y\right)
    =\bigoplus{}_\infty \{\mathcal{SB}(X_\lambda,Y):\lambda\in\Lambda \}
    $$

    \item the operator sequence space 
    $\bigoplus{}_1^0 \{X_\lambda:\lambda\in\Lambda \}$ 
    with natural injections 
    $i_\lambda:X_\lambda\to\bigoplus{}_\infty \{X_\lambda:\lambda\in\Lambda \}$ 
    is a categorical coproduct in $SQNor_1$.
\end{enumerate}
\end{proposition}
\begin{proof} 1) By proposition~\ref{PrSQOpSqQuanIsEquivToStandard} vector 
dualities 
$\mathcal{E}_\lambda
:X_\lambda\times\mathcal{SB}(X_\lambda,Y)
    \to 
Y
:(x_\lambda,\varphi)\mapsto \varphi(x_\lambda)$ satisfy both assumptions of 
proposition~\ref{PrVectDualProdComp}, 
hence $\mathcal{E}^{\bigoplus{}_\infty \{
    \mathcal{SB}(X_\lambda,Y):\lambda\in\Lambda
 \}}$ is a desired isometric isomorphism.

$(ii)$ For all $n\in\mathbb{N}$ and 
$x\in {\left(
    \bigoplus{}_1^0 \{X_\lambda:\lambda\in\Lambda
 \}\right)}^{\wideparen{n}}$ holds
$$
\begin{aligned}
\Vert 
    i_\lambda^{\wideparen{n}}(x)
\Vert_{\wideparen{n}}
&=\sup \{
    \Vert
        \mathcal{D}_{
            \bigoplus{}_1^0 \{
                X_\lambda:\lambda\in \Lambda
             \},
            \bigoplus{}_\infty \{
                X_\lambda^*:\lambda\in \Lambda
             \}}^{\wideparen{n\times n}}
            (i_\lambda^{\wideparen{n}}(x),f)
    \Vert_{\wideparen{n\times n}}
    : f\in B_{
        {(
            \bigoplus{}_\infty \{X_\lambda^\triangle:\lambda\in \Lambda \}
        )}^{\wideparen{n}}
    }
 \} \\
&=\sup \{
    \Vert
        \mathcal{D}_{X_\lambda,X_\lambda^*}^{\wideparen{n\times n}}(
            \tilde{p}_\lambda^{\wideparen{n}}(f),x
        )\Vert_{\wideparen{n\times n}}: 
        f\in B_{
            {(\bigoplus{}_\infty \{
                X_\lambda^\triangle:\lambda\in \Lambda
             \})}^{\wideparen{n}}
        }
 \} \\
&=\sup \{
    \Vert
        \mathcal{D}_{X_\lambda,X_\lambda^*}^{\wideparen{n\times n}}(f,x)
    \Vert_{\wideparen{n\times n}}
    : f\in B_{{(X_\lambda^\triangle)}^{\wideparen{n}}}
 \} \\
&=\Vert x\Vert_{\wideparen{n}}
\end{aligned}
$$ 
so $i_\lambda$ is sequentially bounded, and even sequentially isometric. 
Now consider any family of sequentially contractie operators 
$ \{\varphi_\lambda\in\mathcal{SB}(X_\lambda,Y):\lambda\in\Lambda \}$. 
By previous paragraph for  $\varphi
=\mathcal{E}^{
    \bigoplus{}_\infty \{\mathcal{SB}(X_\lambda,Y):\lambda\in\Lambda \}
}(\oplus_\infty \{\varphi_\lambda:\lambda\in\Lambda \})$ we have 
$\Vert\varphi\Vert_{sb}
=\sup \{\Vert\varphi_\lambda\Vert_{sb}:\lambda\in\Lambda \}\leq 1$. Moreover, 
for all $y\in Y$ we have
$$
\varphi i_\lambda(x_\lambda)
=\mathcal{E}^{
    \bigoplus{}_\infty \{\mathcal{SB}(Y,X_\lambda):\lambda\in\Lambda \}
}(\oplus_\infty \{\varphi_\lambda:\lambda\in\Lambda \})(i_\lambda(x_\lambda))
=\sum\limits_{\lambda'\in\Lambda}\varphi_{\lambda'}(i_\lambda(x_\lambda))
=\varphi_\lambda(x_\lambda)
$$
i.e. $\varphi i_\lambda=\varphi_\lambda$. Since $Y$ and the 
family $ \{\varphi_\lambda:\lambda\in\Lambda \}$ are arbitrary, 
then $\bigoplus{}_1^0 \{X_\lambda:\lambda\in\Lambda \}$ is indeed 
a coproduct in $SQNor_1$.
\end{proof}

\begin{proposition}\label{PrDualOfCoprodIsProd}
Let $ \{X_\lambda:\lambda\in \Lambda \}$ be a family 
of operator sequence spaces, then there exist sequentially 
isometric isomorphism
$$
{\left(\bigoplus{}_1^0 \{X_\lambda:\lambda\in \Lambda \}\right)}^\triangle
=\bigoplus{}_\infty \{X_\lambda^\triangle:\lambda\in \Lambda \}
$$
\end{proposition}
\begin{proof}
The result follows from proposition~\ref{PrSQCoProdUnivProp} 
with $Y=\mathbb{C}$.
\end{proof}

\begin{definition}\label{DefSQc0Sum}
Let $ \{X_\lambda: \lambda \in \Lambda \}$ be a family of operator sequence
space. By definition their $\bigoplus_0^0$-sum  is an operator sequence space
structure on $\bigoplus_0^0 \{X_\lambda^{\wideparen{1}}:\lambda\in \Lambda \}$,
considered 
as subspace of operator sequence space 
$\bigoplus_\infty \{X_\lambda:\lambda\in \Lambda \}$.
\end{definition}

\begin{proposition}\label{PrDensSubsetOfSumOfDoubleDuals} 
Let $ \{X_\lambda:\lambda\in\Lambda \}$ be a family of operator sequence spaces, 
then the set 
$ \{
    \bigoplus{}_\infty \{
        \iota_{X_\lambda}^{\wideparen{n}}(x_\lambda):\lambda\in \Lambda
     \}
    :x\in B_{{(\bigoplus{}_0^0 \{
        X_\lambda:\lambda\in \Lambda 
    \})}^{\wideparen{n}}}
 \}$ is weak${}^*$ dense in 
$B_{{(\bigoplus{}_\infty \{
    X_\lambda^{\triangle\triangle}:\lambda\in \Lambda
 \})}^{\wideparen{n}}}$
\end{proposition}
\begin{proof}
Let $\psi\in 
{(\bigoplus{}_\infty \{X_\lambda^{**}:\lambda\in \Lambda \})}^{\wideparen{m}}$ 
with $\Vert\psi\Vert_{\wideparen{m}}\leq 1$. 
In particular $\Vert\psi_{i,\lambda}\Vert\leq 1$ 
for all $i\in\mathbb{N}_m$ and $\lambda\in\Lambda$. 
For any $\lambda\in\Lambda$ by theorem 3.96~\cite{FabZizBanSpTh} 
we have that $\iota(B_{X_\lambda})$ is weak${}^*$ 
dense in $X_\lambda^{**}$ so for each $i\in\mathbb{N}_m$ 
we have a net $(x_{\nu,i,\lambda}'':\nu\in N_{i,\lambda})\subset B_{X_\lambda}$ 
that is weak${}^*$ converges to $\psi_{i,\lambda}$. For each $i\in\mathbb{N}_m$ 
consider poset $N_i=\prod_{\lambda\in\Lambda}N_{i,\lambda}$ 
with standard product order, 
natural projections $\pi_{i,\lambda}:N_i\to N_{i,\lambda}$ and 
define a subnet $x_{\nu,i,\lambda}'=x_{\pi_{i,\lambda}(\nu),i,\lambda}''$ 
for all $\nu\in N_i$. 
So we get a net $(x_{\nu,i,\lambda}':\nu\in N_i)$ that is weak${}^*$ 
converges to $\psi_{i,\lambda}$. The latter is equivalent to the 
weak${}^*$ convergence of the net 
$(\bigoplus_\infty \{
    \iota_{X_\lambda}(x_{\nu,i,\lambda}'):\lambda\in\Lambda
 \}:\nu\in N_i)
\subset B_{\bigoplus{}_\infty \{X_\lambda^{**}:\lambda\in \Lambda \}}$ 
to $\psi_i$. Again, consider poset $N=\prod_{i=1}^m N_i$ 
with standard product order, natural projections $\pi_i:N\to N_i$ 
and define a subnet $x_{\nu,i,\lambda}=x_{\pi_i(\nu),i,\lambda}'$ 
for all $\nu\in N$. Then we get a net 
$(\bigoplus_\infty \{
    \iota_{X_\lambda}(x_{\nu,i,\lambda}):\lambda\in\Lambda
 \}:\nu\in N)$ that weak${}^*$ converges to $\psi_i$. 
By propositon~\ref{PrDConvEquivCoordwsConv} 
we get that the net 
$(\bigoplus_\infty \{
    \iota_{X_\lambda}(x_{\nu,\lambda}):\lambda\in\Lambda
 \}:\nu\in N)$ weak${}^*$ converges to $\psi$ and thanks to  the defiition of 
the norm in $\bigoplus_\infty$-sum this net is in the unit ball of 
${(\bigoplus{}_\infty \{
    X_\lambda^{\triangle\triangle}:\lambda\in \Lambda
 \})}^{\wideparen{m}}$. The last is equivalent to the desired density result. 
\end{proof}

\begin{proposition}\label{PrDualOfc0SumIsCoProd}
Let $ \{X_\lambda:\lambda\in \Lambda \}$ be a family 
of operator sequence spaces, then there exist sequentially 
isometric isomorphism
$$
{\left(\bigoplus{}_0^0 \{X_\lambda:\lambda\in \Lambda \}\right)}^\triangle
=\bigoplus{}_1 \{X_\lambda^\triangle:\lambda\in \Lambda \}
$$
\end{proposition}
\begin{proof}
For each $n\in\mathbb{N}$ and 
$f\in {\left(
    \bigoplus{}_0^0 \{X_\lambda^\triangle:\lambda\in \Lambda \}
\right)}^{\wideparen{n}}$ we have 
$$
\Vert
    {(
        \mathcal{D}_{
            \bigoplus{}_0^0 \{X_\lambda:\lambda\in\Lambda \},
            \bigoplus{}_1 \{X_\lambda^*:\lambda\in \Lambda \}
        }^{
            \bigoplus{}_1 \{X_\lambda^*:\lambda\in \Lambda \}
        }
    )}^{\wideparen{n}}(f)
\Vert_{\wideparen{n}}=
$$
$$=\sup \{
    \Vert
        \mathcal{D}_{
            \bigoplus{}_0^0 \{X_\lambda:\lambda\in \Lambda \},
            \bigoplus{}_1 \{X_\lambda^*:\lambda\in \Lambda \}
        }^{m\times n}(x,f)
    \Vert_{\wideparen{m\times n}}
    :m\in\mathbb{N}, 
    x\in B_{\bigoplus{}_0^0 \{X_\lambda:\lambda\in \Lambda \}}
 \}
$$
$$
=\sup \{
    \Vert
        \mathcal{D}_{
            \bigoplus{}_1 \{X_\lambda^*:\lambda\in \Lambda \},
            \bigoplus{}_\infty \{X_\lambda^{**}:\lambda\in \Lambda \}
        }^{m\times n}
        (
            f,
            \oplus_\infty \{
                \iota_{X_\lambda}(x_\lambda):\lambda\in\Lambda
             \})
    \Vert_{\wideparen{m\times n}}
    :m\in\mathbb{N}, 
    x\in B_{\bigoplus{}_0^0 \{X_\lambda:\lambda\in \Lambda \}}
 \}
$$
Since tautologically $\mathcal{D}$ is weak${}^*$ continuous in the second 
variable, then from proposition~\ref{PrDensSubsetOfSumOfDoubleDuals} we get
$$
\Vert
    {(
        \mathcal{D}_{
            \bigoplus{}_0^0  \{X_\lambda:\lambda\in \Lambda  \},
            \bigoplus{}_1  \{X_\lambda^*:\lambda\in \Lambda  \}
        }^{
            \bigoplus{}_1  \{X_\lambda^*:\lambda\in \Lambda  \}
        }
    )}^{\wideparen{n}}(f)
\Vert_{\wideparen{n}}=
$$
$$
=\sup \{
    \Vert
        \mathcal{D}_{
            \bigoplus{}_1 \{X_\lambda^*:\lambda\in \Lambda \},
            \bigoplus{}_\infty \{X_\lambda^{**}:\lambda\in \Lambda \}
        }^{m\times n}
        (
            f,
            \psi
        )
    \Vert_{\wideparen{m\times n}}
    :m\in\mathbb{N}, 
    x\in B_{
        \bigoplus{}_\infty \{
            X_\lambda^{\triangle\triangle}:\lambda\in \Lambda 
        \}
    }
 \}
=\Vert f\Vert_{\wideparen{n}}
$$
Therefore 
$\mathcal{D}_{
    \bigoplus{}_0^0 \{X_\lambda:\lambda\in \Lambda \},
    \bigoplus{}_1 \{X_\lambda^*:\lambda\in \Lambda \}
}^{
    \bigoplus{}_1 \{X_\lambda^*:\lambda\in \Lambda \}
}$ is a sequential isometry, but by proposition~\ref{PrSumDuality} it is also 
bijective, hence this is the desired sequential isometric isomorphism.
\end{proof}

Similar results holds for Banach operator sequence spaces 
(just replace $\bigoplus{}_1^0$-sums and $\bigoplus{}_0^0$-sums 
with $\bigoplus{}_1$-sums and $\bigoplus{}_0$-sums).

Next proposition extensively uses terminology and 
results of~\cite{BrownItoUniquePredual}.

\begin{proposition}\label{PrUniquePredualForCoproduct}
Let $ \{X_\lambda:\lambda\in\Lambda \}$ be a family of reflexive operator 
sequence spaces, then $\bigoplus{}_\infty \{X_\lambda:\lambda\in\Lambda \}$ 
have unique (up to sequential isometry) Banach operator sequence space 
predual $\bigoplus{}_1 \{X_\lambda^\triangle:\lambda\in\Lambda \}$
\end{proposition} 
\begin{proof} For each $\lambda\in\Lambda$ the space $X_\lambda$ is reflexive, 
so it belongs to the class $(L_0)$, so by 
theorem 1~\cite{BrownItoUniquePredual} 
the space $\bigoplus_0 \{X_\lambda:\lambda\in\Lambda \}$ 
is in the class $(L_0)$. 
By remark after proposition 4~\cite{BrownItoUniquePredual} 
and propositions~\ref{PrDualOfCoprodIsProd},~\ref{PrDualOfc0SumIsCoProd} 
we get that 
$\bigoplus_0 {\{X_\lambda:\lambda\in\Lambda \}}^{**}
=\bigoplus_\infty \{X_\lambda^{**}:\lambda\in\Lambda \}
=\bigoplus_\infty \{X_\lambda:\lambda\in\Lambda \}$ 
have as Banach space unique up to isometric isomorphism predual Banach space 
${(\bigoplus_0 \{X_\lambda:\lambda\in\Lambda \})}^{*}
=\bigoplus_1 \{X_\lambda^{*}:\lambda\in\Lambda \}$. 
Since being operator seqence space predual is a stronger property than being 
Banach space predual, then the only candidate for operator 
sequence space predual of $\bigoplus{}_\infty \{X_\lambda:\lambda\in\Lambda \}$ 
is $\bigoplus{}_1 \{X_\lambda^\triangle:\lambda\in\Lambda \}$. 
By remark~\ref{RemSqReflexiv} the space $X_\lambda$ is sequentially reflexive 
for each $\lambda\in\Lambda$ and by proposition~\ref{PrDualOfCoprodIsProd} 
we get
$$
{\left(\bigoplus{}_1 \{
    X_\lambda^\triangle:\lambda\in\Lambda 
\}\right)}^\triangle
=\bigoplus{}_\infty \{X_\lambda^{\triangle\triangle}:\lambda\in\Lambda \}
=\bigoplus{}_\infty \{X_\lambda:\lambda\in\Lambda \}
$$
\end{proof}





























\subsection{Minimal and maximal structure of operator sequence space}

\begin{definition}[\cite{LamOpFolgen}, 2.1.1]\label{DefSQMin} 
Minimal structure of operator sequence space $\min(E)$ 
for a normed space $E$ is given by 
identifications ${\min(E)}^{\wideparen{n}} = \mathcal{B}(l_2^n, E)$, 
so for each $x \in E^n$ we have
$$
\Vert x\Vert_{\wideparen{n}}=\sup\left \{\left\Vert\sum\limits_{i=1}^n \xi_i
x_i\right\Vert:\xi\in B_{l_2^n}\right \}
$$
\end{definition}

\begin{proposition}[\cite{LamOpFolgen}, 2.1.4]\label{PrCharMinSQ}
Let $X$ be an operator sequence space, then the following are equivalent
\begin{enumerate}[label = (\roman*)]
    \item $X=\min(X^{\wideparen{1}})$

    \item for every operator sequence space 
    $Y$ each bounded linear operator $\varphi:Y\to X$ 
    is sequentially bounded and $\Vert\varphi\Vert_{sb}=\Vert\varphi\Vert$

    \item for every operator sequence space 
    $Y$ there is isometric isomorphism 
    ${\mathcal{SB}(Y,X)}^{\wideparen{1}}
    =\mathcal{B}(Y^{\wideparen{1}},X^{\wideparen{1}})$
\end{enumerate}
\end{proposition}

\begin{proposition}[\cite{LamOpFolgen}, 1.1.11, 2.1.5]\label{PrMinFucntor}
The map
$$
\begin{aligned}
\min : Nor_1 \to SQNor_1 : X&\mapsto \min(X)\\
\varphi&\mapsto\varphi
\end{aligned}
$$
is a covariant functor from category of normed spaces 
into the category of operator sequence spaces
\end{proposition}

Clearly, the following definition is a generalization of example~\ref{ExT2nSQ}.

\begin{definition}[\cite{LamOpFolgen}, 2.1.7]\label{DefSQMax} 
Maximal structure of operator sequence space $\max(E)$ 
for a given normed space $E$ is given by family of norms
$$
\Vert
x\Vert_{\wideparen{n}}
=\inf\left \{\Vert\alpha\Vert_{M_{n,k}}{\left(
    \sum\limits_{i=1}^k\Vert \tilde x_i\Vert^2
\right)}^{1/2}:x=\alpha\tilde x
\right \}
$$
where $x\in E^{\wideparen{n}}$, $\alpha\in M_{n,k}$, $\tilde{x}\in E^k$.
\end{definition}

\begin{proposition}[\cite{LamOpFolgen}, 2.1.9]\label{PrCharMaxSQ}
Let $X$ be an operator sequence space, then the following are equivalent
\begin{enumerate}[label = (\roman*)]
    \item $X=\max(X^{\wideparen{1}})$

    \item for every operator sequence space $Y$ 
    each bounded linear operator $\varphi:X\to Y$ 
    is sequentially bounded and $\Vert\varphi\Vert_{sb}=\Vert\varphi\Vert$

    \item for every operator sequence space 
    $Y$ there is isometric isomorphism 
    ${\mathcal{SB}(X,Y)}^{\wideparen{1}}
    =\mathcal{B}(X^{\wideparen{1}},Y^{\wideparen{1}})$
\end{enumerate}
\end{proposition}

\begin{proposition}[\cite{LamOpFolgen}, 1.1.11, 2.1.10]\label{PrMaxFucntor}
The map
$$
\begin{aligned}
\max : Nor_1 \to SQNor_1 : X&\mapsto \max(X)\\
\varphi&\mapsto\varphi
\end{aligned}
$$
is a covariant functor from the category of normed 
spaces into the category of operator sequence spaces.
\end{proposition}

\begin{proposition}\label{PrMinPreserveEmbedings} 
Let $\varphi:E\to F$ be bounded linear operator 
between normed spaces $E$ and $F$, then 
\begin{enumerate}[label = (\roman*)]
    \item if $\varphi$ is $c$-topologically injective, 
    then $\min(\varphi)$ is sequentially $c$-topologically injective

    \item if $\varphi$ is isometric, then 
    $\min(\varphi)$ is sequentially isometric
\end{enumerate}
\end{proposition}
\begin{proof} $(i)$ For each $n\in\mathbb{N}$ and 
$x\in {\min(E)}^{\wideparen{n}}$ we have
$$
\Vert {\min(\varphi)}^{\wideparen{n}}(x)\Vert_{\wideparen{n}}
=\sup\left \{\left\Vert\sum\limits_{i=1}^n\xi_i
\varphi^{\wideparen{n}}{(x)}_i\right\Vert:\xi\in B_{l_2^n}\right \}
=\sup\left \{\left\Vert\sum\limits_{i=1}^n\xi_i \varphi(x_i)\right\Vert:\xi\in
B_{l_2^n}\right \}
$$
$$
=\sup\left \{\left\Vert\varphi\left(\sum\limits_{i=1}^n\xi_i
x_i\right)\right\Vert:\xi\in B_{l_2^n}\right \} \geq
c^{-1}\sup\left \{\left\Vert\sum\limits_{i=1}^n\xi_i x_i\right\Vert:\xi\in
B_{l_2^n}\right \} =c^{-1}\Vert x\Vert_{\wideparen{n}}
$$
Hence $\min(\varphi)$ is sequentially $c$-topologically injective.

$(ii)$ By previous paragraph $\min(\varphi)$ is $1$-topologically injective. 
On the other hand, by proposition~\ref{PrCharMinSQ} we have 
$\Vert\min(\varphi)\Vert_{sb}=\Vert\varphi\Vert=1$. 
Therefore $\min(\varphi)$ is sequentially isometric.
\end{proof}

\begin{proposition}\label{PrMinCommuteWithProd} 
Let $ \{X_\lambda:\lambda\in\Lambda \}$ be a family of minimal operator 
sequence spaces, then $\bigoplus{}_\infty \{X_\lambda:\lambda\in\Lambda \}$ 
is also minimal.
\end{proposition} 
\begin{proof}
Let $Y$ be arbitrary operator sequence space, 
then from propositions~\ref{PrSQProdUnivProp},~\ref{PrCharMinSQ} 
and~\ref{PrCharMaxSQ} we have isometric identifications
$$
\mathcal{SB}{\left(
    Y,\bigoplus{}_\infty \{X_\lambda:\lambda\in\Lambda \}
\right)}^{\wideparen{1}}
=\bigoplus{}_\infty \{
    {\mathcal{SB}(Y,X_\lambda)}^{\wideparen{1}}:\lambda\in\Lambda
 \}
=\bigoplus{}_\infty \{
    \mathcal{B}(Y^{\wideparen{1}},X_\lambda^{\wideparen{1}}):\lambda\in\Lambda
 \}
$$
$$
=\bigoplus{}_\infty \{
    {\mathcal{SB}(\max(Y^{\wideparen{1}}),X_\lambda)}^{\wideparen{1}}
    :\lambda\in\Lambda
 \}
=\mathcal{SB}{\left(
    \max(Y^{\wideparen{1}}),\bigoplus{}_\infty \{
        X_\lambda:\lambda\in\Lambda
     \}
\right)}^{\wideparen{1}}
$$
$$
=\mathcal{B}\left(
    Y^{\wideparen{1}},{\left(\bigoplus{}_\infty \{
        X_\lambda:\lambda\in\Lambda
     \}\right)}^{\wideparen{1}}
\right)
$$
Since $Y$ is arbitrary, from proposition~\ref{PrCharMinSQ} we conclude 
that $\bigoplus{}_\infty \{X_\lambda:\lambda\in\Lambda \}$ have minimal operator 
sequence space structure.
\end{proof}

\begin{proposition}\label{PrCommCstarAlgIsMin} Let $A$ be a commutative $C^*$ 
algebra and $X$ be an operator sequence space, then every bounded linear 
operator $\varphi:X\to A$ is sequentially bounded 
with $\Vert\varphi\Vert_{sb}=\Vert\varphi\Vert$. As the consequence the 
standard operator sequence space structure of $A$ is minimal.
\end{proposition}
\begin{proof} As $A$ is a commutatitive $C^*$ algebra, by Gelfand-Naimark 
theorem 2.1.10~\cite{MurphCstarOpTh} we may assume that $A=C_0(\Omega)$. 
Using proposition~\ref{PrCstarAlgSQ} for any $n\in\mathbb{N}$ 
and $x\in X^{\wideparen{n}}$ we have 
$$
\Vert\varphi^{\wideparen{n}}(x)\Vert_{\wideparen{n}} 
=\Vert i_C(\varphi^{\wideparen{n}}(x))\Vert 
=\sup \{
    \Vert
        i_C(\varphi^{\wideparen{n}}(x))(\omega)
    \Vert
    :\omega\in\Omega
 \} 
=\sup \{
    \langle
        i_C(\varphi^{\wideparen{n}}(x))(\omega),\xi
    \rangle
    :\omega\in\Omega,\xi\in B_{\mathbb{C}^n}
 \}
$$
$$
=\sup\left \{
    \left|\sum_{i=1}^n \varphi(x_i)(\omega)\overline{\xi_i}\right|
    :\omega\in\Omega,\xi\in B_{\mathbb{C}^n}
\right \}
=\sup\left \{
    \left| \varphi\left(\sum_{i=1}^n \overline{\xi_i} x_i\right)(\omega)\right|
    :\omega\in\Omega,\xi\in B_{\mathbb{C}^n}
\right \}
$$
$$
=\sup\left \{\left\Vert \varphi\left(\sum_{i=1}^n \overline{\xi_i}
x_i\right)\right\Vert:\xi\in B_{\mathbb{C}^n}\right \}
\leq\Vert\varphi\Vert\sup\left \{\left\Vert \sum_{i=1}^n \overline{\xi_i}
x_i\right\Vert_{\wideparen{n}}:\xi\in B_{\mathbb{C}^n}\right \}
$$
$$
\leq
\Vert\varphi\Vert
\Vert x\Vert_{\wideparen{n}}
\sup \{
    \Vert
        \operatorname{diag}_n(\overline{\xi_1},\ldots,\overline{\xi_n})
    \Vert
:\xi\in B_{\mathbb{C}^n}
 \}
=\Vert\varphi\Vert\Vert x\Vert_{\wideparen{n}}
\sup\left \{
    \max_{i\in\mathbb{N}_n}|\overline{\xi_i}|
    :\xi\in B_{\mathbb{C}^n}
\right \}
\leq\Vert\varphi\Vert\Vert x\Vert_{\wideparen{n}}
$$
Therefore $\Vert\varphi\Vert_{sb}\leq\Vert\varphi\Vert$. Since we always 
have $\Vert\varphi\Vert\leq\Vert\varphi\Vert_{sb}$, then we get the desired 
equality. As operator sequence space $X$ is arbitrary, from 
proposition~\ref{PrCharMinSQ} we see that $A$ have minimal operator 
sequence space structure.
\end{proof}

\begin{proposition}\label{PrMinIsSubspOfCommCstarAlg} Let $X$ be an operator 
sequence space, then $X$ is minimal if and only if there exist sequential 
isometry from $X$ into $C(\Omega)$ for some compact topological space $\Omega$.
\end{proposition}
\begin{proof} 
Assume $X$ have minimal structure. Consider natural 
isometry $i:X\to C(B_{X^*})$ (see A1~\cite{DefFloTensNorOpId}). By proposition 
~\ref{PrMinPreserveEmbedings} we know that 
$\min(i):\min(X^{\wideparen{1}})\to\min({C(B_{X^*})}^{\wideparen{1}})$ is 
sequentially isometric. By proposition~\ref{PrCommCstarAlgIsMin} we 
have $\min({C(B_{X^*})}^{\wideparen{1}})=C(B_{X^*})$ and by 
assumption $\min(X^{\wideparen{1}})=X$, so we get the desired sequential 
isometry $\min(i):X\to C(B_{X^*})$.

Conversely, assume we are given sequential isometry $i:X\to C(\Omega)$. 
Since $i^{\wideparen{1}}:X^{\wideparen{1}}\to {C(\Omega)}^{\wideparen{1}}$ is 
an isometry, by proposition~\ref{PrMinPreserveEmbedings} we have sequential 
isometry $\min(i):\min(X^{\wideparen{1}})\to\min({C(\Omega)}^{\wideparen{1}})$. 
By proposition~\ref{PrCommCstarAlgIsMin} 
we have $\min({C(\Omega)}^{\wideparen{1}})=C(\Omega)$, so we have one more 
sequential isometry $\min(i):\min(X^{\wideparen{1}})\to C(\Omega)$. 
Since $i=\min(i)$ as linear maps we conclude that $X=\min(X^{\wideparen{1}})$ 
\end{proof}

\begin{proposition}\label{PrMaxPreserveQuotients} Let $\varphi:E\to F$ be 
bounded linear operator between normed spaces $E$ and $F$, then

$(i)$ if $\varphi$ is $c$-topologically surjective, then $\max(\varphi)$ 
is sequentially $c$-topologically surjective

$(ii)$ is $\varphi$ is coisometric, then $\max(\varphi)$ 
is sequentially coisometric
\end{proposition}
\begin{proof} $(i)$ By lemma A.2.1~\cite{EROpSp} we know that 
$\widehat{\varphi}:E/\operatorname{Ker}(\varphi)\to F$ 
is $c^{-1}$-topologically injective isomorphism of normed spaces. 
Then it have right inverse 
bounded operator $\psi:F\to E/\operatorname{Ker}(\varphi)$ 
with $\Vert\psi\Vert\leq c$. By proposition~\ref{PrCharMaxSQ} we have 
sequentially bounded operrator 
$\psi'
:\max(F)\to\max(E)/\operatorname{Ker}(\varphi)
:x\mapsto \psi(x)$ 
with $\Vert\psi'\Vert_{sb}=\Vert\psi\Vert\leq c$. 
From proposition~\ref{PrFactorSQOp} we have factorization 
$\max(\varphi)=\widehat{\max(\varphi)}\pi_{\operatorname{Ker}(\varphi),E}$, 
where $\widehat{\max(\varphi)}:E/\operatorname{Ker}(\varphi)\to F$ is 
sequentially bounded operator. Clearly 
$\widehat{\max(\varphi)}=\widehat{\varphi}$ and $\psi=\psi'$ as linear maps, 
hence $\widehat{\max(\varphi)}$ and $\psi'$ are sequentially bounded linear 
operators which are inverse to each other. Now, for any $n\in\mathbb{N}$  
and $y\in{\max(F)}^{\wideparen{n}}$ consider $x={(\psi')}^{\wideparen{n}}(y)$, 
then ${(\widehat{\max(\varphi)})}^{\wideparen{n}}(x)=y$ 
and $\Vert x\Vert_{\wideparen{n}}
=\Vert{(\psi')}^{\wideparen{n}}(y)\Vert_{\wideparen{n}}
\leq\Vert{(\psi')}^{\wideparen{n}}\Vert\Vert y\Vert_{\wideparen{n}}
\leq\Vert \psi'\Vert_{sb}\Vert y\Vert_{\wideparen{n}}
\leq c\Vert y\Vert_{\wideparen{n}}$. Since $n\in\mathbb{N}$ 
and $y\in {\max(F)}^{\wideparen{n}}$ are arbitrary, 
then $\widehat{\max(\varphi)}$ is sequentially $c$-topologically surjective. 
Since $\pi_{\operatorname{Ker}(\varphi),E}$ is sequentially $1$-topologically 
surjective, then by proposition~\ref{PrComposeSQTopInjSur} 
$\max(\varphi)=\widehat{\max(\varphi)}\pi_{\operatorname{Ker}(\varphi),E}$ 
is $c$-topologically surjective.

$(ii)$ By previous paragraph $\max(\varphi)$ is $1$-topologically surjective. On 
the other hand, by proposition~\ref{PrCharMaxSQ} we have 
$\Vert\max(\varphi)\Vert_{sb}=\Vert\varphi\Vert=1$. 
Therefore $\max(\varphi)$ is sequentially coisometric.
\end{proof}

\begin{proposition}\label{PrMaxCommuteWithCoprod} Let 
$ \{X_\lambda:\lambda\in\Lambda \}$ be a family of maximal operator sequence 
spaces, then $\bigoplus{}_1 \{X_\lambda:\lambda\in\Lambda \}$ is also maximal.
\end{proposition} 
\begin{proof}
Let $Y$ be arbitrary operator sequence space, then from 
propositions~\ref{PrSQCoProdUnivProp},~\ref{PrCharMinSQ} 
and~\ref{PrCharMaxSQ} we have isometric identifications
$$
\mathcal{SB}{\left(
    \bigoplus{}_1^0 \{X_\lambda:\lambda\in\Lambda \},Y
    \right)}^{\wideparen{1}}
=\bigoplus{}_\infty \{
    {\mathcal{SB}(X_\lambda,Y)}^{\wideparen{1}}:\lambda\in\Lambda
 \}
=\bigoplus{}_\infty \{
    \mathcal{B}(X_\lambda^{\wideparen{1}},Y^{\wideparen{1}}):\lambda\in\Lambda
 \}
$$
$$
=\bigoplus{}_\infty \{
    {\mathcal{SB}(X_\lambda,\min(Y^{\wideparen{1}}))}^{\wideparen{1}}
    :\lambda\in\Lambda
 \}
=\mathcal{SB}{\left(
    \bigoplus{}_1^0 \{X_\lambda:\lambda\in\Lambda \},\min(Y^{\wideparen{1}})
\right)}^{\wideparen{1}}
$$
$$
=\mathcal{B}\left(
    {\left(
        \bigoplus{}_1^0 \{X_\lambda:\lambda\in\Lambda \}
    \right)}^{\wideparen{1}},
    Y^{\wideparen{1}}
\right)
$$
Since $Y$ is arbitrary, from proposition~\ref{PrCharMaxSQ} we conclude 
that $\bigoplus{}_1^0 \{X_\lambda:\lambda\in\Lambda \}$ have maximal operator 
sequence space structure.
\end{proof}

\begin{proposition}\label{Prl1IsMax} Let $\Lambda$ be an arbitrary, set, 
then $l_1^0(\Lambda):=\bigoplus_1 \{\mathbb{C}:\lambda\in\Lambda \}$ 
have maximal operator sequence structure.
\end{proposition}
\begin{proof} By proposition~\ref{PrCHaveUniqueOSS} operator sequence space 
structure of $\mathbb{C}$ is unique and in particular maximal. Now result 
follows from proposition~\ref{PrMaxCommuteWithCoprod}.
\end{proof}

\begin{proposition}\label{PrMaxIsQuotientOfl1} Let $X$ be an operator sequence 
space, then $X$ is maximal if and only if there exist sequential coisometry 
from $l_1(\Lambda)$ onto $X$ for some set $\Lambda$.
\end{proposition}
\begin{proof} 
Assume $X$ have maximal structure. Consider natural coisometry 
$\pi:l_1(B_X)\to X$ (see A1~\cite{DefFloTensNorOpId}). 
By proposition~\ref{PrMaxPreserveQuotients} we know 
that $\max(\pi):\max({l_1(B_X)}^{\wideparen{1}})\to\max(X^{\wideparen{1}})$ 
is sequentially coisometric. By proposition~\ref{Prl1IsMax} 
we have $\max({l_1(B_X)}^{\wideparen{1}})=l_1(B_X)$ 
and by assumption $\max(X^{\wideparen{1}})=X$, 
so we get the desired sequential coisometry $\max(\pi):l_1(B_X)\to X$.

Conversely, assume we are given sequential coisometry $\pi:l_1(\Lambda)\to X$, 
then by proposition~\ref{PrFactorSQOp} we have that $X$ 
and $l_1(\Lambda)/\operatorname{Ker}(\pi)$ are sequentially 
isometrically isomorphic via $\widehat{\pi}$. 
Since $\pi^{\wideparen{1}}:{l_1(\Lambda)}^{\wideparen{1}}\to X^{\wideparen{1}}$ 
is coisometric too, by proposition~\ref{PrMaxPreserveQuotients} we have 
sequential coisometry 
$\max(\pi):\max({l_1(\Lambda)}^{\wideparen{1}})\to \max(X^{\wideparen{1}})$. 
From proposition~\ref{Prl1IsMax} it is known that 
$\max({l_1(\Lambda)}^{\wideparen{1}})=l_1(\Lambda)$, so we have one more 
sequential coisometry $\max(\pi):l_1(\Lambda)\to\max(X^{\wideparen{1}})$. 
Again by proposition~\ref{PrFactorSQOp} we see that $\max(X^{\wideparen{1}})$ 
and $l_1(\Lambda)/\operatorname{Ker}(\pi)$ are sequentially isometrically 
isomorphic via $\widehat{\max(\pi)}$. 
Therefore $X=l_1(\Lambda)/\operatorname{Ker}(\pi)=\max(X^{\wideparen{1}})$.
\end{proof}

\begin{proposition}[\cite{LamOpFolgen}, 2.1.11]\label{PrDualityAndMinMax}
Let $E$ be a normed space, then identity operator gives sequential 
isometric isomorphisms
$$
\max(E^*)={\min(E)}^\triangle,
\qquad
\min(E^*)={\max(E)}^\triangle
$$
\end{proposition}

Similar results holds in categories $SQNor$, $SQBan$ and $SQBan_1$.





































\subsection{Tensor products of operator sequence spaces}

It is natural to expect some kind of tensor product linearizing 
sequentially bounded bilinear operators.
\begin{definition}[\cite{LamOpFolgen}, 3.1.1]\label{DefSQMaxTenProd}
Let $X$ and $Y$ be operator sequence spaces, then their maximal tensor 
product is a operator sequence space $X\otimes_{\mathrm{Max}}Y$ with the 
family of norms 
${(
    \Vert\cdot\Vert_{{(X\otimes_{\mathrm{Max}}Y)}^{\wideparen{n}}}
)}_{n\in\mathbb{N}}$ 
given by equalities
$$
\Vert u\Vert_{{(X\otimes_{\mathrm{Max}}Y)}^{\wideparen{n}}}
=\inf\left \{
    \Vert[\alpha_1,\ldots,\alpha_k]\Vert_{M_{n,kl^2}}
    {\left(
        \sum\limits_{i=1}^k
            \Vert x_i\Vert_{X^{\wideparen{l}}}^2
            \Vert y_i\Vert_{Y^{\wideparen{l}}}^2
    \right)}^{1/2}
    :u=\sum\limits_{i=1}^k\alpha_i(x_i\otimes y_i)
\right \}
$$
where $u\in {(X\otimes_{\mathrm{Max}}Y)}^{\wideparen{n}}$, 
$\alpha_1,\ldots,\alpha_k\in M_{n,l^2}$ and $x\in X^{\wideparen{l}}$, 
$y\in Y^{\wideparen{l}}$. Using standard completion procedure for operator 
sequence spaces, we define completed version of this tensor product, which 
we will denote $X\otimes^{\mathrm{Max}} Y$.
\end{definition}

In [\cite{LamOpFolgen} 3.1.2] it is proved that, the norm defined above is 
the maximal cross norm making $X\otimes Y$ a operator sequence space. This 
tensor product is called \textit{maximal} and denoted by 
$X \otimes_{\mathrm{Max}} Y$. Maximal tensor product have universal property 
with respect to the class of  sequentially bounded bilinear operators.

\begin{proposition}[\cite{LamOpFolgen}, 
    3.1.3, 3.1.4]\label{PrSQUnivPropMaxTenProd}
Let $X$, $Y$ and $Z$ be operator sequence spaces, then there exist 
sequential isometric isomorphisms 
$$
\mathcal{SB}(X\otimes_{\mathrm{Max}}Y, Z)
=\mathcal{SB}(X\otimes^{\mathrm{Max}}Y, Z) 
=\mathcal{SB}(X\times Y, Z)
=\mathcal{SB}(X,\mathcal{SB}(Y,Z)) 
=\mathcal{SB}(Y,\mathcal{SB}(X,Z))
$$
natural in $X$, $Y$ and $Z$.
\end{proposition}

\begin{corollary}\label{CorSQUnivPropMaxTenProd}
Let $X$, $Y$ be operator sequence spaces, then there exist sequential 
isometric isomorphisms
$$
\mathcal{SB}(X^\triangle,Y)
=\mathcal{SB}(X,Y^\triangle)
={(X\otimes_{\operatorname{Max}} Y)}^\triangle
$$
natural in $X$ and $Y$. 
\end{corollary}































\section{Rigged categories}

\subsection{Projectivity and injectivity. Freedom and cofreedom}

Now we will quote some definitions and results from~\cite{HelMetrFrQmod}. 
Let $\mathcal{K}$ be an arbitrary category.
\begin{definition}[\cite{HelMetrFrQmod}, 2.1]\label{DefRigCat}
A pair ($\mathcal{K}, \square:\mathcal{K}\to\mathcal{L}$), where $\square$ is 
a faithful covariant functor, is called a rigged category. A dual rigged 
category of $(\mathcal{K}, \square)$ 
is a rigged category 
$(\mathcal{K}^{o},\square^{o}:\mathcal{K}^{o}\to\mathcal{L}^{o})$. 
\end{definition}
\begin{definition}[\cite{HelMetrFrQmod}, 2.1]\label{DefAdmMorph}
A morphism $\tau$ in $\mathcal{K}$ is called $\square$-admissible epimorphism 
(monomorphism) if $\square (\tau)$ is a retraction 
(coretraction) in $\mathcal{L}$.
\end{definition}
\begin{definition}[\cite{HelMetrFrQmod}, 2.2]\label{DefProjInj}
An object $P\in \mathcal{K}$ ($I \in \mathcal{K}$) is called 
$\square$-projective ($\square$-injective) with respect to a rigged category 
$(\mathcal{K}, \square)$, if for every $X,Y\in\mathcal{K}$ and every 
$\square$-admissible epimorphism $\tau : Y \to X$ (monomorphism 
$\tau : X \to Y$)  the map $\operatorname{Hom}_{\mathcal{K}}(P,\tau)$ 
($\operatorname{Hom}_{\mathcal{K}}(\tau, I)$) is surjective.
\end{definition}
\begin{definition}[\cite{HelMetrFrQmod}, 2.10]\label{DefFrAndCoFr}
An object $F \in \mathcal{K}$ is called $\square$-free ($\square$-cofree) 
with base $M \in \mathcal{L}$, if there exist a morphism 
$j : M \to \square(F)$ ($j : \square(F) \to M$), 
such that for each $X \in \mathcal{K}$ and each morphism 
$\varphi : M \to \square(X)$ ($\varphi : \square(X) \to M$) there exist 
a unique $\psi : F \to X$ ($\psi : X \to F$), 
making the following diagram
$$
\xymatrix{{\square (F)} \ar@{-->}[dr]^{\square (\psi)} & \\  % chktex 3
{M} \ar[u]^{j} \ar[r]^{\varphi} &{\square (X)}} \qquad\qquad\quad  % chktex 3
\xymatrix{{\square (F)} \ar[d]_{j} & \\  % chktex 3
{M} &{\square (X)} \ar[l]_\varphi \ar@{-->}[ul]_{\square(\psi)}}  % chktex 3
$$
commutative. The morphism $j$ is called universal arrow.
\end{definition}
\begin{definition}\label{DefFrAndCoFrLove}
A rigged category $(\mathcal{K},\square)$ is called freedom-loving 
(cofreedom-loving), if every object in $\mathcal{L}$ is a base of some 
$\square$-free ($\square$-cofree) object in $\mathcal{K}$.
\end{definition}

The following results will be extremely useful in near future.

\begin{proposition}\label{PrUniqFr}
Let $M\in\mathcal{L}$ be a base of $\square$-free ($\square$-cofree) 
objects $F_1$, $F_2$ in the rigged category $(\mathcal{K},\square)$, 
then $F_1$ and $F_2$ are isomorphic.  
\end{proposition}
\begin{proof}
Consider category $\mathcal{K}_M$, whose objects are pairs of the form 
$(X,\varphi:M\to\square X)$, and morphisms from $(X_1,\varphi_1)$ to 
$(X_2,\varphi_2)$ are morphisms $\psi$ in 
$\mathcal{K}$, such that $\varphi_2=\square(\psi)\varphi_1$. Composition of 
morphisms in $\mathcal{K}_M$ is the same as in $\mathcal{K}$. Clearly, every 
object $F$ is $\square$-free in $\mathcal{K}$ with base $M$ and universal 
arrow $j$ if and only if $(F,j)$ is the terminal object in $\mathcal{K}_M$. 
Recall that every terminal object in any category is unique up to isomorphism. 
It is remains to note that every isomorphism in $\mathcal{K}_M$ is an 
isomorphism in $\mathcal{K}$.
\end{proof}

\begin{proposition}\label{PrCompOfFrIsFr} 
Let $\square_{12}:\mathcal{K}_1\to\mathcal{K}_2$, 
$\square_{23}:\mathcal{K}_2\to\mathcal{K}_3$ be faithful functors. 
Denote $\square_{13}=\square_{23}\square_{12}$. Let 
$F_1$ be $\square_{12}$-free ($\square_{12}$-cofree) object with base $F_2$ 
and universal arrow $j_{12}$ in the rigged 
category $(\mathcal{K}_1,\square_{12})$. Let $F_2$ be $\square_{23}$-free 
($\square_{23}$-cofree) object with base $F_3$ and universal arrow $j_{23}$ 
in the rigged category $(\mathcal{K}_2,\square_{23})$. Then $F_1$ is 
a $\square_{13}$-free ($\square_{13}$-cofree) 
object with base $F_3$ and universal arrow $\square_{23}(j_{23})j_{12}$ in 
the rigged category $(\mathcal{K}_1,\square_{13})$. 
$$
\xymatrix{& { \mathcal{K}_2} \ar[dr]^{\square_{23}} & \\  % chktex 3
{\mathcal{K}_1} \ar[ur]^{\square_{12}} \ar[rr]_{\square_{13}} & &  % chktex 3
{\mathcal{K}_3}}
$$
As the consequence, if rigged categories $(\mathcal{K}_1,\square_{12})$, 
$(\mathcal{K}_1,\square_{12})$ are freedom-loving (cofreedom-loving), then so 
does $(\mathcal{K}_1,\square_{13})$. For cofree objects the proof is the same.
\end{proposition}
\begin{proof}
Consider arbitrary object $X\in\mathcal{K}_1$ and a morphism 
$\varphi:F_3\to \square_{13}(X)$. Since $F_2$ is a $\square_{23}$-free object, 
then there exist unique $\psi:F_2\to \square_{12}(X)$, 
such that $\varphi=\square_{23}(\psi)j_{23}$. Since $F_1$ is 
$\square_{12}$-free, then there exist the unique $\chi:F_1\to X$, 
such that $\psi=\square_{12}(\chi)j_{12}$.
$$
\xymatrix
{{\square_{23}(\square_{12}(F_1))}
\ar[rr]^{\square_{23}(\square_{12}(\chi))}&  % chktex 3
&{\square_{23}(\square_{12}(X))}\\
{\square_{23}(F_2)} \ar[u]^{\square_{23}(j_{12})}  % chktex 3
\ar[urr]^{\square_{23}(\psi)}  % chktex 3
& &\\
{F_3} \ar[u]^{j_{23}} \ar[uurr]^{\varphi} & \\  % chktex 3
} 
$$
Therefore 
$\varphi
=\square_{23}(\psi)j_{23}
=\square_{23}(\square_{12}(\chi))\square_{23}(j_{23})j_{12}
=\square_{13}(\chi) j_{13}$, where $j_{13}=\square_{23}(j_{23})j_{12}$ 
is a universal arrow. Since $X$ and $\varphi$ are arbitrary, then 
$F_1$ is a $\square_{13}$-free object with base $F_3$. For cofree objects 
the proof is the same.
\end{proof}

\begin{proposition}[\cite{HelMetrFrQmod}, 2.3]\label{PrRetractsProjInj} 
Let $(\mathcal{K},\square)$ be a rigged category, and $P\in\mathcal{K}$ 
($I\in\mathcal{K}$) be $\square$-projective ($\square$-injective) object, 
then
\begin{enumerate}[label = (\roman*)]
    \item  if $\sigma:P\to Q$ ($\sigma:I\to J$) is a retraction, 
    then $Q$ is $\square$-projective ($J$ $\square$-injective)

    \item if $\sigma:X\to P$ ($\sigma:X\to I$) is $\square$-admissible 
    epimorphism (monomorphism), then $\sigma$ is a retraction (coretraction)
\end{enumerate}
\end{proposition}

\begin{proposition}[\cite{HelMetrFrQmod}, 2.11]\label{PrFrCoFrProjInjObjProp} 
Let $(\mathcal{K},\square)$ be a rigged category, then

\begin{enumerate}[label = (\roman*)]
    \item if $F\in\mathcal{K}$ is a $\square$-free ($\square$-cofree), 
    then it is $\square$-projective ($\square$-injective)

    \item if $X$ such object in $\mathcal{K}$, that $\square(X)$ is a base of 
    $\square$-free ($\square$-cofree) object $F$, then there exist 
    $\square$-admissible epimorphism (monomorphism) from $F$ to $X$ 
    (from $X$ to $F$).

    \item if $\mathcal{K}$ is freedom-loving (cofreedom-loving), 
    then $P\in\mathcal{K}$ ($I\in\mathcal{K}$) is $\square$-projective 
    ($\square$-injective) if and only if it is a retract of $\square$-free 
    ($\square$-cofree) object.
\end{enumerate}
\end{proposition}

\begin{proposition}[\cite{HelMetrFrQmod}, 2.13]\label{PrCoprodFrIsFr} 
Let $(\mathcal{K},\square)$ be a rigged category, and $\Lambda$ be any set. 
Assume for each $\lambda \in \Lambda$ an object $F_{\lambda} \in \mathcal{K}$ 
is $\square$-free ($\square$-cofree) with base $M_{\lambda} \in \mathcal{L}$. 
Assume that $ \{ F_\lambda:\lambda \in \Lambda \}$ 
admits coproduct (product) $F$, and the family 
$ \{ M_\lambda:\lambda \in \Lambda \}$ admits coproduct 
(product) $M$. Then the object $F$ is $\square$-free ($\square$-cofree) 
with base $M$.
\end{proposition}

\begin{proposition}[\cite{HelMetrFrQmod}, 4.5]\label{PrFunctorMapFrToFr} 
Let $(\mathcal{K}_1, \square_1: \mathcal{K}_1 \to \mathcal{L}_1)$ and 
$(\mathcal{K}_2, \square_2 : \mathcal{K}_2 \to \mathcal{L}_2)$ are 
rigged categories, and we are given covariant functors 
$\Phi : \mathcal{K}_1 \to \mathcal{K}_2$ and 
$\Psi : \mathcal{L}_1 \to \mathcal{L}_2$ such that the diagram
$$
\xymatrix{{\mathcal{K}_1}\ar[d]_{\Phi} \ar[rr]^{\square_1} & &  % chktex 3
{\mathcal{L}_1}\ar[d]^{\Psi}\\  % chktex 3
{\mathcal{K}_2}\ar[rr]^{\square_2} & & {\mathcal{L}_2}}  % chktex 3
$$
is commutative. Assume that $\Phi$ and $\Psi$ has left (right) adjoint 
functors $\Phi^*$ and $\Psi^*$ (${}^*\Phi$ and ${}^*\Psi$) respectively, 
and $F\in\mathcal{K}_2$ is a $\square_2$-free ($\square_2$-cofree) object 
with base $M\in\mathcal{L}_2$. Then 
$\Phi^*(F)\in\mathcal{K}_1$ (${}^*\Phi(F)\in\mathcal{K}_1$) is a 
$square_1$-free ($\square_1$-cofree) object with base 
$\Psi^*(M)\in\mathcal{L}_1$ (${}^*\Psi(M)\in\mathcal{L}_1$). 
\end{proposition}




























\subsection{Normed semilinear spaces}

In what follows we will need the following construction.

\begin{definition}\label{DefSemiLinSp} A semilinear space $V$ over field $K$ is 
an ordered triple $(V, K, \cdot)$, where $V$ is a nonempty set, whose elements 
are called vectors, $K$ is a field, whose elements are called scalars, 
$\cdot : K \times V \to V$ is a map satisfying the following axioms
\begin{enumerate}[label = (\roman*)]
    \item 
    \item for all $x\in V$, $\alpha,\beta\in K$ holds 
    $\alpha \cdot (\beta \cdot x) = (\alpha \beta) \cdot x $

    \item for all $x\in V$ holds $1_K \cdot x = x$

    \item there exist a vector $0 \in V$, such that $0_K \cdot x = 0$ 
    for all $x\in V$.
\end{enumerate}

The vector $0\in V$ is called a zero vector.
\end{definition}

Clearly, zero vector is unique and $\alpha \cdot 0 = 0$ for all $\alpha\in K$.

\begin{example}\label{ExSemiLinModelSp}
Consider wedge sum $\bigvee \{K: \lambda \in\Lambda \}$ of copies of the field 
$K$, which intersects by zero vector, for some set $\Lambda$. Multiplication in 
wedge sum is inherited from the filed. Obviously, this is a semilinear space 
over field $K$, which we will denote $K^{\Lambda}$. By $K^{\varnothing}$ we 
will understand semilinear space, consisting of single zero vector. 
\end{example}

\begin{definition}\label{DefSemiLinOp} A map $\varphi : V \to W$ between 
semilinear spaces $V$ and $W$ is called semilinear operator, if 
$\varphi(\alpha \cdot x) = \alpha \cdot \varphi(x)$ for 
all $\alpha \in K$ and $x \in V$.
\end{definition}

Consider category $Lin_{0}^{K}$, whose objects are semilinear spaces over 
field $K$, and morphisms --- semilinear operator. We can easily get complete 
characterization of objects of this category.

\begin{proposition}\label{PrSemiLinSpDesc}
Every semilinear space is isomorphic in $Lin_{0}^{K}$ to $K^{\Lambda}$ for 
some set $\Lambda$.
\end{proposition}
\begin{proof} We say that two vectors $x,y\in V$ are equivalent 
if $x=\alpha y$ for some $\alpha\in K\setminus \{0 \}$. This relation $\sim$ is 
an equivalence relation. Let $ \{x_\lambda:\lambda\in \Lambda \}$  be a set of 
representatives of each equivalence class except equivalence class of zero 
vector. Then the semilinear operator 
$\varphi: K^\Lambda\to V: z_\lambda\mapsto z_\lambda x_\lambda$ is an 
isomorphism in  $Lin_0^K$
\end{proof}

\begin{definition}\label{DefSemiLinNorSp} A semilinear normed space over normed 
field $K$ is a pair $(E, \Vert \cdot \Vert)$, where $E$ is a semilinear space 
over field $K$ and $ \Vert \cdot \Vert : E \to \mathbb{R}_+$ is a map, which 
we will call a norm, satisfying the following relations:
\begin{enumerate}[label = (\roman*)]
    \item if $x\in E$ and $\Vert x \Vert = 0$, then $x = 0$;

    \item for all $x\in E$ and $\alpha\in K$ holds 
    $\Vert \alpha \cdot x \Vert = | \alpha| \Vert x \Vert$.
\end{enumerate}
\end{definition}

\begin{example}\label{ExSemiLinNorModelSp}
For a given normed field $K$ we define a norm on $K^{\Lambda}$, by equality 
$\Vert z_\lambda\Vert:=|z_\lambda|_K$ for each $z_\lambda\in K^\Lambda$. 
\end{example}

\begin{definition}\label{DefSemiLinBndOp} A semilinear operator 
$\varphi : E \to F$ between semilinear normed spaces $E$ and $F$ is called 
bounded, if $\Vert \varphi (x)\Vert \leq C \Vert x \Vert$ for  some 
constant $C\in\mathbb{R}_+$. Infima of all such constants we will call a 
norm of $\varphi$ and will denote it by $\Vert\varphi\Vert$.
\end{definition}

Now consider category $Nor_0^K$, whose objects are semilinear normed spaces, 
and morphisms --- bounded semilinear operators. It is not hard to classify 
objects of this category.

\begin{proposition}\label{PrSemiLinNorSpDesc} Every semilinear normed space in 
$Nor_0^K$ is isomorphic to $K^{\Lambda}$ for some set $\Lambda$.
\end{proposition}

\begin{proof} Using proposition~\ref{PrSemiLinSpDesc} consider equivalence 
relation $\sim$ and a set $ \{x_\lambda:\lambda\in\Lambda \}$ of representatives 
of equivalence classes, except equivalence class of zero vector. 
Fix some $\alpha\in K$ such that $0<|\alpha|<1$. For each $\lambda\in\Lambda$ 
there exist $m_\lambda\in\mathbb{Z}$ such that 
$|\alpha|^{-m_\lambda}\leq\Vert x_\lambda\Vert< |\alpha|^{-m_\lambda+1}$. 
Define $y_\lambda=\alpha^{m_\lambda} x_\lambda$, 
then $1\leq \Vert y_\lambda\Vert<|\alpha|$. Now it is easy to see that the 
semilinear operator 
$\varphi: K^\Lambda\to E: z_\lambda\mapsto z_\lambda y_\lambda$ is 
an isomorphism in $Nor_0^K$
\end{proof}

In what follows by $Nor_0$ we will denote the category $Nor_0^\mathbb{C}$.






































\subsection{Examples of rigged categories}

Consider several examples. For simplicity we will deal only with normed spaces. 
One can easily extend these constructions to the case of normed modules.

\begin{example}[Metric freedom,~\cite{HelMetrFrQmod}]\label{ExMetrFr}
Let $\mathcal{K} = Nor_1$, $\mathcal{L} = Set$. The functor $\square$ sends a 
normed space to its closed unit ball, and morphism is mapped to its 
birestriction to unit balls in domain and range space. In this case 
$\square$-admissible epimorphisms are strict coisometries., $\square$-free 
object with one point base is $\mathbb{C}$. Hence from 
proposition~\ref{PrCoprodFrIsFr} immediately follows, 
that $\square$-free object with base $\Lambda$ is $l_1^0(\Lambda)$.
\end{example}

\begin{example}[Topological freedom]\label{ExTopFr}
Let $\mathcal{K} = Nor$, $\mathcal{L} = Nor_0$. The functor $\square$ sends 
a normed space to its underlying semilinear normed space with the same norm, 
and a morphism remains the same. In this case $\square$-admissible epimorphisms 
are are topologically surjective operators, $\square$-free object with 
base $\mathbb{C}^\Lambda$ is $l_1^0(\Lambda)$.
\end{example}

\begin{example}[Metric cofreedom,~\cite{HelMetrFrQmod}]\label{ExMetrCoFr}
Let $\mathcal{K} = Nor_1$, $\mathcal{L} = Set^0$. The functor $\square$ send 
a normed space $X$ into the unit ball of $X^*$, a morphism is mapped to 
birestriction of dual morphism to unit balls of domain and range spaces. 
In this case $\square$-admissible epimorphisms are isometries, $\square$-cofree 
object with base $\Lambda$ easily constructed from example~\ref{ExMetrFr} 
and proposition~\ref{PrCoprodFrIsFr} --- this is the space $l_\infty(\Lambda)$.
\end{example}

\begin{example}[Topological cofreedom,~\cite{ShtTopFr}]\label{ExTopCoFr}
Let $\mathcal{K} = Nor$, $\mathcal{L} = Nor_0^o$. The functor $\square$ send a 
normed space to underlying semilinear normed space of its dual, a morphism is 
mapped to the its adjoint. In this case $\square$-admissible epimorphisms are 
topologically injective operators, $\square$-cofree objects with 
base $\mathbb{C}^\Lambda$ easily constructed from example~\ref{ExTopFr} and 
proposition~\ref{PrCoprodFrIsFr} --- this is the space $l_\infty(\Lambda)$. 
\end{example}

All these examples have their obvious Banach analogues, given by completion of 
free and cofree objects mentioned above. Moreover these examples have their 
quantum versions: the role of free object with one point base instead 
of $\mathbb{C}$ plays the operator space 
$\mathcal{N}_{\infty}
:=\bigoplus{}_1^0 \{\mathcal{N}(\mathbb{C}^n):n\in\mathbb{N} \}$
([\cite{HelMetrFrQmod}, 5.9], 
see also~\cite{ShtTopFr}). Our immediate goal is to show the same role for 
operator sequence spaces is played by 
$t_2^{\infty} :=  \bigoplus_1^0 \{t_2^n:n\in\mathbb{N} \}$.  






































\section{Free operator sequence spaces}

\subsection{Metric freedom}

We begin with metric version of freedom for operator sequence spaces. 
Consider functor
$$
\begin{aligned}
\square_{sqMet}
:SQNor_1 \to Set
:X&\mapsto\prod\left \{B_{X^{\wideparen{n}}}:n \in \mathbb{N}\right \} \\
\varphi&\mapsto
\prod\left \{
    \varphi^{\wideparen{n}}|_{B_{X^{\wideparen{n}}}}^{B_{Y^{\wideparen{n}}}}
    :n\in\mathbb{N}
\right \}
\end{aligned}
$$
sending a operator sequence space to the cartesian product of unit balls of 
its amplifications. 

\begin{proposition}\label{PrDecsMetrAdmEpiMorph}
$\square_{sqMet}$-admissible epimorphisms are exactly sequentially strictly 
coisometric operators.
\end{proposition}
\begin{proof}
A morphism $\varphi$ is $\square_{sqMet}$-admissible epimorphism 
if $\square_{sqMet}(\varphi)$ is invertible from the right as morphism in $Set$. 
This is equivalent to surjectivity of $\square_{sqMet}(\varphi)$, which is 
equivalent to surjectivity of 
$\varphi^{\wideparen{n}}|_{B_{X^{\wideparen{n}}}}^{B_{Y^{\wideparen{n}}}}$ 
for all $n\in\mathbb{N}$. The latter means that $\varphi^{\wideparen{n}}$ 
strictly coisometric for each $n\in\mathbb{N}$. 
So $\varphi^{\wideparen{n}}$ sequentially strictly coisometric.
\end{proof}


By $I_n$ we denote the element of 
${(t_2^n)}^{\wideparen{n}} = \mathcal{B}(l_2^n, l_2^n)$, corresponding to 
the identity operator.

\begin{proposition}\label{PrMetrFrLem} Let $X$ be a operator sequence space 
and $x \in B_{X^{\wideparen{n}}}$. Then there exist unique sequentially 
contractive operator 
$\psi_n \in \mathcal{SB}(t_2^n, X)$, 
such that $\psi_n^{\wideparen{n}}(I_n) = x$.
\end{proposition}
\begin{proof}
Since, $I_n = {(e_i)}_{i\in\mathbb{N}_n}$, where $e_i$ is the $i$-th orth of 
underlying space $t_2^n$. Obviously, there exist unique linear operator 
$\psi_n$, satisfying $\psi_n(e_i) = x_i$, $i\in\mathbb{N}_n$. 
It is remains to check that $\psi_n$ is sequentially contractive. 
Let $k \in \mathbb{N}$ and $y \in B_{{(t_2^n)}^{\wideparen{k}}} $, 
then $y_i = \sum_{j = 1}^n \alpha_{ij}e_j$, $i\in\mathbb{N}_k$ 
for some matrix $\alpha\in M_{k,n}$. Then
$$
\Vert\psi_n^{\wideparen{k}}(y)\Vert_{\wideparen{k}}
=\left\Vert
    {\left(\psi_n(y_i)\right)}_{i\in\mathbb{N}_k}
\right\Vert_{\wideparen{k}}
=\left\Vert
{\left(
    \sum\limits_{j=1}^n\alpha_{ij}\psi_n(e_j)
\right)}_{i\in\mathbb{N}_k}\right\Vert_{\wideparen{k}}
=\left\Vert
    {\left(\sum\limits_{j=1}^n\alpha_{ij}x_j\right)}_{i\in\mathbb{N}_k}
\right\Vert_{\wideparen{k}}
$$
$$
=\Vert\alpha x\Vert_{\wideparen{k}} \leq\Vert\alpha\Vert\Vert
x\Vert_{\wideparen{n}} =\Vert y\Vert_{{(t_2^n)}^{\wideparen{k}}}\Vert
x\Vert_{\wideparen{n}}\leq 1
$$
Therefore $\psi_n$ is sequentially contractive.
\end{proof}

\begin{proposition}\label{PrOnePtMetrFr} Metrically free operator sequence 
space with one point base is a space 
$t_2^{\infty}:=\bigoplus_1^0  \{t_2^n: n \in \mathbb{N} \}$.
\end{proposition}
\begin{proof}
Define universal arrow as such 
$j: \{\lambda \}\to t_2^\infty:\lambda\mapsto(I_1,I_2,\ldots,I_n,\ldots)$. 
Let $X$ be arbitrary operator sequence space and 
$\varphi: \{\lambda \}\to \prod_{n \in \mathbb{N}} B_{X^{\wideparen{n}}}$ 
be some map. Denote $x=\varphi(\lambda)$. From proposition~\ref{PrMetrFrLem} and 
properties of corpducts it  follows that, there exist unique sequentially 
contractive operator 
$\psi=\bigoplus_1^0 \{\psi_n:n\in\mathbb{N} \}
\in\mathcal{SB}\left(\bigoplus_1^0 \{ t_2^n:n\in\mathbb{N} \}, X\right)$, such 
that $\psi^{\wideparen{n}}(i_n(I_n)) = x$, for all $n \in \mathbb{N}$. 
Here $i_n:t_2^n\to t_2^\infty$ stands for standard embedding.
$$
\xymatrix{{\square_{sqMet} (t_2^\infty)} 
\ar@{-->}[dr]^{\square_{sqMet} (\psi)}  % chktex 3
& \\
{ \{\lambda \}} 
\ar[u]^{j} \ar[r]^{\varphi} &{\square_{sqMet} (X)}}  % chktex 3
$$
In this case $\varphi=\square_{sqMet}(\psi) j$. Since $X$ and $\varphi$ 
are arbitrary, then $t_2^\infty$ is metrically free with one point base. 
\end{proof}

Thus we are ready to state the final result.

\begin{theorem}\label{ThMetrFrDesc} Metrically free operator sequence space 
with base $\Lambda$ is up to sequential isometric isomorphism 
a $\bigoplus{}_1^0$-sum of copies of the space $t_2^{\infty}$, 
indexed by elements of the set $\Lambda$. 
\end{theorem}
\begin{proof}
Result follows from propositions~\ref{PrCoprodFrIsFr} and~\ref{PrOnePtMetrFr}
\end{proof}

\begin{corollary}\label{CorSQSpaceIsImgMetrAdmEpiMorph}
Every operator sequence space is an image of sequentially strictly 
coisometric operator from $\bigoplus_1^0 \{t_2^\infty:\lambda\in\Lambda \}$ 
for some set $\Lambda$.
\end{corollary}
\begin{proof}
From theorem~\ref{ThMetrFrDesc} we see that $(SQNor_1,\square_{sqMet})$ is 
freedom-loving. Now the desired result follows from 
propositions~\ref{PrFrCoFrProjInjObjProp} and~\ref{PrDecsMetrAdmEpiMorph}.
\end{proof}

Similar propositions are valid in Banach case 
(just replace $\bigoplus{}_1^0$-sums with $\bigoplus{}_1$-sums).








































\subsection{Topological freedom}

Let's proceed to consideration of sequential version of topological freedom. 
Consider functor
$$
\begin{aligned}
\square_{sqTop} : SQNor \to Nor_0: X &\mapsto \bigoplus{}_\infty
 \{X^{\wideparen{n}} : n \in \mathbb{N} \} \\
\varphi&\mapsto\bigoplus{}_\infty \{\varphi^{\wideparen{n}}:n\in\mathbb{N} \},\\
\end{aligned}
$$
sending operator sequence space to underlying semilinear normed 
space of $\bigoplus_\infty$-sum of its amplifications.

\begin{proposition}\label{PrCTopSurIsRetrInNor0} Let $\varphi:X\to Y$ be 
bounded linear operator between normed spaces $X$ and $Y$, then it is 
$c$-topologically surjective if and only if there exist bounded semilinear 
operator $\rho:Y\to X$ such that $\Vert\rho\Vert\leq c$ and $\varphi\rho=1_Y$.
\end{proposition}
\begin{proof} Assume $\varphi$ is $c$-topologically surjective. 
Consider relation $\sim$ on $S_Y$ defined as follows: $e_1\sim e_2$  if and 
only if  there exist $\alpha\in\mathbb{T}$ such that $e_1=\alpha e_2$. 
Clearly, $\sim$ is an eqivalence relation, so we can consider a set of 
non-zero representatives of equivalence classes, 
say $ \{r_\lambda:\lambda\in\Lambda \}$. By construction, for 
each $e\in S_Y$ we have unique $\alpha(e)\in\mathbb{T}$ 
and $\lambda(e)\in\Lambda$ such that $e=\alpha(e)r_{\lambda(e)}$. Clearly, 
for any $z\in\mathbb{T}$ and $e\in S_Y$ we have $\alpha(ze)=z\alpha(e)$ 
and $\lambda(ze)=\lambda(e)$. Since $\varphi$ is $c$-topologically surjective, 
then, in particular, for each $\lambda\in\Lambda$ we have $x(\lambda)\in X$ 
such that $\Vert x(\lambda)\Vert\leq c\Vert r_\lambda\Vert$ 
and $\varphi(x(\lambda))=r_\lambda$. Consider, 
map $\tilde{\rho}:S_Y\to X:e\mapsto \alpha(e)x(\lambda(e))$. It is easy to see 
that for all $z\in\mathbb{T}$ and $e\in S_Y$ 
holds $\tilde{\rho}(z e)=z\tilde{\rho}(e)$, $\Vert\tilde{\rho}(e)\Vert\leq c$ 
and $\varphi(\tilde{\rho}(e))=e$. Now consider 
map $\rho:Y\to X: y\mapsto \Vert y\Vert\tilde{\rho}(\Vert y\Vert^{-1} y)$ 
and $\rho(0)=0$. Using properties of $\tilde{\rho}$ it is trivial to check 
that $\rho$ is semilinear operator such that $\Vert\rho\Vert\leq c$ 
and $\varphi\rho=1_Y$.

Conversely, assume there exist bounded semilinear operator $\rho:Y\to X$ such 
that $\Vert\rho\Vert\leq c$ and $\varphi\rho=1_Y$. Take any $y\in Y$ and 
consider $x=\rho(y)$, then $\Vert x\Vert\leq C\Vert y\Vert$ 
and $\varphi(x)=y$. Hence $\varphi$ is $c$-topologically surjective.
\end{proof}

\begin{proposition}\label{PrDecsTopAdmEpiMorph} $\square_{sqTop}$-admissible 
epimorphisms are exactly sequentially topologically surjective operators.
\end{proposition}
\begin{proof}
For a given operator sequence space $Z$ 
by $i_n^Z:Z^{\wideparen{n}}\to\square_{sqTop}(Z)$ we denote natural embedding, 
and by $p_n^Z:\square_{sqTop}(Z)\to Z^{\wideparen{n}}$ we denote natural 
projection. Assume that $\varphi:X\to Y$ is $c$-sequentially topologically 
surjective. Fix $n\in\mathbb{N}$, then by 
proposition~\ref{PrCTopSurIsRetrInNor0} there exist bounded semilinear 
operator $\rho^n$ such 
that $\varphi^{\wideparen{n}}\rho^n=1_{Y^{\wideparen{n}}}$ 
and $\Vert\rho^n\Vert\leq c$. Consider 
map $ \rho=\bigoplus{}_\infty \{\rho^n:n\in\mathbb{N} \}$. For 
each $y\in \square_{sqTop}(Y)$ we have 
$$
\Vert \rho(y)\Vert=\sup \{\Vert\rho^n(p_n^Y(y))\Vert_{\wideparen{n}}:
n\in\mathbb{N} \}\leq c\sup \{\Vert p_n^Y(y)\Vert_{\wideparen{n}}:
n\in\mathbb{N} \}=c\Vert y\Vert
$$
so $\rho$ is semilinear bounded operator. 
Moreover, $\square_{sqTop}(\varphi)\rho=1_{\square_{sqTop}(Y)}$, 
hence $\varphi$ is $\square_{sqTop}$-admissible epimorphism. Conversely, if 
$\varphi$ is $\square_{sqTop}$-admissible epimorphism, then there exist 
bounded right inverse semilinear operator $\rho$ to $\square_{sqTop}(\varphi)$. 
Then for every $y\in Y^{\wideparen{n}}$ 
holds $\square_{sqTop}(\varphi)\\\rho(i_n^Y(y))=i_n^Y(y)$. 
In particular $\varphi^{\wideparen{n}}(p_n^X(\rho(i_n^Y(y))))=y$. 
Denote $x=p_n^X(\rho(i_n^Y(y)))$ and 
$c=\Vert\rho\Vert$, then $\varphi^{\wideparen{n}}(x)=y$ 
and $\Vert x\Vert_{\wideparen{n}}
\leq\Vert\rho(i_n^Y(y))\Vert
\leq c\Vert i_n^Y(y)\Vert=c\Vert y\Vert_{\wideparen{n}}$. Therefore, 
$\varphi$ is sequentially topologically surjective.
\end{proof}

We are ready to state the main result of this section.

\begin{proposition}\label{PrMetrFrIsTopFr} Let $F$ be metrically free 
operator sequence space with base $\Lambda$. Then $F$ is topologically 
free operator sequence space with base $\mathbb{C}^{\Lambda}$.
\end{proposition}
\begin{proof} Let $j':\Lambda\to \square_{sqMet}(F)$ be universal arrow in the 
diagram of metric freedom of $F$. Define semilinear bounded 
operator 
$j:\mathbb{C}^{\Lambda}\to\square_{sqTop}(F)
:z_\lambda\mapsto z_\lambda j(\lambda)$. Consider arbitrary bounded semilinear 
operator $\varphi : \mathbb{C}^{\Lambda} \to \square_{sqTop}(X)$, where 
$X$ is arbitrary operator sequence space. Then 
for $\varphi':=\Vert \varphi \Vert_{sb}^{-1}\varphi $ there exist 
a unique $\psi^{'}$, such that $\varphi'=\square_{sqMet}(\psi')j$. Now, it is 
easy to see that for $\psi:=\Vert \varphi \Vert_{sb} \psi^{'}$ the diagram
$$
\xymatrix{{\square_{sqTop}(F)}
\ar@{-->}[dr]^{\square_{sqTop}(\psi)} & \\  % chktex 3
{\mathbb{C}^{\Lambda}}
\ar[u]_{j}\ar[r]_{\varphi}  &{\square_{sqTop}(X)} }  % chktex 3
$$
is commutative. Assume there two morphisms $\psi_1$ and $\psi_2$ making the 
diagram above commutative. 
Denote $C=\max(
    \Vert\varphi\Vert_{sb},
    \Vert\psi_1 \Vert_{sb}, 
    \Vert\psi_2\Vert_{sb}
)$, 
then morphisms $C^{-1}\psi_1$ and $C^{-1}\psi_2$ make the following diagram
$$
\xymatrix{{\square_{sqMet}(F)}
\ar@{-->}[dr]^{?} & \\  % chktex 3
{\mathbb{C}^\Lambda}
\ar[u]_{j'}\ar[r]_{ C^{-1}\varphi'}  &{\square_{sqMet}(X)} }  %chktex 3
$$
commutative. This contradicts uniqueness of morphism $\psi'$, 
so $\psi$ is unique.
\end{proof}

As the consequence we get complete description of topologically 
free operator sequence spaces

\begin{theorem}\label{ThTopFrDesc} 
A operator sequence space is topologically free if and only if it is 
sequentially topologically isomorphic to $\bigoplus{}_1^0$-sum of copies of 
the space $t_2^\infty$, indexed by elements of some set $\Lambda$.
\end{theorem}

\begin{corollary}\label{CorSQSpaceIsImgTopAdmEpiMorph}
Every operator sequence space is an image of sequentially topologically 
surjective operator from $\bigoplus_1^0 \{t_2^\infty:\lambda\in\Lambda \}$ 
for some set $\Lambda$.
\end{corollary}
\begin{proof}
From theorem~\ref{ThMetrFrDesc} it follows that the rigged 
category $(SQNor,\square_{sqTop})$ is freedom-loving. Now the desired result 
follows from propositions~\ref{PrFrCoFrProjInjObjProp} 
and~\ref{PrDecsTopAdmEpiMorph}
\end{proof}

Similar propositions are valid in Banach case 
(just replace $\bigoplus{}_1^0$-sums with $\bigoplus{}_1$-sums).







































\subsection{Pseudotopological freedom and projectivity}

One may ask, whether existence of uniform constant $C$ in the definition of 
sequential topological surjectivity is necessary? Indeed more natural 
definition would require just topological surjectivity of all amplifications 
of sequentially bounded operator. Later we will see that this class of 
admissible epimorphisms doesn't give rich homological theory. 

The type of projectivity given by this kind of sequentially bounded admissible 
epimorphisms we will call pseudotopological. Consider functor
$$
\begin{aligned}
\square_{sqpTop} : SQNor \to Nor_0: X &\mapsto X^{\wideparen{1}}\\
\varphi&\mapsto\varphi \\
\end{aligned}
$$
sending operator sequence space to the underlying semilinear normed space of 
the first amplification.

\begin{definition}\label{DefPsSQTopSurjOp} A sequentially bounded 
operator $\varphi:X\to Y$is called pseudotopologically surjective if for 
every $n\in\mathbb{N}$ there exist $c_n>0$ such that for 
all $y\in Y^{\wideparen{n}}$ we can find $x\in X^{\wideparen{n}}$ 
with $\varphi^{\wideparen{n}}(x)=y$ 
and $\Vert x\Vert_{\wideparen{n}}\leq c_n\Vert y\Vert_{\wideparen{n}}$ 
\end{definition}

\begin{proposition}\label{PrDecsPsTopAdmEpiMorph} Let $\varphi:X\to Y$ be 
sequentially bounded operator, then the following are equivalent

\begin{enumerate}[label = (\roman*)]
    \item $\varphi$ $\square_{sqpTop}$-admissible epimorphism

    \item $\varphi$ is pseudotopologically surjective

    \item $\varphi^{\wideparen{1}}$ is topologically surjective
\end{enumerate}
\end{proposition}
\begin{proof}
$(i)\implies (ii)$ Assume $\varphi$ is $\square_{sqpTop}$-admissible epimorphism, 
then for some $c_1>0$ and any $y\in Y$ there exist $x\in X$ 
with $\varphi(x)=y$ and $\Vert x\Vert\leq c_1\Vert y\Vert$. 
Let $n\in\mathbb{N}$ and $y\in Y^{\wideparen{n}}$, then 
consider $x\in X^{\wideparen{n}}$, such that $\varphi(x_i)=y_i$ 
and $\Vert x_i\Vert\leq c_1\Vert y_i\Vert$ for all $i\in\mathbb{N}_n$. 
Let $e_i\in M_{1,n}$ be row-matrix with $1$ in the $i$-th place 
and $0$ in others, then 
$$
\Vert x\Vert_{\wideparen{n}} 
\leq  {\left(\sum\limits_{i=1}^n\Vert x_i\Vert_{\wideparen{1}}^2\right)}^{1/2} 
\leq  {\left(\sum\limits_{i=1}^n c_1^2
\Vert y_i\Vert_{\wideparen{1}}^2\right)}^{1/2} 
\leq
c_1{\left(\sum\limits_{i=1}^n\Vert e_i y\Vert_{\wideparen{n}}^2\right)}^{1/2}
$$
$$
\leq c_1{\left(\sum\limits_{i=1}^n\Vert e_i\Vert^2 \Vert
y\Vert_{\wideparen{n}}^2\right)}^{1/2} =c_1n^{1/2}\Vert y\Vert_{\wideparen{n}}
$$
Clearly, $\varphi(x)=y$ so $\varphi$ is pseudotopologically surjective.

$(ii)\implies (iii)$ Obvious.

$(iii)\implies (i)$ By proposition~\ref{PrCTopSurIsRetrInNor0} 
there exist bounded semilinear operator $\rho$ such that 
$\varphi\rho=1_Y$. This means, that $\square_{sqpTop}(\varphi)$ 
have right inverse, i.e. $\varphi$ 
is $\square_{sqpTop}$-admissible epimorphism. 
\end{proof}

Consider functors
$$
\begin{aligned}
\square_{sqRel} : SQNor \to Nor: X &\mapsto X^{\wideparen{1}}\\
\varphi&\mapsto\varphi \\
\end{aligned}
\qquad\qquad
\begin{aligned}
\square_{norTop} : Nor \to Nor_0: X &\mapsto X\\
\varphi&\mapsto\varphi \\
\end{aligned}
$$
Note the obvious identity $\square_{sqpTop}=\square_{norTop}\square_{sqRel}$.

\begin{proposition}\label{PrSQRelChar}
In the rigged category $(SQNor,\square_{sqRel})$
\begin{enumerate}[label = (\roman*)]
    \item $\square_{sqRel}$-free objects are exactly operator sequence spaces 
    sequentially topologically isomorphic to $\max(E)$ for some normed 
    space $E$. This category is freedom-loving. 

    \item Every retract of $\square_{sqRel}$-free object have maximal 
    structure of operator sequence space

    \item each $\square_{sqRel}$-projective object is $\square_{sqRel}$-free.
\end{enumerate}
\end{proposition}
\begin{proof}
$(i)$ Let $E\in Nor$. We will show that $\max(E)$ is $\square_{sqRel}$-free 
object with base $E$. Universal  arrow will be as 
such $j:E\to\square_{sqRel}(\max(E)):x\mapsto x$. Let $X$ be arbitrary operator 
sequence space and $\varphi:E\to\square_{sqRel}(X)$ be arbitrary bounded 
linear operator. Consider linear 
operator $\psi: \max(E)\to X:x\mapsto\varphi(x)$. 
From proposition~\ref{PrCharMaxSQ} 
it follows, that $\psi$ is sequentially bounded. 
Clearly, $\varphi=\square_{sqRel}(\psi)j$.
Since $X$ and $\varphi$ are arbitrary, then $\max(E)$ is $\square_{sqRel}$-free 
object. From proposition~\ref{PrUniqFr} it follows that $\square_{sqRel}$-free 
objects are sequentially topologically isomorphic to 
$\max(E)$. Since $E$ is arbitrary normed space, then the rigged 
category $(SQNor,\square_{sqRel})$ is freedom-loving.

$(ii)$ Let $\sigma:\max(E)\to X$ be a retraction in $SQNor$. Then $\sigma$ is 
topologically surjective and by 
proposition~\ref{PrMaxPreserveQuotients} $X$ have maximal structure.

$(iii)$ Let $P$ be $\square_{sqRel}$-projective object,then from 
proposition~\ref{PrFrCoFrProjInjObjProp} it follows that it is a 
retract of $\square_{sqRel}$-free object, and from paragraph $(ii)$ 
that $P$ have maximal structure of operator sequence 
space, i.e. $P=\max(\square_{sqRel}(P))$. Now from 
paragraph $(i)$ we see that $P$ is $\square_{sqRel}$-free.  
\end{proof}

\begin{proposition}\label{PrNorTopChar}
In the rigged category $(Nor, \square_{norTop})$
\begin{enumerate}[label = (\roman*)]
    \item $\square_{norTop}$-admissible epimorphisms are exactly 
    topologically surjective operators

    \item $\square_{norTop}$-free objects are normed spaces topologically 
    isomorphic to $l_1^0(\Lambda)$ with base $\mathbb{C}^\Lambda$.

    \item $\square_{norTop}$-projective objects are normed spaces 
topologically isomorphic to $l_1^0(\Lambda)$ for some set $\Lambda$. 
\end{enumerate}
\end{proposition}
\begin{proof}
$(i)$ Follows from proposition~\ref{PrCTopSurIsRetrInNor0}/

$(ii)$ Consider 
map $j:\mathbb{C}^\Lambda\to l_1^0(\Lambda)
:z_\lambda\to z_\lambda\delta_\lambda$. For a given semilinear 
bounded operator $\varphi:\mathbb{C}^\Lambda\to\square_{norTop}(X)$, 
where $X$ is an arbitrary normed space consider linear operator 
$\psi:l_1^0(\Lambda)\to X
:f\mapsto\sum_{\lambda\in\Lambda}f(\lambda)\varphi(1_\lambda)$. 
Since $\Vert\psi(f)\Vert\leq\Vert\varphi\Vert\Vert f\Vert$, 
then $\psi$ is bounded. Moreover it is straightforward to check 
that $\square_{norTop}(\psi)j=\varphi$. Uniqueness of $\psi$ follows 
from the chain of equalities 
$$
\psi(f)=\sum_{\lambda\in\Lambda}
f(\lambda)\psi(\delta_\lambda)=\sum_{\lambda\in\Lambda}
f(\lambda)\square_{norTop}(\psi)(j(1_\lambda))=\sum_{\lambda\in\Lambda}
f(\lambda)\varphi(1_\lambda)
$$

$(iii)$ See~\cite{GroTopNorPr} theorem 0.12
\end{proof}

\begin{theorem}\label{ThPsTopFrDesc}
A sequential  operator space is pseudotopologically projective if and only if 
it is sequentially topologically isomorphic to $\max(l_1^0(\Lambda))$ 
for some set $\Lambda$.
\end{theorem}
\begin{proof}
From proposition~\ref{PrNorTopChar} it follows that $l_1^0(\Lambda)$ 
is $\square_{norTop}$-free with base  $\mathbb{C}^\Lambda$. 
From proposition~\ref{PrSQRelChar} we get that $\max(l_1^0(\Lambda))$ 
is  $\square_{sqRel}$-free with base $l_1^0(\Lambda)$. Then 
from proposition~\ref{PrCompOfFrIsFr} we see that $\max(l_1^0(\Lambda))$ 
is $\square_{sqpTop}$-free with base $\mathbb{C}^\Lambda$. Now 
from proposition~\ref{PrUniqFr} we know that all pseudotopologically free 
objects are of the form $\max(l_1^0(\Lambda))$ for some set $\Lambda$.
\end{proof}

\begin{corollary}\label{CorSQSpaceIsImgPsTopAdmEpiMorph}
Every operator sequence space is an image of topologically surjective 
operator from $\max(l_1^0(\Lambda))$ for some set $\Lambda$.
\end{corollary}
\begin{proof}	
From theorem~\ref{ThPsTopFrDesc} it follows that the rigged 
category $(SQNor,\square_{sqpTop})$ is freedom-loving. Now the desired 
result follows from propositions~\ref{PrFrCoFrProjInjObjProp} 
and~\ref{PrDecsPsTopAdmEpiMorph}.
\end{proof}

\begin{theorem}\label{ThPsTopProjDesc}
Every pseudotopologically projective operator sequence space is sequentially 
topologically isomorphic to  $\max(l_1^0(\Lambda))$ for some set $\Lambda$.
\end{theorem}
\begin{proof}
Let $P$ be pseudotopologically projective operator sequence space. 
From proposition~\ref{ThPsTopFrDesc},~\ref{CorSQSpaceIsImgPsTopAdmEpiMorph} we 
see that there exist $\square_{sqpTop}$-admissible 
epimorphism $\sigma:\max(l_1^0(\Lambda))\to P$ for some set $\Lambda$. 
Since $\max(l_1^0(\Lambda))$ is  $\square_{sqpTop}$-free object, then from 
proposition~\ref{PrRetractsProjInj} we get that $\sigma$ is a  retraction 
in $SQNor$. From paragraph $(ii)$ of proposition~\ref{PrSQRelChar} we get, the 
structure of operator sequence space $P$ is maximal, 
i.e. $P=\max(\square_{sqRel}(P))$. Since $\sigma$ is a retraction in $SQNor$, 
it is also a retraction in $Nor$ from the space $l_1^0(\Lambda)$. 
By proposition~\ref{PrNorTopChar} the space $l_1^0(\Lambda)$ 
is $\square_{sqRel}$-free. As $\square_{sqRel}(P)$ is its retract in $Nor$, 
then from proposition~\ref{PrFrCoFrProjInjObjProp} we 
see that $\square_{sqRel}(P)$ is  $\square_{norTop}$-projective. In this case 
again from proposition~\ref{PrNorTopChar} we conclude, 
that $\square_{sqRel}(P)$ is topologically isomorphic to $l_1^0(\Lambda')$ 
for some set $\Lambda'$. Applying $\max$ functor to this isomorphism, 
we establish sequential topological isomorphism 
between  $P=\max(\square_{sqRel}(P))$ and $\max(l_1^0(\Lambda'))$.
\end{proof}

Similar propositions are valid in Banach case 
(just replace $l_1^0$ spaces with $l_1$ space).






























\section{Cofree operator sequence spaces}

In what follows we will use the following simple observation 

From propositions~\ref{PrDualOfCoprodIsProd} 
and~\ref{PrSQSpaceIsSBFromT2n} it follows, that there exist 
sequential isometric isomorphisms
$$
{(t_2^\infty)}^\triangle =\bigoplus{}_\infty \{
    {(t_2^n)}^\triangle:n\in\mathbb{N} 
\}
=\bigoplus{}_\infty \{l_2^n:n\in\mathbb{N} \} =l_2^\infty
$$
Therefore applying again proposition~\ref{PrDualOfCoprodIsProd} we get a 
sequential isometric isomorphism:
$$
{\left(\bigoplus{}_1^0 \{t_2^\infty:\lambda\in\Lambda \}\right)}^\triangle
=\bigoplus{}_\infty \{l_2^\infty:\lambda\in\Lambda \}
$$
 
\subsection{Metric cofreedom}

Consider functor
$$
\begin{aligned}
\square_{sqMet}^d 
: SQNor_1 \to Set^o
: X &\mapsto \prod \left \{
    B_{{(X^\triangle)}^{\wideparen{n}}}
    :n\in\mathbb{N}\right \} \\
\varphi&\mapsto\prod\left \{ 
    {(\varphi^\triangle)}^{\wideparen{n}}|_{
        B_{{(Y^\triangle )}^{\wideparen{n}}}
    }^{
        B_{{(X^\triangle)}^{\wideparen{n}}}
    }
    :n\in\mathbb{N}\right \} \\
\end{aligned}
$$
\begin{proposition}\label{PrDecsMetrAdmMonoMorph}
$\square_{sqMet}^d$-admissible monomorphisms are exactly 
sequentially isometric operators.
\end{proposition}
\begin{proof}
A morphism $\varphi$ is $\square_{sqMet}^d$-admissible monomorphism 
if and only if $\square_{sqMet}^d(\varphi)$ is invertible from the 
left in $Set^o$. This is equivalent to surjectivity of 
$\square_{sqMet}^d(\varphi^\triangle)$. The latter is equivalent to 
surjectivity of 
${(\varphi^\triangle)}^{\wideparen{n}}
|_{B_{X^{\wideparen{n}}}}^{B_{Y^{\wideparen{n}}}}$ 
for all $n\in\mathbb{N}$. This means that 
${(\varphi^\triangle)}^{\wideparen{n}}$ 
is strictly coisometric for each $n\in\mathbb{N}$, i.e. $\varphi^\triangle$ is 
sequentially strictly coisometric. By theorem 
~\ref{ThDualSQOps} it is equivalent to $\varphi$ being sequentially isometric.
\end{proof}

\begin{theorem}\label{ThMetCoFrDesc}
Metrically cofree operator sequence space with base $\Lambda$ is up to 
sequential isometric isomorphism  a $\bigoplus{}_\infty$-sum of copies of 
the space $l_2^\infty:=\bigoplus{}_\infty \{l_2^n:n\in\mathbb{N} \}$, indexed 
by elements $\Lambda$.
\end{theorem}
\begin{proof}
Let $\Lambda$ be an arbitrary. Consider commutative diagram
$$
\xymatrix{SQNor_1^o 
\ar[d]_{\nabla } \ar[rr]^{(\square_{sqMet}^{d})^o} & &  % chktex 3
{Set}\ar[d]^{1_{Set}} \\  % chktex 3
SQNor_1\ar[rr]^{\square_{sqMet}}&  &{Set}}  % chktex 3
$$
Here ${}^\nabla$ is a covariant version of ${}^\triangle$ functor. That diagram 
is indeed commutative since for any operator sequence spaces $X$, $Y$ and 
arbitrary $\varphi\in\mathcal{SB}(X,Y)$ holds
$$
1_{Set}({(\square_{sqMet}^d)}^o(\varphi))
=\prod\limits_{n\in\mathbb{N}}
{(\varphi^\triangle )}^{\wideparen{n}}|_{
    B_{{(Y^\triangle)}^{\wideparen{n}}}
}^{
    B_{{(X^\triangle )}^{\wideparen{n}}}
}
=\square_{sqMet}({}^\nabla(\varphi))
$$
From remark~\ref{CorSQUnivPropMaxTenProd} we see that ${}^\nabla$ have left 
adjoint functor, which is  ${}^\triangle$. Analogously $1_{Set}$ is adjoint 
to itself from the left and from the right. 
By theorem~\ref{ThMetrFrDesc} 
the object $\bigoplus{}_1^0 \{t_2^\infty:\lambda\in\Lambda \}$ 
is  $\square_{sqMet}$-free, so by proposition~\ref{PrFunctorMapFrToFr} 
the object 
${(\bigoplus{}_1^0 \{t_2^\infty:\lambda\in\Lambda \})}^\triangle
=\bigoplus{}_\infty \{l_2^\infty:\lambda\in\Lambda \}$ 
is ${(\square_{sqMet}^d)}^o$-free, which is the same 
as being $\square_{sqMet}^d$-cofree. Since the set $\Lambda$ is arbitrary, 
using proposition~\ref{PrUniqFr} we get that all $\square_{sqMet}$-cofree 
objects with base $\Lambda$ are sequentially isometrically isomorphic to 
the space constructed above.
\end{proof}

\begin{corollary}\label{CorSQSpaceIsFromMetrAdmMonoMorph}
From every operator sequence space there exist a sequentially isometic 
operator into $\bigoplus_\infty \{l_2^\infty:\lambda\in\Lambda \}$  for 
some set $\Lambda$.
\end{corollary}
\begin{proof}
From theorem~\ref{ThMetrFrDesc} it follows that the rigged 
category $(SQNor_1,\square_{sqMet}^d)$ is cofreedom-loving. Now the desired 
result from propositions~\ref{PrFrCoFrProjInjObjProp} 
and~\ref{PrDecsMetrAdmMonoMorph}.
\end{proof}

\begin{proposition}\label{PrCharacDualSQSp} An operator sequence space $X$ 
is a dual operator sequence space if and there is sequentially isometric 
weak${}^*$-weak${}^*$ homeomorphism onto weak${}^*$ closed 
subspace of $\bigoplus_\infty \{l_2^\infty:\lambda\in\Lambda \}$ 
for some set $\Lambda$.
\end{proposition}
\begin{proof}
Assume $X$ is a dual operator sequence space with sequential 
predual $X_\triangle$. By proposition~\ref{CorSQSpaceIsImgMetrAdmEpiMorph} for 
some set $\Lambda$ we have sequentially coisometric 
operator $\pi:\bigoplus_1^0 \{t_2^\infty:\lambda\in\Lambda \}\to X_\triangle$. 
By theorem~\ref{ThDualSQOps} operator $\pi^\triangle$ is a sequential isometry 
from $X_\triangle^\triangle=X$ 
into ${(\bigoplus_1^0 \{t_2^\infty:\lambda\in\Lambda \})}^\triangle
=\bigoplus_\infty \{l_2^\infty:\lambda\in\Lambda \}$. 
By lemma A.2.5~\cite{BleOpAlgAndMods} operator $\pi^\triangle$ is 
weak${}^*$-weak${}^*$ homeomorphism onto its weak${}^*$ closed image.

Conversely, if $X$ is a weak${}^*$ closed subspace of 
$Y:=\bigoplus_\infty \{l_2^\infty:\lambda\in\Lambda \}$  for some 
set $\Lambda$, then by proposition~\ref{PrDualForWStarClQuotsAndSubsp} we 
have $X={(Y/X_\perp)}^\triangle$. Hence $X$ is dual operator sequence space 
with sequential predual $X_\triangle:=Y/X_\perp$.
\end{proof}

Similar propositions are valid in Banach case.






























\subsection{Topological cofreedom}

Consider functor
$$
\begin{aligned}
\square_{sqTop}^d : SQNor \to Nor_0^o, X &\mapsto \bigoplus{}_\infty
 \{{(X^{\triangle })}^{\wideparen{n}} : n \in \mathbb{N} \} \\
\varphi&\mapsto\bigoplus{}_\infty  \{
    {(\varphi^\triangle )}^{\wideparen{n}}:n\in\mathbb{N}
 \}
\end{aligned}
$$

\begin{proposition}\label{PrDecsTopAdmMonoMorph}
$\square_{sqTop}^d$-admissible monomorphisms are exactly sequentially 
topologically injective operators.
\end{proposition}
\begin{proof}
A morphism $\varphi$ is a $\square_{sqTop}^d$-admissible monomorphism if and 
only if $\square_{sqTop}^d(\varphi)$ is invertible as morphism in $Nor_0^o$. 
This is equivalent to say that 
$\square_{sqTop}^d(\varphi)=\square_{sqTop}(\varphi^\triangle)$ is invertible 
from the right as morphism in $Nor_0$. 
From proposition~\ref{PrDecsTopAdmEpiMorph} this is equivalent to sequential 
topological surjectivity of $\varphi^\triangle$. 
By theorem~\ref{ThDualSQOps} this is equivalent to $\varphi$ being 
sequentially topologically injective.
\end{proof}

\begin{theorem}\label{ThTopCoFrDesc}
A operator sequence space is topologically cofree if and on;y if it is 
sequentially topologically isomorphic to $\bigoplus{}_\infty$ sum of copies of 
the space $l_2^\infty$ indexed by elements of some set $\Lambda$.
\end{theorem}
\begin{proof}
Let $\Lambda$ be an arbitrary set. Consider commutative diagram
$$
\xymatrix{SQNor^o 
\ar[d]_{\nabla } \ar[rr]^{(\square_{sqTop}^d)^o} & & {Nor_0}  % chktex 3
\ar[d]^{1_{Nor_0}} \\  % chktex 3
SQNor\ar[rr]^{\square_{sqTop}} & & {Nor_0}}  % chktex 3
$$
Here $\nabla$ is a covariant version of $\triangle$ functor.
This diagram is commutative since for any operator sequence spaces$X$, $Y$ 
and arbitrary $\varphi\in\mathcal{SB}(X,Y)$ holds
$$
1_{Nor_0}({(\square_{sqTop}^d)}^o(\varphi)) =\bigoplus{}_\infty
 \{{(\varphi^\triangle )}^{\wideparen{n}} : n \in \mathbb{N} \}
=\square_{sqTop}({}^\nabla(\varphi))
$$
From remark~\ref{CorSQUnivPropMaxTenProd} we see that ${}^\nabla$ have left 
adjoint functor,which is ${}^\triangle$. Analogously $1_{Nor_0}$ is adjoint to 
itself from the left and from the right. By theorem~\ref{ThTopFrDesc} 
the object $\bigoplus{}_1^0 \{t_2^\infty:\lambda\in\Lambda \}$ 
is $\square_{sqTop}$-free, so by proposition~\ref{PrFunctorMapFrToFr} the object 
${(\bigoplus{}_1^0 \{t_2^\infty:\lambda\in\Lambda \})}^\triangle
=\bigoplus{}_\infty \{l_2^\infty:\lambda\in\Lambda \}$ 
is ${(\square_{sqTop}^d)}^o$-free which is the same as 
being $\square_{sqTop}^d$-cofree. Using proposition~\ref{PrUniqFr} we get, 
that all $\square_{sqTop}$-cofree objects with base 
$\mathbb{C}^\Lambda$ are sequentially topologically isomorphic to the 
space constructed above.
\end{proof}

\begin{corollary}\label{PrMetrCoFrIsTopFr}
Every metrically cofree operator sequence space is topologically cofree.
\end{corollary}


\begin{corollary}\label{CorSQSpaceIsFromTopAdmMonoMorph}
From every operator sequence space there exist sequentially topologically 
injective operator into $\bigoplus_\infty \{l_2^\infty:\lambda\in\Lambda \}$  
for some set $\Lambda$.
\end{corollary}
\begin{proof} From theorem~\ref{ThTopCoFrDesc} it follows that the rigged 
category $(SQNor,\square_{sqTop}^d)$ is cofreedom-loving. Now the desired 
result follows from propositions~\ref{PrFrCoFrProjInjObjProp} 
and~\ref{PrDecsTopAdmMonoMorph}.
\end{proof} 

Similar propositions are valid in Banach case.





























\subsection{Pseudotopological cofreedom and injectivity}

Consider functor
$$
\begin{aligned}
\square_{sqpTop}^d : SQNor \rightarrow Nor_0^o, X &\mapsto
{(X^\triangle)}^{\wideparen{1}}\\
\varphi&\mapsto\varphi^\triangle
\end{aligned}
$$
sending operator sequence space to the underlying semilinear normed space of 
the first amplification of its sequential dual, and morphism is mapped to its 
adjoint  considered as bounded semilinear operator.

\begin{definition}\label{DefPsSQTopInjOp} A sequentially bounded 
operator $\varphi:X\to Y$ is called pseudotopologically injective, 
if for every $n\in\mathbb{N}$ 
there exist $c_n>0$ such that for all $x\in X^{\wideparen{n}}$ 
holds $c_n\Vert\varphi^{\wideparen{n}}(x)\Vert_{\wideparen{n}}
\geq \Vert x\Vert_{\wideparen{n}}$
\end{definition}

\begin{proposition}\label{PrDecsPsTopAdmMonoMorph}
Let $\varphi:X\to Y$ be sequentially bounded operator between operator 
sequence spaces, then the following are equivalent
\begin{enumerate}[label = (\roman*)]
    \item $\varphi$ is $\square_{sqpTop}$-admissible monomorphism

    \item $\varphi$ is pseudotopologically injective

    \item $\varphi^{\wideparen{1}}$ is topologically injective
\end{enumerate}
\end{proposition}
\begin{proof}
$(i)\implies (ii)$ Let $\varphi$ 
be $\square_{sqpTop}^d$-admissible monomorphism. 
Then $\square_{sqpTop}^d(\varphi)$ is invertible from the left as morphism 
in $Nor_0^o$. This is equivalent to say that
$\square_{sqpTop}^d(\varphi)=\square_{sqpTop}(\varphi^\triangle)$ is invertible 
from the right as morphism in $Nor_0$. 
From proposition~\ref{PrDecsPsTopAdmEpiMorph} it is equivalent to 
pseudotopological surjectivity of $\varphi^\triangle$. 
By proposition~\ref{PrDualOps} this is equivalent to $\varphi$ being 
pseudotopologically injective.

$(ii)\implies (iii)$ Obvious.

$(iii)\implies (i)$ Let $\varphi^{\wideparen{1}}$ be topologically injective 
then by proposition~\ref{PrDualOps} ${(\varphi^\triangle)}^{\wideparen{1}}$ 
is topologically surjective. From proposition~\ref{PrDecsPsTopAdmEpiMorph} 
$\varphi^\triangle$ is $\square_{sqpTop}$-admissible epimorphism, 
i.e. $\square_{sqpTop}(\varphi^\triangle)$ is invertible from the right as 
morphism in $Nor_0$. Hence $\square_{sqpTop}^d(\varphi)
=\square_{sqpTop}(\varphi^\triangle)$ is invertible from the left as 
morphism in $Nor_0^o$. Hence $\varphi$  is $\square_{sqpTop}^d$-admissible 
monomorphism. 
\end{proof}

Consider functors
$$
\begin{aligned}
\square_{sqRel}^d : SQNor \to Nor^o: X &\mapsto{(X^\triangle)}^{\wideparen{1}}\\
\varphi&\mapsto\varphi \\
\end{aligned}
\qquad\qquad
\begin{aligned}
\square_{norTop}^d : Nor^o \to Nor_0^o: X &\mapsto X\\
\varphi&\mapsto\varphi \\
\end{aligned}
$$
Note the obvious identity 
$\square_{sqpTop}^d=\square_{norTop}^d\square_{sqRel}^d$.

\begin{proposition}\label{PrSQReldChar}
In the rigged category $(SQNor,\square_{sqRel}^d)$
\begin{enumerate}[label = (\roman*)]
    \item $\square_{sqRel}^d$-cofree objects are exactly operator 
    sequence spaces  sequentiall topologically isomorphic to $\min(E^*)$ 
    for some normed space $E$. This category is cofreedom-loving. 

    \item Every retract of $\square_{sqRel}^d$-cofree object have minimal 
    structure of operator sequence space.
\end{enumerate}
\end{proposition}
\begin{proof}
$(i)$ Let $E\in Nor$. Consider commutative diagram
$$
\xymatrix{SQNor^o 
\ar[d]_{\nabla } \ar[rr]^{(\square_{sqRel}^{d})^o} & &  % chktex 3
{Nor}\ar[d]^{1_{Nor}} \\  % chktex 3
SQNor\ar[rr]^{\square_{sqRel}}&  &{Nor}}  % chktex 3
$$
Here ${}^\nabla$ is a covariant version of ${}^\triangle$ functor.
This diagram is commutative since for any operator sequence spaces $X$, $Y$ 
and arbitrary $\varphi\in\mathcal{SB}(X,Y)$ holds
$$
1_{Nor}({(\square_{sqRel}^d)}^o(\varphi)) =\varphi^\triangle
=\square_{sqRel}({}^\nabla(\varphi))
$$
From remark~\ref{CorSQUnivPropMaxTenProd} we see that ${}^\nabla$ have 
left adjoint functor, which is ${}^\triangle$. Analogously $1_{Nor}$ is adjoint 
to itself from the left and from the right. By proposition~\ref{PrSQRelChar} 
the object $\max(E)$ is $\square_{sqRel}$-free, so 
from propositions~\ref{PrFunctorMapFrToFr},~\ref{PrDualityAndMinMax} the object 
${(\max(E))}^\triangle=\min(E^*)$ is ${(\square_{sqRel}^d)}^o$-free, which is 
the same as being $\square_{sqRel}^d$-cofree. 
Since the space $E$ is arbitrary, using proposition~\ref{PrUniqFr} we get 
that all $\square_{sqRel}^d$-cofree objects with base $E$ are sequentially 
topologically isomorphic to the space constructed above. As the consequence 
the rigged category $(SQNor,\square_{sqRel}^d)$ is cofreedom-loving.

$(ii)$ Let $\sigma:\min(E^*)\to X$ be a retraction in $SQNor$. Right inverse of 
$\sigma$ we will denote by $\rho$. Since $\rho$ is topologically injective, 
then by proposition~\ref{PrMinPreserveEmbedings} we see that $X$ 
have minimal struture.
\end{proof}

\begin{proposition}\label{PrNorTopdChar}
In the rigged category $(Nor^o, \square_{norTop}^d)$
\begin{enumerate}[label = (\roman*)]
    \item $\square_{norTop}^d$-admissible monomorphisms are exactly 
    topologically surjective operators

    \item $\square_{norTop}^d$-cofree objects are exactly normed spaces 
    topologically isomorphic to $l_1^0(\Lambda)$ with base $\mathbb{C}^\Lambda$.

    \item $\square_{norTop}^d$-injective objects are normed spaces 
    topologically isomorphic to  $l_1^0(\Lambda)$ for some set $\Lambda$. 
\end{enumerate}
\end{proposition}
\begin{proof}
All results follow from proposition~\ref{PrNorTopdChar} if one note 
that $\square_{norTop}^d=\square_{norTop}^o$.
\end{proof}

\begin{theorem}\label{ThPsTopCoFrDesc} 
A operator sequence space is pseudotopologically cofree if and only if it 
is sequentially topologically isomorphic to $\min(l_\infty(\Lambda))$ 
for some set $\Lambda$.
\end{theorem}
\begin{proof}
From proposition~\ref{PrNorTopdChar} it follows that $l_1^0(\Lambda)$ 
is $\square_{norTop}^d$-cofree with base $\mathbb{C}^\Lambda$. 
From proposition~\ref{PrSQReldChar} we get that 
$\min({l_1^0(\Lambda)}^*)
=\min(l_\infty(\Lambda))$ is  $\square_{sqRel}^d$-cofree with base 
$l_1^0(\Lambda)$. Then from proposition~\ref{PrCompOfFrIsFr} we see 
that  $\min(l_\infty(\Lambda))$ is $\square_{sqpTop}^d$-cofree with 
base $\mathbb{C}^\Lambda$. Now from proposition~\ref{PrUniqFr} we know that 
all pseudotopologically cofree objects are of the 
form $\min(l_\infty(\Lambda))$ for some set $\Lambda$.
\end{proof}

\begin{corollary}\label{CorSQSpaceIsFromPsTopAdmMonoMorph}
From every operator sequence space there exist topologically 
injective operator into $\min(l_\infty(\Lambda))$ for some set $\Lambda$.
\end{corollary}
\begin{proof}
From theorem~\ref{ThPsTopCoFrDesc} it follows that the 
rigged category $(SQNor,\square_{sqTop}^d)$ is cofreedom-loving. Now 
the desired result  follows from propositions~\ref{PrFrCoFrProjInjObjProp} 
and~\ref{PrDecsPsTopAdmMonoMorph}.
\end{proof}

\begin{theorem}\label{ThPsTopInjDesc} 
Every pseudotopologically injective operator sequence space is sequentially 
topologically isomorphic to $\min(F)$, where $F$ is a retract in $Nor$ of 
the space $l_\infty(\Lambda)$ for some set $\Lambda$.
\end{theorem}
\begin{proof}
Let $I$ be pseudotopologically injective operator sequence space. 
From propositions~\ref{ThPsTopCoFrDesc},~\ref{CorSQSpaceIsFromPsTopAdmMonoMorph} 
we see that there exist $\square_{sqpTop}^d$-admissible 
monomorphism $\sigma:I\to\min(l_\infty(\Lambda))$ for some set $\Lambda$. 
Since $\min(l_\infty(\Lambda))$ is $\square_{sqpTop}^d$-cofree object, then 
from proposition~\ref{PrRetractsProjInj} it follows that $\sigma$ is a 
coretraction in $SQNor$. Let $\rho$ be right inverse morphism of $\sigma$ 
in $SQNor$. It is a retraction in $SQNor$, then from paragraph $(ii)$ of 
proposition~\ref{PrSQReldChar} we get that the structure of operator sequence 
space $I$ is minimal, i.e. $I=\min(\square_{sqRel}(I))$. Since $\rho$ is a 
retraction in $SQNor$, it is retraction in $Nor$ from $l_\infty(\Lambda)$ 
to $F:=\square_{sqRel}(I)$.
\end{proof}

Similar propositions are valid in Banach case.

\begin{remark} Unfortunately, in theorem~\ref{ThPsTopInjDesc} by analogy with 
theorem~\ref{ThPsTopProjDesc} we can't state that retracts 
of $l_\infty(\Lambda)$ are of the form $l_\infty(\Lambda')$ for some 
set $\Lambda'$. Indeed, in~\cite{RosInjLmuSp} corollary 4.4 it was shown 
existence of topologically injective space $F$ which can't be topologically 
isomorphic to any dual space and in particular to $l_\infty(\Lambda')$ for 
any set $\Lambda'$. On the other hand for some set $\Lambda$ there exist 
isometric embedding $i:F\to l_\infty(\Lambda)$. Since $F$ is topologically 
injective, then by proposition~\ref{PrRetractsProjInj} this is a coretraction. 
Therefore, $F$ is a retract of $l_\infty(\Lambda)$, which can not be 
topologically isomorphic to $l_\infty(\Lambda')$ for any set $\Lambda'$.
\end{remark}


\newpage
\begin{thebibliography}{999}
\bibitem{HelQFA}
\textit{Helemskii A. Ya.} Quantum Functional Analysis: Non-Coordinate Approach. 
American Mathematical Society; 
1st edition (December 10, 2010)
\bibitem{HelFA}
\textit{Helemskii A. Ya.} Lectures and Exercises on Functional Analysis. 
Translations of Mathematical Monographs, vol. 233, 2006.
\bibitem{EROpSp}
\textit{Ruan Z.-J., Effros E.} Operator Spaces, 
London Math. Soc. Monographs, New Series 23, 
Oxford University Press, New York 2000.
\bibitem{MurphCstarOpTh}
\textit{Murphy G. J.} $C^*$-Algebras and Operator Theory. 
Academic Press, 1 edition (September 11, 1990)
\bibitem{RudinFA}
\textit{Rudin W.} Functional Analysis. McGraw-Hill Science/Engineering/Math; 
2 edition (January 1, 1991).
\bibitem{LamOpFolgen}\textit{Lambert A.} Operatorfolgenr\"{a}ume. 
Eine Kategorie auf dem Weg von den Banach-R\"{a}umen zu den Operatorr\"{a}umen. 
Dissertation zur Erlangung des Grades Doktor der Naturwissenschaften der 
Technisch-Naturwissenschaftlichen Fakult\"{a}t I der Universit\"{a}t 
des Saarlandes. Saarbr\"{u}cken, 2002.
\bibitem{DefFloTensNorOpId}
\textit{Defant A., Floret K.} Tensor norms and operator ideals, 
North-Holland Amsterdam,
London, New York, Tokio 1993.
\bibitem{BleOpAlgAndMods}
\textit{Blecher P., Le Merdy C.} Operator Algebras and Their Modules: 
An Operator Space Approach, Oxford University Press, USA (January 13, 2005)
\bibitem{FabZizBanSpTh} 
\textit{M. Fabian, P. Habala, V. Montesinos, V. Zizler} Banach Space Theory: 
The Basis for Linear and Nonlinear Analysis, Springer; 
2011 edition (December 15, 2010)
\bibitem{BrownItoUniquePredual}
\textit{L. Brown, T. Ito} Classes of Banach spaces with unique 
isometric preduals. Pacific J. Math. Volume 90, Number 2 (1980), 249--492
\bibitem{HelMetrFrQmod}
\textit{Helemskii A. Ya.} Metric freeness and projectivity for classical and 
quantum normed modules. Sb. Math. 204, No. 7, 1056--1083 (2013); 
translation from Mat. Sb. 204, No. 7, 127--158 (2013).
\bibitem{ShtTopFr}
\textit{Steiner S.} Topological freedom for classical and quantum modules. 
Perprint.
\bibitem{GroTopNorPr}
\textit{N. Groenbaek.} Lifting problems for normed spaces. Preprint.
\bibitem{RosInjLmuSp}
\textit{Haskell P. Rosenthal} On injective Banach spaces and the spaces 
$L_\infty(\mu)$ for finite measures $\mu$. Acta Mathematica, 
July 1970, Volume 124, Issue 1, pp 205--248
\end{thebibliography}
\end{document}