% chktex-file 19
\documentclass[12pt]{article}
\usepackage[left=2cm,right=2cm, top=2cm,bottom=2cm,bindingoffset=0cm]{geometry}
\usepackage{amssymb,amsmath,amsthm}
\usepackage[T1,T2A]{fontenc}
\usepackage[utf8]{inputenc}
\usepackage[russian]{babel}
\usepackage[colorlinks=true, urlcolor=blue, linkcolor=blue, citecolor=blue,
    pdfborder={0 0 0}]{hyperref}

%\pagestyle{plain}

\begin{document}

\begin{center}

\Large \textbf{Топологически инъективные $C^*$-алгебры}\\[0.5cm]
\small {Н. Т. Немеш}\\[0.5cm]

\end{center}
\thispagestyle{empty}

\textbf{Аннотация:} В данной заметке дан критерий топологической инъективности
$AW^*$-алгебры как правого банахова модуля над собой. Также дано необходимое
условие топологической инъективности произвольной $C^*$-алгебры. 

\medskip
\textbf{Ключевые слова:} топологическая инъективность, метрическая
инъективность, $AW^*$-алгебра, $C^*$-алгебра, свойство l.u.st.

\medskip
\textbf{Abstract:} In this short note a criterion of topological injectivity of
an $AW^*$-algebra as a right Banach module over itself is given. A necessary
condition for a $C^*$-algebra to be topologically injective is obtained.

\medskip
\textbf{Keywords:} topological injectivity, metric injectivity, $AW^*$-algebra,
$C^*$-algebra, the l.u.st.\ property.

\bigskip

Во многих категориях функционального анализа инъективные объекты часто
оказываются тесно связанными с $C^*$-алгебрами. Например, 1-инъективные банаховы
пространства являются $C^*$-алгебрами~\cite{NachThOfHahnBanachType},
~\cite{GooProjInNorLinSp},~\cite{KellBanSpWithExtProp},
~\cite{HasumiExtPropComplBanSp}, инъективные операторные пространства являются
углами инъективных $C^*$-алгебр [\cite{RuanOpSp}, теорема 6.1.6]. Среди
различных типов инъективности нам понадобятся два: метрический и топологический.
Первый требует существования продолжения морфизма с сохранением нормы, а второй
требует существования какого-нибудь ограниченного продолжения. В этой заметке
исследуется вопрос метрической и топологической инъективности $C^*$-алгебр как
правых модулей над собой\footnote{Работа выполнена при поддержке Российского
фонда фундаментальных исследований (грант номер 15--01--08392).}. Эти два типа
инъективности банаховых модулей изучены значительно меньше, чем хорошо известная
относительная инъективность, определенная Хелемским в
~\cite{HelemSheinbergFlatBanModAndAmenBanAlg}.

В дальнейшем $A$ обозначает необязательно унитальную банахову алгебру. Через
$A_+$ мы обозначим унитизацию $A$ как банаховой алгебры. Если $A$ ---
$C^*$-алгебра, то ее унитизацию как $C^*$-алгебры будем обозначать $A_\#$.
$A$-морфизмом будем называть непрерывный морфизм правых банаховых $A$-модулей.
Сформулируем два определения инъективности, которые упоминались ранее для
категории банаховых модулей:

\medskip
\textbf{Определение 1.} (\cite{HelMetrFrQMod}, определение 4.3) \textit{Правый
$A$-модуль $J$ называется метрически инъективным, если для любого
изометрического $A$-морфизма $\xi:Y\to X$ и любого $A$-морфизма $\phi:Y\to J$
существует $A$-морфизм $\psi:X\to J$ такой, что $\psi\xi=\phi$  и
$\Vert\psi\Vert=\Vert\phi\Vert$.}

\medskip
\textbf{Определение 2.} (\cite{HelMetrFrQMod}, определение 4.3) \textit{Правый
$A$-модуль $J$ называется топологически инъективным, если для любого
топологически инъективного $A$-морфизма $\xi:Y\to X$ и любого $A$-морфизма
$\phi:Y\to J$ существует $A$-морфизм $\psi:X\to J$ такой, что $\psi\xi=\phi$.}

\medskip
Далее символ $\bigoplus_\infty$ означает $\ell_\infty$-сумму банаховых
пространств. Стандартный пример метрически и топологически инъективного
$A$-модуля --- это $\bigoplus{}_\infty \{A_+^*:\lambda\in\Lambda \}$, то есть
$\ell_\infty$-сумма копий банахова пространства $A_+^*$ в количестве равном
мощности множества $\Lambda$. В терминологии Хелемского такие модули называются
метрически косвободными~\cite{HelMetrFrQMod}. Простейший способ проверки
топологической инъективности некоего модуля --- это доказательство того, что он
дополняем как подмодуль в некотором метрически косвободном $A$-модуле. В случае
метрической инъективности требуется $1$-дополняемость, то есть существование
проектора нормы $1$ являющегося морфизмом правых $A$-модулей. Любой банахов
модуль можно изометрически вложить как подмодуль в некоторый метрически
косвободный модуль.

Необходимым условием инъективности $C^*$-алгебры является унитальность.
Действительно, так как $A$ инъективна, то она дополняема как подмодуль в своей
унитизации $A_\#$ посредством некоторого проектора $P:A_\#\to A_\#$. Более того,
$P$ является морфизмом правых $A$-модулей, поэтому образ единицы алгебры $A_\#$
под действием $P$ есть левая единица в $A$. Так как $A$ --- $C^*$-алгебра, то
она имеет и двустороннюю единицу. Теперь из работ Хаманы~\cite{HamInjEnvBanMod}
и Такесаки~\cite{TakHanBanThAndJordDecomOfModMap} следует:

\medskip
\textbf{Предложение 3.} (Хамана, Такесаки) \textit{$C^*$-алгебра метрически
инъективна как правый модуль над собой тогда и только тогда, когда она является
коммутативной $AW^*$-алгеброй.}

\medskip
Отметим, что в оригинальной статье утверждение было доказано для левых модулей,
но его легко модифицировать и для правых модулей.

Таким образом, вопрос о метрической инъективности $C^*$-алгебр решен полностью.
Перейдем к топологической инъективности. Здесь нам понадобится
банахово-геометрическое свойство l.u.st [\cite{DiestAbsSumOps}, параграф 17].
Самое короткое его определение звучит так: банахово пространство $E$ обладает
свойством l.u.st. если $E^{**}$ изоморфно дополняемому подпространству некоторой
банаховой решетки. 

Будем считать, что $C^*$-алгебра $A$ реализована как конкретная $C^*$-алгебра на
некотором гильбертовом пространстве $H$. Тогда $A$ можно считать подмодулем в
правом $A$-модуле $F:=\bigoplus_\infty \{H^*:f\in H^*, \Vert f\Vert\leq 1 \}$
посредством вложения
$$
I
:A\to\bigoplus_\infty \{
    H^* :f\in H^*,\Vert f\Vert\leq 1 \}
:a\mapsto \bigoplus_\infty \{a^*(f):f\in H^*, \Vert f\Vert\leq 1 \}.
$$
Отметим, что $F$ является банаховой решеткой, а значит имеет свойство l.u.st
[\cite{DiestAbsSumOps}, теорема 17.1]. Если $A$ --- топологически инъективная
$C^*$-алгебра, то она дополняема в $F$. Как легко видеть, свойство l.u.st.
наследуется дополняемыми подпространствами. Отсюда получается следующее
необходимое условие топологической инъективности. 

\medskip
\textbf{Предложение 4.} \textit{Пусть $A$ --- $C^*$-алгебра, топологически
инъективная как правый $A$-модуль. Тогда $A$ обладает свойством l.u.st.}

\medskip
Известно, что всякая полная матричная алгебра $M_n(\mathbb{C})$ $1$-дополняема
как банахово пространство в любой объемлющей $C^*$-алгебре
~\cite{LauLoyWillisAmnblOfBanAndCStarAlgsOfLCG}. Отсюда и из результатов Гордона
и Льюиса~\cite{GorLewAbsSmOpAndLocUncondStrct} следует, что $C^*$-алгебры со
свойством l.u.st не могут содержать полную матричную алгебру $M_n(\mathbb{C})$
как ${}^*$-подалгебру для произвольного $n\in\mathbb{N}$. Как следствие,
$A^{**}$ не может содержать $M_\infty:=\bigoplus_\infty
\{M_n(\mathbb{C}):n\in\mathbb{N} \}$ как ${}^*$-подалгебру. Теперь, из
[\cite{LauLoyWillisAmnblOfBanAndCStarAlgsOfLCG}, теорема 2.5] следует, что все
неприводимые представления $A$ конечномерны и их размерность не превосходит
общей константы. $C^*$-алгебры с таким свойством называют субоднородными. Для
них есть своя теорема представления: они являются замкнутыми ${}^*$-подалгебрами
матричных алгебр $M_n(C(K))$ для некоторого компактного хаусдорфова пространства
$K$ и некоторого натурального числа $n$ [\cite{BlackadarOpAlg}, предложение
IV.1.4.3]. Из этой теоремы представления, нестрого говоря, следует, что
топологически инъективные $C^*$-алгебры почти коммутативны. Отметим, что по
предложению 3 все метрически инъективные $C^*$-алгебры коммутативны.

Теперь приведем несколько примеров топологически инъективных $C^*$-алгебр.

\medskip
\textbf{Предложение 5.} \textit{Пусть $H$ --- конечномерное гильбертово
пространство. Тогда $\mathcal{B}(H)$ топологически инъективен как правый
$\mathcal{B}(H)$-модуль.}

\medskip
Из сказанного ранее следует, что для бесконечномерного $H$ предложение 5
неверно.

\medskip
\textbf{Предложение 6.} \textit{Пусть $K$ --- стоуново пространство. Тогда
$C(K)$ топологически инъективен как правый $C(K)$-модуль.}

\medskip
Этот предложение легко следует из того факта, что всякий метрически инъективный
модуль топологически инъективен. Оба эти примера обобщает следующее предложение:

\medskip
\textbf{Предложение 7.} \textit{Пусть $K$ --- стоуново пространство и
$n\in\mathbb{N}$, тогда $M_n(C(K))$ топологически инъективен как правый
$M_n(C(K))$-модуль.}

\medskip
Доказательство состоит из трех шагов. На первом шаге для каждой точки $s\in K$
рассматривается правый $M_n(C(K))$-модуль $M_n(\mathbb{C}_s)$ c внешним
умножением определенным по формуле ${(x\cdot a)}_{i,j}=\sum_{k=1}^n
x_{i,k}a_{k,j}(s)$ для всех $x\in M_n(\mathbb{C}_s)$, $a\in M_n(C(K))$. Из
аменабельности $M_n(C(K))$ легко вывести, что $M_n(C(K))$-модуль
$M_n(\mathbb{C}_s)$ топологически инъективен. На втором шаге доказывается, что
произведение $\bigoplus{}_\infty \{M_n(\mathbb{C}_s):s\in K \}$ топологически
инъективно как $M_n(C(K))$-модуль. На третьем шаге остается показать, что
$M_n(C(K))$ является дополняемым подмодулем в $\bigoplus{}_\infty
\{M_n(\mathbb{C}_s):s\in K \}$.

Отметим, что все упомянутые примеры принадлежат к более узкому классу
$C^*$-алгебр, а именно к классу $AW^*$-алгебр. И если ограничиться рассмотрением
только $AW^*$-алгебр, то можно доказать следующий критерий.

\medskip
\textbf{Теорема 8.} \textit{Пусть $A$ --- $C^*$-алгебра. Тогда следующие условия
эквивалентны:
\newline
$(i)$ $A$ --- $AW^*$-алгебра, топологически инъективная как правый $A$-модуль;
\newline
$(ii)$ $A$ изоморфна как $C^*$-алгебра алгебре $\bigoplus_\infty
\{M_{n_i}(C(K_i)):i=1,\ldots,n \}$ для некоторого конечного набора стоуновых
пространств ${(K_i)}_{i=1,\ldots,n}$ и натуральных чисел
${(n_i)}_{i=1,\ldots,n}$.}

\medskip
Идея доказательства основывается на предложениях 4, 7 и дихотомии Смита-Уильямса
~\cite{SmithDecompPropCStarAlg}. Они показали, что $AW^*$-алгебра либо изоморфна
как $C^*$-алгебра алгебре $\bigoplus_\infty \{M_{n_i}(C(K_i)):i=1,\ldots,n \}$
для некоторого конечного набора стоуновых пространств ${(K_i)}_{i=1,\ldots,n}$ и
натуральных чисел ${(n_i)}_{i=1,\ldots,n}$, либо содержит $M_\infty$ как
${}^*$-подалгебру. 

Для полного описания топологически инъективных $C^*$-алгебр теперь хотелось бы
показать, что все они являются $AW^*$-алгебрами. Но похоже, что это --- сложная
задача даже в коммутативном случае. Пока ни в одной стандартной категории
функционального анализа, начиная с категории банаховых пространств, не получено
полного описания топологически инъективных объектов.

\begin{thebibliography}{999}
\bibitem{NachThOfHahnBanachType}
\textit{L. Nachbin.} A theorem of the Hahn-Banach type for linear
transformations, Transactions of the American Mathematical Society, (1950),
28--46
\bibitem{GooProjInNorLinSp}
\textit{D. Goodner.} Projections in normed linear spaces, Transactions of the
American Mathematical Society, (1950), 89--108
\bibitem{KellBanSpWithExtProp}
\textit{J. L. Kelley.} Banach spaces with the extension property, Transactions
of the American Mathematical Society, (1952), 323--326
\bibitem{HasumiExtPropComplBanSp}
\textit{M. Hasumi.} The extension property of complex Banach spaces, Tohoku
Mathematical Journal, Second Series, 10:2 (1958), 135--142
\bibitem{RuanOpSp} 
\textit{E. G. Effros, Z.-J. Ruan.} Operator spaces, Oxford University Press,
(2000)
\bibitem{HelemSheinbergFlatBanModAndAmenBanAlg}
\textit{А. Я. Хелемский.} Плоские банаховы модули и аменабельные алгебры, Труды
Московского математического общества, 47:0 (1984), 179--218
\bibitem{HelMetrFrQMod}
\textit{А. Я. Хелемский.} Метрическая свобода и проективность для классических и
квантовых нормированных модулей, Матем. сб., 204:7 (2013), 127–-158 
\bibitem{HamInjEnvBanMod} 
\textit{M. Hamana.} Injective envelopes of Banach modules, T{\^o}hoku
Mathematical Journal, 30:3 (1978), 439--453
\bibitem{TakHanBanThAndJordDecomOfModMap} 
\textit{M. Takesaki.} On the Hahn-Banach type theorem and the Jordan
decomposition of module linear mapping over some operator algebras, Kodai
Mathematical Seminar Reports, 12:1 (1960), 1--10
\bibitem{DiestAbsSumOps} 
\textit{J. Diestel, H. Jarchow, A. Tonge.} Absolutely summing operators,
Cambridge University Press, 43 (1995)
\bibitem{LauLoyWillisAmnblOfBanAndCStarAlgsOfLCG}
\textit{A.T.-M. Lau, R. J. Loy, G. A. Willis.} Amenability of Banach and
$\it{C^*}$-algebras on locally compact groups, Studia Mathematica, 119:2,
(1996), 161--178
\bibitem{GorLewAbsSmOpAndLocUncondStrct} 
\textit{Y. Gordon, D. R. Lewis.} Absolutely summing operators and local
unconditional structures, Acta Mathematica, 133:1 (1974), 27--48
\bibitem{BlackadarOpAlg} 
\textit{B. Blackadar.} Operator algebras, Springer, 122 (2006)
\bibitem{SmithDecompPropCStarAlg} 
\textit{R. R. Smith, D. P. Williams.} The decomposition property for
$\it{C^*}$-algebras, J. Operator Theory, 16 (1986), 51--74
\end{thebibliography}
\end{document}