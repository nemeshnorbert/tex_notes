% chktex-file 35
 \documentclass[12pt]{article}
 \usepackage[left=2cm,right=2cm,top=2cm,bottom=2cm,bindingoffset=0cm]{geometry}
 \usepackage{amssymb}
 \usepackage{amsmath}
 \usepackage{amsthm}
\usepackage{mathrsfs}
 \usepackage{enumerate}
 \usepackage[T1,T2A]{fontenc}
 \usepackage[utf8]{inputenc}
 \usepackage[matrix,arrow,curve]{xy}
 \usepackage[colorlinks=true, urlcolor=blue, linkcolor=blue, citecolor=blue,
     pdfborder={0 0 0}]{hyperref}
\usepackage{enumitem}

 %------------------------------------------------------------------------------
 \newtheorem{theorem}{Theorem}[section]
 \newtheorem{lemma}[theorem]{Lemma}
 \newtheorem{proposition}[theorem]{Proposition}
 \newtheorem{remark}[theorem]{Remark}
 \newtheorem{corollary}[theorem]{Corollary}
 \newtheorem{definition}[theorem]{Definition}
 \newtheorem{example}[theorem]{Example}

 \newcommand{\projtens}{\mathbin{\widehat{\otimes}}}
 \newcommand{\convol}{\ast}
 \newcommand{\projmodtens}[1]{\mathbin{\widehat{\otimes}}_{#1}}
 \newcommand{\isom}[1]{\mathop{\mathbin{\cong}}\limits_{#1}}
 %------------------------------------------------------------------------------

\begin{document}

\begin{center}
    \Large \textbf{Lack of metric projectivity, injectivity, 
    and flatness for modules $L_p$}\\[0.5cm]
    \small {N. T. Nemesh}\\[0.5cm]
\end{center}

\thispagestyle{empty}

\medskip
\textbf{Abstract:} In this paper we show that for a locally compact Hausdorff 
space $S$ and a decomposable Borel measure $\mu$ metric projectivity, 
injectivity or flatness of $C_0(S)$-module $L_p(S,\mu)$ implies 
that $\mu$ is purely atomic with at most one atom.
\medskip

\textbf{Keywords:} metric projectivity, metric injectivity, metric flatness, 
$L_p$-space.

\bigskip

%-------------------------------------------------------------------------------
%	Introduction
%-------------------------------------------------------------------------------

\section{Introduction}\label{SctnIntro}

This paper finalizes author's research on homological properties of modules $L_p$.
In \cite{NemRelProjModLp} it was shown that modules $L_p$ are relatively 
projective for a small class of underlying measure spaces, namely purely atomic 
measure spaces, with atoms being isolated points. In this paper we solve the same
problem for metric projectivity, injectivity and flatness. It was expected that
for metric theory the class of measure spaces has to be even narrower. 
As we show in this paper it is extremely poor. It includes only purely atomic 
measure spaces with at most one atom. It is safe to say that $L_p$-spaces 
are almost never metrically projective, injective or flat. 

Before we proceed to the main topic we shall give a few definitions. For any 
natural number $n\in\mathbb{N}$ by $\mathbb{N}_n$ we denote the set of 
first $n$ natural numbers. Let $M$ be a subset of a set $N$, then $\chi_M$ 
denotes the indicator function of $M$. The symbol $1_N$ denotes the identity 
map on $N$. If $n',n''\in N$, then $\delta_{n'}^{n''}$ stands for their 
Kronecker symbol.

All Banach spaces in this paper are considered over the complex numbers. 
We shall actively use the following Banach space constructions.
For given family of Banach spaces $\{E_\lambda: \lambda\in\Lambda\}$ 
by $\bigoplus_p\{E_\lambda: \lambda\in\Lambda\}$ we denote their $\ell_p$-sum 
(see [\cite{HelLectAndExOnFuncAn}, proposition 1.1.7]).
Similarly, for a family of linear operators $T_\lambda:E_\lambda\to F_\lambda$
where $\lambda\in\Lambda$ we denote their $\ell_p$-sum as
$\bigoplus_p\{T_\lambda:\lambda\in\Lambda\}$ 
(see [\cite{HelLectAndExOnFuncAn}, proposition 1.1.7]). We use 
symbol $E\projtens F$ do denote the projective tensor product of Banach 
spaces $E$ and $F$ [\cite{HelLectAndExOnFuncAn}, theorem 2.7.4].

Let $T:E\to F$ be a bounded linear operator between Banach spaces. Now we give 
quantitative version of the definition of embedding. If for some 
constant $c>0$ and all $x\in E$ holds $c\Vert T(x)\Vert\geq \Vert x\Vert$, 
then $T$ is called $c$-topologically injective. Similarly, a quantitative 
definition of the open map is as follows: if there exists a 
constant $c>0$ such that for any $y\in F$ we can find an $x\in E$ such 
that $T(x)=y$ and $c\Vert y\Vert\geq \Vert x\Vert$, then $T$ is 
called $c$-topologically surjective. Finally, the operator $T$ is 
called contractive if its norm is not greater than $1$.

Let $A$ be a Banach algebra. We shall consider both left and right 
Banach $A$-modules with contractive outer action $\cdot:A\times X\to X$. 
Let $X$ and $Y$ be two Banach $A$-modules, then a map $\phi:X\to Y$ is called 
an $A$-morphism if it is a continuous $A$-module map. All left 
Banach $A$-modules and their $A$-morphisms form a category which we denote 
by $A-\mathbf{mod}$. Similarly, one can define the category of 
right $A$-modules $\mathbf{mod}-A$. By $X\otimes_A Y$ we denote the projective 
module tensor product of a left $A$-module $X$ and a 
right $A$-module $Y$ (see [\cite{HelBanLocConvAlg}, definition VI.3.18]) 

Now we can give definitions of metric projectivity, injectivity and flatness.
The first paper on this subject was written in 1978 by 
Graven \cite{GravInjProjBanMod}. These notions were later rediscovered by 
White \cite{WhiteInjmoduAlg} and 
Helemeskii \cite{HelMetrFrQMod,HelMetrFlatNorMod}.

A left Banach $A$-module $P$ is called metrically projective if for 
any $c$-topologically surjective $A$-morphism $\xi:X\to Y$ and any
$A$-morphism $\phi:P\to Y$ there exists an $A$-morphism $\psi:P\to X$ such that
$\Vert\psi\Vert\leq c$ and the diagram
$$
    \xymatrix{
    & {X} \ar[d]^{\xi}\\  % chktex 3
    {P} \ar@{-->}[ur]^{\psi} \ar[r]^{\phi} &{Y}}  % chktex 3
$$
is commutative. The original definition, was somewhat 
different [\cite{GravInjProjBanMod}, definition 2.4], but it is still equivalent
to the one above. The simplest example of a metrically projective $A$-module is 
the algebra $A$ itself, provided it is 
unital [\cite{GravInjProjBanMod}, theorem 2.5]. 

A right Banach $A$-module $J$ is called metrically injective if for 
any $c$-topologically injective $A$-morphism $\xi:Y\to X$ and any
$A$-morphism $\phi:Y\to J$ there exists an $A$-morphism $\psi:X\to J$ such that
$\Vert\psi\Vert\leq c$ and the diagram
\[
    \xymatrix{
    & {X} \ar@{-->}[dl]_{\psi} \\  % chktex 3
    {J} &{Y} \ar[l]_{\phi} \ar[u]_{\xi}}  % chktex 3
\]
is commutative. Our definition is equivalent to the original 
one [\cite{GravInjProjBanMod}, definition 3.1]. Note that $P^*$ is a metrically 
injective $A$-module, whenever $P$ is metrically 
injective [\cite{GravInjProjBanMod}, theorem 3.2]. Therefore, the right Banach 
module $A^*$ is metrically injective, whenever $A$ is unital. 

A left $A$-module $F$ is called metrically flat if for each $c$-topologically 
injective $A$-morphism $\xi:X\to Y$ of right $A$-modules the 
operator $\xi\projmodtens{A} 1_F:X\projmodtens{A} F\to Y\projmodtens{A} F$ 
is $c$-topologically injective. This definition was implicitly given in the 
theorem [\cite{GravInjProjBanMod}, theorem 3.10]. From this theorem and remarks 
above we conclude that any metrically projective module is metrically flat. In
particular, a unital Banach algebra $A$ is a metrically flat $A$-module.

%-------------------------------------------------------------------------------
%	Metric injectivity of linftyn modules lpn
%-------------------------------------------------------------------------------

\section{Metric injectivity of finite-dimensional 
\texorpdfstring{$\ell_\infty(\Lambda)$}{linftyLmbd}-module 
\texorpdfstring{$\ell_p(\Lambda)$}{lpLmbd}}
\label{MetrInjlinftynlpn}

For an index set $\Lambda$ and $1\leq p\leq +\infty$ by $\ell_p(\Lambda)$ we 
denote the standard $\ell_p$-space. We denote its norm 
by $\Vert\cdot\Vert_p$ and its natural basis 
by $(e_\lambda)_{\lambda\in\Lambda}$. For $1\leq p<+\infty$ we shall often 
exploit the standard identification $\ell_p(\Lambda)^*=\ell_{p^*}(\Lambda)$, 
where $p^*=p/(p-1)$. By convention, we set $1^*=+\infty$. The 
space $\ell_p(\Lambda)$ can be regarded both as left and right Banach module 
over the Banach algebra $\ell_\infty(\Lambda)$. In this section 
we shall show that for finite $\Lambda$ the 
right $\ell_\infty(\Lambda)$-module $\ell_p(\Lambda)$ is metrically injective 
only if $\Lambda$ has at most $1$ element.

\begin{definition}\label{StdEmbd} 
    Let $\Lambda$ be a set, $1<p<+\infty$ 
    and $\mathcal{F}\subset\ell_{p^*}(\Lambda)$, then we define a linear 
    operator
    \[
        \xi_{\mathcal{F}}: 
        \ell_p(\Lambda)\to\bigoplus_\infty\{\ell_1(\Lambda):f\in\mathcal{F}\},\,
        x \mapsto \bigoplus_\infty\{ x\cdot f: f\in\mathcal{F}\}.
    \]
\end{definition}

\begin{definition}\label{StdEmbdCoercv}
    Let $\Lambda$ be a set, $1<p<+\infty$ 
    and $\mathcal{F}\subset \ell_{p^*}(\Lambda)$, 
    then we define the coercivity constant for the operator $\xi_{\mathcal{F}}$ 
    as
    \[
        \gamma_{\mathcal{F}}=\sup\{
            \Vert x\Vert_p: 
            x\in\ell_p^n,\,\, \Vert \xi_{\mathcal{F}}(x)\Vert\leq 1
        \}.
    \]
\end{definition}

Note that $\xi_{\mathcal{F}}$ is $\gamma_{\mathcal{F}}$-topologically injective
whenever $\gamma_{\mathcal{F}}$ is finite.

\begin{proposition}\label{LinfnMorphlpntolqnCharac}
    Let $\Lambda$ be a set, $1\leq p,q\leq +\infty$ 
    and $\phi:\ell_p(\Lambda)\to \ell_{q}(\Lambda)$ is 
    an $\ell_\infty(\Lambda)$-morphism of right modules. Then there exists a 
    vector $\eta\in\ell_\infty(\Lambda)$ such 
    that $\phi(x)=\eta\cdot x$ for all $x\in \ell_p(\Lambda)$.
\end{proposition}
\begin{proof}
    Denote $\eta_\lambda=\phi(e_\lambda)_\lambda$ for $\lambda\in\Lambda$, then 
    for any $x\in\ell_p(\Lambda)$ and $\lambda\in\Lambda$ we have
    \[
        \phi(x)_\lambda
        =(\phi(x)\cdot e_\lambda)_\lambda
        =\phi(x\cdot e_\lambda)_\lambda
        =\phi(x_\lambda e_\lambda)_\lambda
        =x_\lambda\phi(e_\lambda)_\lambda
        =x_\lambda\eta_\lambda
        =(\eta\cdot x)_\lambda.
    \]
    Therefore, $\phi(x)=\eta\cdot x$. By construction 
    $\Vert\eta\Vert_\infty\leq\Vert\phi\Vert$, so $\eta\in\ell_\infty(\Lambda)$.
\end{proof}

\begin{proposition}\label{ExtMorphSuml1ntlpnCharac}
    Let $\Lambda$ be a set, $1<p<+\infty$ 
    and $\mathcal{F}\subset \ell_{p^*}(\Lambda)$ be a finite set. Then for any 
    morphism of right $\ell_\infty(\Lambda)$-modules
    $\psi:\bigoplus_\infty\{\ell_1(\Lambda):f\in\mathcal{F}\}\to\ell_p(\Lambda)$ 
    there exists a family of vectors $\eta\in\ell_\infty(\Lambda)^\mathcal{F}$ 
    such that
    \[
        \psi(t)=\sum_{f\in\mathcal{F}} \eta_f \cdot t_f
    \]
    for all $t\in \bigoplus_\infty\{ \ell_1(\Lambda):f\in\mathcal{F}\}$.
\end{proposition}
\begin{proof}
    For each $f\in\mathcal{F}$  we define a natural embedding
    \[
        \operatorname{in}_f:
        \ell_1(\Lambda)\to\bigoplus_\infty\{\ell_1(\Lambda):f\in\mathcal{F}\}
    \]
    which is a morphism of right $\ell_\infty(\Lambda)$-modules. Then we define 
    an $\ell_\infty(\Lambda)$-morphism $\psi_f=\psi\circ \operatorname{in}_f$. 
    By proposition \ref{LinfnMorphlpntolqnCharac} there exists 
    a vector $\eta_f\in\ell_\infty(\Lambda)$ such that $\psi_f(x)=\eta_f\cdot x$ 
    for all $x\in\ell_1(\Lambda)$. Since $\mathcal{F}$ is finite, then for 
    all $t\in \bigoplus_\infty\{ \ell_1(\Lambda) : f\in \mathcal{F}\}$ we have
    \[
        \psi(t)
        =\psi\left(\bigoplus_\infty\{ t_f : f\in\mathcal{F}\}\right)
        =\psi\left(\sum_{f\in\mathcal{F}} \operatorname{in}_f(t_f)\right)
        =\sum_{f\in\mathcal{F}}\psi_f(t_f)
        =\sum_{f\in\mathcal{F}} \eta_f\cdot t_f.
    \]
\end{proof}

\begin{definition}\label{ParamExtMorph}
    Let $\Lambda$ be a set, $1<p<+\infty$ 
    and $\mathcal{F}\subset \ell_{p^*}(\Lambda)$ be
    a finite set. For a given family $\eta\in \ell_\infty(\Lambda)^\mathcal{F}$
    we define
    \[
        \psi_{\eta}:
        \bigoplus_\infty\{\ell_1(\Lambda):f\in\mathcal{F}\}\to\ell_p(\Lambda),\,
        t\mapsto\sum_{f\in\mathcal{F}} \eta_f\cdot t_f.
    \]
\end{definition}

\begin{definition}\label{ExtMorphs}
    Let $\Lambda$ be a set, $1<p<+\infty$ 
    and $\mathcal{F}\subset\ell_{p^*}(\Lambda)$ be a finite set, then we define
    \[
        \mathcal{N}_{\mathcal{F}}=\left\{
            \eta\in \ell_\infty(\Lambda)^{\mathcal{F}} : 
            \sum_{f\in\mathcal{F}} \eta_{f,\lambda}f_\lambda=1,\,\,
            \lambda\in\Lambda
        \right\}.
    \]
\end{definition}

\begin{proposition}\label{StdEmbdLeftInvCharac}
    Let $\Lambda$ be a set, $1<p<+\infty$ 
    and $\mathcal{F}\subset\ell_{p^*}(\Lambda)$ be a finite set. 
    Then $\psi_\eta$ is a left inverse $\ell_\infty(\Lambda)$-morphism 
    to $\xi_{\mathcal{F}}$ iff $\eta\in\mathcal{N}_{\mathcal{F}}$.
\end{proposition}
\begin{proof} 
    Suppose $\psi_{\eta}$ is a left inverse of $\xi_{\mathcal{F}}$, then
    for any $\lambda\in\Lambda$ we have
    \[
        1=e_\lambda
        =\psi_{\eta}(\xi_{\mathcal{F}}(e_\lambda))
        =\psi_{\eta}\left(\bigoplus_\infty\{
            e_\lambda\cdot f: f\in\mathcal{F}
        \}\right)
        =\sum_{f\in\mathcal{F}} \eta_{f}\cdot e_\lambda\cdot f.
    \]
    At $\lambda$-th coordinate  we get
    \[
        1=(e_\lambda)_\lambda
        =\left(
            \sum_{f\in\mathcal{F}} \eta_f\cdot e_\lambda\cdot f
        \right)_\lambda
        =\sum_{f\in\mathcal{F}} \eta_{f,\lambda}f_\lambda.
    \]
    Hence, $\eta\in\mathcal{N}_{\mathcal{F}}$.
    Conversely, let $\eta\in\mathcal{N}_{\mathcal{F}}$, then for 
    any $x\in\ell_p(\Lambda)$ holds
    \[
    \begin{aligned}
        \psi_\eta(\xi_{\mathcal{F}}(x))
        &=\psi_{\eta}\left(\bigoplus_\infty\{x\cdot f:f\in\mathcal{F}\}\right) 
        =\sum_{f\in\mathcal{F}}\eta_f\cdot x\cdot f 
        =\sum_{f\in\mathcal{F}}\sum_{\lambda\in\Lambda} 
            (\eta_f\cdot x\cdot f)_\lambda e_\lambda 
        =\sum_{f\in\mathcal{F}}\sum_{\lambda\in\Lambda} 
            \eta_{f,\lambda} x_\lambda f_\lambda e_\lambda \\
        &=\sum_{\lambda\in\Lambda} 
            \left(\sum_{f\in\mathcal{F}}\eta_{f,\lambda}f_\lambda\right) 
            x_\lambda e_\lambda 
        =\sum_{\lambda\in\Lambda} x_\lambda e_\lambda 
        =x. 
    \end{aligned}
    \]
    Therefore, $\psi_\eta$ is a left inverse for $\xi_{\mathcal{F}}$.
\end{proof}

\begin{proposition}\label{ExtMorphNorm}
    Let $\Lambda$ be a finite set, $1<p<+\infty$ 
    and $\mathcal{F}\subset\ell_{p^*}(\Lambda)$ be a finite set. 
    Suppose $\eta\in\ell_\infty(\Lambda)^\mathcal{F}$, then 
    \[
        \Vert \psi_{\eta}\Vert
        =\max\left\{
            \left(\sum_{\lambda\in\Lambda}
                \left|
                    \sum_{f\in\mathcal{F}} |\eta_{f,\lambda}| 
                    \delta_{\lambda}^{d(f)}
                \right|^p
            \right)^{1/p} : 
            d\in\Lambda^\mathcal{F}
        \right\}.
    \]
\end{proposition}
\begin{proof}
    By definition of operator norm
    \[
    \begin{aligned}
        \Vert\psi_{\eta}\Vert
        &=\sup\left\{
            \Vert\psi_{\eta}(t)\Vert_p:
            t\in \bigoplus_\infty\{\ell_1(\Lambda):f\in\mathcal{F}\},\,\,
            \Vert t\Vert\leq 1
        \right\} \\
        &=\sup\left\{
            \left \Vert\sum_{f\in\mathcal{F}}\eta_f\cdot t_f\right \Vert_p:
            t_f\in\ell_1(\Lambda),\, f\in\mathcal{F},\,\,
            \max\{\Vert t_f\Vert:f\in\mathcal{F}\}\leq 1
        \right\} \\
        &=\sup\left\{
            \left(\sum_{\lambda\in\Lambda}
                \left|
                    \sum_{f\in\mathcal{F}}\eta_{f,\lambda} t_{f,\lambda}
                \right|^p
            \right)^{1/p}:
            \sum_{\lambda\in\Lambda} |t_{f,\lambda}|\leq 1,\,\, 
            t_{f,\lambda}\in\mathbb{C},\,\, f\in\mathcal{F}, \lambda\in\Lambda
        \right\}. \\
    \end{aligned}
    \]
    For each $\lambda\in\Lambda$ and $f\in\mathcal{F}$ we 
    denote $r_{f,\lambda}=|t_{f,\lambda}|$ 
    and $\alpha_{f,\lambda}=\operatorname{arg}(t_{f,\lambda})$. 
    Then $t_{f,\lambda}=r_{f,\lambda} e^{i \alpha_{f,\lambda}}$. So
    \[
    \begin{aligned}
        \Vert \psi_{\eta}\Vert
        &=\sup\left\{
            \left(\sum_{\lambda\in\Lambda}
                \left|
                    \sum_{f\in\mathcal{F}}
                        \eta_{f,\lambda} r_{f,\lambda} e^{i \alpha_{f,\lambda}}
                \right|^p
            \right)^{1/p}:
            \sum_{\lambda\in\Lambda} r_{f,\lambda}\leq 1,\,\, 
            r_{f,\lambda}\in\mathbb{R}_+,\,\, 
            \alpha_{f,\lambda}\in\mathbb{R},\,\, 
            f\in\mathcal{F},\, \lambda\in\Lambda
        \right\}. \\
    \end{aligned}
    \]
    For any $\lambda\in\Lambda$ the maximum of the expression 
    \[
        \left|
            \sum_{f\in\mathcal{F}}
                \eta_{f,\lambda}\cdot r_{f,\lambda} e^{i \alpha_{f,\lambda}}
        \right|
    \]
    is attained 
    if $e^{i \alpha_{f,\lambda}}=\operatorname{sgn}(\eta_{f,\lambda})$ for 
    all $f\in\mathcal{F}$ and $\lambda\in\Lambda$. In this case
    \[
    \begin{aligned}
        \Vert \psi_{\eta}\Vert
        &=\sup\left\{
            \left(\sum_{\lambda\in\Lambda}
                \left|
                    \sum_{f\in\mathcal{F}}|\eta_{f,\lambda}| r_{f,\lambda}
                \right|^p
            \right)^{1/p}:
            \sum_{\lambda\in\Lambda} r_{f,\lambda}\leq 1,\,\, 
            r_{f,\lambda}\in\mathbb{R}_+,\,\, 
            f\in\mathcal{F},\, \lambda\in\Lambda
        \right\}. \\
    \end{aligned}
    \]
    Consider linear operators
    \[
        \tau_f:\mathbb{R}^\Lambda\to\ell_p(\Lambda): r\mapsto \eta_f\cdot r
    \]
    for $f\in\mathcal{F}$. Then 
    \[
    \begin{aligned}
        \Vert\psi_{\eta}\Vert
        &=\sup\left\{
            \left \Vert\sum_{f\in\mathcal{F}}^m\tau_f(r_f)\right \Vert_p:
            \sum_{\lambda\in\Lambda} r_{f,\lambda}\leq 1,\,\, 
            r_{f,\lambda}\in\mathbb{R}_+,\,\, 
            f\in\mathcal{F},\, \lambda\in\Lambda
        \right\}. \\
    \end{aligned}
    \]
    Since linear operators $(\tau_f)_{f\in\mathcal{F}}$ attain their values 
    in $\ell_p(\Lambda)$, whose norm is strictly convex, then the function
    \[
        F:(\mathbb{R}^\Lambda)^\mathcal{F}\to\mathbb{R}_+,\, 
        r\mapsto \left \Vert\sum_{f\in\mathcal{F}} \tau_f(r_f)\right \Vert_p
    \]
    is strictly convex. Since the set
    \[
        C=\left\{ 
            r\in(\mathbb{R}^\Lambda)^\mathcal{F} :
            \sum_{\lambda\in\Lambda} r_{f,\lambda}\leq 1,\,\, 
            r_f\in\mathbb{R}^\Lambda_+,\,\, 
            f\in\mathcal{F},\, \lambda\in\Lambda
        \right\}
    \]
    is a convex polytope in a finite-dimensional space, then $F$ attains its 
    maximum on $\operatorname{ext}(C)$ --- the set of extreme points of $C$. So
    \[
        \Vert\psi_\eta\Vert=\max\{F(r) : r\in \operatorname{ext}(C)\},
    \]
    Clearly, $r\in \operatorname{ext}(C)$ iff $r=0$ or for some 
    function $d:\mathcal{F}\to\Lambda$ and 
    all $\lambda\in\Lambda$, $f\in\mathcal{F}$ 
    holds $r_{f,\lambda}=\delta_{\lambda}^{d(f)}$. So,
    \[
    \begin{aligned}
        \Vert\psi_{\eta}\Vert
        &=\max\left\{
            \left(\sum_{\lambda\in\Lambda}
                \left|
                    \sum_{f\in\mathcal{F}}
                        |\eta_{f,\lambda}| \delta_{\lambda}^{d(f)}
                \right|^p
            \right)^{1/p}:
            d\in\Lambda^\mathcal{F}
        \right\}. \\
    \end{aligned}
    \]
\end{proof}

\begin{definition}\label{ExtMorphsNormInf}
    Let $\Lambda$ be a set, $1<p<+\infty$ 
    and $\mathcal{F}\subset\ell_{p^*}(\Lambda)$ is a finite set, then we define
    \[
        \nu_{\mathcal{F}}=\inf\{
            \Vert\psi_{\eta}\Vert : \eta\in\mathcal{N}_{\mathcal{F}}
        \}.
    \]
\end{definition}

\begin{definition}\label{SpclFuncFam}
    Let $\Lambda$ be a finite set and $\kappa\in\mathbb{R}$, then we set
    $f_\lambda=e_\lambda$ and $f_*=\kappa\sum_{\lambda\in\Lambda} e_\lambda$. 
    Now we define
    \[
        \mathcal{F}_{\kappa}(\Lambda)
        =\{f_\lambda: \lambda\in\Lambda\}
        \cup
        \{f_\star\}
        \subset \ell_{p^*}(\Lambda).
    \]
\end{definition}

\begin{proposition}\label{StdEmbdSpclCoerciv}
    Assume $\Lambda$ is a finite set with $n>1$ elements, $1<p<+\infty$ 
    and $n^{-1}<\kappa<(n-1)^{-1}$. Then
    \[
        \gamma_{\mathcal{F}_{\kappa}(\Lambda)}
        =(n-1+(\kappa^{-1}-(n-1))^p)^{1/p}.
    \]
\end{proposition}
\begin{proof}
    By definition of coercivity constant
    \[
    \begin{aligned}
        \gamma_{\mathcal{F}_{\kappa}(\Lambda)}
        &=\sup\{
            \Vert x\Vert_p : 
            x\in\ell_p(\Lambda),\, 
            \Vert \xi_{\mathcal{F}_{\kappa}(\Lambda)}(x)\Vert\leq 1
        \} \\
        &=\sup\left\{
            \left( \sum_{\lambda\in\Lambda} |x_\lambda|^p\right)^{1/p} : 
            x\in\mathbb{C}^\Lambda,\, 
            \max\left\{
                \max\{|x_\lambda|:\lambda\in\Lambda\},
                \kappa\sum_{\lambda\in\Lambda} |x_\lambda|
            \right\}\leq 1
        \right\} \\
        &=\sup\left\{
            \left( \sum_{\lambda\in\Lambda} |t_\lambda|^p\right)^{1/p} : 
            t\in\mathbb{R}^\Lambda,\, 
            \sum_{\lambda\in\Lambda} |t_\lambda|\leq \kappa^{-1},\,
            |t_\lambda|\leq 1,\,\lambda\in\Lambda,\,
        \right\}. \\
    \end{aligned}
    \]
    Consider function
    \[
        F:
        \mathbb{R}^\Lambda\to\mathbb{R},\, 
        t\mapsto \left(\sum_{\lambda\in\Lambda}|t_\lambda|^p\right)^{1/p}
    \]
    and a convex polytope
    \[
        C=\left\{ 
            t\in\mathbb{R}^\Lambda : 
            \sum_{\lambda\in\Lambda} |t_\lambda|\leq \kappa^{-1},\,
            |t_\lambda|\leq 1,\,\lambda\in\Lambda,\,
        \right\}
    \]
    in a finite-dimensional space. Since $F$ is strictly convex, then $F$ 
    attains its maximum on $\operatorname{ext}(C)$ --- the set of extreme 
    points of $C$. Therefore,
    \[
        \gamma_{\mathcal{F}_{\kappa}(\Lambda)}=\max\{
            F(t):t\in\operatorname{ext}(C)
        \},
    \]
    Clearly, any point $t\in \operatorname{ext}(C)$ has all coordinates but one
    equal to $1$ or $-1$. Therefore,
    \[
        \operatorname{ext}(C)=\left\{ 
            t\in\mathbb{R}^\Lambda : 
            \exists \lambda'\in\Lambda\quad |t_{\lambda'}|=\kappa^{-1}-(n-1)\,
            \wedge\, 
            \forall \lambda\in\Lambda\setminus\{\lambda'\}\quad |t_\lambda|=1
        \right\}.
    \]
    As a consequence
    \[
        \gamma_{\mathcal{F}_{\kappa}(\Lambda)}=(n-1+(\kappa^{-1}-(n-1))^p)^{1/p}.
    \]
\end{proof}

\begin{proposition}\label{ExtMorphsNormLwrBnd}
    Assume $\Lambda$ is a finite set with $n>1$ elements, $1<p<+\infty$ 
    and $n^{-1}<\kappa<(n-1)^{-1}$, then
    \[
        \nu_{\mathcal{F}_{\kappa}(\Lambda)}
        \geq 
        \kappa^{-1}\frac{
            \left(
                \left(\frac{n-1}{\kappa^{-1}-1}\right)^{\frac{p}{p-1}}+n-1
            \right)^{1/p}
        }{
            \left(\frac{n-1}{\kappa^{-1}-1}\right)^{\frac{1}{p-1}}-1+\kappa^{-1}
        }.
    \]
\end{proposition}
\begin{proof}
    By construction
    \[
        \mathcal{N}_{\mathcal{F}_{\kappa}(\Lambda)}=\{
            \eta\in\ell_\infty(\Lambda)^{\mathcal{F}_{\kappa}(\Lambda)}:
            \eta_{f_\lambda,\lambda}+\kappa \eta_{f_*, \lambda}=1,\, 
            \lambda\in\Lambda
        \}.
    \]
    For each $\lambda\in\Lambda$ consider function
    \[
        d_\lambda:\mathcal{F}_\kappa(\Lambda)\to\Lambda,\,
        f_i\mapsto
        \begin{cases}
            i\quad &\mbox{ if }i\neq \star,\\
            \lambda\quad &\mbox{ if }i=\star.
        \end{cases}
    \]
    Then from proposition \ref{ExtMorphNorm} for 
    any $\eta\in\ell_\infty(\Lambda)^{\mathcal{F}_{\kappa}(\Lambda)}$ we would 
    get
    \[
    \begin{aligned}
        \Vert\psi_{\eta}\Vert
        &\geq\max\left\{
            \left(\sum_{\lambda\in\Lambda}
                \left|
                    \sum_{f\in\mathcal{F}_{\kappa(\Lambda)}} 
                        |\eta_{f,\lambda}|\delta_{\lambda}^{d_{\lambda'}(f)}
                \right|^p
            \right)^{1/p}:
            \lambda'\in\Lambda
        \right\} \\
        &=\max\left\{
            \left(
                \sum_{\lambda\in\Lambda,\lambda\neq \lambda'} 
                    |\eta_{f_\lambda,\lambda}|^p+
                    (|\eta_{f_{\lambda'},\lambda'}|+|\eta_{f_\star,\lambda'}|)^p
            \right)^{1/p}:
            \lambda'\in\Lambda
        \right\}. \\
    \end{aligned}
    \]
    For any $\eta\in\mathcal{N}_{\mathcal{F}_{\kappa}(\Lambda)}$ 
    and $\lambda\in\Lambda$ we 
    have $\eta_{f_\lambda,\lambda}+\kappa \eta_{f_\star, \lambda}=1$. 
    Therefore,
    \[
    \begin{aligned}
        \Vert \psi_{\eta}\Vert
        &\geq\max\left\{
            \left(
                \sum_{\lambda\in\Lambda,\lambda\neq \lambda'}
                    |\eta_{f_\lambda,\lambda}|^p+
                    (
                        |\eta_{f_{\lambda'},\lambda'}|+
                        |\kappa^{-1}(1-\eta_{f_{\lambda'},\lambda'})|
                    )^p
            \right)^{1/p}:
            \lambda'\in\Lambda
        \right\} \\
        &\geq\max\left\{
            \left(
                \sum_{\lambda\in\Lambda,\lambda\neq \lambda'} 
                    |\eta_{f_\lambda,\lambda}|^p+
                    (
                        |\eta_{f_{\lambda'},\lambda'}|+
                        \kappa^{-1}|1-|\eta_{f_{\lambda'},\lambda'}||
                    )^p
            \right)^{1/p}:
            \lambda'\in\Lambda
        \right\} \\
    \end{aligned}
    \]
    for any $\eta\in\mathcal{N}_{\mathcal{F}_{\kappa}(\Lambda)}$. Denote
    \[
    \begin{aligned}
        \alpha_i&:\mathbb{R}^n\to\mathbb{R},\,
        t\mapsto \left(
            \sum_{k=1,k\neq i}^n |t_k|^p+(|t_i|+\kappa^{-1}|1-|t_i||)^p
        \right)^{1/p} \quad\mbox{ for }\quad i\in\mathbb{N}_n \\
        \alpha&:\mathbb{R}^n\to\mathbb{R},\,
        t\mapsto\max\{\alpha_i(t):i\in\mathbb{N}_n\}.
    \end{aligned}
    \]
    Then for any enumeration $\Lambda=\{\lambda_1,\ldots,\lambda_n\}$ we get
    \[
        \nu_{\mathcal{F}_{\kappa}(\Lambda)}
        \geq\inf\{
            \Vert \psi_{\eta}\Vert : 
            \eta\in\mathcal{N}_{\mathcal{F}_{\kappa}(\Lambda)}
        \}
        =\inf\{
            \alpha(
                |\eta_{f_{\lambda_1},\lambda_1}|,
                \ldots,
                |\eta_{f_{\lambda_n},\lambda_n}|
            ) : 
            \eta\in\mathcal{N}_{\mathcal{F}_{\kappa}(\Lambda)}
        \}
        =\inf\{\alpha(t) : t\in\mathbb{R}_+^n\}.
    \]
    Consider functions
    \[
    \begin{aligned}
        F&:\mathbb{R}_+\to\mathbb{R},\, 
            t\mapsto t \\
        G&:\mathbb{R}_+\to\mathbb{R},\, 
            t\mapsto t+\kappa^{-1}|1-t| \\
        H&:\mathbb{R}^n\to\mathbb{R},\, t\mapsto 
            \left(\sum_{k=1}^n|t_k|^p\right)^{1/p} \\
    \end{aligned}
    \]
    Clearly, for each $i\in\mathbb{N}_n$ the function $\alpha_i$ is a 
    composition of $F$, $G$ and $H$. Since $F$ and $G$ are convex 
    and $H$ is strictly convex, then all 
    functions $(\alpha_i)_{i\in\mathbb{N}_n}$ are strictly convex 
    on $\mathbb{R}_+^n$. Hence, so is their maximum $\alpha$. Note 
    that $\alpha$ is continuous, strictly convex 
    and $\lim_{\Vert t\Vert\to+\infty}\alpha(t)=+\infty$. 
    Therefore, $\alpha$ has a unique global minimum at some 
    point $t_0\in\mathbb{R}_+^n$. Observe, that $\alpha$ is invariant under 
    permutation of its arguments. Then from the uniqueness of the global minimum 
    at $t_0$ we can conclude that all coordinates of $t_0$ are equal. 
    As a corollary
    \[
    \begin{aligned}
        \nu_{\mathcal{F}_{\kappa}(\Lambda)}
        &\geq\inf\{\alpha(t) : t\in\mathbb{R}_+^n\} \\
        &=\inf\{\alpha(s,\ldots,s) : s\in\mathbb{R}_+\} \\
        &=\inf\{((n-1)s^p+(s+\kappa^{-1}|1-s|)^p)^{1/p} : s\in\mathbb{R}_+\}.
    \end{aligned}
    \]
    Consider function
    \[
        F:\mathbb{R}_+\to\mathbb{R},\,
        s\mapsto ((n-1)s^p+(s+\kappa^{-1}|1-s|)^p)^{1/p}
    \]
    We have 
    \[
        F(s)=
        \begin{cases}
            ((s+\kappa^{-1}(1-s))+(n-1)s^p)^{1/p}
            \quad&\mbox{ if }\quad 0\leq s\leq 1, \\
            \left(
                (\kappa^{-1}+1)^p
                \left(s-\frac{\kappa^{-1}}{\kappa^{-1}+1}\right)+
                (n-1)s^p
            \right)^{1/p}
            \quad&\mbox{ if }\quad s>1.
        \end{cases}
    \]
    Since $F$ is continuous and clearly increasing on $(1,+\infty)$, 
    then $F$ attains its minimum on $[0, 1]$. Let us find stationary points 
    of $F$ on $[0, 1]$. For $s\in[0,1]$ we have
    \[
        F'(s)=
        ((s+\kappa^{-1}(1-s))+(n-1)s^p)^{1/p-1}
        ((\kappa^{-1}+(1-\kappa^{-1})s)^{p-1}+(n-1)s^{p-1}).
    \]
    The stationary point can be found from the equation $F'(s)=0$. 
    The solution is
    \[
        s_0
        =\frac{
            \kappa^{-1}
        }{
            \left(
                \frac{n-1}{\kappa^{-1}-1}
            \right)^{\frac{1}{p-1}}
            -1+\kappa^{-1}
        }.\\
    \]
    By assumption $n^{-1}<\kappa$, so $\frac{n-1}{\kappa^{-1}-1}<1$ and 
    therefore $0<s_0<1$. Since $F$ is convex, then $s_0$ is the point of 
    minimum on $[0,1]$. The minimum equals
    \[
    \begin{aligned}
        F(s_0)
        &=\kappa^{-1}
            \frac{
                \left(
                    \left(\frac{n-1}{\kappa^{-1}-1}
                    \right)^{\frac{p}{p-1}}
                    -1+\kappa^{-1}
                \right)^{1/p}
            }{
                \left(\frac{n-1}{\kappa^{-1}-1}
                \right)^{\frac{1}{p-1}}
                -1+\kappa^{-1}
            }. \\
    \end{aligned}
    \]
    This gives the desired lower bound 
    for $\nu_{\mathcal{F}_{\kappa}(\Lambda)}$.
\end{proof}

\begin{proposition}\label{SpclLyapIneq}
    Let $n\in\mathbb{N}$, $r>1$ and $x\in\mathbb{C}^n$. Then
    \[
        \Vert x\Vert_p\leq \Vert x\Vert_1^{1/r}\Vert x\Vert_\infty^{1-1/r}.
    \]
    The equality holds iff $|x_1|=\ldots=|x_n|$.
\end{proposition}
\begin{proof}
    For any $r>1$ and any $x\in\mathbb{C}^n$ we have
    \[
    \begin{aligned}
        \Vert x\Vert
        =\left(\sum_{k=1}^n |x_k|^r \right)^{1/r}
        =\left(\sum_{k=1}^n |x_k| |x_k|^{r-1} \right)^{1/r}
        \leq\left(\sum_{k=1}^n |x_k| \right)^{1/r} 
        \left(\max\limits_{k\in\mathbb{N}_n}|x_k|^{r-1}\right)^{1/r}
        =\Vert x\Vert_1^{1/r}\Vert x\Vert_\infty^{1-1/r}.
    \end{aligned}
    \]
    Clearly, the equality holds iff $|x_k|=\max\limits_{k\in\mathbb{N}_n}|x_k|$
    for all $k\in\mathbb{N}_n$.
\end{proof}

\begin{proposition}\label{CompStdEmbdCoercvAndExtMorphsNormInf}
    Let $\Lambda$ be a finite set with $n>1$ elements, $1<p<+\infty$ 
    and $n^{-1}<\kappa<(n-1)^{-1}$. Then
    $
    \nu_{\mathcal{F}_{\kappa}(\Lambda)}
    >
    \gamma_{\mathcal{F}_{\kappa}(\Lambda)}.
    $
\end{proposition}
\begin{proof}
    Using results of propositions \ref{StdEmbdSpclCoerciv} 
    and \ref{ExtMorphsNormLwrBnd} it is enough to show that
    \[
        \kappa^{-1}
        \frac{
            \left(
                \left(\frac{n-1}{\kappa^{-1}-1}
                \right)^{\frac{p}{p-1}}
                -1+\kappa^{-1}
            \right)^{1/p}
        }{
            \left(\frac{n-1}{\kappa^{-1}-1}
            \right)^{\frac{1}{p-1}}
            -1+\kappa^{-1}
        }
        >
        (n-1+(\kappa^{-1}-(n-1))^p)^{1/p}.
    \]
    Let us make a substitution $m=n-1$ and $\rho=\kappa^{-1}$. 
    Then $m\in\mathbb{N}$ and $m<\rho<m+1$. Then the last inequality is 
    equivalent to
    \[
        \rho
        \frac{
            \left(
                \left(\frac{m}{\rho-1}
                \right)^{\frac{p}{p-1}}
                -1+\rho
            \right)^{1/p}
        }{
            \left(\frac{m}{\rho-1}
            \right)^{\frac{1}{p-1}}
            -1+\rho
        }
        >
        (m+(\rho-m)^p)^{1/p}.
    \] 
    After simplifications, we arrive at the inequality
    \[
        \frac{m\rho}{\rho-1}
        >
        (m+(\rho-m)^p)^{1/p}
        \left(
            m+\left(\frac{m}{\rho-1}\right)^{p^*}
        \right)^{1/p^*}.
    \]
    To prove this inequality we apply proposition \ref{SpclLyapIneq} to the 
    vector $x=(1,\ldots,1,\rho-m)^T\in\mathbb{C}^{m+1}$ with $r=p$ and to the 
    vector $x=(1,\ldots,1,\frac{m}{\rho-1})\in\mathbb{C}^{m+1}$ with $r=p^*$. 
    Since $m<\rho<m+1$, then the components of these vectors are not all equal,
    so the inequalities are strict:
    \[
    \begin{aligned}
        &(m+(\rho-m)^p)^{1/p}
        <
        (m+(\rho-m))^{1/p} 1^{1-1/p}\\
        &\left(m+\left(\frac{m}{\rho-1}\right)^{p^*}\right)^{1/p^*}
        <
        \left(m+\frac{m}{\rho-1}\right)^{1/p^*}
        \left(\frac{m}{\rho-1}\right)^{1-1/p^*}.\\
    \end{aligned}
    \]
    By multiplying these inequalities we get the desired result.
\end{proof}

\begin{proposition}\label{RetrPrblmNoSln}
    Let $\Lambda$ be a finite set with $n>1$ elements, $1<p<+\infty$ 
    and $n^{-1}<\kappa<(n-1)^{-1}$. Then for any $\ell_\infty(\Lambda)$-morphism
    $\psi$ which is a left inverse to $\xi_{\mathcal{F}_{\kappa}(\Lambda)}$ 
    holds $\Vert \psi\Vert>\gamma_{\mathcal{F}_{\kappa}(\Lambda)}$.  
\end{proposition}
\begin{proof}
    By proposition \ref{ExtMorphSuml1ntlpnCharac} there exists a family of 
    vectors $\eta\in\ell_\infty(\Lambda)^{\mathcal{F}_{\kappa}(\Lambda)}$ such 
    that $\psi=\psi_{\eta}$. From the definition \ref{ExtMorphsNormInf} we 
    have $\Vert \psi_{\eta}\Vert\geq \nu_{\mathcal{F}_{\kappa}(\Lambda)}$. Now 
    from proposition \ref{CompStdEmbdCoercvAndExtMorphsNormInf} we get 
    $
    \Vert\psi\Vert\geq\nu_{\mathcal{F}_{\kappa}(\Lambda)}
    >
    \gamma_{\mathcal{F}_{\kappa}(\Lambda)}.
    $
\end{proof}

\begin{proposition}\label{linftnModlpnIsntMetInjCharac}
    Let $\Lambda$ be a finite set, then the right $\ell_\infty(\Lambda)$-module 
    $\ell_p(\Lambda)$ is metrically injective iff $Lambda$ has at most $1$ 
    element.
\end{proposition}
\begin{proof}
    Suppose that the right $\ell_\infty(\Lambda)$-module $\ell_p(\Lambda)$ is 
    metrically injective and $\Lambda$ has $n$ elements. Assume that $n>1$. 
    Pick any real number $\kappa\in(n^{-1},(n-1)^{-1})$. By 
    proposition \ref{StdEmbdSpclCoerciv} we 
    have $\gamma_{\mathcal{F}_{\kappa}(\Lambda)}$ is finite, 
    therefore $\xi_{\mathcal{F}_{\kappa}(\Lambda)}$ 
    is $\gamma_{\mathcal{F}_{\kappa}(\Lambda)}$-topologically injective. 
    From assumption, it follows that there exists 
    an $\ell_\infty(\Lambda)$-morphism $\psi$ which is a left inverse 
    to $\xi_{\mathcal{F}_{\kappa}(\Lambda)}$ such 
    that $\Vert\psi\Vert\leq \gamma_{\mathcal{F}_{\kappa}(\Lambda)}$. 
    This contradicts proposition \ref{linftnModlpnIsntMetInjCharac}, 
    therefore $n\leq 1$.

    Now assume that $\Lambda$ has at most 1 element. If $\Lambda$ is empty, then
    $\ell_p(\Lambda)=\{0\}$. Zero module is always injective. If $\Lambda$ has 
    one element, then $\ell_p(\Lambda)$ is isometrically isomorphic 
    to $\ell_\infty(\Lambda)^*$ as an $\ell_\infty(\Lambda)$-module. The dual of
    the unital algebra is always metrically injective.
\end{proof}

%-------------------------------------------------------------------------------
%   Measure theory preliminaries
%-------------------------------------------------------------------------------

\section{Measure theory preliminaries}
\label{MeasThPrelim}

In this section we set the stage for the main theorem. Even though it is stated
for Borel measures on locally compact spaces we shall prove propositions of this 
section for general measure spaces. A comprehensive study of general measure 
spaces can be found in~\cite{FremMeasTh2}. 

Let $\Omega$ be a set. By measure we mean a countably additive set function with
values in $[0,+\infty]$ defined on a $\sigma$-algebra $\Sigma$ of measurable
subsets of a set $\Omega$. A measurable set $E$ is called an atom if
$\mu(A)>0$ and for every measurable subset $B\subset A$ holds either $\mu(B)=0$
or $\mu(A\setminus B)=0$. A measure $\mu$ is called purely atomic if every
measurable set of positive measure has an atom. A measure $\mu$ is semi-finite
if for any measurable set $A$ of infinite measure there exists a measurable
subset of $A$ with finite positive measure. A family $\mathcal{D}$ of measurable
subsets of finite measure is called a decomposition of $\Omega$ if for any 
measurable set $E$ $\mu(E)=\sum_{D\in\mathcal{D}}\mu(E\cap D)$ and a set $F$ 
is measurable whenever $F\cap D$ is measurable for all $D\in\mathcal{D}$. 
Finally, a measure $\mu$ is called decomposable if it is semi-finite and admits 
a decomposition of $\Omega$. Most measures encountered in functional analysis 
are decomposable. A pair $(\Omega,\mu)$ is called a measure space.

We shall define a few Banach spaces constructed from measure spaces. 
Let $(\Omega,\mu)$ be a measure space. By $B(\Sigma)$ we denote the algebra of
bounded measurable functions with the  $\sup$ norm. For $1\leq p\leq +\infty$ 
by $L_p(\Omega,\mu)$ we denote the Banach space of equivalence
classes of $p$-integrable (or essentially bounded if $p=+\infty$) functions on
$\Omega$. Elements of $L_p(\Omega,\mu)$ are denoted by $[f]$.

\begin{definition}\label{GnrlzdMean}
    Let $(\Omega,\mu)$ be a measure space, $E$ be a measurable set of finite 
    positive measure and $f:\Omega\to\mathbb{C}$ be a measurable function. 
    For any real $r\neq 0$ we define a linear map
    \[
        m_{E,r}(f)=\mu(E)^{\frac{1}{r}-1}\int_E f(\omega)d\mu(\omega).
    \]
\end{definition}

\begin{proposition}\label{GnrlzdMeanProp}
    Let $E$ be a finite measure subset of a measure space $(\Omega,\mu)$ 
    and $r\neq 0$. Then for any measurable 
    functions $f:\Omega\to\mathbb{C}$, $g:\Omega\to\mathbb{C}$ holds
    \begin{enumerate}[label = (\roman*)]
        \item $m_{E,r}(\chi_E)=\mu(E)^{1/r}$;
        \item $m_{E,r}(f)=m_{E,r}(f\chi_E)$;
        \item If $E$ is an atom, then $f=m_{E,\infty}(f)$ almost 
        everywhere on $E$;
        \item If $E$ is an atom, 
        then $m_{E,\infty}(f)m_{E,\infty}(g)=m_{E,\infty}(f\cdot g)$.
        \item If $E$ is an atom and $s\neq 0$, 
        then $m_{E,r}(f)m_{E,s}(g)=m_{E,\frac{rs}{r+s}}(f\cdot g)$.
    \end{enumerate}
\end{proposition}
\begin{proof}
    Paragraphs $(i)$ and $(ii)$ are obvious.

    $(iii)$ Without loss of generality we assume, that $f$ is real-valued. 
    Denote $k=m_{E,\infty}(f)$. Clearly, $k$ is a mean value of $f$ on $E$. 
    Consider set $A_+=\{\omega\in E: f(\omega)>k\}$. Since $A_+$ is a 
    subset of an atom $E$ of finite measure, then either $\mu(A_+)=0$ 
    or $\mu(A_+)=\mu(E)>0$. In the latter case we get 
    \[
        \int_E f(\omega)d\mu(\omega)
        =\int_{A_+}f(\omega)d\mu(\omega)
        >k\mu(A_+)
        =k\mu(E)
        =\int_E f(\omega)d\mu(\omega).
    \]
    Contradiction, so $\mu(A_+)=0$. Similarly, one can show that the 
    set $A_-=\{\omega\in E:f(\omega)<k\}$ also has measure zero. 
    Thus, $f=k$ almost everywhere on $E$.

    Paragraph $(v)$ immediately follows from $(iv)$, which in turn an easy 
    consequence of $(iii)$.
\end{proof}

\begin{proposition}\label{LpOnPurAtomMeasSpRepr}
    Let $(\Omega,\mu)$ be a purely atomic measure space. Let $\mathcal{A}$ be a 
    decomposition of $\Omega$ into atoms of finite measure. Then for 
    any $1\leq p<+\infty$ the linear maps
    \[
    \begin{aligned}
        &I_p:
        L_p(\Omega,\mu)\to \ell_p(\mathcal{A}),\,
        [f]\mapsto\sum_{A\in\mathcal{A}} m_{A,p}(f) e_A, \\
        &J_p:
        \ell_p(\mathcal{A})\to L_p(\Omega,\mu),\,
        \omega\mapsto\sum_{A\in\mathcal{A}} \omega_A m_{A,-p}(\chi_A) [\chi_A]\\
    \end{aligned}
    \]
    are isometric isomorphisms which are inverse to each other.
\end{proposition}
\begin{proof} 
    Since $\mathcal{A}$ is a decomposition of $X$ into atoms, then
    \[
        [f]=\sum_{A\in\mathcal{A}} m_{A,\infty}(f)[\chi_A]
    \]
    for any $f\in L_p(\Omega,\mu)$. Using proposition \ref{GnrlzdMeanProp} for 
    each $f\in L_p(\Omega,\mu)$ and $A\in\mathcal{A}$ we get
    \[
        \int_A |f(\omega)|^pd\mu(\omega)
        =\int_A\left|m_{A,\infty}(f)\right|^pd\mu(\omega)
        =\mu(A)\left|m_{A,\infty}(f)\right|^p
        =\left|m_{A,p}(\chi_A) m_{A,\infty}(f)\right|^p
        =|m_{A,p}(f)|^p.
    \]
    So for any $f\in L_p(\Omega,\mu)$ we have
    \[
    \begin{aligned}
        \Vert I_p([f])\Vert
        =\left( \sum_{A\in\mathcal{A}} |m_{A,p}(f)|^p\right)^{1/p} 
        =\left( 
            \sum_{A\in\mathcal{A}} \int_A |f(\omega)|^pd\mu(\omega)
        \right)^{1/p} 
        =\left( \int_{\Omega} |f(\omega)|^pd\mu(\omega)\right)^{1/p} 
        =\Vert [f]\Vert.
    \end{aligned}
    \]
    Again, using proposition \ref{GnrlzdMeanProp} for 
    each $A\in\mathcal{A}$ we get
    \[
        |m_{A,-p}(\chi_A)|^p \int_{\Omega} |\chi_A(\omega)|^p d\mu(\omega)
        =(\mu(A)^{-1/p})^p\mu(A)
        =1.
    \]
    So for any $x\in\ell_p(\mathcal{A})$ we have
    \[
    \begin{aligned}
        \Vert J_p(x)\Vert
        =\left(
            \int_{\Omega}
                \sum_{A\in\mathcal{A}}
                |x_A|^p|m_{A,-p}(\chi_A)|^p|\chi_A(\omega)|^p 
            d\mu(\omega)
        \right)^{1/p}
        =\left(\sum_{A\in\mathcal{A}} |x_A|^p \right)^{1/p}
        =\Vert x\Vert.
    \end{aligned}
    \]
    Thus, $I_p$ and $J_p$ are isometric maps. Obviously these maps are linear. 
    Note that, any $x\in\ell_p(\mathcal{A})$ and $A\in\mathcal{A}$ holds
    \[
        m_{A,p}(J_p(x))
        =m_{A,p}(J_p(x)\chi_A)
        =m_{A,p}(x_A m_{A,-p}[\chi_A])
        =x_A m_{A,-p}(\chi_A)m_{A,p}(\chi_A)
        =x_A.
    \]
    Hence, for any $x\in\ell_p(\mathcal{A})$ we have
    \[
        I_p(J_p(x))
        =\sum_{A\in\mathcal{A}}m_{A,p}(J_p(x))e_A
        =\sum_{A\in\mathcal{A}}x_A e_A
        =x.
    \]
    Note that for any $A\in\mathcal{A}$ holds
    \[
        J_p(e_A)
        =\sum_{A'\in\mathcal{A}} (e_A)_{A'}m_{A',-p}(\chi_{A'})[\chi_{A'}]
        =\sum_{A'\in\mathcal{A}} \delta_{A}^{A'}m_{A',-p}(\chi_{A'})[\chi_{A'}]
        =m_{A,-p}(\chi_{A})[\chi_{A}].
    \]
    Therefore, for any $f\in L_p(\Omega,\mu)$ we have
    \[
        J_p(I_p(f))
        =\sum_{A\in\mathcal{A}}m_{A,p}(f)J_p(e_A)
        =\sum_{A\in\mathcal{A}}m_{A,p}(f)m_{A,-p}(\chi_A)[\chi_A]
        =\sum_{A\in\mathcal{A}}m_{A,\infty}(f)[\chi_A]
        =[f].
    \]
    Thus, $J_p$ and $I_p$ are inverse to each other.
\end{proof}

\begin{proposition}\label{SwtchMorphBtwnAtomMeasSp}
    Let $(\Omega,\mu)$ be a purely atomic measure space and $\mathcal{A}$ be 
    a decomposition of $\Omega$ into atoms of finite measure. 
    Suppose $1\leq p,q<+\infty$, then
    \begin{enumerate}[label = (\roman*)]
        \item If $\Phi:L_p(\Omega,\mu)\to L_q(\Omega,\mu)$ is 
        a $B(\Sigma)$-morphism the map $I_q\circ \Phi\circ J_p$ is 
        an $\ell_\infty(\mathcal{A})$-morphism of the same norm;

        \item If $\phi:\ell_p(\mathcal{A})\to \ell_q(\mathcal{A})$ is 
        an $\ell_\infty(\mathcal{A})$-morphism the
        map $J_q\circ \phi\circ I_p$ is a $B(\Sigma)$-morphism of the same norm;
    \end{enumerate} 
    
\end{proposition}
\begin{proof} 
    $(i)$ Denote $\phi=I_q\circ \Phi\circ J_p$, then by 
    proposition \ref{GnrlzdMeanProp} for any atom $A\in\mathcal{A}$ holds
    \[
        \phi(e_A)
        =I_q(\Phi(J_p(e_A)))
        =I_q(\Phi(m_{A,-p}(\chi_A)[\chi_A]))
        =m_{A,-p}(\chi_A)I_q(\Phi([\chi_A])).
    \]
    Since $A$ is an atom, then  we get
    \[
        \Phi([\chi_A])
        =\Phi([\chi_A]\cdot\chi_A)
        =\Phi([\chi_A])\cdot\chi_A
        =m_{A,\infty}(\Phi([\chi_A]))[\chi_A],
    \]
    so
    \[
    \begin{aligned}
        \phi(e_A)
        &=m_{A,-p}(\chi_A)I_q(m_{A,\infty}(\Phi([\chi_A]))[\chi_A]) 
        =m_{A,\infty}(\Phi([\chi_A]))m_{A,-p}(\chi_A)I_q([\chi_A]) \\
        &=m_{A,\infty}(\Phi([\chi_A]))m_{A,-p}(\chi_A)m_{A,p}(\chi_A)e_A 
        =m_{A,\infty}(\Phi([\chi_A]))e_A. \\
    \end{aligned}
    \]
    Now for any $x\in\ell_p(\mathcal{A})$ and $a\in\ell_\infty(A)$ we have
    \[
    \begin{aligned}
        \phi(x\cdot a)
        =\sum_{A\in\mathcal{A}} x_A a_A \phi(e_A) 
        =\sum_{A\in\mathcal{A}} x_A a_A m_{A,\infty}(\Phi([\chi_A]))e_A 
        =\sum_{A\in\mathcal{A}} (x_A \phi(e_A))\cdot a 
        =\phi(x)\cdot a. \\
    \end{aligned}
    \]
    Therefore, $\phi$ is an $\ell_\infty(\mathcal{A})$-morphism. By 
    proposition \ref{LpOnPurAtomMeasSpRepr} the maps $I_q$ and $J_p$ are 
    isometric isomorphisms, hence $\phi$ and $\Phi$ have the same norm.

    $(ii)$ Denote $\Phi=J_q\circ \phi\circ I_p$, then by 
    proposition \ref{GnrlzdMeanProp} for any atom $A\in\mathcal{A}$ holds
    \[
        \Phi([\chi_A])
        =J_q(\phi(I_p([\chi_A])))
        =J_q(\phi(m_{A,p}(\chi_A)e_A))
        =m_{A,p}(\chi_A)J_q(\phi(e_A)).
    \]
    Moreover,
    \[
        \phi(e_A)
        =\phi(e_A\cdot e_A)
        =\phi(e_A)\cdot e_A
        =\phi(e_A)_A e_A,
    \]
    so
    \[
    \begin{aligned}
        \Phi([\chi_A])
        &=m_{A,p}(\chi_A)J_q(\phi(e_A)_A e_A) 
        =\phi(e_A)_A m_{A,p}(\chi_A)J_q( e_A) \\
        &=\phi(e_A)_A m_{A,p}(\chi_A)m_{A,-p}(\chi_A) [\chi_A] 
        =\phi(e_A)_A [\chi_A]. \\
    \end{aligned}
    \]
    Now for any $f\in L_p(\Omega, \mu)$ and $a\in B(\Sigma)$ we have
    \[
    \begin{aligned}
        \Phi([f]\cdot a)
        &=\Phi\left(
            \sum_{A\in\mathcal{A}} m_{A,\infty}(f\cdot a)[\chi_A]
        \right) 
        =\sum_{A\in\mathcal{A}} m_{A,\infty}(f\cdot a) \Phi([\chi_A]) \\
        &=\sum_{A\in\mathcal{A}} 
            m_{A,\infty}(f) m_{A,\infty}(a) \phi(e_A)_A [\chi_A] 
        =\sum_{A\in\mathcal{A}} 
            (m_{A,\infty}(f) \phi(e_A)_A [\chi_A])\cdot a \\
        &=\sum_{A\in\mathcal{A}} 
            m_{A,\infty}(f) \Phi([\chi_A])\cdot a 
        =\Phi\left(\sum_{A\in\mathcal{A}} 
            m_{A,\infty}(f) [\chi_A]\right)\cdot a 
        =\Phi([f])\cdot a. \\
    \end{aligned}
    \]
    Therefore, $\Phi$ is a $B(\Sigma)$-morphism. By 
    proposition \ref{LpOnPurAtomMeasSpRepr} the maps $J_q$ and $I_p$ are 
    isometric isomorphisms, hence $\Phi$ and $\phi$ have the same norm.
\end{proof}

\begin{proposition}\label{LpFinDimCharac}
    Let $(\Omega,\mu)$ be a decomposable measure space 
    and $1\leq p\leq+\infty$. If $L_p(\Omega,\mu)$ is finite-dimensional, 
    then $(\Omega,\mu)$ is purely atomic with finitely many atoms of 
    finite measure.
\end{proposition} 
\begin{proof}
    Suppose $(\Omega,\mu)$ is not purely atomic, then there exists an atomless 
    measurable set $E\subset \Omega$. By~[\cite{FremMeasTh2}, proposition 215D] 
    we may assume that $E$ has a positive finite measure. 
    By~[\cite{FremMeasTh2}, exercise 215X(e)] there is a countable 
    family $(E_n)_{n\in\mathbb{N}}$ of disjoint sets of positive finite 
    measures. In this case $(\chi_{E_n})_{n\in\mathbb{N}}$ is a countable 
    linearly independent set in $L_p(\Omega,\mu)$, hence $L_p(\Omega,\mu)$ is 
    infinite-dimensional. Contradiction, so $(\Omega,\mu)$ is purely atomic. 
    Let $\mathcal{A}$ be a family of atoms whose union is $\Omega$. 
    Since $\Omega$ is decomposable, then all these atoms have a finite measure. 
    Therefore, $(\chi_{A})_{A\in\mathcal{A}}$ is a linearly independent set. 
    Since $L_p(\Omega,\mu)$ is finite-dimensional, then $\mathcal{A}$ is finite.
\end{proof}

%-------------------------------------------------------------------------------
%	Metric projectivity, injectivity and flatness of C_0(S) modules L_p(S,mu)
%-------------------------------------------------------------------------------

\section{Metric projectivity, injectivity and flatness of 
\texorpdfstring{$C_0(S)$}{C0(S)}-modules 
\texorpdfstring{$L_p(S,\mu)$}{LpSmu}}
\label{MetrProInjFltOfC0SModLp}

Let $S$ be a locally compact Hausdorff space. By $\operatorname{Bor}(S)$ we 
denote the $\sigma$-algebra generated by open subsets of $S$. In this section 
we shall consider only decomposable Borel measures. Let $\mu$ be such measure 
on $S$. We shall show that for $1<p<+\infty$ Banach $C_0(S)$-modules 
$L_p(S,\mu)$ are almost never metrically projective, injective or flat. 

\begin{proposition}\label{C0SMorphBtwnReflxSpIsBMorph}
    Let $S$ be a locally compact Hausdorff space and $\mu$ be a decomposable 
    Borel measure on $S$. Then any $C_0(S)$-morphism between 
    right reflexive modules is a $B(\operatorname{Bor}(S))$-morphism. 
\end{proposition}
\begin{proof} 
    Suppose $Z$ is a right $C_0(S)$-module, then $Z^{**}$ is a 
    right $C_0(S)^{**}$-module 
    [\cite{DalBanAlgAutCont}, proposition 2.6.15(iii)]. If $Z$ is reflexive, 
    then the natural embedding $\iota_Z:Z\to Z^{**}$ is an isometric 
    isomorphism. Recall that $B(\operatorname{Bor}(S))$ is a subalgebra 
    of $C_0(S)^{**}$ [\cite{DalBanAlgAutCont}, proposition 4.2.30], 
    therefore we can endow $Z$ with the structure 
    of $B(\operatorname{Bor}(S))$-module via 
    formula $z\cdot b=\iota_Z^{-1}(\iota_Z(z)\cdot b)$ for $z\in Z$ 
    and $b\in B(\operatorname{Bor}(S))$.
    
    Let $\phi:X\to Y$ be a morphism of right reflexive $C_0(S)$-modules. 
    Then $\phi^{**}$ is a $C_0(S)^{**}$-morphism
    [\cite{DalBanAlgAutCont}, proposition A.3.53]. As we have noted above, $X$
    and $Y$ are $B(\operatorname{Bor}(S))$-modules 
    and $\iota_X$, $\iota_Y$ --- are isometric isomorphisms. 
    Since $\phi=\iota_Y^{-1}\circ \phi^{**}\circ \iota_X$, then $\phi$ is 
    a $B(\operatorname{Bor}(S))$-morphism.
\end{proof}

\begin{proposition}\label{MetInjC0SModLpSmuOnFinAtmMeasSpCharac}
    Let $S$ be a locally compact Hausdorff space and $\mu$ be a purely atomic 
    Borel measure on $S$ with finitely many atoms of finite measure. 
    Suppose $1<p<+\infty$ and the $C_0(S)$-module $L_p(S,\mu)$ is metrically 
    injective, then $\mu$ has at most $1$ atom.
\end{proposition}
\begin{proof}
    Let $\mathcal{A}$ be a decomposition of $S$ into $n$ atoms of finite 
    measure. Suppose $n>1$, then pick any $\kappa\in(n^{-1}, (n-1)^{-1})$. 
    Now we set $\mathcal{F}=\mathcal{F}_{\kappa}(\mathcal{A})$. For 
    each $f\in \mathcal{F}$ we define an $\ell_\infty(\mathcal{A})$-morphism 
    \[
        m_f:
        \ell_p(\mathcal{A})\to\ell_1(\mathcal{A}),\,
        x\mapsto x\cdot f.
    \]
    We shall also use a natural embedding and a natural projection
    \[
        \operatorname{in}_f:
        L_1(S,\mu)\to\bigoplus_\infty\{L_1(S,\mu):f'\in\mathcal{F}\},
        \qquad
        \operatorname{pr}_f:
        \bigoplus_\infty\{
            \ell_1(\mathcal{A}):f'\in\mathcal{F}
        \}\to\ell_1(\mathcal{A}),
    \]
    which are $B(\operatorname{Bor}(S))$-morphism 
    and $\ell_\infty(A)$-morphism respectively. 
    Now consider $B(\operatorname{Bor}(S))$-morphisms 
    \[
        I_p^\infty=\bigoplus_\infty\{I_p:f\in\mathcal{F}\},
        \qquad
        J_p^\infty=\bigoplus_\infty\{J_p:f\in\mathcal{F}\}.
    \]
    These are isometric isomorphisms and one can easily check that 
    \[
        I_p^\infty \circ \operatorname{in}_f=\operatorname{in}_f\circ I_p,
        \qquad 
        \operatorname{pr}_f\circ I_p^\infty=I_p\circ \operatorname{pr}_f.
    \]

    By proposition \ref{SwtchMorphBtwnAtomMeasSp} the 
    map $\Xi_f=J_1\circ m_f\circ I_p$ is 
    a $B(\operatorname{Bor}(S))$-morphism. Hence, the map 
    \[
        \Xi_{\mathcal{F}}=\sum_{f\in\mathcal{F}}\operatorname{in}_f\circ \Xi_f
    \]
    is a $B(\operatorname{Bor}(S))$-morphism and a fortiori a $C_0(S)$-morphism. 
    Note that
    \[
    \begin{aligned}
        I_1^\infty\circ\Xi_\mathcal{F}\circ J_p
        = \sum_{f\in\mathcal{F}} 
            I_1^\infty\circ\operatorname{in}_f \circ \Xi_f \circ J_p
        = \sum_{f\in\mathcal{F}} 
            \operatorname{in}_f\circ I_1 \circ 
            J_1 \circ m_f \circ I_p \circ J_p
        =\sum_{f\in\mathcal{F}} \operatorname{in}_f\circ m_f 
        =\xi_{\mathcal{F}}.
    \end{aligned}
    \]
    By proposition \ref{StdEmbdSpclCoerciv} the coercivity 
    constant $\gamma_{\mathcal{F}}$ is finite and positive, 
    so $\xi_{\mathcal{F}}$ is $\gamma_{\mathcal{F}}$-topologically injective.
    As $I_1$ and $J_p$ are isometric isomorphisms the map $\Xi_{\mathcal{F}}$ is
    also $\gamma_{\mathcal{F}}$-topologically injective. By assumption, 
    the $C_0(S)$-module $L_p(S,\mu)$ is metrically injective, hence there 
    exists a $C_0(S)$-morphism 
    \[
        \Psi:
        \bigoplus_\infty\{ L_1(S,\mu):f\in\mathcal{F}\}\to L_p(S,\mu)
    \]
    such that $\Psi\circ \Xi_{\mathcal{F}}=1_{L_p(S,\mu)}$ 
    and $\Vert \Psi\Vert\leq \gamma_{\mathcal{F}}$.

    Again, for each $f\in\mathcal{F}$ we define 
    a $C_0(S)$-morphism $\Psi_f=\Psi\circ \operatorname{in}_f$. 
    Since $\mathcal{A}$ is finite, then $L_p(S,\mu)$ and $L_1(S,\mu)$ are
    finite-dimensional and therefore reflexive. Hence, by 
    proposition \ref{C0SMorphBtwnReflxSpIsBMorph} for any $f\in\mathcal{F}$ the
    map $\Psi_f$ is a $B(\operatorname{Bor}(S))$-morphism. Note that
    \[
        \Psi=\sum_{f\in\mathcal{F}} \Psi_f\circ\operatorname{pr}_f.
    \]
    Now we define a bounded linear operator
    \[
    \begin{aligned}
        \psi
        =I_p\circ\Psi\circ J_1^\infty
        =I_p\circ\left(
            \sum_{f\in\mathcal{F}} \Psi_f\circ \operatorname{pr}_f
        \right)\circ J_1^\infty 
        =\sum_{f\in\mathcal{F}} 
            I_p\circ\Psi_f\circ\operatorname{pr}_f\circ J_1^\infty 
        =\sum_{f\in\mathcal{F}} 
            I_p\circ\Psi_f\circ J_1\circ\operatorname{pr}_f.
    \end{aligned} 
    \]
    By proposition \ref{SwtchMorphBtwnAtomMeasSp} for each $f\in \mathcal{F}$ 
    the map $I_p\circ \Psi_f\circ J_1$ is 
    an $\ell_\infty(\mathcal{A})$-morphism. Therefore, $\psi$ is 
    an $\ell_\infty(\mathcal{A})$-morphism too. As $I_p$ and $J_1^{\infty}$ are
    isometric isomorphisms we have $\Vert\psi\Vert=\Vert\Psi\Vert$. Moreover,
    \[
    \begin{aligned}
        \psi\circ\xi_\mathcal{F}
        = I_p\circ\Psi\circ J_1^{\infty}\circ 
            I_1^{\infty}\circ \Xi_{\mathcal{F}}\circ J_p
        = I_p\circ\Psi\circ \Xi_{\mathcal{F}}\circ J_p
        = I_p\circ J_p
        = 1_{\ell_p(\mathcal{A})}.
    \end{aligned}
    \]
    Thus, we have constructed an $\ell_\infty(\mathcal{A})$-morphism $\psi$
    such that $\psi\circ\xi_{\mathcal{F}}=1_{\ell_p(\mathcal{A})}$ 
    and $\Vert \psi\Vert\leq\gamma_{\mathcal{F}}$. Since we assumed that $n>1$
    we arrive at contradiction with proposition \ref{RetrPrblmNoSln}. 
    Hence, $n\leq 1$, i.e. $\mathcal{A}$ has at most one atom.
\end{proof}

\begin{proposition}\label{MetInjC0SModLpSmuCharac}
    Let $S$ be a locally compact Hausdorff space and $\mu$ be a decomposable 
    Borel measure on $S$. Suppose $1<p<+\infty$ and 
    the $C_0(S)$-module $L_p(S,\mu)$ is metrically injective, 
    then $\mu$ has at most $1$ atom.
\end{proposition}
\begin{proof} 
    Let $K$ be the Alexandroff's compactification of $S$, then $C_0(S)$ is 
    complemented in $C(K)$. By [\cite{DefFloTensNorOpId}, lemma 4.4] the space
    $C(K)$ is an $\mathscr{L}_\infty^g$-space, hence so is $C_0(S)$ as its 
    complemented subspace. [\cite{DefFloTensNorOpId}, corollary 23.1.2(1)]. 
    Since $L_p(S,\mu)$ is reflexive [\cite{FremMeasTh2}, theorem 244K], then
    by [\cite{NemGeomProjInjFlatBanMod}, corollary 3.14] this $C_0(S)$-module 
    must be finite-dimensional. Now from proposition \ref{LpFinDimCharac} we 
    get that $\mu$ is purely atomic with finitely many atoms of finite measure.
    Finally, from proposition \ref{MetInjC0SModLpSmuOnFinAtmMeasSpCharac} we 
    conclude that $\mu$ has at most one atom.
\end{proof}

\begin{theorem}\label{MetInjPlotjFlatC0SModLpSmuCharac}
    Let $S$ be a locally compact Hausdorff space, $\mu$ be a decomposable 
    Borel measure on $S$. Suppose $1<p<+\infty$ and 
    the $C_0(S)$-module $L_p(S,\mu)$ is metrically projective, injective or 
    flat then $\mu$ has at most $1$ atom.
\end{theorem}
\begin{proof} 
    If $L_p(S,\mu)$ is a metrically injective $C_0(S)$-module, then the 
    result follows from proposition \ref{MetInjC0SModLpSmuOnFinAtmMeasSpCharac}. 
    
    Suppose $L_p(S,\mu)$ is a metrically flat $C_0(S)$-module. Then by 
    [\cite{NemGeomProjInjFlatBanMod}, proposition 2.21] the 
    dual $C_0(S)$-module $L_p(S,\mu)^*$ is metrically injective. 
    It remains to note that $C_0(S)$-modules $L_p(S,\mu)^*$ and $L_{p^*}(S,\mu)$ 
    are isometrically isomorphic and that $1<p^*<+\infty$ for $1<p<+\infty$. 
    Now the result follows from the previous paragraph.
    
    If $L_p(S,\mu)$ is a metrically projective $C_0(S)$-module, then by
    [\cite{NemGeomProjInjFlatBanMod}, proposition 2.26] it is metrically flat. 
    Now the result follows from the previous paragraph.
\end{proof}

\begin{thebibliography}{999}
    %
    \bibitem{NemRelProjModLp}\textit{N. T. Nemesh} Relative projectivity of 
    modules $\it{L_p}$, Math. Notes, 2022, Volume 111, p 103--114.
    %
    %
    \bibitem{HelLectAndExOnFuncAn}\textit{A. Ya. Helemskii} Lectures and 
    exercises on functional analysis, Vol. 233, {2006}, 
    American Mathematical Society Providence, RI.
    %
    %
    \bibitem{HelBanLocConvAlg}\textit{A. Ya. Helemskii} Banach and locally 
    convex algebras, {1993}, Oxford University Press.
    %
    %
    \bibitem{GravInjProjBanMod}\textit{A. W. M. Graven}, Injective and 
    projective Banach modules, Indag. Math. (Proceedings), Vol. 82, p. 253--272,
    Elsevier.
    %
    \bibitem{WhiteInjmoduAlg}\textit{M. C. White} Injective modules for uniform 
    algebras, Proc. Lond. Math. Soc., Vol. 3(1), p. 155--184, Oxford University 
    Press.
    %
    \bibitem{HelMetrFrQMod}\textit{A. Ya. Helemskii} Metric freeness and 
    projectivity for classical and quantum normed modules, Sb. Mat., 
    Vol. 204(7), p. 1056--1083, IOP Publishing.
    %
    %
    \bibitem{HelMetrFlatNorMod}\textit{A. Ya. Helemskii} Metric version of 
    flatness and Hahn-Banach type theorems for normed modules over sequence 
    algebras, Stud. Math., Vol. 206(2), p. 135--160, Institute of Mathematics.
    %
    %
    \bibitem{FremMeasTh2}\textit{D. H. Fremlin} Measure Theory, Vol. 2,
    {2003}, Torres Fremlin.
    %
    %
    \bibitem{DalBanAlgAutCont}\textit{H. G. Dales} Banach algebras and 
    automatic continuity, {2000}, Clarendon Press.
    %
    %
    \bibitem{DefFloTensNorOpId}\textit{A. Defant, K. Floret} Tensor norms and 
    operator ideals, Vol. 176, {1992}, Elsevier.
    %
    %
    \bibitem{NemGeomProjInjFlatBanMod}\textit{N. T. Nemesh} The Geometry of 
    projective, injective and flat Banach modules, J. Math. Sci. (New York), 
    2019, Volume 237, Issue 3, Pages 445–459.
\end{thebibliography}

Norbert Nemesh, Faculty of Mechanics and Mathematics, Moscow State University,
Moscow 119991 Russia

\textit{E-mail address:} nemeshnorbert@yandex.ru


\end{document}