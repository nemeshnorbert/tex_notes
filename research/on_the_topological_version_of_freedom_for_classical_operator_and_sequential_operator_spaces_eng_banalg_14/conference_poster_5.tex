% chktex-file 35
%%%%%%%%%%%%%%%%%%%%%%%%%%%%%%%%%%%%%%%%%
% a0poster Landscape Poster
% LaTeX Template
% Version 1.0 (22/06/13)
%
% The a0poster class was created by:
% Gerlinde Kettl and Matthias Weiser (tex@kettl.de)
% 
% This template has been downloaded from:
% http://www.LaTeXTemplates.com
%
% License:
% CC BY-NC-SA 3.0 (http://creativecommons.org/licenses/by-nc-sa/3.0/)
%
%%%%%%%%%%%%%%%%%%%%%%%%%%%%%%%%%%%%%%%%%

%-------------------------------------------------------------------------------
%	PACKAGES AND OTHER DOCUMENT CONFIGURATIONS
%-------------------------------------------------------------------------------

\documentclass[a0b,landscape]{a0poster}
% This is so we can have multiple columns of text side-by-side
\usepackage{multicol} 
% This is the amount of white space between the columns in the poster
\columnsep=100pt 
% This is the thickness of the black line between the columns in the poster
\columnseprule=3pt 
% Specify colors by their 'svgnames', for a full list of all colors 
% available see here: http://www.latextemplates.com/svgnames-colors
\usepackage[svgnames]{xcolor} 

% Use the times font
\usepackage{times} 
%\usepackage{palatino} % Uncomment to use the Palatino font

% Required for including images
\usepackage{graphicx} 
% Location of the graphics files
\graphicspath{{figures/}} 
% Top and bottom rules for table
\usepackage{booktabs} 
% Required for specifying captions to tables and figures
\usepackage[font=small,labelfont=bf]{caption} 
% For math fonts, symbols and environments
\usepackage{amsfonts, amsmath, amsthm, amssymb} 
% Allows wrapping text around tables and figures
\usepackage{wrapfig} 

\usepackage[matrix,arrow,curve]{xy}

\newtheorem*{theorem}{Theorem}
\newtheorem*{lemma}{Lemma}
\newtheorem*{proposition}{Proposition}
\newtheorem*{remark}{Remark}
\newtheorem*{corollary}{Corollary}
\newtheorem*{definition}{Definition}
\newtheorem*{example}{Example}
\newtheorem*{scheme}{Scheme}

\begin{document}

%-------------------------------------------------------------------------------
%	POSTER HEADER 
%-------------------------------------------------------------------------------

% The header is divided into three boxes:
% The first is 55% wide and houses the title, subtitle, 
% names and university/organization
%
% The second is 25% wide and houses contact information
% 
% The third is 19% wide and houses a logo for your 
% university/organization or a photo of you
% The widths of these boxes can be easily edited to 
% accommodate your content as you see fit

\begin{minipage}[b]{0.85\linewidth}
\veryHuge{} \color{NavyBlue} \textbf{
    On the topological version of freedom for classical, 
    operator and sequential operator spaces} \color{Black}\\ % Title
\Large\textbf{Sergei Shteiner \& Norbert Nemesh}\\ % Author(s)
\Large Moscow State University\\ % University/organization
\end{minipage}
%
%
\begin{minipage}[b]{0.15\linewidth}
\begin{center}
% Logo or a photo of you, adjust its dimensions here
\includegraphics[width=10cm]{Eep8BvgUZEk.jpg} 
\end{center}
\end{minipage}

%\vspace{1cm}  A bit of extra whitespace between the header and poster content

%-------------------------------------------------------------------------------
% This is how many columns your poster will be broken into, a poster with many
% figures may benefit from less columns whereas a text-heavy poster benefits
% from more

\begin{multicols}{4} 

\small

\section*{Introduction}


Consider the category $A-mod$ of left normed modules over a fixed normed unital
algebra $A$ and their bounded morphisms. A normed $A$-module $P$ is called
\textit{topologically (metrically) projective} if for each open map (strict
coisometry) $\tau : Y \to X$, where $X, Y \in A-mod$, and each bounded morphism
$\varphi : P \to X$, there exists a lifting (a lifting with the same norm)
$\psi$ of $\varphi$ across $\tau$, that is a morphism making the diagram
$$
\xymatrix{
& {Y} \ar[d]^{\tau}\\  % chktex 3
{P} \ar@{-->}[ur]^{\psi} \ar[r]^{\varphi} &{X}}  % chktex 3
$$
commutative. Similarly, a normed $A$-module $I$ is called \textit{topologically
(metrically) injective} if for each bounded below map (isometry) $\tau : X \to
Y$, where $X, Y \in A-mod$, and each bounded morphism $\varphi : X \to I$, there
exists an extension (an extension with the same norm) $\psi$ of $\varphi$ across
$\tau$, that is a morphism making the diagram
$$
\xymatrix{
& {Y} \ar@{-->}[dl]_{\psi}\\  % chktex 3
{I}  &{X} \ar[l]_{\varphi} \ar[u]_{\tau}  % chktex 3
}
$$
commutative.

The topological projectivity was studied by Gottfried K\"{o}te in the context of
Banach spaces and by Niels Gr{\o}nbaek in the context of normed spaces. The
metric projectivity was recently introduced by Alexander Helemskii. The both
types of projectivity and injectivity can be considered as particular cases of a
certain general categorical scheme. Let $\mathcal{K}$ be an arbitrary category.
A \textit{rig} of $\mathcal{K}$ is a faithful covariant functor $\square :
\mathcal{K} \to \mathcal{L}$, where $\mathcal{L}$ is a category. A pair
consisting of a category and its rig is called \textit{rigged category}. Fix,
for a time, a rigged category, say $\left (\mathcal{K}, \square : \mathcal{K}
\to \mathcal{L}\right )$. We call a morphism $\tau$ in $\mathcal{K}$ an
\textit{admissible epimorphism (admissible monomorphism)} if the morphism
$\square(\tau)$ is a retraction (coretraction) in $\mathcal{L}$. An object $P$
($I$) in $\mathcal{K}$ is called projective (injective) with respect to the rig
$\square$ if for every admissible epimorphism (monomorphism) $\tau$ the map
$\operatorname{Hom}_{\mathcal{K}}(P,\tau)
:\operatorname{Hom}_{\mathcal{K}}(P,Y)\to\operatorname{Hom}_{\mathcal{K}}(P,X)$
($\operatorname{Hom}_{\mathcal{K}}(\tau,I)
:\operatorname{Hom}_{\mathcal{K}}(Y,I)\to\operatorname{Hom}_{\mathcal{K}}(X,I)$)
is surjective. Later on we will give the definitions of rigs describing metric
and topological projectivity and injectivity.

Now we define a free object for the rig. Let $M$ be an object in $\mathcal{L}$.
An object $F$ in $\mathcal{K}$ is called \textit{$\square$-free
($\square$-cofree)} object with the \text{base} $M$ if there exists an
isomorphism of functors
$\operatorname{Hom}_{\mathcal{K}}(F,-)
\cong\operatorname{Hom}_{\mathcal{L}}(M,\square(-))$
($\operatorname{Hom}_{\mathcal{K}}(-,F)
\cong\operatorname{Hom}_{\mathcal{L}}(\square(-),M)$).
A rigged category is called \textit{freedom (cofreedom)-loving} if every object
in $\mathcal{L}$ is a base of a free (cofree) object in $\mathcal{K}$. The
following fact is the main reason to study these objects.

\begin{proposition}[Helemskii] If a rigged category $(\mathcal{K},\square)$ is
$\square$-freedom (cofreedom)-loving, then all its $\square$-projective
(injective) objects are retracts of $\square$-free (cofree) objects.
\end{proposition}

It is worth to mention that in general for a given type of projectivity there
are rigs giving different classes of free objects but the same capacity of
projective objects. Our nearest goal is to find $\square_{Met}$- and
$\square_{Top}$-free and cofree objects for the case $A = \mathbb{C}$ and the
categories of normed spaces, operator spaces, sequential operator spaces and
non-archimedian spaces.





%%%%%%%%%%%%%%%%%%%%%%%%%%%%%%%%%%%%%%%%%%%%%%%%%%%%%%%%%%%%%%%%%%%%%%%%%%%%%%%%
%	Normed spaces
%%%%%%%%%%%%%%%%%%%%%%%%%%%%%%%%%%%%%%%%%%%%%%%%%%%%%%%%%%%%%%%%%%%%%%%%%%%%%%%%





\section*{Normed spaces}

To put the metric projectivity in the general scheme, consider the functor
$$
\square_{Met}
:Nor_1\to Set
: X\mapsto Ball_X\quad \varphi\mapsto \varphi|_{Ball_X}^{Ball_Y}.
$$
It is almost immediate that in our case $\square_{Met}$-admissible epimorphisms
are exactly strictly coisometric morphisms, and $\square_{Met}$-projective
objects are exactly metrically projective modules.


Let us now define the topological projectivity in this scheme. A
\textit{semilinear} space over a field $K$ is a set $V$ together with a binary
operation $\cdot:K\times V\to V$ that satisfies the three axioms listed below:

(1) for all $a,b\in K$ and $v\in V$ holds $a\cdot (b\cdot v) = ab\cdot v$,

(2) for all $v\in V$ we have $1_K\cdot v=v$,

(3) there exists $\mathbf{0} \in V$, such that $0_K\cdot v = \mathbf{0}$ for all
$v \in V$.

A map $\varphi : V \to W$  between two semilinear spaces is said to be a
\textit{semilinear} map if for each vector $v \in V$ and each scalar $\alpha\in
K$, the following condition is satisfied:  $\varphi(\alpha \cdot v) = \alpha
\cdot \varphi(v)$.

A norm on a semilinear space $V$ over a valued field $K$ is a function $ \Vert
\cdot  \Vert  : V \to \mathbb{R}_{+}$, such that

(1) $ \Vert  v  \Vert  = 0$ if and only if $v = \mathbf{0}$;

(2) $ \Vert  \alpha \cdot v  \Vert  = | \alpha |  \Vert  v  \Vert $ for all
$\alpha \in K$, and $v \in V$.

A normed semilinear space is a pair $(V,  \Vert  \cdot  \Vert)$, where $V$ is
a semilinear space and $  \Vert  \cdot  \Vert  $ is a norm on $V$. By $Nor_0^K$
we denote the category of normed semilinear spaces over $K$ and bounded
semilinear maps.  Let $\Lambda$ be a set and $K_{\lambda}=K$ for each $\lambda
\in \Lambda$. One can show that any semilinear space is isomorphic in $Nor_0^K$
to the wedge sum $\bigvee \{K_{\lambda}: \lambda \in\Lambda \}$ of copies of
pointed spaces $(K,0_K)$ for a certain set $\Lambda$.

Let now $K := \mathbb{C}$, then consider the functor
$$
\square_{Top}:Nor\to Nor_0: X\mapsto X\quad \varphi\mapsto \varphi
$$
which forgets about addition. One can check that in this case
$\square_{Top}$-admissible epimorphisms are exactly open morphisms, and
$\square_{Top}$-projective objects are exactly topologically projective modules.

We will use the following observation.

\begin{proposition}[Helemskii] Suppose that $\mathcal{K}$ admits coproducts,
$\mathcal{L}$ is just $Set$, and there exists a free module, say $F$, with a
one-point base. Then our rigged category is freedom-loving. Moreover, if
$\Lambda$ is an arbitrary set, then the coproduct of the family of copies of
$F$, indexed by points of $\Lambda$, is the free object with the base $\Lambda$.
\end{proposition}

Metrically or topologically free objects with a one-point base are obviously one
dimensional spaces, and coproducts in $Nor_1$ are their $l_1^0$-sums. Thus
metrically free objects are exactly $l_1^0(\Lambda)$ spaces and rigged category
$(Nor_1, \square_{Met})$ is freedom-loving. For the topological freedom in
general there is no coproducts in $Nor$, but the following proposition saves the
day.

\begin{proposition} Let $E$ be a metrically free normed space with a base
$\Lambda$. Then $E$ is topologically free with the base $\bigvee \{K_{\lambda} :
\lambda \in\Lambda \}$. Conversely, let $E$ be a topologically free normed space
with the base $\bigvee \{K_{\lambda}: \lambda \in\Lambda \}$. Then it is
topologically isomorphic to some metrically free normed space $E^{'}$ with the
base $\Lambda$.
\end{proposition}

Thus we obtain that topologically free spaces are also $l_1^0(\Lambda)$ (but in
the category $Nor$) and the rigged category $(Nor, \square_{Top})$ is
freedom-loving. Of course there is an easy straightforward proof of this fact,
but this approach seems to be more convenient in more complicated cases
discussed below.

Similarly, one can show that admissible monomorphisms for functors
$$
\square_{Met}^d
:Nor_1\to Set^o
: X\mapsto Ball_{X^*},\quad\varphi\mapsto \varphi^*|_{Ball_{Y^*}}^{Ball_{X^*}}
$$
$$
\square_{Top}^d:Nor\to Nor_0^o: X\mapsto X^*,\quad\varphi\mapsto \varphi^*
$$
are exactly isometric and bounded below morphisms respectively. We obtain the
description of free objects. Now it is easy to describe cofree ones using the
following result.

\begin{proposition}[Helemskii] Assume we are given two rigged categories
$(\mathcal{K}_1, \square_1: \mathcal{K}_1 \to \mathcal{L}_1)$, $(\mathcal{K}_2,
\square_2 : \mathcal{K}_2 \to \mathcal{L}_2)$, and two covariant functors $\Phi
: \mathcal{K}_1 \to \mathcal{K}_2$, $\Psi : \mathcal{L}_1 \to \mathcal{L}_2$
such that the diagram is
$$
\xymatrix{
{\mathcal{K}_1}\ar[d]_{\Phi}  %chktex 3
\ar[rr]^{\square_1} && {\mathcal{L}_1}\ar[d]^{\Psi}\\  % chktex 3
{\mathcal{K}_2}\ar[rr]^{\square_2} &&{\mathcal{L}_2}}  % chktex 3
$$
commutative. Assume $\Phi$ and $\Psi$ have left adjoint functors $\Phi^*$ and
$\Psi^*$ respectively, and $F\in\mathcal{K}_2$ is a $\square_2$-free object with
base $M\in\mathcal{L}_2$. Then $\Phi^*(F)\in\mathcal{K}_1$ is a $\square_1$-free
object with base $\Psi^*(M)\in\mathcal{L}_1$. In particular if
$(\mathcal{K}_2,\square_2)$ is freedom-loving then so does
$(\mathcal{K}_1,\square_1)$.
\end{proposition}

One can check that the assumptions of this proposition hold for the pairs of
functors $({(\square_{Met}^d)}^o,\square_{Met})$ and
$({(\square_{Top}^d)}^o,\square_{Top})$ with $\Phi$ as ${}^*$ functor and $\Psi$
as an identity functor. Hence metrically (topologically) cofree normed spaces
are exactly ${(\ell_1^0(\Lambda))}^*=\ell_\infty(\Lambda)$ for some 
set $\Lambda$.

\begin{scheme}
All in all, the solution above consists of four basic steps:

(1) We find a free object with a one-point base.

(2) We notice that the category in question admits coproducts, so free objects
are exactly the coproducts of copies of free objects with a one-point base.

(3) We prove that all metrically free objects are topologically free.

(4) We check that the conditions of the previous proposition hold, so metrically
and topologically cofree objects are duals of free ones.
\end{scheme}

This scheme can be applied to the categories we are going to discuss right now
--- operator spaces, sequential operator spaces and non-archimedean spaces.






%%%%%%%%%%%%%%%%%%%%%%%%%%%%%%%%%%%%%%%%%%%%%%%%%%%%%%%%%%%%%%%%%%%%%%%%%%%%%%%%
%	Operator spaces
%%%%%%%%%%%%%%%%%%%%%%%%%%%%%%%%%%%%%%%%%%%%%%%%%%%%%%%%%%%%%%%%%%%%%%%%%%%%%%%%




\section*{Operator spaces}

Operator spaces are well known but we feel obliged to recall necessary
definitions. Let $E$ be a normed space, then for given $m,n\in\mathbb{N}$ by
$M_{m,n}(E)$ we denote a complex linear space of $m\times n$ matrices with
entries in $E$. By $\Vert\cdot\Vert_n$ we denote a norm on $M_{n,n}(E)$. We say
that the pair $X=(E,\{\Vert\cdot\Vert_n:n\in\mathbb{N}\})$ defines the structure
of \textit{operator space} on $E$ if 

(1) for all $x\in M_{n,n}(E)$ and $\alpha\in M_{m,n}(\mathbb{C})$, $\beta\in
M_{n,m}(\mathbb{C})$ we have
$
\Vert\alpha x\beta\Vert_m\leq\Vert\alpha\Vert\Vert x\Vert_n\Vert\beta\Vert,
$

(2) for all $x\in M_{n,n}(E)$, $y\in M_{m,m}(E)$ we have
$
\Vert x\oplus y\Vert_{n+m}\leq\max(\Vert x\Vert_n,\Vert y\Vert_m),
$\\
where $m,n\in\mathbb{N}$. For example $\mathcal{B}(\mathbb{C}^k)$ can be endowed
with this structure via the identification
$M_{n,n}(\mathcal{B}(\mathbb{C}^k))=\mathcal{B}(\mathbb{C}^{nk})$.

For a given operator spaces $X$ and $Y$ and a linear operator $\varphi:X\to Y$
we define its $n$-th \textit{amplification} as a linear operator
$\varphi^n:M_{n,n}(X)\to M_{n,n}(Y):x\mapsto ({\varphi(x_{ij})}_{i,j=1,n})$. A
linear operator $\varphi$ is called \textit{completely bounded} if its $cb$-norm
$\Vert \varphi\Vert_{cb}=\sup \{\Vert
\varphi^n\Vert_{\mathcal{B}(M_n(X),M_n(Y)):n\in\mathbb{N}}\}$ is finite. The set
of completely bounded operators between $X$ and $Y$ forms an operator space
$\mathcal{CB}(X,Y)$ with the sequence of norms given by the identification
$M_{n,n}(\mathcal{CB}(X,Y))=\mathcal{CB}(X,M_{n,n}(Y))$. Therefore in the
operator space theory we have the analogue of continuous dual space defined by
$X^\triangle=\mathcal{CB}(X,\mathbb{C})$. In this case we can endow the space of
nuclear operators $\mathcal{N}(H,K)$ between two Hilbert spaces $H$, $K$ with an
operator space structure via the natural inclusion
$\mathcal{N}(H,K)\hookrightarrow {\mathcal{B}(K,H)}^\triangle$. 
Then one can show
that ${\mathcal{B}(\mathbb{C}^n)}^\triangle=\mathcal{N}(\mathbb{C}^n)$ and
${\mathcal{N}(\mathbb{C}^n)}^\triangle=\mathcal{B}(\mathbb{C}^n)$.

Thus we can define two categories --- $QNor$ and $QNor_1$. Their objects are
operator spaces. Their morphisms are completely bounded and completely
contractive linear operators respectively.

The category $QNor_1$ admits products and coproducts. Let $\{X_ \lambda :
\lambda \in \Lambda \}$ be a family of objects in $QNor_1$, 
then their product is
a normed space $\bigoplus_\infty \{M_{1,1}(X_ \lambda ): \lambda \in \Lambda \}$
with a sequence of norms induced by the identification
$M_{n,n}\left(\bigoplus{}_\infty \{X_ \lambda : \lambda  \in
\Lambda \}\right)=\bigoplus{}_\infty \{M_{n,n}(X_ \lambda ): \lambda \in
\Lambda \}$. The coproduct is a normed space 
$\bigoplus{}_1^0\{M_{1,1}(X_ \lambda): \lambda \in \Lambda \}$ 
with a sequence of norms induced by the inclusion
$\bigoplus{}_1^0\{
    X_\lambda:\lambda\in\Lambda 
 \}\hookrightarrow{
     \left(\bigoplus{}_\infty \{
         X_\lambda^\triangle:\lambda\in\Lambda 
     \}\right)}^\triangle$.

Let us define two types of admissible morphisms to introduce metric and
topological projectivity and injectivity. We say that a morphism of operator
spaces $\varphi$ is called \textit{completely strictly coisometric (open map)}
if all its amplifications are strictly coisometric (open maps with the same
coercivity constant). Along the same lines we say that a morphism of operator
spaces $\varphi$ is called \textit{completely isometric (bounded below)} if all
its amplifications are isometric (bounded below with the same constant). One can
check that the following four faithful functors
$$
\begin{aligned}
&\square_{qMet} 
:QNor_1 \to Set 
:X\mapsto\prod_{n \in \mathbb{N}} Ball_{M_{n,n}(X)}\quad
\varphi\mapsto 
    \prod_{n\in\mathbb{N}} \varphi^{n}|_{
        Ball_{M_{n,n}(X)}}^{Ball_{M_{n,n}(Y)}},\\
&\square_{qTop}
:QNor \to Nor_0
:X \mapsto \bigoplus_{ n \in \mathbb{N}}{}_\infty M_{n,n}(X) \quad
\varphi\mapsto\bigoplus_{n\in\mathbb{N}}{}_\infty \varphi^{n},\\
&\square_{qMet}^d
:QNor_1 \to Set
:X\mapsto\prod_{n \in \mathbb{N}} Ball_{M_{n,n}(X^\triangle)}\quad
\varphi\mapsto \prod_{n\in\mathbb{N}} {(\varphi^\triangle)}^{n}|_{
    Ball_{M_{n,n}(Y^\triangle)}}^{Ball_{M_{n,n}(X^\triangle)}},\\
&\square_{qTop}^d
:QNor\to Nor_0^o
:X \mapsto \bigoplus_{ n \in \mathbb{N}}{}_\infty M_{n,n}(X^\triangle) \quad
\varphi\mapsto\bigoplus_{n\in\mathbb{N}}{}_\infty{(\varphi^\triangle)}^{n}
\end{aligned}
$$
describe metric and topological projectivity and injectivity in the category of
operator spaces. One can show that $\mathcal{N}_{\infty} := \bigoplus_1^0
\{\mathcal{N}(\mathbb{C}^n): n \in \mathbb{N}\}$ is metrically free with a
one-point base. Moreover, all metrically free operator spaces are topologically
free. So using the general scheme we conclude that all metrically and
topologically free objects are of the form $\bigoplus_1^0\{\mathcal{N}_\infty:
\lambda \in \Lambda \}$. Finally, following the general scheme we obtain the
description of metrically and topologically cofree objects. They are
$\bigoplus_\infty \{\mathcal{B}_\infty: \lambda \in \Lambda \}$, where
$\mathcal{B}_{\infty} := \bigoplus_\infty \{\mathcal{B}(\mathbb{C}^n): n \in
\mathbb{N}\}$.





%%%%%%%%%%%%%%%%%%%%%%%%%%%%%%%%%%%%%%%%%%%%%%%%%%%%%%%%%%%%%%%%%%%%%%%%%%%%%%%%
%	Sequential operator spaces
%%%%%%%%%%%%%%%%%%%%%%%%%%%%%%%%%%%%%%%%%%%%%%%%%%%%%%%%%%%%%%%%%%%%%%%%%%%%%%%%




\section*{Sequential operator spaces}

Sequential operator spaces were introduced and studied by Anselm Lambert in his
PhD thesis in 2002. Informally speaking these spaces are situated between
classical Banach spaces and operator spaces. Let $E$ be a linear space, and for
each $n\in\mathbb{N}$ we have a linear space $E^n:=\oplus_{k=1}^n E$ and a norm
on $\Vert \cdot \Vert_{\widehat{n}}:E^n\to\mathbb{R}_+$. We say that the pair $X
= (E^n, \{\Vert \cdot \Vert_{\widehat{n}}:n\in\mathbb{N}\})$ defines the
structure of \textit{sequential operator space} on $E$, if the following
conditions are satisfied:

(1) for all $x \in E^{\widehat{n}}$, $\alpha \in M_{m, n}(\mathbb{C})$ holds
$
\Vert \alpha x \Vert_{\widehat{m}}\leq\Vert\alpha\Vert\Vert x\Vert_{\widehat{n}}
$,

(2) for all $x \in E^n$, $y \in E^m$ holds
$
\left\Vert ( x, y ) \right\Vert^2_{\widehat{n + m}} 
\leq \Vert x \Vert_{\widehat{n}}^2 + \Vert y \Vert_{\widehat{m}}^2,
$\\
where $m,n\in\mathbb{N}$. By $X^{\widehat{n}}$ we will denote the normed space
$(E^n,\Vert \cdot \Vert_{\widehat{n}})$.

An obvious example of such structure is the one-dimensional complex linear space
$l_2^n$, which is just $\mathbb{C}$ with norms defined by the identification
$\mathbb{C}^{\widehat{n}}=\mathbb{C}^n$. Next example is the $n$-dimensional
complex linear space $t_2^n$ with a sequence of norms defined by the
identification ${(t_2^n)}^{\widehat{k}}=\mathcal{B}(\mathbb{C}^k,\mathbb{C}^n)$.

If $X$ and $Y$ are two sequential operator spaces and $\varphi : X \rightarrow
Y$ is a linear operator, then for each $n\in\mathbb{N}$ its $n$-th
\textit{amplification} is  
a linear operator $\varphi^{\widehat{n}} : X^{\widehat{n}} \to Y^{\widehat{n}}$
defined by $\varphi^{\widehat{n}}(x)={(\varphi(x_i))}_{i=1,n}$. We say that
$\varphi$ is \textit{sequentially bounded}, if $sb$-norm $\Vert \varphi
\Vert_{sb} := \sup \{\Vert
    \varphi^{\widehat{n}}
\Vert_{\mathcal{B}(X^{\widehat{n}},Y^{\widehat{n}})}:n\in\mathbb{N}\}$
is finite. The set of such operators forms a normed space $\mathcal{SB}(X,Y)$,
which in turn can be endowed with a sequential operator space structure via the
identification
${\mathcal{SB}(X,Y)}^{\widehat{n}}=\mathcal{SB}(X,Y^{\widehat{n}})$. This allows
us to introduce the notion of sequential dual of sequential operator space
$X^\triangle:=\mathcal{SB}(X,\mathbb{C})$. One can show that
${(t_2^n)}^\triangle=l_2^n$ and ${(l_2^n)}^\triangle=t_2^n$.

Now we define two categories $SQNor$ and $SQNor_1$. Their objects are sequential
operator spaces. Morphisms of the first category are sequentially bounded linear
operators, morphisms of the second category are sequentially bounded linear
operators with $sb$-norm not greater than $1$.

Let $\{X_ \lambda : \lambda \in \Lambda \}$ be a family of objects in $SQNor_1$,
then their product is a normed space $\bigoplus_\infty \{X_ \lambda
^{\widehat{1}}: \lambda \in \Lambda \}$ with the sequence of norms induced by the
identification ${\left(\bigoplus{}_\infty \{X_ \lambda : \lambda  \in
\Lambda \}\right)}^{\widehat{n}} =\bigoplus{}_\infty \{X_ \lambda ^{\widehat{n}}:
\lambda \in \Lambda \}$. The coproduct is a normed space $\bigoplus{}_1^0\{X_
\lambda ^{\widehat{1}}: \lambda \in \Lambda \}$ with a sequence of norms induces
by the inclusion $\bigoplus{}_1^0\{X_ \lambda : \lambda  \in
\Lambda \}\hookrightarrow {\left(\bigoplus{}_\infty \{X_ \lambda ^\triangle:
\lambda \in \Lambda \}\right)}^\triangle$.

Let us define two types of admissible morphisms to introduce metric and
topological projectivity and injectivity. We say that a morphism of sequential
operator spaces $\varphi$ is called \textit{sequentially strictly coisometric
(open map)} if all its amplifications are strictly coisometric (open maps with
the same coercivity constant). Similarly we say that a morphism of sequential
operator spaces $\varphi$ is \textit{sequentially isometric (bounded below)} if
all its amplifications are isometric (bounded below with the same constant). One
can check that for the following four faithful functors
$$
\begin{aligned}
&\square_{sqMet}
:SQNor_1 \to Set
:X\mapsto\prod_{n \in \mathbb{N}} Ball_{X^{\widehat{n}}}\quad
\varphi\mapsto \prod_{n\in\mathbb{N}} 
    \varphi^{\widehat{n}}|_{Ball_{X^{\widehat{n}}}}^{Ball_{Y^{\widehat{n}}}},\\
&\square_{sqTop}
:SQNor \to Nor_0
:X \mapsto \bigoplus_{n \in \mathbb{N}}{}_\infty X^{\widehat{n}}  \quad
\varphi\mapsto\bigoplus_{n\in\mathbb{N}}{}_\infty \varphi^{\widehat{n}},\\
&\square_{sqMet}^d
:SQNor_1 \to Set
:X\mapsto\prod_{n \in \mathbb{N}} Ball_{{(X^\triangle)}^{\widehat{n}}}\quad
\varphi\mapsto \prod_{n\in\mathbb{N}} {(\varphi^\triangle)}^{\widehat{n}}|_{
    Ball_{{(Y^\triangle)}^{\widehat{n}}}
}^{
    Ball_{{(X^\triangle)}^{\widehat{n}}}
},\\
&\square_{sqTop}^d
:SQNor \to Nor_0^o
:X\mapsto \bigoplus_{n\in\mathbb{N}}{}_\infty {(X^\triangle)}^{\widehat{n}}\quad
\varphi\mapsto\bigoplus_{n\in\mathbb{N}}{}_\infty {(
    \varphi^\triangle
)}^{\widehat{n}}
\end{aligned}
$$
describe metric and topological projectivity and injectivity in the category of
sequential operator spaces. One can show that $t_2^{\infty} := \bigoplus_1^0
\{t_2^n: n \in \mathbb{N}\}$ is a metrically free with a one-point base and that
every metrically free object is topologically free, so using the general scheme
we conclude that all metrically and topologically free objects are of the form
$\bigoplus_1^0\{t_2^\infty: \lambda \in \Lambda \}$. Follow the general scheme
one can obtain the description of metrically and topologically cofree objects.
They are $\bigoplus_\infty \{l_2^\infty: \lambda \in \Lambda \}$, where
$l_2^{\infty} := \bigoplus_\infty \{l_2^n: n \in \mathbb{N}\}$.






%%%%%%%%%%%%%%%%%%%%%%%%%%%%%%%%%%%%%%%%%%%%%%%%%%%%%%%%%%%%%%%%%%%%%%%%%%%%%%%%
%	Non-archimedean spaces
%%%%%%%%%%%%%%%%%%%%%%%%%%%%%%%%%%%%%%%%%%%%%%%%%%%%%%%%%%%%%%%%%%%%%%%%%%%%%%%%





\section*{Non-Archimedean spaces}

Let $K$ be a field, then \textit{non-Archimedean valuation} on $K$ is a function
$|\cdot|:K\to \mathbb{R}_+$ such that for all $a,b\in K$ holds

(1) $|ab|=|a||b|$,

(2) $|a+b|\leq\max(|a|,|b|)$,

(3) $|a|=0$ if and only if $a=0$.

We additionally assume that the valuation is non-trivial, i.e.~there exists
$a\in K$ with $|a|\notin \{0,1 \}$. Each valuation makes $K$ a metric space
with a distance function $d(a,b)=|a-b|$. So $K$ is called a non-Archimedean
field if it has a non-trivial valuation which makes $K$ a complete metric space.
Most of results of classical functional analysis remains the same if we
additionally require $K$ to be \textit{spherically complete}, i.e.~any
descending by inclusion chain of balls in $K$ has non-empty intersection.

Let $E$ be a linear space over $K$. The \textit{non-Archimedean norm} on $E$ is
a function $\Vert\cdot\Vert:E\to\mathbb{R}_+$ such that

(1) $\Vert a x\Vert=|a|\Vert x\Vert$,

(2) $\Vert x+y\Vert\leq\max(\Vert x\Vert,\Vert y\Vert)$,

(3) $\Vert x\Vert=0$ if and only if $x=0$ \\
for all $x,y\in V$ and $a\in K$. A pair $(X,\Vert\cdot\Vert)$ is called a
\textit{non-Archimedean normed space}. As in the classical case we define
bounded linear operators. Now we deal with two categories $Nor^K$ and $Nor^K_1$.
Their objects are non-Archimedean normed spaces over $K$. Morphisms of the first
category are bounded linear operators, morphisms of the second category are
contractive ones.

The definitions of metrically and topologically admissible epimorphisms and
monomorphisms, the definitions of metrically and topologically projective and
injective spaces, the definitions of faithful functors describing these types of
projectivity and injectivity, all is the same as in the case of normed spaces.

For a given family of non-Archimedean normed spaces $\{E_ \lambda : \lambda \in
\Lambda \}$ their coproduct in $Nor_1^K$ is $c_{00}$ sum, i.e.
$\bigoplus_{00}\{E_ \lambda : \lambda \in \Lambda \}$. Obviously free object with
a one-point base is a one-dimensional space, i.e.~the field $K$ itself.
Applying the general scheme we get that both metrically and topologically free
objects are exactly $c_{00}(\Lambda)$ for a some set $\Lambda$.


If $K$ is spherically complete then we have the whole analogue of the classical
duality theory of normed spaces, so the general scheme tells that the only
metrically and topologically cofree spaces are $\ell_\infty(\Lambda)$ for a some
set $\Lambda$. If $K$ is not spherically complete then the ground field is not
metrically and topologically injective. Hence there is no injective and cofree
objects in both senses for the case of non-sperically complete fields.





%%%%%%%%%%%%%%%%%%%%%%%%%%%%%%%%%%%%%%%%%%%%%%%%%%%%%%%%%%%%%%%%%%%%%%%%%%%%%%%%
%	Generalization to modules and Banach spaces
%%%%%%%%%%%%%%%%%%%%%%%%%%%%%%%%%%%%%%%%%%%%%%%%%%%%%%%%%%%%%%%%%%%%%%%%%%%%%%%%

\section*{Generalization to modules and Banach spaces}


We need the following observation.

\begin{proposition}Let $\square_{12}:\mathcal{K}_1\to\mathcal{K}_2$,
$\square_{23}:\mathcal{K}_2\to\mathcal{K}_3$ be faithful functors. Denote
$\square_{13}=\square_{23}\square_{12}$. Let $Fr_1(M)$ be a $\square_{23}$-free
object with a base $M \in \mathcal{K}_3$ and let $Fr_2(Fr_1(M))$ be a
$\square_{23}$-free object with a base $Fr_1(M) \in \mathcal{K}_2$. Then
$Fr_2(Fr_1(M))$ is a $\square_{13}$-free object with a base $M$.
\end{proposition}

Consider the natural forgetful functor $\square^{A}_{1}:A-mod_1\to Nor_1$. The
rigged category $(A-mod_1, \square^{A}_{1})$ is freedom-loving; free objects are
$A \otimes_p E$ where $E$ is a normed space. Using previous proposition we
obtain that metrically free $A$-modules are exactly $A \otimes_p
l_1^0(\Lambda)$. The same result holds for the topological freedom. Both
metrically and topologically cofree $A$-modules are $\mathcal{B}(A,
l_{\infty}(\Lambda))$ with the natural module structure. This approach with the
obvious modifications works in all categories described above. For example,
metrically and topologically free weak operator modules over a weak unital
operator algebra $A$ are $\bigoplus_1^0\{A \otimes_{op} \mathcal{N}_\infty:
\lambda \in \Lambda \}$ and free strong operator modules over a strong unital
operator algebra $A$ are $\bigoplus_1^0\{A \otimes_{h} \mathcal{N}_\infty:
\lambda \in \Lambda \}$.

The same idea works in the Banach case. Consider the inclusion functor
$\square_{Ban_1}: Ban_1\to Nor_1$. The rigged category $(Ban_1,
\square_{Ban_1})$ is freedom-loving; free object with a base $E$ is its
completion $\widehat{E}$ of a normed space $E$. We obtain that metrically free
Banach spaces are exactly the completions of metrically free normed spaces. The
same result holds for the topological freedom and both types of cofreedom. It
can be obviously expanded to the categories described above (including the
categories of normed modules).

\end{multicols}
\end{document}