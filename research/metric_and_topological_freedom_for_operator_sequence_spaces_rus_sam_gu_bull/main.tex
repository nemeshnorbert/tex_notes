%\title{Метрическая и топологическая свобода для секвенциальных операторных пространств}
\documentclass[12pt]{article}
\usepackage[left=2cm,right=2cm,
top=2cm,bottom=2cm,bindingoffset=0cm]{geometry}
\usepackage{amssymb,amsmath}
\usepackage[T1,T2A]{fontenc}
\usepackage[utf8]{inputenc}
\usepackage[matrix,arrow,curve]{xy}
\usepackage[russian]{babel} 
\usepackage[final]{graphicx} 
\usepackage{mathrsfs}
\usepackage[colorlinks=true, urlcolor=blue, linkcolor=blue, citecolor=blue, pdfborder={0 0 0}]{hyperref}
\usepackage{yhmath}

\pretolerance=1000
\tolerance=200
\emergencystretch=10pt
\newtheorem{theorem}{Теорема}[section]
\newtheorem{lemma}[theorem]{Лемма}
\newtheorem{proposition}[theorem]{Предложение}
\newtheorem{remark}[theorem]{Замечание}
\newtheorem{corollary}[theorem]{Следствие}
\newtheorem{definition}[theorem]{Определение}
\newtheorem{example}[theorem]{Пример}

\newenvironment{proof}{\par $\triangleleft$}{$\triangleright$}

\pagestyle{plain}

\begin{document}

\begin{center}

\Large \textbf{Метрическая и топологическая свобода для секвенциальных операторных пространств}\\[0.5cm]
\small {Норберт Немеш, Сергей Штейнер}\\[0.5cm]

\end{center}
\thispagestyle{empty}

\begin{abstract}
В 2002 году году Ансельм Ламберт в своей диссертации \cite{LamOpFolgen} ввел определение секвенциального операторного пространства и доказал аналоги многих фактов теории операторных пространств. Говоря неформально, категория секвенциальных операторных пространств находится   <<между>> категориями нормированных и операторных пространств. Цель данной статьи --- описание свободных и косвободных объектов для различных версий  гомологии в категории секвенциальных операторных пространств. Сначала мы покажем, что в этой категории теория двойственности во многом аналогична таковой для нормированных пространств. Затем, основываясь на этих результатах, мы дадим полное описание метрически и топологически свободных и косвободных объектов.
\end{abstract}


\section{Секвенциальные операторнве пространства.}


\subsection{Некоторые напоминания.}

Далее все линейные пространства будут рассматриваться над полем комплексных чисел. Через $B_E$ мы будем обозначать замкнутый единичный шар нормированного пространства $E$. Если $E$, $F$ --- два нормированных пространства, то $\mathcal{B}(E, F)$ --- нормированное пространство ограниченных линейных операторов из $E$ в $F$. Для заданного $1\leq p\leq\infty$ через $\bigoplus_p^0\{E_\lambda:\lambda\in\Lambda\}$ мы обозначаем $\bigoplus_p^0$ сумму семейства нормированных пространств $\{E_\lambda:\lambda\in\Lambda\}$. Это нормированное пространство, у которого каждый вектор имеет лишь конечное число ненулевых координат. Аналогично $\bigoplus_p\{E_\lambda:\lambda\in\Lambda\}$ обозначает $\bigoplus_p$-сумму банаховых пространств. Отметим, что $\bigoplus_\infty$-cуммы являются произведениями, а $\bigoplus_1$-суммы --- копроизведениями в категории нормированных простанств. Через $\mathbb{N}_n$ мы будем обозначать множество $\{1,\ldots,n\}$.

\medskip

Пусть $n, k\in\mathbb{N}$, тогда через $M_{n,k}$ мы будем обозначать линейное пространство комплекснозначных матриц размера $n\times k$. Пространство $M_{n,k}$ по умолчанию наделяется операторной нормой $\Vert\cdot\Vert$, но нам также понадобится норма Гильберта-Шмидта. Пусть $\alpha\in M_{n,k}$, тогда норму Гильберта-Шмидта определим равенством $\Vert\alpha\Vert_{hs}=\operatorname{trace}(|\alpha|^2)^{1/2}$
где $|\alpha|=(\alpha^*\alpha)^{1/2}$. Отметим, что всегда выполнены соотношения $\Vert\alpha\Vert\leq\Vert\alpha\Vert_{hs}$ и $\Vert|\alpha|\Vert_{hs}=\Vert|\alpha^*|\Vert=\Vert\alpha\Vert_{hs}$ .

\medskip

Для линейного пространства $E$ через $E^k$ будем обозначать пространство столбцов высоты $k$ с элементами из $E$. Для $\alpha\in M_{n,k}$ и $x\in E^k$ через $\alpha x$ будем обозначать такой столбец из $E^n$, что
$(\alpha x)_i=\sum_{j=1}^n \alpha_{ij} x_j$. Эта  формула является естественным обощением  матричного умножения. Теперь мы готовы дать два основных определения: определение секвенциального операторного пространства и определение секвенциально ограниченного оператора.

\medskip

{\bf Определение 1.1.1}[\cite{LamOpFolgen}, 1.1.7]\label{DefSQSpace} Пусть $E$ --- линейное пространство, и для каждого $n\in\mathbb{N}$ на пространстве $E^n$ задана некоторая норма $\Vert \cdot \Vert_{\wideparen{n}}$. 
Будем говорить, что семейство $X = (E^n, (\Vert \cdot \Vert_{\wideparen{n}})_{n \in \mathbb{N}})$, задаёт на $E$ структуру \textit{секвенциального операторного} пространства, если выполнены следующие условия:

$i)$ $\Vert \alpha x \Vert_{\wideparen{m}} \leq \Vert \alpha \Vert  \Vert x \Vert_{\wideparen{n}}$ для всех $m, n \in \mathbb{N}$, $x \in E^{\wideparen{n}}$, $\alpha \in M_{m, n}$.

$ii)$ $\left\Vert ( x, y)^{tr} \right\Vert^2_{\wideparen{n + m}} \leq   \Vert x \Vert_{\wideparen{n}}^2 + \Vert y \Vert_{\wideparen{m}}^2$ для всех $m, n \in \mathbb{N}$, $x \in E^n$, $y \in E^m$ 
\newline
Пространство $E^n$ с нормой $\Vert \cdot \Vert_{\wideparen{n}}$ будем обозначать через $X^{\wideparen{n}}$.

\medskip

Легко заметить, что если $X$ --- секвенциальное операторное пространство, то каждое 
нормированное пространство $X^{\wideparen{n}}$ наделено естественной структурой секвенциального операторного пространства: достаточно отождествить $(X^{\wideparen{n}})^{\wideparen{k}}$ с $X^{\wideparen{nk}}$. Для любого нормированного пространства $E$ можно задать семейство наименьших или наибольших норм, делающих $E$ секвенциальным операторным пространством [\cite{LamOpFolgen}, 2.1.1, 2.1.2]. Мы обозначим эти пространства $\min(E)$ и $\max(E)$ соответственно. Их нормы задаются равенствами
$$
\Vert x\Vert_{\min(E)^{\wideparen{n}}}=\sup_{\xi\in B_{l_2^n}}\left\Vert\sum\limits_{i=1}^n \xi_i x_i\right\Vert
\qquad\qquad
\Vert x\Vert_{\max(E)^{\wideparen{n}}}=\inf_{x=\alpha\tilde x, \alpha\in M_{n,k}, \tilde{x}\in E^{\wideparen{k}}}\Vert\alpha\Vert_{M_{n,k}}\left(\sum\limits_{i=1}^k\Vert\tilde  x_i\Vert^2\right)^{1/2}
$$
Мы будем использовать обозначения $t_2^n=\min(\mathbb{C}^n)$, $l_2^n=\max(\mathbb{C}^n)$, причем здесь $\mathbb{C}^n$ рассматривается как $n$-мерное гильбертово пространство. Отсюда, кстати, легко видеть, что $\mathcal{C}$ обладает единственной секвенциальной операторной структурой.

\medskip

{\bf Определение 1.1.2}[\cite{LamOpFolgen}, 1.2.1]\label{DefSBOp}
Пусть $X$ и $Y$ ---- секвенциальные операторные проcтранства, а $\varphi : X \to Y$ --- линейный оператор. Его \textit{размножением} называется семейство операторов $\varphi^{\wideparen{n}} : X^{\wideparen{n}} \to Y^{\wideparen{n}}$, 
 $n\in\mathbb{N}$, определённых равенством $\varphi^{\wideparen{n}}(x)=(\varphi(x_i))_{i\in\mathbb{N}_k}$. Будем называть оператор $\varphi$ \textit{секвенциально ограниченным}, если 
$$
\Vert \varphi \Vert_{sb} := \sup\{\Vert \varphi^{\wideparen{n}}\Vert_{\mathcal{B}(X^{\wideparen{n}},Y^{\wideparen{n}})}:n\in\mathbb{N}\}  < \infty
$$

\medskip

Множество секвенциально ограниченных операторов между секвенциальными операторными пространствами $X$ и $Y$ будем обозначать через $\mathcal{SB}(X,Y)$. Это линейное подпространство в $\mathcal{B}(X,Y)$, которое также можно наделить структурой секвенциального операторного пространства [\cite{LamOpFolgen}, 1.2.7] посредством отождествления $\mathcal{SB}(X,Y)^{\wideparen{n}}=\mathcal{SB}(X, Y^{\wideparen{n}})$. Теперь мы можем ввести две категории секвенциальных операторных пространств: $SQNor$ и $SQNor_1$. Объекты обеих категорий --- секвенциальные операторные пространства. Морфизмы в $SQNor$ --- секвенциально ограниченные операторы, а в $SQNor_1$ --- секвенциально ограниченные операторы с $sb$-нормой, не превосходящей $1$. 

\medskip

Теперь легко проверить что $\mathcal{SB}(-,-):SQNor\times SQNor\to SQNor$ задает бифунктор, ковариантный по первому аргументу и контравариантный по второму. Как и в случае нормированных пространств, логично рассмотреть действие этого функтора с пространством $\mathbb{C}$ в качестве второго аргумента. Мы получим функтор ${}^\triangle=\mathcal{SB}(-,\mathbb{C})$, который логично называть функтором сопряжения для секвенциальных операторных пространств. Он действительно  ведет себя подобно функтору банаховой сопряженности [\cite{LamOpFolgen}, 1.3]. Категория $SQNor_1$  (как и категория операторных пространств с вполне сжимающими операторами в качестве морфизмов) обладает категорными произведениями и копроизведениями. 

\medskip

{\bf Определение 1.1.3}[\cite{LamOpFolgen}, 1.1.28]\label{DefSQProd}
Пусть $\{X_\lambda: \lambda \in \Lambda\}$ --- произвольное семейство секвенциальных операторных пространств. Их $\bigoplus_\infty$-суммой называется секвенциальное операторное пространство 
$\bigoplus_\infty\{X_\lambda^{\wideparen{1}}:\lambda\in \Lambda\}$, с семейством норм, задаваемых отождествлениями 
$$
\left(\bigoplus{}_\infty\{X_\lambda:\lambda \in \Lambda\}\right)^{\wideparen{n}}
=\bigoplus{}_\infty\{X_\lambda^{\wideparen{n}}:\lambda\in \Lambda\}
$$

\medskip

{\bf Определение 1.1.4}\label{DefSQCoProd}
Пусть $\{X_\lambda: \lambda \in \Lambda\}$ --- произвольное семейство секвенциальных операторных пространств. Их $\bigoplus_1^0$-суммой называется секвенциальное операторное пространство  
$\bigoplus_1^0\{X_\lambda^{\wideparen{1}}:\lambda\in \Lambda\}$, с нормами, индуцированными вложением
$$
\bigoplus{}_1^0\{X_\lambda:\lambda \in \Lambda\}\hookrightarrow
\left(\bigoplus{}_\infty\{X_\lambda^\triangle:\lambda\in \Lambda\}\right)^\triangle
$$

\medskip

Как и в случае операторных пространств, легко показать, что $\bigoplus_\infty$-суммы являются произведениями, а $\bigoplus_1^0$-суммы --- копроизведениями в $SQNor_1$. Более того, имеет место изоморфизм в $SQNor_1$:
$$
\left(\bigoplus{}_1^0\{X_\lambda:\lambda\in \Lambda\}\right)^\triangle
=\bigoplus{}_\infty\{X_\lambda^\triangle:\lambda\in \Lambda\}
$$


\subsection{Двойственность для секвенциально ограниченных операторов}

Основные результаты этого раздела получены Н. Немешем. Для начала нам нужно напомнить некоторые определения и факты, касающиеся ограниченных линейных операторов.

{\bf Определение 1.2.1}\label{DefNorOpType} Пусть $ T:E\to F$ --- ограниченный линейный оператор между нормированными пространствами, тогда $ T$ называется

$i)$ $c$-топологически инъективным, если для каждого $x \in E$ выполнено $\Vert x\Vert\leq c\Vert  T(x)\Vert$. Если упоминание константы $c$ не нужно, будем говорить, что $ T$ топологически инъективен.

$ii)$ (строго) $c$-топологически сюрьективным, если любого $c'>c$ и любого $y\in F$ существует такой $x \in E$, что ($\Vert x \Vert \leq c \Vert y \Vert$) $\Vert x \Vert < c' \Vert y \Vert$ и $ T(x) = y$. 
Если упоминание константы $c$ не нужно то будем говорить, что $ T$ (строго) топологически сюръективен.

$iii)$ (строго) коизометрическим, если он сжимающий и (строго) $1$-топологически сюръективный

\medskip

{\bf Предложение 1.2.2}\label{PrDualOps} Пусть $ T:E\to F$ ограниченный оператор между нормированными пространствами и $c>0$, тогда

$i)$ если $ T$ (строго) $c$-топологически сюръективен, то $ T^*$ $c$-топологически инъективен

$ii)$ если $ T$ $c$-топологически инъективен, то $ T^*$ строго $c$-топологически сюръективен

$iii)$ если $ T^*$ (строго) $c$-топологически сюръективен, то $ T$ $c$-топологически инъективен

$iv)$ если $ T^*$ $c$-топологически инъективен и $E$ полно, то $ T$ $c$-топологически сюръективен

\medskip

Аналогичные определения можно дать и для секвенциально ограниченных операторов. Например, оператор $\varphi\in\mathcal{SB}(X,Y)$ между секвенциальными операторными пространствами $X$ и $Y$ называется секвенциально $c$-топологически инъективным, если для любого $n\in\mathbb{N}$ оператор $\varphi^{\wideparen{n}}$ $c$-топологически инъективен.

\medskip

Далее мы докажем несколько технических предложений, необходимых для описания двойственности между секвенциально ограниченными операторами.

\medskip

{\bf Предложение 1.2.3}[\cite{LamOpFolgen}, 1.3.14]\label{PrDualSBOp}
Пусть $X$, $Y$ --- секвенциальные операторные пространства и $\varphi\in \mathcal{SB}(X,Y)$. Тогда $\varphi^\triangle \in\mathcal{SB}(Y^\triangle ,X^\triangle )$ и для каждого  $n\in\mathbb{N}$ выполнено 
$\Vert(\varphi^\triangle )^{\wideparen{n}}\Vert=\Vert\varphi^{\wideparen{n}}\Vert$. Как следствие, $\Vert\varphi^\triangle \Vert_{sb}=\Vert\varphi\Vert_{sb}$.

\medskip

{\bf Определение 1.2.4}[\cite{LamOpFolgen}, 1.3.15]\label{DefT2n}
Пусть $X$ --- секвенциальное операторное пространство и $n\in\mathbb{N}$, тогда через $t_2^n(X)$ будем обозначть нормированное пространство $X^n$ c нормой
$$
\Vert x\Vert_{t_2^n(X)}:=\inf\left\{\Vert\tilde{\alpha}\Vert_{hs}\Vert \tilde{x}\Vert_{\wideparen{k}}:x=\tilde{\alpha} \tilde{x}\right\}
$$
где $\tilde{\alpha}\in M_{n,k}$, $x\in X^k$ и $k\in\mathbb{N}$. Если $Y$ --- секвенциальное операторное пространство, и $\varphi\in\mathcal{SB}(X,Y)$, то через $t_2^n(\varphi)$ будем обозначать линейный оператор
$$
t_2^n(\varphi): t_2^n(X)\to t_2^n(Y): x\mapsto \varphi^{\wideparen{n}}(x)
$$

\medskip

{\bf Предложение 1.2.5}\label{PrT2nNormProperty}
Пусть $X$ --- секвенциальное операторное пространство и $n\in\mathbb{N}$, тогда
$$
\Vert x\Vert_{t_2^n(X)}=\inf\left\{\Vert\alpha'\Vert_{hs}\Vert x'\Vert_{\wideparen{k}}:x=\alpha'x'\right\}
$$
где $\alpha'\in M_{n,n}$ --- обратимая матрица, $x'\in X^{n}$.


{\bf Доказательство.} Обозначим правую часть доказываемого равенства через $\Vert x\Vert_{t_2^n(X)}'$. Фиксируем $\varepsilon>0$, тогда существуют $\tilde{\alpha}\in M_{n,k}$ и $\tilde{x}\in X^{k}$, $k\in\mathbb{N}$ такие, что 
$x=\tilde{\alpha}\tilde{x}$ и $\Vert\tilde{\alpha}\Vert_{hs}\Vert\tilde{x}\Vert_{\wideparen{k}}<\Vert x\Vert_{t_2^n(X)}+\varepsilon$. Рассмотрим полярное разложение 
$\tilde{\alpha}=|\tilde{\alpha}^*| \rho$ матрицы $\tilde{\alpha}$. Пусть $p$ --- ортогональный проектор на $\operatorname{Im}(|\tilde{\alpha}^*|)^\perp$. Тогда для любого $\delta\in\mathbb{R}$ матрица 
$\alpha_\delta'=|\tilde{\alpha}^*|+\delta p$ обратима так как $\operatorname{Ker}(\alpha_\delta')=\{0\}$. Так как $\alpha'_0=|\tilde{\alpha}|$ и функция $\Vert\alpha_\delta'\Vert_{hs}$ непрерывна при 
$\delta\in\mathbb{R}$, то существует такое значение $\delta_0$, что 
$\Vert\alpha_{\delta_0}'\Vert_{hs}<\Vert|\tilde{\alpha}^*|\Vert_{hs}+\varepsilon\Vert \tilde{x}\Vert_{\wideparen{k}}^{-1}=\Vert\tilde{\alpha}\Vert_{hs}+\varepsilon\Vert \tilde{x}\Vert_{\wideparen{k}}^{-1}$. 
Обозначим $\alpha'=\alpha_{\delta_0}'\in M_{n,n}$ и $x'=\rho\tilde{x}\in Y^n$, тогда 
$$
\alpha'x'
=(|\tilde{\alpha}^*|+\delta_0 p)\rho \tilde{x}
=|\tilde{\alpha}^*|\rho \tilde{x}+\delta_0 p\rho \tilde{x}
=\tilde{\alpha}\tilde{x}
$$
По построению полярного разложения $\Vert \rho\Vert\leq 1$, поэтому с учетом определения $\Vert x\Vert_{t_2^n(X)}'$ получаем
$$
\Vert x\Vert_{t_2^n(X)}'\leq
\Vert\alpha'\Vert_{hs}\Vert x'\Vert_{\wideparen{n}}
\leq (\Vert\tilde{\alpha}\Vert_{hs}+\varepsilon\Vert \tilde{x}\Vert_{\wideparen{k}})\Vert \rho\Vert\Vert\tilde{x}\Vert_{\wideparen{n}}
\leq\Vert\tilde{\alpha}\Vert_{hs}\Vert\tilde{x}\Vert_{\wideparen{k}}+\varepsilon
\leq \Vert x\Vert_{t_2^n(X)}+2\varepsilon
$$
Так как $\varepsilon>0$ произвольно, то $\Vert x\Vert_{t_2^n(X)}'\leq\Vert x\Vert_{t_2^n(X)}$. Обратное неравенство очевидно, поэтому $\Vert x\Vert_{t_2^n(X)}=\Vert x\Vert_{t_2^n(X)}'$.

\medskip

{\bf Предложение 1.2.5}\label{PrT2nOfOpIsWellDef}
Пусть $X$, $Y$ --- секвенциальные операторные пространства, $\varphi\in\mathcal{SB}(X,Y)$ и $n,k\in\mathbb{N}$. Тогда 

$i)$ Для любых $\alpha\in M_{n,k}$ и $x\in t_2^k(X)$ выполнено $t_2^n(\varphi)(\alpha x)=\alpha t_2^k(\varphi)(x)$

$ii)$ $t_2^n(\varphi)\in\mathcal{B}(t_2^n(X),t_2^n(Y))$, причем $\Vert t_2^n(\varphi)\Vert\leq\Vert\varphi^{\wideparen{n}}\Vert$

$iii)$ если $\varphi^{\wideparen{n}}$ (строго) $c$-топологически сюръективно, то $t_2^n(\varphi)$ так же (строго) $c$-топологически сюръективно

$iv)$  если $\varphi^{\wideparen{n}}$ $c$-топологически инъективно, то $t_2^n(\varphi)$ так же $c$-топологически инъективно


{\bf Доказательство.}
$i)$ Проверяется непосредственно. 

$ii)$ Пусть $x\in t_2^n(X)$ и $x=\alpha'x'$, где $\alpha\in M_{n,n}$ --- обратимая матрица и $x'\in X^{n}$, тогда $t_2^n(\varphi)(x)=\alpha't_2^n(\varphi)(x')=\alpha'\varphi^{\wideparen{n}}(x')$, поэтому из 
определения нормы в $t_2^n(Y)$ следует, что
$$
\Vert t_2^n(\varphi)(x)\Vert_{t_2^n(Y)}
\leq\Vert\alpha'\Vert_{hs}\Vert\varphi^{\wideparen{n}}(x')\Vert_{\wideparen{n}}
\leq\Vert\alpha'\Vert_{hs}\Vert\varphi^{\wideparen{n}}\Vert\Vert x'\vert_{\wideparen{n}}
$$
Теперь возьмем инфимум по всем представлениям $x$ описанным выше, тогда предложение \ref{PrT2nNormProperty} дает
$$
\Vert t_2^n(\varphi)(x)\Vert_{t_2^n(Y)}\leq\Vert\varphi^{\wideparen{n}}\Vert\Vert x\Vert_{t_2^n(X)}
$$
Следовательно $\Vert t_2^n(\varphi)\Vert\leq\Vert\varphi^{\wideparen{n}}\Vert$ и $t_2^n(\varphi)\in\mathcal{B}(t_2^n(X),t_2^n(Y))$.

$iii)$ Пусть $\varphi^{\wideparen{n}}$  $c$-топологически сюръективен. Пусть $y\in t_2^n(Y)$ и $y=\alpha' y'$, где $\alpha'\in M_{n,n}$ --- обратимая матрица, $y'\in Y^n$. Пусть $c<c''<c'$. Так как 
$\varphi^{\wideparen{n}}$ $c$-топологически сюръективно, то существует $x'\in X^n$ такое что $\varphi^{\wideparen{n}}(x')=y'$ и $\Vert x'\Vert_{\wideparen{n}}< c''\Vert y'\Vert_{\wideparen{n}}$. Рассмотрим 
$x:=\alpha'x'$, тогда $t_2^n(\varphi)(x)=\alpha't_2^n(\varphi)(x')=\alpha'\varphi^{\wideparen{n}}(x')=\alpha' y'=y$. Из определения нормы в $t_2^n(X)$ получаем
$$
\Vert x\Vert_{t_2^n(X)}
\leq\Vert\alpha'\Vert_{hs}\Vert x'\Vert_{\wideparen{n}}
\leq\Vert\alpha'\Vert_{hs} c''\Vert y'\Vert_{\wideparen{n}}
$$
Теперь возьмем инфимум по всем представлениям $y$ описанным выше, тогда предложение \ref{PrT2nNormProperty} дает $\Vert x\Vert_{t_2^n(X)}\leq c''\Vert y\Vert_{t_2^n(Y)}<c'\Vert y\Vert_{t_2^n(Y)}$
Таким образом, для любого $y\in t_2^n(Y)$ и любого $c'>c$ существует $x\in t_2^n(X)$ такой что $t_2^n(\varphi)(x)=y$ и $\Vert x\Vert_{t_2^n(X)}< c'\Vert y\Vert_{t_2^n(Y)}$. Следовательно $t_2^n(\varphi)$ 
$c$-топологически сюръективен.
\newline
Пусть $\varphi^{\wideparen{n}}$ строго $c$-топологически сюръективен. Пусть $y\in t_2^n(Y)$ и $y=\alpha' y'$, где $\alpha'\in M_{n,n}$ --- обратимая матрица, $y'\in Y^n$. Так как $\varphi^{\wideparen{n}}$ $c$-топологически 
сюръективно, то существует $x'\in X^n$ такое что $\varphi^{\wideparen{n}}(x')=y'$ и $\Vert x'\Vert_{\wideparen{n}}\leq c\Vert y'\Vert_{\wideparen{n}}$. Рассмотрим $x:=\alpha'x'$, тогда 
$t_2^n(\varphi)(x)=\alpha't_2^n(\varphi)(x')=\alpha'\varphi^{\wideparen{n}}(x')=\alpha' y'=y$. Из определения нормы в $t_2^n(X)$ получаем
$$
\Vert x\Vert_{t_2^n(X)}
\leq\Vert\alpha'\Vert_{hs}\Vert x'\Vert_{\wideparen{n}}
\leq\Vert\alpha'\Vert_{hs} c\Vert y'\Vert_{\wideparen{n}}
$$
Теперь возьмем инфимум по всем представлениям $y$, описанным выше, тогда предложение \ref{PrT2nNormProperty} дает $\Vert x\Vert_{t_2^n(X)}\leq c\Vert y\Vert_{t_2^n(Y)}$
Таким образом, для любого $y\in t_2^n(Y)$ существует $x\in t_2^n(X)$ такой что $t_2^n(\varphi)(x)=y$ и $\Vert x\Vert_{t_2^n(X)}\leq c\Vert y\Vert_{t_2^n(Y)}$. Следовательно $t_2^n(\varphi)$ строго $c$-топологически сюръективен.

$iv)$ Пусть $x\in t_2^n(X)$, обозначим $y:=t_2^n(\varphi)(x)$. Пусть имеется представление $y=\alpha' y'$, где $\alpha'\in M_{n,n}$ --- обратимая матрица, $y'\in Y^n$. Тогда 
$y'=(\alpha')^{-1}y=(\alpha')^{-1}t_2^n(\varphi)(x)=t_2^n(\varphi)((\alpha')^{-1}x)\in\operatorname{Im}(t_n^2(\varphi))
$. Так как $\varphi^{\wideparen{n}}$ $c$-топологически инъективен, то он инъективен, поэтому для $y'\in \operatorname{Im}(t_2^n(\varphi))$ существует $x'\in X^n$ такой что 
$y'=t_2^n(\varphi)(x')=\varphi^{\wideparen{n}}(x')$. Так как $\varphi^{\wideparen{n}}$ $c$-топологически инъективен, то $\Vert x'\Vert_{\wideparen{n}}\leq c\Vert y'\Vert$. Из определения нормы в $t_2^n(X)$ следует, что
$$
\Vert x\Vert_{t_2^n(X)}\leq\Vert\alpha'\Vert_{hs}\Vert x'\Vert_{\wideparen{n}}\leq c\Vert\alpha'\Vert_{hs}\Vert y'\Vert_{\wideparen{n}}
$$
Теперь возьмем инфимум по всем представлениям $y$, описанным выше, тогда предложение \ref{PrT2nNormProperty} дает $\Vert x\Vert_{t_2^n(X)}\leq c\Vert y\Vert_{t_2^n(Y)}=c\Vert t_2^n(\varphi)(x)\Vert_{t_2^n(Y)}$. 
Таким образом, для любого $x\in t_2^n(X)$ выполнено $\Vert t_2^n(\varphi)(x)\Vert_{t_2^n(Y)}\geq c^{-1}\Vert x\Vert_{t_2^n(X)}$. Следовательно, $t_2^n(\varphi)$ $c$-топологически инъективен.

\medskip

{\bf Предложение 1.2.6}[\cite{LamOpFolgen}, 1.3.16]\label{PrT2nTraingDuality}
Пусть $X$ --- секвенциальное операторное пространство и $n\in\mathbb{N}$. Тогда имеют место изометрические изоморфизмы
$$
\alpha_X^n:t_2^n(X^\triangle)\to (X^{\wideparen{n}})^*: f\mapsto\left(x\mapsto\sum\limits_{i=1}^n f_i(x_i)\right)
\qquad
\beta_X^n:(X^\triangle)^{\wideparen{n}}\to t_2^n(X)^*:f\mapsto\left(x\mapsto\sum\limits_{i=1}^n f_i(x_i)\right)
$$

\medskip

{\bf Предложение 1.2.7}\label{PrTwoTypesDualOpEquiv}
Пусть $X$, $Y$ --- секвенциальные операторные пространства, $\varphi\in \mathcal{SB}(X,Y)$ и $n\in\mathbb{N}$, тогда 

$i)$ $(\varphi^\triangle)^{\wideparen{n}}$ $c$-топологически инъективен (сюръективен) тогда и только тогда когда $t_2^n(\varphi)^*$ $c$-топологически инъективен (сюръективен)

$ii)$ $t_2^n(\varphi^\triangle)$ $c$-топологически инъективен (сюръективен) тогда и только тогда когда $(\varphi^{\wideparen{n}})^*$ $c$-топологически инъективен (сюръективен)

$iii)$ верны равенства $\Vert (\varphi^\triangle)^{\wideparen{n}}\Vert=\Vert t_2^n(\varphi)^*\Vert$ и $\Vert t_2^n(\varphi^\triangle)\Vert=\Vert (\varphi^{\wideparen{n}})^*\Vert$ и  $\Vert t_2^n(\varphi)\Vert=\Vert\varphi^{\wideparen{n}}\Vert$

{\bf Доказательство.}
Пусть $g\in (Y^\triangle)^{\wideparen{n}}$ и $x\in t_2^n(X)$, тогда
$$
(\alpha_X^n(\varphi^\triangle)^{\wideparen{n}})(g)(x)
=\alpha_X^n((\varphi^\triangle)^{\wideparen{n}}(g))(x)
=\sum\limits_{k=1}^n (\varphi^\triangle)^{\wideparen{n}}(g)_k(x_k)
=\sum\limits_{k=1}^n (\varphi^\triangle)(g_k)(x_k)
=\sum\limits_{k=1}^n g_k(\varphi(x_k))
$$
$$
(t_2^n(\varphi)^* \alpha_Y^n)(g)(x)
=t_2^n(\varphi)^*(\alpha_Y^n(g))(x)
=\alpha_Y^n(g)(t_2^n(\varphi)(x))
=\sum\limits_{k=1}^n g_k(t_2^n(\varphi)(x)_k)
=\sum\limits_{k=1}^n g_k(\varphi(x_k))
$$
Так как $g$ и $x$ произвольны, то $\alpha_X^n(\varphi^\triangle)^{\wideparen{n}}=t_2^n(\varphi)^* \alpha_Y^n$. Так как $\alpha_Y^n$ и $\alpha_X^n$ изометрические изоморфизмы, то мы получаем утверждение $i)$ и равенство $\Vert (\varphi^\triangle)^{\wideparen{n}}\Vert=\Vert t_2^n(\varphi)^*\Vert$.
Пусть $g\in t_2^n(Y^\triangle)$ и $x\in X^{\wideparen{n}}$, тогда
$$
(\beta_X^n t_2^n(\varphi^\triangle))(g)(x)
=\beta_X^n(t_2^n(\varphi^\triangle)(g))(x)
=\sum\limits_{k=1}^n t_2^n(\varphi^\triangle)(g)_k(x_k)
=\sum\limits_{k=1}^n (\varphi^\triangle)(g_k)(x_k)
=\sum\limits_{k=1}^n g_k(\varphi(x_k))
$$
$$
((\varphi^{\wideparen{n}})^*\beta_Y^n)(g)(x)
=(\varphi^{\wideparen{n}})^*(\beta_Y^n(g))(x)
=\beta_Y^n(g)(\varphi^{\wideparen{n}})(x))
=\sum\limits_{k=1}^n g_k(\varphi^{\wideparen{n}})(x)_k)
=\sum\limits_{k=1}^n g_k(\varphi(x_k))
$$
Так как $g$ и $x$ произвольны, то $\beta_X^n t_2^n(\varphi^\triangle)=(\varphi^{\wideparen{n}})^*\beta_Y^n$. Так как $\beta_Y^n$ и $\beta_X^n$ изометрические изоморфизмы, то мы получаем утверждение $ii)$ и равенство $\Vert t_2^n(\varphi^\triangle)\Vert=\Vert (\varphi^{\wideparen{n}})^*\Vert$.

Наконец, из предложений \ref{PrT2nOfOpIsWellDef}, \ref{PrDualSBOp} следует что $\Vert t_2^n(\varphi)\Vert\leq\Vert\varphi^{\wideparen{n}}\Vert=\Vert(\varphi^\triangle)^{\wideparen{n}}\Vert=\Vert t_2^n(\varphi)^*\Vert=\Vert t_2^n(\varphi)\Vert$, т.е. $\Vert t_2^n(\varphi)\Vert=\Vert\varphi^{\wideparen{n}}\Vert$.



{\bf Теорема 1.2.8}\label{ThDualSQOps}
Пусть $X$, $Y$ --- секвенциальные операторные пространства и $\varphi\in\mathcal{SB}(X,Y)$, тогда

$i)$ $\varphi$ (строго) секвенциально $c$-топологически сюръективен $\Longrightarrow$
$ \varphi^\triangle$ секвенциально $c$-топологически инъективен

$ii)$ $\varphi$ секвенциально $c$-топологически инъективен $\Longrightarrow$ строго
$ \varphi^\triangle$ строго секвенциально $c$-топологически сюръективен

$iii)$ $\varphi^\triangle$ (строго) секвенциально $c$-топологически сюръективен $\Longrightarrow$
$ \varphi$ секвенциально $c$-топологически инъективен

$iv)$ $\varphi^\triangle$ секвенциально $c$-топологически инъективен $\Longrightarrow$
$ \varphi$ строго секвенциально $c$-топологически сюръективен

$v)$ $\varphi$ секвенциально коизометричен $\Longrightarrow$ 
$\varphi^\triangle$ секвенциально изометричен, если $X$ полно, то верно и обратное

$vi)$ $ \varphi$ секвенциально изометричен $\Longleftrightarrow$ 
$\varphi^\triangle$ секвенциально строго коизометричен


{\bf Доказательство.}
Для каждого натурального числа $n\in\mathbb{N}$ имеем цепочку импликаций
\newline
{\small
\begin{tabular}{llllll}
$\varphi^{\wideparen{n}}$ & $c$-топологически инъективен & $\implies$ & $t_2^n(\varphi)$                    & $c$-топологически инъективен       &\ref{PrT2nOfOpIsWellDef}\\
                        &                              & $\implies$ & $t_2^n(\varphi)^*$                  & строго $c$-топологически сюръективен      &\ref{PrDualOps}\\
                        &                              & $\implies$ & $(\varphi^\triangle)^{\wideparen{n}}$ & строго $c$-топологически сюръективен &\ref{PrTwoTypesDualOpEquiv}\\
                        &                              & $\implies$ & $t_2^n(\varphi^\triangle)$          & строго $c$-топологически сюръективен &\ref{PrT2nOfOpIsWellDef}\\
                        &                              & $\implies$ & $(\varphi^{\wideparen{n}})^*$         & строго $c$-топологически сюръективен &\ref{PrTwoTypesDualOpEquiv}\\
                        &                              & $\implies$ & $\varphi^{\wideparen{n}}$             & $c$-топологически инъективен       &\ref{PrDualOps}\\
\end{tabular}
}
\newline
Откуда мы получаем $ii)$ и $iii)$. Снова для любого $n\in\mathbb{N}$ мы имеем цепочку ипликаций
{\small
\begin{tabular}{llclll}
$\varphi^{\wideparen{n}}$ & \begin{tabular}{c}(строго) $c$-топологически\\ сюръективен\end{tabular}  & $\implies$ & $t_2^n(\varphi)$                    & $c$-топологически сюръективен     &\ref{PrT2nOfOpIsWellDef}\\
                        &                               & $\implies$ & $t_2^n(\varphi)^*$                  & $c$-топологически инъективен      &\ref{PrDualOps}\\
                        &                               & $\implies$ & $(\varphi^\triangle)^{\wideparen{n}}$ & $c$-топологически инъективен &\ref{PrTwoTypesDualOpEquiv}\\
                        &                               & $\implies$ & $t_2^n(\varphi^\triangle)$          & $c$-топологически инъективен &\ref{PrT2nOfOpIsWellDef}\\
                        &                               & $\implies$ & $(\varphi^{\wideparen{n}})^*$         & $c$-топологически инъективен &\ref{PrTwoTypesDualOpEquiv}\\
                        &                               & $\overset{\mbox{$X$ полно}}{\implies}$ & $\varphi^{\wideparen{n}}$             & $c$-топологически сюръективен     &\ref{PrDualOps}\\
\end{tabular}
}
\newline
Откуда мы получаем $i)$ и $iv)$. Пункты $v)$ и $vi)$ являются прямым следствием $i)$---$iv)$ при $c=1$ если учесть что $\varphi$ секвенциально сжимающий тогда и только тогда $\varphi^\triangle$ секвенцильно сжимающий (см. предложение \ref{PrDualSBOp}).

%%%%%%%%%%%%%%%%%%%%%%%%%%%%%%%%%%%%%%%%%%%%%%%%%%%%%%%%%%%%%%%%%%%%%%%%%%%%%%%%%%%%%%%%%%
%%%%%%%%%%%%%%%%%%%%%%%%%%%%%%%%%%%%%%%%%%%%%%%%%%%%%%%%%%%%%%%%%%%%%%%%%%%%%%%%%%%%%%%%%%
%%%%%%%%%%%%%%%%%%%%%%%%%%%%%%%%%%%%%%%%%%%%%%%%%%%%%%%%%%%%%%%%%%%%%%%%%%%%%%%%%%%%%%%%%%
%%%%%%%%%%%%%%%%%%%%%%%%%%%%%%%%%%%%%%%%%%%%%%%%%%%%%%%%%%%%%%%%%%%%%%%%%%%%%%%%%%%%%%%%%%
%%%%%%%%%%%%%%%%%%%%%%%%%%%%%%%%%%%%%%%%%%%%%%%%%%%%%%%%%%%%%%%%%%%%%%%%%%%%%%%%%%%%%%%%%%
%%%%%%%%%%%%%%%%%%%%%%%%%%%%%%%%%%%%%%%%%%%%%%%%%%%%%%%%%%%%%%%%%%%%%%%%%%%%%%%%%%%%%%%%%%
%%%%%%%%%%%%%%%%%%%%%%%%%%%%%%%%%%%%%%%%%%%%%%%%%%%%%%%%%%%%%%%%%%%%%%%%%%%%%%%%%%%%%%%%%%
%%%%%%%%%%%%%%%%%%%%%%%%%%%%%%%%%%%%%%%%%%%%%%%%%%%%%%%%%%%%%%%%%%%%%%%%%%%%%%%%%%%%%%%%%%
%%%%%%%%%%%%%%%%%%%%%%%%%%%%%%%%%%%%%%%%%%%%%%%%%%%%%%%%%%%%%%%%%%%%%%%%%%%%%%%%%%%%%%%%%%
%%%%%%%%%%%%%%%%%%%%%%%%%%%%%%%%%%%%%%%%%%%%%%%%%%%%%%%%%%%%%%%%%%%%%%%%%%%%%%%%%%%%%%%%%%
%%%%%%%%%%%%%%%%%%%%%%%%%%%%%%%%%%%%%%%%%%%%%%%%%%%%%%%%%%%%%%%%%%%%%%%%%%%%%%%%%%%%%%%%%%
%%%%%%%%%%%%%%%%%%%%%%%%%%%%%%%%%%%%%%%%%%%%%%%%%%%%%%%%%%%%%%%%%%%%%%%%%%%%%%%%%%%%%%%%%%
%%%%%%%%%%%%%%%%%%%%%%%%%%%%%%%%%%%%%%%%%%%%%%%%%%%%%%%%%%%%%%%%%%%%%%%%%%%%%%%%%%%%%%%%%%
%%%%%%%%%%%%%%%%%%%%%%%%%%%%%%%%%%%%%%%%%%%%%%%%%%%%%%%%%%%%%%%%%%%%%%%%%%%%%%%%%%%%%%%%%%



\section{Свободные и косвободные объекты}

Основные результаты этого раздела получены С. Штейнером. Все необходимые определения, связанные с общекатегорным подходом к проективности, можно найти в работе \cite{HelMetrFrQmod}. Категория полулинейных нормированных пространств описана в \cite{ShteinerTopFr}.

\subsection{Метрически свободные секвенциальные пространства}

Начнём с рассмотрения метрической версии свободы для секвенциальных операторных пространств. Рассмотрим функтор 
$$
\begin{aligned}
\square_{sqMet} : SQNor_1 \to Set : X&\mapsto\prod\left\{ B_{X^{\wideparen{n}}}:n \in \mathbb{N}\right\}\\
\varphi&\mapsto \prod\left\{\varphi^{\wideparen{n}}|_{B_{X^{\wideparen{n}}}}^{B_{Y^{\wideparen{n}}}}:n\in\mathbb{N}\right\}
\end{aligned}
$$
отправляющий  секвенциальное операторное пространство $X$ в декартово произведение единичных шаров каждого из пространств $X^{\wideparen{n}}$. Легко заметить, что справедливо

{\bf Предложение 2.1.1} $\square_{sqMet}$-допустимыми эпиморфизмами являются в точности секвенциально строго коизометрические операторы.

\medskip


\textit{Метрически свободными} секвенциальными пространствами естественно называть $\square_{sqMet}$-свободные объекты. Обозначим через $I_n$ элемент из $(t_2^n)^{\wideparen{n}} = \mathcal{B}(l_2^n, l_2^n)$, соответствующий тождественному оператору.

\medskip

{\bf Предложение 2.1.2} Пусть $X$ --- произвольное секвенциальное операторное пространство и $x \in B_{X^{\wideparen{n}}}$. Тогда существует единственный секвенциально сжимающий оператор 
$\psi_n \in \mathcal{SB}(t_2^n, X)$, такой что $\psi_n^{\wideparen{n}}(I_n) = x$.

{\bf Доказательство.} Итак, $I_n = (e_i)_{i\in\mathbb{N}_n}$, где $e_i$ - $i$-й орт подлежащего пространства $t_2^n$. Ясно, что есть только один линейный оператор $\psi_n$, удовлетворяющий условиям $\psi_n(e_i) = x_i$, $i\in\mathbb{N}_n$. 
Осталось проверить, что $\psi_n$ является секвенциально сжимающим. Итак, пусть $k \in \mathbb{N}$ и $y \in B_{(t_2^n)^{\wideparen{k}}} $, тогда $y_i = \sum\limits_{j = 1}^n \alpha_{ij}e_j$, $i\in\mathbb{N}_k$ 
для некоторой матрицы $\alpha\in M_{k,n}$. Тогда 
$$
\Vert\psi_n^{\wideparen{k}}(y)\Vert_{\wideparen{k}}
=\left\Vert\left(\psi_n(y_i)\right)_{i\in\mathbb{N}_k}\right\Vert_{\wideparen{k}}
=\left\Vert\left(\sum\limits_{j=1}^n\alpha_{ij}\psi_n(e_j)\right)_{i\in\mathbb{N}_k}\right\Vert_{\wideparen{k}}
=\left\Vert\left(\sum\limits_{j=1}^n\alpha_{ij}x_j\right)_{i\in\mathbb{N}_k}\right\Vert_{\wideparen{k}}
$$
$$
=\Vert\alpha x\Vert_{\wideparen{k}}
\leq\Vert\alpha\Vert\Vert x\Vert_{\wideparen{n}}
=\Vert y\Vert_{(t_2^n)^{\wideparen{k}}}\Vert x\Vert_{\wideparen{n}}\leq 1
$$
Предложение доказано.

\medskip

{\bf Предложение 2.1.3} Метрически свободным секвенциальным операторным пространством с базой из одноточечного множества является пространство $t_2^{\infty} := \bigoplus_1^0 \{t_2^n: n \in \mathbb{N}\}$.

{\bf Доказательство.} Универсальную стрелку определим следующим образом $j:\{\lambda\}\to t_2^\infty:\lambda\mapsto(I_1,I_2,\ldots,I_n,\ldots)$. Пусть $X$ - произвольное секвенциальное операторное пространство, и 
$\varphi:\{\lambda\}\to \prod_{n \in \mathbb{N}} B_{X^{\wideparen{n}}}$. Обозначим $x=\varphi(\lambda)$. Тогда из предложения \ref{PrMetrFrLem} и свойств копроизведения ясно, что существует 
единственный секвенциально сжимающий морфизм $\psi=\bigoplus_1^0\{\psi_n:n\in\mathbb{N}\}\in\mathcal{SB}\left(\bigoplus_1^0\{ t_2^n:n\in\mathbb{N}\}, X\right)$, такой что $\psi^{\wideparen{n}}(i_n(I_n)) = x$, 
для всех $n \in \mathbb{N}$. Здесь $i_n:t_2^n\to t_2^\infty$ --- стандартное вложение.
$$
\xymatrix{
{\square_{sqMet} (t_2^\infty)} \ar@{-->}[dr]^{\square_{sqMet} (\psi)} & \\
{\{\lambda\}} \ar[u]^{j} \ar[r]^{\varphi} &{\square_{sqMet} (X)}}
$$
В этом случае $\varphi=\square_{sqMet}(\psi) j$. Так как $X$ и $\varphi$ произвольны то $t_2^\infty$ метрически свободен и имеет одноточечную базу. 

\medskip

Итак, теперь мы готовы сформулировать итоговый результат, справедливость которого мгновенно вытекает из доказанного выше предложения.

{\bf Теорема 2.1.4} Метрически свободным секвенциальным операторным пространством с базой $\Lambda$ является, с точностью до секвенциального изометрического изоморфизма, 
$\bigoplus{}_1^0$-сумма копий пространства $t_2^{\infty}$, заиндексированных элементами множества $\Lambda$. 

\subsection{Топологически свободные секвенциальные пространства}

Перейдём теперь к рассмотрению секвенциальной операторной версии  топологической свободы. Рассмотрим функтор 
$$
\begin{aligned}
\square_{sqTop} : SQNor \to Nor_0: X &\mapsto \bigoplus{}_\infty \{X^{\wideparen{n}} : n \in \mathbb{N}\}\\
\varphi&\mapsto\bigoplus{}_\infty\{\varphi^{\wideparen{n}}:n\in\mathbb{N}\},\\
\end{aligned}
$$ 
то есть секвенциальное операторное пространство $X$ отображается в $\bigoplus{}_\infty$-сумму своих размножений без аддитивной структуры.

{\bf Предложение 2.2.1} Пусть $\varphi:X\to Y$ --- ограниченный оператор между нормиированными пространствами $X$ и $Y$, тогда он $c$-топологически сюръективен тогда и только тогда когда существует ограниченный полулинейный оператор $\rho:Y\to X$ такой что $\Vert\rho\Vert\leq c$ и $\varphi\rho=1_Y$.

{\bf Доказательство.} Допустим, что $\varphi$ $c$-топологически сюръективен. Расссмотрим отношение $\sim$ на $S_Y$ определенное следующим образом: $e_1\sim e_2$ тогда и только тогда когда существует $\alpha\in\mathbb{T}$ такое, что $e_1=\alpha e_2$. Очевидно, $\sim$ есть отношение эквивалентности, поэтому рассмотрим множество ненулевых представителей классов эквивалентностей, которое обозначим $\{r_\lambda:\lambda\in\Lambda\}$. По построению, для каждого $e\in S_Y$ сущетсвует единственные $\alpha(e)\in\mathbb{T}$ и $\lambda(e)\in\Lambda$ такие, что $e=\alpha(e)r_{\lambda(e)}$. Ясно, что для любых $z\in\mathbb{T}$ и $e\in S_Y$ выполнено $\alpha(ze)=z\alpha(e)$ и $\lambda(ze)=\lambda(e)$. Так как $\varphi$ $c$-топологически сюръективен, то, в частности, для каждого $\lambda\in\Lambda$ существует $x(\lambda)\in X$ такой что $\Vert x(\lambda)\Vert\leq c\Vert r_\lambda\Vert$ и $\varphi(x(\lambda))=r_\lambda$. Рассмотрим, отображение $\tilde{\rho}:S_Y\to X:e\mapsto \alpha(e)x(\lambda(e))$. Легко видеть, что для всех $z\in\mathbb{T}$ и $e\in S_Y$ выполнено $\tilde{\rho}(z e)=z\tilde{\rho}(e)$, $\Vert\tilde{\rho}(e)\Vert\leq C$ и $\varphi(\tilde{\rho}(e))=e$. Теперь рассмотрим отображение $\rho:Y\to X: y\mapsto \Vert y\Vert\tilde{\rho}(\Vert y\Vert^{-1} y)$ и $\rho(0)=0$. Используя свойства $\tilde{\rho}$ легко проверить, что $\rho$ --- полулинейный оператор такой, что $\Vert\rho\Vert\leq C$ и $\varphi\rho=1_Y$.

Обратно, допустим, что существует ограниченный полулинейный оператор $\rho:Y\to X$ такой, что $\Vert\rho\Vert\leq c$ и $\varphi\rho=1_Y$. Возьмем произвольный $y\in Y$ и рассмотрим $x=\rho(y)$, тогда $\Vert x\Vert\leq C\Vert y\Vert$ и $\varphi(x)=y$. Следовательно $\varphi$ $c$-топологически сюръективен.

\medskip

{\bf Предложение 2.2.2} $\square_{sqTop}$-допустимыми эпиморфизмами являются в точности секвенциальные топологически сюрьективные операторы.

{\bf Доказательство.} Для произвольного секвенциального операторного пространства $Z$ через $i_n^Z:Z^{\wideparen{n}}\to\square_{sqTop}(Z)$ обозначим стандартное вложение, а через $p_n^Z:\square_{sqTop}(Z)\to Z^{\wideparen{n}}$ обозначим 
стандартную проекцию. Допустим что $\varphi:X\to Y$ $c$-секвенциально топологически сюръективен. Фиксируем $n\in\mathbb{N}$, тогда по предложению \ref{PrCTopSurIsRetrInNor0} существует ограниченный полулинейный оператор $\rho^n$ такой, что $\varphi^{\wideparen{n}}\rho^n=1_{Y^{\wideparen{n}}}$ и $\Vert\rho^n\Vert\leq c$. Рассмотрим отображение 
$ \rho=\bigoplus{}_\infty\{\rho^n:n\in\mathbb{N}\}$. Для любого $y\in \square_{sqTop}(Y)$ имеем 
$$
\Vert \rho(y)\Vert=\sup\{\Vert\rho^n(p_n^Y(y))\Vert_{\wideparen{n}}: n\in\mathbb{N}\}\leq
c\sup\{\Vert p_n^Y(y)\Vert_{\wideparen{n}}: n\in\mathbb{N}\}=c\Vert y\Vert
$$
следовательно $\rho$ --- полулинейный ограниченый оператор. Более того, $\square_{sqTop}(\varphi)\rho=1_{\square_{sqTop}(Y)}$, значит $\varphi$ $\square_{sqTop}$-допустимый эпиморфизм. Обратно, если 
$\varphi$ $\square_{sqTop}$-допустимый эпиморфизм, то существует ограниченный правый обратный полулинейный оператор  $\rho$ к $\square_{sqTop}(\varphi)$. Тогда для 
любого $y\in Y^{\wideparen{n}}$ выполнено $\square_{sqTop}(\varphi)\rho(i_n^Y(y))=i_n^Y(y)$. В частности $\varphi^{\wideparen{n}}(p_n^X(\rho(i_n^Y(y))))=y$. Положим $x=p_n^X(\rho(i_n^Y(y)))$ и 
$c=\Vert\rho\Vert$, тогда $\varphi^{\wideparen{n}}(x)=y$ и $\Vert x\Vert_{\wideparen{n}}\leq\Vert\rho(i_n^Y(y))\Vert\leq c\Vert i_n^Y(y)\Vert=c\Vert y\Vert_{\wideparen{n}}$. Следовательно, 
$\varphi$  секвенциально топологически сюръективен.


\medskip

\textit{Топологически свободными} секвенциальными пространствами естественно называть $\square_{sqTop}$-свободные объекты. Сформулируем и докажем основное утверждение раздела.

\medskip

{\bf Предложение 2.2.3} Пусть $F$ -  секвенциальное метрически свободное пространство с базой $\Lambda$. Тогда $F$ является секвенциальным  операторным топологически свободным с базой $\mathbb{C}^{\Lambda}$.

{\bf Доказательство.} Пусть $j':\Lambda\to \square_{sqMet}(F)$ --- универсальная стрелка в диаграмме для секвенциальной метрической свободы. Определим полулинейный ограниченный оператор $j: \mathbb{C}^{\Lambda} \to \square_{sqTop}(F): z_\lambda\mapsto z_\lambda j(\lambda)$. Рассмотрим произвольный ораниченный полулинейный оператор $\varphi : \mathbb{C}^{\Lambda} \to \square_{sqTop}(X)$, где $X$ --- произвольное секвенциальное операторное пространство.  Тогда для $\varphi':=\Vert \varphi \Vert_{sb}^{-1}\varphi $ существует единственный морфизм $\psi^{'}$, такой что $\varphi'=\square_{sqMet}(\psi')j$. Теперь, легко видеть что для морфизма $\psi:=\Vert \varphi \Vert_{sb} \psi^{'}$ диаграмма
$$
\xymatrix{
{\square_{sqTop}(F)}\ar@{-->}[dr]^{\square_{sqTop}(\psi)} & \\
{\mathbb{C}^{\Lambda}}\ar[u]_{j}\ar[r]_{\varphi}  &{\square_{sqTop}(X)} }
$$
коммутативна.
	
Единственность $\psi$ доказывается следующим образом. Пусть для диаграммы выше есть два различных подходящих морфизма $\psi_1$ и $\psi_2$. Обозначим $C=\max( \Vert \varphi\Vert_{sb}, \Vert \psi_1 \Vert_{sb}, \Vert \psi_2\Vert_{sb})$, тогда ясно что морфизмы $C^{-1}\psi_1$ и $C^{-1}\psi_2$ подходят для следующей диаграммы, соответствующей секвенциальной метрической проективности:
$$
\xymatrix{
{\square_{sqMet}(F)}\ar@{-->}[dr]^{?} & \\
{\mathbb{C}^\Lambda}\ar[u]_{j'}\ar[r]_{ C^{-1}\varphi'}  &{\square_{sqMet}(X)} }
$$
Это противоречит единственности морфизма $\psi'$, значит $\psi$ единственен.

\medskip

Как следствие мы получаем описание топологически свободных секвенциальных операторных пространств.

{\bf Теорема 2.2.4} Секвенциальное операторное пространство является топологически свободным тогда и только тогда, когда оно секвенциально топологически изоморфно $\bigoplus{}_1^0$-сумме пространств $t_2^\infty$, заиндексированных 
элементами некоторого множества $\Lambda$.


\subsection{Метрически косвободные секвенциальные пространства}

Рассмотрим функтор 
$$
\begin{aligned}
\square_{sqMet}^d : SQNor_1 \to Set^o: X &\mapsto \prod \left\{B_{(X^\triangle )^{\wideparen{n}}}:n\in\mathbb{N}\right\}\\
\varphi&\mapsto\prod\left\{ (\varphi^\triangle )^{\wideparen{n}}|_{B_{(Y^\triangle )^{\wideparen{n}}}}^{B_{(X^\triangle )^{\wideparen{n}}}}:n\in\mathbb{N}\right\}\\
\end{aligned}
$$

{\bf Предложение 2.3.1} $\square_{sqMet}^d$-допустимыми мономорфизмами являются в точности секвенциально  изометрические операторы.

{\bf Доказательство.}  Морфизм $\varphi$ является $\square_{sqMet}^d$-допустмым мономорфизмом только если $\square_{sqMet}^d(\varphi)$ обратим слева как морфизм в $Set^o$. Это равносильно тому что 
$\square_{sqMet}^d(\varphi^\triangle)$ сюръективно. Последнее эквивалентно  сюръективности $(\varphi^\triangle)^{\wideparen{n}}|_{B_{(X^\triangle)^{\wideparen{n}}}}^{B_{(Y^\triangle)^{\wideparen{n}}}}$ 
для всех $n\in\mathbb{N}$. Это означает, что $(\varphi^\triangle)^{\wideparen{n}}$ строго коизометрично для каждого $n\in\mathbb{N}$, т.е. $\varphi^\triangle$ секвенциально строго коизометричен. 
По теореме 1.2.8 это равносильно тому, что $\varphi$ секвенциально изометричен.

\medskip

\textit{Метрически косвободными} секвенциальными пространствами естественно называть $\square_{sqMet}^d$-косвободные объекты.

\medskip

{\bf Теорема 2.3.2} Метрически косвободным секвенциальным операторным пространством с базой $\Lambda$ является, с точностью до секвенциального изометрического изоморфизма, 
$\bigoplus{}_\infty$-сумма копий пространства $l_2^\infty:=\bigoplus{}_\infty\{l_2^n:n\in\mathbb{N}\}$, заиндексированных элементами множества $\Lambda$.

{\bf Доказательство.}  Пусть $\Lambda$ --- произвольное множество. Рассмотрим коммутативную диаграмму
$$
\xymatrix{
SQNor_1^o \ar[d]_{\nabla } \ar[rr]^{(\square_{sqMet}^{d})^o} & & {Set}\ar[d]^{1_{Set}}\\
SQNor_1\ar[rr]^{\square_{sqMet}}&  &{Set}}
$$
Здесь ${}^\nabla$ есть ковариантная версия функтора ${}^\triangle$.
Эта диаграмма коммутативна так как для произвольных секвенциальных операторных пространств $X$, $Y$ и любого $\varphi\in\mathcal{SB}(X,Y)$ выполнено
$$
1_{Set}((\square_{sqMet}^d)^o(\varphi))
=\prod\limits_{n\in\mathbb{N}} (\varphi^\triangle )^{\wideparen{n}}|_{B_{(Y^\triangle )^{\wideparen{n}}}}^{B_{(X^\triangle )^{\wideparen{n}}}}
=\square_{sqMet}({}^\nabla(\varphi))
$$
Заметим, что функтор ${}^\nabla$ имеет левый сопряженный функтор, а именно ${}^\triangle$. Аналогично $1_{Set}$ сопряжен слева к самому себе. 
По теореме 2.1.4 объект $\bigoplus{}_1^0\{t_2^\infty:\lambda\in\Lambda\}$ $\square_{sqMet}$-свободен, поэтому по предложению [\cite{HelMetrFrQmod}, 4.5] объект 
$(\bigoplus{}_1^0\{t_2^\infty:\lambda\in\Lambda\})^\triangle=\bigoplus{}_\infty\{l_2^\infty:\lambda\in\Lambda\}$ является $(\square_{sqMet}^d)^o$ свободным, или что то же самое $\square_{sqMet}^d$-косвободным. 
Так как множество $\Lambda$ произвольно, получаем, что все $\square_{sqMet}$-косвободные объекты с базой 
$\Lambda$ секвенциально изометрически изоморфны пространствам указанного вида.


\subsection{Топологически косвободные секвенциальные пространства}

Рассмотрим 
функтор 
$$
\begin{aligned}
\square_{sqTop}^d : SQNor \to Nor_0^o, X &\mapsto \bigoplus{}_\infty \{(X^{\triangle })^{\wideparen{n}} : n \in \mathbb{N}\}\\
\varphi&\mapsto\bigoplus{}_\infty \{(\varphi^\triangle )^{\wideparen{n}} : n \in \mathbb{N}\}
\end{aligned}
$$

{\bf Предложение 2.4.1}
$\square_{sqTop}^d$-допустимыми мономорфизмами являются в точности секвенциально топологически инъективные операторы.

{\bf Доказательство.}  Морфизм $\varphi$ является $\square_{sqTop}^d$-допустмым мономорфизмом только если $\square_{sqTop}^d(\varphi)$ обратим слева как морфизм в $Nor_0^o$. Это равносильно тому что 
$\square_{sqTop}^d(\varphi)=\square_{sqTop}(\varphi^\triangle)$ обратим справа в как морфизм в $Nor_0$. По предложению 2.2.2 это эквивалентно секвенциальной топологической сюръективности $\varphi^\triangle$. 
По теореме 1.2.8 это равносильно тому, что $\varphi$ секвенциально топологически инъективен.

\medskip

\textit{Топологически косвободными} секвенциальными пространствами естественно называть $\square_{sqTop}^d$-косвободные объекты.

\medskip
 
{\bf Теорема 2.4.2} Секвенциальное операторное пространство является топологически косвободным тогда и только тогда, когда оно секвенциально топологически изоморфно $\bigoplus{}_\infty$ сумме пространств $l_2^\infty$ заиндексированных элементами множества $\Lambda$.


{\bf Доказательство.} Пусть $\Lambda$ произвольное множество. Рассмотрим коммутативную диаграмму
$$
\xymatrix{
SQNor^o \ar[d]_{\nabla } \ar[rr]^{(\square_{sqTop}^d)^o} & & {Nor_0} \ar[d]^{1_{Nor_0}}\\
SQNor\ar[rr]^{\square_{sqTop}} & & {Nor_0}}
$$
Здесь $\nabla$ есть ковариантная версия функтора $\triangle$.
Эта диаграмма коммутативна, так как для произвольных секвенциальных операторных пространств $X$, $Y$ и любого $\varphi\in\mathcal{SB}(X,Y)$ выполнено
$$
1_{Nor_0}((\square_{sqTop}^d)^o(\varphi))
=\bigoplus{}_\infty \{(\varphi^\triangle )^{\wideparen{n}} : n \in \mathbb{N}\}
=\square_{sqTop}({}^\nabla(\varphi))
$$
Функтор ${}^\nabla$ имеет левый сопряженный функтор, а именно ${}^\triangle$. Аналогично $1_{Nor_0}$ сопряжен слева к самому себе. 
По теореме 2.2.4 объект $\bigoplus{}_1^0\{t_2^\infty:\lambda\in\Lambda\}$ $\square_{sqTop}$-свободен, поэтому по предложению [\cite{HelMetrFrQmod}, 4.5] объект 
$(\bigoplus{}_1^0\{t_2^\infty:\lambda\in\Lambda\})^\triangle=\bigoplus{}_\infty\{l_2^\infty:\lambda\in\Lambda\}$ является $(\square_{sqTop}^d)^o$-свободным, или что то же самое $\square_{sqTop}^d$-косвободным. Получаем, что все $\square_{sqTop}$-косвободные объекты с базой 
$\mathbb{C}^\Lambda$ секвенциально топологически изоморфны пространствам указанного вида.

\begin{thebibliography}{99}
\bibitem{LamOpFolgen}\textit{Lambert A.} Operatorfolgenr\"{a}ume. Eine Kategorie auf dem Weg von den Banach-R\"{a}umen zu den Operatorr\"{a}umen. Dissertation zur Erlangung des Grades Doktor der Naturwissenschaften der Technisch-Naturwissenschaftlichen Fakult\"{a}t I der Universit\"{a}t des Saarlandes. Saarbr\"{u}cken, 2002.
\bibitem{HelMetrFrQmod}
\textit{Хелемский А. Я.} Метрическая свобода и проективность для классических и квантовых нормированных модулей,, Матем. сб., 204:7 (2013), 127–158 
\bibitem{ShteinerTopFr}
\textit{Штейнер С. М.} 
Топологическая свобода для классических и квантовых нормированных модулей // Вестник СамГУ. 2013. № 9/1 (110). С.49-57.
\end{thebibliography}


\end{document}