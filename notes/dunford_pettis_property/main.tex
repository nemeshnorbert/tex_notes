%\title{The Dunford-Pettis property}
\documentclass[12pt]{article}
\usepackage[left=2cm, right=2cm, top=2cm, bottom=2cm,
    bindingoffset=0cm]{geometry}
\usepackage{amssymb,amsmath}
\usepackage[utf8]{inputenc}
\usepackage{mathrsfs}
\usepackage[colorlinks=true, urlcolor=blue, linkcolor=blue, citecolor=blue,
    pdfborder={0 0 0}]{hyperref}
\usepackage{enumitem}

\hypersetup{frenchlinks=true}

\newtheorem{theorem}{Theorem}[section]
\newtheorem{lemma}[theorem]{Lemma}
\newtheorem{proposition}[theorem]{Proposition}
\newtheorem{remark}[theorem]{Remark}
\newtheorem{corollary}[theorem]{Corollary}
\newtheorem{definition}[theorem]{Definition}
\newtheorem{example}[theorem]{Example}

\newenvironment{proof}{\par $\triangleleft$}{$\triangleright$}

\pagestyle{plain}

\begin{document}

\begin{center}

    \Large \textbf{The Dunford-Pettis property}\\[0.5cm]
    \small {Nemesh N. T.}\\[0.5cm]

\end{center}

\begin{abstract}
    This is a short note on the Dunford-Pettis property was written for
    self-educational purposes and future reference. A few additional subjects
    are discussed to give a firm introduction.
\end{abstract}

\section{Weak topologies}

\begin{definition} Let $X$ be a Banach space, then the weakest topology making
    all functionals in $X^*$ continuous is called the weak topology.

    If $X$ is a Banach space, then the weak${}^*$ topology on $X^*$ is the
    weakest topology making all functionals
    $\iota_X(x):X^*\to\mathbb{C}:f\mapsto f(x)$ continuous.
\end{definition}

The sequences that converge in $\langle$ weak / weak${}^*$ $\rangle$ topology to
$0$ we shall call $\langle$ weakly null / weakly${}^*$ null $\rangle$. A subset
$A$ in $\langle$ $X$ / $X^*$ $\rangle$ is called $\langle$ weakly / weakly${}^*$
$\rangle$ bounded if $\langle$ for all $f\in X^*$ the set $\{f(x):x\in A\}$ /
for all $x\in X$ the set $\{f(x):f\in A\}$ $\rangle$ is bounded in $\mathbb{C}$.

We shall list the following well known properties of weak topologies. Let $X$ be
a Banach space.

\begin{enumerate}[label = (\roman*)]

    \item A subset $A$ of $X$ is weakly bounded iff it is norm bounded
              [\cite{FabHabBanSpTh}, theorem 3.88].

    \item If $X$ is infinite dimensional, then every non empty weakly open
          subset of $X$ is unbounded [\cite{FabHabBanSpTh}, proposition 3.89].

    \item If the weak topology of $X$ is metriazable, then $X$ is finite
          dimensional.

    \item if $A$ is a convex subset in $X$, then the norm and the weak closure
          of $A$ coincide [\cite{FabHabBanSpTh}, theorem 3.45].

\end{enumerate}

Now we shall say a few words on weak${}^*$ topologies. Let $X$ be a Banach
space.

\begin{enumerate}[label = (\roman*)]

    \item $B_{X^*}$ is weak${}^*$ compact [\cite{FabHabBanSpTh}, theorem 3.37]

    \item $B_X$ is weak${}^*$ dense in $B_{X^{**}}$ [\cite{FabHabBanSpTh},
                  theorem 3.96]

\end{enumerate}

We proceed to discussion of weak compactness in Banach spaces. A subset $A$ of
$X$ is called $\langle$ weakly / relatively weakly $\rangle$ compact if
$\langle$ $A$ / the weak closure of $A$ $\rangle$ is compact in the weak
topology of $X$.

Again we list basic properties of weakly compact sets. Let $X$ be a Banach
space.

\begin{enumerate}[label = (\roman*)]

    \item Any weakly compact set in $X$ is norm closed and norm bounded.

    \item $B_X$ is weakly compact iff $X$ is reflexive [\cite{FabHabBanSpTh},
                  theorem 3.111]

    \item if $X$ is reflexive, then any bounded set is relatively weakly compact

    \item if $X$ is reflexive, then for any bounded linear operator $T:X\to Y$
          the set $T(B_X)$ is weakly compact.

\end{enumerate}

\begin{definition} Let $A$ be a subset of a topological space $M$, then $A$ is
    called

    \begin{enumerate}[label = (\roman*)]
        \item $\langle$ sequentially / relatively sequentially $\rangle$ compact
              if every sequence in $A$ has a subsequence convergent to a point
              $\langle$ in $A$ / in $M$ $\rangle$.

        \item $\langle$ countably / relatively countably $\rangle$ compact if
              every sequence in $A$ has a subnet convergent to a point $\langle$
              in $A$ / in $M$ $\rangle$.
    \end{enumerate}
\end{definition}

Countable compactness is implied by both compactness and sequential compactness.
If $M$ is metriazable, then all three concepts coincide, but the converse is not
true. For example $B_{\ell_\infty^*}$ with weak${}^*$ topology is compact by
Banach-Alaoglu theorem, but not sequentially weak${}^*$ compact, because the
sequence of functionals $e_n^*:\ell_\infty\to\mathbb{C}:x\mapsto x(n)$ has no
convergent subsequence. Though weak topology of Banach space is not metriazable
in general its bounded subsets behave much like they are metriazable. They
indeed metriazable if $X$ is separable.

\begin{theorem}[Eberlein-Smulian] Let $A$ be a subset of a Banach space $X$. The
    following are equivalent
    \begin{enumerate}[label = (\roman*)]
        \item $A$ is $\langle$ weakly / relatively weakly $\rangle$ compact
        \item $A$ is $\langle$ weakly / relatively weakly $\rangle$ sequentially
              compact
        \item $A$ is $\langle$ weakly / relatively weakly $\rangle$ countably
              compact
    \end{enumerate}
\end{theorem}
\begin{proof} See [\cite{KalAlbTopicsBanSpTh}, theorem 1.6.3]
\end{proof}

Here are some examples of characterization of relatively weakly compact subsets:

\begin{definition} A subset of $A$ of a Banach space $L_1(\Omega,\mu)$ is called
    equi-integrable if for any $\varepsilon>0$ there exists a $\delta>0$ such
    that for any measurable subset $E$ of $\Omega$ with $\mu(E)<\delta$ and any
    $f\in A$ holds
    $$
        \int_E |f(\omega)|d\mu(\omega)<\varepsilon
    $$
\end{definition}

\begin{theorem}[Dunford, Pettis] A bounded subset $A$ of $L_1(\Omega,\mu)$ is
    relatively weakly compact iff it is equi-integrable.
\end{theorem}
\begin{proof} See [\cite{KalAlbTopicsBanSpTh}, 5.2.9].
\end{proof}

By $M(K)$ we denote the Banach space of complex Borel regular measures on a
Hausdorff compact $K$.

\begin{definition} A subset $A$ of a Banach space $M(K)$ is said to be uniformly
    regular if for any open subset $U$ in $K$ and any $\varepsilon>0$, there is
    a compact set $H$ in $U$ such that $\sup_{\mu\in A}|\mu|(U\setminus
        H)<\varepsilon$.
\end{definition}

\begin{theorem}[Grothendieck] A bounded subset $A$ of $M(K)$ is relatively
    weakly compact iff it is uniformly regular.
\end{theorem}
\begin{proof} See [\cite{KalAlbTopicsBanSpTh}, 5.3.2]
\end{proof}

\section{Classes of bounded linear operators}

Note: by $\mathcal{B}(X,Y)$ we denote the Banach space of all bounded linear
operators between Banach spaces $X$ and $Y$.

\begin{definition} A bounded linear operator $T:X\to Y$ between Banach space $X$
    and $Y$ is called
    \begin{enumerate}[label = (\roman*)]
        \item compact if $T(B_X)$ is relatively compact in $Y$

        \item weakly compact if $T(B_X)$ is relatively weakly compact in $Y$

        \item completely continuous if $T(W)$ is compact in $Y$ for any weakly
              compact subset $W$ of $X$
    \end{enumerate}

    By $\langle$ $\mathcal{K}(X,Y)$ / $\mathcal{W}(X,Y)$ / $\mathcal{CC}(X,Y)$
    $\rangle$ we denote the Banach space of $\langle$ compact / weakly compact /
    completely continuous $\rangle$ operators.
\end{definition}

\begin{proposition} A bounded linear operator is completely continuous iff it is
    weak-to-norm sequentially  continuous.
\end{proposition}
\begin{proof} See [\cite{KalAlbTopicsBanSpTh}, 5.4.2].
\end{proof}

Just for comparison we mention the following fact

\begin{proposition} Let $X$ and $Y$ be two Banach spaces. Then

    \begin{enumerate}[label = (\roman*)]
        \item a linear operator $T:X\to Y$ is bounded iff $T$ is weak-to-weak
              continuous;

        \item if $T:X\to Y$ is a bounded linear operator, then $T^*$ is
              weak${}^*$-to-weak${}^*$ continuous;

        \item if $S:Y^*\to X^*$ is weak${}^*$-to-weak${}^*$ continuous then
              $S=T^*$ for some bounded linear operator $T:X\to Y$.

        \item if $T:X\to Y$ is weak-to-norm continuous, then it is a finite rank
              operator
    \end{enumerate}
\end{proposition}
\begin{proof} $(i)$ See [\cite{DunfSchwLinOpsVol1}, theorem 5.3.15].

    $(ii)$ Straightforward.

    $(iii)$ See [\cite{FabHabBanSpTh}, exercise 3.60]

    $(iv)$ See [\cite{FabHabBanSpTh}, exercise 15.3]  % chktex 15
\end{proof}

\begin{proposition} For any Banach spaces $X$ and $Y$we have isometric
    inclusions
    \begin{enumerate}[label = (\roman*)]
        \item $\mathcal{K}(X,Y)\subset\mathcal{W}(X,Y)\subset\mathcal{B}(X,Y)$
        \item $\mathcal{K}(X,Y)\subset\mathcal{CC}(X,Y)\subset\mathcal{B}(X,Y)$
    \end{enumerate}
\end{proposition}
\begin{proof} The only non-trivial inclusion here is
    $\mathcal{K}(X,Y)\hookrightarrow \mathcal{CC}(X,Y)$. See
        [\cite{FabHabBanSpTh}, exercise 1.77].
\end{proof}

\begin{theorem}[Schauder] A bounded linear operator between Banach spaces is
    compact iff its adjoint is compact.
\end{theorem}
\begin{proof} See theorem 15.3 in~\cite{FabHabBanSpTh}
\end{proof}

\begin{theorem}[Davis, Figel, Johnson, Pelczynski] A bounded linear operator
    $T:X\to Y$ is weakly compact iff there exists a reflexive Banach space $Z$
    and bounded linear operators $R:X\to Z$, $Q:Z\to Y$ such that $T=QR$, that
    is $T$ factors through a reflexive Banach space.
\end{theorem}
\begin{proof} Necessity is proved in [\cite{FabHabBanSpTh}, theorem 13.33]. The
    converse is obvious. Indeed, since $X$ is reflexive, then $Q$ is weakly
    compact and so does $T=QR$.
\end{proof}

\begin{theorem}[Gantmacher] Let $T:X\to Y$ be a bounded linear operator between
    Banach spaces $X$ and $Y$. Then the following are equivalent

    \begin{enumerate}[label = (\roman*)]
        \item $T$ is weakly compact;

        \item $T^{**}(X^{**})\subset Y$;

        \item $T^*$ is weak${}^*$-to-weak continuous;

        \item $T^*$ is weakly compact.
    \end{enumerate}
\end{theorem}
\begin{proof} $(i)\implies (ii)$ It is enough to check that all elements of
    $T^{**}(X^{**})$ are weak${}^*$-continuous.

    $(ii)\implies (iii)$ Straightforward.

    $(iii)\implies (iv)$ By Davis-Jhonson-Figel-Pelczynski theorem $T$ factors
    through a reflexive space. Clearly, so does $T^*$.


\end{proof}

\section{The Dunford-Pettis property}

\begin{definition}
    We say that a Banach space $X$ has the Dunford-Pettis property if
    $\mathcal{W}(X,Y)\subset \mathcal{CC}(X,Y)$ for any Banach space $Y$.
\end{definition}

Directly from definition it follows that the square of any weakly compact
operator on a Banach space with Dunford-Pettis property is necessarily compact.
If $X$ is reflexive Banach space, then $1_X$ is weakly compact. Therefore $X$
has the  Dunford-Pettis property iff $1_X$ is compact. The latter is possible
only for finite dimensional $X$. In other words, for infinite dimensional
reflexive Banach space $X$ the operator $1_X$ provides an example of weakly
compact but not completely continuous operator.

\begin{theorem} A Banach space $X$ has the Dunford-Pettis property iff for any
    weakly null sequences ${(x_n)}_{n\in\mathbb{N}}\subset X$,
    ${(x_n^*)}_{n\in\mathbb{N}}\subset X^*$ holds $\lim_n x_n^*(x_n)=0$.
\end{theorem}
\begin{proof} Let $Y$ be a Banach space and $T:X\to Y$ a weakly compact
    operator. Let us suppose that $T$ is not Dunford-Pettis. Then there is
    weakly null sequence ${(x_n)}_{n\in\mathbb{N}}$ in $X$ such that for some
    $\delta>0$ holds $\Vert T(x_n)\Vert\geq \delta$ for all $n\in\mathbb{N}$.
    Pick ${(y_n^*)}_{n\in\mathbb{N}}\subset Y^*$ such that $y^*(T(x_n))=\Vert
        T(x_n)\Vert$ and $\Vert y_n\Vert=1$ for all $n\in\mathbb{N}$. By
    Gantmacher’s theorem $T^*$ is weakly compact hence $T^*(B_{Y^*})$ is a
    relatively weakly compact subset of $X^*$. By the Eberlein-Smulian theorem
    the sequence $T^*(y_n^*)\subset T^*(B_{Y^*})$ can be assumed weakly
    convergent to some $x^*\in X^*$. Then ${(T^*(y_n^*)-x^*)}_{n\in\mathbb{N}}$
    is weakly null. Therefore, the Dunford-Pettis property of $X$ gives that
    $$
        (T^*(y_n^*)-x^*)(x_n)\to_n 0
    $$
    when $n\to\infty$. Note that $x^*(x_n)\to_n 0$ since
    ${(x_n)}_{n\in\mathbb{N}}$ is weakly null. Therefore $\Vert
        T(x_n)\Vert=T^*(y_n)(x_n)\to_n 0$. Contradiction.

    For the converse, let ${(x_n)}_{n\in\mathbb{N}}\subset X$ and
    ${(x_n^*)}_{n\in\mathbb{N}}\subset X^*$ be weakly null sequences. Consider
    the operator
    $$
        T:X \to c_0: x\mapsto {(x_n^*(x))}_{n\in\mathbb{N}}
    $$
    The adjoint operator $T^*$ of $T$ satisfies $T^*(e_k)=x_k^*$ for all
    $k\in\mathbb{N}$, where ${(e_k)}_{k\in\mathbb{N}}$ denotes the canonical
    basis of $\ell_1$. This implies that $T^*(B_{\ell_1})$ is contained in the
    convex hull of the weakly null sequence ${(x_n^*)}_{n\in\mathbb{N}}$.
    Therefore $T^*$ is weakly compact, hence by Gantmacher’s theorem so is $T$.
    As $T$ is weakly compact, $T$ is also Dunford-Pettis by the hypothesis.
    Therefore ${(T(x_n))}_{n\in\mathbb{N}}$ is norm null sequence. Since
    $$
        |x_n^*(x_n)|\leq\max_k |x_k^*(x_n)|=\Vert T(x_n)\Vert
    $$
    then ${(x_n^*(x_n))}_{n\in\mathbb{N}}$ is norm null sequence too.
\end{proof}

Any Banach space $X$ with the Schur property (that is any weakly null sequence
is norm null sequence) has the Dunford-Pettis property. Indeed, let
${(x_n)}_{n\in\mathbb{N}}\subset X$ and ${(x_n^*)}_{n\in\mathbb{N}}\subset X^*$
be weakly null sequences. Since ${(x_n^*)}_{n\in\mathbb{N}}$ is weakly null it
is norm bounded by some constant $C$. The Schur property of $X$ gives that
${(x_n)}_{n\in\mathbb{N}}$ is norm null, so $|x_n^*(x_n)|\leq C\Vert
    x_n\Vert\to_n 0$. In particular $\ell_1$ has the Dunford-Pettis property.

\begin{example} Consider operator $T:L_1([0,1])\to
        C([0,1]):x\mapsto\left(t\mapsto\int_0^t x(s)ds\right)$. One can show
    that for the sequence
    $x_n(t)=2n(\chi_{[1/2-1/n,1/2]}-\chi_{[1/2,1/2+1/n]})$ the sequence
    $T(x_n)$ has no weakly convergent subsequence. So $T$ is not weakly
    compact. Using Arzela-Ascoli criterion of compactness in $C([0,1])$ and
    Dunford-Pettis criterion of weak compactness in $L_1([0,1])$ it is
    routine to check that $T$ is completely continuous.
\end{example}

\begin{proposition} If $X^*$ has the Dunford-Pettis property, then so does $X$.
\end{proposition}
\begin{proof} Let ${(x_n)}_{n\in\mathbb{N}}\subset X$ and
    ${(x_n^*)}_{n\in\mathbb{N}}\subset X^*$ be weakly null sequences. Take
    arbitrary $x^{***}\in X^{***}$, then since ${(x_n)}_{n\in\mathbb{N}}$ is
    weakly null, then $x^{***}(\iota_X(x_n))=\iota_X^*(x^{***})(x_n)\to_n 0$.
    Therefore ${(\iota_X(x_n))}_{n\in\mathbb{N}}\subset X^{**}$ is weakly null
    sequence. Since $X^*$ has the Dunford-Pettis property, then
    $x_n^*(x_n)=\iota_X(x_n)(x_n^*)\to_n 0$. Therefore $X$ has the
    Dunford-Pettis property.
\end{proof}

\begin{proposition} The Dunford-Pettis property is inherited by complemented
    subspaces of Banach spaces.
\end{proposition}
\begin{proof} Let $Y$ be a complemented subspace of $X$ with bounded linear
    projection $P:X\to Y$. Let ${(y_n)}_{n\in\mathbb{N}}\subset Y$ and
    ${(y_n^*)}_{n\in\mathbb{N}}\subset Y^*$ be weakly null sequences. Then
    ${(y_n)}_{n\in\mathbb{N}}$ is a weakly null in $X$. Since $P^*$ is bounded,
    then it is weak-to-weak continuous, so from assumption
    ${(P^*(y_n))}_{n\in\mathbb{N}}$ is weakly null in $X^*$. Since $X$ has the
    Dunford-Pettis property, then
    $$
        y_n^*(y_n)=y_n^*(P(y_n))=P^*(y_n^*)(y_n)\to_n 0
    $$
    Therefore $Y$ has the Dunford-Pettis property.
\end{proof}

Proving a certain Banach space has the Dunford-Pettis property is always a
challenge. Most proofs require understanding of weakly null sequences of dual
spaces. It is known since the times of Grothendieck that $L_1$ spaces and
$C(K)$-spaces have the Dunford-Pettis property.

\newpage
\begin{thebibliography}{999}
    \bibitem{FabHabBanSpTh}
    \textit{Fabian M., Habala P.} Banach space theory. Springer, 2011.
    \bibitem{KalAlbTopicsBanSpTh}
    \textit{Albiac F., Kalton N. J.} Topics in Banach space theory. Springer,
    2006, vol 233.
    \bibitem{DunfSchwLinOpsVol1}
    \textit{Dunford N. Schwartz J.T.} Linear operators. Wiley-interscience New
    York, 1971, vol 1.
\end{thebibliography}
\end{document}
