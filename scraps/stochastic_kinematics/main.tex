
\documentclass[12pt]{article}
\usepackage[utf8]{inputenc}
\usepackage[russian]{babel}
\usepackage{amssymb}
\usepackage{amsmath}
\usepackage{mathrsfs}
\usepackage{esvect}
\usepackage[left=2cm, right=2cm, top=2cm, bottom=2cm,
    bindingoffset=0cm]{geometry}
\usepackage[colorlinks=true, urlcolor=blue, linkcolor=blue, citecolor=blue,
    pdfborder={0 0 0}]{hyperref}
\usepackage{enumitem}

\hypersetup{frenchlinks=true}

\newtheorem{theorem}{Theorem}[section]
\newtheorem{lemma}[theorem]{Лемма}
\newtheorem{proposition}[theorem]{Предложение}
\newtheorem{remark}[theorem]{Замечание}
\newtheorem{corollary}[theorem]{Следствие}
\newtheorem{definition}[theorem]{Определение}
\newtheorem{example}[theorem]{Пример}

\newenvironment{proof}{\par $\triangleleft$}{$\triangleright$}

\pagestyle{plain}

\begin{document}

\begin{center}

    \Large \textbf{Преобразования кинематических величин при переходе между разными системами координат}\\[0.5cm]
    \small {Немеш Н. Т.}\\[0.5cm]

\end{center}
\date{March 2022}

\section{Справочные материалы}

\subsection{Некоторые свойства ортогональных матриц и векторного произведения}

Пусть $a$, $b$ и $c$ -- три вектора из $\mathbb{R}^3$.
Тогда, векторное произвдение будем обозначать через $a \times b$, a скалярное
через $\langle a,b\rangle$.

Напомним определение смешанного произвдения
$$
    (a, b, c)=\langle a, b\times c\rangle
$$
Поскольку,
$$
    (a, b, c)=\det[a, b, c],
$$
то смешанное произвдение (как и определитель) меняет знак при перестановке любых двух
соседних аргументов.


\begin{definition}
    Пусть $a$ и $b$ -- два вектора из $R^3$.
    Поскольку векторное произведение линейно по каждому аргументу,
    то корректно определен линейный оператор
    $$
        [a]_\times : \mathbb{R}^3\to\mathbb{R}^3: x\mapsto a\times x
    $$
\end{definition}

\begin{proposition}
    Пусть $a, b\in\mathbb{R}^3$, тогда
    \begin{enumerate}[label = (\roman*)]
        \item $[a]_\times b=-[b]_\times a$
        \item $([a]_\times)^T=-[a]_\times$
        \item $[a]_\times [b]_\times=b a^T - (a^T b) E$
    \end{enumerate}
\end{proposition}
\begin{proof}
    $(i)$ Непосредственно из определения
    $$
        [a]_\times b=a \times b=-b \times a=-[b]_\times a
    $$
    $(ii)$ Пусть $x,y\in\mathbb{R}^3$, тогда
    $$
        \begin{aligned}
            \langle y, [a]_\times^T x\rangle
             & =\langle [a]_\times y, x\rangle  \\
             & =\langle x, [a]_\times y\rangle  \\
             & =\langle x, a\times y\rangle     \\
             & =(x, a, y)                       \\
             & =-(y, a, x)                      \\
             & =-\langle y, a\times x\rangle    \\
             & =-\langle y, [a]_\times x\rangle
        \end{aligned}
    $$
    Поскольку $x,y\in\mathbb{R}^3$ произвольны, то $([a]_\times)^T=-[a]_\times$

    $(iii)$ Пусть $x\in\mathbb{R}^3$, тогда
    $$
        \begin{aligned}
            \,[a]_\times [b]_\times x
             & = a \times (b \times x)                        \\
             & =\langle a, x\rangle b - \langle a, b\rangle x \\
             & =(a^T x) b - (a^T b) x                         \\
             & =b(a^T x) - (a^T b) x                          \\
             & =(b a^T) x - (a^T b) x                         \\
             & =(b a^T  - (a^T b) E)x                         \\
        \end{aligned}
    $$
    Поскольку $x\in\mathbb{R}^3$
    произвольно $[a]_\times [b]_\times=b a^T - (a^T b) E$.
\end{proof}

\begin{proposition}
    Пусть $M$ -- произвольная матрица и $u,v\in\mathbb{R}^3$, тогда
    $$
        M^T(M u\times M v)=\det(M) (u \times v)
    $$
    В частности, для любой ортогональной матрицы $Q$
    $$
        Qu\times Qv=\det(Q) Q(u\times v)
    $$
    Наконец, если $R$ -- матрица вращения $\mathbb{R}^3$, то
    \begin{equation}\label{eq:1}
        Ru\times Rv=R(u\times v)
    \end{equation}
\end{proposition}
\begin{proof}
    Пусть $x\in\mathbb{R}^3$, тогда
    $$
        \begin{aligned}
            \det(M)\langle x, u\times v\rangle
             & =\det(M)(x,u,v)                      \\
             & =\det(M)\det[x,u,v]                  \\
             & =\det(M\cdot [x,u,v])                \\
             & =\det[Mx,Mu,Mv]                      \\
             & =(Mx,Mu,Mv)                          \\
             & =\langle Mx, Mu\times Mv\rangle      \\
             & =\langle x, M^T(Mu \times Mv)\rangle
        \end{aligned}
    $$
    Поскольку $x\in\mathbb{R}^3$
    произвольно, то $\det(M) (u\times v)=M^T(Mu\times Mv)$.

    Если в качестве $M$ взять ортогональную матрицу $Q$, то так как $Q^T=Q^{-1}$,
    то $\det(Q)(u\times v)=Q^{-1}(Qu\times Qv)$. Откуда
    $$
        Qu\times Qv=\det(Q) Q(u\times v).
    $$
    Осталось напомнить, что у матриц вращения определитель равен $1$.
\end{proof}

\begin{proposition}
    Пусть $R$ -- матрица вращения и $a$ -- произвольный вектор
    в $\mathbb{R}^3$, тогда
    \begin{enumerate}[label = (\roman*)]
        \item
              \begin{equation}
                  [a]_\times R=R[R^Ta]_\times, \quad [Ra]_\times=R[a]_\times R^T
              \end{equation}
        \item
              \begin{equation}
                  (E+[a]_\times)R=R(E+[R^Ta]_\times)
              \end{equation}
        \item
              \begin{equation}
                  (R(E+[a]_\times))^T=R^T(E-[Ra]_\times)
              \end{equation}
        \item
              $$
                  (R(E+[a]_\times))^{-1}=R^T(E-[Ra]_\times)+o(\Vert a\Vert^2)
              $$
    \end{enumerate}
\end{proposition}
\begin{proof}
    $(i)$ Пусть $x\in\mathbb{R}^3$, тогда
    $$
        \begin{aligned}
            \,[a]_\times R x
             & = a\times Rx         \\
             & = RR^Ta\times Rx     \\
             & = R(R^Ta\times x)    \\
             & = R([R^Ta]_\times x) \\
        \end{aligned}
    $$
    Поскольку $x\in\mathbb{R}^3$ произвольно $[a]_\times R=R[R^Ta]_\times$.

    Заменяя $a$ на $Ra$ получим $[Ra]_\times R=R[R^TRa]_\times$. Откуда
    $$
        [Ra]_\times=[Ra]_\times RR^T=([Ra]_\times R)R^T
        =R[R^TRa]_\times R^T=R[a]_\times R^T
    $$

    $(iii)$ Так как $[a]_\times^T=-[a]_\times$, то
    $$
        \begin{aligned}
            (R(E+[a]_\times))^T
             & =(E+[a]_\times)^T R^T \\
             & =(E+[a]_\times^T) R^T \\
             & =(E-[a]_\times) R^T   \\
        \end{aligned}
    $$

    $(ii)$ Из $(ii)$ следует, что
    $$
        \begin{aligned}
            (E+[a]_\times) R
             & = R+[a]_\times R     \\
             & = R+R[R^Ta]_\times   \\
             & = R(E+[R^Ta]_\times) \\
        \end{aligned}
    $$

    $(iii)$ Из пункта $(ii)$ следует, что
    $$
        \begin{aligned}
            R(E+[a]_\times) R^T(E-[Ra]_\times)
             & =(R+R[a]_\times)(R^T - R^T[Ra]_\times)                                \\
             & =R R^T+R[a]_\times R^T - R R^T[Ra]_\times- R[a]_\times R^T[Ra]_\times \\
             & =E+R[a]_\times R^T - [Ra]_\times- R[a]_\times R^T[Ra]_\times          \\
             & =E - R[a]_\times R^T[Ra]_\times                                       \\
             & =E - R[a]_\times R^TR[a]_\times R^T                                   \\
             & =E - R[a]_\times [a]_\times R^T                                       \\
             & =E + o(\Vert a\Vert^2)                                                \\
        \end{aligned}
    $$
    Откуда
    $$
        (R(E+[a]_\times))^{-1}=R^T(E-[Ra]_\times)+o(\Vert a\Vert^2)
    $$
\end{proof}


\subsection{Напоминания из теории многомерных нормальных распределений}

\begin{definition} Пусть $\vv{\mu}\in\mathbb{R}^n$ -- $n$-мерный вектор,
    и $\Sigma$ -- положительно определенная матрица размера $n\times n$. Тогда
    случайный вектор $\pmb{X}=[X_1,\ldots,X_n]^T$ имеет многомерное
    нормальное распределение если его плотность распределения  имеет вид
    $$
        f_{\pmb{X}}:\mathbb{R}^n\to\mathbb{R}
        :\vv{x}\mapsto \frac{1}{(2\pi)^{n/2}\sqrt{\det(\Sigma)}}
        \exp\left(
        -\frac{1}{2}(\vv{x}-\vv{\mu})\Sigma^{-1}(\vv{x}-\vv{\mu})
        \right).
    $$
    В этом случае мы будем писать
    $\pmb{X}\sim \mathcal{N}(\vv{\mu}, \Sigma)$.
\end{definition}

\begin{remark} Допустим, что случайный вектор $\pmb{X}$ имеет
    многомерное нормальное распределение с параметрами $\vv{\mu}$ и $\Sigma$,
    т.е. $\pmb{X}\sim \mathcal{N}(\vv{\mu}, \Sigma)$.
    Пусть $A$ -- произвольная матрица размера $k\times n$
    и $\vv{b}\in\mathbb{R}^k$ -- произвольный $k$-мерный вектор,
    тогда случайный вектор $\pmb{Y}=A \pmb{X}+\vv{b}$ имеет многомерное
    нормальное распределение, причем
    $$
        \pmb{Y}\sim\mathcal{N}(A\vv{\mu}+\vv{b}, A\Sigma A^T)
    $$
\end{remark}



\section{Преобразования кинематических величин при замене системы координат}

В этом параграфе мы выведем формулы преобразования координат векторов
различных кинематических величин: поза, скорость, угловая скорость, ускорение.
Напомним, что поза -- это радиус-вектор и ориентация твердого тела.
Во многих ситуациях мы будем считать, что эти величины имеют многомерное
нормальное распределение и наша основная задача найти параметры
этого распределения.

Большинство векторных величин (поза, скорость, угловая скорость, ускорение)
в нашем случае имеют простую модель. Если $\vv{x}$ некоторый вектор то мы
будем считать, что к нему добваляется некоторый шум,
т.е. случайный вектор с многомерным нормальным распределением с нулевым средним. 
Такой случайный вектор мы будем обозначать $\pmb{x}$. Более подробно, 
пусть $A$ некоторая система координат, тогда координаты $\pmb{x}$ в $A$ 
будут иметь распределение $\mathcal{N}(x^A, \Sigma_x^A)$.

Мы также будем предполагать, что переходы между некоторыми системами
координат нам точно не известны и имеют некоторый шум. Точнее
пусть $\pmb{R}$ -- случайная велична описывающая матрицу перехода между
двумя системами координат. Будем считать, что существует фиксированная матрица
перехода $R$ и случайный вектор $\delta\pmb{\theta}\sim\mathcal{N}(0, \Sigma)$,
такие, что $\pmb{R}=R(E+[\delta\pmb{\theta}]_\times)$

Теперь рассмотрим различные сценарии преобразования между двумя системами
координат. В этих сценариях системы координат
могут быть
\begin{itemize}
    \item неподвижными друг относительно друга
    \item двигаться с линейной или с угловой или с обеими скоростями
          друг относительно друга
    \item иметь нетривиальную матрицу поворота
    \item иметь матрицу поворота являющуюся случайной величиной
\end{itemize}

\subsection{Преобразования позиции для константной замены координат
    относительно инерциальной системы отсчета
}

Допустим у нас есть три системы координат $A$, $B$ и $C$,
причем $C$ -- инерциальная система координат.
Пусть нам известны поза системы координат $A$ относительно $C$ и $B$
относительно $A$, надо найти позу системы координат $B$ относительно $C$.
Будем считать, что поза $A$ в $C$ является случайной величиной, а поза $B$
в $A$ доподлинно известна. За модельный пример здесь можно
взять $A=\mbox{imu}$, $B=\mbox{base link}$ и $C=\mbox{map}$.

Пусть $(\pmb{R}_A^C, \pmb{p}_A^C)$ -- поза $A$ в $C$, тогда
$$
    \pmb{R}_A^C=R_A^C(E+[\delta\pmb{\theta}_A^C]_\times),
    \quad
    \delta\pmb{\theta}_A^C \sim\mathcal{N}(0, \Sigma_{\delta\theta}^A)
$$
$$
    \pmb{p}_A^C=p_A^C+\delta \pmb{p}_A^C,
    \quad
    \delta\pmb{p}_A^C\sim\mathcal{N}(0, \Sigma_{\delta p}^A),
$$
Поза $B$ в $C$ описывается формулами
$$
    \begin{aligned}
        \pmb{R}_B^C
         & =\pmb{R}_A^C R_B^A                                              \\
         & =R_A^C(E+[\delta\pmb{\theta}_A^C]_\times) R_B^A                 \\
         & =R_A^C R_B^A(E+[(R_B^A)^T\delta\pmb{\theta}_A^C]_\times)        \\
        \pmb{p}_B^C
         & =\pmb{p}_A^C + \pmb{R}_A^C p_B^A                                \\
         & =p_A^C + \delta \pmb{p}_A^C
        + R_A^C(E+[\delta\pmb{\theta}_A^C]_\times) p_B^A                   \\
         & =p_A^C + R_A^C p_B^A
        + \delta \pmb{p}_A^C + R_A^C [\delta\pmb{\theta}_A^C]_\times p_B^A \\
         & =p_A^C + R_A^C p_B^A
        + \delta \pmb{p}_A^C - R_A^C [p_B^A]_\times \delta\pmb{\theta}_A^C \\
    \end{aligned}
$$
Следовательно,
\begin{equation}
    \pmb{R}_B^C=R_A^CR_B^A(E+[\delta\pmb{\theta}_B^C]_\times),
    \quad
    \delta\pmb{\theta}_B^C \sim\mathcal{N}(0, (R_B^A)^T\Sigma_{\delta\theta}^A R_B^A)
\end{equation}
\begin{equation}
    \pmb{p}_B^C=p_A^C + R_A^C p_B^A+\delta \pmb{p}_A^C,
    \quad
    \delta\pmb{p}_B^C\sim\mathcal{N}(
    0,
    \Sigma_{\delta p}^A
    +R_A^C [p_B^A]_\times \Sigma_{\delta\theta}^A (R_A^C [p_B^A]_\times)^T
    )
\end{equation}


\subsection{Преобразования скорости, угловой скорости и ускорения для
    константной замены координат между инерциальными
    системами координат связанными с твердым телом
}

Допустим у нас есть две инерциальные системы координат $A$ и $B$ связанные с
некоторым твердым телом. Пускай нам известна некоторая векторная
велична $\pmb{x}$ в системе координат $A$, и поза системы координат $A$
относительно $B$. Надо найти $\pmb{x}$ в системе координат $B$. Будем считать,
что поза $A$ в $B$ доподлинно известна. За модельный пример здесь можно
взять $A=\mbox{imu}$, $B=\mbox{base link}$.

Пусть
$$
    \pmb{x}^A=x^A+\delta \pmb{x}^A,
    \quad
    \delta x^A\sim\mathcal{N}(0, \Sigma_{\delta x}^A)
$$
Тогда
$$
    \begin{aligned}
        \pmb{x}^B
         & =R_A^B \pmb{x}^A                    \\
         & =R_A^B (x^A + \delta \pmb{x}^A)     \\
         & =R_A^B x^A + R_A^B \delta \pmb{x}^A \\
    \end{aligned}
$$
Следовательно,
\begin{equation}
    \pmb{x}^B=R_A^Bx^A+\delta \pmb{x}^B,
    \quad
    \delta \pmb{x}^B\sim\mathcal{N}(0, R_A^B \Sigma_{\delta x}^A (R_A^B)^T)
\end{equation}
В последней формуле в качестве $\pmb{x}$ можно брать скороть $\pmb{v}$, угловую
скорость $\pmb{\omega}$ или ускорение $\pmb{a}$.

\subsection{Преобразования скорости, угловой скорости и ускорения для
    неконстантной замены координат между инерциальными системами
    координат связанными с твердым телом
}

Допустим у нас есть две системы координат $A$ и $B$ связанные с некоторым
твердым телом. Пускай нам известна некоторая векторная велична $\pmb{x}$ в
системе координат $A$, и поза системы координат $A$ относительно $B$.
Надо найти $\pmb{x}$ в системе координат $B$. Будем считать, что поза $A$
в $B$ является случайной величиной. За модельный пример здесь можно
взять $A=\mbox{imu}$, $B=\mbox{base link}$.

Пусть
$$
    \pmb{x}^A=x^A+\delta \pmb{x}^A, \quad \delta x^A\sim\mathcal{N}(0, \Sigma_{\delta x}^A)
$$
$$
    \pmb{R}_A^B=R_A^B(E+[\delta \pmb{\theta}_A^B]_\times),
    \quad
    \delta\pmb{\theta}_A^B \sim \mathcal{N}(0, \Sigma_{\delta \theta})
$$
Тогда,
$$
    \begin{aligned}
        \pmb{x}^B
         & =\pmb{R}_A^B \pmb{x}^A                                        \\
         & =R_A^B (E+[\delta \pmb{\theta}_A^B]) (x^A + \delta \pmb{x}^A) \\
         & =R_A^B x^A + R_A^B \delta \pmb{x}^A
        + R_A^B [\delta \pmb{\theta}_A^B]_\times x^A
        + R_A^B [\delta \pmb{\theta}_A^B]_\times \delta \pmb{x}^A        \\
         & =R_A^B x^A + R_A^B \delta \pmb{x}^A
        - R_A^B [x^A]_\times \delta \pmb{\theta}_A^B
        + o(\Vert \delta\pmb{x}^A \Vert)                                 \\
    \end{aligned}
$$
Следовательно,
\begin{equation}
    \pmb{x}^B=R_A^Bx^A+\delta \pmb{x}^B,
    \quad
    \delta \pmb{x}^B\sim\mathcal{N}(
    0,
    R_A^B \Sigma_{\delta x}^A (R_A^B)^T
    + R_A^B [x^A]_\times \Sigma_{\delta\theta} (R_A^B [x^A]_\times)^T
    )
\end{equation}
В последней формуле в качестве $\pmb{x}$ можно брать скорость $\pmb{v}$,
угловую скорость $\pmb{\omega}$ или ускорение $\pmb{a}$.

\subsection{Нахождение скорости точек твердого тела в различных системах координат}

Допустим, у нас есть две системы координат $A$ и $B$ связанные с твердым телом.
Пусть в некоторой инерциальной системе координат $C$ твердое тело имеет
угловую скорость $\pmb{\omega}$. Пусть $\pmb{v}_A$ линейная
скорость начала системы координат $A$ относительно системы координат $C$.
Аналогично $\pmb{v}_B$ линейная скорость начала системы координат $B$
относительно системы координат $C$

Так как $A$ и $B$ связаны с одним
и тем же твердым телом, то $A$ и $B$ будет иметь в системе координат $C$ ту же
самую угловую скорость. Тогда
\begin{equation}
    \pmb{v}_B=\pmb{v}_A+\pmb{\omega} \times \vv{AB}
\end{equation}
Мы выведем параметры распределения вектора $\pmb{v}_B$ в базисах
$A$ и $C$.

За модельный пример здесь можно взять $A=\mbox{imu}$, $B=\mbox{base link}$
и $C=\mbox{map}$.

Пусть координаты $\pmb{v}_A^A$ вектора $\pmb{v}_A$ в системе координат $A$ имеют
распределение
$$
    \pmb{v}_A^A=v_A^A+\delta \pmb{v}_A^A,
    \quad
    \delta \pmb{v}_A^A \sim \mathcal{N}(0, \Sigma_{\delta v}^A)
$$
и угловая скорость в системе координат $A$ имеет распределение
$$
    \pmb{\omega}^A=\omega^A+\delta \pmb{\omega}^A,
    \quad
    \delta \pmb{\omega}^A\sim \mathcal{N}(0, \Sigma_{\delta \omega}^A)
$$
Тогда
$$
    \begin{aligned}
        \pmb{v}_B^A
         & =\pmb{v}_A^A+\pmb{\omega}^A \times p_B^A                 \\
         & =\pmb{v}_A^A+\delta \pmb{v}_A^A
        + (\omega^A+\delta \pmb{\omega}^A) \times p_B^A             \\
         & =\pmb{v}_A^A+ \omega^A\times p_B^A
        + \delta \pmb{v}_A^A + \delta \pmb{\omega}^A \times p_B^A   \\
         & =\pmb{v}_A^A+ \omega^A\times p_B^A
        + \delta \pmb{v}_A^A - [p_B^A]_\times \delta \pmb{\omega}^A \\
    \end{aligned}
$$
Следовательно,
\begin{equation}
    \pmb{v}_B^A=\pmb{v}_A^A+ \omega^A\times p_B^A+\delta \pmb{v}_B^A,
    \quad
    \delta \pmb{v}_B^A\sim\mathcal{N}(
    0,
    \Sigma_{\delta v}^A
    + [p_B^A]_\times \Sigma_{\delta\omega} ([p_B^A]_\times)^T
    )
\end{equation}

Пусть координаты $\pmb{v}_A^C$ вектора $\pmb{v}_A$ в системе координат $C$ имеют
распределение
$$
    \pmb{v}_A^C=v_A^C+\delta \pmb{v}_A^C,
    \quad
    \delta \pmb{v}_A^C \sim \mathcal{N}(0, \Sigma_{\delta v_A}^C)
$$
и угловая скорость в системе координат $A$ имеет распределение
$$
    \pmb{\omega}^A=\omega^A+\delta \omega^A,
    \quad
    \delta\omega^A\sim \mathcal{N}(0, \Sigma_{\delta \omega}^A)
$$
Будем считать, что матрица перехода от $A$ к $C$ сама является случайной
величиной с распределением
$$
    \pmb{R}_A^C=R_A^C(E+[\delta \pmb{\theta}_A^C]_\times),
    \quad
    \delta\pmb{\theta}_A^C \sim \mathcal{N}(0, \Sigma_{\delta \theta})
$$

Тогда
$$
    \begin{aligned}
        \pmb{v}_B^C
         & =\pmb{v}_A^C+\pmb{\omega}^C \times \vv{AB}^C                                \\
         & =\pmb{v}_A^C+\pmb{R}_A^C \pmb{\omega}^A \times \pmb{R}_A^C p_B^A            \\
         & =\pmb{v}_A^C+\pmb{R}_A^C (\pmb{\omega}^A \times p_B^A)                      \\
         & =v_A^C + \delta \pmb{v}_A^C + R_A^C (E+[\delta \pmb{\theta}_A^C]_\times)
        ((\omega^A + \delta \pmb{\omega}^A) \times p_B^A)                              \\
         & =v_A^C + \delta \pmb{v}_A^C + (R_A^C+R_A^C[\delta \pmb{\theta}_A^C]_\times)
        (\omega^A \times p_B^A + \delta \pmb{\omega}^A \times p_B^A)                   \\
         & =v_A^C + \delta \pmb{v}_A^C
        + R_A^C (\omega^A \times p_B^A)
        + R_A^C[\delta \pmb{\theta}_A^C]_\times (\omega^A \times p_B^A)
        + R_A^C (\delta \pmb{\omega}^A \times p_B^A)
        + R_A^C[\delta \pmb{\theta}_A^C]_\times (\delta \pmb{\omega}^A \times p_B^A)   \\
         & =v_A^C + R_A^C (\omega^A \times p_B^A)
        + \delta \pmb{v}_A^C
        - R_A^C[\omega^A \times p_B^A]_\times \delta \pmb{\theta}_A^C
        - R_A^C [p_B^A]_\times \delta \pmb{\omega}^A
        + o(\Vert \delta\pmb{\omega}^A\Vert)
    \end{aligned}
$$
Следовательно,
\begin{equation}
    \begin{aligned}
        \pmb{v}_B^C        & =v_A^C + R_A^C (\omega^A \times p_B^A) + \delta \pmb{v}_B^C, \\
        \delta \pmb{v}_B^C & \sim\mathcal{N}(
        0,
        \Sigma_{\delta v_A}^C
        +  R_A^C[\omega^A \times p_B^A]_\times \Sigma_{\delta \theta} (R_A^C[\omega^A \times p_B^A]_\times)^T
        + R_A^C[p_B^A]_\times \Sigma_{\delta\omega}^A (R_A^C[ p_B^A]_\times)^T
        )
    \end{aligned}
\end{equation}

\subsection{Нахождение ускорений точек твердого тела в различных системах координат}

Допустим, у нас есть две системы координат $A$ и $B$ связанные с твердым телом.
Пусть в некоторой инерциальной системе координат $C$ твердое тело имеет
угловую скорость $\pmb{\omega}$ и угловое ускорение $\pmb{\epsilon}$.
Пусть $\pmb{a}_A$ линейное ускорение начала системы координат $A$ относительно
системы координат $C$. Аналогично $\pmb{a}_B$ линейное ускорение начала
системы координат $B$ относительно системы координат $C$.

Поскольку $\pmb{v}_B=\pmb{v}_A+\pmb{\omega} \times \vv{AB}$, то
\begin{equation}
    \begin{aligned}
        \pmb{a}_B
         & =\frac{d\pmb{v}_B}{dt}                                \\
         & =\frac{d}{dt} (\pmb{v}_A+\pmb{\omega} \times \vv{AB}) \\
         & =\frac{d \pmb{v}_A}{dt}
        + \frac{d}{dt}(\pmb{\omega} \times \vv{AB})              \\
         & =\pmb{a}_A
        + \frac{d \pmb{\omega}}{dt} \times \vv{AB}
        + \pmb{\omega} \times \frac{d}{dt} \vv{AB}               \\
         & =\pmb{a}_A + \pmb{\epsilon} \times \vv{AB}
        + \pmb{\omega} \times (\pmb{\omega} \times \vv{AB})      \\
    \end{aligned}
\end{equation}

Мы выведем параметры распределения вектора $\pmb{a}_B$ в базисах
$A$ и $C$.

За модельный пример здесь можно взять $A=\mbox{imu}$, $B=\mbox{base link}$
и $C=\mbox{map}$.

Пусть координаты $\pmb{a}_A^A$ вектора $\pmb{a}_A$ в системе координат $A$ имеют
распределение
$$
    \pmb{a}_A^A=a_A^A+\delta \pmb{a}_A^A,
    \quad
    \delta \pmb{a}_A^A \sim \mathcal{N}(0, \Sigma_{\delta a}^A)
$$
и угловая скорость в системе координат $A$ имеет распределение
$$
    \pmb{\omega}^A=\omega^A+\delta \pmb{\omega}^A,
    \quad
    \delta \pmb{\omega}^A\sim \mathcal{N}(0, \Sigma_{\delta \omega}^A)
$$
Угловое ускорение будем считать пренебрежимо малым, тогда
$$
    \begin{aligned}
        \pmb{a}_B^A
         & =\pmb{a}_A^A+\pmb{\omega}^A \times (\pmb{\omega}^A \times p_B^A)                                                   \\
         & =a_A^A + \delta \pmb{a}_A^A
        + (\omega^A+\delta \pmb{\omega}^A) \times ((\omega^A+\delta \pmb{\omega}^A) \times p_B^A)                             \\
         & =a_A^A + \delta \pmb{a}_A^A
        + (\omega^A+\delta \pmb{\omega}^A) \times (\omega^A\times p_B^A+\delta \pmb{\omega}^A\times p_B^A)                    \\
         & =a_A^A + \delta \pmb{a}_A^A
        + \omega^A \times (\omega^A\times p_B^A)+\delta \pmb{\omega}^A \times (\omega^A\times p_B^A)
        + \omega^A\times (\delta \pmb{\omega}^A\times p_B^A)+\delta \pmb{\omega}^A \times (\delta \pmb{\omega}^A\times p_B^A) \\
         & =a_A^A + \omega^A \times (\omega^A\times p_B^A)
        + \delta \pmb{a}_A^A
        - (\omega^A\times p_B^A) \times \delta \pmb{\omega}^A
        - \omega^A\times (p_B^A \times \delta \pmb{\omega}^A)+o(\Vert \delta\omega^A\Vert^2)                                  \\
         & =a_A^A + \omega^A \times (\omega^A\times p_B^A)
        + \delta \pmb{a}_A^A
        - [\omega^A\times p_B^A]_\times \delta \pmb{\omega}^A
        - [\omega^A]_\times [p_B^A]_\times \delta \pmb{\omega}^A+o(\Vert \delta\omega^A\Vert^2)                               \\
         & =a_A^A + \omega^A \times (\omega^A\times p_B^A)
        + \delta \pmb{a}_A^A
        - ([\omega^A\times p_B^A]_\times + [\omega^A]_\times [p_B^A]_\times) \delta \pmb{\omega}^A
        +o(\Vert \delta\omega^A\Vert^2)                                                                                       \\
    \end{aligned}
$$
Следовательно,
\begin{equation}
    \begin{aligned}
        \pmb{a}_B^A        & =\pmb{a}_A^A
        + \omega^A\times (\omega^A \times p_B^A)+\delta \pmb{a}_B^A \\
        \delta \pmb{a}_B^A & \sim\mathcal{N}(
        0,
        \Sigma_{\delta a}^A
        + ([\omega^A\times p_B^A]_\times + [\omega^A]_\times [p_B^A]_\times)
        \Sigma_{\delta\omega}^A
        ([\omega^A\times p_B^A]_\times + [\omega^A]_\times [p_B^A]_\times)^T
        )
    \end{aligned}
\end{equation}

Пусть координаты $\pmb{a}_A^C$ вектора $\pmb{a}_A$ в системе координат $C$ имеют
распределение
$$
    \pmb{a}_A^C=a_A^A+\delta \pmb{a}_A^C,
    \quad
    \delta \pmb{a}_A^C \sim \mathcal{N}(0, \Sigma_{\delta a}^C)
$$
и угловая скорость в системе координат $A$ имеет распределение
$$
    \pmb{\omega}^A=\omega^A+\delta \pmb{\omega}^A,
    \quad
    \delta \pmb{\omega}^A\sim \mathcal{N}(0, \Sigma_{\delta \omega}^A)
$$
Будем считать, что матрица перехода от $A$ к $C$ сама является случайной
величиной с распределением
$$
    \pmb{R}_A^C=R_A^C(E+[\delta \pmb{\theta}_A^C]_\times),
    \quad
    \delta\pmb{\theta}_A^C \sim \mathcal{N}(0, \Sigma_{\delta \theta})
$$
Угловое ускорение будем считать пренебрежимо малым, тогда
$$
    \small{
        \begin{aligned}
            \pmb{a}_B^C
             & =\pmb{a}_A^C+\pmb{\omega}^C \times (\pmb{\omega}^C \times \vv{AB}^C)                                 \\
             & =\pmb{a}_A^C+\pmb{R}_A^C \pmb{\omega}^A \times (\pmb{R}_A^C \pmb{\omega}^A \times \pmb{R}_A^C p_B^A) \\
             & =\pmb{a}_A^C+\pmb{R}_A^C \pmb{\omega}^A \times \pmb{R}_A^C( \pmb{\omega}^A \times p_B^A)             \\
             & =\pmb{a}_A^C+\pmb{R}_A^C (\pmb{\omega}^A \times ( \pmb{\omega}^A \times p_B^A))                      \\
             & =\pmb{a}_A^C
            +R_A^C (E+[\delta \pmb{\theta}_A^C]_\times)
            ((\omega^A + \delta\pmb{\omega}^A) \times
            ( (\omega^A + \delta \pmb{\omega}^A) \times p_B^A))                                                     \\
             & =\pmb{a}_A^C
            +R_A^C (E+[\delta \pmb{\theta}_A^C]_\times)
            (\omega^A \times (\omega^A\times p_B^A)
            +\delta \pmb{\omega}^A \times (\omega^A\times p_B^A)
            + \omega^A\times (\delta \pmb{\omega}^A\times p_B^A)
            +\delta \pmb{\omega}^A \times (\delta \pmb{\omega}^A\times p_B^A))                                      \\
             & =\pmb{a}_A^C
            + (R_A^C+R_A^C[\delta \pmb{\theta}_A^C]_\times)
            (\omega^A \times (\omega^A\times p_B^A)
            - ([\omega^A\times p_B^A]_\times+ [\omega^A]_\times [p_B^A]_\times) \delta \pmb{\omega}^A
            +o(\Vert\delta \pmb{\omega}^A\Vert^2))                                                                  \\
             & =\pmb{a}_A^C
            + R_A^C (\omega^A \times (\omega^A\times p_B^A))
            + R_A^C[\delta \pmb{\theta}_A^C]_\times (\omega^A \times (\omega^A\times p_B^A))
            - R_A^C ([\omega^A\times p_B^A]_\times+ [\omega^A]_\times [p_B^A]_\times) \delta \pmb{\omega}^A
            + o(\Vert\delta \pmb{\omega}^A\Vert)                                                                    \\
             & =\pmb{a}_A^C
            + R_A^C (\omega^A \times (\omega^A\times p_B^A))
            - R_A^C[\omega^A \times (\omega^A\times p_B^A)]_\times \delta \pmb{\theta}_A^C
            - R_A^C ([\omega^A\times p_B^A]_\times+ [\omega^A]_\times [p_B^A]_\times) \delta \pmb{\omega}^A
            + o(\Vert\delta \pmb{\omega}^A\Vert)                                                                    \\
        \end{aligned}
    }
$$
Следовательно,
\begin{equation}
    \small{
        \begin{aligned}
            \pmb{a}_B^C                         & =a_A^C + R_A^C (\omega^A \times (\omega^A \times p_B^A)) + \delta \pmb{a}_B^C,                                                                                              \\
            \delta \pmb{a}_B^C                  & \sim\mathcal{N}(0, \widetilde{\Sigma}_{\delta\theta} + \widetilde{\Sigma}_{\delta\omega}^A)                                                                                 \\
            \widetilde{\Sigma}_{\delta\theta}   & =R_A^C[\omega^A \times (\omega^A\times p_B^A)]_\times\Sigma_{\delta\theta} (R_A^C[\omega^A \times (\omega^A\times p_B^A)]_\times)^T                                         \\
            \widetilde{\Sigma}_{\delta\omega}^A & =R_A^C ([\omega^A\times p_B^A]_\times+ [\omega^A]_\times [p_B^A]_\times)\Sigma_{\delta\omega}^A (R_A^C ([\omega^A\times p_B^A]_\times+ [\omega^A]_\times [p_B^A]_\times))^T
        \end{aligned}
    }
\end{equation}

\end{document}