% chktex-file 1
% chktex-file 19
\documentclass[6pt,pdf,utf8,russian]{beamer}

\usetheme[progressbar=frametitle]{metropolis}
\usepackage{appendixnumberbeamer}

\usepackage{booktabs}
\usepackage[scale=2]{ccicons}

\usepackage{pgfplots}
\usepgfplotslibrary{dateplot}

\usepackage{xspace}
\newcommand{\themename}{\textbf{\textsc{metropolis}}\xspace}

\usepackage[T2A]{fontenc}
\usepackage[utf8]{inputenc}
\usepackage[russian]{babel}

\usepackage{amssymb,amsmath}
\usepackage[utf8]{inputenc}
\usepackage{mathrsfs}
\usepackage[matrix,arrow,curve]{xy}

\title{Алгоритмы регистрации облаков точек}
\author{Немеш Норберт}
\begin{document}

\maketitle

\section{Введение}

\begin{frame}[fragile]{Трехмерные изображения}
    \begin{block}{}
        В последнее десятилетие стали достаточно распространены и доступны
        устройства для получения трехмерных изображений.
    \end{block}

    \pause

    \begin{block}{}
        Основное преимущество --- более точное описание объектов.
    \end{block}

    \pause

    \begin{block}{}
        Применения:
        \begin{itemize}
            \item Восстановление 3D моделей
            \item Инспекция состояния зданий
            \item Локализация в робототехнике
            \item ...
        \end{itemize}
    \end{block}
\end{frame}

\begin{frame}[fragile]{Облака точек}
    \begin{block}{}
        Более распространенное название --- облако точек.
    \end{block}

    \pause

    \begin{figure}
        \includegraphics[width=0.6\textwidth]{images/point_cloud_example.jpg}
    \end{figure}

\end{frame}


\begin{frame}[fragile]{Облака точек. Определение}
    \begin{block}{}
        Точка --- кортеж чисел содержащий координаты точки и возможно некоторые дополнительные
        атрибуты учитывающие специфику сенсора. Координаты точки даются в системе координат сеносра.
    \end{block}

    \pause

    \begin{block}{}
        Пример: координаты $x$, $y$, $z$ и цвет в формате $RGB$.
    \end{block}

    \pause

    \begin{block}{}
        Облако точек -- множество точек и возможно некоторая дополнительная информация
        учитывающая специфику сенсора из которого получено облако.
        \[
            C = (\{p_1,\ldots,p_n\}, \ldots)
        \]
        Пример дополнительной информации: порядок точек в облаке.
    \end{block}
\end{frame}

\section{Типы облаков точек}

\begin{frame}[fragile]{Способы получения облаков точек}
    \begin{block}{}
        Структура облака и свойства точек зависят от способа получения.
    \end{block}

    \pause

    \begin{block}{}
        Пассивный способ --- используем свет приходящий от объекта и
        с помощью дополнительных вычислений строим облако точек.
    \end{block}

    \pause

    \begin{block}{}
        Активный способ --- излучаем свет и по отражениям восстанавливаем облако точек.
    \end{block}
\end{frame}

\begin{frame}[fragile]{Пассивный способ. Structure from motion}
    \begin{block}{}
        Движемся вокруг объекта и делаем снимки, по
        изменению положения ключевых точкек на изображении и известной собственной
        траектории восстанавливаем облако
    \end{block}

    \pause

    \begin{figure}
        \includegraphics[width=0.9\textwidth]{images/structure_from_motion.jpg}
    \end{figure}

\end{frame}

\begin{frame}[fragile]{Пассивный способ. Multi view stereo}
    \begin{block}{}
        Получаем изображение с двух близкорасположенных камер. Зная взаимное расположение камер
        и их внутренние характеристики восстанавливаем облако точек.
    \end{block}

    \pause

    \begin{figure}
        \includegraphics[width=\textwidth]{images/multi_view_stereo.png}
    \end{figure}

\end{frame}

\begin{frame}[fragile]{Активный способ. Time of flight}
    \begin{block}{}
        Выпускаем в различных направлениях сигналы, находим время до возврата, оцениваем расстояние
        до препятствия и восстанавливаем облако точек.

        Пример: лидар
    \end{block}

    \pause

    \begin{figure}
        \includegraphics[width=0.9\textwidth]{images/time_of_flight.jpg}
    \end{figure}

\end{frame}

\begin{frame}[fragile]{Активный способ. Structured light}
    \begin{block}{}
        Проецируем на объект свет опрделенного формата, по искажениям восстанавливаем
        облако точек.
    \end{block}

    \pause

    \begin{figure}
        \includegraphics[width=\textwidth]{images/structured_light.jpg}
    \end{figure}

\end{frame}

\begin{frame}[fragile]{Облака точек. Проблемы описания окружения}
    \begin{block}{}
        Ограничения:
        \begin{itemize}
            \item Окклюзии
            \item Плохое покрытие точками на больших расстояниях
            \item Ограниченный угол обзора
            \item Шумные показания сенсоров, неточные координаты
            \item У лидаров могут быть лишние точки, из-за отраженных лучей
        \end{itemize}
    \end{block}

\end{frame}

\section{Регистрация облаков точек}

\begin{frame}[fragile]{Проблемы описания окружения}

    \begin{block}{}
        Проблема: одного облака недостаточно, надо получать трехмерные изображения с разных позиций.
    \end{block}

    \pause

    \begin{figure}
        \includegraphics[width=\textwidth]{images/multiple_point_clouds.jpg}
    \end{figure}

    \pause

    \begin{block}{}
        Решение: ``склеиваем'' точки в некоторой общей системе координат.

        Более строго этот процесс называется регистрацией облаков.
    \end{block}
\end{frame}

\begin{frame}[fragile]{Постановка задачи}
    \begin{block}{}
        У нас есть два облака точек. Мы хотим найти такое преобразвание,
        которое ``наиболее точно встраивало'' бы одно облако в другое.
    \end{block}

    \pause

    \begin{block}{}
        Применения:
        \begin{itemize}
            \item Локализация робота в карте
            \item Локализация и одновременное построение карты (Simultaneous localization and mapping)
            \item Виртуальная и дополненная реальность
        \end{itemize}
    \end{block}

    \pause

    \begin{block}{}
        Каждое облако задано в системе координат сенсора на момент
        получения данных. Достаточно найти преобразвания между этими системами координат.
    \end{block}

\end{frame}


\begin{frame}[fragile]{Классификация регистраций. По количеству облаков}
    \begin{block}{}
        Попарная регистрация --- нужно зарегистрировать одно облако точек в другом
    \end{block}

    \pause

    \begin{block}{}
        Множественная регистрация --- есть более двух облаков точек, нужно собрать их вместе в одно
        большое облако точек. Наивным образом сводится к попарной регистрации.
    \end{block}

\end{frame}

\begin{frame}[fragile]{Классификация регистраций. По точности}
    \begin{block}{}
        Точная регистрация --- ошибка должна быть минимальна
    \end{block}

    \pause

    \begin{block}{}
        Грубая регистрация --- разрешается большая ошибка. Грубая регистрация обычно используется
        как начальное приближение для точной.
    \end{block}
\end{frame}

\begin{frame}[fragile]{Классификация регистраций. По свойствам объекта}
    \begin{block}{}
        Жесткая регистрация --- преобразование между облаками точек должно сохранять расстояния.
        Чаще всего ищут жесткую регистрацию.
    \end{block}

    \pause

    \begin{figure}
        \includegraphics[width=0.5\textwidth]{images/rigid_registration.jpg}
    \end{figure}
\end{frame}

\begin{frame}[fragile]{Классификация регистраций. По свойствам объекта}
    \begin{block}{}
        Нежесткая регистрация --- преобразование между облаками точек может дополнительно
        использовать растяжение и инверсию.
    \end{block}

    \pause
    \begin{figure}
        \includegraphics[width=0.5\textwidth]{images/nonrigid_registration.jpg}
    \end{figure}

\end{frame}

\begin{frame}[fragile]{Классификация регистраций. По алгоритму нахождения}
    \begin{block}{}
        Условно есть 5 типов алгоритмов регистрации:
        \begin{itemize}
            \item Итеративные методы ближайшей точки (Iterative closest point)
            \item Вероятностные методы
            \item Методы ключевых точек (Interest points)
            \item Методы машинного обучения
            \item Все остальное
        \end{itemize}
    \end{block}
\end{frame}

\begin{frame}[fragile]{Проблемы и ограничения регистраций}
    \begin{block}{}
        С регистрацией облаков точек много проблем и нюансов
        \begin{itemize}
            \item Облака точек могут иметь разное количество точек
            \item Облака точек могут лишь частично перекрываться
            \item Облака точек содержат выбросы и координаты зашумлены
            \item Нахождение регистрации может занимать очень вычислительных ресурсов
            \item Решение задачи, вообще говоря, неединственно
            \item Нет однозначного определения близости облаков точек
        \end{itemize}
    \end{block}
\end{frame}

\section{Преобразования между облаками точек}

\begin{frame}[fragile]{Аффинные преобразвания}
    \begin{block}{}
        Любая регистрация определяется аффинным преобразованием
        \[
            T:\mathbb{R}^3\to\mathbb{R}^3:x \mapsto R x + t,
        \]
        где $t\in\mathbb{R}^3$, $R\in Mat_{3,3}(\mathbb{R})$.
    \end{block}

    \pause

    \begin{block}{}
        Жесткие регистрации описываются изометриями, т.е. сдвигом и вращением: $R^TR=I$, $\det(R)=1$
    \end{block}

    \pause

    \begin{block}{}
        Нежесткие регистрации описываются невырожденными аффинными преобразованиями: $\det(R)\neq 0$.
    \end{block}

\end{frame}

\section{Итеративные методы ближайшей точки (ICP методы)}

\begin{frame}[fragile]{Схема ICP алгоритмов}
    \begin{block}{}
        Пусть нам заданы два облака точек $P=\{p_1,\ldots,p_n\}$ и $Q=\{q_1,\ldots,q_m\}$.
        Облако $P$ будем называть референсным, а облако $Q$ текущим.
    \end{block}

    \pause

    \begin{block}{}
        Задача регистрации заключается в нахождении аффинного преобразвания $T$ такого, что облако
        $T(Q)\stackrel{def}{=}\{T(q_1),\ldots,T(q_n)\}$ будет ``мало отличаться'' от $P$.
    \end{block}

    \begin{block}{}
        Отличие облаков оценивается некотрой функцией потерь $L$, а цель алгоритма 
        минимизировать эту функцию итеративно подбирая преобразование $T$.
    \end{block}

\end{frame}

\begin{frame}[fragile]{Схема ICP алгоритмов}
    \begin{block}{}
        Алгоритм ICP состоит из шагов
        \begin{itemize}
            \item Выбор точек из облаков (point selection)
            \item Нахождение связок между точками (point matching)
            \item Отклонение связок (point rejection)
            \item Минимизация ошибки
        \end{itemize}
        Шаги повторяются пока алгоритм не сойдется (например, превышено максимальное количество итераций)
    \end{block}

    \begin{figure}
        \includegraphics[width=0.5\textwidth]{images/icp_scheme.png}
    \end{figure}

\end{frame}

\begin{frame}[fragile]{Схема ICP алгоритмов. Выбор точек}
    \begin{block}{}
        Выбор точек (downsampling) в ICP алгоритме помогает
        \begin{itemize}
            \item Снизить количество точек которые надо обрабатывать
            \item Выкинуть шумы и выбросы для увеличения точности алгоритма
        \end{itemize}
    \end{block}

    \pause

    \begin{figure}
        \includegraphics[width=0.5\textwidth]{images/icp_downsampling.jpg}
    \end{figure}

\end{frame}

\begin{frame}[fragile]{Схема ICP алгоритмов. Выбор точек}

    \begin{block}{}
        Стратегии выбора точек:
        \begin{itemize}
            \item Случайный выбор (random sampling)
            \item Каждый раз выбираем новую точку которая находится на некотором расстоянии от предыдущей (distance limit)
            \item Разобъем пространство на ячейки в каждой ячеке все точки заменим одной - средним всех точек ячейки (uniform sampling)
        \end{itemize}
    \end{block}

\end{frame}

\begin{frame}[fragile]{Схема ICP алгоритмов. Нахождение связок}

    \begin{block}{}
        Нахождение связок необходимо для вычисления функции потерь.
    \end{block}

    \pause

    \begin{block}{}
        В связку попадают точка $p$ из референсного облака и точка $q$ из текущего облака, причем
        просле преобразования $T$ точка $T(q)$ --- ближайшая к $p$ в смысле какой-то метрики.
    \end{block}

    \begin{figure}
        \includegraphics[width=0.6\textwidth]{images/correspondence_example.jpg}
    \end{figure}


\end{frame}

\begin{frame}[fragile]{Схема ICP алгоритмов. Нахождение связок}

    \begin{block}{}
        Лобовое решение имело бы сложность $O(n\cdot m)$, где $n$ --- количество точек в референсном облаке, $m$ --- в текущем облаке.
    \end{block}

    \pause

    \begin{block}{}
        Конечно, нужно что-то побыстрее. Обычно ускорения добиваются предварительным подсчетом над точками референсного облака.
        Строится какая-нибудь структура данных которая позволит бысро находить соседнюю точку для заданной. Обычно получаем
        ускорение до $O(m \log n)$.
        Примеры таких структур данных:
        \begin{itemize}
            \item multi-z-buffer
            \item kd-tree
            \item octree
        \end{itemize}
    \end{block}

\end{frame}

\begin{frame}[fragile]{Схема ICP алгоритмов. Нахождение связок. Multi-z-buffer}

    \begin{block}{}
        Принцип работы:
        \begin{itemize}
            \item Разрезаем прострнство на ячейки (воксели)
            \item Для каждой точки за $O(1)$ находим ее ячейку
            \item В ячейке любым способом ищем соседей
        \end{itemize}
    \end{block}

    \pause

    \begin{block}{}
        \begin{figure}
            \includegraphics[width=0.4\textwidth]{images/multi_z_buffer.png}
        \end{figure}
    \end{block}

\end{frame}

\begin{frame}[fragile]{Схема ICP алгоритмов. Нахождение связок. Kd-tree}

    \begin{block}{}
        Принцип работы:
        \begin{itemize}
            \item Разрезаем пространство `пополам' каждый раз переключая нормаль плоскости разрезания ($Ox$, $Oy$, $Oz$)
            \item Этот процесс описывает построение некоторого бинарного дерева
            \item Поиск ближайшей точки похож на поиск в бинарном дереве
        \end{itemize}
        \begin{figure}
            \includegraphics[width=0.7\textwidth]{images/kd_tree.jpg}
        \end{figure}
    \end{block}

\end{frame}

\begin{frame}[fragile]{Схема ICP алгоритмов. Нахождение связок. Octree}

    \begin{block}{}
        Принцип работы:
        \begin{itemize}
            \item Разрезаем пространство на 8 октантов
            \item В тех октантах где есть точки повторяем предыдущйи шаг
            \item Поиск ближайшей точки похож на поиск в 8-арном дереве
        \end{itemize}
        \begin{figure}
            \includegraphics[width=0.7\textwidth]{images/octree.png}
        \end{figure}
    \end{block}

\end{frame}

\begin{frame}[fragile]{Схема ICP алгоритмов. Удаление связок}
    \begin{block}{}
        Одной точке из текущего облака может соответствовать несколько точек из референсного.
        Нам это будет мешать, от таких связок надо избавляться.
    \end{block}

    \begin{figure}
        \includegraphics[width=0.6\textwidth]{images/correspondence_erasure_1.jpg}
    \end{figure}

\end{frame}

\begin{frame}[fragile]{Схема ICP алгоритмов. Удаление связок}
    \begin{block}{}
        Существует несколько алгоритмов удаления связок:
        \begin{itemize}
            \item Удаление определенного процента самых длинных связок
            \item Удаление связок с длиной больше заданной
            \item Удаление связок в которых есть задваивание точек
            \item Удаление связок с длиной больше медианной
            \item Удаление связок c несогласованными нормалями
            \item ...
        \end{itemize}
    \end{block}

    \pause

    \begin{block}{}
        Можно применить сразу несколько алгоритмов
    \end{block}

\end{frame}

\begin{frame}[fragile]{Схема ICP алгоритмов. Удаление связок по длине}
    \begin{block}{}
        Удаление связок с длиной больше заданной интуитивно понятно, но требует подбора
        максмального значения. Можно высчитывать медианную длину связок по всему облаку
        или по окрестности
    \end{block}

    \pause

    \begin{block}{}
        \begin{figure}
            \includegraphics[]{images/correspondence_erasure_2.jpg}
        \end{figure}
    \end{block}

\end{frame}

\begin{frame}[fragile]{Схема ICP алгоритмов. Удаление связок c задваиванием}
    \begin{block}{}
        Удаление связок с задваиванием, гарантирует, что у нас каждой точке текущего облака будет
        соответствовать одна точка из референсного. Не факт, что это будет ``удачная'' точка.
    \end{block}

    \pause

    \begin{block}{}
        \begin{figure}
            \includegraphics[]{images/correspondence_erasure_3.jpg}
        \end{figure}
    \end{block}

\end{frame}

\begin{frame}[fragile]{Схема ICP алгоритмов. Удаление связок по нормалям}
    \begin{block}{}
        Удаление связок с по нормалям, убирает точки у которых нормали к поверхности ``сильно'' разнонаправлены.
    \end{block}

    \pause

    \begin{block}{}
        \begin{figure}
            \includegraphics[]{images/correspondence_erasure_4.jpg}
        \end{figure}
    \end{block}

\end{frame}

\begin{frame}[fragile]{Схема ICP алгоритмов. Функция потерь}
    \begin{block}{}
        Ошибку регистрации облаков можно посчитать когда для каждой точки $q_i$ из текущего облака $Q$ найдена связанная
        точка $p_i$ из референсного облака $P$.
    \end{block}

    \pause

    \begin{block}{}
        Обзначим эту ошибку $L(P, Q, T)$, тогда искомое преобразование для жесткой регистрации находится из
        оптимизационной задачи
        \[
            \hat{T}=\operatorname{argmin}\limits_{T\in \operatorname{Isometry}(\mathbb{R}^3)} L(P, Q, T)
        \]
    \end{block}

\end{frame}

\begin{frame}[fragile]{Схема ICP алгоритмов. Функции потерь}
    \begin{block}{}
        Типы функций потерь:
        \begin{itemize}
            \item Точка к точке \[L(P,Q,T)=\sum_{i=1}^m\Vert p_i - T(q_i)\Vert^2 \]
            \item Точка к плоскости \[L(P,Q,T)=\sum_{i=1}^m(n_i (p_i - T(q_i)))^2 \]
            \item Точка к проекции. Находм ближашую точку в референсном облаке по направлению собственной нормали
            \item ...
        \end{itemize}
    \end{block}
\end{frame}

\begin{frame}[fragile]{Известные ICP алгоритмы}
    \only<1>{
        \begin{block}{}
            Go-ICP --- ищет оптимальное преобразование по всему пространству $Isometry(\mathbb{R}^3)$
            используя метод ветвей и границ
        \end{block}
        \begin{block}{}
            \begin{figure}
                \includegraphics[width=0.7\textwidth]{images/go_icp.png}
            \end{figure}
        \end{block}
    }

    \pause

    \only<2>{
        \begin{block}{}
            GICP (Generalized-ICP) --- вводит вероятностную модель на распределение точек в облаке. Варьируя параметры вероятностной модели
            можно получить ICP алгоритмы с функцией потерь точка к точке или точека к плоскости
        \end{block}
        \begin{block}{}
            \begin{figure}
                \includegraphics[width=0.7\textwidth]{images/gicp.jpg}
            \end{figure}
        \end{block}
    }

    \pause

    \only<3>{
        \begin{block}{}
            VGICP (voxelized GICP) --- модфикация GICP в которой пространство разбивается на воксели и по точкам в вокселе
            оцениваются параметры вероятностного распределения GICP. Хорошо адаптируется для видеокарт.
        \end{block}
        \begin{block}{}
            \begin{figure}
                \includegraphics[width=0.7\textwidth]{images/gicp.jpg}
            \end{figure}
        \end{block}
    }
\end{frame}

\begin{frame}[fragile]{ICP алгоритмы}
    \only<1>{
        \begin{block}{}
            Плюсы:
            \begin{itemize}
                \item Хорошо параллелизуются
                \item Стабильные и робастные
                \item Не требуют feature-инжениринга
                \item Обобщаются на многомерный случай
                \item Могут быть использованы в связке с другими алгоритмами
            \end{itemize}
        \end{block}
    }
    \pause

    \only<2>{
        \begin{block}{}
            Минусы:
            \begin{itemize}
                \item Требуется начальное приближение
                \item Требуются предварительные вычисления (например построение kd-дерева)
                \item Медленные из-за необходимости построения связок
                \item Во время оптимизации могут застрять в локальном минимуме
            \end{itemize}
        \end{block}
    }

\end{frame}

\section{Вероятностные методы}

\begin{frame}[fragile]{Вероятностные методы}
    \only<1>{
        \begin{block}{}
            Основная идея:
            \begin{itemize}
                \item Все облака являются выборками из некоторого вероятностного распределения
                \item Обычно вероятностное распределение это смесь Гауссовых распределений (Gaussian mixture model)
                \item Оптимальное преобразвание минимизирует ``расстояние'' между этими распределениями
            \end{itemize}
        \end{block}
    }

    \pause

    \only<2>{
        \begin{block}{}
            Мы обсудим два метода:
            \begin{itemize}
                \item Смесь гауссовских распределений (Gaussian mixture model, GMM)
                \item Преобразование нормальных распределений (Normal distributions transform, NDT)
            \end{itemize}
        \end{block}
    }
\end{frame}

\begin{frame}[fragile]{Вероятностные методы. GMM}
    \only<1>{
        \begin{block}{}
            Для начала нам потребуется несколько предварительных сведений
        \end{block}
    }
    \only<2>{
        \begin{block}{}
            Любое облако $C$ можно описать как смесь гауссовых распределений.

            Допустим, что облако точек $C$ описывается смесью $N_C$ гауссовских распределений. Пусть
            \begin{itemize}
                \item вес $i$-го рапределения -- $\phi_i^C$
                \item плотность $i$-го гауссовского распределения
                      \[
                          f_i^C(r)=\Gamma(r; \mu_i^c, \Sigma_i^C)
                      \]
            \end{itemize}
            Тогда плотность распределения облака
            \[
                f^C(r)=\sum_{i=1}^{N_C}\phi_i^C f_i^C(r)
            \]
        \end{block}
    }

    \pause

    \only<3>{
        \begin{block}{}
            Если к облаку применить преобразование $T$ задаваемое сдвигом $t$ и матрицей $R$, то плотность распределения
            изменится следующим образом
            \[
                f^{T(C)}(r)=\sum_{i=1}^{N_c} \phi_i^C f_i^{T(C)}(r)
            \]
            где
            \[
                f_i^{T(C)}(r)=\Gamma(r; \mu_i^C + t, R \Sigma_i^C R^T)
            \]
        \end{block}
    }

    \pause

    \only<4>{
        \begin{block}{}
            Мы хотим найти преобразование $T$ при котором распределение $f^P$ референсного облака как можно меньше отличалось
            от распределения $f^{T(Q)}$ преобразованного текущего облака $Q$.
        \end{block}
        \begin{block}{}
            В методе GMM расстояние между вероятностными распределениями вычисляется как интегральная $L_2$ норма
            \[
                d(f_1, f_2)=\int_{\mathbb{R}^3} (f_1(r)-f_2(r))^2dr
            \]
        \end{block}
    }

    \pause

    \only<5>{
        \begin{block}{}
            В итоге мы получаем задачу минимизации
            \[
                \hat{T}=\operatorname{argmin}\limits_{T\in\operatorname{Isometry(\mathbb{R}^3)}}\int_{\mathbb{R}^3} (f^{T(Q)}(r)-f^P(r))^2dr
            \]
        \end{block}
        \begin{block}{}
            Поскольку величины
            \[
                \int_{\mathbb{R}^3} f^P(r)^2dr \quad \mbox{ и } \quad  \int_{\mathbb{R}^3} f^{T(Q)}(r)^2dr
            \]
            не зависят от $T$ то задача минимизации сводится к следующей
            \[
                \hat{T}=\operatorname{argmin}\limits_{T\in\operatorname{Isometry(\mathbb{R}^3)}}\int_{\mathbb{R}^3}(-2\cdot f^{T(Q)}(r) \cdot f^P(r))dr
            \]
        \end{block}
    }

    \pause

    \only<6>{
        \begin{block}{}
            Эта задача может быть решена $EM$-алгоритмом или с помощью $SVM$.

            Алгоритм вычислительно тяжелый. Поэтому люди искали альтернативы
        \end{block}
    }
\end{frame}


\begin{frame}[fragile]{Вероятностные методы. NDT}
    \only<1>{
        \begin{block}{}
            Идея метода NDT похожа на GMM, но мы делаем несколько инженерных упрощений:
            \begin{itemize}
                \item Мы разбиваем референсное облако на воксели и считаем, что внутри вокселя точки распределены как смесь нормального и равномерного распределения
                \item Параметры нормального распределения оцениваются по координатам точек вокселя
                \item Для нахождения оптимального преобразвания мы применяем метод максимального правдоподобия к облаку $T(Q)$ и используем плотность распределения
                      референсного облака
            \end{itemize}
        \end{block}
    }
    \only<2>{
        \begin{block}{}
            Обсудим алгоритм подробнее.

            Пусть референсное облако разбито на воксели $\{V_1,\ldots,V_N\}$.

            Рассмотрим произвольный воксель $V$. Пусть он содержит точки ${x_1,\ldots,x_{k_V}}$

            Будем считать, что внутри вокселя распределение точек является смесью нормального и равномерного.
            \[
                p_V(r)=\phi_1 \Gamma(r; \mu_V, \Sigma_V) + \phi_2 p_2
            \]
            причем
            \[
                \mu_V=\frac{1}{k_V}\sum_{i=1}^{k_V} x_i \quad\;\;\;
                \Sigma_V = \frac{1}{k_V-1}\sum_{i=1}^{k_V}(x_i-\mu_V)(x_i-\mu_V)^T
            \]
        \end{block}
    }

    \pause

    \only<3>{
        \begin{block}{}
            Искомое преобразование $T$ должно максимизировать вероятность того что облако $T(Q)$ получено из
            распределения облака $P$.

            Таким образом мы получаем задачу оптимизации
            \[
                \hat{T}=\operatorname{argmin}\limits_{T\in\operatorname{Isometry(\mathbb{R}^3)}} \prod_{i=1}^m p_{V(T(q_i))}(T(q_i))
            \]
            где $V(r)$ это воксель в который попадает точка с радиус вектором $r$
        \end{block}
    }

    \pause

    \only<4>{
        \begin{block}{}
            В алгоритме NDT минимизируемую функцю $p_V$ заменяют на функцию вида
            \[
                \tilde{p}_V(r) = c_1 \exp\left( \frac{-c_2}{2}(r-\tilde{\mu}_v)^T \tilde{\Sigma}_V^{-1}(r-\tilde{\mu}_V)\right)
            \]

            Она ведет себя похожим образом, и более того такая ``замена'' позволяет записать простые аналитические выражения
            для градиента и гессиана минимизируемой функции.

            Это в свою очередь позволяет быстро решить задачу оптимизации с помощью методов второго порядка.
        \end{block}
    }

    \pause

\end{frame}

\begin{frame}[fragile]{Вероятностные алгоритмы}
    \only<1>{
        \begin{block}{}
            Плюсы:
            \begin{itemize}
                \item Хорошо параллелизуются
                \item Стабильные и робастные
                \item Не требуют feature-инжиниринга
                \item Обобщается на многомерный случай
                \item Могут быть использованы в связке с другими алгоритмами
                \item Работают быстрее ICP так как не требуют вычисления связок точек
            \end{itemize}
        \end{block}
    }
    \pause

    \only<2>{
        \begin{block}{}
            Минусы:
            \begin{itemize}
                \item Требуется начальное приближение
                \item Во время оптимизации могут застрять в локальном минимуме
            \end{itemize}
        \end{block}
    }

\end{frame}



\section{Методы ключевых точек}

\begin{frame}[fragile]{Методы ключевых точек}
    \begin{block}{}
        Часто в облаке не все точки полезны и нужны.

        Пример: большая часть точек на ровной стене не несет полезной информации.
    \end{block}

    \pause

    \begin{block}{}
        Идея: найти ``интресные'' точки и оставить в облаке только их.
    \end{block}

    \pause

    \begin{block}{}
        Какие точки ``интересные''?
        \begin{itemize}
            \item Углы объектов
            \item Переход цвета или интенсивности
            \item Границы объектов
        \end{itemize}
    \end{block}

    \pause

    \begin{block}{}
        Далее ``интересные'' точки будет называть ключевыми.
    \end{block}
\end{frame}

\begin{frame}[fragile]{Методы ключевых точек. Дескрипторы}
    \begin{block}{}
        Какие найти ключевые точки?
        \begin{itemize}
            \item Посчитать геометрические характеристики ее окретсности
            \item Переход цвета или интенсивности
            \item Границы объектов
        \end{itemize}
    \end{block}

    \pause

    \begin{block}{}
        Такие характеристики называются локальными дескрипторами.
    \end{block}

    \pause

    \begin{block}{}
        Основное требование к дескрипторам --- они должны быть инвариантны относительно преобразований используемых в регистрации.
        Так как мы чаще всего используем жесткие регистрации, то требуется инвариантность относительно изометрий.
    \end{block}

\end{frame}

\begin{frame}[fragile]{Методы ключевых точек. Простешие дескрипторы}
    \begin{block}{}
        Примеры локальных дескрипторов:
        \begin{itemize}
            \item нормаль к поверхности в точке
            \item кривизны поверхности в точке
            \item плотность точек в окрестности
        \end{itemize}
    \end{block}

    \pause

    \begin{block}{}
        \begin{figure}
            \includegraphics[width=0.4\textwidth]{images/simple_descriptor.jpg}
        \end{figure}
    \end{block}

\end{frame}


\begin{frame}[fragile]{Методы ключевых точек. Продвинутые дескрипторы}
    \only<1>{
        \begin{block}{}
            Придумано множество дескрипторов:
            \begin{itemize}
                \item Point Feature Histogram (PFH)
                \item Signature of Histograms OrienTations (SHOT)
                \item Intinsic shape signature (ISS)
                \item BOrder Aware Repeatable Directions (BOARD)
                \item ...
            \end{itemize}
        \end{block}
    }

    \pause

    \only<2>{
        \begin{block}{}
            Для примера обсудим SHOT дескриптор. Он вычисляется следующим образом:
            \begin{itemize}
                \item Для каждой точки облака подсчитать какую либо характеристику (например угол между нормалью в точке и осью z)
                \item Для каждой точки выбрать окрестность и по всем точкам из окрестности построить гистограмму выбранной характеристики
                \item Данная гистограмма и будет дескрипторам точки
            \end{itemize}
        \end{block}
    }

    \pause

    \only<3>{
        \begin{block}{}
            \begin{figure}
                \includegraphics[width=\textwidth]{images/shot_descriptor.jpg}
            \end{figure}
        \end{block}
    }
\end{frame}

\begin{frame}[fragile]{Методы ключевых точек. Регистрация}
    \only<1>{
        \begin{block}{}
            \begin{itemize}
                \item Как только получены дескрипторы всех точек можно начинать процесс регистрации.
                \item Теперь сопоставлять между собой мы будем не сами координаты точек, а их локальные дескрипторы.
                \item Сопоставляя дескрипторы мы получаем связки между точками
                \item Перед сопоставлением дескрипторов можно уменьшить количество точек в референсном и текущем облаке, например с помощью кластеризации.
                \item Как только получены связки можно вернутся к шагам алгоритма ICP
            \end{itemize}
        \end{block}
    }
\end{frame}

\begin{frame}[fragile]{Методы ключевых точек}
    \only<1>{
        \begin{block}{}
            Плюсы:
            \begin{itemize}
                \item Дают хорошее начальное приближение
                \item Большой набор готовых дескрипторов
                \item Можно использовать как начальное приближение для более точных алгоритмов регистрации
            \end{itemize}
        \end{block}
    }

    \pause

    \only<2>{
        \begin{block}{}
            Минусы:
            \begin{itemize}
                \item Не дают точного решения
                \item Медленно работают
            \end{itemize}
        \end{block}
    }

\end{frame}

\begin{frame}[fragile]{}
    \begin{block}{}
        В этом обзоре мы не обсудили регистрации точек методами машинного обучения, но это слишком большая тема 
        чтобы уместить ее в этой презентации. Здесь стоит сказать, что лидерами в этой области являются нейронные
        сети.
        \newline
        \newline
        \newline
        Спасибо за внимание!
    \end{block}

\end{frame}

\end{document}