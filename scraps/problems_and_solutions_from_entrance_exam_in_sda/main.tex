\documentclass[12pt]{article}
\usepackage[left=2cm,right=2cm,top=2cm,bottom=2cm,bindingoffset=0cm]{geometry}
\usepackage{amssymb,amsmath}
\usepackage[utf8]{inputenc}
\usepackage{mathrsfs}

\newtheorem{problem}{Problem}[subsection]
\newenvironment{solution}{\par $\triangleleft$}{$\triangleright$}


\begin{document}

\begin{center}

    \Large \textbf{Problems and solutions from entrance exam in Yandex school of
        data analysis}\\[0.5cm]
    \small {Nemesh N. T.}\\[0.5cm]

\end{center}

\section{Interview questions}

\subsection{Algebra}

\begin{problem} Count all bases of $n$-dimensional vector space over field
$\mathbb{F}_k$.
\end{problem}
\begin{solution} There are $k^n-1$ ways to choose the first vector, Now there
    are $k^n-k$ ways to choose the second linearly independent vector. Then
    there are $k^n-k^2$ ways to choose the third linearly independent vector and
    etc. Hence there are
    $$
        (k^n-1)(k^n-k)(k^n-k^2)\cdot
        \ldots
        \cdot(k^n-k^{n-1})
    $$
    ways to choose $n$ linearly independent vectors in ${(\mathbb{F}_k)}^n$,
    i.e.\ to construct a basis.
\end{solution}

\begin{problem} Find the probability that a random matrix of size $n\times n$
with entries in the finite field $\mathbb{F}_k$ will be non-singular.
\end{problem}
\begin{solution} The total number of matrices of size $n\times n$ with entries
    in $\mathbb{F}_k$ is $k^{n^2}$. Now we want to compute the number of
    non-singular matrices. Rows of these matrices are linearly independent.
    There are $k^n-1$ ways to choose the first row, i.e.\ the first vector. Now
    there are $k^n-k$ ways to choose the second vector. Then there are $k^n-k^2$
    ways to choose the third vector and  etc. Hence there are
    $(k^n-1)\cdot\ldots\cdot(k^n-k^{n-1})$  % chktex 11
    ways to choose $n$ linearly independent rows. This is the number of
    non-singular matrices. Hence the probability is
    $$
        \frac{(k^n-1)\cdot \ldots\cdot(k^n-k^{n-1})}{k^{n^2}}  % chktex 11
    $$
\end{solution}

\begin{problem} Find all $f\in \mathbb{R}[x]$ such that $xf(x-2)=(x-2)f(x-1)$
\end{problem}
\begin{solution} After substitution $x=2$ we get $2f(0)=0$, i.e. $f(0)=0$. This
    means that $f(x)=xg(x)$ for some $g\in\mathbb{R}[x]$. In this case our
    equation can be rewritten as $x(x-2)g(x-2)=(x-2)(x-1)g(x-1)$, which is
    equivalent to $xg(x-2)=(x-1)g(x-1)$. After substitution $x=1$ we get
    $g(-1)=0$, hence $g(x)=(x+1)h(x)$ for some $h\in\mathbb{R}[x]$. Now the last
    equation can be rewritten as $x(x-1)h(x-2)=(x-1)xh(x-1)$, i.e.
    $h(x-2)=h(x-1)$. In particular $h(\mathbb{Z})=\{h(0)\}$. The only polynomial
    attaining the same value infinitely many times is the constant polynomial,
    i.e. $h(x)=C$. Finally, $f(x)=xg(x)=x(x+1)h(x)=Cx(x+1)$.
\end{solution}

\begin{problem} Find all $f\in \mathbb{R}[x]$ such that $(x-3)f(x-1)=(x-2)f(x)$
\end{problem}
\begin{solution} Substitute $x=3$ to get $f(3)=0$, hence $f(x)=(x-3)f_1(x)$ for
    some $f_1\in\mathbb{R}[x]$. Then our equation can be rewritten as
    $(x-3)(x-4)f_1(x-1)=(x-2)(x-3)f_1(x)$, so $(x-4)f_1(x-1)=(x-2)f_1(x)$. Again
    substitute $x=4$ to get $f_1(4)=0$, hence $f_1(x)=(x-4)f_2(x)$ for some
    $f_2\in\mathbb{R}[x]$. We can repeat this process and for any
    $n\in\mathbb{N}$ to construct a polynomial $f_n\in\mathbb{R}[x]$ such that
    $(x-3-n)f_n(x-1)=(x-2)f_n(x)$ and $f_{n-1}(x)=(x-n-2)f_{n}(x)$. In this case
    we have $f(x)=(x-3)(x-4)\cdot\ldots\cdot(x-n-2)f_n(x)$  % chktex 11
    for all $n\in\mathbb{N}$. This is possible iff $f(x)=0$.
\end{solution}





















\newpage

\subsection{Linear algebra}

\begin{problem} Given a Jordan matrix of an operator count all its invariant
subspaces
\end{problem}
\begin{solution} By Jordan's decomposition
    $$
        \mathcal{A}=\bigoplus_{i=1}^m \mathcal{A}_i\qquad
        \mathcal{A}_i=\bigoplus_{l=1}^{m_i} \mathcal{J}_{n_i,l}(\lambda_i)
    $$
    where ${(\lambda_i)}_{i\in\mathbb{N}_m}$ are eigenvalues of $\mathcal{A}$
    and
    $$
        V_{\lambda_i}(\mathcal{A})
        :=\operatorname{Ker}(\mathcal{A}-\lambda_i\mathcal{E})
        \qquad
        k_i:=\dim V_{\lambda_i}(\mathcal{A})
    $$
    If $k_i>1$ for some $i\in\mathbb{N}_m$, then consider linearly independent
    vectors $e_1,e_2\in V_{\lambda_i}(\mathcal{A})$. Clearly,
    $\operatorname{span}\{e_1\}$, $\operatorname{span}\{e_2\}$ are an invariant
    subspaces of $\mathcal{A}$. For all $\mu_1,\mu_2\in\mathbb{C}$ we see that
    $\operatorname{span}\{\mu_1e_1+\mu_2e_2\}$ is an invariant subspace of
    $\mathcal{A}$ too. Hence we have infinitely many invariant subspaces.

    If $k_i=1$ for all $i\in\mathbb{N}_m$, then
    $\mathcal{A}_i=\mathcal{J}_{n_i}(\lambda_i)$. For each $i\in\mathbb{N}_m$
    the subspaces
    $U_{i,r}=\operatorname{Ker}{(\mathcal{A}-\lambda_i\mathcal{E})}^r$ with
    $r\in\mathbb{N}_{n_i}^*$ are an increasing chain of invariant subspaces. An
    arbitrary invariant subspace is of the form
    $$
        \bigoplus_{i=1}^m U_{i,r_i}
    $$
    where $r_i\in\mathbb{N}_{n_i}^*$. Consequently, th number of invariant
    subspaces is
    $$
        \prod_{i=1}^m(n_i+1)
    $$
\end{solution}


\begin{problem} Give an example of a linear operator with only one eigenvector
\end{problem}
\begin{solution} Consider Jordan's matrix $\mathcal{J}_n(\lambda)$. Since
    $$
        \det(\mathcal{J}_n(\lambda)-t\mathcal{E})={(\lambda-t)}^n
    $$
    then we have only one eigenvalue $\lambda$ with multiplicity $n$. Now we
    define eigenvectors from the equation
    $$
        (\mathcal{J}_n(\lambda)-\lambda\mathcal{E})x=0
    $$
    This is equivalent to $\mathcal{J}_n(0)x=0$. It is easily seen that
    solutions of this equation are of the form
    $$
        x={( c, 0, 0, \ldots, 0 )}^{tr}\qquad c\in\mathbb{C}
    $$
    Hence $\mathcal{J}_n(\lambda)$ has only one eigenvector.
\end{solution}


\begin{problem} Give an example of a linear operator without eigenvalues, but
with proper innvariant subspaces.
\end{problem}
\begin{solution} By $\mathcal{R}(\alpha)$ we denote the linear operator
    corresponding to rotation by angle $\alpha$. Its matrix is
    $$
        [ \mathcal{R}(\alpha)]
        =\begin{pmatrix}
            \cos\alpha & -\sin\alpha \\
            \sin\alpha & \cos\alpha
        \end{pmatrix}
    $$
    Now we find
    $$
        \det(\mathcal{R(\alpha)}-t\mathcal{E})={(t-\cos\alpha)}^2+\sin^2\alpha
    $$
    Which is strictly greater than zero provided $\alpha\notin\pi \mathbb{Z}$.
    Now consider angles
    ${(\alpha_i)}_{i\in\mathbb{N}_m}\subset\mathbb{R}\setminus\pi\mathbb{Z}$. We
    define a linear operator
    $$
        \mathcal{R}=\bigoplus_{i=1}^m\mathcal{R}(\alpha_i)
    $$
    Since $[\mathcal{R}]$ is a block-diagonal matrix, then $\mathcal{R}$ admits
    invariant subspaces. In fact it has at least $2^m$ invariant subspaces. But
    $\mathcal{R}$ has no eigenvalues. Indeed
    $$
        \det(\mathcal{R}-t\mathcal{E})
        =\prod_{i=1}^m\det(\mathcal{R}(\alpha_i)-t\mathcal{E})
        =\prod_{i=1}^m({(t-\cos\alpha_i)}^2+\sin^2\alpha_i)>0
    $$
    because of the choice of the angles ${(\alpha_i)}_{i\in\mathbb{N}_m}$.
\end{solution}


\begin{problem} Compute ${\mathcal{J}_n(\lambda)}^k$ for all $k\in\mathbb{N}$.
\end{problem}
\begin{solution} Note that for commuting operators $\mathcal{A}$ and
    $\mathcal{B}$ and for all $k\in\mathbb{N}$ we have the binomial identity
    $$
        {(\mathcal{A}+\mathcal{B})}^k
        =\sum_{s=0}^k \binom{k}{s}\mathcal{A}^s\mathcal{B}^{k-s}
    $$
    Hence,
    $$
        {\mathcal{J}_n(\lambda)}^k
        ={(\mathcal{J}_n(0)+\lambda\mathcal{E})}^k
        =\sum_{s=0}^k
        \binom{k}{s}{\mathcal{J}_n(0)}^s{(\lambda\mathcal{E})}^{k-s}
        =\sum_{s=0}^k\binom{k}{s}\lambda^{k-s}{\mathcal{J}_n(0)}^s
    $$
    Now it remains to note that ${[{\mathcal{J}_n(0)}^s]}_{i,j}=\delta_{i,j-s}$
    for all $s\in\mathbb{N}^*$. We can compute arbitrary entry of
    ${\mathcal{J}_n(\lambda)}^k$. Indeed,
    $$
        {[ {\mathcal{J}_n(\lambda)}^k]}_{i,j}
        =\sum_{s=0}^k\binom{k}{s}\lambda^{k-s}\delta_{i,j-s}
        =\binom{k}{j-i}\lambda^{k-(j-i)}
    $$
    For example
    $$
        [ {\mathcal{J}_3(\lambda)}^4]=
        \begin{pmatrix}
            \binom{4}{0}\lambda^4  &
            \binom{4}{1}\lambda^3  &
            \binom{4}{2}\lambda^2    \\
            \binom{4}{-1}\lambda^5 &
            \binom{4}{0}\lambda^4  &
            \binom{4}{1}\lambda^3    \\
            \binom{4}{-2}\lambda^6 &
            \binom{4}{-1}\lambda^5 &
            \binom{4}{0}\lambda^4
        \end{pmatrix}
        =
        \begin{pmatrix}
            \lambda^4 & 4\lambda^3 & 6\lambda^2 \\
            0         & \lambda^4  & 4\lambda^3 \\
            0         & 0          & \lambda^4
        \end{pmatrix}
    $$
\end{solution}

\begin{problem} Which eigenvalues can a matrix $A$ satisfying $A^4=A^2$ have?
Give an example of matrix which have all these eigenvalues.
\end{problem}
\begin{solution} If $\lambda$ is an eigenvalue of a matrix $A$, then $Ax=\lambda
        x$ for some $x\neq 0$. Hence, $A^k x=\lambda^k x$, Since $A^4=A^2$, then
    $\lambda^4x=\lambda^2 x$. Since $x\neq 0$, then $\lambda^4=\lambda^2$.
    This equation has three roots: $0$ and $\pm 1$. An example of matrix
    which has all these eigenvalues is
    $$
        A=\operatorname{diag}(0,-1,1)
    $$
\end{solution}

\begin{problem} Given two symmetric $2\times 2$ matrices determine whether they
specify the same quadratic form.
\end{problem}
\begin{solution} In general we must compute positive and negative indices of
    inertia of matrices. They are the same if and only if these matrices
    describe the same quadratic form. In order to compute indices of inertia we
    must find eigenvalues of the matrices. Number of positive/negative
    eigenavalues are positive/negative indices of inertia respectively.
\end{solution}

\begin{problem} Given two $2\times 2$ matrices determine whether they specify
the same linear operator.
\end{problem}
\begin{solution} In general we must compare Jordan's forms of given matrices.

    In this particular case it is enough to say if matrices are different. If
    matrices describe the same linear operator, then they have the same
    characteristic polynomial. In case of $2\times 2$ matrices we have
    $$
        \det(\mathcal{A}-t\mathcal{E})
        =\det(\mathcal{A})-\operatorname{tr}(\mathcal{A}) t+t^2
    $$
    Hence if matrices have different determinants or traces, then the specify
    different linear operators. In general this method can't be used to
    distinguish matrices. For example the matrices
    $$
        A_1=\begin{pmatrix}0&0\\0&0\end{pmatrix}\qquad\qquad
        A_2=\begin{pmatrix}0&1\\0&0\end{pmatrix}
    $$
    have the same trace and determinant, but they specify different linear
    operators because
    $$
        \dim(\operatorname{Ker}(\mathcal{A}_1))=2\qquad\qquad
        \dim(\operatorname{Ker}(\mathcal{A}_2))=1
    $$
\end{solution}














\newpage

\subsection{Analysis}

\begin{problem} Let $f\colon [0,1 ]\to[0,1]$ be a continuous map. Prove it has a
fixed point.
\end{problem}
\begin{solution} Consider continuous function $g(x)=f(x)-x$. Obviously,
    $$
        g(0)=f(0)-0=f(0)\in[0,1]\qquad g(1)=f(1)-1\in[-1,0]
    $$
    Since $g(0)\geq 0$, $g(1)\leq 0$ by mean value theorem there exists
    $x\in[0,1]$ such that $g(x)=0$. This means that $f(x)=x$, i.e. $x$ is a
    fixed point of $f$.
\end{solution}


\begin{problem} Find derivative and domain of the function $f(x)={(\sin
    x)}^{\cos x}$
\end{problem}
\begin{solution} Obviously,
    $$
        \operatorname{dom}(f)=\{x\in\mathbb{R}:\sin x\geq 0\}
        =\bigcup_{n\in\mathbb{Z}}\{x\in\mathbb{R}:2\pi n\leq x\leq \pi+2\pi n\}
    $$
    Now note that $\ln f(x)=\cos x\ln\sin x$. After taking the derivative we get
    $$
        \frac{f'(x)}{f(x)}=(\cos x)'\ln(\sin x)+\cos x(\ln(\sin x))'
    $$
    $$
        f'(x)={(\sin x)}^{\cos x}(\cos x\cot x-\sin x\ln(\sin x))
    $$
\end{solution}

\begin{problem} Compute Maclaurin series of $\arctan x$
\end{problem}
\begin{solution} Note that for all $t\in(-1,1)$ we have
    $$
        \frac{1}{1+t^2}
        =\sum_{k=0}^\infty{(-t^2)}^k
        =\sum_{k=0}^\infty{(-1)}^k t^{2k}
    $$
    After integrating this equality over the interval $[0,x]$ with $x\in(-1,1)$
    we get
    $$
        \arctan x=\int_{[0,x]}\frac{dt}{1+t^2}
        =\int_{[0,x]}\sum_{k=0}^\infty{(-1)}^k t^{2k}dt
        =\sum_{k=0}^\infty{(-1)}^k\int_{[0,x]}t^{2k}dt
        =\sum_{k=0}^\infty\frac{x^{2k+1}}{2k+1}
    $$
\end{solution}


\begin{problem} Compute $d({(xy)}^{xy})$.
\end{problem}
\begin{solution} Denote $f(x,y)={(xy)}^{xy}$, then $\ln f(x,y)=xy\ln(xy)$. After
    taking differentials we get
    $$
        \frac{d f(x,y)}{f(x,y)}
        =d(xy)\ln (xy)+xy\cdot d(\ln(xy))=(xdy+ydx)\ln(xy)+xy\frac{xdy+ydx}{xy}
    $$
    Hence
    $$
        df(x,y)={(xy)}^{xy}(\ln(xy)+1)(xdy+ydx)
    $$
\end{solution}


\begin{problem} Compute $\int \arctan(x)dx$
\end{problem}
\begin{solution} We use integration by parts
    $$
        \int\arctan(x)dx
        =x\arctan(x)-\int xd(\arctan(x))
        =x\arctan(x)-\int\frac{x}{1+x^2}dx=
    $$
    $$
        =x\arctan(x)-\frac{1}{2}\int\frac{d(1+x^2)}{1+x^2}
        =x\arctan(x)-\frac{1}{2}\ln(1+x^2)
    $$
\end{solution}

\begin{problem} Compute $\int_{\mathbb{R}} e^{-x^2}dx$
\end{problem}
\begin{solution} Denote $I=\int_{\mathbb{R}} e^{-x^2}dx$, then
    $$
        I^2
        =\left(\int_{\mathbb{R}} e^{-x^2}dx\right)
        \left(\int_{\mathbb{R}} e^{-y^2}dy\right)
        =\int_{\mathbb{R}} e^{-x^2-y^2}dxdy
    $$
    Now we use polar coordinates
    $$
        I^2
        =\int_{\mathbb{R}_+\times[0,2\pi]}e^{-r^2}rdrd\varphi
        =\int_{[0,2\pi]}d\varphi\int_{\mathbb{R}_+}re^{-r^2}dr
        =2\pi\frac{1}{2}\int_{\mathbb{R}_+}e^{-r^2}d(r^2)
        =\pi e^{-r^2}\Biggl|_{0}^{+\infty}=\pi
    $$
    Hence $I=\sqrt{\pi}$
\end{solution}
















\newpage

\subsection{Combinatorics}

\begin{problem} How many matrices of size $M\times N$ are there with entries
equal to $\pm 1$ such that the product of elements in each row and each column
is $1$
\end{problem}
\begin{solution} Each such matrix $A$ can be written as $[{(-1)}^{P_{i,j}}]$ for
    some matrix $[P_{i,j}]$ with entries in $\mathbb{Z}/2\mathbb{Z}$ and the
    property that
    $$
        \sum_{i=1}^N P_{i,j}\equiv 0 \pmod 2
        \qquad\qquad \sum_{j=1}^M P_{i,j}\equiv 0 \pmod 2
    $$
    Clearly this correspondence between $A$ and $P$ is bijective, so it is
    enough to count matrices $P$. We can arbitrarily choose submatrix of $P$
    consisting of first $N-1$ rows and first $M-1$ columns. In this case
    elements ${(P_{i,M})}_{i\in\mathbb{N}_{N-1}}$ and
    ${(P_{N,j})}_{j\in\mathbb{N}_{M-1}}$ are uniquely determined by equalities
    given above, i.e.
    $$
        P_{i,M}\equiv-\sum_{j=1}^{M-1}P_{i,j} \pmod 2\qquad\qquad
        P_{N,j}\equiv-\sum_{i=1}^{N-1}P_{i,j} \pmod 2
    $$
    Constraints put on matrix $P$ allows us to determine $P_{N,M}$ via two
    equalities
    $$
        P_{N,M}\equiv-\sum_{i=1}^{N-1}P_{i,M} \pmod 2\qquad\qquad
        P_{N,M}\equiv-\sum_{j=1}^{M-1}P_{N,j} \pmod 2
    $$
    We need to check that both formulae give the same value for $P_{N,M}$. They
    are indeed equal
    $$
        -\sum_{i=1}^{N-1}P_{i,M}
        \equiv-\sum_{i=1}^{N-1}\left(-\sum_{j=1}^{M-1}P_{i,j}\right)
        \equiv\sum_{i=1}^{N-1}\sum_{j=1}^{M-1}P_{i,j}
    $$
    $$
        \equiv\sum_{j=1}^{M-1}\sum_{i=1}^{N-1}P_{i,j}
        \equiv-\sum_{j=1}^{M-1}\left(-\sum_{i=1}^{N-1} P_{i,j}\right)
        \equiv-\sum_{j=1}^{M-1} P_{N,j}
    $$
    Therefore matrix $P$ is completely determined by its submatrix consisting of
    the first $N-1$ rows and $M-1$ columns. What is more values of its entries
    can be arbitrary. There are $2^{(N-1)\cdot(M-1)}$ such matrices, so this is
    the number of all possible matrices $P$ and all possible matrices $A$.
\end{solution}

\begin{problem} Compute $\sum_{i=0}^k i^r$ for $r\in\mathbb{N}^*$.
\end{problem}
\begin{solution}
    We will derive the recurrence formula. For the beginning
    $$
        S_0(k)=\sum\limits_{i=0}^k i^0=k+1
    $$
    Now to derive the recurrence equation note that
    $$
        {(i+1)}^{r+1}-i^{r+1}
        =\left(\sum\limits_{p=0}^{r+1}
        \binom{r+1}{p} 1^{r+1-p} i^p\right) -i^{r+1}
        =\left(\sum\limits_{p=0}^{r+1}
        \binom{r+1}{p} i^p\right) -i^{r+1}
        =\sum\limits_{p=0}^{r} \binom{r+1}{p} i^p
    $$
    Now lets sum this equalities by $i$ from $0$ to $k$. Then we get
    $$
        \sum\limits_{i=0}^{k} \left({(i+1)}^{r+1}-i^{r+1}\right)=
        \sum\limits_{i=0}^{k} \sum\limits_{p=0}^{r} \binom{r+1}{p} i^p=
        \sum\limits_{p=0}^{r} \binom{r+1}{p}\sum\limits_{i=0}^{k}  i^p=
        \sum\limits_{p=0}^{r} \binom{r+1}{p}S_p(k)=
    $$
    $$
        \binom{r+1}{r}S_{r}(k)+\sum\limits_{p=0}^{r-1} \binom{r+1}{p}S_p(k)=
        (r+1)S_{r}(k)+\sum\limits_{p=0}^{r-1} \binom{r+1}{p}S_p(k)
    $$
    Note that $\sum\limits_{i=0}^{k} \left({(i+1)}^{r+1}-i^{r+1}\right)$ is a
    telescopic sum and it equals to ${(k+1)}^{r+1}$. So we get the following
    equality
    $$
        {(k+1)}^{r+1}=(r+1)S_{r}(k)+\sum\limits_{p=0}^{r-1} \binom{r+1}{p}S_p(k)
    $$
    This gives us
    $$
        S_r(k)=\frac{1}{r+1}\left({(k+1)}^{r+1}
        -\sum\limits_{p=0}^{r-1} \binom{r+1}{p}S_p(k)\right)
    $$
\end{solution}

















\newpage

\subsection{Probability theory}

\begin{problem} Let $p$ and $q$ be probabilities of heads and tails respectively
in  a coin trials. Find the expected value of heads in $n$ tosses.
\end{problem}
\begin{solution} Let $X_i$ be a random variable defined by equality
    $$
        X_i=\begin{cases}
            1 &
            \mbox{there was head in $i$-th toss} \\
            0 &
            \mbox{otherwise}
        \end{cases}
    $$
    Obviously,
    $$
        \mathbb{E}[X_i]=1\cdot\mathbb{P}({X_i=1})+0\cdot\mathbb{P}({X_i=0})=p
    $$
    Let $X$ be a random variable equal to the number of heads after $n$ tosses,
    then

    $$
        X=\sum_{i=1}^n X_i
    $$
    $$
        \mathbb{E}[X]=\mathbb{E}\left[\sum_{i=1}^n X_i\right]
        =\sum_{i=1}^n\mathbb{E}[X_i]=np
    $$
\end{solution}


\begin{problem} There are 20 cards in a deck with distinct numbers from $0$ to
$19$ on them. Five cards where taken. Find the expected value of the sum of
numbers on these cards.
\end{problem}
\begin{solution} Let $X_i$ where $i\in \mathbb{N}_5$ be random variables equal
    to the number on the $i$-th card. Obviously,
    $$
        \mathbb{P}({X_i=k})=0.05\qquad k\in\mathbb{N}_{19}^*
    $$
    $$
        \mathbb{E}[X_i]=\sum_{k=0}^{19}k\mathbb{P}({X_i=k})
        =0.05\sum_{k=0}^{19} k=9.5
    $$
    Let $X$ be a random variable equal to the sum of numbers on five cards
    taken, then
    $$
        X=\sum_{i=1}^5 X_i
    $$
    $$
        \mathbb{E}[X]=\mathbb{E}\left[\sum_{i=1}^5 X_i\right]
        =\sum_{i=1}^5\mathbb{E}[X_i]=5\cdot 9.5=47.5
    $$
\end{solution}


\begin{problem} A drunk guy moves along the line. Each moment he makes moves to
the right or to the left with probabilities $p$ and $1-p$ respectively. $N$
steps to the left from the guy there is a precipice. Find the probability of
drunk guy falling into the abyss.
\end{problem}
\begin{solution} Let $P_k(s)$ be the probability of a drunk guy falling into the
    abyss after making at most $k$ moves provided his current position is $s$
    steps from the precipice. Then we have
    $$
        P_k(s)=pP_{k-1}(s-1)+(1-p)P_{k-1}(s+1)
    $$
    with initial conditions $P_0(s)=0$ for $s<N$ and $P_k(N)=1$ for all
    $k\in\mathbb{N}$. Since the number of possible moves is not bounded, we need
    to consider asymptotical behaviour of the probability, i.e.\ find the closed
    form of $P(s)=\lim_{k\to\infty} P_k(s)$. So we have the following recurrence
    equation
    $$
        P(s)=pP(s-1)+(1-p)P(s+1)
    $$
    with bounddary condition $P(N)=1$. This is a recurrence equation of the
    second order with characteristic equation
    $$
        (1-p)\lambda-1+\frac{p}{\lambda}=0
    $$
    Its roots are $\lambda=p/(1-p)$ and $\lambda=1$. So
    $$
        P(s)=C_1{\left(\frac{p}{1-p}\right)}^s+C_2
    $$
    Since $P(N)=1$ we get $C_2=1-C_1{(p/(1-p))}^N$, so
    $$
        P(s)=1+C_1\left({\left(\frac{p}{1-p}\right)}^s
        -{\left(\frac{p}{1-p}\right)}^N\right)
    $$
    Clearly, $\lim\limits_{s\to\infty}P(s)=0$. If $p\geq 0.5$, this is possible
    iff $C_1=0$, hence $P(s)=1$. If $p<0.5$, then we get an equation
    $$
        1-C_1{\left(\frac{p}{1-p}\right)}^N=0
    $$
    from which we get $C_1={\left(\frac{1-p}{p}\right)}^N$ and
    $P(s)={\left(\frac{p}{p-1}\right)}^{s-N}$. Finally we get
    $$
        P(s)=\begin{cases}
            1                                  &
            \mbox{ if } 0.5 \leq p \leq 1        \\
            {\left(\frac{p}{p-1}\right)}^{s-N} &
            \mbox{ if } 0 \leq p < 0.5
        \end{cases}
    $$
    The desired  probability equals
    $$
        P(0)=\begin{cases}
            1                                &
            \mbox{ if } 0.5 \leq p \leq 1      \\
            {\left(\frac{1-p}{p}\right)}^{N} &
            \mbox{ if }0 \leq p <0.5
        \end{cases}
    $$
\end{solution}


\begin{problem} Random variables $X_1,\ldots,X_n$ are independent and uniformly
distributed on $[0,1]$. Find the expected value and the variance of the random
variables
$$
    \min(X_1,\ldots,X_n)\quad\mbox{ and }\quad \max(X_1,\ldots,X_n).
$$
\end{problem}
\begin{solution} Denote $X=\max(X_1,\ldots,X_n)$. By $f$ and $F$ we denote the
    probability density function and the cumulative distributive function of the
    uniform distribution on $[0,1]$. So
    $$
        f(x)=\begin{cases}
            1 &
            \mbox{ if } x\in[0,1] \\
            0 &
            \mbox{ if }x\notin[0,1]
        \end{cases}\qquad
        F(x)=\begin{cases}
            1 &
            \mbox{ if } x<0           \\
            x &
            \mbox{ if } 0\leq x\leq 1 \\
            0 &
            \mbox{ if } x>1
        \end{cases}
    $$
    Now we find probability density function of $X$. For all $x\in\mathbb{R}$ we
    have
    $$
        F_X(x)=\mathbb{P}(\{X<x\})
        =\mathbb{P}(\{\max(X_1,\ldots,X_n)<x\})
        =\mathbb{P}(\{X_1<x,\ldots,X_n<x\})=
    $$
    $$
        =\mathbb{P}(\{X_1<x\})\cdot\ldots\cdot\mathbb{P}(\{X_n<x\}) % chktex 11
        =F_{X_1}(x)\cdot\ldots\cdot F_{X_n}(x)=F^n(x)  % chktex 11
    $$
    $$
        f_X(x)=(F_X(x))'=n F^{n-1}(x)F'(x)=nF^{n-1}(x)f(x)
    $$
    Now we compute the expected value and the variance of $X$
    $$
        \mathbb{E}[X]=\int_{\mathbb{R}}xf_X(x)dx
        =\int_{\mathbb{R}}xnF^{n-1}(x)f(x)dx
        =\int_{[0,1]}xnx^{n-1}dx=\frac{n}{n+1}
    $$
    $$
        \mathbb{E}[X^2]=\int_{\mathbb{R}}x^2f_X(x)dx
        =\int_{\mathbb{R}}x^2nF^{n-1}(x)f(x)dx
        =\int_{[0,1]}x^2nx^{n-1}dx=\frac{n}{n+2}
    $$
    $$
        \mathbb{V}[X]=\mathbb{E}[X^2]-{\mathbb{E}[X]}^2
        =\frac{n}{n+1}-{\left(\frac{n}{n+2}\right)}^2
    $$
    Denote $Y=\min(X_1,\ldots,X_n)$ and we shall find cummulative and the
    probability density function of $Y$. For all $x\in\mathbb{R}$ we have
    $$
        F_Y(x)=\mathbb{P}(\{Y<x\})
        =1-\mathbb{P}(\{Y\geq x\})
        =1-\mathbb{P}(\{\min(X_1,\ldots,X_n)\geq x\})=
    $$
    $$
        =1-\mathbb{P}(\{X_1\geq x,\ldots,X_n\geq x\})=
        1-\mathbb{P}(\{X_1\geq x\})\cdot
        \ldots
        \cdot\mathbb{P}(\{X_n\geq x\})
    $$
    $$
        =1-(1-\mathbb{P}(\{X_1<x\}))\cdot
        \ldots
        \cdot(1-\mathbb{P}(\{X_n<x\}))
        =1-(1-F_{X_1}(x))\cdot
        \ldots
        \cdot(1-F_{X_n}(x))
        =1-{(1-F(x))}^n
    $$
    $$
        f_Y(x)=(F_Y(x))'=(1-{(1-F(x))}^n)'
        =-n{(1-F(x))}^{n-1}(-F'(x))
        =n{(1-F(x))}^{n-1}f(x)
    $$
    Now we compute the expected value and the variance of $Y$
    $$
        \mathbb{E}[Y]=\int_{\mathbb{R}} xf_Y(x)dx
        =\int_{\mathbb{R}}x n {(1-F(x))}^{n-1}dx
        =\int_{[0,1]}x n {(1-x)}^{n-1}dx
        =\int_{[0,1]}(1-t)nt^{n-1}dt
    $$
    $$
        =1-\frac{n}{n+1}=\frac{1}{n+1}
    $$
    $$
        \mathbb{E}[Y^2]=\int_{\mathbb{R}} x^2f_Y(x)dx
        =\int_{\mathbb{R}}x^2n{(1-F(x))}^{n-1}dx
        =\int_{[0,1]}x^2n{(1-x)}^{n-1}dx
        =\int_{[0,1]}{(1-t)}^2nt^{n-1}dt
    $$
    $$
        =1-\frac{2n}{n+1}+\frac{n}{n+2}=\frac{2}{(n+1)(n+2)}
    $$
    $$
        \mathbb{V}[Y]=\mathbb{E}[Y^2]-{\mathbb{E}[Y]}^2
        =\frac{1}{n+1}-{\left(\frac{2}{(n+1)(n+2)}\right)}^2
    $$
\end{solution}


\begin{problem} There are three towers in apexes of the equilateral triangle.
The dragon sits on the first tower. With equal probabilities dragon flies to two
other towers. Find the expected value of the number flights after which the
dragon will come back to the first tower.
\end{problem}
\begin{solution} Clearly in order to come back to the first tower the dragon
    must make at least two flights. Assume he made $n$ flights. There are two
    scenarios: on the first flight the dragon flies to the second tower, or the
    dragon flies to the third tower. Both scenarios have the same probability
    $0.5^{n-2}$, so the total probability is $0.5^{n-1}$. The expected value of
    flights is
    $$
        \sum_{n=2}^\infty n 0.5^{n-1}
        =\sum_{n=2}^\infty nx^{n-1}\Biggl|_{x=0.5}
        =\left(\frac{d}{dx}\sum_{n=2}^\infty x^n\right)\Biggl|_{x=0.5}
        =\frac{d}{dx}\left(\frac{x^2}{1-x}\right)\Biggl|_{x=0.5}
        =\frac{2x-x^2}{{(1-x)}^2}\Biggl|_{x=0.5}
        =3
    $$
\end{solution}


\begin{problem} There are $n$ places on the airplane. An old crazy lady comes
first and randomly chooses a sit. Then come the other passengers. If a passenger
sees his place is already taken then he arbitrarily chooses another sit. Find
the probability that the last passenger will sit at his place.
\end{problem}
\begin{solution} Let $p(n)$ be the probability that the last passenger will not
    sit at his place in case of $n$-man problem. An lady woman can choose any
    sit with probability $1/n$. If she chooses her own sit then $p(n) = 1$. If
    she chooses a sit of the last passenger, then $p(n) = 0$. If she chooses
    $k$-th place where $k\in \{2,\ldots,n-1\}$, then all the passengers with
    numbers $2,\ldots, k-1$ will sit at their places. The $k$-th passenger will
    choose its place randomly. Hence we reduce our problem to the $n-k$-man
    problem, where the role of an old woman is played by $k$-th passenger. Thus
    the total probability is
    $$
        p(n)=1\cdot\frac{1}{n}+p(n-1)\cdot\frac{1}{n}
        +\ldots
        +p(1)\cdot\frac{1}{n}
        =\frac{1}{n}\left(1+\sum_{k=2}^{n-1} p(k)\right)
    $$
    Now we prove by induction that $p(n)=1/2$. Obviously, $p(2)=1/2$. Assume
    that $p(k)=1/2$ for $k\in\mathbb{N}_m$, then
    $$
        p(m+1)=\frac{1}{m+1}\left(1+\sum_{k=2}^{m}p(k)\right)
        =\frac{1}{m+1}\left(1+\sum_{k=2}^{m}\frac{1}{2}\right)=\frac{1}{2}
    $$
    The desired probability is $1-p(n)=1-1/2=1/2$.
\end{solution}











\newpage

\subsection{Programming}

\begin{problem} Write a program which computes the greatest common divisor of
two integers. Prove correctness.
\end{problem}
\begin{solution} This algorithm is called the Euclidean algorithm.
    $$
        \gcd(a,b)=\begin{cases}
            a                &
            \mbox{ if } b=0    \\
            \gcd(b, a\mod b) &
            \mbox{ if } b\neq 0
        \end{cases}
    $$
    Since each iteration of the algorithm decreases the second argument, then
    this algorithm will always stop. To prove correctness we need to show that
    $\gcd(a,b)=\gcd(b, a\mod b)$. It enough to show that the sets of common
    divisors of $a$, $b$ and $b$, $a\mod b$ coincide. Assume $d|a$ and $d|b$.
    Since $a\mod b=a-kb$ for some $k\in\mathbb{Z}$, then $d|(a\mod b)$. Thus
    $d|b$ and $d|(a\mod b)$. Now  assume $d|b$ and $d|(a\mod b)$. Since
    $a=kb+(a\mod b)$ for some $k\in\mathbb{Z}$, then $d|a$. Thus $d|a$ and
    $d|b$.
    \begin{verbatim}
#include <iostream>

int gcd(int a, int b)
{
    while(b !=0 )
    {
        a %= b;
        std::swap(a,b);
    }
    return a;
}

int main()
{
    int a, b;
    std::cin >> a >> b;
    std::cout << gcd(a, b);

    return 0;
}
    \end{verbatim}
\end{solution}

\begin{problem} Write a program which finds inverse element in a finite field.
\end{problem}
\begin{solution} Assume we are given a finite field $\mathbb{F}_k$, then by
    little Fermat’s theorem for all $x\in \mathbb{F}_k$ we have
    $$
        x^{k-1}=e
    $$
    Hence $x^{-1}=x^{p-2}$. It is remains to write a fast powerization in finite
    fields.
    \begin{verbatim}
#include <iostream>
#include <vector>
int power_mod(int x, int degree, int k)
{
    if (degree == 0)
    {
        return 1;
    }
    if (degree % 2 == 1)
    {
        return (x * power_mod(x, degree - 1, k)) % k;
    }
    int half_power = power_mod(x, degree / 2, k);
    return (x * x) % k;
}

int inverse_mod(int x, int k)
{
    return power_mod(x, k - 2, k);
}

int main()
{
    int x, k;
    std::cin >> x >> k;
    std::cout << inverse_mod(x, k);
    return 0;
}
    \end{verbatim}
\end{solution}

\begin{problem} Let $a$ be an array of integers of length $n$. Write a program
which effectively finds the sum of elements in each of $m$ consecutive requests
of the form (begin,end).
\end{problem}
\begin{solution} The main idea is to store additional array of sums of elements
    of each prefix of array.
    \begin{verbatim}
#include <iostream>
#include <vector>

class accumulator
{
public:
    accumulator(const std::vector<int>& numbers)
      : numbers_(numbers), sums_(numbers.size() + 1, 0)
    {}

    long long get_sum(size_t begin, size_t end)
    {
        return end >= begin ? sums_[end + 1] - sums_[begin] : 0;
    }

    void initialize()
    {
        for (size_t i = 1; i < numbers_.size(); ++i)
        {
            sums_[i] = sums_[i - 1] + numbers_[i - 1];
        }
    }

private:
    const std::vector<int>& numbers;
    std::vector<long long> sums_;
};

int main()
{
    size_t n;
    std::cin >> n;
    std::vector<int> numbers(n, 0);
    for (size_t i = 0; i < numbers.size(); ++i)
    {
        std::cin >> numbers[i];
    }

    accumulator accumulator_engine(numbers);
    accumulator_engine.initialize();

    size_t m;
    std::cin >> m;
    for (size_t i = 0; i < m; ++i)
    {
        int begin, end;
        std::cin >> begin >> end;
        std::cout << accumulator_engine.get_sum(begin, end);
    }

    return 0;

}
    \end{verbatim}
\end{solution}

\begin{problem} There are $2n+1$ integers in the array, with one unique integer
and $n$ integers that repeat twice. Write a program with $O(n)$ complexity which
finds this integer with usage of $O(1)$ memory.
\end{problem}
\begin{solution} Consider operation $\oplus$ which makes bitwise addition modulo
    2 of binary representations of integers. This is a commutative associative
    operation. Note that for all $x\in \mathbb{N}^*$ holds
    $$
        x\oplus x=0\qquad x\oplus 0=x
    $$
    Let ${(a_i)}_{i\in\mathbb{N}_n}$ be elements of array. Let $\sigma\in
        S_{2n+1}$ be such permutation of elements of array that after
    permutation twin integers will follow one by one and the unique integer
    will occur at the end. Then
    $$
        \bigoplus_{i=1}^{2n+1}a_i
        =\bigoplus_{i=1}^{2n+1}a_{\sigma(i)}
        =\left(\bigoplus_{k=1}^n (a_{\sigma(2k-1)}\oplus a_{\sigma(2k)})\right)
        \oplus a_{\sigma(2n+1)}
        =\left(\bigoplus_{k=1}^n 0\right)
        \oplus
        a_{\sigma(2n+1)}
        =a_{\sigma(2n+1)}
    $$
    Thus the unique element of array is equal to
    $$
        \bigoplus_{i=1}^{2n+1} a_i
    $$
    \begin{verbatim}
#include <iostream>

int main()
{
    int n;
    std::cin >> n;
    int unique_element = 0;
    for (size_t i = 0; i < 2 * n + 1; ++i)
    {
        int integer;
        std::cin >> integer;
        unique_element ^= integer;
    }
    std::cout << unique_element;
    return 0;
}
\end{verbatim}
\end{solution}


\begin{problem} Write a program with $O(\log n)$ complexity, which computes
$n$-th power of a given integer.
\end{problem}
\begin{solution} In order to powerize quickly we will use the following
    identities
    $$
        a^{2k+1}=aa^{2k}\qquad a^{2k}={(a^k)}^2
    $$
    \begin{verbatim}
#include <iostream>
#include <vector>

long long power(int integer, int degree)
{
    if (degree == 0)
    {
        return 1;
    }
    if (degree % 2 == 1)
    {
        return integer * power(integer, degree - 1);
    }

    long long half_power = power(integer, degree / 2);
    return half_power * half_power;
}

int main()
{
    int integer, degree;
    std::cin >> integer >> degree;
    std::cout << power(integer, degree);
    return 0;
}
    \end{verbatim}
\end{solution}


\begin{problem} Write a program with $O(\log n)$ complexity which finds $n$-th
Fibbonacci number
\end{problem}
\begin{solution} Let's rewrite the recurrence relation for Fibbonacci numbers
    $f_n=f_{n-1}+f_{n-2}$ in the matrix form.
    $$
        \begin{pmatrix}
            f_n \\
            f_{n-1}\end{pmatrix}
        =
        \begin{pmatrix}
            1 & 1 \\
            1 & 0
        \end{pmatrix}
        \begin{pmatrix}
            f_{n-1} \\
            f_{n-2}
        \end{pmatrix}
    $$
    Then we get
    $$
        \begin{pmatrix}
            f_n \\
            f_{n-1}
        \end{pmatrix}
        =\begin{pmatrix}
            1 & 1 \\
            1 & 0
        \end{pmatrix}^{n-1}
        \begin{pmatrix}
            f_1 \\ f_0
        \end{pmatrix}
    $$
    It remains to quickly compute $n-1$-th power of the matrix
    $$
        \begin{pmatrix} 1 & 1\\ 1 & 0\end{pmatrix}
    $$
    which is easily done with the help of identities
    $$
        a^{2k}=a a^{2k}\qquad a^{2k}={(a^k)}^2
    $$
    \begin{verbatim}
#include <iostream>
#include <vector>

class matrix2x2
{
public:

    matrix2x2() : matr_(2,std::vector<int>(2,0))
    {}

    matrix2x2(int matr00, int matr01, int matr10, int matr11)
      : matr_(2, std::vector<int>(2,0))
    {
        matr_[0][0] = matr00;
        matr_[0][1] = matr01;
        matr_[1][0] = matr10;
        matr_[1][1] = matr11;
    }

    matrix2x2(const matrix2x2& matrix)
    {
        matr_ = matrix.matr_;
    }

    const matrix2x2& operator = (const matrix2x2& rhs)
    {
        if (this != &rhs)
        {
            matr = rhs.matr;
        }
        return *this;
    }

    int& operator() (size_t i, size_t j)
    {
        return matr_[i][j];
    }

    const int& operator()(size_t i, size_t j) const
    {
        return matr_[i][j];
    }

    matrix2x2& operator * (const matrix2x2& ths)
    {
        return product(*this, rhs);;
    }

private:

    matrix2x2 product(const matrix2x2& lhs, const matrix2x2& rhs) const
    {
        retrn matrix2x2(lhs(0, 0) * rhs(0, 0) + lhs(0, 1) * rhs(1, 0),
                        lhs(0, 0) * rhs(0, 1) + lhs(0, 1) * rhs(1, 1)
                        lhs(1, 0) * rhs(0, 0) + lhs(1, 1) * rhs(1, 0)
                        lhs(1, 0) * rhs(0, 1) + lhs(1, 1) * rhs(1, 1));
    }


private:
    std::vector<int> matr_;
};

matrix2x2 matrix_power(const matrix2x2& matrix, size_t power)
{
 if (power == 0)
    {
     return matrix2x2(1, 0, 0, 1);
    }
    if (power % 2 == 1)
    {
     return matrix_power(matrix, power - 1) * matrix;
    }

    matrix2x2 half_power = matrix_power(matrix, power / 2);
    return half_power * half_power;
}

long long fibbonacci_number(int n)
{
    matrix2x2 fibbonacci_matrix = matrix_power(matrix2x2(1, 1, 1, 0), n - 1);
    return fibbonacci_matrix(0,0);
}

int main()
{
    size_t order;
    std::cin >> order;
    std::cout << fibbonacci_number(order);
    return 0;
}
    \end{verbatim}
\end{solution}


\begin{problem} There is an array of $n$ integers. It is known that some number
repeats in the array more than $n/2$ times. Give an algorithm that finds this
number. Time complexity $O(n)$, memory restrictions $O(1)$.
\end{problem}
\begin{solution}
    We call popular the number we are looking for. First we present the
    algorithm and then explain why it is correct.
    \begin{verbatim}
#include <iostream>
#include <vector>

int find_popular_number(const std::vector<int>& values)
{
    int popular_number;
    size_t counter = 0;

    for (size_t number_index = 0; number_index < values.size(); ++number_index)
    {
        if (counter == 0)
        {
            popular_number = values[number_index];
            counter++;
        }
        else
        {
            counter += popular_number != values[number_index] ? -1 : 1;
        }
    }

    return popular_number;
}

int main()
{
    size_t values_count;
    std::cin >> values_count;
    std::vector<int> values(values_count);
    for (size_t index = 0; index < values.size(); ++index)
    {
        std::cin >> values[index];
    }

    std::cout << find_popular_number(values);

    return 0;
}
    \end{verbatim}

    If the counter is equal to 1 we are looking for the popular number. And we
    set the popular number to be the first incoming number. If meet the number
    equal to the current popular we increase the counter, otherwise we decrease
    it. Each decrease of the counter is equivalent to throwing away the number
    at current iteration and the current popular number. By assumption we have
    at least $n/2$ equal numbers, hence such reduction will exclude all non
    popular numbers, and the remaining current popular number will be the
    genuine popular number.
\end{solution}

















\newpage

\section{Entrance exams in SDA}

\subsection{Exam on 10.06.2012}

\begin{problem} There are 2012 weights with different masses. They are divided
into two groups with 1006 weights in each group. Weights in each group are
sorted in ascending order. Give an algorithm that finds 1006's weight among all
weights with at most 11 weighings.
\end{problem}
\begin{solution} Denote that groups $A$ and $B$. Without loss of generality we
    may assume that $A[503]>B[503]$ (this is one weighing). Clearly the desired
    weight can not belong to $B[1\ldots 502]$ or $A[504\ldots 1006]$. So we
    reduce the problem to finding that element in $A[1\ldots 503]$ and
    $B[503\ldots 1006]$ with 10 weighings left. We can repeat the procedure
    until we arrive at comparison of two single element arrays. Since
    $2^{11}>2012$ we will definitely reach this comparison.
\end{solution}

\begin{problem} Compute $\int_{0}^{2\pi} \sin ^8xdx$.
\end{problem}
\begin{solution}
    $$
        \int_0^{2\pi}\sin^8xdx
        =\int_0^{2\pi}{\left(\frac{e^{ix}-e^{-ix}}{2i}\right)}^8dx
    $$
    $$
        =\int_0^{2\pi}2^{-8}{(e^{ix}-e^{-ix})}^8dx
        =2^{-8}\int_0^{2\pi}\sum_{k=0}^8
        \binom{8}{k}{(e^{ix})}^k{(-e^{-ix})}^{8-k}dx
    $$
    $$
        =2^{-8}\sum_{k=0}^8\binom{8}{k}{(-1)}^{8-k}\int_0^{2\pi}e^{i(2k-8)x}dx
        =2^{-8}\sum_{k=0}^8\binom{8}{k}{(-1)}^{k}2\pi\delta_{2k-8,0}
        =2^{-8}\binom{8}{4}{(-1)}^{4}2\pi
        =\frac{\binom{8}{4}}{2^7}\pi
    $$
\end{solution}

\begin{problem} Assume a polynomial $f\in \mathbb{R}[x]$ attains only positive
values. Prove it can be represented as a sum of squares of some polynomials.
\end{problem}
\begin{solution} By fundamental theorem of algebra
    $f(x)=A\prod_{i=1}^n(x-a_i)\prod_{j=1}^m(x^2+p_j x+q_j)$ for some real
    numbers $A$, ${(a_i)}_{i\in\mathbb{N}_n}$, ${(p_j)}_{j\in\mathbb{N}_m}$ and
    ${(q_j)}_{j\in\mathbb{N}_m}$, where $p_j^2-4q_j<0$ for all
    $j\in\mathbb{N}_m$. Clearly, $f(a_i)=0$ for all $i\in\mathbb{N}_n$, which
    contradicts assumption of positivity, hence $n=0$ and therefore
    $f(x)=A\prod_{j=1}^m(x^2+p_j x+q_j)$. Since $p_j^2-4q_j<0$ for all
    $j\in\mathbb{N}_m$, then for all $x\in\mathbb{R}$ and all $j\in\mathbb{N}_m$
    we have $x^2+p_j x+q_j>0$. Again, since $f$ is always positive we conclude
    that $A>0$. Without loss of generality we may assume that $A=1$.

    Now we prove by induction on $m$ that each polynomial of the form
    $f(x)=\prod_{j=1}^m(x^2+p_j x+q_j)$ can be represented as a sum of two
    squared polynomials. Basis of induction: $m=1$. We have
    $$
        f(x)=x^2+px+q
        ={\left(x+\frac{p}{2}\right)}^2+{\left(\frac{\sqrt{4q-p^2}}{2}\right)}^2
    $$
    and we are done. Assume our claim is true for $m>1$. Consider polynomial
    $f(x)=\prod_{j=1}^{m+1} (x^2+p_j x+q_j)$. By inductive assumption there
    exist polynomials $g_1, g_2$ such that
    $$
        \prod_{j=1}^{m} (x^2+p_j x+q_j)=g_1^2(x)+g_2^2(x)
    $$
    Denote $h_1(x)=x+\frac{1}{2}p_{m+1}$,
    $h_2(x)=\frac{1}{2}\sqrt{4q_{m+1}-p_{m+1}^2}$, then
    $$
        f(x)=\left(\prod_{j=1}^{m} (x^2+p_j x+q_j)\right) (x^2+p_{m+1}+q_m)
        =(g_1^2(x)+g_2^2(x))(h_1^2(x)+h_2^2(x))
    $$
    $$
        ={(g_1(x)h_1(x)+g_2(x)h_2(x))}^2+{(g_1(x)h_2(x)-g_2(x)h_1(x))}^2
    $$
    Induction step completed.
\end{solution}

\begin{problem} What is the maximal possible variance of a random variable $X$
with values in $[0,1]$
\end{problem}
\begin{solution} Assume a random variable is defined on probability space
    $(\Omega,\mathcal{F},\mathbb{P})$. Denote $m=\mathbb{E}[X]$, then
    $$
        \mathbb{V}[X]-\int_{\Omega}
        {\left(X(\omega)-\frac{1}{2}\right)}^2 d\mathbb{P}
        =\int_{\Omega}\left({\left(X(\omega)-m\right)}^2-
        {\left(X(\omega)-\frac{1}{2}\right)}^2\right)d\mathbb{P}
    $$
    $$
        =\int_{\Omega}\left(X(\omega)(1-2m)+m^2-
        \frac{1}{4}\right)d\mathbb{P}
        =(1-2m)\int_{\Omega}X(\omega)d\mathbb{P}+
        \left(m^2-\frac{1}{4}\right)\int_{\Omega}d\mathbb{P}
    $$
    $$
        =(1-2m)m+m^2-\frac{1}{4}=
        -{\left(m-\frac{1}{2}\right)}^2\leq 0
    $$
    From these inequalities it follows that $\mathbb{V}[X]$ is maximal if
    $m=1/2$. Since $X$ takes values in $[0,1]$, then
    ${\left(X(\omega)-\frac{1}{2}\right)}^2\leq \frac{1}{4}$. Thus the upper
    bound for $\mathbb{V}[X]$ is $\frac{1}{4}$. Now assume there is an event
    $A_1\in\mathcal{F}$ with $\mathbb{P}(A_1)=\frac{1}{2}$. Consider random
    variable defined as $X=\chi_{A_1}$, then
    $$
        \mathbb{V}[X]
        =\mathbb{E}[X^2]-{\mathbb{E}[X]}^2
        =\int_{\Omega}\chi_{A_1}(\omega)d\mathbb{P}-
        \left(\int_{\Omega}\chi_{A_1}^2(\omega)d\mathbb{P}\right)
        =\mathbb{P}(A_1)-{(\mathbb{P}(A_1))}^2
        =\frac{1}{4}
    $$
    Thus $X$ is a random variable with values in $[0,1]$ and the maximal
    possible variance equals to $\frac{1}{4}$.
\end{solution}

\begin{problem} In the group of $n$ people each man may know or may not know any
other. All acquaintances are represented via boolean $n\times n$ matrix. We say
that a man is a celebrity if all others know him, but he doesn't know anyone.
Give an algorithm that finds a celebrity or says it doesn't exist. Algorithms
complexity $O(n)$, memory restriction $O(1)$.
\end{problem}
\begin{solution} Notice that in the group there can be at most one celebrity. So
    we'll find that candidate and check whether he is a real celebrity. Let $a$
    be the boolean matrix representing acquaintances. The main idea of the
    algorithm is the fact that if $a[i][j]=0$ ($a[i][j]=1$), then $j$ ($i$) is
    not a celebrity.
    \begin{verbatim}
#include <iostream>
#include <vector>

size_t find_celebrity(
    const std::vector<std::vector<bool>>& acquaintances)
{
    size_t crowd_size = acquaintances.size();

    // find the candidate
    size_t candidate = 0;
    for (size_t man = 1; man < crowd_size; ++man)
    {
        if (acquaintances[candidate][man] == true)
        {
            candidate = man;
        }
    }

    //check candidate
    for (size_t man = 0; man < crowd_size; ++man)
    {
        if (acquaintances[man][candidate] == false ||
           (man != candidate &&
            acquaintances[candidate][man] == true))
        {
            candidate = -1;
            break;
        }
    }

    return candidate;
}

int main()
{
    size_t crowd_size;
    std::cin >> crowd_size;
    std::vector<std::vector<bool>> acquaintances(
        crowd_size, std::vector<bool>(crowd_size, true));

    for (size_t man = 0; man < acquaintances.size(); ++man)
    {
        for (size_t buddy = 0;
            buddy < acquaintances[man].size();
            ++buddy)
        {
            size_t acquaintance;
            std::cin >> acquaintance;
            acquaintances[man][buddy] = (
                acquaintance != 0 ? true : false);
        }
    }

    std::cout << find_celebrity(acquaintances);
    return 0;
}
    \end{verbatim}
\end{solution}

\begin{problem} Consider arbitrary permutation of $n$ elements. Prove that any
$k$ given  elements will belong to the same circle with probability $1/k$.
\end{problem}
\begin{solution} Let's count number of permutations of $n$ elements that contain
    cycle of length $k+i$ with $k$ given elements. Denote this quantity
    $c_n(i)$. There are $\binom{n-k}{i}$ ways to choose $i$ elements that
    together with given $k$ elements will constitute a cycle. There are
    $(i+k-1){!}$ ways to build a cycle from $k$ given elements and $i$ chosen
    elements. Finally, there are $(n-k-i){!}$ ways to arrange permutation of
    remaining $n-k-i$ elements. Therefore
    $c_n(i)=\binom{n-k}{i}(i+k-1)!(n-k-i){!}$. The desired probability is
    $$
        p_n(k)
        =\frac{1}{n!}\sum_{i=0}^{n-k} c_n(i)
    $$
    $$
        =\frac{1}{n!}\sum_{i=0}^{n-k} \binom{n-k}{i}(i+k-1)!(n-k-i)!
        =\frac{1}{n!}\sum_{i=0}^{n-k} \frac{(n-k)!}{(n-k-i)!i!}(i+k-1)!(n-k-i)!
    $$
    $$
        =\frac{(n-k)!}{n!}\sum_{i=0}^{n-k} \frac{(i+k-1)!}{i!}
        =\frac{(n-k)!(k-1)!}{n!}\sum_{i=0}^{n-k} \frac{(i+k-1)!}{i!(k-1)!}
        =\frac{(n-k)!(k-1)!}{n!}\sum_{i=0}^{n-k} \binom{i+k-1}{i}
    $$
    Note that
    $$
        \sum_{i=0}^{p}\binom{i+q}{i}
        =\binom{q}{0}+\sum_{i=1}^{p}\binom{i+q}{i}
        =\binom{q+1}{0}+
        \sum_{i=1}^{p}\left(\binom{i+1+q}{i}-\binom{i+q}{i-1}\right)
    $$
    $$
        =\binom{q+1}{0}+\binom{p+1+q}{p}-\binom{q+1}{0}
        =\binom{p+1+q}{p}
    $$
    so
    $$
        p_n(k)
        =\frac{(n-k)!(k-1)!}{n!}\sum_{i=0}^{n-k} \binom{i+k-1}{i}
        =\frac{(n-k)!(k-1)!}{n!}\binom{n}{n-k}
    $$
    $$
        =\frac{(n-k)!(k-1)!}{n!}\frac{n!}{(n-k)!k!}
        =\frac{1}{k}
    $$
\end{solution}

\begin{problem} There is an unknown quadratic form $Q$ on $\mathbb{R}^n$. One
may ask questions of the form `What is the value of $Q(v)$?'. What is the
minimal number of questions required to determine if $Q$ is positive definite?
\end{problem}
\begin{solution} For the beginning we claim that one needs to know all the
    entries of the matrix of quadratic form to say if it is positive definite.
    Indeed, if at least one entry is not fixed, then we can fix all the
    remaining entries and consider sufficiently big positive values of non fixed
    entry, and sufficiently big negative values of non fixed entry. For these
    values of non fixed entry quadratic form will have different definiteness
    type. Therefore to give a definite answer whether quadratic form $Q$ is
    positive definite we need to know all its entries. Since the matrix $q$ of
    the quadratic form is symmetric we need to know only entries over the main
    diagonal and on the main diagonal. That is $n(n+1)/2$ elements. For the
    beginning we determine entries of main diagonal by formula the
    $q_{i,i}=Q(e_i)$, where $i\in\mathbb{N}_n$. This requires $n$ questions. In
    order to  determine entries over the main diagonal we use the formula
    $q_{i,j}=\frac{1}{2}(Q(e_i+e_j)-Q(e_i)-Q(e_j))$. This will require
    $\frac{1}{2}(n-1)n$ questions. The total number of questions is
    $n+\frac{1}{2}n(n-1)=\frac{1}{2}n(n+1)$. We claim this is the minimal number
    of questions. Indeed, assume there exists an algorithm $A$ that completely
    determines matrix $q$ with $m<\frac{1}{2}n(n+1)$ questions. Then it uses
    values of $Q$ on $m$ vectors $v^{(1)},\ldots,v^{(m)}$. Which gives us $m$
    equations of the form $\sum_{i,j=1}^n v_i^{(k)}q_{i,j}v_j^{(k)}=Q(v^{(k)})$
    where $k\in\mathbb{N}_m$. This is a system of $m$ equations with
    $\frac{1}{2}n(n+1)$ unknowns which is greater than $m$. Then there
    infinitely many solutions of this system of linear equations, i.e.\ there
    are infinitely many quadratic forms with values
    ${(Q(v^{(k)}))}_{k\in\mathbb{N}_m}$ on vectors
    ${(v^{(k)})}_{k\in\mathbb{N}_m}$. Therefore the algorithm $A$ cannot
    definitely determine the matrix $q$. Contradiction, hence at least
    $\frac{1}{2}n(n+1)$ questions required to determine $q$. As we showed above
    there is an algorithm that answers the question using exactly
    $\frac{1}{2}n(n+1)$ questions.
\end{solution}








\newpage

\subsection{Exam on 26.05.2013}

\begin{problem} Find the following recurrent sequence
$$
    x_0=0,\quad x_1=1,\quad x_{n+1}=\frac{x_n+nx_{n-1}}{n+1}
$$
is convergent and find its limit.
\end{problem}
\begin{solution} Consider sequence $y_n=x_{n+1}-x_n$, then
    $y_n=-\frac{n}{n+1}y_{n-1}$ and $y_0=1$. Clearly,
    $$
        y_n=y_0\cdot\prod_{k=1}^n\frac{-k}{k+1}=\frac{{(-1)}^n}{n+1}
    $$
    hence
    $$
        x_n-x_0
        =\sum_{k=0}^{n-1} y_k
        =\sum_{k=0}^{n-1} \frac{{(-1)}^k}{k+1}
    $$
    Now we compute the desired limit
    $$
        \lim\limits_{n\to\infty} x_n
        =\lim\limits_{n\to\infty}
        \left(x_0+\sum_{k=0}^{n-1} \frac{{(-1)}^k}{k+1}\right)
        =\sum_{k=0}^\infty\frac{{(-1)}^k}{k+1}
        =\sum_{k=0}^\infty{(-1)}^k\int_0^1 x^k dx
        =\int_0^1\sum_{k=0}^\infty{(-1)}^k x^k dx
    $$
    $$
        =\int_0^1\sum_{k=0}^\infty{(-x)}^k dx
        =\int_0^1\frac{1}{1+x}dx
        =\ln 2
    $$
\end{solution}

\begin{problem} Assume we are given $100$ subsets of $\{0,1,\ldots,9\}$. Prove
there two subsets with symmetric difference not greater than $2$.
\end{problem}
\begin{solution} Each subset of $\{0,1,\dots, 9\}$ can be identified with some
    vector in ${\{0,1 \}}^{10}$. Hence the original problem can be reformulated
    as follows: does there exist $100$ vectors which differs from each other in
    at least $2$ positions, i.e.\ the Hamming distance is greater than $2$. We
    prove that such vectors do not exist. It will be enough to show that there
    is no $100$ disjoint balls of radius $1$ in ${\{0, 1 \}}^{10}$.


    We let $A_q(n,d)$ denote the maximum number of $q$-ary vectors of dimension
    $n$ with minimum pairwise distance $d$. For a given vector $c\in
        {\{0,\ldots,q \}}^n$ we can count $q$-ary vectors of length $n$ that differs
    from $c$ in at most $t$ positions. There are $\binom{n}{k}$ ways to choose
    $0\leq k\leq t$ positions where a vector can differ from $c$. At each
    position we can place $q-1$ values. So there are $\binom{n}{k}{(q-1)}^k$
    vectors that differ from a given vector $c$ in $k$ positions. Thus there are
    $\sum_{k=0}^ t\binom{n}{k}{(q-1)}^k$ that differ from $c$ in at most $t$
    positions. Since we are looking for vectors that differ in at most $d$
    positions we need to set $t=(d-1)/2$. Denote
    $m=\sum_{k=0}^{(d-1)/2}\binom{n}{k}{(q-1)}^k$. The total number of vectors
    is $q^n$, so
    $$
        A_q(n,d)\leq\frac{q^n}{m}
        =\frac{q^n}{\sum_{k=0}^{(d-1)/2}\binom{n}{k}{(q-1)}^k}
    $$

    Using this bound we get $A_2(10,3) \leq \frac{2^{10}}{1 + 10} < 94<100$,
    therefore such collection of subsets doesn't exist.
\end{solution}

\begin{problem} A random point $P$ is taken on the unit circle $x^2+y^2=1$. A
random point $Q$ is taken from the unit ball $x^2+y^2\leq 1$. Let $R$ be a
rectangle with diagonal $PQ$, whose sides are parallel to the coordinates axis.
What is the probability that $R$ is contained in the unit ball.
\end{problem}
\begin{solution} Let $P$ has coordinates $(a,b)$ and $Q$ has coordinates
    $(c,d)$, then $a^2+b^2=1$ and $c^2+d^2\leq 1$. In order for rectangle to be
    inside the circle it is necessary and sufficient that $a^2+d^2\leq 1$ and
    $c^2+b^2\leq 1$. Therefore $|c|\leq|a|$ and $|d|\leq |b|$. For the fixed
    point $P$ the point $Q$ must belong to the rectangle
    $[-|a|,|a|]\times[-|b|,|b|]$. The probablity of this event is
    $$
        \frac{\operatorname{area}(circle)}{\operatorname{area}(rectangle)}
        =\frac{2|a|\cdot2|b|}{\pi \cdot 1^2}=\frac{4}{\pi}|ab|
    $$
    Since $P$ is uniformly distributed over the circle, then $a=\cos(2\pi t)$
    and $b=\sin(2\pi t)$ with $t$ uniformly disturbed on $[0,1]$. Therefore the
    desired probability is
    $$
        p
        =\int_{0}^{1} \frac{4}{\pi}|\cos(2\pi t)\sin(2\pi t)|dt
        =\frac{2}{\pi}\int_{0}^{1} |\sin(4\pi t)|dt
        =\frac{2}{\pi}\int_{0}^{4\pi} |\sin(z)|\frac{dz}{4\pi}
        =\frac{1}{2\pi^2}\int_0^{4\pi}|\sin z|dz
    $$
    $$
        =\frac{4}{2\pi^2}\int_0^{\pi}|\sin z|dz
        =\frac{2}{\pi^2}\int_0^\pi \sin z dz
        =\frac{4}{\pi^2}
    $$
\end{solution}

\begin{problem} Let $f:\mathbb{R}\to\mathbb{R}_+$ be a continuous function with
the property that $\int_{\mathbb{R}} f(x)dx=1$. Let $\alpha\in(0,1)$ and $[a,b]$
is  the interval of minimal length such that $\int_{[a,b]} f(x)dx=\alpha$. Show
that $f(a)=f(b)$.
\end{problem}
\begin{solution} We have a constrained optimization problem. We need to minimize
    the function $b-a$ with restriction that $\int_{[a,b]}f(x)dx=\alpha$. The
    Lagrangian of this problem is
    $$
        L(a,b,\lambda)=b-a+\lambda\left(\int_{[a,b]}f(x)dx-\alpha\right)
    $$
    By assumption $L$ attains minimum for some $a$ and $b$, hence we have
    $L_a'=L_b'=0$. This gives us two equations
    $$
        -1+\lambda(-f(b))=0,\qquad 1+\lambda f(a)=0
    $$
    If $\lambda=0$ these equations give a contradiction, so $\lambda\neq 0$.
    After summation of the equations we obtain $\lambda(f(b)-f(a))=0$. Since
    $\lambda\neq 0$ we have $f(b)=f(a)$.
\end{solution}

\begin{problem} Assume we a given a $n\times n$ matrix $M$, such that
$$
    M_{ij}=\begin{cases}
        a_i a_j & \mbox{ if }\quad i\neq j \\
        a_i^2+k & \mbox{ if }\quad i=j
    \end{cases}
$$
Find $\det(M)$.
\end{problem}
\begin{solution}
    Let $X$ be an $m\times n$ matrix and $Y$ be $n\times m$ matrix, then we have
    the following identity
    $$
        \begin{pmatrix}
            kE_n+YX & 0    \\
            X       & kE_m
        \end{pmatrix}
        \begin{pmatrix}
            E_n & Y   \\
            0   & E_m
        \end{pmatrix}
        =\begin{pmatrix}
            E_n & Y   \\
            0   & E_m
        \end{pmatrix}
        \begin{pmatrix}
            kE_n & 0         \\
            X    & kE_m + XY
        \end{pmatrix}
    $$
    Now we compute the determinant of both sides
    $$
        \det\left(
        \begin{pmatrix}
                kE_n+YX & 0    \\
                X       & kE_m
            \end{pmatrix}
        \begin{pmatrix}
                E_n & Y   \\
                0   & E_m
            \end{pmatrix}
        \right)
        =
        \det\left(
        \begin{pmatrix}
                E_n & Y   \\
                0   & E_m
            \end{pmatrix}
        \begin{pmatrix} kE_n & 0       \\
                X    & kE_m+XY
            \end{pmatrix}
        \right)
    $$
    $$
        \det\begin{pmatrix}
            kE_n + YX & 0    \\
            X         & kE_m
        \end{pmatrix}
        \det\begin{pmatrix}
            E_n & Y \\ 0 & E_m
        \end{pmatrix}
        =
        \det\begin{pmatrix}
            E_n & Y   \\
            0   & E_m
        \end{pmatrix}
        \det\begin{pmatrix}
            kE_n & 0       \\
            X    & kE_m+XY
        \end{pmatrix}
    $$
    Recall the following formulae
    $$
        \det\begin{pmatrix} A&B\\0&D\end{pmatrix}=\det(A)\det(D)\qquad
        \det\begin{pmatrix} A&0\\B&D\end{pmatrix}=\det(A)\det(D)\qquad
    $$
    then
    $$
        \det(kE_n+YX)\det(kE_m)\det(E_n)\det(E_m)
        =
        \det(E_m)\det(E_n)\det(kE_n)\det(kE_m+XY)
    $$
    $$
        k^m\det(kE_n+YX)=k^{n}\det(kE_m+XY)
    $$
    $$
        \det(kE_n+YX)=k^{n-m}\det(kE_m+XY)
    $$
    Let $X=(a_1,\ldots,a_n)$, $Y={(a_1,\ldots,a_n)}^{tr}$, then $M=k E_n+YX$. By
    previous result we have
    $$
        \det(M)=\det(kE_n+YX)=k^{n-1}\det(kE_1+XY)
        =k^{n-1}\left(k+\sum_{i=1}^n a_i^2\right)
    $$
\end{solution}

\begin{problem} Assume we are given a binary $n\times n$ matrix (each entry take
1 bit of memory). We say that a row or column is bad if it contains at least one
zero. Give an algorithm which with usage of $O(1)$ memory sets to zero all
entries of bad rows and columns.
\end{problem}
\begin{solution} Denote that matrix by $a$. If $a[i][j]=0$, then $i$-th row and
    $j$-th column are bad. Hence it is enough to iterate through the matrix and
    each time we met a zero element we set $a[i][0]$ and $a[0][j]$ to zero.
    After this in the first row and first column we indicate all the bad rows
    and columns. It remains to set them all to zero.
    \begin{verbatim}
#include <iostream>
#include <vector>

void print_matrix(const std::vector<std::vector<bool>>& matrix)
{
    for (size_t i = 0; i < matrix.size(); ++i)
    {
        for (size_t j = 0; j < matrix[i].size(); ++j)
        {
            std::cout << matrix[i][j] << ' ';
        }
        std::cout << '\n';
    }
}

void fix_matrix(std::vector<std::vector<bool>>& matrix)
{
    for (size_t i = 0; i < matrix.size(); ++i)
    {
        for (size_t j = 0; j < matrix[i].size(); ++j)
        {
            if (matrix[i][j] == false)
            {
                matrix[0][j] = matrix[i][0] = false;
            }
        }
    }

    for (size_t i = 1; i < matrix.size(); ++i)
    {
        for (size_t j = 1; j < matrix[i].size(); ++j)
        {
            if (matrix[i][0] == false || matrix[0][j] == false)
            {
                matrix[i][j] = false;
            }
        }
    }
}

int main()
{
    size_t matrix_size;
    std::cin >> matrix_size;
    std::vector<std::vector<bool>> matrix(matrix_size,
                                          std::vector<bool>(matrix_size, 0));
    for (size_t i = 0; i < matrix.size(); ++i)
    {
        for (size_t j = 0; j < matrix[i].size(); ++j)
        {
            size_t value;
            std::cin >> value;
            matrix[i][j] = value != 0 ? true : false;
        }
    }

    fix_matrix(matrix);

    print_matrix(matrix);

    return 0;
}
    \end{verbatim}
\end{solution}

\begin{problem} Consider a real vector space of polynomials of two variables
with degree at most $2013$ and its subspace $V$ consisting of polynomials with
the property that
$$
    \oint_{x^2+y^2=R^2}f(x,y)ds=0
$$
for all real numbers $R$. Find $\dim(V)$.
\end{problem}
\begin{solution} Denote $N=2013$ and $C_R=\{(x,y)\in\mathbb{R}^2:
        x^2+y^2=R^2\}$. If we denote the homogeneous part of $f$ of degree $k$
    by $f_k$, we see that

    $$
        I(R)
        :=\int_{C_R} f(x,y)\,ds
        =\int_{C_R} \sum_{k=0}^N f_k(x,y) ds
        =\sum_{k=0}^N \int_{C_R} f_k(x,y) ds
        =\sum_{k=0}^N \int_{C_1}f_k(x,y) ds\cdot R^{k+1},
    $$

    so the integral of each homogeneous part must vanish.

    By symmetry, the integral of each monomial $x^\alpha y^\beta$ vanishes when
    $\alpha$ or $\beta$ is odd, and is strictly positive when $\alpha$ and
    $\beta$ are both even.

    So for odd $k$, the integral vanishes for all monomials of degree $k$, and
    for even $k = 2m$, we have $m$ monomials where the exponent of $x$ and $y$
    is odd, so their integral vanishes, and for $\mu = 0,\dotsc,m-1$, we have a
    homogeneous polynomial $x^{2\mu}y^{2(m-\mu)} - c_\mu\cdot x^{2m}$ of degree
    $k$ whose integral vanishes, and these $k$ homogeneous polynomials are
    linearly independent.

    So overall, we lose one dimension for every even $0 \leq k \leq N$, whence

    $$\dim V = \sum_{k=0}^N (k+1) - \left(\left\lfloor \frac{N}{2}\right\rfloor
        + 1\right) = \frac{(N+1)(N+2)}{2} - \left\lfloor
        \frac{N+2}{2}\right\rfloor.$$
\end{solution}













\newpage

\subsection{Exam on 02.06.2013}

\begin{problem} Compute $\prod_{k=1}^\infty \cos(x\cdot 2^{-k})$
\end{problem}
\begin{solution} For $x=0$ the answer is trivially $1$. If $x\neq 0$, then note
    that
    $$
        \prod_{k=1}^n \cos(x\cdot 2^{-k})
        =\frac{1}{2^n\sin(x\cdot 2^{-n})}
        \prod_{k=1}^n 2^n\cos(x\cdot 2^{-k})\sin(x\cdot 2^{-n})
        =\frac{\sin x}{2^n\sin(x\cdot 2^{-n})}
    $$
    so
    $$
        \prod_{k=1}^\infty \cos(x\cdot 2^{-k})
        =\lim\limits_{n\to\infty}\prod_{k=1}^n \cos(x\cdot 2^{-k})
        =\lim\limits_{n\to\infty}\frac{\sin x}{2^n\sin(x\cdot 2^{-n})}
        =\lim\limits_{n\to\infty}\frac{x}{2^n x\cdot 2^{-n}}=1
    $$
\end{solution}

\begin{problem} Compute the rank of $n\times n$ matrix $A$ such that
$A_{i,j}={(i-j)}^2$.
\end{problem}
\begin{solution} We claim that the determinant of the matrix $M$ is given by
    $M_{i,j}={(a_i+b_j)}^m$ where $i,j\in\mathbb{N}_n$ is $0$ if $n>m+1$. We
    have
    $$
        \begin{aligned}
            \det(M)
             & =\begin{vmatrix}
                {(a_1+b_1)}^m &
                {(a_1+b_2)}^m &
                \ldots        &
                {(a_1+b_n)}^m   \\
                {(a_2+b_1)}^m &
                {(a_2+b_2)}^m &
                \ldots        &
                {(a_2+b_n)}^m   \\
                \ldots        &
                \ldots        &
                \ldots        &
                \ldots          \\
                {(a_n+b_1)}^m &
                {(a_n+b_2)}^m &
                \ldots        &
                {(a_n+b_n)}^m   \\
            \end{vmatrix} \\
             & =\begin{vmatrix}
                \sum\limits_{k_1=0}^m\binom{m}{k_1} a_1^{k_1}b_1^{m-k_1} &
                \sum\limits_{k_1=0}^m\binom{m}{k_1} a_1^{k_1}b_2^{m-k_1} &
                \ldots                                                   &
                \sum\limits_{k_1=0}^m\binom{m}{k_1} a_1^{k_1}b_n^{m-k_1}   \\
                \sum\limits_{k_2=0}^m\binom{m}{k_2} a_2^{k_2}b_1^{m-k_2} &
                \sum\limits_{k_2=0}^m\binom{m}{k_2} a_2^{k_2}b_2^{m-k_2} &
                \ldots                                                   &
                \sum\limits_{k_2=0}^m\binom{m}{k_2} a_2^{k_2}b_n^{m-k_2}   \\
                \ldots                                                   &
                \ldots                                                   &
                \ldots                                                   &
                \ldots                                                     \\
                \sum\limits_{k_n=0}^m\binom{m}{k_n} a_n^{k_n}b_1^{m-k_n} &
                \sum\limits_{k_n=0}^m\binom{m}{k_n} a_n^{k_n}b_2^{m-k_n} &
                \ldots                                                   &
                \sum\limits_{k_n=0}^m\binom{m}{k_n} a_n^{k_n}b_n^{m-k_n}   \\
            \end{vmatrix} \\
             & =\sum\limits_{k_1=0}^m
            \sum\limits_{k_2=0}^m
            \ldots
            \sum\limits_{k_n=0}^m
            \binom{m}{k_1}
            \binom{m}{k_2}
            \ldots\binom{m}{k_n}
            \begin{vmatrix}
                a_1^{k_1} b_1^{m-k_1} &
                a_1^{k_1} b_2^{m-k_1} &
                \ldots                &
                a_1^{k_1} b_n^{m-k_1}   \\
                a_2^{k_2} b_1^{m-k_2} &
                a_2^{k_2} b_2^{m-k_2} &
                \ldots                &
                a_2^{k_2} b_n^{m-k_2}   \\
                \ldots                &
                \ldots                &
                \ldots                &
                \ldots                  \\
                a_n^{k_n} b_1^{m-k_n} &
                a_n^{k_n} b_2^{m-k_n} &
                \ldots                &
                a_n^{k_n} b_n^{m-k_n}   \\
            \end{vmatrix}     \\
             & =\sum\limits_{k_1=0}^m
            \sum\limits_{k_2=0}^m
            \ldots
            \sum\limits_{k_n=0}^m
            \binom{m}{k_1}
            \binom{m}{k_2}
            \ldots\binom{m}{k_n}
            a_1^{k_1}a_2^{k_2}
            \ldots
            a_n^{k_n} b_1^m b_2^m\ldots b_n^m
            \begin{vmatrix}
                b_1^{-k_1} &
                b_2^{-k_1} &
                \ldots     &
                b_n^{-k_1}   \\
                b_1^{-k_2} &
                b_2^{-k_2} &
                \ldots     &
                b_n^{-k_2}   \\
                \ldots     &
                \ldots     &
                \ldots     &
                \ldots       \\
                b_1^{-k_n} &
                b_2^{-k_n} &
                \ldots     &
                b_n^{-k_n}   \\
            \end{vmatrix}     \\
        \end{aligned}
    $$
    Denote
    $$
        M_{k_1,k_2,\ldots,k_n}=
        \begin{vmatrix}
            b_1^{-k_1} &
            b_2^{-k_1} &
            \ldots     &
            b_n^{-k_1}   \\
            b_1^{-k_2} &
            b_2^{-k_2} &
            \ldots     &
            b_n^{-k_2}   \\
            \ldots     &
            \ldots     &
            \ldots     &
            \ldots       \\
            b_1^{-k_n} &
            b_2^{-k_n} &
            \ldots     &
            b_n^{-k_n}   \\
        \end{vmatrix}
    $$
    Note that if $k_i=k_j$ for some $i,j\in\mathbb{N}_n$, then the determinant
    $M_{k_1,k_2,\ldots,k_n}$ have two equal columns and therefore equals zero.
    Now note that if $n>m+1$ there always exist $i,j\in\mathbb{N}_n$ such that
    $k_i=k_j$, so $M_{k_1,k_2,\ldots,k_n}=0$ for all tuples
    $(k_1,k_2,\ldots,k_n)\in {\{0,\ldots,m\}}^n$ and as the consequence
    $\det(M)=0$.

    Now we return to the original problem. For $n=1,2,3$ we have the following
    matrices
    $$
        A=(0)\qquad
        A=\begin{pmatrix}0&1\\1&0\end{pmatrix}\qquad
        A=\begin{pmatrix}0&1&4\\1&0&1\\4&1&0 \end{pmatrix}
    $$
    respectively. Their ranks are $0$, $2$ and $3$ respectively. If $n>3$ by
    previous claim all minors of $A$ of size greater than $3$ are zero, hence
    $\operatorname{rank}(A)\leq 3$. Since $n>3$, then $A$ contains a minor
    $$
        \begin{pmatrix}0&1&4\\1&0&1\\4&1&0 \end{pmatrix}
    $$
    whose rank is $3$, so $\operatorname{rank}(A)=3$. Finally
    $$
        \operatorname{rank}(A)=
        \begin{cases}
            0 & \quad\mbox{if}\quad n=1 \\
            2 & \quad\mbox{if}\quad n=2 \\
            3 & \quad\mbox{if}\quad n>2 \\
        \end{cases}
    $$
\end{solution}

\begin{problem} Given set $A=\{1,2,\ldots,256\}$ find the size of maximal subset
$A'\subset A$ such that it doesn't contain elements $x$ and $y$ with the
property $x=2y$.
\end{problem}
\begin{solution} Let's split $A$ into subsets $A_0,\ldots,A_8$ where $A_i$
    contains all numbers in $A$ with exactly $i$ powers of 2 in its
    factorization. Then $A_{i+1} \subset 2A_i$ and since the sets diminish in
    size the maximal subset would be $A_0\cup A_2\cup\ldots\cup A_8$. It remains
    to compute the cardinality of $A_i$ for each even $i$. Since $2^8$ is the
    biggest number in $A$, then $|A_8|=1$. For the remaining values of $i$ we
    have $|A_i|=\frac{256}{2^{i+1}}$, so the desired cardinality is
    $$
        N=1+\sum\limits_{0\leq 2i\leq 7}\frac{256}{2^{2i+1}}=171
    $$
\end{solution}

\begin{problem} There are $n$ uniformly disturbed points on a circle. Find the
probability that they belong to some semicircle.
\end{problem}
\begin{solution} Let $A_n(x)$ denote the event that all points fit into the arc
    of length $2\pi x$ and belong to some semicircle. Let $B_n(x)$ denote the
    event that the minimal arc containing $n$ points is of length $2\pi x$, then
    $$
        \mathbb{P}(A_{n+1}(x))=
        \int_0^x \mathbb{P}(A_{n+1}(x)|B_n(t)) d(\mathbb{P}(A_n(t)))
    $$
    Now we need to find $\mathbb{P}(A_{n+1}(x)|B_n(t))$. If the minimal arc
    containing $n$ points is of the length $2\pi x$, then in order to have $n+1$
    points inside the arc of length at most $2\pi x$ we need that the $n+1$-th
    point to be at most $2\pi(x-t)$ radians to the left from the most left point
    or to the right from the most right point of the first $n$ points. Therefore
    $\mathbb{P}(A_{n+1}(x)|B_n(t))=2x-t$ and we get
    $$
        \mathbb{P}(A_{n+1}(x))=\int_0^x (2x-t) d(\mathbb{P}(A_n(t)))
        =x\mathbb{P}(A_n(x))+\int_0^x \mathbb{P}(A_n(t))dt
    $$
    Note that $\mathbb{P}(A_1(x))=1$. We claim that $\mathbb{P}(A_n(x))=n
        x^{n-1}$. Base of induction is obvious. Step of induction: assume
    $\mathbb{P}(A_n(x))=n x^{n-1}$, then
    $$
        \mathbb{P}(A_{n+1}(x))
        =x\mathbb{P}(A_n(x))+\int_0^x \mathbb{P}(A_n(t))dt
        =xn x^{n-1}+\int_0^x n t^{n-1}dt
        =(n+1)x^n
    $$
    Thus we proved that $\mathbb{P}(A_n(x))=n x^{n-1}$. The desired probability
    is
    $$
        \mathbb{P}(A_n(0.5))=\frac{n}{2^{n-1}}
    $$
\end{solution}

\begin{problem} We say that a two-dimensional array $A[1\ldots n][1\ldots n]$ of
real numbers is increasing, if $A[k][l]\geq A[i][j]$ provided $i\leq k$ and
$j\leq l$. Prove that there is no algorithm that finds a real number $X$ in the
increasing array $A[1\ldots n][1\ldots n]$ in less than $n$ comparisons.
\end{problem}
\begin{solution} Take an $n$ tuple of distinct integers $(a_1,\ldots,a_n)$ and
    place them on the secondary diagonal of array $A$. If $i+j<n$ we set
    $A[i][j]=\min(a_1,\ldots,a_n)$ and if $i+j>n$ we set
    $A[i][j]=\max(a_1,\ldots,a_n)$. It is clear that $A$ is an increasing array
    in the sense given above. Since the knowledge of the value of the element on
    the secondary diagonal doesn't say anything about values of other elements
    on the secondary diagonal, and these values could be arbitrary we need at
    least iterate through the whole secondary diagonal. Since it is of length
    $n$, then we need at least $n$ comparisons.
\end{solution}

\begin{problem} A linear map of $n$-dimensional space has $n+1$ eigenvector,
such that any $n$ of them is linearly independent. Describe all such matrices.
\end{problem}
\begin{solution} By assumption the linear map $A$ has a $n+1$ eigenvectors, say
    ${(e_i)}_{i\in\mathbb{N}_{n+1}}$ with eigenvalues
    ${(\lambda_i)}_{i\in\mathbb{N}_{n+1}}$. By assumption the vectors
    ${(e_i)}_{i\in\mathbb{N}_n}$ are linearly independent. Since $A$ is defined
    on $n$ dimensional linear space, then ${(e_i)}_{i\in\mathbb{N}_n}$ is a
    basis. Hence there exist coefficients ${(\alpha_i)}_{i\in\mathbb{N}_n}$ such
    that $e_{n+1}=\sum_{i=1}^n \alpha_i e_i$. By assumption
    $(e_1,\ldots,e_{m-1},e_{n+1},e_{m+1},\ldots,e_n)$ is a system of linearly
    independent vectors for all $m\in\mathbb{N}_n$. Since $A$ is defined on $n$
    dimensional linear space, then this system of vectors is a basis. Therefore
    the determinant
    $$
        \begin{vmatrix}
            1            &
            0            &
            \ldots       &
            0            &
            \alpha_1     &
            0            &
            \ldots       & 0 \\
            0            &
            1            &
            \ldots       &
            0            &
            \alpha_2     &
            0            &
            \ldots       &
            0                \\
            \ldots       &
            \ldots       &
            \ldots       &
            \ldots       &
            \ldots       &
            \ldots       &
            \ldots       &
            \ldots           \\
            0            &
            0            &
            \ldots       &
            1            &
            \alpha_{m-1} &
            0            &
            \ldots       &
            0                \\
            0            &
            0            &
            \ldots       &
            0            &
            \alpha_{m}   &
            0            &
            \ldots       &
            0                \\
            0            &
            0            &
            \ldots       &
            0            &
            \alpha_{m+1} &
            1            &
            \ldots       &
            0                \\
            \ldots       &
            \ldots       &
            \ldots       &
            \ldots       &
            \ldots       &
            \ldots       &
            \ldots       &
            \ldots           \\
            0            &
            0            &
            \ldots       &
            0            &
            \alpha_n     &
            0            &
            \ldots       &
            1                \\
        \end{vmatrix}
    $$
    is non-zero. Clearly, this determinant equals to $\alpha_m$, so we get
    $\alpha_m\neq 0$ for all $m\in\mathbb{N}_n$. Since $e_{n+1}$ is an
    eigenvector, then $\lambda_{n+1}=\lambda_k$ for some $k\in\mathbb{N}_n$.
    Thus $A(e_{n+1})=\lambda_k e_{n+1}$. Recall the formula for $e_{n+1}$, and
    the fact that $A(e_n)=\lambda_n e_n$. Therefore we get
    $\sum_{i=1}^n\alpha_i(\lambda_i-\lambda_k)e_i=0$. Since
    ${(e_i)}_{i\in\mathbb{N}_n}$ is a basis $\alpha_i(\lambda_i-\lambda_k)=0$
    for all $i\in\mathbb{N}_n$. As we showed earlier $\alpha_i\neq 0$ for all
    $i\in\mathbb{N}_n$, so for $\lambda_i=\lambda_k$ for all $i\in\mathbb{N}_n$.
    In other words all eigenvalues of $A$ are equal. It remains to show that any
    matrix with $n$ equal eigenvalues has $n+1$ eigenvector with the property
    that any $n$ of them are linearly independent. Indeed we can take first $n$
    vectors ${(e_i)}_{i\in\mathbb{N}}$ to be eigenvectors corresponding to each
    eigenvalue and set $e_{n+1}=\sum_{i=1}^n e_i$. By argument with determinant
    given above it follows that any $n$ vectors of the system
    ${(e_i)}_{i\in\mathbb{N}}$ are linearly independent.
\end{solution}

\begin{problem} Compute
$$
    \sum_{n=1}^\infty\frac{f(n)}{n(n+1)}
$$
where $f(n)$ is the number of 1's in the binary representation of $n$.
\end{problem}
\begin{solution} For each $i\in\mathbb{N}^*$ by $f_i(n)$ we denote $i$-th digit
    in the binary exapnsion of $n$, then for any $n\in\mathbb{N}$ we have
    $n=\sum_{i=0}^\infty f_i(n) 2^i$. Note that $f_i(n)=1$ iff $n=k+2^i+j
        2^{i+1}$ for some $k\in \{0,\ldots,2^i-1\}$ and $j\in\mathbb{N}$. Now we
    compute the desired sum
    $$
        \sum_{n=1}^\infty\frac{f(n)}{n(n+1)}
        =\sum_{n=1}^\infty\frac{1}{n(n+1)}\sum_{i=0}^\infty f_i(n)
    $$
    $$
        =\sum_{i=0}^\infty\sum_{n=1}^\infty \frac{f_i(n)}{n(n+1)}
        =\sum_{i=0}^\infty\sum_{j=0}^\infty\sum_{k=0}^{2^i-1}
        \frac{1}{(k+2^i+j 2^{i+1})(k+1+2^i+j2^{i+1})}
    $$
    $$
        =\sum_{i=0}^\infty\sum_{j=0}^\infty\sum_{k=0}^{2^i-1}
        \left(\frac{1}{k+2^i+j 2^{i+1}}-\frac{1}{k+1+2^i+j 2^{i+1}}\right)
        =\sum_{i=0}^\infty\sum_{j=0}^\infty
        \left(\frac{1}{2^i+j 2^{i+1}}-\frac{1}{2^i+2^i+j 2^{i+1}}\right)
    $$
    $$
        =\sum_{i=0}^\infty\frac{1}{2^i}\sum_{j=0}^\infty
        \left(\frac{1}{1+2j}-\frac{1}{2+2j}\right)
        =\left(\sum_{i=0}^\infty\frac{1}{2^i}\right)\sum_{j=0}^\infty
        \frac{{(-1)}^j}{j+1}
        =\frac{1}{1-1/2}\sum_{j=0}^\infty{(-1)}^j\int_0^1 x^j dx=
    $$
    $$
        =2\int_0^1 \sum_{j=0}^\infty{(-1)}^j x^j dx
        =2\int_0^1 \sum_{j=0}^\infty{(-x)}^j dx
        =2\int_0^1 \frac{1}{1+x} dx
        =2\ln2
    $$
\end{solution}













\newpage

\subsection{Exam on 01.06.2012}

\begin{problem} A recurrence sequence defined as follows $x_1=a$, $x_2=b$ and
$x_{n+1}=\frac{1}{2}(x_n+x_{n-1})$ for $n> 1$. Find $\lim\limits_{n\to\infty}
    x_n$.
\end{problem}
\begin{solution} Denote $y_n=x_{n+1}-x_n$, then $y_1=b-a$ and
    $y_n=-\frac{1}{2}y_{n-1}$. Therefore
    $y_n={\left(-\frac{1}{2}\right)}^{n-1}(b-a)$. Note that
    $$
        x_n-x_1
        =\sum\limits_{k=1}^{n-1} y_k
        =\sum\limits_{k=1}^{n-1} {\left(-\frac{1}{2}\right)}^{k-1}(b-a)
    $$
    so
    $$
        \lim\limits_{n\to\infty} x_n
        =\lim\limits_{n\to\infty}
        \left(
        x_1+
        \sum\limits_{k=1}^{n-1} {\left(-\frac{1}{2}\right)}^{k-1}(b-a)
        \right)
        =a+(b-a)\sum\limits_{k=1}^\infty {\left(-\frac{1}{2}\right)}^{k-1}
        =a+\frac{2}{3}(b-a)
        =\frac{a+2b}{3}
    $$
\end{solution}

\begin{problem} Compute $\int_0^1 \varphi(x)\varphi'(x)dx$ where
$\varphi(x)=\sum_{k=1}^\infty2^{-2\lfloor \log_2k\rfloor} x^k$.
\end{problem}
\begin{solution} Direct computation shows
    $$
        I:=\int_0^1 \varphi(x)\varphi'(x)dx
        =\frac{1}{2}\int_0^1 d(\varphi^2(x))
        =\frac{1}{2}(\varphi^2(1)-\varphi^2(0))
    $$
    Clearly, $\varphi(0)=0$, so $I=\frac{1}{2}\varphi^2(1)$. Note that
    $$
        \varphi(1)
        =\sum_{k=1}^\infty2^{-2\lfloor \log_2k\rfloor}
        =\sum_{n=0}^\infty\sum_{i=2^n}^{2^{n+1}} 2^{-2n}
        =\sum_{n=0}^\infty 2^{-2n}\cdot 2^n
        =\sum_{n=0}^\infty 2^{-n}
        =2
    $$
    so $I=\frac{1}{2}\varphi^2(1)=2$
\end{solution}

\begin{problem} Consider all nonempty subsets of $\mathbb{N}_n$. For each subset
multiply its elements. Find the sum of inverses of these $2^n-1$ products.
\end{problem}
\begin{solution} This sum can be written as
    $$
        \prod_{k=1}^n\left(1+\frac{1}{k}\right)-1
        =\frac{2}{1}\cdot\frac{3}{2}\cdot
        \ldots
        \cdot\frac{n+1}{n}-1=n+1-1=n
    $$
\end{solution}

\begin{problem} Wolf Palme and Ravi Shankar toss a fair coin. Wolf tosses the
coin $n$ times, and Ravi $n+1$ times. Find the probability that Ravi had more
heads than Wolf.
\end{problem}
\begin{solution} By $\xi_i$ and $\eta_i$ we denote the number of heads got by
    Ravi and Wolf at $i$-th toss respectively. Then we set $X_k=\sum_{i=1}^k
        \xi_i$ and $Y_k=\sum_{i=1}^k \eta_i$. These random variables are numbers of
    heads got by Ravi and Wolf after $k$  tosses respectively. Denote
    $p=\mathbb{P}(\{X_n>Y_n\})$, then by symmetry $\mathbb{P}(\{X_n<Y_n\})=p$
    and, as the consequence, $\mathbb{P}(\{X_n=Y_n\})=1-2p$. Clearly,
    $X_{n+1}>Y_n$ in two cases $X_n>Y_n$ or $X_n=Y_n$ and on $\xi_{n+1}=1$.
    Therefore
    $$
        \mathbb{P}(\{X_{n+1}>Y_n\})
        =\mathbb{P}(\{X_n>Y_n\})+\mathbb{P}(\{\xi_{n+1}=1|X_n=Y_n\})
    $$
    $$
        =\mathbb{P}(\{X_n>Y_n\})+
        \mathbb{P}(\{\xi_{n+1}=1\})\cdot\mathbb{P}(\{X_n=Y_n\})
        =p+\frac{1}{2}(1-2p)=\frac{1}{2}
    $$
\end{solution}

\begin{problem} Prove that each uncountable set of positive numbers admits a
countable subset with infinite sum.
\end{problem}
\begin{solution} Let $A$ be an uncountable set of positive numbers. Denote $B_n=
        \{x: x>n^{-1}\}$, $A_n=A\cap B_n$, then $A=\cup_{n=1}^\infty A_n$. Since
    $A$ is uncountable and represented as countable  union of sets, then
    there exists at least one $m\in\mathbb{N}$ such that $A_m$ is
    uncountable. Take any sequence ${(a_i)}_{i\in\mathbb{N}}\subset
        A_m\subset A$. Since for all $i\in\mathbb{N}$ we have $a_i\geq m^{-1}$,
    then the series $\sum_{i=1}^\infty a_i$ diverges.
\end{solution}

\begin{problem} Give an algorithm that checks if an array of $n$ integers is a
permutation of integers ${1,\ldots,n}$.  Time complexity $O(n)$.
\end{problem}
\begin{solution} We treat the array as permutation. Thus we need to iterate
    through all its elements and each time we meet a new element we consider it
    as the begining of new cycle. As we iterate through the cycle we set
    respective values in array to $-1$ to indicate that this element was
    visited. We run through this hypothetical cycle until we meet its begining
    or an already iterated element. In the latter case the array is not a
    permutation.

    \begin{verbatim}
#include <iostream>
#include <vector>

bool is_permutation_(std::vector<int> integers)
{
    // dummy check for not being a permutation
    for (int index = 0; index < integers.size(); ++index)
    {
        if (integers[index] > integers.size() || integers[index] < 1)
        {
            return false;
        }
        else
        {
            integers[index]--;
        }
    }

    for (int index = 0; index < integers.size(); ++index)
    {
        // we met not iterated element
        if (integers[index] >= 0)
        {

            int current_position = index;
            int next_position = integers[current_position];
            // run through the cycle and fill it with -1
            do
            {
                integers[current_position] = -1;
                if (next_position == index)
                {
                    //we found begining of our cycle
                    break;
                }
                if (next_position < 0)
                {
                    //we met another cycle!
                    return false;
                }
                current_position = next_position;
                next_position = integers[current_position];
            } while (true);
        }
    }
    return true;
}

int main()
{
    size_t array_size;
    std::cin >> array_size;
    std::vector<int> integers(array_size, 0);
    for (int i = 0; i < integers.size(); ++i)
    {
        std::cin >> integers[i];
    }

    std::cout << (
        is_permutation_(integers) ?
        "is permutation" :
        "not a permutation"
    );

    std::cin >> array_size;
    return 0;
}
\end{verbatim}
\end{solution}

\begin{problem} Let ${(A_i)}_{i\in\mathbb{N}_n}$ be a family of finite sets.
Prove that the matrix $A={(|A_i\cap A_j|)}_{i,j\in\mathbb{N}_n}$ is positive
semidefinite.
\end{problem}
\begin{solution} Denote $A=\cup_{i=1}^n A_i$. For each $x\in A$ define matrix
    $B_x$ such that
    $$
        {(B_x)}_{i,j}=
        \begin{cases}
            1 & \quad\mbox{if}\quad x\in A_i\cap A_j    \\
            0 & \quad\mbox{if}\quad x\notin A_i\cap A_j \\
        \end{cases}
    $$
    We claim that each $B_x$ is positive definite. Fix $x\in A$. Note that
    ${(B_x)}_{i,i}=0$, then $x\notin A_i$ and as the consequence, for all
    $j\in\mathbb{N}_n$ we have ${(B_x)}_{i,j}={(B_x)}_{j,i}=0$. We call this
    property $(S)$.  Denote $I_x$ the set of indices for which
    ${(B_x)}_{i,i}\neq 0$. Now rearragnge enumeration of sets
    ${(A_i)}_{i\in\mathbb{N}_n}$ such that elements of $I_x$ goes first. After
    rearrangement we get the matrix $B_x'$. Since $B_x$ is symmetric and has
    property $(S)$ then matrix $B_x'$ will look like
    $$
        \begin{pmatrix}
            1      &
            1      &
            1      &
            1      &
            0      &
            0      &
            0      &
            \ldots &
            0        \\
            1      &
            1      &
            1      &
            1      &
            0      &
            0      &
            0      &
            \ldots &
            0        \\
            1      &
            1      &
            1      &
            1      &
            0      &
            0      &
            0      &
            \ldots &
            0        \\
            1      &
            1      &
            1      &
            1      &
            0      &
            0      &
            0      &
            \ldots &
            0        \\
            0      &
            0      &
            0      &
            0      &
            0      &
            0      &
            0      &
            \ldots &
            0        \\
            0      &
            0      &
            0      &
            0      &
            0      &
            0      &
            0      &
            \ldots &
            0        \\
            0      &
            0      &
            0      &
            0      &
            0      &
            0      &
            0      &
            \ldots &
            0        \\
            \hdots &
            \hdots &
            \hdots &
            \hdots &
            \hdots &
            \hdots &
            \hdots &
            \hdots &
            \hdots   \\
            0      &
            0      &
            0      &
            0      &
            0      &
            0      &
            0      &
            \ldots &
            0        \\
        \end{pmatrix}
    $$
    Obviously all its minors are zero, so $B_x'$ is positive semidefinite. Since
    after rearrangement positive semidefiniteness is preserved, then $B_x$ is
    also positive semidefinite. Note that $A_x=\sum_{x\in A} B_x$, so $A$ is
    positive semidefinite as sum of positive semidefinite matrices.
\end{solution}
















\newpage

\subsection{Exam on 09.06.2013}

\begin{problem} A sequence ${(a_n)}_{n\in\mathbb{N}}$ is defined recursively
$$
    a_0=1\quad a_{n+1}=\frac{a_n}{1+n a_n}
$$
Find a closed form for $a_n$.
\end{problem}
\begin{solution} Denote $b_n=a_n^{-1}$, then $b_0=1$ and $b_{n+1}=b_n+n$. So
    $$
        b_n-b_0=\sum_{k=1}^{n-1}(b_{k+1}-b_k)=\sum_{k=1}^{n-1}k
        =\frac{1}{2}n(n-1)
    $$
    and finally we get
    $$
        a_n=b_n^{-1}={\left(b_0+\frac{1}{2}n(n-1)\right)}^{-1}
        =\frac{2}{n^2-n+2}
    $$
\end{solution}

\begin{problem} Assume we are given a set $A=\{1,\ldots,n\}$. Find the
probability that $\bigcap_{i=1}^k A_i=\varnothing$ for $k$ randomly chosen
subsets $A_1,\ldots,A_n$ of $A$.
\end{problem}
\begin{solution} Each subset of $A$ can be represented as binary vector of
    length $n$, hence a group of $k$ sets can be represented as $k\times n$
    binary matrix. There are $2^{nk}$ such matrices. We are interested in groups
    of sets with empty intersection. These are matrices without columns of
    $1$'s. There are $2^k-1$ columns which doesn't consist of $1$'s. Since we
    have $n$ columns, the total number of matrices in question is ${(2^k-1)}^n$.
    The desired probability is
    $$
        \frac{{(2^k-1)}^n}{2^{kn}}={(1-2^{-k})}^n
    $$
\end{solution}

\begin{problem} Give an algorithm that finds the longest subarray of binary
array of length $n$ with equal number of units and zeros. Restrictions: $O(n)$
time complexity, $O(n)$ memory.
\end{problem}
\begin{solution} Let's change all zeros in the original array to $-1$. Call this
    array $b$. Now we need to find a subarray of $b$ with zero sum of elements.
    Now consider the graph of the function $F(i)=n+H(i)$ where $H(i)$ is the sum
    of the first $i$ elements of $b$. For each $h\in[0,2n+1]$ we can draw a
    horizontal line on the height $h$. If it intersects the graph of $F$, then
    consider abscissas of the most left and the most right points of
    intersections. Call them $f_h$ and $l_h$, then $b[f_h\ldots l_h]$ is a
    subarray with zero sum. If there is no intersections with horizontal lines
    we set $f_h=l_h=-1$. It remains to iterate through all $h\in[0,2n+1]$ and
    find $h$ with maximal difference $l_h - f_h$ provided $l_h$ and $f_h$ are
    positive.
    \begin{verbatim}
#include <algorithm>
#include <iostream>
#include <vector>

std::pair<size_t, size_t> max_balanced_subarray_edges(
    const std::vector<bool>& values)
{
    std::vector<int> height(values.size() + 1, values.size());
    for (size_t index = 1; index < height.size(); ++index)
    {
        height[index] = height[index - 1] +
            (values[index - 1] ? 1 : -1);
    }

    int flag = -1;

    std::vector<int> first_occurence_of_height(
        2 * values.size() + 1, flag);
    std::vector<int>  last_occurence_of_height(
        2 * values.size() + 1, flag);

    for (size_t index = 0; index < height.size(); ++index)
    {
        if (first_occurence_of_height[height[index]] == flag)
        {
            first_occurence_of_height[height[index]] = index;
        }
        else
        {
            last_occurence_of_height[height[index]] = index;
        }
    }

    size_t begin, end;
    size_t max_length = 0;

    for (size_t height = 0; height < 2 * values.size() + 1; ++height)
    {
        if (first_occurence_of_height[height] != flag &&
            last_occurence_of_height[height] != flag)
        {
            if (last_occurence_of_height[height] -
                first_occurence_of_height[height] > max_length)
            {
                max_length = last_occurence_of_height[height] -
                             first_occurence_of_height[height];
                begin = first_occurence_of_height[height];
                end = last_occurence_of_height[height];
            }
        }
    }

    return std::make_pair(begin, end - 1);
}

int main()
{
    size_t array_size;
    std::cin >> array_size;
    std::vector<bool> values(array_size, false);
    for (size_t index = 0; index < values.size(); ++index)
    {
        int value;
        std::cin >> value;
        values[index] = value != 0 ? true : false;
    }

    std::pair<size_t, size_t> edges = max_balanced_subarray_edges(values);
    std::cout << edges.first << ' ' << edges.second;

    return 0;
}
    \end{verbatim}
\end{solution}

\begin{problem} Find all $m\in \{1,\ldots,10\}$ such that
$$
    I_m=\int_0^{2\pi} \cos(x)\cdot
    \ldots
    \cdot\cos(mx)dx\neq 0
$$
\end{problem}
\begin{solution} Let $C(l,m)$ denote the number or representations of $l$ in the
    form $\sum_{r=1}^m\alpha_r$ where $\alpha_r\in \{-r,r\}$. Then
    $$
        I_m=\int_0^{2\pi}\prod_{k=1}^m\cos(kx)dx
        =\int_0^{2\pi}\prod_{k=1}^m\frac{1}{2}(e^{ikx}+e^{-ikx})dx
        =2^{-m}\int_0^{2\pi}\sum_{l=-m}^m C(l,m) e^{ilx}dx
    $$
    $$
        =2^{-m}\sum_{l=-m}^m C(l,m)\int_0^{2\pi} e^{ilx}dx
            =2^{-m}\sum_{l=-m}^m C(l,m)2\pi \delta_{l,0}
        =2^{1-m}\pi C(0,m)
    $$
    Therefore $I_m\neq 0$ iff $C(0,m)\neq 0$, so we had to decide for each
    $m\in\mathbb{N}_{10}$ whether $0$ is representable in the form
    $\sum_{r=1}^m\alpha_r$ for some $\alpha_r\in \{-r,r\}$. Note that
    $\sum_{r=1}^m\alpha_r$ have the same parity as $S_m:=\sum_{r=1}^m r$. Since
    $S_1, S_2, S_5, S_6, S_9, S_{10}$ are odd then for $m\in \{1,2,5,6,9,10\}$
    we have $C(0,m)=0$ and as the consequence $I_m=0$. For the remaining values
    of $m$ we can present at least one representation: $1+2-3=0$ for $m=3$,
    $1-2-3+4=0$ for $m=4$, $1+2-3-4+5+6-7=0$ for $m=7$, $1+2+3-4+5-6+7-8=0$ for
    $m=8$. Hence, for $m\in \{3, 4, 7, 8\}$ we have $C(0,m)\neq 0$ and $I_m\neq
        0$ too.
\end{solution}

\begin{problem} Let $G$ be a nonoriented graph without loops. Prove its
adjacency matrix has negative a eigenvalue.
\end{problem}
\begin{solution} Since graph is non oriented then its adjacency matrix $A$ is a
    real symmetric matrix. Hence all eigenvalues of $A$ are real. Since $G$ has
    no loops, then its main diagonal contains only zeros, hence trace of $A$ is
    zero. Since trace also equals to the sum of all eigenvalues, then we have
    two separate cases. First case $A$ has positive and negaitve eigenvalues,
    Second case all eigenvalues of $A$ are zero. In the first case we are done,
    in the latter case $A$ is zero, which means that $G$ has no edges.
\end{solution}

\begin{problem} Consider infinite two dimensional array
${(a_{i,j})}_{i,j\in\mathbb{N}}$ with the property that each its element repeats
exactly $8$ times. Prove that there exist $m$ and $n$ such that $a_{m,n}> mn$.
\end{problem}
\begin{solution} Assume $a_{m,n}\leq mn$ for all $m,n\in\mathbb{N}$. For any
    fixed $k\in\mathbb{N}$  consider set $A_k=\{(m,n):mn\leq k\}$. Clearly
    $|A_k|\leq \int_2^k \frac{k}{x}dx=k\ln(k/2)$. From assumption $|A_k|\geq 8k$
    for all $k\in\mathbb{N}$. Therefore $8k\leq k\ln(k/2)$, for all
    $k\in\mathbb{N}$. Contradiction, hence there exist $(m,n)\in\mathbb{N}^2$
    such that $a_{m,n}>mn$.
\end{solution}

\begin{problem} Assume we are given a binary matrix such that each its row is
zero or contains exactly one group of $1$'s followed one after another. Prove
that determinant of such matrix is $0$ or $1$ or $-1$.
\end{problem}
\begin{solution} By basic properties of determinants we can arrange rows of the
    matrix $A$ to get upper trapezoid form $B$. During rearrangement, if we have
    a group of rows where $1$'s start at the same index we place on the top a
    row with smaller length of group of $1$'s. After such rearrangemnt of rows
    determinant of original matrix can only change its sign. Thus it is enough
    to prove the result for matrices of the upper trapezoid form with the
    property that rows having the same index of the first $1$ sorted from top to
    bottm by increase of the length of the group of $1$'s. In other words we
    need to prove that determinant of the matrix of the form like
    $$
        \begin{pmatrix}
            1      &
            1      &
            1      &
            0      &
            0      &
            0      &
            0      &
            \ldots &
            0        \\
            1      &
            1      &
            1      &
            1      &
            0      &
            0      &
            0      &
            \ldots &
            0        \\
            0      &
            1      &
            1      &
            0      &
            0      &
            0      &
            0      &
            \ldots &
            0        \\
            0      &
            1      &
            1      &
            1      &
            0      &
            0      &
            0      &
            \ldots &
            0        \\
            0      &
            1      &
            1      &
            1      &
            1      &
            1      &
            0      &
            \ldots &
            0        \\
            0      &
            0      &
            0      &
            1      &
            1      &
            0      &
            0      &
            \ldots &
            0        \\
            0      &
            0      &
            0      &
            0      &
            1      &
            1      &
            1      &
            \ldots &
            0        \\
            \ldots &
            \ldots &
            \ldots &
            \ldots &
            \ldots &
            \ldots &
            \ldots &
            \ldots &
            \ldots   \\
            0      &
            0      &
            0      &
            0      &
            0      &
            0      &
            0      &
            \ldots &
            1        \\
        \end{pmatrix}
    $$
    is $0$ or $\pm 1$.

    We prove this claim by induction on size of the matrix. Basis of induction:
    if $n=1$ then matrix is either $(0)$ or $(1)$, so its determinant is $0$ or
    $1$. Induction step: assume claim is valid for $n=k$. Consider arbitrary
    $(k+1)\times(k+1)$-matrix $B$ of the form described above. If its first
    column is zero, then $\det(B)=0$. Otherwise, first element of the first row
    is $1$. We use it as pivot (if necessary) to make zeros in the first column
    under it. This is possible since subtraction of rows doesn't change the
    deteminant. Thanks to arrangement of rows after such subtractions we get the
    matrix $B'$ of the same form. For example, for matrix described above we get
    $$
        \begin{pmatrix}
            1      &
            1      &
            1      &
            0      &
            0      &
            0      &
            0      &
            \ldots &
            0        \\
            0      &
            0      &
            0      &
            1      &
            0      &
            0      &
            0      &
            \ldots &
            0        \\
            0      &
            1      &
            1      &
            0      &
            0      &
            0      &
            0      &
            \ldots &
            0        \\
            0      &
            1      &
            1      &
            1      &
            0      &
            0      &
            0      &
            \ldots &
            0        \\
            0      &
            1      &
            1      &
            1      &
            1      &
            1      &
            0      &
            \ldots &
            0        \\
            0      &
            0      &
            0      &
            1      &
            1      &
            0      &
            0      &
            \ldots &
            0        \\
            0      &
            0      &
            0      &
            0      &
            1      &
            1      &
            1      &
            \ldots &
            0        \\
            \ldots &
            \ldots &
            \ldots &
            \ldots &
            \ldots &
            \ldots &
            \ldots &
            \ldots &
            \ldots   \\
            0      &
            0      &
            0      &
            0      &
            0      &
            0      &
            0      &
            \ldots &
            1        \\
        \end{pmatrix}
    $$
    Now we expand this deteminant in the first columnn and see that $\det(B')=1
        M_{1,1}=M_{1,1}$. After sutable rearrangement of rows the minor
    $M_{1,1}$ of matrix $B'$ becomes a determinant of the $k\times k$ matrix
    of the form described above. By inductive assummption we have $M_{1,1}$
    is either $0$ or $\pm 1$. Thus, $\det(B)=\det(B')=\det(M_{1,1})$ is
    either $0$ or $\pm 1$. Induction step finished.
\end{solution}














\newpage

\subsection{Exam on 19.02.2014}

\begin{problem} Find all $3\times 3$ matrices $X$ such that $X^2+E=0$.
\end{problem}
\begin{solution} Since we consider $3\times 3$ matrices, then $\det(-E)=-1$.
    Hence, ${\det(X)}^2=\det(X)\det(X)=\det(X^2)=\det(-E)=-1<0$. Contradiction,
    hence such matrix doesn't exist.
\end{solution}

\begin{problem} During camping among any four guys at least one is acquainted
with other three guys. Prove that each participant save at most three is
acquainted with others.
\end{problem}
\begin{solution} We call participants that are not acquainted with all others
    non-friendly with respect to his group. The remaining participants are
    called friendly. We perform proof by induction on the number of participants
    $n$. Base of induction for $n=4$ is obvious. Assume the claim is true for
    any group of at most $n$ participants. Consider group $G$ of $n+1$
    participants. Take any subgroup $G'$ of $n$ participants. From statement of
    the problem it is clear that there cannot exist a unique non-friendly
    participant with respect to $G'$, because all other participants must be
    acquainted with him, and therefore non-friendly participant must be
    friendly. Therefore from here we have two cases.

    First case: there are two non-friendly participants in $G'$. Call them $a$
    and $b$. Clearly all participants from $G'\setminus \{a,b\}$ are acquainted
    with $a$ and $b$. Therefore $a$ and $b$ don't know each other. Consider a
    unique participant $z\in G\setminus G'$ and any participant $p\in
        G'\setminus \{a,b\}$. Consider group $S=\{a,b,p,z\}$. By assumption we have
    a participant in $S$ who knows others. Since $a$ and $b$ don't know each
    other this must be either $p$ or $z$. In both cases we get that $z$ is
    acquainted with $p$. Since $p$ is arbitrary we get that $z$ knows all
    participants from $G'\setminus \{a,b\}$. If $z$ knows $a$ and $b$ he is
    friendly in $G$, so $G$ has two non-friendly participants and inductive step
    is finished in this subcase. If $z$ does not know $a$ or $b$, then he is
    non-friendly in $G$, but $G$ still has the property that among any $4$
    participants there is a friendly participant. Hence the claim is proved in
    the this subcase too.

    Second case: there are three non-friendly participants in $G'$. Call them
    $a$, $b$ and $c$. Clearly all participants from $G'\setminus \{a,b,c\}$ are
    familiar with $a$, $b$ and $c$. Hence among $a$, $b$ and $c$ there is at
    least one pair of participants who don't know each other. Consider a unique
    participant $z\in G\setminus G'$ and consider group $\{z,a,b,c\}$. If there
    is no acquaintances among $a$, $b$ and $c$ then from the problem statement
    it follows that $z$ knows $a$, $b$ and $c$. Assume now there are
    acquaintances among $a$, $b$ and $c$. Clearly $a$, $b$ and $c$ can't know
    each other because they are non friendly. Clearly there is no participant in
    $\{a,b,c\}$ that knows the remaining two, because otherwise he will be
    friendly in $G'$. Therefore there is only one acquaintance, say $a$ knows
    $b$. Again from problem statement there is a participant in $\{z,a,b,c\}$
    that knows the remaining participants. The only possibility here is that $z$
    knows $a$, $b$ and $c$. All in all we proved that $z$ knows $a$, $b$ and
    $c$. Now take any $p\in G'\setminus \{a,b,c\}$ and consider group
    $S=\{z,p,a,c\}$. Since $p$ knows $a$ and $c$ and $c$ doesn't know $a$, then
    from problem statement we have that either $z$ or $p$ knows the remaining
    participants in $S$. In both cases $z$ knows $p$. Since $p\in G'\setminus
        \{a,b,c\}$ is arbitrary, then $z$ knows all participants in $G'$. But he
    also knows $a$, $b$ and $c$. So he is friendly in $G$. Therefore the only
    non-friendly participants in $G$ are $a$, $b$ and $c$. Thus the claim is
    proved in the second case too and the step of induction is completed.
\end{solution}

\begin{problem} Describe all non-singular real-valued matrices $A$ such that all
entries of $A$ and $A^{-1}$ are non negative.
\end{problem}
\begin{solution} Assume $A$ is invertible $n\times n$ matrix and $A$ with
    $A^{-1}$ have non negative entries. Denote $B=A^{-1}$, then $AB=BA=E$. Hence
    for all $i,k\in\mathbb{N}_n$
    $$
        \sum_{j=1}^n A_{i,j} B_{j,k}=\delta_{i,j}
    $$
    Since $A$ and $B$ have non negative entries, then for any
    $i,j,k\in\mathbb{N}_n$ such that $i\neq k$ we have $A_{i,j}=0$ or
    $B_{j,k}=0$. Assume there exist $i,j_1,j_2\in\mathbb{N}_n$ such that
    $j_1\neq j_2$ and $A_{i,j_1}\neq 0$, $A_{i,j_2}\neq 0$, then for any $k\neq
        i$ we have $B_{j_1,k}=0$ and $B_{j_2,k}=0$. Clearly, $B_{j_1,i}\neq 0$ and
    $B_{j_2,i}\neq 0$, otherwise $B$ will have a zero row and will be singular.
    Since $B_{j_1,i}\neq 0$ and $B_{j_2,i}\neq 0$ while $B_{j_1,k}=B_{j_2,k}=0$
    for all $k\neq i$, then $B$ have two linearly dependent rows,hence it is
    singular. Contradiction, therefore for any $i\in\mathbb{N}_n$ there is
    unique $\pi(i)\in\mathbb{N}_n$ such that $A_{i,\pi(i)}\neq 0$. Assume
    $\pi(i_1)=\pi(i_2)$ for some $i_1,i_2\in\mathbb{N}_n$ such that $i_1\neq
        i_2$, then $A$ have two linearly dependent rows, hence $A$ is singular.
    Contradiction, so $\pi:\mathbb{N}_n\to\mathbb{N}_n$ is an injective
    function, hence $\pi$ is a permutation. Thus matrix $A$ is a permutation of
    columns of a diagonal matrix $D$. Diagonal elements of $D$ are nonzero,
    because $A$ is invertible, and positive because $A$ is non-negative.

    Conversely, let $A$ be constructed from diagonal matrix
    $D=\operatorname{diag}(a_1,\ldots,a_n)$ where $a_i>0$ for all
    $i\in\mathbb{N}_n$ by permutation of its columns. Denote this permutation by
    $\pi$, then $A(e_{\pi(i)})=a_i e_i$, where ${(e_i)}_{i\in\mathbb{N}_n}$ is a
    standard basis of $\mathbb{R}^n$. It is clear now, that
    $A^{-1}(e_i)=a_i^{-1}e_{\pi(i)}$. Hence $A^{-1}$ have non negative entries.
\end{solution}

\begin{problem} Give an algorithm that finds the maximal value of sums of
subarrays of a given array $a[1\ldots n]$.  Time complexity $O(n)$, memory
restriction $O(1)$.
\end{problem}
\begin{solution} The maximum is initially zero. Suppose that we've solved the
    problem for subarray $a[1\ldots i-1]$ how can we extend that to a solution
    for the first $i$ elements? The maximum sum in the first $i$ elements is
    either the maximum sum in the first $i-1$ elements (which we'll call max so
    far), or it is that of a subvector that ends in position $i$ (which we'll
    call max ending here). Instead of computing the maximum subvector ending in
    position I from scratch, we'll use the maximum subvector that ends in
    position $i-1$.
    \begin{verbatim}
#include <iostream>
#include <algorithm>
#include <vector>

double maximal_subarray_sum(const std::vector<double>& values)
{
    double max_so_far = 0.;
    double max_ending_here = 0.;
    for (size_t index = 0; index < values.size(); ++index)
    {
        max_ending_here = std::max(max_ending_here + values[index], 0.);
        max_so_far = std::max(max_so_far, max_ending_here);
    }

    return max_so_far;
}

int main()
{
    size_t size;
    std::cin >> size;
    std::vector<double> values(size);
    for (size_t index = 0; index < values.size(); ++index)
    {
        std::cin >> values[index];
    }

    std::cout << maximal_subarray_sum(values);

    return 0;
}
\end{verbatim}
\end{solution}

\begin{problem} There are $10$ coins. With one weighing one can determine which
coin is heavier. Is it possible to find the heaviest coin in $20$ weighings?
\end{problem}
\begin{solution} There are $10{!}$ different rearragements of $10$ coins. This
    knowledge is worth of $\lfloor\log_2(10!)\rfloor+1$ bits, which is $23$ bits
    of information. Each weighing gives us $1$ bit of information, hence $20$
    weighings is not enough.
\end{solution}

\begin{problem} Compute
$$
    \int_{\sqrt{\pi/6}}^{\sqrt{\pi/3}}\sin(x^2)dx+
    \int_{1/2}^{\sqrt{3}/2}\sqrt{\arcsin(x)}dx
$$
\end{problem}
\begin{solution} Note that if $f$ is a strictly increasing continuous function,
    then
    $$
        \int_a^b f(x)dx+\int_{f(a)}^{f(b)} f^{-1}(x)dx=bf(b)-af(a)
    $$
    Applying this result to $f=\sin(x^2)$ with $a=\sqrt{\pi/6}$ and
    $b=\sqrt{\pi/3}$ we get
    $$
        \int_{\sqrt{\pi/6}}^{\sqrt{\pi/3}}\sin(x^2)dx+
        \int_{1/2}^{\sqrt{3}/2}\sqrt{\arcsin(x)}dx
        =\sqrt{\frac{\pi}{3}}\cdot\frac{\sqrt{3}}{2}-
        \sqrt{\frac{\pi}{6}}\cdot\frac{1}{2}
        =\frac{\sqrt{\pi}(\sqrt{6}-1)}{2\sqrt{6}}
    $$
\end{solution}

\begin{problem} A game consist of equal independent rounds, with probability of
win equal to $p$. If a player wins he gets $1$ dollar, otherwise he pays $1$
dollar. As soon as he collect $N$ dollars he finishes the game. Find the
probability that the player will lose all his money if he starts the game with
$K$ dollars.
\end{problem}
\begin{solution} Let $P_k(s)$ be the probaility of player to loose after making
    at most $k$ moves provided his current current stock is $s$ dollars. Then we
    have
    $$
        P_k(s)=pP_{k-1}(s+1)+(1-p)P_{k-1}(s-1)
    $$
    with initial conditions $P_0(s)=0$ for $s>0$, $P_0(s)=1$ for $s\leq $ and
    $P_k(0)=1$, $P_k(N)=0$ for all $k\in\mathbb{N}$. Since the number of
    possible moves is not bounded, we need to consider asymptotical behaviour of
    the probability, i.e.\ find the closed form of $P(s)=\lim_{k\to\infty}
        P_k(s)$. So we have the following recurrence equation
    $$
        P(s)=pP(s+1)+(1-p)P(s-1)
    $$
    with bounddary condition $P(0)=1$, $P(N)=0$. This is recurrence equation of
    the second order with characteristic equation
    $$
        p\lambda-1+\frac{1-p}{\lambda}=0
    $$
    Its roots are $\lambda=1-p/p$ and $\lambda=1$. So
    $$
        P(s)=C_1{\left(\frac{1-p}{p}\right)}^s+C_2
    $$
    Since $P(0)=1$, $P(N)=0$ we get
    $$
        C_1+C_2=1\qquad C_1{\left(\frac{1-p}{p}\right)}^N+C_2=1
    $$
    so
    $$
        C_1=\frac{p^N}{p^N-{(1-p)}^N}\qquad C_2=-\frac{{(1-p)}^N}{p^N-{(1-p)}^N}
    $$
    $$
        P(s)=\frac{p^N}{p^N-{(1-p)}^N}\left({\left(\frac{1-p}{p}\right)}^s-
        {\left(\frac{1-p}{p}\right)}^N\right)
    $$
    The desired  probability equals
    $$
        P(K)=\frac{p^N}{p^N-{(1-p)}^N}\left({\left(\frac{1-p}{p}\right)}^K-
        {\left(\frac{1-p}{p}\right)}^N\right)
    $$
\end{solution}

\begin{problem} For a given real number $a$ and an integer $n\geq 0$ by $a_n$ we
denote the distantce from $a$ to the closest number of the form $m/2^n$ where
$m$ is an integer. Find the maximal possible value of
$$
    \sum_{n=0}^\infty a_n
$$
\end{problem}
\begin{solution} Clearly, for any real $a$ we have $a_n\leq 2^{-n-1}$ so
    $\sum_{n=0}^\infty a_n\leq \sum_{n=0}^\infty 2^{-n-1}=1$. This is maximal
    possible value for the sum in question and maximum is attained for $a=0$.
    Indeed, $a^*=0$ and $a_n^*=2^{-n}$ for $n\in\mathbb{N}$, hence
    $\sum_{n=0}^\infty a_n^*=0+\sum_{n=1}^\infty 2^{-n}=1$.
\end{solution}

\end{document}