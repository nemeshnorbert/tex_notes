\documentclass[12pt]{article}
\usepackage[
    left=2cm,right=2cm, top=2cm,bottom=2cm,bindingoffset=0cm]{geometry}
\usepackage{amssymb,amsmath}
\usepackage[utf8]{inputenc}
\usepackage{mathrsfs}

\newtheorem{theorem}{Theorem}[subsection]
\newtheorem{lemma}[theorem]{Lemma}
\newtheorem{proposition}[theorem]{Proposition}
\newtheorem{remark}[theorem]{Remark}
\newtheorem{corollary}[theorem]{Corollary}
\newtheorem{definition}[theorem]{Definition}
\newtheorem{example}[theorem]{Example}

\newenvironment{proof}{\par $\triangleleft$}{$\triangleright$}

\begin{document}

\begin{center}

    \Large \textbf{Quick introduction to probability and statistics}\\[0.5cm]
    \small {Nemesh N. T.}\\[0.5cm]

\end{center}

\section{Probability theory}

Probability theory studies randomness in a rigorous mathematical fashion. The
following notions are the key concepts in probability theory:
\begin{itemize}
    \item Probability space $\Omega$ --- describes possible outcomes we can
          expect while studying a random phenomena;
    \item Probability measure --- describes how probable a specific 
          random outcome;
    \item Random variable --- a quantity whose value is random;
    \item Distribution function --- a function that describes how probable a
          specific value is for a given random variable
    \item Expected value --- an average value of a random variable;
    \item Variance --- a quantity describing volatility of a random variable.
\end{itemize}


%%%%%%%%%%%%%%%%%%%%%%%%%%%%%%%%%%%%%%%%%%%%%%%%%%%%%%%%%%%%%%%%%%%%%%%%%%%%%%%%

\subsection{Probability space}

We shall discuss all these notions, give definitions and list their properties.

\begin{definition}
    Probablity space is a triple $(\Omega,\mathcal{F}, \mathbb{P})$, where

    \begin{itemize}
        \item $\Omega$ is a set of all elementary events;
        \item $\mathcal{F}\subset 2^\Omega$ is a family of set representing events 
              we are interested in;
        \item $\mathbb{P}$ is a probablity measure 
              $\mathbb{P}:\mathcal{F}\to [0,1]$.
    \end{itemize}
    The family $\mathcal{F}$ has the follwing properties
    \begin{itemize}
        \item $\varnothing\in\mathcal{F}$, that is we always can speak of the 
              event that defenitely will not happen; 
        \item if $A\in\mathcal{F}$, then $\Omega\setminus A\in\mathcal{F}$, that
              is if an event $A$ happens, then we can reason about the opposite
              event, that $A$ didn't happen;
        \item if ${(A_n)}_{n=1}^\infty\subset\mathcal{F}$, then 
              $\bigcup_{i=1}^\infty A_n\in\mathcal{F}$, that is if we have an 
              infinite sequence of events ${(A_n)}_{n=1}^\infty$ we can speak of 
              the event that at least one event $A_n$ happened.   
    \end{itemize}

    The function $\mathbb{P}$ must satisfy equations
    \begin{itemize}
        \item $\mathbb{P}(\Omega)=1$
        \item for any family ${(A_n)}_{n=1}^\infty\subset\mathcal{F}$ such that 
              $A_n\cap A_m=\varnothing$ for $n\neq m$ holds
              $$
              \mathbb{P}\left(\bigcup_{n=1}^\infty A_n\right)
              =\sum_{n=1}^\infty \mathbb{P}(A_n)
              $$
    \end{itemize}
\end{definition}

Any single element $\{\omega \} \in\Omega$ is called an elementary event.

\begin{remark} As a consequence 
    $$
        \mathbb{P}(A\cup B)=\mathbb{P}(A)+\mathbb{P}(B)-\mathbb{P}(A\cap B)
    $$
    whenever $A,B\in\mathcal{F}$ and $A\cap B=\varnothing$. In general
    $$
        \mathbb{P}(A\cup B)=\mathbb{P}(A)+\mathbb{P}(B)-\mathbb{P}(A\cap B)
    $$
\end{remark}



\begin{example} Consider an experimant where we toss a dice once. 
    Then $\Omega=\{1,2,3,4,5,6\}$. 
    Elementary events: $\{1\}$, $\{2\}$, $\ldots$, $\{6\}$. 
    An event --- any subset of $\Omega$, e.g. $\{1,4,5\}$.
\end{example}

\begin{example} Consider an experiment where we toss a coin once. 
    Then $\Omega=\{H, T\}$, $\mathcal{F}$ ---
    all subsets of $\Omega$ and $\mathbb{P}(\{H \})=1/2$, 
    $\mathbb{P}(\{T \})=1/2$
\end{example}

\begin{example} Consider an experiment where we toss a coin twice. 
    Then define we can define a probability space as follows
    $\Omega=\{(H,H), (H, T), (T, H), (T,T)\}$, 
    $\mathcal{F}$ --- all subset of $\Omega$
    and
    $$
        \mathbb{P}(\{(H,H) \})=\mathbb{P}(\{(H,T) \})
        =\mathbb{P}(\{(T,H) \})
        =\mathbb{P}(\{(T,T) \})=1/4.
    $$
\end{example}

\begin{example} Throwing a dot on a segment. Let $\Omega=[0,1]$, $\mathcal{F}$ 
    is a set of Borel subset of $\Omega$ (it is quite complicated, so you can 
    think that $\mathcal{F}$ contains almost any subset of $\Omega$). Finally, 
    let $\mathbb{P}(A)=\mbox{length}(A)$. If $A=[0.2, 0.4]$, then $P(A)=0.2$. If
    $B=[0.7,0.8]$, then $\mathbb{P}(B)=0.1$ and 
    $\mathbb{P}(A\cup B)=\mathbb{P}(A)+\mathbb{P}(B)=0.3$
\end{example}

\begin{definition} Two events $A,B\subset \Omega$ are called inpedendent if
    $$
        \mathbb{P}(A\cap B)=\mathbb{P}(A)\cdot\mathbb{P}(B)
    $$
\end{definition}

This mathematical definition of independent events is consistent with our usual
understanding of independent events.

\begin{example} Tossing a coin twice. Let $A$ denote the event that first toss 
    gave tails, and $B$ denote the event that second toss gave heads. Then
    $$
        A=\{(T,H),(T,T)\},\quad B=\{(T,H),(H,H)\},\quad A\cap B=\{(T,H)\},
    $$
    $$
        \mathbb{P}(A)=\frac{1}{4}+\frac{1}{4}=\frac{1}{2},\quad
        \mathbb{P}(B)=\frac{1}{4}+\frac{1}{4}=\frac{1}{2},\quad
        \mathbb{P}(A\cap B)=\frac{1}{4}
    $$
    As we see $A$ and $B$ are independent in usual sense and $\mathbb{P}(A\cap
        B)=\mathbb{P}(A)\cdot\mathbb{P}(B)$.
\end{example}

\begin{definition} Events $A_1,\ldots,A_n$ are called independent if
    $$
        \mathbb{P}(A_{i_1}\cap\ldots\cap A_{i_m})
        =\mathbb{P}(A_{i_1})\cdot
        \ldots
        \cdot\mathbb{P}(A_{i_m})
    $$
    for any $\{i_1,\ldots,i_m\}\subset \{1,\ldots,n\}$
\end{definition}

\begin{definition} Let $A$ and $B$ be two events and $P(B)\neq 0$, then the
    conditional probability of $A$ given $B$ is defined by
    $$
        \mathbb{P}(A|B)=\frac{\mathbb{P}(A\cap B)}{\mathbb{P}(B)}
    $$
\end{definition}

\begin{example} Conditional probability measures how probable the event $A$ is
    \underline{given} that event $B$ already happened. For example, consider
    experiment where we toss a coin twice. Then $\Omega=\{(H, H), (H, T), (T,H),
        (T,T)\}$. Let $A$ be the event that after two tosses we got two different
    outcomes, and let $B$ be the event that there was at first toss we got head.
    In this case
    $$
        A=\{(H,T),(T,H)\}\quad\quad B=\{(H,T),(H,H)\}
    $$
    Then
    $$
        \mathbb{P}(A)=\mathbb{P}(\{(H,T),(T,H)\})
        =\frac{1}{2}\quad\quad \mathbb{P}(B)
        =\mathbb{P}(\{(H,T),(H,H)\})=\frac{1}{2}
    $$
    $$
        \mathbb{P}(A\cap B)=\mathbb{P}(\{(H,T)\})
        =\frac{1}{4}\quad\quad \mathbb{P}(A|B)
        =\frac{\mathbb{P}(A\cap B)}{\mathbb{P}(B)}
        =\frac{\frac{1}{4}}{\frac{1}{2}}=\frac{1}{2}
    $$
\end{example}

\begin{remark} If $A$ and $B$ are independent events and $\mathbb{P}(B)\neq 0$,
    then
    $$
        \mathbb{P}(A|B)=\mathbb{P}(A)
    $$
    Indeed,
    $$
        \mathbb{P}(A|B)
        =\frac{\mathbb{P}(A\cap B)}{\mathbb{P}(B)}
        =\frac{\mathbb{P}(A)\mathbb{P}(B)}{\mathbb{P}(B)}=\mathbb{P}(A)
    $$
\end{remark}

%%%%%%%%%%%%%%%%%%%%%%%%%%%%%%%%%%%%%%%%%%%%%%%%%%%%%%%%%%%%%%%%%%%%%%%%%%%%%%%%

\subsection{Random variables}

\begin{definition} A random varialbe is a function $X:\Omega\to \mathbb{R}$ on a
    probability space $(\Omega,\mathcal{F},\mathbb{P})$, such that for any 
    $c\in\mathbb{R}$, the set $\{ \omega\in\Omega: X(\omega)<c\}$ lies 
    in $\mathcal{F}$.
\end{definition}

\begin{example} Let $A$ be any event in a probability space 
    $(\Omega,\mathcal{F},\mathbb{P})$. Then the funciton
    $$
        1_A:\Omega\to\mathbb{R}:\omega\to
        \begin{cases}
            1 & \mbox{ if }\omega\in A    \\
            0 & \mbox{ if }\omega\notin A
        \end{cases}
    $$
    is called the indicator function of $A$.
\end{example}

\begin{example} Consider an experiment where we toss a dice once. Then
    $$
        X:\Omega\to\mathbb{R}:\omega\to \omega+10
    $$
    is a random variable.
\end{example}

\begin{example} Suppose we are tossing a coin. With probability $p$ we get tails
    and we set $X=1$, otherwise we set $X=0$. This random variable is called 
    the Bernoulli random variable. We denote this fact as $X\sim Ber(p)$.
\end{example}

\begin{example} Suppose we are tossing a coin $n$ times. If $X$ is the numbers
    of tails after $n$ tosses, with say that $X$ is binomial random variable. We
    write this fact as $X\sim Bin(n,p)$. Clearly $X$ can be represented as a sum
    of $n$ Bernoulli random variables $X_1,\ldots,X_n$, i.e.
    $X=X_1+\ldots+X_n$.    % chktex 11
    Indeed, just pick $X_i=1$ if we got tails at $i$-th toss and $X_i=0$ if we
    got head at $i$-th toss.
\end{example}

\begin{example} Let $\Omega$ be a unit square (i.e. $\Omega=[0,1]\times[0,1]$),
    let $\mathbb{P}(A)=\mbox{area}(A)$. Then
    $$
        X:\Omega\to\mathbb{R}:(x,y)\to\sqrt{x^2+y^2}
    $$
    is a random variable.
\end{example}

\begin{definition} If $X:\Omega\to\mathbb{R}$ is a random variable and $A$ is
    some subset of real numbers then 
    $$
        \{X\in A\}=\{\omega\in\Omega:X(\omega)\in A\}
    $$
    In particular, for a real number $a\in\mathbb{R}$ we have
    $$
        \{X=a\}=\{\omega\in\Omega:X(\omega)=a\}
    $$
\end{definition}

\begin{definition} Two random variables $X:\Omega\to\mathbb{R}$ and
    $Y:\Omega\to\mathbb{R}$ are called independent if events
    $$
        \{X\in A\},\quad \{Y\in B\}
    $$
    are independent for any $A,B\in\mathcal{F}$.
\end{definition}

This definition of independence is consistent with usual understanding of
independent random quantities. Checking by definition that two random variables
are independent is tedios. Usually it is clear from the context of the problem
being solved that two random variables are independent.

\begin{definition} Random variables $X_1,\ldots,X_n$ are called independent if
    events
    $$
        \{X_{i_1}\in A_1\},\ldots,\{X_{i_m}\in  A_m\}
    $$
    are independent for any $\{i_1,\ldots,i_m\}\subset \{1,\ldots,n\}$ and any
    $A_1,\ldots,A_m\in\mathcal{F}$.
\end{definition}

%%%%%%%%%%%%%%%%%%%%%%%%%%%%%%%%%%%%%%%%%%%%%%%%%%%%%%%%%%%%%%%%%%%%%%%%%%%%%%%%

\subsection{Distributions of random variables}

\begin{definition} Cumulative density function is a function defined by
    $$
        F_X:\mathbb{R}\to [0,1]: t\mapsto \mathbb{P}(X\leq t)
    $$
\end{definition}

\begin{example} Tossing a coin twice. Again
    $\Omega=\{(i,j), i,j\in \{1,\ldots,6\} \}$,
    $\mathbb{P}((i,j))=\frac{1}{36}$ for all
    $(i,j)\in \Omega$. Again consider random variable
    $$
        X:\Omega\to\mathbb{R}:(i,j)\mapsto i+j
    $$
    Now we shall compute $F_X$ for all $t\in\mathbb{R}$.
    $$
        F_X(t)=
        \begin{cases}
            0                                       &
            \mbox { if }t < 2                         \\
            \mathbb{P}(X=2)                         &
            \mbox { if } 2 \leq t < 3                 \\
            \mathbb{P}(X=2)+\mathbb{P}(X=3)         &
            \mbox { if } 3 \leq t < 4                 \\
            \mathbb{P}(X=2)+\ldots+\mathbb{P}(X=4)  & % chktex 11
            \mbox { if } 4 \leq t < 5                 \\
            \mathbb{P}(X=2)+\ldots+\mathbb{P}(X=5)  & % chktex 11
            \mbox { if } 5 \leq t < 6                 \\
            \mathbb{P}(X=2)+\ldots+\mathbb{P}(X=6)  & % chktex 11
            \mbox { if } 6 \leq t < 7                 \\
            \mathbb{P}(X=2)+\ldots+\mathbb{P}(X=7)  & % chktex 11
            \mbox { if } 7 \leq t < 8                 \\
            \mathbb{P}(X=2)+\ldots+\mathbb{P}(X=8)  & % chktex 11
            \mbox { if } 8 \leq t < 9                 \\
            \mathbb{P}(X=2)+\ldots+\mathbb{P}(X=9)  & % chktex 11
            \mbox { if } 9 \leq t < 10                \\
            \mathbb{P}(X=2)+\ldots+\mathbb{P}(X=10) & % chktex 11
            \mbox { if } 10 \leq t < 11               \\
            \mathbb{P}(X=2)+\ldots+\mathbb{P}(X=11) & % chktex 11
            \mbox { if } 11 \leq t < 12               \\
            1                                       &
            \mbox { if }12 \leq t
        \end{cases}
        =
        \begin{cases}
            0             & \mbox { if }t < 2           \\
            \frac{1}{36}  & \mbox { if } 2 \leq t < 3   \\
            \frac{3}{36}  & \mbox { if } 3 \leq t < 4   \\
            \frac{6}{36}  & \mbox { if } 4 \leq t < 5   \\
            \frac{10}{36} & \mbox { if } 5 \leq t < 6   \\
            \frac{15}{36} & \mbox { if } 6 \leq t < 7   \\
            \frac{21}{36} & \mbox { if } 7 \leq t < 8   \\
            \frac{26}{36} & \mbox { if } 8 \leq t < 9   \\
            \frac{30}{36} & \mbox { if } 9 \leq t < 10  \\
            \frac{33}{36} & \mbox { if } 10 \leq t < 11 \\
            \frac{35}{36} & \mbox { if } 11 \leq t < 12 \\
            1             & \mbox { if }12 \leq t
        \end{cases}
    $$
\end{example}

\begin{example} Consider an experimant where we drop a dot on a line. 
    Let $\Omega=[0,1]$,
    $\mathbb{P}(A)=\mbox{length}(A)$ and
    $$
        X:\Omega\to\mathbb{R}:\omega\mapsto \omega
    $$
    We shall compute $F_X$ for all $t\in\mathbb{R}$
    $$
        F_X(t)
        =\mathbb{P}(X\leq t)
        =\mathbb{P}(\{\omega\in\Omega:X(\omega)\leq t\})
        =\mathbb{P}(\{\omega\in[0,1]:\omega\leq t\})
    $$
    $$
        =\begin{cases}
            \mbox{length}(\varnothing) & t < 0          \\
            \mbox{length}([0,t])       & 0\leq t \leq 1 \\
            \mbox{length}([0,1])       & 1 < t          \\
        \end{cases}
        =\begin{cases}
            0 & t < 0          \\
            t & 0\leq t \leq 1 \\
            1 & 1 < t          \\
        \end{cases}
    $$
\end{example}

From these examples we see that $F_X$ is always a non-decreasing function. 
If $X$ attains only finitely many values, then $F_X$ has discontinuities, 
and $X$ is called a discrete random variable. If $F_X$ is continuous, 
then $X$ is called a continuous random variable.

\begin{definition} Probability density function is a function defined by
    \begin{itemize}
        \item $f_X:\mathbb{R}\to [0,1]: t\mapsto \mathbb{P}(X=t)$ if $X$ is
              discrete;
        \item $f_X:\mathbb{R}\to\mathbb{R}: t\mapsto  F_X'(t) $ if $X$ is
              continuous.
    \end{itemize}
\end{definition}

\begin{theorem} If $X$ is a continuous random variable, then
    $$
        F_X(t)=\int_{-\infty}^t f_X(s)ds
    $$
\end{theorem}

\begin{example} Consider an experimant where we toss a dice twice. Define
    $\Omega=\{(i,j), i,j\in \{1,\ldots,6\} \}$,
    $\mathbb{P}((i,j))=\frac{1}{36}$ for all
    $(i,j)\in \Omega$. Consider a random variable
    $$
        X:\Omega\to\mathbb{R}:(i,j)\mapsto i+j
    $$
    Now we shall compute $f_X$ for reasonable values of $X$. That is for
    $t\in \{2,3,\ldots,12\}$
    $$
        f_X(2)=\mathbb{P}(X=2)=\mathbb{P}(\{(1,1)\})=\frac{1}{36}
    $$
    $$
        f_X(3)=\mathbb{P}(X=3)=\mathbb{P}(\{(1,2),(2,1)\})=\frac{2}{36}
    $$
    $$
        f_X(4)=\mathbb{P}(X=4)=\mathbb{P}(\{(1,3),(3,1),(2,2)\})=\frac{3}{36}
    $$
    $$
        \ldots
    $$
    After careful gazing we get
    $$
        f_X(k)=
        \begin{cases}
            \frac{6-|k-7|}{36} & \mbox{ if } k\in \{2,\ldots,12\} \\
            0                  & \mbox{ otherwise }
        \end{cases}
    $$
\end{example}

\begin{example} Consider an experimant where we drop a dot on a line. 
    Let $\Omega=[0,1]$,
    $\mathbb{P}(A)=\mbox{length}(A)$ and
    $$
        X:\Omega\to\mathbb{R}:\omega\mapsto \omega
    $$
    We shall compute $f_X$ for all $t\in\mathbb{R}$
    $$
        f_X(t)=F_X'(t)
        =\begin{cases}
            0' & t < 0          \\
            t' & 0\leq t \leq 1 \\
            1' & 1 < t          \\
        \end{cases}
        =\begin{cases}
            0 & t < 0          \\
            1 & 0\leq t \leq 1 \\
            0 & 1 < t          \\
        \end{cases}
    $$
\end{example}

\begin{example} \textbf{(Uniform distribution)} We say that a random variable is
    uniformly distributed on a segment $[a, b]$ if it has probability density
    function of the form
    $$
        f_X(t)
        =\begin{cases}
            \frac{1}{b-a} & \mbox{ if } t\in[a,b]    \\
            0             & \mbox{ if } t\notin[a,b]
        \end{cases}
    $$
    We express this fact as $X\sim Unif(a, b)$
\end{example}

The following example is central to the whole theory

\begin{definition} \textbf{(Normal distribution)} A random variable $X$ is
    called normal if it has probability density function of the form
    $$
        f_X(t)
        =\frac{1}{\sigma\sqrt{2\pi}} e^{-\frac{{(t-\mu)}^2}{2\sigma^2}}
    $$
    In this case we shall write $X\sim Norm(\mu,\sigma^2)$.
\end{definition}

\begin{remark} Consider function
    $$
        \Phi(t)=\int_{-\infty}^t\frac{1}{\sqrt{2\pi}}e^{-\frac{t^2}{2}}dt
    $$
    then for any $X$ with normal distribution with parameters $\mu$, $\sigma$ we
    have
    $$
        F_X(t)=\Phi\left(\frac{t-\mu}{\sigma}\right)
    $$
    In other words $\Phi$ is the cumulative density function of the normal
    random variable with parameters $\mu=0$, $\sigma^2=1$. Normal random
    variables with such parameters are called standard normal random variables.
\end{remark}

Later we shall discuss the meaning of parameters $\mu$ and $\sigma$.

\begin{example} \textbf{(Binomial distribution)} Let $X$ be a binomial random
    variable with parameters $n$ and $p$. In other words $X$ is a number of
    tails after $n$ tosses of coin such that at each toss tails have probability
    $p$. Clearly $X$ attains values $\{0,\ldots,n\}$. We shall compute $f_X(k)$
    for $k\in \{0,\ldots,n\}$. By definition $f_X(k)=\mathbb{P}(X=k)$. An event
    $\{X=k\}$ consist of some number elemtary events
    $\{\omega_1,\ldots\omega_N\}$. The number $N$ of these elementary events
    equals the number of ways to pick $k$ tosses out of $n$ tosses that will end
    up with tails. This is a standard fact from combinatorics, that the latter
    number is $\binom{n}{k}$. Each elementary event $\omega_i$ corresponds to
    the series of tosses with exactly $k$ tails and $n-k$ heads, so
    $\mathbb{P}(\{\omega_i\})=p^k{(1-p)}^{n-k}$. Thus
    \begin{align*}
        f_X(k)
         & =\mathbb{P}(X=k)                          \\
         & =\mathbb{P}(\{\omega_1,\ldots,\omega_N\}) \\
         & =\mathbb{P}(\{\omega_1\})+
        \ldots+\mathbb{P}(\{\omega_N\})              \\ % chktex 11
         & =p^k{(1-p)}^{n-k}+
        \ldots+p^k{(1-p)}^{n-k}                      \\
         & =\binom{n}{k}p^k{(1-p)}^{n-k}
    \end{align*}
\end{example}

\begin{definition} \textbf{(Poisson distribution)} Let $X$ be a discrete random
    variable with probability density function defined by
    $$
        f_X(k)=\frac{\lambda^k}{k!}e^{-\lambda}
    $$
    then we say that $X$ is Poisson random variable with parameter $\lambda$. We
    denote this fact as $X\sim Pois(\lambda)$.
\end{definition}

%%%%%%%%%%%%%%%%%%%%%%%%%%%%%%%%%%%%%%%%%%%%%%%%%%%%%%%%%%%%%%%%%%%%%%%%%%%%%%%%

\subsection{Quantiles of random variables}

\begin{definition} An $\alpha$-quantile of a random variable $X$ is the smallest
    number $q$ such that
    $$
        \mathbb{P}(X\leq q)\geq\alpha
    $$
    In other words this is the smallest number $q$ such that $F_X(q)\geq
        \alpha$. Notation: $q_\alpha(X)$.
\end{definition}

Put differently $q_\alpha(X)$ is the smallest value $q$ such that at least
$100\alpha$ percent of the values of $X$ are less than $q$.

\begin{example} Tossing a coin twice. Again $\Omega=\{(i,j),
        i,j\in \{1,\ldots,6\} \}$, $\mathbb{P}(\{(i,j) \})=\frac{1}{36}$ for all
    $(i,j)\in \Omega$. Again consider a random variable
    $$
        X:\Omega\to\mathbb{R}:(i,j)\mapsto i+j
    $$
    Then $q_{0.25}(X)=4$ because 25 percent of the values are less than 4 and 4
    is the smallest possible constant here.
\end{example}

Notation:
\begin{itemize}
    \item $q_{0.25}(X)$ --- first quartile of $X$
    \item $q_{0.50}(X)$ --- second quartile of $X$ or median of $X$
    \item $q_{0.75}(X)$ --- third quartile of $X$
\end{itemize}

Question: Find the median value of the total score after tossing two cubes. Find
the 0.99 quantile.

\begin{remark} If $X$ is a continuous random variable, then $q_\alpha(X)$ is a
    (necessarily unique) root of the equation $F_X(q)=\alpha$.
\end{remark}

\begin{example} Dropping a dot on a line. Let $\Omega=[0,1]$,
    $\mathbb{P}(A)=\mbox{length}(A)$ and
    $$
        X:\Omega\to\mathbb{R}:\omega\mapsto \omega
    $$
    As we already know
    $$
        F_X(t)
        =\begin{cases}
            0 & t < 0          \\
            t & 0\leq t \leq 1 \\
            1 & 1 < t          \\
        \end{cases}
    $$
    Since $X$ is a continuous random variable $q_{0.25}(X)$, is a root of the
    equation $F_X(q)=0.25$. Clearly, q=0.25, so $q_{0.25}(X)=0.25$.
\end{example}

%%%%%%%%%%%%%%%%%%%%%%%%%%%%%%%%%%%%%%%%%%%%%%%%%%%%%%%%%%%%%%%%%%%%%%%%%%%%%%%%

\subsection{Expected value of a random variable}

\begin{definition} Expected value of a random variable $X$ is a number
    \begin{itemize}
        \item $\mathbb{E}[X]=\sum_{k\in \mathcal{X}} k \mathbb{P}(X=k)$ if $X$
              is discrete ($\mathcal{X}$ is the set of values attained by $X$);
        \item $\mathbb{E}[X]=\int_{-\infty}^{+\infty}t f_X(t)dt$ if $X$ is
              continuous;
    \end{itemize}
    Another notation for expected value is $m_X$.
\end{definition}

Simply speaking expected value of a random variable is an average value of that
variable.

\begin{example} Tossing a coin twice. Again $\Omega=\{(i,j),
        i,j\in \{1,\ldots,6\} \}$, $\mathbb{P}((i,j))=\frac{1}{36}$ for all
    $(i,j)\in \Omega$. Again consider random variable
    $$
        X:\Omega\to\mathbb{R}:(i,j)\mapsto i+j
    $$
    Then $\mathcal{X}=\{2,3,\ldots,12\}$. Therefore
    $$
        \mathbb{E}[X]
        =2\cdot \mathbb{P}(X=2)+3\mathbb{P}(X=3)+
        \ldots+12\mathbb{P}(X=12)
        =7
    $$
\end{example}

\begin{example} Let $X$ be a Bernoulli random variable with parameter $p$. Then
    $\mathcal{X}=\{0,1\}$, so
    $$
        \mathbb{E}[X]
        =0\cdot \mathbb{P}(X=0) + 1\cdot \mathbb{P}(X=1)
        =0 \cdot (1-p) + 1 \cdot p = p
    $$
\end{example}

\begin{example} Let $X$ be a normal random variable with parameters $\mu$ and
    $\sigma$. Then
    $$
        \mathbb{E}[X]
        =\int_{-\infty}^{+\infty}t f_X(t)dt
        =\int_{-\infty}^{+\infty}
        \frac{1}{\sigma\sqrt{2\pi }}
        e^{-\frac{{(t-\mu)}^2}{2\sigma^2}} dt
        =\ldots
        =\mu
    $$
    Thus parameter $\mu$ in the normal distribution is the average value of the
    random variable.
\end{example}

\begin{theorem} Suppose we are given two random variables $X$ and $Y$, then
    $$
        \mathbb{E}[X+Y]=\mathbb{E}[X]+\mathbb{E}[Y]
    $$
    Even more if $a$ and $b$ --- real numbers, then
    $$
        \mathbb{E}[aX+bY]=a\mathbb{E}[X]+b\mathbb{E}[y]
    $$
\end{theorem}

\begin{theorem} If $X$ and $Y$ are two \underline{independent} random variables
    then
    $$
        \mathbb{E}[XY]=\mathbb{E}[X]\cdot\mathbb{E}[Y]
    $$
\end{theorem}

\begin{example} Let $X$ be a binomial random variable with parameters $n$ and
    $p$. Then $X$ can be represented as sum of $n$ Bernoulli random variables
    $X_1,\ldots,X_n$, i.e. $X=X_1+\ldots+X_n$.     % chktex 11
    From previous example we know
    that $\mathbb{E}[X_1]=\ldots=\mathbb{E}[X_n]=p$. Therefore    % chktex 11
    $$
        \mathbb{E}[X]
        =\mathbb{E}[X_1+\ldots+X_n]     % chktex 11
        =\mathbb{E}[X_1]+\ldots+\mathbb{E}[X_n]    % chktex 11
        =p+\ldots+p=np    % chktex 11
    $$
\end{example}

\begin{theorem} \textbf{(Law of the unconscious statistician, a.k.a. LOTUS)} Let
    $X$ be a random variable and $g:\mathbb{R}\to\mathbb{R}$ be any funciton,
    then
    \begin{itemize}
        \item $\mathbb{E}[g(X)]=\sum_{k\in\mathcal{X}}g(k)\mathbb{P}(X=k)$ if
              $X$ is discrete ($\mathcal{X}$ --- values attained by $X$);
        \item $\mathbb{E}[g(X)]=\int_{-\infty}^{+\infty} g(t)f_X(t)dt$ if $X$ is
              continuous.
    \end{itemize}
\end{theorem}

\begin{example} Let $X$ be a bernoulli random variable with parameter $p$. Then
    $\mathcal{X}={0,1}$ and by the LOTUS
    $$
        \mathbb{E}[X^2]=
        0^2\cdot\mathbb{P}(X=0)+1^2\cdot\mathbb{P}(X=1)
        =0^2\cdot (1-p)+1^2\cdot p=p
    $$
\end{example}

%%%%%%%%%%%%%%%%%%%%%%%%%%%%%%%%%%%%%%%%%%%%%%%%%%%%%%%%%%%%%%%%%%%%%%%%%%%%%%%%

\subsection{Variance and standard deviation of a random variable}

\begin{definition} Variance of a random variable $X$ is the expected value of
    the random variable ${(X-\mathbb{E}[X])}^2$. In other words
    $$
        \mathbb{V}[X]=\mathbb{E}[{(X-\mathbb{E}[X])}^2]
    $$
\end{definition}

\begin{remark} One can show that
    $$
        \mathbb{V}[X]=\mathbb{E}[X^2]-{\mathbb{E}[X]}^2
    $$
    If $X$ is a continuous random variable, then
    $$
        \mathbb{V}[X]
        =\int_{-\infty}^{+\infty} {(t-\mathbb{E}[X])}^2 f_X(t)dt
        =\int_{-\infty}^{+\infty} t^2 f_X(t)dt-
        {\left( \int_{-\infty}^{+\infty} t f_X(t)dt\right)}^2
    $$
\end{remark}

Variance shows how volatile values of $X$ are. Variance shows how much they are
different from the expecqted value. In practice it is more convinient to work
with another quantity called the standard deviation. This characteristic has the
advantage that it is measured in the same units as the expected value.

\begin{definition} The standard deviation of a random variable $X$ is defined as
    $$
        s_X=\sqrt{\mathbb{V}[X]}
    $$
\end{definition}

\begin{example} Let $X$ be a Bernoulli random variable with parameter $p$. As we
    showed earlier $\mathbb{E}[X]=p$ and $\mathbb{E}[X^2]=p$, so
    $$
        \mathbb{V}[X]=\mathbb{E}[X^2]-{\mathbb{E}[X]}^2=p-p^2=p(1-p)
    $$
    $$
        s_X=\sqrt{p(1-p)}
    $$
\end{example}

\begin{example} Let $X$ be a normal random variable with parameters $\mu$ and
    $\sigma$. As we already know $\mathbb{E}[X]=\mu$, so
    $$
        \mathbb{V}[X]
        =\int_{-\infty}^{+\infty}{(t-\mathbb{E}[X])}^2 f_X(t)dt
        =\int_{-\infty}^{+\infty}{(t-\mu)}^2
        \frac{1}{\sqrt{2\pi\sigma}}e^{-\frac{{(t-\mu)}^2}{2\sigma^2}}dt
        =\ldots
        =\sigma^2
    $$
    $$
        s_X=\sqrt{\mathbb{V}[X]}=\sigma
    $$
    Thus the parameter $\sigma$ in the normal distribution is the standard 
    deviation of the random variable.
\end{example}

\begin{theorem} Suppose we are given two \underline{independent} random
    variables $X$ and $Y$, then
    $$
        \mathbb{V}[X+Y]=\mathbb{V}[X]+\mathbb{V}[Y]
    $$
    Even more if $a$ and $b$ are real numbers, then
    $$
        \mathbb{V}[aX+bY]=a^2\mathbb{V}[X]+b^2\mathbb{V}[Y]
    $$
\end{theorem}


\begin{example} Let $X$ be a binomial random variable, then $X$ can be
    represented as a sum of $n$ Bernoulli variables
    $X=X_1+\ldots+X_n$,  % chktex 11
    where $X_1,\ldots,X_n\sim Ber(p)$. As we showed earlier
    $\mathbb{V}[X_1]=\ldots=\mathbb{V}[X_n]=p(1-p)$,   % chktex 11
    so from independece of
    $X_1,\ldots, X_n$ and the previous remark we get
    $$
        \mathbb{V}[X]
        =\mathbb{V}[X_1+\ldots+X_n]    % chktex 11
        =\mathbb{V}[X_1]+\ldots+\mathbb{V}[X_n]    % chktex 11
        =p(1-p)+\ldots+p(1-p)=np(1-p)    % chktex 11
    $$
    $$
        s_X=\sqrt{\mathbb{V}[X]}=\sqrt{np(1-p)}
    $$
\end{example}

\subsection{De Moivre–Laplace theorem}

The following thoerem states informally that for big values of $n$ 
binomial random variables behave like normal varaibles

\begin{theorem} Let $X$ be a binomial random variable with parameters $n$ and
    $p$ where $0<p<1$. Then $f_X(k)\approx f_Y(k)$ for the normal random
    variable $Y$ with parameters $\mu=np$ and $\sigma^2=np(1-p)$. More
    explicitly
    $$
        f_X(k)
        =\binom{n}{k}p^k {(1-p)}^{n-k}
        \underset{n\to\infty}{\to}
        \frac{1}{\sqrt{2\pi \cdot np(1-p)}}e^{-\frac{{(k-np)}^2}{2np(1-p)}}
    $$
\end{theorem}

This theorem has another form (the so called integral form).

\begin{theorem} Let $X$ be a binomial random variable with parameters $n$ and
    $p$ where $0<p<1$. Then $F_X(t)\approx F_Y(t)$ for the normal random
    variable $Y$ with parameters $\mu=np$ and $\sigma=np(1-p)$. More explicitly
    $$
        F_X(t)
        =\mathbb{P}(X\leq t)
        \underset{n\to\infty}{\to}
        \int_{-\infty}^k
        \frac{1}{\sqrt{2\pi \cdot np(1-p)}}
        e^{-\frac{{(s-np)}^2}{2np(1-p)}}ds
        =\Phi\left(\frac{k-np}{\sqrt{np(1-p)}}\right)
    $$
\end{theorem}

These theorems assume that $n$ has to be big enough. How big should $n$ be in
practice?

\begin{remark} Let $X$ be a binomial random variable with parameters $n$ and $p$
    where $pn>10$ and $n(1-p)>10$.  Then
    $$
        F_X(k)\approx\Phi\left(\frac{k+\frac{1}{2}-np}{\sqrt{np(1-p)}}\right)
    $$
\end{remark}

\subsection{Law of large numbers}

The following theorem states informally that average of a big number of equally
distributed random variables approaches their expected value. This is one of the
foundational theorems of probability theory which gives a firm basis for
applications in real world problems.

\begin{definition} We say that a sequence of random variables $X_1,\ldots,X_n$
    is i.i.d.\ if its random variables are independent and identically
    distributed. In other words they are independent and have the same
    cumulative density function.
\end{definition}

Clearly, if $X_1,\ldots,X_n$ are i.i.d., then these random variables have equal
characterstics like quantiles, mean, standard deviation and many others.

\begin{theorem} \textbf{(Weak law of large numbers)}. Let $X_1,\ldots,X_n$ be
    i.i.d.\ with expected value $\mu$, then for all $\epsilon>0$ we have
    $$
        \mathbb{P}\left(\left|\frac{X_1+\ldots+X_n}{n}    % chktex 11
        -\mu\right|<\epsilon\right)\underset{n\to\infty}{\to}1
    $$
\end{theorem}

\begin{theorem} \textbf{(Strong law of large numbers)}.
    Let $X_1,\ldots,X_n$ be    % chktex 11
    i.i.d.\ with expected value $\mu$, then
    $$
        \mathbb{P}\left(\frac{X_1+\ldots+X_n}{n}    % chktex 11
        =\mu\right)\underset{n\to\infty}{\to}1
    $$
\end{theorem}

\subsection{Central limit theorem}

Cental limit theorem is no doubt the most important theorem of probability
theory. Most non-trivial results are based on this fact. It essentially says
that the average of i.i.d.\ random variables behave like a normal random
variable. Compare this with the law of large numbers.

Before stating the theorem we shall give a short remark on characteristics of
the average of i.i.d.\ random variables.

\begin{remark} Let $X_1,\ldots,X_n$ be i.i.d.\ random variables. Let
    $Y=\frac{1}{n}(X_1+\ldots+X_n)$  % chktex 11
    be their average. Since $X_1,\ldots,X_n$
    identically distributed, then they have the same expected value and
    varianece
    $$
        \mathbb{E}[X_1]
        =\ldots=\mathbb{E}[X_n]             % chktex 11
        =\mu, \quad \mathbb{V}[X_1]
        =\ldots=\mathbb{V}[X_n]=\sigma^2    % chktex 11
    $$
    Then
    $$
        \mathbb{E}[Y]
        =\mathbb{E}\left[\frac{1}{n}(X_1+\ldots+X_n)\right]     % chktex 11
        =\frac{1}{n}\mathbb{E}[X_1+\ldots+X_n]                  % chktex 11
        =\frac{1}{n}(\mathbb{E}[X_1]+\ldots+\mathbb{E}[X_n])    % chktex 11
        =\frac{1}{n}\cdot n\mu
        =\mu
    $$
    $$
        \mathbb{V}[Y]
        =\mathbb{V}\left[\frac{1}{n}(X_1+\ldots+X_n)\right]       % chktex 11
        ={\left(\frac{1}{n}\right)}^2\mathbb{V}[X_1+\ldots+X_n]     % chktex 11
        =\frac{1}{n^2}(\mathbb{V}[X_1]+\ldots+\mathbb{V}[X_n])    % chktex 11
        =\frac{1}{n^2}\cdot n\sigma^2
        =\frac{\sigma^2}{n}
    $$
    Therefore
    $$
        m_Y=\mu, \quad s_Y=\frac{\sigma}{\sqrt{n}}
    $$
\end{remark}

\begin{theorem} Let $X_1,\ldots,X_n$ be i.i.d.\ with expected value $\mu$ and
    variance $\sigma^2$. Let $Y=\frac{1}{n}(X_1+\ldots+X_n)$     % chktex 11
    be their average. Then the random variable
    $Z_Y=\sqrt{n}\frac{Y-\mu}{\sigma}$ has cumulative
    density function approximately equal to the cumulative density funciton of
    the standard normal distribution:
    $$
        F_{Z_Y}(t)\underset{n\to\infty}{\to}\Phi(t)
    $$
    More explicitly
    $$
        \mathbb{P}\left(
        \frac{
            \frac{1}{n}(X_1+\ldots+X_n)-\mu    % chktex 11
        }{\frac{1}{\sqrt{n}}\sigma}
        <t\right)
        \underset{n\to\infty}{\to}
        \frac{1}{\sqrt{2\pi}}\int_{-\infty}^{t} e^{-\frac{s^2}{2}}ds
    $$
\end{theorem}

\begin{remark} One can note that De Moivre-Laplace theorem is nothing more than
    a central limit theorem applied to i.i.d. Bernoulli random variables.
\end{remark}

\section{Statistics}

In probability theory we study characteristics and behaviour of random variables
assuming that we know the probability space or at least the cumulative density
functions. In real world it is not possible to get the exact description of a
probability space for our problem or a precise formula for the densities of
random variables.

The best we can do is to make a good guess about random variable distributions
based on some numbers of observations. For example, tossing a coin 10000 times
and observing heads in 5053 time we can be quite confident that this coin is
described by a Bernoulli random variable with parameter $p=1/2$. To be
absolutely sure about distribution of the coin we had to make infinitely many
tossing, which is impossible in practice. Therefore we need to study random
variables given only finitely many observations.

The goal of statistics is to give us tools to
\begin{enumerate}
    \item approximately recover distributions of random variables given finite
          number of observations;
    \item approximately compute random variables characteristics given finite
          number of observations;
    \item for a given level of confidence answer questions regarding random
          variable behaviour;
    \item predict values of dependent random variables given finite number of
          observations of independent random variables.
\end{enumerate}
These four big problems have their names: distribution estimation, point
estimation, hypothesis testing and regression respectively.

\subsection{Samples, observations, sample statistics}

\begin{definition} A sample $\mathscr{X}$ of size $n$ is a sequence of random
    variables $X_1,\ldots,X_n$ defined on some probability 
    space $(\Omega,\mathcal{F},\mathbb{P})$.
\end{definition}

\begin{definition} Let $X$ be a random variable. A sample $\mathscr{X}$ from $X$
    of size $n$ is a sequence $X_1,\ldots,X_n$ of i.i.d.\ random variables with
    the same distribution as $X$. We shall denote this fact as $\mathscr{X}\sim
        X$.
\end{definition}

\begin{example} Assume we have a fair coin and we toss it $n$ times. Outcomes of
    the tossed coin are described by a random variable $X$. Let $X_i$ denote the
    random variable describing coin side on the $i$-th toss. Then
    $\mathscr{X}=(X_1,\ldots,X_n)$ is a sample of the size $n$ of the random
    variable $X$, where $X$ is a random variable representing outcome of the
    flipped coin.
\end{example}

\begin{definition} Let $\mathscr{X}$ be a sample of the random variable
    $X:\Omega\to\mathbb{R}$ defined on a probability 
    space $(\Omega,\mathcal{F},\mathbb{P})$. For any
    fixed elementary event $\omega\in\Omega$ the sequence of numbers
    $x=(X_1(\omega),\ldots,X_n(\omega))$ is called on observation of $X$ of size
    $n$.
\end{definition}

\begin{example} Assume we have a fair coin and we toss it 5 times. Outcomes of
    the tossed coin are described by a random variable $X$. Suppose we got the
    following outcomes $x=(H,H,T,T,H)$. Then $x$ is called the observations of
    the random variable $X$.
\end{example}

\begin{definition} Let $\mathscr{X}$ be a sample of size $n$. A statistic $T$ is
    a random variable which is a function of sample $\mathscr{X}$.
\end{definition}

\begin{example} Let $\mathscr{X}$ be a sample of size $n$ of the random variable
    $X$. Then we define the following statistics
    \begin{itemize}
        \item sample mean $$m(\mathscr{X})=\frac{1}{n}\sum_{i=1}^n X_i$$
        \item sample $k$-th moment $$m_{k}(\mathscr{X})=\frac{1}{n}\sum_{i=1}^n
                  X_i^k$$
        \item sample variance
              $$
                  s_b^2(\mathscr{X})
                  =\frac{1}{n}\sum_{i=1}^n{(X_i-m(\mathscr{X}))}^2
              $$
        \item unbiased sample variance
              $$
                  s^2(\mathscr{X})
                  =\frac{1}{n-1}\sum_{i=1}^n{(X_i-m(\mathscr{X}))}^2
              $$
    \end{itemize}
\end{example}

\begin{remark} Clearly,
    $$
        m(\mathscr{X})=m_{1}(\mathscr{X})
        \quad\quad
        s^2(\mathscr{X})=\frac{n}{n-1}s_b^2(\mathscr{X})
    $$
    One can show that
    $$
        s_b^2(\mathscr{X})
        =\frac{1}{n}
        \sum_{i=1}^n X_i^2-{\left(\frac{1}{n}\sum_{i=1}^n X_i\right)}^2
        =m_2(\mathscr{X})-{m(\mathscr{X})}^2
    $$
\end{remark}

\begin{remark} Let $\mathscr{X}$ be a sample from the random variable $X$, then
    $$
        \mathbb{E}[m_{k}(\mathscr{X})]=\mathbb{E}[X^k]
    $$
    Indeed,
    $$
        \mathbb{E}[m_{k}(\mathscr{X})]
        =\mathbb{E}\left[\frac{1}{n}\sum_{i=1}^n X_i^k\right]
        =\frac{1}{n}\mathbb{E}\left[\sum_{i=1}^n X_i^k\right]
        =\frac{1}{n}\sum_{i=1}^n \mathbb{E}[X_i^k]
        =\frac{1}{n}\sum_{i=1}^n \mathbb{E}[X^k]
        =\frac{1}{n}\cdot n\mathbb{E}[X^k]
        =\mathbb{E}[X^k]
    $$
\end{remark}

\begin{remark}\label{ExpecStats} Let $\mathscr{X}$ be a sample from the random
    variable $X$, then
    $$
        \mathbb{E}[m(\mathscr{X})]=m_X,
        \quad\quad
        \mathbb{E}[s_b^2(\mathscr{X})]=\frac{n-1}{n}s_X^2,
        \quad\quad
        \mathbb{E}[s^2(\mathscr{X})]=s_X^2
    $$
    Indeed,
    $$
        \mathbb{E}[m(\mathscr{X})]
        =\mathbb{E}[m_{1}(\mathscr{X})]
        =\mathbb{E}[X^1]=m_X
    $$
    Now note that
    \begin{align*}
        \mathbb{E}\left[{\left(\frac{1}{n}\sum_{i=1}^n X_i\right)}^2\right]
         & =\frac{1}{n^2}
        \mathbb{E}\left[{\left(\sum_{i=1}^n X_i\right)}^2\right]        \\
         & =\frac{1}{n^2}
        \mathbb{E}\left[\sum_{i=1}^n\sum_{j=1}^n X_i X_j\right]         \\
         & =\frac{1}{n^2}
        \mathbb{E}\left[\sum_{i=1}^n X_i^2+\sum_{i\neq j}X_i X_j\right] \\
         & =\frac{1}{n^2}
        \left(\sum_{i=1}^n\mathbb{E}[X_i^2]
        +\sum_{i\neq j}\mathbb{E}[X_i X_j]\right)                       \\
         & =\frac{1}{n^2}
        \left(\sum_{i=1}^n\mathbb{E}[X^2]
        +\sum_{i\neq j}\mathbb{E}[X_i]\mathbb{E}[X_j]\right)            \\
         & =\frac{1}{n^2}
        \left(n\mathbb{E}[X^2]
        +\sum_{i\neq j}\mathbb{E}[X]\mathbb{E}[X]\right)                \\
         & =\frac{1}{n^2}
        \left(n\mathbb{E}[X^2]+(n^2-n)\mathbb{E}[X]\mathbb{E}[X]\right) \\
         & =\frac{1}{n}
        \left(\mathbb{E}[X^2]+(n-1){\mathbb{E}[X]}^2\right)             \\
    \end{align*}
    So
    \begin{align*}
        \mathbb{E}[s_b^2(\mathscr{X})]
         & =\mathbb{E}\left[\frac{1}{n}\sum_{i=1}^n X_i^2
        -{\left(\frac{1}{n}\sum_{i=1}^n X_i\right)}^2\right]                 \\
         & =\mathbb{E}\left[\frac{1}{n}\sum_{i=1}^n X_i^2\right]
        -\mathbb{E}\left[{\left(\frac{1}{n}\sum_{i=1}^n X_i\right)}^2\right] \\
         & =\mathbb{E}[m_{2}(\mathscr{X})]
        -\frac{1}{n}\left(\mathbb{E}[X^2]+(n-1){\mathbb{E}[X]}^2\right)      \\
         & =\mathbb{E}[X^2]
        -\frac{1}{n}\left(\mathbb{E}[X^2]+(n-1){\mathbb{E}[X]}^2\right)      \\
         & =\frac{n-1}{n}(\mathbb{E}[X^2]
        -{\mathbb{E}[X]}^2)                                                  \\
         & =\frac{n-1}{n}\mathbb{V}[X]                                       \\
         & =\frac{n-1}{n}s_X^2
    \end{align*}
    and
    $$
        \mathbb{E}[s^2(\mathscr{X})]
        =\mathbb{E}[\frac{n}{n-1}s_b^2(\mathscr{X})]
        =\frac{n}{n-1}\mathbb{E}[s_b^2(\mathscr{X})]
        =\frac{n}{n-1}\frac{n-1}{n}s_X^2
        =s_X^2
    $$
\end{remark}

\subsection{Distribution estimates}

As we have seen earlier the most single important characteristic of a random
variable is its distribution function. Given a set of observations of a random
variable one can construct an approximation of this distribution function. The
construction is pretty straightforward.

\begin{definition} Let $\mathscr{X}$ be a sample from a random variable $X$.
    Then the parametric random variable
    $$
        F_{\mathscr{X}}(t)=\frac{1}{n}\sum_{i=1}^n 1_{\{X_i<t\}}
    $$
    is called the empirical cumulative distribution function.
\end{definition}

For every $t$ this statistic gives a good approximation for $F_X(t)$.

\begin{theorem} Let $\mathscr{X}$ be a sample from a random variable $X$. Then
    for all $t\in\mathbb{R}$ and $\epsilon>0$
    $$
        \mathbb{P}(|F_{\mathscr{X}}(t)-F_X(t)|>\epsilon)
        \underset{n\to\infty}{\to}0
    $$
\end{theorem}

Even stronger result is true

\begin{theorem} Let $\mathscr{X}$ be a sample from a random variable $X$. Then
    for all $\epsilon>0$
    $$
        \mathbb{P}\left(\sup_{t\in\mathbb{R}}|
        F_{\mathscr{X}}(t)-F_X(t)|>\epsilon\right)\underset{n\to\infty}{\to}0
    $$
\end{theorem}

\begin{example} Let $x=(1,2,4,2,4,1,3,1)$ be an observation of a random variable
    $X$. Find a function that approximates cumulative distribution function of
    $X$. The observation $x$ corresponds to some elementary event $\omega$, that
    is $(X_1(\omega),\ldots,X_{8}(\omega))=(1,2,4,2,4,1,3,1)$. The desired
    approximation will be
    \begin{align*}
        F_{\mathscr{X}}(t)(\omega)
         & =\frac{1}{n}\sum_{i=1}^{n} 1_{\{X_i(\omega)<t\}} \\
         & =\frac{1}{8}\sum_{i=1}^{8} 1_{\{X_i(\omega)<t\}} \\
         & =\frac{1}{8}\left(
        1_{1<t}+1_{2<t}+1_{4<t}+1_{2<t}+1_{4<t}+1_{1<t}+1_{3<t}+1_{1<t}
        \right)                                             \\
         & =\frac{1}{8}\left(
        3\cdot 1_{1<t}+2\cdot 1_{2<t}+1_{3<t}+2\cdot 1_{4<t}
        \right)                                             \\
    \end{align*}
    Therefore
    $$
        F_{\mathscr{X}}(t)(\omega)
        =\begin{cases}
            0           & < t \leq 1   \\
            \frac{3}{8} & 1 < t \leq 2 \\
            \frac{5}{8} & 2 < t \leq 3 \\
            \frac{6}{8} & 3 < t \leq 4 \\
            1           & 4 < t
        \end{cases}
    $$
\end{example}

\subsection{Point estimates}

Suppose we study a random variable $X$. We have $n$ observations
$x=(x_1,\ldots,x_n)\in\mathbb{R}^n$ and we want to know the distribution of $X$.
In practice we usually have a good guess of what type of distribution $X$ should
have. It might be a normal distribution or a binomial distribution or a Poisson
distribution. All these classes of distributions are parametric meaning that you
need to specify some parameters to pick a concrete distribution of the class.
For example you need to know exact values of $\mu$ and $\sigma^2$ to speak of 
the normal distribution $Norm(\mu,\sigma^2)$. Now given observations
$x\in\mathbb{R}^n$ and a class of distributions $X$ belongs to we can hope to
find parameters of the specific distribution of $X$. This is the primary goal of
point estimations theory.

\begin{definition} Let $\mathcal{D}$ be a family of distributions. We say that
    $\mathcal{D}$ is a parametric family if there is a set of parameters
    $\Theta\subset\mathbb{R}^k$ such that any distribution $D\in\mathcal{D}$ is
    uniquely determined by some group of parameters $\theta\in\Theta$. We write
    this fact as
    $\mathcal{D}=\{D_\theta:\theta=(\theta_1,\ldots,\theta_k)\in\Theta \}$.
\end{definition}

\begin{example} Let $\Theta=\mathbb{R}_+$. Then the class of Poisson
    distributions $\mathcal{P}$ is a parametric family because
    $$
        \mathcal{P}=\{Pois(\lambda):\lambda\in\Theta \}
    $$
\end{example}

\begin{example} Let $\Theta=\mathbb{R}\times\mathbb{R}_+$. Then the class of
    normal distributions $\mathcal{N}$ is a parametric family because
    $$
        \mathcal{N}=\{Norm(\mu,\sigma^2): (\mu,\sigma^2)\in\Theta \}
    $$
\end{example}

\begin{example} Let $\Theta=\mathbb{N}\times[0,1]$. Then the class of binomial
    distributions $\mathcal{B}$ is a parametric family
    $$
        \mathcal{B}=\{Bin(n,p):(n,p)\in\Theta \}
    $$
\end{example}

\begin{definition} Let $X$ be a random variable with distribution from a
    parametric family
    $\mathcal{D}=\{D_\theta:\theta=(\theta_1,\ldots,\theta_k)\in\Theta \}$. Let
    $\mathscr{X}$ be a sample of $X$. A statistic $T$ is called a point
    estimation of $\theta_i\in\mathbb{R}$ if
    $$
        \mathbb{E}[T(\mathscr{X})]=\theta_i
    $$
\end{definition}

\begin{example} Let $X\sim Norm(\mu,\sigma^2)$ be a normal random variable. Let
    $\mathscr{X}$ be a sample from $X$. Then $m(\mathscr{X})$ is a point
    estimation of $\mu$ and $s^2(\mathscr{X})$ is a point estimation of
    $\sigma^2$. Indeed, since $X$ is normal, then $m_X=\mu$ and
    $s_X^2=\sigma^2$. Now using remark~\ref{ExpecStats} we get
    $$
        \mathbb{E}[m(\mathscr{X})]=m_X=\mu,
        \quad\quad
        \mathbb{E}[s^2(\mathscr{X})]=s_X=\sigma^2
    $$
\end{example}

\begin{example} Let $X\sim Unif(a, b)$ be a random variable uniformly
    distributed on $[a, b]$. Let $\mathscr{X}$ be a sample from $X$. Consider
    statistics $L(\mathscr{X})=\min(X_1,\ldots,X_n)$ and
    $U(\mathscr{X})=\max(X_1,\ldots,X_n)$ then one can show that
    $$
        \mathbb{E}[L(\mathscr{X})]=a+\frac{1}{n+1}(b-a),
        \quad\quad
        \mathbb{E}[U(\mathscr{X})]=a+\frac{n}{n+1}(b-a)
    $$
    Therefore
    $$
        a=\frac{n\mathbb{E}[L(\mathscr{X})]-\mathbb{E}[U(\mathscr{X})]}{n-1}
        =\mathbb{E}\left[\frac{n L(\mathscr{X})-U(\mathscr{X})}{n-1}\right],
    $$
    $$
        b=\frac{n\mathbb{E}[U(\mathscr{X})]-\mathbb{E}[L(\mathscr{X})]}{n-1}
        =\mathbb{E}\left[\frac{n U(\mathscr{X})-L(\mathscr{X})}{n-1}\right]
    $$
    These equalities show that statistics
    $$
        A(\mathscr{X})=\frac{n L(\mathscr{X})-U(\mathscr{X})}{n-1},
        \quad\quad
        B(\mathscr{X})=\frac{n U(\mathscr{X})-L(\mathscr{X})}{n-1}
    $$
    are point estimates for $a$ and $b$.
\end{example}

\begin{example} Let $x=(1,2,3,1,3,1,4)$ be observations of the random variable
    $X$ with uniform distribution on some segment $[a,b]$. Our observation
    corresponds to some elementary event $\omega$, so
    $(X_1(\omega),\ldots,X_n(\omega))=(1,2,3,1,3,1,4)$ In our case $n=7$ and
    $$
        L(\mathscr{X})(\omega)=\min(1,2,3,1,3,1,4)=1,
        \quad\quad
        U(\mathscr{X})(\omega)=\max(1,2,3,1,3,1,4)=4
    $$
    $$
        A(\mathscr{X})(\omega)=
        \frac{n L(\mathscr{X})(\omega)-U(\mathscr{X})(\omega)}{n-1}=\frac{1}{2},
        \quad\quad
        B(\mathscr{X})(\omega)=
        \frac{n U(\mathscr{X})(\omega)-L(\mathscr{X})(\omega)}{n-1}=\frac{9}{2}
    $$
    Therefore $a\approx 0.5$, $b\approx 4.5$.
\end{example}

\subsection{Hypothesis testing}

Given observation of a random variable $X$ we can make several guesses about
random variable distribution. These guesses are called hypotheses. Our goal is
to construct a function that chooses one of the hypothesis given observations of
$X$. Such functions are called a criteria. Since we can inspect only finitely
many observations there is always a possibility that our criterion chooses a
wrong hypothesis. We want this to happen as rarely as possible.

\begin{definition} Let $\mathscr{X}$ be a sample. A hypothesis $H$ is any
    proposition regarding $\mathscr{X}$.
\end{definition}

\begin{example} Let $\mathscr{X}$ be a sample. The following statements are
    hypotheses:
    \begin{itemize}
        \item all random variables $X_1,\ldots,X_n$ in $\mathscr{X}$ are
              independent;
        \item all random variables $X_1,\ldots,X_n$ in $\mathscr{X}$ are i.i.d;
        \item $\mathscr{X}$ is a sample from Bernoulli random variable with
              $p=1/2$;
        \item $\mathscr{X}$ is a sample from normal random variable with
              $\mu\in[0.25,0.75]$ and $\sigma^2\in[1,2]$;
    \end{itemize}
\end{example}

\begin{definition} A hypothesis $H$ is called simple if it has the form:
    $\mathscr{X}$ is a sample from distribution $D$. Otherwise $H$ is
    composite.
\end{definition}

\begin{example} Here are a few examples of simple and composite hypotheses:
    \begin{itemize}
        \item $\mathscr{X}\sim Norm(1, 2^2)$ --- simple hypothesis;
        \item $\mathscr{X}\sim Unif(1, 5)$ --- simple hypothesis;
        \item $\mathscr{X}\sim Norm(\mu, 2^2)$ where $\mu\in(-1,1)$ ---
              composite hypothesis;
        \item all random variables in $\mathscr{X}=(X_1,\ldots,X_n)$ are
              independent --- composite hypothesis;
        \item all random variables in $\mathscr{X}=(X_1,\ldots,X_n)$ are i.i.d.
              --- composite hypothesis.
    \end{itemize}
\end{example}

Now we shall formalize the notion of criterion for hypotheses testing.

\begin{definition} Let $H_1,\ldots H_k$ be a set of hypotheses. Then any
    function of the form
    $$
        \delta:\mathbb{R}^n\to \{H_1,\ldots, H_k\}
    $$
    is called a criterion.
\end{definition}

\begin{remark} In practice we usually consider criteria with two hypotheses
    $\{H_1, H_2\}$. The hypothesis $H_1$ is called the null hypothesis, and
    $H_2$ is called the alternative.
\end{remark}

It is rarely possible to develop a criterion that does not make mistakes. In
order to quantify mistakes that a criterion can make we give the following
definition.

\begin{definition} Let $\delta:\mathbb{R}^n\to \{H_1,\ldots,H_k\}$ be a
    criterion. We say that $\delta$ made an error of the $i$-th kind on the
    observation $x=(x_1,\ldots,x_n)$ if the hypothesis $H_i$ is true but
    $\delta(x_1,\ldots,x_n)\neq H_i$. The probability of the error of the $i$-th
    kind is defined by
    $$
        \alpha_i(\delta)=\mathbb{P}(\delta(X_1,\ldots,X_n)\neq H_i | H_i)
    $$
\end{definition}

\begin{example} Let $\mathscr{X}$ be a sample of size $2n$ from Bernoulli random
    variable. Consider two hypotheses:
    $$
        H_1=\{\mathscr{X}\sim Ber(0.5)\}
        \quad\quad
        H_2=\{\mathscr{X}\sim Ber(p), p>0.5\}
    $$
    For example, for these two hypotheses we can define a criterion
    $$
        \delta(x_1,\ldots,x_{2n})
        =\begin{cases}
            H_1 & \mbox{ if } \overline{x}=0.5 \\
            H_2 & \mbox{ otherwise }
        \end{cases}
    $$
    where $\overline{x}=\frac{1}{2n}\sum_{i=1}^{2n} x_i$. Intuitively this
    criterion rarely chooses the null hypothesis and often makes the error of
    the first kind. We shall compute exact values for the errors of the first
    and second kind. For the beginning note that $\mathbb{P}(\overline{x}\neq
        0.5)=1$, so
    $$
        \alpha_1(\delta)
        =\mathbb{P}(\delta(x)\neq H_1|H_1)
        =\frac{\mathbb{P}(\delta(x)\neq H_1 \cap H_1)}{\mathbb{P}(H_1)}
        =\frac{\mathbb{P}(\overline{x}\neq 0.5 \cap H_1)}{\mathbb{P}(H_1)}
        =\frac{\mathbb{P}(H_1)}{\mathbb{P}(H_1)}
        =1
    $$
    $$
        \alpha_2(\delta)
        =\mathbb{P}(\delta(x)\neq H_2|H_2)
        =\frac{\mathbb{P}(\delta(x)\neq H_2 \cap H_2)}{\mathbb{P}(H_2)}
        =\frac{\mathbb{P}(\overline{x}= 0.5 \cap H_2)}{\mathbb{P}(H_2)}
        =\frac{0}{\mathbb{P}(H_2)}
        =0
    $$
    Our criterion almost always makes the error of the first kind and never
    makes the error of the second kind.

    Clearly the problem with this criterion is that it requires exact equality
    for the average rate of heads in Bernoulli trials. The better criterion
    would check that $\overline{x}$ falls into some neighbourhood of 0.5. In
    this case the probability of the error of the first kind would fall
    dramatically but there would be a room for the errors of the second kind.
\end{example}

\begin{remark} In practice we usually consider criteria $\delta$ with two
    hypotheses. In this case we denote $\alpha_1(\delta)$ as $\alpha(\delta)$
    and $\alpha_2(\delta)$ as $\beta(\delta)$. If the criterion in question is
    clear from the context, we denote probabilities of the errors as $\alpha$
    and $\beta$. Even more, the quantity $1-\beta$ is called the power of the
    criterion $\delta$.
\end{remark}

Errors of the first and second kind are somewhat opposite to each other. If you
minimize $\alpha$, the $\beta$ grows bigger and if you try to make $\beta$
smaller you get larger $\alpha$.

\begin{example} Let $\mathscr{X}$ be a sample of size $1$ from a normal random
    variable $X$. We have two simple hypotheses regarding distribution of $X$:
    $$
        H_1=\{\mathscr{X}\sim Norm(0, 1)\}
        \quad\quad
        H_2=\{\mathscr{X}\sim Norm(1,1)\}
    $$
    Consider criterion
    $$
        \delta(x_1)=
        \begin{cases}
            H_1 & \mbox{ if } x_1\leq b \\
            H_2 & \mbox{ if } x_1> b
        \end{cases}
    $$
    Note, that this criterion depends on parameter $b$. By definition we have
    $$
        \alpha=\mathbb{P}(\delta(x_1)\neq H_1|H_1)=\mathbb{P}(x_1>b|H_1)
        \quad\quad
        \beta=\mathbb{P}(\delta(x_1)\neq H_2|H_2)=\mathbb{P}(x_1\leq b|H_2)
    $$
    As $b$ grows bigger we get smaller values of $\alpha$ and larger values
    $\beta$ and vice versa.
\end{example}


\subsection{Statistical tests}

From now on we shall build criteria for some specific but practically important
problems. Suppose we have a sample $\mathscr{X}$ from a random variable $X$.  We
consider only two hypotheses $H_1$ and $H_2=\{H_1\mbox{ is not true }\}$. For
these to hypotheses we shall build criteria of the form
$$
    \delta(x)=
    \begin{cases}
        H_1 & \mbox{ if }\quad |\rho(x)|\leq C \\
        H_2 & \mbox{ if }\quad |\rho(x)|> C
    \end{cases}
$$
Our goal is to invent a function $\rho$ such that the criterion $\delta$
`mostly' gives correct answers. By `mostly' we mean that we do not expect to 
make more than a specified percent (say $\alpha$) of errors of the first kind. 
To tweak the error rate of our criterion we need to choose $C$ such 
that $\alpha(\delta)\leq \alpha$.

\begin{definition} Let $\mathscr{X}$ be a sample of size $n$ from $X$. Consider
    two hypotheses $H_1$ and $H_2=\{H_1\mbox{ is not true }\}$. Let
    $\rho(\mathscr{X})$ be a statistic (called test statistic) such that
    \begin{itemize}
        \item if $H_1$ is true, then cumulative density function of
              $\rho(\mathscr{X})$ pointwise converges to cumulative density
              function of some continuous random variable $\eta$
              $$
                  F_{\rho(\mathscr{X})}(t)\underset{n\to\infty}{\to}F_{\eta}(t)
                  \quad\mbox{ for all }\quad t\in\mathbb{R}
              $$
        \item if $H_1$ is not true, then
              $$
                  \mathbb{P}(|\rho(\mathscr{X})|>\epsilon)
                  \underset{n\to\infty}{\to} 1
                  \quad\mbox{ for all }\quad \epsilon>0.
              $$
    \end{itemize}
    Then a statistical test is a criterion of the form
    $$
        \delta(x)=
        \begin{cases}
            H_1 & \mbox{ if }\quad |\rho(x)|\leq C \\
            H_2 & \mbox{ if }\quad |\rho(x)|> C
        \end{cases}
    $$
    where $C>0$.
\end{definition}

\begin{remark} One needs to clarify the requirements for the test statistic 
    in the previous definition. The first requirement says that if $H_1$ holds 
    true, then $\rho(\mathscr{X})$ attains the same values as $\eta$. 
    If $H_2$ holds true, then for $n$ big enough $\rho(\mathscr{X})$ will 
    attain big values.
\end{remark}

In the very definition of the criterion in statistical tests we assume that test
statistic has distribution close to some distribution which does not depend on a
sample being studied. Now we shall discuss typical distributions encountered in
statistical tests.

\begin{definition} Let $X_1,\ldots,X_n\sim Norm(0, 1)$. Consider random variable
    $X=X_1^2+\ldots+X_n^2$.  % chktex 11
    Its distribution is called the $\chi^2$ distribution
    with $n$ degrees of freedom. Notation: $X\sim Chi(n)$.
\end{definition}

\begin{definition} Let $X_1\sim Norm(0, 1)$ and $X_2\sim Chi(n)$. Consider
    random variable $X=\frac{X_1}{\sqrt{X_2/n}}$. Its distribution is called the
    student distribution with $n$ degrees of freedom. Notation: $X\sim St(n)$.
\end{definition}

\begin{definition} Let $X_1\sim Chi(k)$ and $X_2\sim Chi(n)$. Consider random
    variable $X=\frac{X_1/k}{X_2/n}$. Its distribution is called the Fisher
    distribution with parameters $k$ and $n$. Notation $X\sim F(k,n)$.
\end{definition}

\begin{definition} We say that a random variable $X$ has Kolmogorov's
    distribution if its cumulative distribution function is
    $$
        F_X(t)=\sum_{k=-\infty}^{+\infty} {(-1)}^k e^{-2k^2t^2}.
    $$
    Notation $X\sim Kolm$
\end{definition}

Now we shall list a few constructions that lead to these distributions

\begin{remark} Let $\mathscr{X}=(X_1,\ldots, X_n)\sim Norm(\mu,\sigma^2)$. Then
    \begin{itemize}
        \item $\sqrt{n}\frac{m(\mathscr{X})-\mu}{\sigma}\sim Norm(0,1)$;
        \item $\frac{(n-1)s^2(\mathscr{X})}{\sigma^2}\sim Chi(n-1)$;
        \item $\sqrt{n}\frac{m(\mathscr{X})-\mu}{s(\mathscr{X})}\sim St(n-1)$.
    \end{itemize}
\end{remark}

Now we shall discuss statistical tests used in practice.

\begin{remark}
    Suppose we are given a statistical test. It must be used as follows:
    \begin{itemize}
        \item Choose a level of confidence $\alpha$ (i.e.\ probability of errors
              of the first kind);
        \item Find $C$ such that $\mathbb{P}(|\eta|\geq C)=\alpha$;
        \item Given observations $x$ compute $\rho(x)$;
        \item If $|\rho(x)|>C$ we reject $H_1$, otherwise reject $H_2$.
    \end{itemize}
\end{remark}

\begin{example} \textbf{(Kolmogorov's test)} This criterion tests if a given
    sample $\mathscr{X}$ of size $n$ was sampled from a continuous random
    variable $X$ with cumulative distribution function $F_X$.
    \begin{itemize}
        \item hypotheses:
              $$H_1=\{\mathscr{X}\sim X\},\quad\quad H_2=\{H_1\mbox{ is not
                      true}\}$$
        \item test statistic:
              $$\rho(\mathscr{X})
                  =\sqrt{n}\sup_{t\in\mathbb{R}}|F_{\mathscr{X}}(t)-F_X(t)|$$
        \item distribution of the test statistic if $H_1$ is true: $\eta\sim Kolm$
    \end{itemize}
\end{example}

\begin{example} \textbf{($z$-test)} This criterion checks if a given sample
    $\mathscr{X}$ of size $n$ sampled from a normal random variable $X\sim
        Norm(\mu,\sigma^2)$ (with \underline{unknown} $\mu$ and \underline{known}
    $\sigma$) has mean equal to $\mu_0$.
    \begin{itemize}
        \item hypotheses:
              $$H_1=\{\mu=\mu_0\},\quad\quad H_2=\{H_1\mbox{ is not true }\}$$
        \item test statistic:
              $$\rho(\mathscr{X})=\sqrt{n}\frac{m(\mathscr{X})-\mu_0}{\sigma}$$
        \item distribution of the test statistic if $H_1$ is true: $\eta\sim
                  St(n-1)$
    \end{itemize}
\end{example}

\begin{example} \textbf{($t$-test)} This criterion checks if a given sample
    $\mathscr{X}$ of size $n$ sampled from a normal random variable
    $X\sim Norm(\mu,\sigma^2)$ (with \underline{unknown} $\mu$ and
    \underline{unknown} $\sigma$) has mean equal to $\mu_0$.
    \begin{itemize}
        \item hypotheses:
              $$H_1=\{\mu=\mu_0\},\quad\quad H_2=\{H_1\mbox{ is not true }\}$$
        \item test statistic:
              $$\rho(\mathscr{X})
                  =\sqrt{n}\frac{m(\mathscr{X})-\mu_0}{s(\mathscr{X})}$$
        \item distribution of the test statistic if $H_1$ is true: $\eta\sim
                  Norm(0,1)$
    \end{itemize}
\end{example}

\begin{example} \textbf{(Pirson's test)} This criterion tests if a given sample
    $\mathscr{X}$ of size $n$ was sampled from a continuous random variable $X$
    satisfying certain restrictions on its distribution. Let $A_1,\ldots,A_k$ be
    a sequence of disjoint segments whose union contains all possible values of
    $X$. Let $p_1,\ldots,p_k$ be the expected probabilities that $X$ fall into
    segments $A_1,\ldots,A_k$ respectively. Clearly $p_1,\ldots,p_k$ must sum up
    to 1.
    \begin{itemize}
        \item hypotheses:
              $$H_1=\{\mathbb{P}(X_1\in A_i)=p_i\mbox{ for all }
                  i\in \{1,\ldots,k\} \},\quad\quad H_2=\{H_1\mbox{ is not
                      true}\}$$
        \item test statistic:
              $$\rho(\mathscr{X})=\sum_{i=1}^k
                  \frac{{(\nu_i(\mathscr{X})-np_i)}^2}{n p_i}
              $$
              where
              $$
                  \nu_i(\mathscr{X})
                  =\sum_{j=1}^n 1_{X_j\in A_i}
                  \quad\mbox{the number of }X_j\mbox{ that fall into }A_i
              $$
        \item distribution of the test statistic if $H_1$ is true:
              $\eta\sim Chi(k-1)$
    \end{itemize}
    In practice we do not explicitly specify probabilities
    $p_1,\ldots,p_k$, but compute them from distribution of some
    random variable $X$ using formulae
    $$
        p_i=\mathbb{P}(X\in A_i)
    $$
    This approach gives rise to the false belief that Pirson's test check
    that a sample $\mathscr{X}$ was sampled from random variable $X$.
    This would be true if the number of segments $A_1,\ldots,A_k$ grew to
    infinity while their sizes would uniformly approach zero.
\end{example}

\begin{example} \textbf{(Two-sample Kolmogorov's test)} This criterion tests
    if a given a sample $\mathscr{X}$ of size $n$ from a random variable $X$
    and a sample $\mathscr{Y}$ of size $m$ from a random variable $Y$
    have the same distribution.
    \begin{itemize}
        \item hypotheses:
              $$
                  H_1=\{F_X=F_Y\},\quad\quad H_2=\{H_1\mbox{ is not true }\}
              $$
        \item test statistic:
              $$
                  \rho(\mathscr{X},\mathscr{Y})
                  =\sqrt{\frac{mn}{m+n}}
                  \sup_{t\in\mathbb{R}}|F_{\mathscr{X}}(t)-F_{\mathscr{Y}}(t)|
              $$
        \item distribution of the test statistic if $H_1$ is true: $\eta\sim Kolm$
    \end{itemize}
\end{example}

\begin{example} \textbf{(Two-sample Fisher's test)} This criterion tests if a
    given sample $\mathscr{X}$ of size $n$ from a normal random variable
    $X\sim Norm(\mu_X,\sigma_X^2)$ and a sample $\mathscr{Y}$ of size $m$ from
    a normal random variable $Y\sim Norm(\mu_Y,\sigma_Y^2)$ have 
    the same standard deviation.
    \begin{itemize}
        \item hypotheses:
              $$
                  H_1=\{\sigma_X=\sigma_Y\},\quad\quad H_2
                  =\{H_1\mbox{ is not true}\}
              $$
        \item test statistic:
              $$
                  \rho(\mathscr{X},\mathscr{Y})
                  =\frac{s^2(\mathscr{X})}{s^2(\mathscr{Y})}
              $$
        \item distribution of the test statistic if $H_1$ is true:
              $\eta\sim F(n-1,m-1)$
    \end{itemize}
\end{example}

\begin{example} \textbf{(Two-sample Student's test)} This criterion tests if a
    given sample $\mathscr{X}$ of size $n$ from a normal random variable 
    $X\sim Norm(\mu_X,\sigma^2)$ and a sample $\mathscr{Y}$ of size $m$ from a 
    normal random variable $Y\sim Norm(\mu_Y,\sigma^2)$ have the same mean.
    \begin{itemize}
        \item hypotheses:
              $$
                  H_1=\{\mu_X=\mu_Y\},\quad\quad H_2=\{H_1\mbox{ is not true }\}
              $$
        \item test statistic:
              $$
                  \rho(\mathscr{X},\mathscr{Y})
                  =\sqrt{\frac{mn(n+m-2)}{m+n}}\frac{m(\mathscr{X})
                      -m(\mathscr{Y})}{\sqrt{(n-1)s^2(\mathscr{X})
                          +(m-1)s^2(\mathscr{Y})}}
              $$
        \item distribution of the test statistic if $H_1$ is true:
              $\eta\sim St(n-1,m-1)$
    \end{itemize}
\end{example}

\begin{example} \textbf{(Pirson's test)} This criterion tests if a given sample
    $\mathscr{X}$ of size $n$ of a random variable $X$ and a sample
    $\mathscr{Y}$ of size $n$ from a random variable $Y$ has the property that
    $X$ and $Y$ are independent. Let $A_1,\ldots,A_k$ be a sequence of disjoint
    segments whose union contains all possible values of $X$. Analogously, let
    $B_1,\ldots,B_l$ be a sequence of disjoint segments whose union contains all
    possible values of $Y$. Let $p_{i,j}$ be the expected probability that $X\in
        A_i$ and $Y\in B_j$. Clearly all $p_{i,j}$ must sum up to 1.
    \begin{itemize}
        \item hypotheses:
              $$H_1=\{X\mbox{ and }Y\mbox{ are independent}\},\quad\quad
                  H_2=\{H_1\mbox{ is not true }\}$$
        \item test statistic:
              $$
                  \rho(\mathscr{X},\mathscr{Y})
                  =n\sum_{i=1}^k\sum_{j=1}^l\frac{
                      {\left(\nu_{i,j}-\frac{1}{n}\nu_{i,*}\nu_{*,j}\right)}^2
                  }{\nu_{i,*}\nu_{*,j}}
              $$
              where
              $$
                  \nu_{i,j}=\sum_{s=1}^n 1_{X_s\in A_i}1_{Y_s\in B_j}
                  \quad \mbox{number of pairs }(X_s,Y_s)
                  \mbox{ that fall into }(A_i,B_j)
              $$
              $$
                  \nu_{i,*}=\sum_{j=1}^{l}\nu_{i,j}
                  \quad \mbox{number of }X_s\mbox{ that fall into }A_i
              $$
              $$
                  \nu_{*,j}=\sum_{i=1}^{k}\nu_{i,j}
                  \quad \mbox{number of }Y_s\mbox{ that fall into }B_j
              $$

        \item distribution of the test statistic if $H_1$ is true:
              $\eta\sim Chi((k-1)\cdot(l-1))$
    \end{itemize}
\end{example}

\begin{example} \textbf{(Bartlett test)} This criterion tests if samples from
    different normal random variables have the same variance. Let
    $\mathscr{X}_1,\ldots,\mathscr{X}_k$ be samples of sizes $n_1,\ldots,n_k$
    respectively. Suppose $\mathscr{X}_i\sim Norm(\mu_i,\sigma_i^2)$ and
    $n_i>3$.
    \begin{itemize}
        \item hypotheses:
              $$
                  H_1=\{\sigma_i=\sigma_j
                  \mbox{ for all}i,j\in \{1,\ldots,k\} \},
                  \quad\quad H_2=\{H_1\mbox{ is not true}\}
              $$
        \item test statistic:
              $$
                  \rho(\mathscr{X}_1,\ldots,\mathscr{X}_k)
                  =\frac{
                      (N-k)\ln(s_p^2(\mathscr{X}_1,\ldots,\mathscr{X}_k))
                      -\sum_{i=1}^k (n_i-1)\ln(s^2(\mathscr{X}_i))
                  }{
                      1+\frac{1}{3(k-1)}
                      \left(\sum_{i=1}^k \frac{1}{n_i-1}-\frac{1}{N-k}\right)
                  }
              $$
              where
              $$
                  N=\sum_{i=1}^k n_i,
                  \quad\quad
                  s_p^2(\mathscr{X}_1,\ldots,\mathscr{X}_k)
                  =\frac{1}{N-k}\sum_{i=1}^k (n_i-1) s^2(\mathscr{X}_i)
              $$
        \item distribution of the test statistic if $H_1$ is true: $\eta\sim
                  Chi(k-1)$
    \end{itemize}
\end{example}

\begin{example} \textbf{(ANOVA test)} This criterion tests if samples from
    different normal variables have the same mean. Let
    $\mathscr{X}_1,\ldots,\mathscr{X}_k$ be samples of sizes $n_1,\ldots,n_k$
    respectively. Suppose $\mathscr{X}_i\sim Norm(\mu_i,\sigma^2)$. Note: all
    samples must be sampled from normal variables with \underline{equal}
    variances.
    \begin{itemize}
        \item hypotheses:
              $$
                  H_1=\{\mu_i=\mu_j
                  \mbox{ for all}i,j\in \{1,\ldots,k\} \},
                  \quad\quad H_2=\{H_1\mbox{ is not true}\}
              $$
        \item test statistic:
              $$
                  \rho(\mathscr{X}_1,\ldots,\mathscr{X}_k)
                  =\frac{s_i^2(\mathscr{X}_1,\ldots,\mathscr{X}_k)}
                  {s_p^2(\mathscr{X}_1,\ldots,\mathscr{X}_k)}
              $$
              where
              $$
                  N=\sum_{i=1}^k n_i,
              $$
              $$
                  M(\mathscr{X}_1,\ldots,\mathscr{X}_k)
                  =\frac{1}{N}\sum_{i=1}^k n_i m(\mathscr{X}_i)
                  \quad\mbox{ overall mean}
              $$
              $$
                  s_i^2(\mathscr{X}_1,\ldots,\mathscr{X}_k)
                  =\frac{1}{k-1}\sum_{i=1}^k n_i{(m(\mathscr{X}_i)
                  -M(\mathscr{X}_1,\ldots,\mathscr{X}_k))}^2
                  \quad \mbox{ inter sample variance}
              $$
              $$
                  s_p^2(\mathscr{X}_1,\ldots,\mathscr{X}_k)
                  =\frac{1}{N-k}\sum_{i=1}^k (n_i-1) s^2(\mathscr{X}_i)
                  \quad \mbox{pooled variance}
              $$
        \item distribution of the test statistic if $H_1$ is true:
              $\eta\sim F(k-1, n-k)$
    \end{itemize}
\end{example}
\end{document}