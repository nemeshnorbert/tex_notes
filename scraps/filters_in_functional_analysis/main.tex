\documentclass[12pt]{article}
\usepackage[left=2cm, right=2cm, top=2cm, bottom=2cm,
    bindingoffset=0cm]{geometry}
\usepackage{amssymb,amsmath}
\usepackage[utf8]{inputenc}
\usepackage{mathrsfs}
\usepackage[colorlinks=true, urlcolor=blue, linkcolor=blue, citecolor=blue,
    pdfborder={0 0 0}]{hyperref}
\usepackage{enumitem}

\hypersetup{frenchlinks=true}

\newtheorem{theorem}{Theorem}[section]
\newtheorem{lemma}[theorem]{Lemma}
\newtheorem{proposition}[theorem]{Proposition}
\newtheorem{remark}[theorem]{Remark}
\newtheorem{corollary}[theorem]{Corollary}
\newtheorem{definition}[theorem]{Definition}
\newtheorem{example}[theorem]{Example}

\newenvironment{proof}{\par $\triangleleft$}{$\triangleright$}

\pagestyle{plain}

\begin{document}

\begin{center}

    \Large \textbf{Filters in functional analysis}\\[0.5cm]
    \small {Nemesh N. T.}\\[0.5cm]

\end{center}

\begin{abstract}
    In this note we give a brief introduction into the theory of filters. Then
    we demonstrate several applications of filters in the proof of inevitably
    non-constructive theorems of functional analysis.
\end{abstract}

\section{Set theoretic preliminaries}

For a given set $M$ by $\mathcal{P}(M)$ we denote the set of all its subsets. By
$\mathcal{P}_0(M)$ we denote the set of all its finite subsets.

\begin{definition}\label{DefFilter} Let $M$ be a set, a family
    $\mathcal{F}\subset\mathcal{P}(M)$ with the following properties
    \begin{enumerate}[label = (\roman*)]
        \item $A,B\in\mathcal{F}\implies A\cap B\in\mathcal{F}$
        \item $A\in\mathcal{F}, A\subset B\implies B\in\mathcal{F}$
        \item $\varnothing\notin\mathcal{F}$
    \end{enumerate}
    is called a filter on the set $M$.
\end{definition}

\begin{remark}\label{RemAnyFilterContainsItsSet} Directly from these axioms it
    follows that for a filter $\mathcal{F}$ on a set $M$ we have
    \begin{enumerate}[label = (\roman*)]
        \item $M\in\mathcal{F}$
        \item $A_1,\ldots A_n\in\mathcal{F}\implies A_1\cap\ldots\cap
                  A_n\in\mathcal{F}$

        \item $A\in\mathcal{F}\implies M\setminus A\notin\mathcal{F}$
    \end{enumerate}
\end{remark}

\begin{definition}\label{DefFilterTypes} Let $\mathcal{F}$ be a filter on the
    set $M$, then
    \begin{enumerate}[label = (\roman*)]
        \item $\mathcal{F}$ is called free if $\bigcap\mathcal{F}=\varnothing$
        \item $\mathcal{F}$ is called fixed if $\bigcap\mathcal{F}=\{m\}$, for
              some $m\in M$
    \end{enumerate}
\end{definition}

\begin{definition}\label{DefFilterBase} Let $M$ be a set, then a family
    $\mathcal{B}\subset\mathcal{P}(M)$ is called a filterbase if
    \begin{enumerate}[label = (\roman*)]
        \item $\mathcal{B}\neq\varnothing$
        \item $\varnothing\notin\mathcal{B}$
        \item $A,B\in\mathcal{B}\implies \exists C\in\mathcal{B}\quad C\subset
                  A\cap B$
    \end{enumerate}
\end{definition}

\begin{proposition}\label{PrFilterFromFilterBase} Let $\mathcal{B}$ be a
    filterbase on the set $M$, then the family
    $$
        \mathcal{F}_{\mathcal{B}}=
        \{A\in\mathcal{P}(M):\exists B\in\mathcal{B}: B\subset A\}
    $$
    is a filter on $M$.
\end{proposition}
\begin{proof} Obvious.
\end{proof}

Thus we can describe filters via their filterbases.

\begin{example}\label{ExFrechetFilter} A family
    $\mathcal{F}_0(M)=\{A\in\mathcal{P}(M):\operatorname{Card}(M\setminus A)
    <\aleph_0\}$ is a filter called Frechet filter. Clearly, this is 
    a free filter.
\end{example}

\begin{example}\label{ExSectionFilter} Let $(N,\leq )$ be a directed set, then
    the family $\mathcal{B}_N=\{ \{\nu'\in N:\nu\leq\nu'\}:\nu\in N\}$ is a
    filterbase. The respective filter
    $\mathcal{F}_N=\mathcal{F}_{\mathcal{B}_N}$ is called a section filter or a
    filter of tails.
\end{example}

\begin{example}\label{ExNeighbourhoodsFilter} Let $(X,\tau )$ be a topological
    space, and $x\in X$. Then the set of open neighbourhoods $\mathcal{N}(x)$ of
    $x$ is a filterbase. The respective filter $\mathcal{F}_{\mathcal{N}(x)}$ is
    called a neighbourhoods filter.
\end{example}

Clearly, any filter has the finite intersection property.

\begin{definition} Let $\mathcal{I}$ be a family of subsets of $M$. We say that
    $F$ has the finite intersection property (f.i.p.\ for short) if $A\cap B\in
        \mathcal{I}$ whenever $A,B\in \mathcal{I}$.
\end{definition}

\begin{proposition} Let $\mathcal{I}$ be a non-empty family of subsets of a set
    $M$ with finite intersection property, then
    $$
        \mathcal{I}_{\cap}:=\big \{\cap \mathcal{A} :
        \mathcal{A}\subset\mathcal{P}_0(\mathcal{I})\big \}
    $$
    is a filterbase on $M$.
\end{proposition}
\begin{proof} Since $\mathcal{I}$ is not empty there is a set $A\in\mathcal{I}$.
    Consider $\mathcal{A}=\{A \}\in\mathcal{P}_0(\mathcal{I})$, then
    $A=\cap\mathcal{A}\in\mathcal{I}_{\cap}$, so
    $\mathcal{I}_{\cap}\neq\varnothing$. Suppose, $\varnothing \notin
        \mathcal{I}_{\cap}$, then there is a finite family
    $\mathcal{A}\subset\mathcal{P}_0(\mathcal{I})$ with
    $\cap\mathcal{A}=\varnothing$. This contradicts finite intersection property
    of $\mathcal{I}_{\cap}$, hence $\varnothing \notin \mathcal{I}_{\cap}$.
    Finally, let $A_1, A_2\in\mathcal{I}_{\cap}$, then $A=\cap\mathcal{A}_1$ and
    $A_2=\cap\mathcal{A}_2$ for some
    $\mathcal{A}_1,\mathcal{A}_2\in\mathcal{P}_0(\mathcal{I})$. Clearly,
    $\mathcal{A}:=\mathcal{A}_1\cup\mathcal{A}_2\in\mathcal{P}_0(\mathcal{I})$,
    so $A:=\cap\mathcal{A}\in\mathcal{I}_{\cap}$ and, obviously $A\subset
        A_1\cap A_2$.
\end{proof}

\begin{example}\label{ExFilterGenBySet} Let $A$ be a non-empty subset of a set
    $M$, then the family $F_A=\{B\in\mathcal{P}(M):A\subset B\}$ is a filter,
    called a filter generated by set $A$.
\end{example}

\begin{remark}\label{RemDescOfFiltersOnFiniteSets} Every filter $\mathcal{F}$ on
    a finite set $M$ is of the form $\mathcal{F}_{A}$. Indeed, $\mathcal{F}$ is
    a finite set, then $A=\bigcap\mathcal{F}$ is finite intersection of elements
    of $\mathcal{F}$, so $A\in\mathcal{F}$. Therefore any $B\in\mathcal{F}$
    contains $A$, and $\mathcal{F}\subset\mathcal{F}_A$. On the other hand any
    $B\in\mathcal{P}(M)$ that contains $A\in\mathcal{F}$ is in $\mathcal{F}$ by
    definition of filter. So $\mathcal{F}_A\subset\mathcal{F}$.
\end{remark}


\begin{definition}\label{DefFilterComparsion} Let $\mathcal{F}_1$,
    $\mathcal{F}_2$ be two filter on a set $M$. We say that $\mathcal{F}_2$
    dominates $\mathcal{F}_1$ and write $\mathcal{F}_1\leq \mathcal{F}_2$ if
    $\mathcal{F}_1\subset\mathcal{F}_2$.
\end{definition}

\begin{remark} Let $\mathscr{F}$ be a family of filters on $M$, then
    $\mathcal{F}=\bigcap \mathscr{F}$ is a filter. Clearly $\mathcal{F}$ is
    dominated by any filter of $\mathscr{F}$.
\end{remark}

\begin{definition}\label{DefUltraFilter} A filter $\mathcal{U}$ on a set $M$ is
    called an ultrafilter if any filter that dominates $\mathcal{U}$ equals
    $\mathcal{U}$.
\end{definition}

\begin{remark}\label{RemFixedFilterIsAnUltraFilter} It is easy to see that any
    fixed filter is an ultrafilter, but there are free ultrafilters too.
\end{remark}

Now we present a very important lemma --- an ultrafilter lemma.

\begin{lemma}\label{LemExistenceOfUltraFilters} Let $\mathcal{F}$ be a filter on
    a set $M$, then there exists an ultrafilter $\mathcal{U}$ that dominates
    $\mathcal{F}$.
\end{lemma}
\begin{proof} Let $\mathscr{F}$ be a set of filters on $M$ that dominate
    $\mathcal{F}$. It is easy to check that any linearly ordered chain
    $\mathscr{C}\subset \mathscr{F}$ has a maximal element $\bigcup\mathscr{C}$.
    By Zorn's lemma $\mathscr{F}$ has a maximal element $\mathcal{U}$. By
    construction this is an ultrafilter that dominates $\mathcal{F}$.
\end{proof}


Note: the ultrafilter lemma is weaker than the axiom of choice.

\begin{proposition}\label{PrUltraFilterCharac} Let $\mathcal{F}$ be a filter on
    a set $M$. Then the following are equivalent:
    \begin{enumerate}[label = (\roman*)]
        \item $A_1\cup\ldots\cup A_n\in\mathcal{F}\implies\exists i\in
                  \{1,\ldots,n\}\quad A_i\in\mathcal{F}$;
        \item $A\cup B\in\mathcal{F}\implies (A\in\mathcal{F})\vee
                  (B\in\mathcal{F})$;
        \item $(A\in\mathcal{F})\vee(M\setminus A\in\mathcal{F})$;
        \item $\mathcal{F}$ is an ultrafilter;
    \end{enumerate}
\end{proposition}
\begin{proof} $(i)\implies (ii)$ Obvious

    $(ii)\implies (iii)$ Note that $M=A\cup(M\setminus A)$ and recall that
    $M\in\mathcal{F}$.

    $(iii)\implies (iv)$ Let $\mathcal{G}$ be a filter on $M$ dominating
    $\mathcal{F}$. Consider arbitrary $A\in\mathcal{G}$, then $M\setminus
        A\notin\mathcal{G}$ and a fortiori $M\setminus A\notin\mathcal{F}$. By
    assumption $A\in\mathcal{F}$. Since $A\in\mathcal{G}$ is arbitray
    $\mathcal{F}$ dominates $\mathcal{G}$, but by construction $\mathcal{G}$
    dominates $\mathcal{F}$. Hence $\mathcal{G}=\mathcal{F}$ and therefore
    $\mathcal{F}$ is an ultrafilter.

    $(iv)\implies (ii)$ Assume that $A\notin\mathcal{F}$ and
    $B\notin\mathcal{F}$. One can easily check that
    $\mathcal{G}=\{C\in\mathcal{P}(M):A\cup C\in\mathcal{F}\}$ is a filter on
    $M$. A moment thought shows that $B\in\mathcal{G}$ and $\mathcal{G}$
    dominates $\mathcal{F}$. Since $B\notin \mathcal{F}$, then $\mathcal{F}$ is
    not an ultrafilter.

    $(ii)\implies (i)$ Obvious induction on $n$.
\end{proof}

\begin{remark}\label{RemDescOfUltrafiltersOnFiniteSets} Any ultrafilter
    $\mathcal{U}$ on a finite set $M$ is fixed. As we noted above
    $\mathcal{U}=\mathcal{F}_A$ for some $A\in\mathcal{U}$. If
    $\operatorname{Card}(A)>1$, then $A$ has a proper subset $B$ and
    $\mathcal{F}_B$ dominates $\mathcal{F}_A=\mathcal{U}$ while not equal to
    $\mathcal{U}$. Hence $\mathcal{U}$ is not an ultrafilter, contradiction.
    Therefore $A$ is a singleton and $\mathcal{U}$ is fixed.
\end{remark}

\begin{proposition}\label{PrFreeUltraFilterCharac} An ultrafilter $\mathcal{U}$
    on a infinite set $M$. Then $\mathcal{U}$ is free iff it dominates Frechet
    filter on $M$.
\end{proposition}
\begin{proof} Assume $\mathcal{F}_0(M)\not\subset\mathcal{U}$, then there exists
    $A=\{m_1,\ldots,m_n\}\in\mathcal{P}_0(M)$ such that $M\setminus A\notin
        \mathcal{U}$. Therefore $A\in\mathcal{U}$. Since $A=\{m_1\}\cup
        \ldots\cup \{m_n\}$, then $\{m_i\}\in\mathcal{U}$ for some
    $i\in \{1,\ldots,n\}$.
    Thereofore $\mathcal{F}_{\{m_i\}}\subset\mathcal{U}$. Since
    $\mathcal{F}_{\{m_i\}}$ is an ultrafilter, then
    $\mathcal{U}=\mathcal{F}_{\{m_i\}}$ and
    $\bigcap\mathcal{U}=\{m_i\}\neq\varnothing$. Thus $\mathcal{U}$ is not an
    ultrafilter.

    Conversely, if $\mathcal{U}$ contains Frechet filter, then
    $\bigcap\mathcal{U}\subset\bigcap\mathcal{F}_0(M)=\varnothing$. Therefore
    $\mathcal{U}$ is free.
\end{proof}

\begin{proposition}\label{PrImageAndPreimageOfAFilter} Let $\varphi:M\to N$ be a
    map between sets $M$ and $N$. Then
    \begin{enumerate}[label = (\roman*)]
        \item If $\mathcal{F}$ is a filter on $M$, then
              $\varphi_{\rightarrow}[\mathcal{F}]
                  :=\{\varphi(A):A\in\mathcal{F}\}$
              is a filter base on $N$.
        \item If $\mathcal{F}$ is a filter on $M$, then
              $\varphi_{\leftarrow}[\mathcal{F}]=
                  \{P\in\mathcal{P}(N):\varphi^{-1}(P)\in\mathcal{F}\}$
              is a filter on $N$.
        \item $\varphi_{\rightarrow}[\mathcal{F}]$ is generated by
              $\varphi_{\leftarrow}[\mathcal{F}]$
        \item If $\mathcal{F}$ is an ultrafilter on $M$, then
              $\varphi_{\leftarrow}[\mathcal{F}]$ is an ultrafilter on $N$.
    \end{enumerate}
\end{proposition}
\begin{proof} $(i)$ Let $P_1,P_2\in\varphi_{\rightarrow}[\mathcal{F}]$, then
    $P_1=\varphi(A_1)$, $P_2=\varphi(A_2)$, then there exists
    $P_3=\varphi(A_1\cap A_2)\in\varphi_{\rightarrow}[\mathcal{F}]$ such that
    $P_3\subset\varphi(A_1)\cap\varphi(A_2)=P_1\cap P_2$. Since
    $M\in\mathcal{F}$, then $\varphi(M)\in\varphi_{\rightarrow}[\mathcal{F}]$
    and $\varphi_{\rightarrow}[\mathcal{F}]\neq\varnothing$. If
    $\varnothing\in\varphi_{\rightarrow}[\mathcal{F}]$, then
    $\varnothing=\varphi(A)$ for some $A\in\mathcal{F}$. In fact,
    $A=\varnothing$, contradiction. So
    $\varnothing\notin\varphi_{\rightarrow}[\mathcal{F}]$. Therefore
    $\varphi_{\rightarrow}[\mathcal{F}]$ is a filter base on $N$.

    $(ii)$ Let $P_1,P_2\in\varphi_{\leftarrow}[\mathcal{F}]$. Then
    $\varphi^{-1}(P_1)\in\mathcal{F}$, $\varphi^{-1}(P_2)\in\mathcal{F}$ and
    $\varphi^{-1}(P_1\cap
        P_2)=\varphi^{-1}(P_1)\cap\varphi^{-1}(P_1)\in\mathcal{F}$, i.e.
    $P_1\cap P_2\in\varphi_{\leftarrow}[\mathcal{F}]$ Consider arbitrary
    $A\in\varphi_{\leftarrow}[\mathcal{F}]$ and $B\in\mathcal{P}(M)$ such that
    $A\subset B$. Since $A\in\varphi_{\leftarrow}[\mathcal{F}]$, then
    $\varphi^{-1}(A)\in\mathcal{F}$. Since $A\subset B$, then
    $\varphi^{-1}(B)\supset\varphi^{-1}(A)$. Therefore
    $B\in\varphi_{\leftarrow}[\mathcal{F}]$. Finally, if
    $\varnothing\in\mathcal{G}$, then
    $\varnothing=\varphi^{-1}(\varnothing)\in\mathcal{F}$. Contradiction, so
    $\varnothing\notin\varphi_{\leftarrow}[\mathcal{F}]$. Therefore
    $\varphi_{\leftarrow}[\mathcal{F}]$ is a filer on $N$.

    $(iii)$ Let $P\in\varphi_{\rightarrow}[\mathcal{F}]$, then $P=\varphi(A)$
    for some $A\in\mathcal{F}$. Note that $\varphi^{-1}(P)\supset
        A\in\mathcal{F}$, so $\varphi^{-1}(P)\in\mathcal{F}$ and
    $P\in\varphi_{\rightarrow}[\mathcal{F}]$. This means that
    $\varphi_{\rightarrow}[\mathcal{F}]\subset
        \varphi_{\leftarrow}[\mathcal{F}]$. Take any $P\in
        \varphi_{\leftarrow}[\mathcal{F}]$, then
    $A=\varphi^{-1}(P)\in\mathcal{F}$ and
    $\varphi(A)\in\varphi_{\rightarrow}[\mathcal{F}]$. Clearly,
    $\varphi(A)\subset P$. Since $P$ is arbitrary we conclude that
    $\varphi_{\leftarrow}[\mathcal{F}]$ is generated by
    $\varphi_{\rightarrow}[\mathcal{F}]$.

    $(iv)$ Assume $P\notin\varphi_{\leftarrow}[\mathcal{F}]$, then
    $\varphi^{-1}(P)\notin\mathcal{F}$. As $\mathcal{F}$ is an ultrafilter, then
    $\varphi^{-1}(N\setminus P)=M\setminus\varphi^{-1}(P)\in\mathcal{F}$. Hence
    $N\setminus P\in\varphi_{\leftarrow}[\mathcal{F}]$. Thus
    $\varphi_{\leftarrow}[\mathcal{F}]$ is an ultrfilter.
\end{proof}











\section{Filters in topology}

\begin{definition}\label{DefLimitAlongFilter} Let $\mathcal{F}$ be a filter on a
    set $M$, and $\varphi:M\to X$ be a map from $M$ to the topological space
    $X$. We say that $x$ is a limit of $\varphi$ along $\mathcal{F}$ and write
    $x\in\lim_{\mathcal{F}} \varphi(m)$ if
    $$
        \mathcal{N}(x)\subset \varphi_{\leftarrow}[\mathcal{F}]
    $$
    which is equivalent to
    $$
        \forall U\in\mathcal{N}(x)\quad \varphi^{-1}(U)\in\mathcal{F}
    $$
\end{definition}

\begin{remark}\label{RemReductionFromLimitAlongFilter}

    $(i)$ If $\mathcal{F}=\mathcal{F}_0(\mathbb{N})$, then we get the usual
    limit of a sequence. $(ii)$ If $M$ is a topological space and
    $\mathcal{F}=\mathcal{N}(m)$, then we get the usual definition of limit of
    function between topological spaces. $(iii)$ If $M$ is a directed set and
    $\mathcal{F}=\mathcal{F}_M$ we get the definition of a limit of the net
    ${(\varphi_m)}_{m\in M}$.
\end{remark}

\begin{remark}\label{RemOnCheckingTheDefinitionOfLimit} Let $b$ be a prebase of
    topology $\tau$ of a topological space $X$. Since all open sets are unions
    of finite intersections of elements of b, then it is enough to check the
    definition of limit along the filter not for all neighbourhoods of the point
    but just for elements of prebase.
\end{remark}

\begin{remark}\label{RemLimitOfAConst} If $\varphi(m)=x$ for all $m\in M$ and
    some $x\in X$, then for any filter $\mathcal{F}$ we have
    $\lim_{\mathcal{F}}\varphi(m)=x$. Indeed for any $U\in\mathcal{N}(x)$ we
    have $\varphi^{-1}(U)=M\in\mathcal{F}$.
\end{remark}

\begin{remark}\label{RemLimitAlongFilterGeneratedByASet} If
    $\mathcal{F}=\mathcal{F}_A$ for some $A\in\mathcal{P}_0(M)$, then
    $\varphi(A)\subset \operatorname{cl}_X(\{x\})$. Indeed, for any
    $U\in\mathcal{N}(x)$ we have $A\subset\varphi^{-1}(U)$. This is equivalent
    to $\varphi(A)\subset\operatorname{cl}_X(\{x\})$. If $A=\{m\}$ we get that
    the limit along the fixed ultrafilter $\mathcal{F}_{\{m\}}$ always equals
    $\operatorname{cl}_X(\varphi(m))$.
\end{remark}

\begin{proposition}\label{PrChangeOfFilterInLimit} Let $\mathcal{F}_1,
        \mathcal{F}_2$ be two filters on a set $M$ and $\varphi:M\to X$ be a map
    from $M$ to the topological space $X$. If $\mathcal{F}_2$ dominates
    $\mathcal{F}_1$, then
    $$
        \lim_{\mathcal{F}_1}\varphi(m)\subset\lim_{\mathcal{F}_2}\varphi(m)
    $$
\end{proposition}
\begin{proof} Obvious.
\end{proof}

\begin{proposition}\label{RemUniqueLimitAlongFilter} In a Hausdorff topological
    space, if a limit along the filter exists it is unique. In this case we will
    write $x=\lim_{\mathcal{F}}\varphi(m)$
\end{proposition}
\begin{proof} Let $x,y\in\lim_{\mathcal{F}}\varphi(m)$. Assume $x\neq y$. Since
    $X$ is a Hausdorff space, then there exist $U\in\mathcal{N}(x)$ and
    $V\in\mathcal{N}(y)$ such that $U\cap V=\varnothing$. Since
    $x,y\in\lim_{\mathcal{F}}\varphi(m)$, then
    $\varphi^{-1}(U),\varphi^{-1}(V)\in\mathcal{F}$. In particular,
    $\varnothing=\varphi^{-1}(U\cap
        V)=\varphi^{-1}(U)\cap\varphi^{-1}(V)\in\mathcal{F}$. Contradiction, so
    $x=y$.
\end{proof}

\begin{proposition}\label{PrLimitAlongUltrafilterIntoTheCompact} Let
    $\mathcal{U}$ be an ultrafilter on a set $M$. Let $\varphi:M\to X$ be a map
    from $M$ into a compact topological space $X$. Then
    $\lim_{\mathcal{F}}\varphi(m)$ exists.
\end{proposition}
\begin{proof} Suppose no point in $x\in X$ is a limit of $\varphi$ along
    $\mathcal{U}$. Hence for every $x\in X$ there is a neighbourhood
    $U_x\in\mathcal{N}(x)$ such that $\varphi^{-1}(U_x)\notin\mathcal{U}$. By
    compactness of $X$ a cover $\{U_x:x\in X\}$ have a finite subcover
    $\{U_{x_k}:k\in \{1,\ldots,n\} \}$. Note that
    $$
        \varphi^{-1}(U_{x_1})\cup\ldots\cup\varphi^{-1}(U_{x_k})
        =\varphi^{-1}(X)=M\in\mathcal{F}
    $$
    Since $\mathcal{U}$ is an ultrfilter, then $U_{x_i}\in\mathcal{U}$ for some
    $i\in \{1,\ldots,n\}$. Contradiction. Hence there exists an $x\in X$ such
    that $x\in\lim_{\mathcal{F}}\varphi(m)$.
\end{proof}


\begin{corollary}\label{PrLimitOfASequenceAlongUltrafilter} Let
    ${(x_\nu)}_{\nu\in N}$ be a bounded net in $\mathbb{C}$ and $\mathcal{U}$ be
    an ultrafilter dominating section filter on $N$, then
    $\lim_{\mathcal{U}}x_\nu$ exists and unique.
\end{corollary}
\begin{proof} Existence follows from
    proposition~\ref{PrLimitAlongUltrafilterIntoTheCompact}. Uniqueness follows
    from proposition~\ref{RemUniqueLimitAlongFilter}.
\end{proof}


\begin{proposition}\label{PrFiltersUnderContinuousFunctions} Let $\mathcal{F}$
    be a filter on a set $M$, $X$ and $Y$ be two topological spaces. Assume we
    are given a map $\varphi:M\to X$ and a continuous map $g: X\to Y$. Then
    $$
        g\left(\lim_{\mathcal{F}}\varphi(m)\right)
        \subset\lim_{\mathcal{F}}g(\varphi(m))
    $$
\end{proposition}
\begin{proof} Let $x\in\lim_{\mathcal{F}}\varphi(m)$. For any
    $U\in\mathcal{N}(x)$ we have $\varphi^{-1}(U)\in\mathcal{F}$. Let
    $V\in\mathcal{N}(g(x))$, then $g^{-1}(V)\in\mathcal{N}(x)$ and therefore
    $$
        {(g\circ\varphi)}^{-1}(V)=\varphi^{-1}(g^{-1}(V))\in\mathcal{F}
    $$
    Since $V\in\mathcal{N}(x)$ is arbitrary, then
    $g(x)\in\lim_{\mathcal{F}}(g\circ\varphi)(m)=
        \lim_{\mathcal{F}}g(\varphi(m))$
\end{proof}

For a given family of topological spaces ${(X_\lambda)}_{\lambda\in \Lambda}$ by
$\prod_{\lambda\in \Lambda}X_\lambda$ we denote their Tychonoff's product. By
$\pi_\lambda:\prod_{\lambda\in \Lambda}X_\lambda\to X_\lambda$ and
$i_\lambda:X_\lambda\to \prod_{\lambda\in \Lambda}X_\lambda$ we denote the
natural projections and injections respectively. A prebase of this topology is
$\{i_{\lambda}(U_\lambda):\lambda\in\Lambda, x\in X_\lambda, U_\lambda\in
    \mathcal{N}(x_\lambda)\}$.

\begin{proposition}\label{PrTopProdAndFilters} Let $\mathcal{F}$ be a filter on
    a set $M$, ${(X_\lambda)}_{\lambda\in \Lambda}$ be a family of topological
    spaces. Assume we are given a map $\varphi:M\to \prod_{i\in I}X_i$, then
    $$
        \lim_{\mathcal{F}}\varphi(m)=
        \prod_{\lambda\in\Lambda}\lim_{\mathcal{F}}\pi_\lambda(\varphi(m))
    $$
\end{proposition}
\begin{proof} Let $x\in X=\prod_{i\in I}X_i$. Then
    $$
        \begin{aligned}
            x\in \lim_{\mathcal{F}}\varphi(m)
             & \Longleftrightarrow
            \forall U\in\mathcal{N}(x)\quad\varphi^{-1}(U)\in\mathcal{F} \\
             & \Longleftrightarrow
            \forall \lambda\in\Lambda\quad\forall
            U_\lambda\in\mathcal{N}(x_\lambda)
            \quad\varphi^{-1}(i_\lambda(U_\lambda))\in\mathcal{F}        \\
             & \Longleftrightarrow
            \forall \lambda\in\Lambda\quad\forall
            U_\lambda\in\mathcal{N}(x_\lambda)
            \quad\varphi^{-1}(\pi^{-1}_\lambda(U_\lambda))\in\mathcal{F} \\
             & \Longleftrightarrow
            \forall \lambda\in\Lambda
            \quad x_\lambda\in\lim_{\mathcal{F}}\pi_\lambda(\varphi(m))  \\
             & \Longleftrightarrow
            x\in\prod_{\lambda\in\Lambda}
            \lim_{\mathcal{F}}\pi_\lambda(\varphi(m))
        \end{aligned}
    $$
\end{proof}

\begin{remark}\label{RemBasicPropertiesOfLimitsAlngFilters} As a consequence of
    two previous proposition we see that limits along filters act much like the
    usual limits. We can substitute Tychonoff's product in the role of $X$ in
    the proposition~\ref{PrFiltersUnderContinuousFunctions}. Therefore we can
    handle limits along filters for several variables in the limit. For example
    $$
        \lim_{\mathcal{F}}g(\varphi_1(m), \varphi_2(m))
        =g(\lim_{\mathcal{F}}\varphi_1(m), \lim_{\mathcal{F}}\varphi_2(m))
    $$
    As the consequence, limit along any filter for a scalar valued functions is
    linear and multiplicative.
\end{remark}

\begin{proposition} Let $\mathcal{F}$ be a filter on a set $M$. Assume we are
    given two maps $\varphi:M\to\mathbb{R}$ and $\psi:M\to\mathbb{R}$ such that
    there exist $\lim_{\mathcal{F}}\varphi(m)$ and $\lim_{\mathcal{F}}\psi(m)$.
    Then
    $$
        \forall m\in M\quad\varphi(m)\leq\psi(m) \implies
        \lim_{\mathcal{F}}\varphi(m)\leq\lim_{\mathcal{F}}\psi(m)
    $$
\end{proposition}
\begin{proof} Assume $\lim_{\mathcal{F}}\varphi(m)-\lim_{\mathcal{F}}\psi(m)=
        \lim_{\mathcal{F}}(\varphi(m)-\psi(m))=a>0$. Consider
    $U=(a/2,3a/2)\in\mathcal{N}(a)$, then
    ${(\varphi-\psi)}^{-1}(U)\in\mathcal{F}$. On the other hand
    ${(\varphi-\psi)}^{-1}(U)=\varnothing$ because $\varphi(m)\leq\psi(m)$
    for all $m\in M$. Therefore $\varnothing\in\mathcal{F}$. Contradiction,
    hence $\lim_{\mathcal{F}}\varphi(m)\leq\lim_{\mathcal{F}}\psi(m)$.
\end{proof}

\begin{definition}\label{DefClusterPointAlongTheFilter} Let $\mathcal{F}$ be a
    filter on  a set $M$, and $\varphi:M\to X$ be a map from $M$ to a
    topological space $X$. We say that a point $x\in X$ is a cluster point of
    $\varphi$ along filter $\mathcal{F}$ is
    $$
        \forall U\in\mathcal{N}(x)\quad \forall A\in\mathcal{F}
        \quad \varphi^{-1}(U)\cap A\neq\varnothing
    $$
\end{definition}

\begin{proposition}\label{PrClusterPointAlongTheFilterCharac} Let $\mathcal{F}$
    be a filter on  a set $M$, and $\varphi:M\to X$ be a map from $M$ to the
    topological space $X$. Then the set of cluster point of $\varphi$ along
    filter $\mathcal{F}$ equals
    $$
        \bigcap_{A\in\mathcal{F}} \operatorname{cl}_X(\varphi(A))
    $$
\end{proposition}
\begin{proof} This is just manipulations with definitions:
    $$
        \begin{aligned}
            x\in \bigcap_{A\in\mathcal{F}} \operatorname{cl}_X(\varphi(A))
             & \Longleftrightarrow
            \forall A\in\mathcal{F} \quad
            \forall U\in\mathcal{N}(x)
            \quad U\cap \varphi(A)\neq\varnothing      \\
             & \Longleftrightarrow
            \forall U\in\mathcal{N}(x)\quad
            \forall A\in\mathcal{F}
            \quad \varphi^{-1}(U)\cap A\neq\varnothing \\
        \end{aligned}
    $$
\end{proof}

\begin{proposition}\label{PrClusterPointAlongTheFilterCriterion} Let
    $\mathcal{F}$ be a filter on  a set $M$, and $\varphi:M\to X$ be a map from
    $M$ to a topological space $X$. Then $x\in X$ is a cluster point of
    $\varphi$ along filter $\mathcal{F}$ iff there exists an ultrafilter
    $\mathcal{G}$ dominating $\mathcal{F}$ such that
    $x\in\lim_{\mathcal{G}}\varphi(m)$
\end{proposition}
\begin{proof} $\Longrightarrow$ Consider family
    $\mathcal{B}=\{\varphi^{-1}(U)\cap A: A\in\mathcal{F},
        U\in\mathcal{N}(x)\}$. Since $x$ is a cluster point of $\varphi$ along
    $\varphi$, then $\mathcal{B}$ is non-empty and doesn't contain an empty set.
    If $A,B\in\mathcal{F}$ and $U,V\in\mathcal{N}(x)$ then
    $$
        (\varphi^{-1}(U)\cap A)\cap(\varphi^{-1}(V)\cap B)
        =\varphi^{-1}(U\cap V)\cap (A\cap B)
    $$
    Since $U\cap V\in\mathcal{N}(x)$ and $A\cap B\in\mathcal{F}$, then by
    definition of  cluster point of $\varphi$ along the filter $\mathcal{F}$ the
    intersection is non-empty. As the consequence $\mathcal{B}$ is a filter
    base. Let $\mathcal{G}$ be an ultrafilter containing it, then $\mathcal{G}$
    dominates $\mathcal{F}$. Indeed, for any $A\in\mathcal{F}$,
    $U\in\mathcal{N}(x)$ we have $\varphi^{-1}(U)\cap
        A\in\mathcal{B}\subset\mathcal{G}$, so $A\in\mathcal{G}$ because
    $\varphi^{-1}(U)\cap A\subset A$. For the same reason
    $\varphi^{-1}(U)\in\mathcal{G}$. Since $U$ is arbitrary
    $x\in\lim_{\mathcal{G}}\varphi(m)$

    $\Longleftarrow$  For each $U\in\mathcal{N}(x)$ we have
    $\varphi^{-1}(U)\in\mathcal{G}$ If $A\in\mathcal{F}\subset \mathcal{G}$,
    then $\varphi^{-1}(U)\cap A\in\mathcal{G}$ which implies
    $\varphi^{-1}(U)\cap A\neq \varnothing$. Since $U\in\mathcal{N}(x)$ is
    arbitrary, then $x$ is a cluster point of $\varphi$ along $\mathcal{F}$.
\end{proof}

\begin{corollary}\label{CorClusterPintAreLimitsAlongUltrafilters} Let
    $\mathcal{F}$ be a filter on  a set $M$, and $\varphi:M\to X$ be a map from
    $M$ to a compact Hausdorff space $X$. Then $\lim_{\mathcal{F}} \varphi(m)$
    is the only cluster point of $\varphi$ along $\mathcal{F}$
\end{corollary}
\begin{proof} Consider ultrafilter $\mathcal{G}$ dominating filter
    $\mathcal{F}$. By proposition~\ref{PrLimitAlongUltrafilterIntoTheCompact}
    there exists an $x\in\lim_{\mathcal{G}}\varphi(m)$. It is unique because $X$
    is Hausdorff. Now we apply
    proposition~\ref{PrClusterPointAlongTheFilterCriterion}.
\end{proof}




\section{Filters in functional analysis}

Now we present a short proof of Banach-Alaoglu theorem.

\begin{proposition}\label{PrBanachAlaoglu} Let $E$ be a normed space, then the
    unit ball $B_{E^*}$ of $E^*$ is weak${}^*$ compact.
\end{proposition}
\begin{proof} It is enough to show that every net ${(f_\nu)}_{\nu\in N}\subset
        B_{E^*}$ have weak${}^*$ convergent subnet. Consider ultrafilter
    dominating section filter of the directed set $N$. For each $x\in X$ the
    set $\{f_\nu(x):\nu\in N\}$ is bounded in $\mathbb{C}$, so by
    proposition~\ref{PrLimitOfASequenceAlongUltrafilter} we have a well
    defined map $f:X\to\mathbb{C}: x\mapsto\lim_{\mathcal{U}}f_\nu(x)$. By
    remark~\ref{RemBasicPropertiesOfLimitsAlngFilters} it is a bounded
    linear functional. Thus we proved that $f$ is limit of $1_{E^*}$ anlong
    $\mathcal{U}$ in $(E^*,\sigma(E^*, E))$. By
    proposition~\ref{PrClusterPointAlongTheFilterCriterion} we have that $f$
    is a cluster point of $1_{E^*}$ along $\mathcal{F}_N$, because
    $\mathcal{U}$ is an ultrafilter that dominates section filter
    $\mathcal{F}_N$. Therefore $f$ is an accumulation point of
    ${(f_\nu)}_{\nu\in N}$.
\end{proof}

Here is one more application of ultrafilters to show existence of so called
Banach limits.

\begin{definition}\label{DefBanachLimit} A linear functional
    $f\in{({\ell}_\infty)}^*$ is called a Banach limit if it satisfies the
    following conditions:
    \begin{enumerate}[label = (\roman*)]
        \item $f$ extend the linear functional $f_0:c\to\mathbb{C}:x\mapsto
                  \lim_{n\to\infty} x(n)$.
        \item $\Vert f\Vert=1$ and $\liminf_{n\to\infty}x(n)\leq f(x)\leq
                  \limsup_{n\to\infty} x(n)$.
        \item $f(S(x))=f(x)$ for all $x\in\ell_\infty$ where
              $S:\ell_\infty\to\ell_\infty$ is a left shift operator.
        \item $f(x)\geq 0$ for all $x\geq 0$, $x\in\ell_\infty$.
    \end{enumerate}
\end{definition}

\begin{proposition}\label{PrBanachLimitExists} A Banach limit exists.
\end{proposition}
\begin{proof} For each $x\in \ell_\infty$ and $n\in\mathbb{N}$ consider number
    $f_n(x)=\frac{1}{n}\sum_{k=1}^n x(k)$. We have the following properties for
    this family of functionals

    $(i')$ By Caezaro theorem $\lim_{n\to\infty} f_n(x)=\lim_{n\to\infty}x(n)$
    for all $x\in c$.

    $(ii')$ $|f_n(x)|\leq \Vert x\Vert$ for all $x\in\ell_\infty$ and
    $n\in\mathbb{N}$. Even more, for all real valued $x\in\ell_\infty$ anf
    $\varepsilon>0$ there exist an $N\in\mathbb{N}$ such that
    $\liminf_{n\to\infty}x(n)-\varepsilon<|f_n(x)|\leq
    \limsup_{n\to\infty}x(n)+\varepsilon$ for all $n>N$.

    $(iii')$ $\lim_{n\to\infty} (f_n(S(x))-f_n(x))=0$ for all $x\in\ell_\infty$

    $(iv')$ $f_n(x)\geq 0$ for all $x\in\ell_\infty$

    Now we proceed to the proof of paragraphs $(i)-(iv)$.

    $(i)$ Now $(i')$ gives that ${(f_n(x))}_{n\in\mathbb{N}}$ is bounded
    sequence in $\mathbb{C}$ for any fixed $x\in\ell_\infty$. Therefore we have
    a well defined limit $f(x)=\lim_{\mathcal{U}}f_n(x)$ along an ultrafilter
    $\mathcal{U}$ domintaiting section filter on $\mathbb{N}$. Since
    $\mathcal{U}$  dominates section filter on $\mathbb{N}$, so
    $f(x)=\lim_{n\to\infty}f_n(x)=\lim_{n\to\infty} x(n)$ for all $x\in c$.

    $(ii)$ Taking the limit along $\mathcal{U}$ in $(ii')$ we get that
    $|f(x)|\leq\Vert x\Vert$ for all $x\in\ell_\infty$, i.e. $\Vert f\Vert\leq
        1$. Since $f$ extend and $f_0$, and $\Vert f_0\Vert=1$, then $\Vert
        f\Vert=1$. Again taking the limit along $\mathcal{U}$ in the second
    inequaity of $(ii')$ we get $\liminf_{n\to\infty} x(n)-\varepsilon< f(x)
        \leq\limsup_{n\to\infty}x(n)+\varepsilon$ for all $\varepsilon>0$. Since
    $\varepsilon>0$ is arbitrary we get the desired inequality.

    $(iii)$ Since $\mathcal{U}$ dominates section filter on $\mathbb{N}$ we have
    $f(S(x))-f(x)=\lim_{\mathcal{U}}(f_n(S(x))-f_n(x))
        =\lim_{n\to\infty}(f_n(S(x))-f_n(x))=0$.

    $(iv)$ Again taking the limit along $\mathcal{U}$ in inequality of $(iv')$
    we get $f(x)\geq 0$ for all $x\geq 0$, $x\in\ell_\infty$.
\end{proof}

\begin{remark}\label{RemValueOfTheBanachLimit} Usually it is impossible to find
    the value of a Banach limit for a given sequence in $\ell_\infty$. But there
    could be exceptions. Consider sequence $x\in\ell_\infty(\mathbb{N})$ given
    by the formula $x(n)=(1+{(-1)}^n)/2$. Clearly, $x+S(x)=1_{\mathbb{N}}$, so
    for any Banach limit $f$ we have
    $1=f(1_{\mathbb{N}})=f(x+S(x))=f(x)+f(S(x))=2f(x)$. Therefore, $f(x)=1/2$.
\end{remark}

Now we need to remind a well known definition from topology

\begin{definition}\label{DefStoneCechCompactificationOfDiscerteSet} Let $S$ be a
    discrete set. Then by $\beta S$ we denote the set of ultrafilters on $S$.
    The prebase of the topology of $\beta S$ is given by $\{\mathcal{F}\in\beta
        S:A\notin\mathcal{F}\}$ for some $A\in\mathcal{P}(S)$.
\end{definition}

One can show that

$(i)$ $\beta S$ is an extremelly disconnected compact Hausdorff topological
space

$(ii)$ $S$ may be identified with the set of fixed ultrafilters on $S$ and this
set is dense in $\beta S$

$(iii)$ $\beta$ is freedom functor from the category of descrete spaces into the
category of extremelly disconnected compact Hausdorff topological spaces.

\begin{proposition} The spectrum of the commutative Banach algebra
    $\ell_\infty(S)$ is homeomorphic to $\beta S$.
\end{proposition}
\begin{proof} Take any ultrafilter $\mathcal{U}\in\beta S$, then we have a well
    defined bounded character
    $f:\ell_\infty(S)\to\mathbb{C}:x\mapsto\lim_{\mathcal{U}}x(s)$. It remains
    to show that any bounded character is of this form. Let $f$ be a nonzero
    multiplicative functional on $\ell_\infty(S)$. Since
    $f(1_S)=f(1_S^2)={f(1_S)}^2$, we get that $f(1_S)=1$ (it cannot be zero,
    because then $f=0$). Now let $a\in\ell_\infty(S)$ such that $a(s)\in
        \{0,1\}$ for all $s\in S$. Write $\alpha=f(a)$. As $a(1-a)=0$, we have
    $0=f(a(1-a))=f(a)f(1-a)=\alpha(1-\alpha)$. So either $\alpha=0$ or
    $\alpha=1$. Note that we can write $a=1_A$ where $A=\{s\in S: a(s)=1\}$. Now
    define $ \mathcal U=\{A\in\mathcal{P}(S):\ f(1_A)=1\}$. In fact, $\mathcal
        U$ is an ultrafilter. Indeed,

    $(i)$ $S\in\mathcal U$ (since $f(1_S)=1$)

    $(ii)$ $A\in\mathcal{U}$ iff $S\setminus A\notin\mathcal{U}$ because $1_A
        1_{S\setminus A}=0$

    $(iii)$ If $A,B\in\mathcal{U}$, then $A\cap B\in\mathcal{U}$ because
    $1_{A\cap B}=1_A 1_B$

    $(iv)$ If $A\in\mathcal U$ and $A\subset B$, then $B\in\mathcal U$ because
    $1_A=1_A 1_B$
    \newline
    Now let $c\in\ell_\infty$ be positive, i.e. $0\leq c\leq 1$. Define sets
    $$
        A_j^{(n)}=\left \{s\in S:
        \ \frac{j}{2^n}\leq c(s)<\frac{(j+1)}{2^n}\right \}
    $$
    for $j=\{0,\ldots,2^n-1\}$. For a given $s\in S$, these sets are pairwise
    disjoint and $\bigcup_{j=0}^{2^n-1}A_j^{(n)}=S\in\mathcal{U}$. As
    $\mathcal{U}$ is an ultrafilter, for each $n\in\mathbb{N}$ there is exactly
    one $j(n)\in\mathbb{N}$ such that $A_{j(n)}^{(n)}\in\mathcal U$, and none of
    the others is. Define
    $$
        c_n=\sum_{j=0}^{2^n-1}\,\frac{j}{2^n}\,1_{A_j^{(n)}}.
    $$
    By construction, $\Vert c-c_n\Vert\leq 2^{-n}$, so $c_n\to c$ in
    $\ell_\infty(S)$. As $f$ is continuous, we have
    $$
        f(c)
        =\lim_{n\to\infty} f(c_n)
        =\lim_{n\to\infty}\sum_{j=0}^{2^n-1}\frac{j}{2^n}f(1_{A_j^{(n)}})
        =\lim_{n\to\infty}\frac{j(n)}{2^n},
        =\lim_{n\to\infty} c(j(n))
        =\lim_{\mathcal U}\ c(n).
    $$
    The last step is to extend $f$ by linearity to all of $\ell_\infty(S)$.
    Therefore we showed that
    $$
        \Phi:\operatorname{Spec}(\ell_\infty(S))\to \beta S:f\mapsto \mathcal{U}
    $$
    is a bijection. We claim this is a homeomorphism. Take any element $G$ of
    prebase of the topology of $\beta S$, then $G=\{\mathcal{F}\in\beta S:
        A\notin\mathcal{F}\}$ for some $A\in\mathcal{P}(S)$. As was shown above
    $A\in\mathcal{U}\in\beta S$ iff $f(1_A)=1$ for $f=\Phi^{-1}(\mathcal{U})$.
    Therefore $\Phi^{-1}(G)=\{f\in\operatorname{Spec}(\ell_\infty(S)):
        f(1_A)\neq 1\}$ which is open in
    $({\ell}_\infty(S),\sigma({\ell_\infty(S)}^*,\ell_\infty(S)))$ and therefore
    in $\operatorname{Spec}(\ell_\infty(S))$. Since $G$ is an arbitrary element
    of prebase of the topology of $\beta S$, then $\Phi$ is continuous. Since
    $\operatorname{Spec}(\ell_\infty(S))$ and $\beta S$ are Hausdorff compacts
    and $\Phi$ is a bijection, then $\Phi$ is a homeomorphism.
\end{proof}

This correspondence is of use in the measure theory too. For example, one can
check that for a given ultrafilter $\mathcal{U}$, the map
$$
    \mu:\mathcal{P}(\mathbb{N})\to\mathbb{R}:A\mapsto
    \begin{cases} 1
         & \quad\mbox{ if } A\in\mathcal{U} \\ 0&\quad\mbox{ otherwise }
    \end{cases}
$$
is a finitely additive measure on $\mathbb{N}$. The functional $f\in
    {L_\infty(\mathbb{N}, \mu)}^*$ corresponding to this measure is just the
limit along the ultrafilter $\mathcal{U}$.

Another example of a finitely additive measure is as follows. For example, for a
given $A\in\mathcal{P}(\mathbb{N})$ define
$d_n(A)=\frac{1}{n}\operatorname{Card}(A\cap \{1,\ldots,n\})$. By
propositions~\ref{CorClusterPintAreLimitsAlongUltrafilters}
and~\ref{PrClusterPointAlongTheFilterCharac} the set $\bigcap_{S\in\mathcal{U}}
    \operatorname{cl}_{\mathbb{R}}(\{d_n(A):n\in\mathcal{S}\})$ is a singleton.
Hence it is of the form $\{\mu(A)\}$. We claim that
$$
    \mu:\mathcal{P}(\mathbb{N})\to\mathbb{R}:A\mapsto\mu(A)
$$
is a finitely additive measure on $\mathbb{N}$. It is easy to understand if one
notes that this measure correspond to the Banach limit on ${\ell}_\infty$
constructed in proposition~\ref{PrBanachLimitExists}.

Now we proceed to the one of the numerous applications of ultrafilters in the
local theory of Banach spaces.

\begin{proposition} Let $E$ be a Banach space, such that for any its
    $n$-dimensional subspace $F$ there exists an isomorphism $T:F\to \ell_2^n$
    with the property $\Vert T\Vert\Vert T^{-1}\Vert\leq C$. Then $E$ is
    isomorphic to some Hilbert space.
\end{proposition}
\begin{proof}
    We can find a family of linearly independent vectors
    $\{x_\lambda:\lambda\in\Lambda \}$  such that we have the following
    representation
    $E=\mathrm{cl}_E\left(\mathrm{span}\{x_\lambda:\lambda\in\Lambda \}\right)$.
    Denote $E_S=\mathrm{span}\{x_\lambda:\lambda\in S\}$ for finite subset $S$
    in $\Lambda$, then
    $$
        E=\mathrm{cl}_E(E_\infty)\quad\text{where}\quad
        E_\infty=\bigcup\limits_{S\in\mathcal{P}_0(\Lambda)} E_S
    $$
    Fix $S\in\mathcal{P}_0(\Lambda)$, then by assumption there exists an
    operator $T_S:E_S\to\ell_2^n$ (where $n=\operatorname{Card}(S)$) such that
    $\Vert T_S\Vert\Vert T_S^{-1}\Vert\leq C$. After suitable rescaling of $T_S$
    we can assume that $\Vert T_S\Vert\leq 1$ and $\Vert T_S^{-1}\Vert< C$.
    Consider function
    $$
        \langle\cdot, \cdot\rangle_{E_S}:
        E_S\times E_S\to\mathbb{R}:(x,y)\mapsto
        \langle T_S(x),T_S(y)\rangle_{\ell_2^n}
    $$
    Since $T_S$ is an isomorphism this map is inner product, and what is more
    \begin{equation}\label{EquivNorm}
        C^{-2}\Vert x\Vert^2\leq \langle x, x\rangle_{E_S}\leq\Vert x\Vert^2
    \end{equation}
    As the strange consequence for a fixed $x\in E_\infty$ the sequence
    $\{\langle x, x\rangle_{E_S}:S\in\mathcal{P}_0(\Lambda)\}$ is a subset of
    Hausdorff compact $[0, \Vert x\Vert^2]\subset\mathbb{R}$.

    On a directed set $(\mathcal{P}_0(\Lambda),\subset)$ with standard ordering
    consider respective section filter $\mathcal{F}$ and an ultrafilter
    $\mathcal{U}$ dominating $\mathcal{F}$. Define the map
    $$
        \Vert\cdot\Vert_{E_\infty} :E_\infty\to\mathbb{R}:
        x\mapsto\lim\limits_{\mathcal{U}}\langle x, x\rangle_{E_S}^{1/2}
    $$
    It is well defined becasuse limit along ultrafiler for any sequence
    contained in a Hausdorff compact exists and unique.

    One can check that $\Vert\cdot\Vert_{E_\infty}$ is a norm satisfying
    parallelogram law. By Jordan von Neumann theorem we have well defined inner
    product
    $$
        \langle\cdot,\cdot\rangle_{E_\infty}:E_\infty\times
        E_\infty\to\mathbb{C}:(x,y)\mapsto
        \sum\limits_{k=1}^4\frac{i^k}{4}\Vert x+i^k y\Vert_{E_\infty}
    $$
    Since $E=\mathrm{cl}_E(E_\infty)$, there is continuous extension
    $\langle\cdot,\cdot\rangle_E$ of $\langle\cdot,\cdot\rangle_{E_\infty}$ to
    the inner product on the whole $E$. Now from~\ref{EquivNorm} it follows that
    norm $\Vert\cdot\Vert_E$ induced by $\langle\cdot,\cdot\rangle_E$ is
    equivalent to the original norm of $E$. Hence identity map
    $$
        1_E:(E,\Vert\cdot\Vert)\to(E,\Vert\cdot\Vert_E):x\mapsto x
    $$
    gives the desired isomorphism.
\end{proof}

\newpage
\begin{thebibliography}{999}
    \bibitem{SlezFConvFiltNets}
    \textit{Sleziak M.} $\mathcal{F}$-convergence, filters and nets.
    \bibitem{BourbTopGen}
    \textit{Bourbaki N. } Eléments de mathématiques. Hermann, 1963.
    \bibitem{EngelGenTop}
    \textit{Engelking R.} General topology. Heldermann, 1989.
\end{thebibliography}
\end{document}
