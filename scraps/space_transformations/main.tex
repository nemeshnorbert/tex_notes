\documentclass[12pt]{article}
\usepackage[utf8]{inputenc}
\usepackage[russian]{babel}
\usepackage{amssymb}
\usepackage{amsmath}
\usepackage{mathrsfs}
\usepackage{esvect}
\usepackage[left=2cm, right=2cm, top=2cm, bottom=2cm,
    bindingoffset=0cm]{geometry}
\usepackage[colorlinks=true, urlcolor=blue, linkcolor=blue, citecolor=blue,
    pdfborder={0 0 0}]{hyperref}
\usepackage{enumitem}

\hypersetup{frenchlinks=true}

\newtheorem{theorem}{Theorem}[section]
\newtheorem{lemma}[theorem]{Лемма}
\newtheorem{proposition}[theorem]{Предложение}
\newtheorem{remark}[theorem]{Замечание}
\newtheorem{corollary}[theorem]{Следствие}
\newtheorem{definition}[theorem]{Определение}
\newtheorem{example}[theorem]{Пример}

\newenvironment{proof}{\par $\triangleleft$}{$\triangleright$}

\pagestyle{plain}

\begin{document}

\begin{center}

    \Large \textbf{Преобразования систем координат}\\[0.5cm]
    \small {Немеш Н. Т.}\\[0.5cm]

\end{center}
\date{March 2022}


\section{Системы координат и трансформы}

В этой заметке обсудим преобразования между системами координат. Мы будем использовать только
трехмерное пространство $\mathbb{R}^3$. Строго говоря, мы рассматриваем $\mathbb{R}^3$
как аффинное пространство.

\begin{definition}
    Система координат $A$ -- это выделенная точка в пространстве $O_A\in\mathbb{R}^3$
    называемая началом координат и некоторый ортонормированный базис $\mathcal{B}_A$
    с началом в точке $O_A$. Обозначение: $A=(O_A, \mathcal{B}_A)$.
\end{definition}

\begin{remark}
    В определении выше мы намеренно ограничились
    ортонормированными базисами, потому что другие нам не понадобятся.
\end{remark}

\begin{definition}
    Пусть $A=(O_A,\mathcal{B}_A)$ и $B=(O_B,\mathcal{B}_B)$ -- две системы координат.
    Пусть $Q_A^B$ -- матрица перехода из базиса $\mathcal{B}_B$ в базис $\mathcal{B}_A$,
    пусть $p_A^B$ -- координаты точки $\vv{p}:=\vv{O_BO_A}$ в базисе $\mathcal{B}_B$.
    Тогда трасформом из $A$ в $B$ называется преобразование
    $$
        T_A^B:\mathbb{R}^3\to\mathbb{R}^3: x^A \mapsto Q_A^B x^A + p_A^B,
    $$
    где $Q_A^B$ -- матрица вращения размера $3\times 3$ и $p_A^B$ -- трехмерный вектор. 
    Обозначение: $T_A^B=(Q_A^B,p_A^B)$.
\end{definition}

\begin{remark}
    Трансформы позволяют найти координаты вектора в одной системе координат
    по координатам вектора в другой системе коодинат. Действительно, если $T_A^B$ -- трансформ из
    системы координат $A$ в систему координат $B$, а $\vv{x}$ -- произвольный вектор в $\mathbb{R}^3$,
    то
    $$
        x^B=T_A^B(x^A)
    $$
    При этом, чтобы задать трансформ из $A$ в $B$ надо знать матрицу перехода из $B$ в $A$.
\end{remark}

\begin{proposition}
    Пусть $A$, $B$ и $C$ -- три системы координат и нам даны
    трансформ $T_A^B=(Q_A^B,p_A^B)$ из $A$ и $B$ и трансформ $T_B^C=(Q_B^C,p_B^C)$ из $B$ в $C$.
    Тогда, трансформ из $A$ в $C$ дается формулой
    $$
        T_A^C=(Q_A^C, p_A^C)=(Q_B^C Q_A^B, p_B^C+Q_B^C p_A^B)
    $$
\end{proposition}
\begin{proof}
    Пусть $\vv{x}\in\mathbb{R}^3$ -- произвольный вектор. Тогда c одной стороны
    $$
        T_A^C(x^A)=Q_A^C x^A+p_A^C
    $$
    и с другой
    $$
        T_A^C(x^A)=T_B^C(T_A^B(x^A))
    $$
    Находим
    $$
        T_A^C(x^A)=T_B^C(T_A^B(x^A))
            =T_B^C(Q_A^B x^A+p_A^B)
            =Q_B^C(Q_A^B x^A+p_A^B)+p_B^C
            =Q_B^C Q_A^B x^A + Q_B^C p_A^B + p_B^C
    $$
    Поскольку равенство
    $$
        T_A^C(x^A)=Q_A^C x^A+p_A^C=Q_B^C Q_A^B x^A + Q_B^C p_A^B + p_B^C
    $$
    верно для любого $\vv{x}\in\mathbb{R}^3$, то
    $$
        Q_A^C=Q_B^C Q_A^B, \quad\quad p_A^C=p_B^C+Q_B^C p_A^B
    $$
\end{proof}

\begin{corollary}
    Пусть $A$ и $B$ -- две системы координат и $T_A^B$ -- трансформ из $A$ в $B$.
    Тогда трансформ из $B$ в $A$ имеет вид
    $$
        T_B^A=((Q_A^B)^T,-Q_A^B p_A^B)
    $$
\end{corollary}
\begin{proof}
    Полагая, $C=A$ в предыдущем предложении
    получим, что
    $$
        Q_A^A=Q_B^A Q_A^B, \quad\quad p_A^A=p_B^A+Q_B^A p_A^B
    $$
    Поскольку $Q_A^A=E$, $p_A^A=0$, то
    $$
        Q_B^A=(Q_A^B)^{-1}=(Q_A^B)^T \quad \quad p_B^A=-Q_B^A p_A^B
    $$
\end{proof}


\section{Справочные материалы}

\begin{definition}
    Базис $\mathcal{B}$ это упорядоченный набор из трех линейно независимых векторов
    $\vv{e}_1, \vv{e}_2, \vv{e}_3$.
    Обозначение $\mathcal{B}=(\vv{e}_1, \vv{e}_2, \vv{e}_3)$.
\end{definition}

\begin{definition}
    Базис назваются ортонормированным если все его векторы имеют длину 1
    и попарно перпендикулярны.
\end{definition}

\begin{remark}
    Основное свойство базиса -- любой вектор $\vv{x}\in\mathbb{R}^3$ может быть однозначно
    представлен в виде линейной комбинации
    $$
    \vv{x}=x_1 \vv{e_1}+x_2 \vv{e_2} + x_3 \vv{e_3}.
    $$
    Упорядоченная тройка чисел $(x_1,x_2,x_3)$ назвается координатами вектора $\vv{x}$
    в базисе $\mathcal{B}$. Обозначение:
    $$
        x^A:=\begin{bmatrix}x_1 \\ x_2 \\ x_3\end{bmatrix}
    $$
\end{remark}

\begin{definition}
    Пусть $\mathcal{B}_A$ и $\mathcal{B}_B$ -- два базиса. Матрицей перехода от
    базиса $\mathcal{B}_A$ к $\mathcal{B}_B$ называется матрица в которой по столбцам выписаны
    координаты векторов базиса $\mathcal{B}_B$ в базисе $\mathcal{B}_A$. Будем обозначать такую
    матрицу через $C_B^A$.
\end{definition}

\begin{example}
    Пусть базис $\mathcal{B}_A=(\vv{a}_1,\vv{a}_2,\vv{a}_3)$ состоит из
    единичных векторов направленных вдоль координатных осей $X$, $Y$, $Z$. Тогда
    $$
        a_1^A=\begin{bmatrix}1 \\ 0 \\ 0\end{bmatrix},
        a_2^A=\begin{bmatrix}0 \\ 1 \\ 0\end{bmatrix},
        a_3^A=\begin{bmatrix}0 \\ 0 \\ 1\end{bmatrix},
    $$
    Пусть базис $\mathcal{B}_B=(\vv{b}_1,\vv{b}_2,\vv{b}_3)$ получен из базиса $\mathcal{B}_A$
    поворотом вокруг оси $Z$ на угол
    $\alpha$ против часовой стрелки. Тогда
    $$
        b_1^A=\begin{bmatrix}\cos\alpha \\ \sin\alpha \\ 0\end{bmatrix},
        b_2^A=\begin{bmatrix}-\sin\alpha \\ \cos\alpha \\ 0\end{bmatrix},
        b_3^A=\begin{bmatrix}0 \\ 0 \\ 1\end{bmatrix}
    $$
    Как следствие, матрица перехода $C_B^A$ будет
    $$
        C_B^A=\begin{bmatrix}
            \cos\alpha & -\sin\alpha & 0 \\
            \sin\alpha & \cos\alpha  & 0 \\
            0          & 0           & 1
        \end{bmatrix}
    $$
\end{example}

\begin{remark}
    С помощью матрицы перехода можно находить координаты вектора в разных базисах. Пусть
    $\mathcal{B}_A$, $\mathcal{B}_B$ -- два базиса и $\vv{x}\in\mathbb{R}^3$ -- произвольный
    вектор. Тогда,
    $$
        x^A=C_B^A x^B
    $$
\end{remark}

\begin{remark}
    Если базисы $\mathcal{B}_A$ и $\mathcal{B}_B$ ортонормированы, то чтобы акцентировать
    на этом внимание для матрицы перехода $C_B^A$ мы будем использовать обозначение
    $$
        Q_B^A:=C_B^A
    $$
\end{remark}

\begin{remark}
    Матрицы перехода $Q$ между ортонормированными базисами обладают свойством
    $$
        Q^T Q=Q Q^T=E
    $$
    Такие матрицы называются ортогональными. Если $Q$ ортогональная матрица, то $Q^T$ тоже ортогональная.
    У ортогональных матриц легко находить обратную
    $$
        Q^{-1}=Q^T
    $$
    Более того $\det(Q)=\pm 1$.
\end{remark}

\begin{remark}
    Всякое ортональное преобразование сохраняет скалярное произведение. Ортогональные матрицы
    с определителем равным 1 вдобавок к этому сохраняют ориентацию пространства.
    Любое вращение трехмерного пространства и только оно задается ортогональной
    матрицей с определителем 1.
\end{remark}

\begin{definition}
    Ортогональные матрицы с определителем 1 мы будем обозначать буквой $R$ и будем
    называть их матрицами вращения.
\end{definition}
\end{document}