% chktex-file 35
\documentclass[12pt]{article}
\usepackage[left=2cm,right=2cm,top=2cm,bottom=2cm,bindingoffset=0cm]{geometry}
\usepackage{amssymb}
\usepackage{amsmath}
\usepackage{amsthm}
\usepackage{enumerate}
\usepackage[T1,T2A]{fontenc}
\usepackage[utf8]{inputenc}
% \usepackage[russian]{babel}
\usepackage[matrix,arrow,curve]{xy}
\usepackage[colorlinks=true, urlcolor=blue, linkcolor=blue, citecolor=blue,
    pdfborder={0 0 0}]{hyperref}

%-------------------------------------------------------------------------------
\newtheorem{theorem}{Theorem}[section]
\newtheorem{lemma}[theorem]{Lemma}
\newtheorem{proposition}[theorem]{Proposition}
\newtheorem{remark}[theorem]{Remark}
\newtheorem{corollary}[theorem]{Corollary}
\newtheorem{definition}[theorem]{Definition}
\newtheorem{example}[theorem]{Example}

\newcommand{\projtens}{\mathbin{\widehat{\otimes}}}
\newcommand{\convol}{\ast}
\newcommand{\projmodtens}[1]{\mathbin{\widehat{\otimes}}_{#1}}
\newcommand{\isom}[1]{\mathop{\mathbin{\cong}}\limits_{#1}}
%-------------------------------------------------------------------------------

\begin{document}

\begin{center}
    \Large \textbf{A note on relatively injective $C_0(S)$-modules
        $C_0(S)$}\\[0.5cm]
    \small {N. T. Nemesh}\\[0.5cm]
\end{center}

\thispagestyle{empty}

\medskip
\textbf{Abstract:} In this note we discuss some necessary and some sufficient
conditions for relative injectivity of the $C_0(S)$-module $C_0(S)$, where $S$
is a locally compact Hausdorff space. We also give a Banach module version of
Sobczyk's theorem. The main result of the paper is as follows: if
$C_0(S)$-module $C_0(S)$ is relatively injective then the equality
$S=\beta(S\setminus \{s\})$ holds for any limit point $s\in S$.
\medskip

\textbf{Keywords:} injective Banach module, $C_0(S)$-space, almost compact
space.

\bigskip

%-------------------------------------------------------------------------------
%	Introduction
%-------------------------------------------------------------------------------

\section{Introduction}\label{SectionIntroduction}

Extension problems have been an important topic of functional analysis since its
inception. The first example of successfully solved extension problem is the
Hahn-Banach
theorem~\cite{HahnLinSystInLinSp,BanachOnLinFuncI,BanachOnLinFuncII}. In modern
terms this theorem states that the field of complex numbers is an injective
object in the category of Banach spaces. All known injective Banach spaces are
isomorphic to the space of continuous functions on some compact
space~\cite{BlascIvorConstrInjSpCK}. This fact motivates our study of
injectivity of spaces of continuous functions, but this time we consider them as
Banach modules.

%-------------------------------------------------------------------------------
%	Preliminaries
%-------------------------------------------------------------------------------

\section{Preliminaries}\label{SectionPreliminaries} Before we proceed to the
main topic we shall give a few definitions and notations.

Let $M$ be a subset of a set $N$, then $\chi_M$ denotes the indicator function
of $M$. If $f:N\to L$ is an arbitrary function, then $f|_M$ denotes its
restriction to $M$.

Let $S$ be a Hausdorff topological space. The space $S$ is called
\textit{extremally disconnected} if any open subset of $S$ has open closure;
\textit{Stonean} if it is extremally disconnected and compact;
\textit{pseudocompact} if any continuous function on $S$ is bounded. If $S$ is
non-compact locally compact, then by $\alpha S$ we denote the
\textit{Alexandroff's compactification} of $S$, and by $\beta S$ we denote
\textit{Stone-Cech compactification} of $S$. The \textit{Stone-Cech remainder}
$\beta S\setminus S$ we denote by $S^*$. A non-compact Hausdorff topological
space $S$ is called \textit{almost compact} if $\alpha S=\beta S$. A typical
example of almost compact space is $[0, \omega_1)$, % chktex 9
where $\omega_1$ is the first uncountable ordinal~\cite[paragraph
    1.3]{HrusPsdCompTopSp}. More on extremally disconnected, pseudocompact and amost
compact spaces can be found in~\cite[section 6.2]{EngkingGenTop},~\cite[section
    3.10]{EngkingGenTop} and~\cite[paragraph 1.3]{HrusPsdCompTopSp} respectively.

For a given non-compact locally compact Hausdorff space $S$ we can consider
filter base $\mathcal{B}_S$ consisting of complements of compact subsets of $S$.
The filter $\mathcal{F}_S$ generated by $\mathcal{B}_S$ is called the
\textit{Frechet filter} on $S$. Now we can introduce several functional spaces
on $S$. By $C(S)$ we denote the space of continuous functions on $S$. This space
is normable if $S$ is compact. By $C_b(S)$ we denote the Banach space of
continuous bounded functions on $S$. Symbol $C_l(S)$ denotes the space of
continuous functions that converge to a finite limit along filter
$\mathcal{F}_S$. By $C_0(S)$ we denote the closed subspace of $C_l(S)$ of
functions that converge to zero along $\mathcal{F}_S$ (we also say that these
functions vanish at infinity). If $S$ is compact all these spaces coincide with
$C(S)$.

Let $A$ be a Banach algebra. We shall consider only right Banach modules over
$A$ with contractive outer action $\cdot:X\times A\to X$. Let $X$ and $Y$ be two
right Banach $A$-modules, then a map $\phi:X\to Y$ is an \textit{$A$-morphism}
if it is a continuous $A$-module map. Banach $A$-modules and $A$-morphisms form
a category which we denote by $\mathbf{mod}-A$.

In $\mathbf{mod}-A$ the notion of injectivity can be defined in different ways.
Let $\xi:X\to Y$ be an $A$-morphism. Then it is called \textit{relatively
    admissible} if $\eta\circ \xi=1_X$ for some bounded linear operator $\eta:Y\to
    X$; \textit{topologically admissible} if $\xi$ is a linear homeomorphism on its
image; \textit{metrically admissible} if $\xi$ is isometric. A Banach $A$-module
$J$ is called \textit{relatively injective} (resp. \textit{topologically
    injective}, resp. \textit{metrically injective}) if for any relatively (resp.
topologically, resp.\ metrically) admissible $A$-morphism $\xi:X\to Y$ and any
$A$-morphism $\phi:X\to J$ there exists a continuous (resp.\ continuous, resp.
continuous with the same norm as $\phi$) $A$-module map $\psi:Y\to J$ making the
diagram
$$
    \xymatrix{
    & {Y} \\
    {J} \ar@{<--}[ur]^{\psi} &{X} \ar[u]^{\xi} \ar[l]^{\phi}}  % chktex 3
$$
commutative.

If $A=\{0\}$, the category $\mathbf{mod}-A$ turns into the ordinary category of
Banach spaces. In this case all Banach spaces are relatively injective. As for
topologically and metrically injective Banach spaces, in the standard literature
they are called \textit{$\mathcal{P}_\lambda$-spaces} and
\textit{$\mathcal{P}_1$-spaces} respectively. To this day there is no clear
description of $\mathcal{P}_\lambda$-spaces~\cite[page vi]{AvilSepInjBanSp}, but
for $\mathcal{P}_1$-spaces the question is closed. Any $\mathcal{P}_1$-space is
isometrically isomorphic to $C(S)$-space, where $S$ is a Stonean
space~\cite{HasumExtPropCompBanSp}.

%-------------------------------------------------------------------------------
%   Necessary and sufficient conditions of injectivity
%-------------------------------------------------------------------------------

\section{Necessary and sufficient conditions for
  injectivity}\label{SecionNecessaryAndSufficientConditionsForInjectivity}

In this section we shall discuss necessary and sufficient conditions for the
relative injectivity of $C_0(S)$-module $C_0(S)$, where $S$ is a locally compact
Hausdorff space. We start from a quite restrictive sufficient condition.

\begin{proposition}\label{SStonImplRelInjCSModCS} Let $S$ be a Stonean space.
    Then $C(S)$-module $C(S)$ is relatively injective.
\end{proposition}
\begin{proof} Denote $A=C(S)$. Since $S$ is Stonean, then $A$ is an
    $AW^*$-algebra~\cite[section 1, paragraph 7]{BerbBaerStRng}. It was shown
    in~\cite[theorem 2]{TakHanBanThAndJordDecomOfModMap} that any $AW^*$-algebra
    is a metrically injective bimodule over itself. Careful inspection of the
    proof shows that the same argument is valid for the right $A$-module $A$. It
    remains to recall that any metrically injective module is relatively
    injective.
\end{proof}

To give a rather burdensome necessary condition for relative injectivity of
$C(S)$-module $C(S)$ we start from somewhat specific case.

\begin{proposition}\label{RelInjCaSModCaSImplSAlmComp} Let $S$ be a non-compact
    locally compact Hausdorff space. Suppose $C(\alpha S)$-module $C(\alpha S)$
    is relatively injective. Then $S$ is almost compact.
\end{proposition}
\begin{proof} Obviously $C(\alpha S)$ and $C_l(S)$ are isometrically isomorphic
    as Banach algebras, so $C_l(S)$-module $C_l(S)$ is relatively injective.
    Note that $C_0(S)$ is a two sided ideal of $C_l(S)$ consisting of functions
    that vanish at infinity. This ideal is complemented via projection
    $P:C_l(S)\to C_0(S):x\mapsto x-(\lim_{\mathcal{F}_S}x(s))\chi_{S}$. Consider
    an isometric embedding $\xi:C_0(S)\to C_l(S):x\mapsto x$ which is a
    $C_l(S)$-morphism. Since $P\circ\xi=1_{C_0(S)}$, then $\xi$ is relatively
    admissible.

    Fix $f \in C_b(S)$ and consider $C_l(S)$-morphism $\phi:C_0(S)\to
        C_l(S):x\mapsto f \cdot x$. Since $C_l(S)$ is relatively injective
    $C_l(S)$-module, then there exists a $C_l(S)$-morphism $\psi:C_l(S)\to
        C_l(S)$ such that $\phi=\psi\circ\xi$. In particular, for all $x\in
        C_0(S)$ we have $f\cdot x=\phi(x)=\psi(\xi(x))=\psi(x)=\psi(x\cdot
        \chi_{S})=x\cdot \psi(\chi_{S})$. Fix $s\in S$. Since $S$ is locally
    compact and Hausdorff, then by~\cite[corollary 3.3.3]{EngkingGenTop}
    there exists a continuous function $e\in C_0(S)$ such that $e(s)=1$.
    Hence, $f(s)=f(s)e(s)=(f\cdot
        e)(s)=(e\cdot\psi(\chi_{S}))(s)=e(s)\psi(\chi_{S})(s)=\psi(\chi_{S})(s)$.
    Since $s\in S$ is arbitrary $f=\psi(\chi_{S})$. By construction
    $\psi(\chi_S)\in C_l(S)$, so $f\in C_l(S)$. Recall that $f\in C_b(S)$ is
    arbitrary, so $C_b(S)\subset C_l(S)$. This is possible only if
    $C_b(S)=C_l(S)$.

    Note that in the category of Banach spaces $C_b(S)$ is isometrically
    isomorphic to $C(\beta S)$, and $C_l(S)$ is isometrically isomorphic to
    $C(\alpha S)$. So we conclude that Banach spaces $C(\beta S)$ and $C(\alpha
        S)$ are isometrically isomorphic. By Banach-Stone theorem~\cite[theorem
        83]{StoneBanStAppBoolRngToTop} the spaces $\alpha S$ and $\beta S$ are
    homeomorphic. Therefore $S$ is almost compact.
\end{proof}

Now its time to introduce yet another notion of compactness.

\begin{definition} A compact Hausdorff space $S$ is called uniformly compact if
    for every limit point $s\in S$ the space $S\setminus \{s\}$ is almost
    compact.
\end{definition}

In other words a compact Hausdorff space $S$ is uniformly compact if for every
limit point $s\in S$ we have $S=\beta(S\setminus \{s\})$.

\begin{proposition}\label{StoneSpUnifComp} Stonean spaces are uniformly compact.
\end{proposition}
\begin{proof} Let $S$ be a Stonean space and $s\in S$ be its limit point. Denote
    $S_\circ=S\setminus \{s\}$. Since $S$ is compact and $S_\circ$ is its open
    subset then $S_\circ$ is locally compact. As $s$ is a limit point in $S$,
    then space $S_\circ$ is non-compact and $\alpha S_\circ=S$. Let $A,B\subset
        S_\circ$ be two completely separated subsets. By definition it means that
    there exists a continuous function $f:S_\circ\to\mathbb[0,1]$ such that
    $f|_A=\{0\}$ and $f|_B=\{1\}$. Consider disjoint open sets
    $U=f^{-1}([0,1/3))\subset S_\circ$ and  % chktex 9
    $V=f^{-1}((1/2,1])\subset S_\circ$.  % chktex 9
    Clearly, $U$ and $V$ are open in $S$. Since $S$ is extremally disconnected
    then $U$ and $V$ have disjoint closures in $S$. As $A\subset U$, $B\subset
        V$, the sets $A$ and $B$ also have disjoint closures in $S$. By
    theorem~\cite[theorem 3.6.2]{EngkingGenTop} we get that $\beta
        S_\circ=\alpha S_\circ=S$.
\end{proof}

\begin{corollary}\label{MesSpUnifCompIffFin} A metrizable space is uniformly
    compact if and only if it is finite.
\end{corollary}
\begin{proof} Let $S$ be a metrizable space with topology induced by metric $d$.
    Suppose that $S$ is uniformly compact. Assume that $S$ has a limit point
    $s\in S$. Then the space $S_\circ=S\setminus \{s\}$ is almost compact and
    therefore pseudocompact~\cite[proposition 1.3.10]{HrusPsdCompTopSp}.
    Consider continuous function $f:S_\circ\to\mathbb{R}:x\mapsto
        {d(x,s)}^{-1}$. This function is unbounded because $s\in S$ is a limit
    point. Therefore $S_\circ$ is not pseudocompact. Contradiction. Thus $S$ is
    a compact metric space without limit points, hence $S$ is finite.

    Conversely, if $S$ is finite it is vacuously uniformly compact.
\end{proof}

The following example is due to K. P. Hart.

\begin{proposition}\label{UcountProdCompSpIsUnifComp} The Tychonoff's product of
    an uncountable family of compact Hausdorff spaces that are not singletons is
    uniformly compact.
\end{proposition}
\begin{proof} Let $\mathcal{S}={(S_\lambda)}_{\lambda\in\Lambda}$ be a family of
    compact Hausdorff spaces. By Tychonoff's theorem their product $S$ is
    compact~\cite[theorem 3.2.4]{EngkingGenTop}. Let $s\in S$ be a limit point
    in $S$. Since the spaces $S_\lambda$ are not singletons for all
    $\lambda\in\Lambda$, then there exists a point $s'\in S$ such that
    $s_\lambda\neq s'_\lambda$ for all $\lambda\in\Lambda$. Let $\Sigma(s')$ be
    the $\Sigma$-product of $\mathcal{S}$ at point $s'$, that is $\Sigma(s')$
    consists of all points in $S$ that differ from $s'$ at at most countably
    many coordinates. By~\cite[exercise 3.12.24(c)]{EngkingGenTop} we have
    $S=\beta(\Sigma(s'))$. Since $\Sigma(s')\subset S\setminus \{s\}\subset
        S=\beta(\Sigma(s'))$, then by~\cite[corollary 3.6.9]{EngkingGenTop}
    $\beta(S\setminus \{s\})=\beta(\Sigma(s'))=S$. As $s\in S$ is an arbitrary
    limit point, then $S$ is uniformly compact.
\end{proof}

In some cases the property of being uniformly compact depends on the set of
axioms of the set theory. By \textsc{ZFC} we denote the standard
Zermelo-Fraenkel set theory together with the axiom of choice. By
\textsc{CH}$_n$ we denote the axiom that the cardinality of continuum equals the
$n$-th uncountable cardinal. Finally, \textsc{MA} stands for the Martin's axiom.
For details see~\cite{KunSetThIndepPrf}. On the one hand $\mathbb{N}^*$ is not
uniformly compact in \textsc{ZFC + CH$_1$}~\cite{FinGillExtContFuncbN}. On the
other hand, it is consistent with \textsc{ZFC + MA + CH$_2$} that $\mathbb{N}^*$
is uniformly compact~\cite{DouKunMillCStarEmbdDenPropSbspStoneRemN}.

We are ready to formulate the main result of the paper.

\begin{theorem}\label{RelInjCSModCSImplUnifCompS} Let $S$ be a locally compact
    Hausdorff space. If $C_0(S)$-module $C_0(S)$ is relatively injective, then
    $S$ is uniformly compact.
\end{theorem}
\begin{proof} Since $C_0(S)$-module $C_0(S)$ is relatively injective
    from~\cite[corollary 2.2.8 (i)]{RamsHomPropSemgroupAlg} we know that
    $C_0(S)$ has a left identity. Therefore $\chi_S\in C_0(S)$, hence $S$ is
    compact. Let $s$ be a non-isolated point in $S$ and $S_\circ=S\setminus
        \{s\}$. Since $\alpha S_\circ=S$, we see that $C(\alpha S_\circ)$-module
    $C(\alpha S_\circ)$ is relatively injective. By
    proposition~\ref{RelInjCaSModCaSImplSAlmComp} the space $S_\circ$ is almost
    compact. Since $s\in S$ is arbitrary the space $S$ is uniformly compact.
\end{proof}

\begin{corollary}\label{CSModCSRelInjImplSHasNoConvSeq} Let $S$ be a compact
    metrizable space. If $C(S)$-module $C(S)$ is relatively injective then $S$
    is finite.
\end{corollary}
\begin{proof} The result directly follows from
    theorem~\ref{RelInjCSModCSImplUnifCompS} and
    corollary~\ref{MesSpUnifCompIffFin}.
\end{proof}

Currently all known examples of locally compact spaces $S$ such that
$C_0(S)$-module $C_0(S)$ is relatively injective are extremally disconnected. It
would be interesting to get non-extremally disconnected examples, if they exist.
The first candidate is the space $\mathbb{N}^*$. It is not extremally
disconnected~\cite[example 6.2.31]{EngkingGenTop}, but it is uniformly compact
under certain set theoretic assumptions. However $C(\mathbb{N}^*)$ is not an
injective Banach space~\cite[corollary 2]{AmirProjContFuncSp}. Thanks to
proposition~\ref{UcountProdCompSpIsUnifComp} another possible candidate is an
uncountable power of the discrete space $\{0, 1\}$.

%-------------------------------------------------------------------------------
%   A module version of Sobczyk theorem
%-------------------------------------------------------------------------------

\section{A module version of Sobczyk's theorem}\label{SectionExamples}

In classical Banach space theory all infinite dimensional injective Banach
spaces are non-separable since all these spaces contain a copy of
$\ell_\infty(\mathbb{N})$~\cite[corollary 1.1.4]{RosOnRelDisjFamOfMeas}. Sobczyk
showed that $c_0$ is an injective space but among separable Banach
spaces~\cite[theorem 5]{SobProjmOnc0}. Later Zippin, showed  that all spaces
injective in the category of separable Banach spaces are isomorphic to
$c_0$~\cite{ZipSepExtProbm}.  Here we shall show that for any set $\Lambda$ the
$\ell_\infty(\Lambda)$-module $c_0(\Lambda)$ is relatively injective. Note that
by theorem~\ref{RelInjCSModCSImplUnifCompS} the $c_0(\Lambda)$-module
$c_0(\Lambda)$ is not relatively injective for infinite $\Lambda$.

Now we need to recall a few notions from Banach space theory. A bounded linear
operator $T$ is called \textit{weakly compact} if it maps bounded sets into
relatively weakly compact sets. A bounded linear operator $T$ is called
\textit{completely continuous} if it maps weakly convergent sequences into norm
convergent sequences.

A Banach space $E$ is called a \textit{Grothendieck space} if every weak$^*$
convergent sequence in $E^*$ converges weakly. Obviously all reflexive spaces
are Grothendieck spaces. A Banach space $E$ is called \textit{weakly compactly
    generated} if there is a weakly compact set $K\subset E$ whose linear span is
dense in $E$. Typical examples of weakly compactly generated spaces are
reflexive spaces and separable spaces~\cite[paragraph 13.1]{FabHabBanSpTh}.
Finally, we say that a Banach space $E$ has the \textit{Dunford-Pettis property}
if for any weakly null ${(f_n)}_{n\in\mathbb{N}}\subset E^*$ and any weakly null
sequence ${(x_n)}_{n\in\mathbb{N}}\subset E$ holds $\lim_{n\to\infty}
    f_n(x_n)=0$. For any compact space $S$ the Banach space $C(S)$ has the
Dunford-Pettis property~\cite{DunfPetLinOpSumFunc}.

\begin{proposition}\label{OpLInfc0CompContWeakComp} Any bounded linear operator
    $T:\ell_\infty(\Lambda)\to c_0(\Lambda)$ is weakly compact and even
    completely continuous.
\end{proposition}
\begin{proof} Note that $\ell_\infty(\Lambda)$ is isometrically isomorphic to
    $C(\beta\Lambda)$. Since $\beta\Lambda$ is a Stonean space, then
    by~\cite[theorem 9, p. 168]{GrothApplFabilCompCK} the space
    $\ell_\infty(\Lambda)$ is a Grothendieck space. The space $c_0(\Lambda)$ is
    weakly compactly generated~\cite[paragraph 13.1 example
        (iii)]{FabHabBanSpTh}. Then the operator $T$ is weakly
    compact~\cite[exercise 13.33]{FabHabBanSpTh}. Again, since the space
    $\ell_\infty(\Lambda)$ is a $C(S)$-space for $S=\beta\Lambda$, then it has
    the Dunford-Pettis property~\cite[theorem 13.43]{FabHabBanSpTh}. Therefore
    any weakly compact operator with domain $\ell_\infty(\Lambda)$ is completely
    continuous~\cite[proposition 13.42]{FabHabBanSpTh}.
\end{proof}

\begin{proposition}\label{FrechFiltConvCharac} Let $\Lambda$ be an infinite set
    and $x:\Lambda\to\mathbb{C}$ be a function such that $\lim_{n\to\infty}
        x(\lambda_n)=0$ for all sequences ${(\lambda_n)}_{n\in\mathbb{N}}$ of
    distinct elements in $\Lambda$. Then
    $\lim_{\mathcal{F}_{\Lambda}}x(\lambda)=0$.
\end{proposition}
\begin{proof} Suppose it is not true that
    $\lim_{\mathcal{F}_{\Lambda}}x(\lambda)=0$. Then there exists an $\epsilon >
        0$ such that for any $L\in\mathcal{F}_{\Lambda}$ there exists a $\lambda\in
        L$ with the property $|x(\lambda)|>\epsilon$. By induction we can construct
    a sequence ${(\lambda_k)}_{k\in\mathbb{N}}$ of distinct elements in
    $\Lambda$ such that $|x(\lambda_n)|\geq \epsilon$. Therefore
    $\lim_{k\to\infty} x(\lambda_k)\neq 0$. Contradiction.
\end{proof}

\begin{proposition}\label{OpLInfc0DiagConv0} Let $\Lambda$ be an infinite set.
    Then for any bounded linear operator $T:\ell_\infty(\Lambda)\to
        c_0(\Lambda)$ holds $\lim_{\mathcal{F}_{\Lambda}}T(\chi_{\{\lambda
            \}})(\lambda)=0$.
\end{proposition}
\begin{proof} Take an arbitrary sequence ${(\lambda_n)}_{n\in\mathbb{N}}$ of
    distinct elements in $\Lambda$. Then
    ${(\chi_{\{\lambda_n\}})}_{n\in\mathbb{N}}$ weakly converges to 0 in
    $c_0(\Lambda)$, and a fortiori in $\ell_\infty(\Lambda)$. By
    proposition~\ref{OpLInfc0CompContWeakComp} the operator $T$ is completely
    continuous, so $T(\chi_{\{\lambda_n\}})$ converges to 0 in norm. In
    particular, $\lim_{n\to\infty} T(\chi_{\{\lambda_n\}})(\lambda_n)=0$. Now
    from proposition~\ref{FrechFiltConvCharac} we get the desired equality.
\end{proof}

\begin{theorem}\label{RelInjLInfModc0} For any set $\Lambda$ the right
    $\ell_\infty(\Lambda)$-module $c_0(\Lambda)$ is relatively injective.
\end{theorem}
\begin{proof} Assume that $\Lambda$ is infinite. By~\cite[proposition
        IV.1.39]{HelHomolBanTopAlg} it is enough to show that the morphism of
    right $\ell_\infty(\Lambda)$-modules
    $\rho:c_0(\Lambda)\to\mathcal{B}(\ell_\infty(\Lambda),
        c_0(\Lambda)):x\mapsto(a\mapsto x\cdot a)$ admits a left inverse. It
    does exist. Consider linear operator
    $\tau:\mathcal{B}(\ell_\infty(\Lambda), c_0(\Lambda))\to c_0(\Lambda):
        T\mapsto(\lambda\to T(\chi_{\{\lambda \}})(\lambda))$. By
    proposition~\ref{OpLInfc0DiagConv0} this is a well defined linear
    operator. One can easily check that $\tau$ is even a contractive
    morphism of right $\ell_\infty(\Lambda)$-modules.

    If $\Lambda$ is finite it is a Stonean space. Then
    $c_0(\Lambda)=\ell_\infty(\Lambda)=C(\Lambda)$ and by
    proposition~\ref{SStonImplRelInjCSModCS} the $\ell_\infty(\Lambda)$-module
    $c_0(\Lambda)$ is relatively injective.
\end{proof}

%-------------------------------------------------------------------------------
%   Funding
%-------------------------------------------------------------------------------

\section{Funding}\label{SectionFunding} This research was supported by the
Russian Foundation for Basic Research (grant no. 19--01--00447).

%-------------------------------------------------------------------------------
%   Bibliography
%-------------------------------------------------------------------------------

\begin{thebibliography}{999}
    %
    \bibitem{HahnLinSystInLinSp}
    \textit{Hahn, H.}, {\"U}ber lineare Gleichungssysteme in linearen
    R{\"a}umen., J. Reine Angew. Math., 157, 214--229, 1927.
    %
    %
    \bibitem{BanachOnLinFuncI}
    \textit{Banach, S.}, Sur les fonctionnelles lin{\'e}aires, Stud. Math.,
    1(1), 211--216, 1929.
    %
    %
    \bibitem{BanachOnLinFuncII}
    \textit{Banach, S.}, Sur les fonctionnelles lin{\'e}aires II, Stud. Math.,
    1(1), 223--239, 1929.
    %
    %
    \bibitem{BlascIvorConstrInjSpCK}
    \textit{Blasco, J.L. and Ivorra, C.}, On constructing injective spaces of
    type $C(K)$, Indagationes Mathematicae, 9(12), 161--172, 1998.
    %
    %
    \bibitem{EngkingGenTop}
    \textit{Engelking, R.}, General topology, 1977.
    %
    %
    \bibitem{HrusPsdCompTopSp}
    \textit{Hru{\v{s}}{\'a}k, M. and Tamariz-Mascar{\'u}a, A. and Tkachenko,
    M.}, Pseudocompact topological spaces, Springer, 2018.
    %
    %
    %
    \bibitem{HelHomolBanTopAlg}
    \textit{Helemskii, A. Ya.}, The homology of Banach and topological algebras,
    Springer, 1989.
    %
    %
    \bibitem{AvilSepInjBanSp}
    \textit{Avil{\'e}s, A. and S{\'a}nchez, F. C. and Castillo, J. MF.\@ and
        Gonz{\'a}lez, M. and Moreno, Y.}, Separably injective Banach spaces, 17--65,
    2016.
    %
    %
    \bibitem{HasumExtPropCompBanSp}
    \textit{Hasumi, M.}, The extension property of complex Banach spaces, Tohoku
    Mathematical Journal, Second Series, 10(2), 135--142, 1958.
    %
    %
    \bibitem{BerbBaerStRng}
    \textit{Berberian, S. K.}
    Baer $^*$-rings, 195, 2010.
    %
    %
    \bibitem{TakHanBanThAndJordDecomOfModMap}
    \textit{Takesaki, M.}, On the Hahn-Banach type theorem and the Jordan
    decomposition of module linear mapping over some operator algebras, Kodai
    Mathematical Seminar Reports, 12, 1--10, 1960.
    %
    %
    \bibitem{StoneBanStAppBoolRngToTop}
    \textit{Stone, M. H.}, Applications of the theory of Boolean rings to
    general topology, Transactions of the American Mathematical Society, 41(3),
    375--481, 1937.
    %
    %
    \bibitem{KunSetThIndepPrf}
    \textit{Kunen, K.}, Set theory an introduction to independence proofs,
    Elsevier, 2014.
    %
    %
    \bibitem{FinGillExtContFuncbN}
    \textit{Fine, N. J. and Gillman, L.}
    Extension of continuous functions in $\beta {\mathbf{N}}$ Bull. Amer. Math.
    Soc., 66, 376--381, 1960.
    %
    %
    \bibitem{DouKunMillCStarEmbdDenPropSbspStoneRemN}
    \textit{van Douwen, E. and  Kunen,  K. and van Mill, J.}
    There can be $C^*$-embedded dense proper subspaces in $\beta\omega-\omega$,
    Proc. Amer. Math. Soc., 105(2), 462–-470, 1989.
    %
    %
    \bibitem{RamsHomPropSemgroupAlg}
    \textit{Ramsden, P.}, Homological properties of semigroup algebras, The
    University of Leeds, 2009.
    %
    %
    \bibitem{AmirProjContFuncSp}
    \textit{Amir, D.}, Projections onto continuous function spaces, Proceedings
    of the American Mathematical Society, 15(3), 396--402, 1964.
    %
    %
    \bibitem{RosOnRelDisjFamOfMeas}
    \textit{Rosenthal, H.}, On relatively disjoint families of measures, with
    some applications to Banach space theory Stud. Math., 37(1), 13--36, 1970.
    %
    %
    \bibitem{SobProjmOnc0}
    \textit{Sobczyk, A.}, Projection of the space ($m$) on its subspace ($c_0$),
    Bulletin of the American Mathematical Society, 47(12), 938--947, 1941.
    %
    %
    \bibitem{ZipSepExtProbm}
    \textit{Zippin, M.}, The separable extension problem, Israel Journal of
    Mathematics, 26(3--4), 372--387, 1977.
    %
    %
    \bibitem{FabHabBanSpTh}
    \textit{Fabian, M. and Habala, P. and Hajek, P. and
        Montesinos, V. and Zizler, V.}
    Banach space theory. The basis for linear and non-linear analysis. Springer,
    2011.
    %
    %
    \bibitem{GrothApplFabilCompCK}
    \textit{{Grothendieck, A.}}, Sur les applications lin{\'e}aires faiblement
    compactes d'espaces du type $C(K)$, Canadian Journal of Mathematics, 5,
    129--173, 1953.
    %
    %
    \bibitem{DunfPetLinOpSumFunc}
    \textit{Dunford, N. and Pettis, B. J.}, Linear operations on summable
    functions, Transactions of the American Mathematical Society, 47(3),
    323--392, 1940.
    %
\end{thebibliography}

Norbert Nemesh, Faculty of Mechanics and Mathematics, Moscow State University,
Moscow 119991 Russia

\textit{E-mail:} nemeshnorbert@yandex.ru


\end{document}  % chktex 17
