% chktex-file 35
 \documentclass[12pt]{article}
 \usepackage[left=2cm,right=2cm,top=2cm,bottom=2cm,bindingoffset=0cm]{geometry}
 \usepackage{amssymb}
 \usepackage{amsmath}
 \usepackage{amsthm}
 \usepackage{enumerate}
 \usepackage[T1,T2A]{fontenc}
 \usepackage[utf8]{inputenc}
 \usepackage[matrix,arrow,curve]{xy}
 \usepackage[colorlinks=true, urlcolor=blue, linkcolor=blue, citecolor=blue,
     pdfborder={0 0 0}]{hyperref}
\usepackage{enumitem}

 %------------------------------------------------------------------------------
 \newtheorem{theorem}{Theorem}[section]
 \newtheorem{lemma}[theorem]{Lemma}
 \newtheorem{proposition}[theorem]{Proposition}
 \newtheorem{remark}[theorem]{Remark}
 \newtheorem{corollary}[theorem]{Corollary}
 \newtheorem{definition}[theorem]{Definition}
 \newtheorem{example}[theorem]{Example}

 \newcommand{\projtens}{\mathbin{\widehat{\otimes}}}
 \newcommand{\convol}{\ast}
 \newcommand{\projmodtens}[1]{\mathbin{\widehat{\otimes}}_{#1}}
 \newcommand{\isom}[1]{\mathop{\mathbin{\cong}}\limits_{#1}}
 %------------------------------------------------------------------------------

\begin{document}

\begin{center}
    \Large \textbf{Relative projectivity of modules $L_p$}\\[0.5cm]
    \small {N. T. Nemesh}\\[0.5cm]
\end{center}

\thispagestyle{empty}

\medskip
\textbf{Abstract:} In this paper we give criteria of relative projectivity of
$L_p$-spaces regarded as left Banach modules over the algebras of bounded
measurable functions ($1\leq p\leq +\infty$) and the algebra of continuous
functions vanishing at infinity ($1\leq p< +\infty$). The main result is as
follows: for a locally compact Hausdorff space $S$ and a locally finite inner
compact regular Borel measure $\mu$ relative projectivity of $C_0(S)$-module
$L_\infty(S,\mu)$ forces $\mu$ to be inner open regular with pseudocompact
support.
\medskip

\textbf{Keywords:} projective module, $L_p$-space, atom, normal measure,
pseudocompact space.

\bigskip

%-------------------------------------------------------------------------------
%	Introduction
%-------------------------------------------------------------------------------

\section{Introduction}\label{SectionIntroduction}

The main question of Banach homology sounds like this: what is the homological
dimension of a given Banach algebra $A$? To solve this problem one needs to
answer another question: is a given Banach $A$-module $X$ projective? For many
modules of analysis the answers are known. Still there are examples of classical
modules of analysis for which this question is not addressed, for example the
$L_p$-spaces. We shall regard Lebesgue's spaces as left Banach modules over the
algebra of vanishing at infinity continuous functions defined on a locally
compact Hausdorff space $S$ and as modules over the algebra of bounded
measurable functions. We shall give necessary and sufficient conditions for
relative projectivity of these spaces. Special attention must be paid to the
case of projective $L_\infty$ modules. The reason is that one of the main
results of Banach homology --- the global dimension
theorem~[\cite{HelHomolBanTopAlg}, proposition V.2.21] is based on the fact that
the homological dimension of the module of bounded sequences over the algebra of
vanishing sequences equals 2. In particular this module is not projective. As we
shall see this behaviour is typical for most modules $L_\infty$. Before we
proceed to the main topic we shall give a few definitions.



Let $M$ be a subset of a set $N$, then $\chi_M$ denotes the indicator function
of $M$. If $f:N\to L$ is an arbitrary function, then $f|_M$ denotes its
restriction to $M$. The symbol $1_M$ denotes the identity map on $M$.

Let $S$ be an arbitrary topological space and $M$ be its subset. Then
$\operatorname{cl}_S(M)$ and $\operatorname{int}_S(M)$ denote the closure and
the interior of $M$ in $S$.

All Banach spaces in this paper are considered over the complex field. For a
given Banach spaces $X$ and $Y$ by $X\oplus_1 Y$ we denote their $\ell_1$-sum
and by $X\projtens Y$ their projective tensor product. We say that a Banach
space $X$ is complemented in $Y$ if $X$ is a subspace of $Y$ and there exists a
bounded linear operator $P:Y\to Y$ such that $P|_X=1_X$ and
$\operatorname{Im}(P)=X$. For $1\leq p\leq +\infty$ and a given measure space
$(X,\mu)$ by $L_p(X,\mu)$ we shall denote the Banach space of equivalence
classes of $p$-integrable (or essentially bounded if $p=+\infty$) functions on
$X$. Elements of $L_p(X,\mu)$ are denoted by $[f]$. Note that all $L_p$-spaces
have the approximation property.

For a given Banach algebra $A$ by $A_+:=A\oplus_1 \mathbb{C}$ we denote its
standard unitization. We shall consider only left Banach modules with
contractive outer action $\cdot:A\times X\to X$. If $A$ is a Banach algebra with
the unit $e$, then a Banach $A$-module $X$ is called unital if $e\cdot x=x$ for
all $x\in X$. For a given Banach $A$-module $X$ its essential part $X_{ess}$ is
the closed linear span of the set $A\cdot X$. We say that the module $X$ is
essential if $X=X_{ess}$. Clearly, any unital Banach module is essential. Let
$X$ and $Y$ be two Banach $A$-modules, then a map $\phi:X\to Y$ is an
$A$-morphism if it is a continuous $A$-module map. Banach $A$-modules and
$A$-morphisms form a category which we denote by $A-\mathbf{mod}$.

The category $A$-mod has its own notion of projectivity. An $A$-morphism
$\xi:X\to Y$ is called admissible if there exists a right inverse bounded linear
operator $\eta:Y\to X$, i.e if $\xi\eta=1_Y$. A Banach $A$-module $P$ is called
relatively projective if for any admissible $A$-morphism $\xi:X\to Y$ and any
$A$-morphism $\phi:P\to Y$ there exists an $A$-morphism $\psi:P\to X$ making the
diagram
$$
    \xymatrix{
    & {X} \ar[d]^{\xi}\\  % chktex 3
    {P} \ar@{-->}[ur]^{\psi} \ar[r]^{\phi} &{Y}}  % chktex 3
$$
commutative. Instead of checking by definition one may show that a Banach module
$P$ is relatively projective by constructing an $A$-morphism $\sigma:P\to
    A_+\projtens P$ which is a right inverse for the canonical $A$-morphism
$\pi_P^+:A_+\projtens P\to P:(a\oplus_1 z)\projtens x\mapsto a\cdot x+z x$
[\cite{HelHomolBanTopAlg}, proposition IV.1.1]. If a Banach module $P$ is
essential then it is projective if and only if the canonical $A$-morphism
$\pi_P:A\projtens P\to P: a\projtens x\mapsto a\cdot x$ has a right inverse
$A$-morphism [\cite{HelHomolBanTopAlg}, proposition IV.1.2].

%-------------------------------------------------------------------------------
%	Necessary conditions
%-------------------------------------------------------------------------------

\section{Necessary conditions for relative
  projectivity}\label{NecessaryConditions}

In this section we shall show that for a relatively projective $A$-module $X$
its essential part is complemented and $A$-valued $A$-morphisms separate points
of the essential part. These necessary conditions will play a key role in this
paper.

\begin{proposition}\label{MorphDecomp} Let $X$ be a Banach $A$-module and $E$ be
    a Banach space. Let $j_E:A_+\projtens E\to (A\projtens E)\oplus_1 E$ denote
    the natural isomorphism. Then for any $A$-morphism $\sigma:X\to A_+\projtens
        E$ there exist bounded linear operators $\sigma_1:X\to A\projtens E$,
    $\sigma_2:X\to E$ such that

    \begin{enumerate}[label = (\roman*)]
        \item $j_E(\sigma(x))=\sigma_1(x)\oplus_1 \sigma_2(x)$ for all $x\in X$;
        \item $\sigma_1(a\cdot x)=a\cdot \sigma_1(x)+a\projtens \sigma_2(x)$ for
              all $x\in X$ and $a\in A$;
        \item $\sigma_2(a\cdot x)=0$ for all $x\in X$ and $a\in $.

              \noindent
              As a consequence, $\sigma_1|_{X_{ess}}$ is an $A$-morphism,
              $\sigma_2|_{X_{ess}}=0$.
    \end{enumerate}
\end{proposition}
\begin{proof} Consider bounded linear operators $q_1:A_+\projtens X\to
        A\projtens X: (a\oplus_1 z)\projtens x\mapsto a\projtens x$ and
    $q_2:A_+\projtens X\to X: (a\oplus_1 z)\projtens x\mapsto z x$. Now
    define $\sigma_1=q_1\sigma$, $\sigma_2=q_2\sigma$. Clearly,
    $j_E=q_1\oplus_1 q_2$, hence $j_E(\sigma(x))=\sigma_1(x)\oplus_1
        \sigma_2(x)$ for all $x\in X$. Note that $a\cdot u=a\cdot
        q_1(u)+a\projtens q_2(u)$ for all $a\in A$ and $u\in A_+\projtens X$. As
    $\sigma$ is an $A$-morphism, it is routine to check that
    $\sigma_1(a\cdot x)=a\cdot \sigma_1(x)+a\projtens \sigma_2(x)$ and
    $\sigma_2(a\cdot x)=0$ for all $a\in A$, $x\in X$.
\end{proof}

The following proposition is a slight generalization of
    [\cite{SelivBiprojBanAlg}, lemma 1.4].

\begin{proposition}\label{ProjModEssPartCompl} Let $X$ be a relatively
    projective Banach $A$-module. Then $X_{ess}$ is complemented in $X$ as
    Banach space.
\end{proposition}
\begin{proof} Since $X$ is relatively projective there exists an $A$-morphism
    $\sigma:X\to A_+\projtens X$ such that $\pi_X^+\sigma=1_X$. Let $\sigma_1$
    and $\sigma_2$ be bounded linear operators given by
    proposition~\ref{MorphDecomp}. Now consider $A$-morphism $\pi_X:A\projtens
        X\to X:a\projtens x\mapsto a\cdot x$, then for all $x\in X$ we have
    $x=\pi_X^+(\sigma(x))=\pi_X(\sigma_1(x)) + \sigma_2(x)$. Consider a bounded
    linear operator $\eta = \pi_X\sigma_1$. Since $\sigma_2|_{X_{ess}}=0$ we
    have $\eta|_{X_{ess}}=1_X$. Moreover
    $\operatorname{Im}(\eta)\subset\operatorname{Im}(\pi_X)=X_{ess}$, therefore
    $\eta$ is a bounded linear projection of $X$ onto $X_{ess}$.
\end{proof}

\begin{proposition}\label{RelProjNecesCond} Let $A$ be a Banach algebra and $X$
    be a relatively projective Banach $A$-module. Suppose, that either $A$ or
    $X$ possess the approximation property. Then
    \begin{enumerate}[label = (\roman*)]
        \item for any non-zero $x\in X$ there exists an $A$-morphism $\phi:X\to
                  A_+$
        \item for any non-zero $x\in X_{ess}$ there exists an $A$-morphism
              $\psi:X_{ess}\to A$ such that $\psi(x)\neq 0$;
    \end{enumerate}
\end{proposition}
\begin{proof} Let $i_E:E\projtens \mathbb{C}\to E$ be the natural isomorphism.

    $(i)$ Fix non-zero $x\in X$. Since $X$ is relatively projective there exists
    an $A$-morphism $\sigma:X\to A_+\projtens X$ such that $\pi_X^+\sigma=1_X$.
    Consider $u:=\sigma(x)\in A_+\projtens X$. Since $\pi_X^+(u)=x\neq 0$, then
    $u\neq 0$. Recall that either $A$ or $X$ has the approximation property, so
    there exist $f\in A_+^*$ and $g\in X^*$ such that $(f\projtens g)(u)\neq 0$
    [\cite{GrothProdTenTopNucl}, corollary I.5.1, p. 168]. Consider
    $a:=((1_{A_+}\projtens g)(u))\in A_+\projtens\mathbb{C}$ and $F:=(f\projtens
        1_{\mathbb{C}})\in A_+^*\projtens\mathbb{C}$. Since $F(a)=(f\projtens
        g)(u)\neq 0$, then $a\neq 0$. Now it is routine to check that the linear
    operator $\xi:=(1_{A_+}\projtens g)\sigma$ is an $A$-morphism. Clearly,
    $\xi(x)=a\neq 0$. It remains to set $\phi=i_{A_+}\xi$.

    $(ii)$ Fix a non-zero $x\in X_{ess}$. Let $\xi$ be the morphism constructed
    in paragraph $(i)$. Consider morphisms $\xi_1$ and $\xi_2$ given by
    proposition~\ref{MorphDecomp}. Since
    $\xi(x)=j_{\mathbb{C}}(\xi_1(x)\oplus_1\xi_2(x))\neq 0$ and $\xi_2(x)=0$,
    then $\xi_1(x)\neq 0$. By the same proposition $\xi_1|_{X_{ess}}$ is an
    $A$-morphism, so it remains to set $\psi=i_A \xi_1|_{X_{ess}}$.

\end{proof}

%-------------------------------------------------------------------------------
%   Preliminaries on general measure theory
%-------------------------------------------------------------------------------

\section{Preliminaries on general measure
  theory}\label{SectionPreliminariesOnGeneralMeasureTheory}

A comprehensive study of general measure spaces can be found
in~\cite{FremMeasTh2}. We follow its definitions.

Let $X$ be a set. By measure we mean a countably additive set function with
values in $[0,+\infty]$ defined on a $\sigma$-algebra $\Sigma$ of measurable
subsets of a set $X$. If $F$ is a measurable set, then we have a well defined
measures $\mu^F:\Sigma\to[0,+\infty]:E\mapsto \mu(E\cap F)$ and
$\mu_F:\Sigma_F\to[0,+\infty]: E\mapsto \mu(E)$, where
$\Sigma_F=\{E\in\Sigma:E\subset F\}$. A measurable set $E$ is called an atom if
$\mu(A)>0$ and for every measurable subset $B\subset A$ holds either $\mu(B)=0$
or $\mu(A\setminus B)=0$. A measure $\mu$ is called purely atomic if every
measurable set of positive measure has an atom. A measure $\mu$ is semi-finite
if for any measurable set $A$ of infinite measure there exists a measurable
subset of $A$ with finite positive measure. A family $\mathcal{D}$ of measurable
subsets of finite measure is called a decomposition of $X$ if for any measurable
set $E$ $\mu(E)=\sum_{D\in\mathcal{D}}\mu(E\cap D)$ and a set $F$ is measurable
whenever $F\cap D$ is measurable for all $D\in\mathcal{D}$. Finally, a measure
$\mu$ is called decomposable if it is semi-finite and admits a decomposition of
$X$. In fact a measure space is decomposable if and only if it is a disjoint
union of finite measure spaces [\cite{FremMeasTh2}, exercise 214X (i)]. Most
measures encountered in functional analysis are decomposable.

\begin{definition}\label{AtomCore} Let $A$ be an atom of a measure space
    $(X,\mu)$. Then a measurable set $C\subset A$ is called a core of $A$ if $C$
    is an atom and the only measurable subsets of $C$ are $\varnothing$ and $C$.
    An atom $A$ is called hard if it has a core. Clearly, if core exists it is
    unique and in this case the core is denoted by $A^\bullet$
\end{definition}

\begin{proposition}\label{GenniunelyAtomicMeasCharac} Let $(X,\mu)$ be a
    non-empty finite measure space such that the only set of measure zero in $X$
    is the empty set. Then $(X,\mu)$ is purely atomic and each atom is hard.
\end{proposition}
\begin{proof} Let $E$ be a measurable set of positive measure. Peek any $x\in E$
    and consider value $c:=\inf \{\mu(F):x\in F\in \Sigma,\; F\subset E\}$. For
    any $n\in\mathbb{N}$ there exists $E_n\in\Sigma$ such that $x\in E_n\subset
        E$ and $\mu(E_n)<c+2^{-n}$. Define $A=\bigcap \{E_n:n\in\mathbb{N}\}\subset
        E$, then $x\in A\in\Sigma$ and $\mu(A)=c$. By construction $A$ is not empty,
    so $\mu(A)>0$. Suppose $B$ is a measurable subset of $A$. If $x\in
        A\setminus B$, then $c \leq\mu(A\setminus B)\leq\mu(A)=c$, i.e. $\mu(B)=0$.
    Similarly, if $x\in B$ we get $\mu(A\setminus B)=0$. Thus $A\subset E$ is an
    atom. Since $E$ is arbitrary, $(X,\mu)$ is purely atomic.

    Now let $A$ be an atom of $(X,\mu)$. If $B\in\Sigma$ and $B\subset A$, then
    either $\mu(B)$ or $\mu(A\setminus B)=0$. From assumption on $(X,\mu)$ we
    get either $B$ or $A\setminus B$ is empty. Thus $A^\bullet=A$.
\end{proof}

%-------------------------------------------------------------------------------
%	Projectivity of $B(\Sigma)$-modules L_p(X,\mu)
%-------------------------------------------------------------------------------

\section{Relative projectivity of \texorpdfstring{$B(\Sigma)$}{BSigma}-modules
  \texorpdfstring{$L_p(X,\mu)$}{LpXmu}
 }\label{SectionRelativeProjectivityOfBSigmaModulesLpXmu}

Let $(X,\mu)$ be a measure space. By $B(\Sigma)$ we shall denote the algebra of
bounded measurable functions with the  $\sup$ norm. In this section we give a
criterion of projectivity of $B(\Sigma)$-modules $L_p(X,\mu)$. Speaking
informally all these modules look like $\ell_\infty(\Lambda)$-modules
$\ell_p(\Lambda)$ for some index set $\Lambda$.

\begin{proposition}\label{BSigmaModLpRetrProj} Let $(X,\mu)$ be a measure space.
    Let $1\leq p\leq +\infty$ and $L_p(X,\mu)$ be a relatively projective
    $B(\Sigma)$-module. Then for any measurable set $B\subset X$ the
    $B(\Sigma)$-module $L_p(X,\mu^B)$ is relatively projective.
\end{proposition}
\begin{proof} It is easy to check that for $B(\Sigma)$-morphisms
    $\pi:L_p(X,\mu)\to L_p(X,\mu^B):[f]\mapsto [f]\chi_B$ and
    $\sigma:L_p(X,\mu^B)\to L_p(X,\mu):[f]\mapsto [f]$ holds
    $\pi\sigma=1_{L_p(X,\mu^B)}$. In other words $L_p(X,\mu^B)$ is a retract of
    $L_p(X,\mu)$ in $B(\Sigma)-\mathbf{mod}$. Now the result follows from
        [\cite{HelBanLocConvAlg}, proposition VII.1.6].
\end{proof}

\begin{proposition}\label{LpBSigmaModNecessCond} Let $(X,\mu)$ be a decomposable
    measure space and $L_p(X,\mu)$ be a relatively projective
    $B(\Sigma)$-module. Then $(X,\mu)$ is a disjoint union of hard atoms of
    finite measure.
\end{proposition}
\begin{proof} Let $\mathcal{D}$ be a decomposition of $X$ onto measurable
    subsets of finite measure. Fix $D\in\mathcal{D}$ and denote $\nu:=\mu^D$. By
    proposition~\ref{BSigmaModLpRetrProj} the $B(\Sigma)$-module $L_p(X,\nu)$ is
    relatively projective. Peek any $E\in\Sigma$ of positive measure $\nu$.
    Since $\nu$ is finite, so is $\nu(E)$. Then $[f]:=[\chi_E]$ is well defined
    and non-zero in $L_p(X,\nu)$. As $B(\Sigma)$ is a unital algebra the module
    $L_p(X,\nu)$ is essential. Now from proposition~\ref{RelProjNecesCond} we
    get a $B(\Sigma)$-morphism $\psi:L_p(X,\nu)\to B(\Sigma)$ such that
    $\psi([f])\neq 0$. Therefore the set $F:=a^{-1}(\mathbb{C}\setminus
        \{0\})\in\Sigma$ is not empty. Note that $[f]=[f]\chi_E$, so
    $a=\psi([f]\chi_E)=\psi([f])\chi_E=a\chi_E$. Hence $a|_{X\setminus E}=0$ and
    $F\subset E$. Consider arbitrary measurable set $A\subset F$ with $\nu$
    measure zero. Then $[\chi_A]$ is zero in $L_p(X,\nu)$ and
    $[\chi_A]=[\chi_E]\chi_A$. Therefore
    $a\chi_A=\psi([\chi_E])\chi_A=\psi([\chi_E]\chi_A)=\psi([\chi_A])=0$. As
    $A\subset F$ and $a$ is not zero at any point of $F$ we get $A=\varnothing$.
    Since $F\neq \varnothing$, then from
    proposition~\ref{GenniunelyAtomicMeasCharac} we get that the measure space
    $(F,\nu_F)$ has a hard atom. Thus we have shown that any measurable set $E$
    of positive $\nu$ measure has a hard atom, hence by a standard application
    of Zorn's lemma we get that $(X,\mu^D)$ is a disjoint union of hard atoms.
    The same conclusion holds for $(X,\mu_D)$. Since $\mu_D$ is finite, then so
    is every atom. Since $D$ is arbitrary the result follows from
        [\cite{FremMeasTh2}, exercise 214X (i)].
\end{proof}

Let $(X,\mu)$ be a measure space and $A$ be a measurable set of finite positive
measure. Then we have a well defined bounded linear functional
$m_A:B(\Sigma)\to\mathbb{C}:a\mapsto {\mu(A)}^{-1}\int_A f(x)d\mu(x)$ of norm 1.

\begin{proposition}\label{HardAtomicMeasProp} Let $(X,\mu)$ be a disjoint union
    of the family $\mathcal{A}$ of finite hard atoms. Then
    \begin{enumerate}[label = (\roman*)]
        \item the set $X^\bullet:=\bigcup \{A^\bullet:A\in\mathcal{A}\}$ is
              measurable and $\mu(X\setminus X^\bullet)=0$;

        \item for any $A\in\mathcal{A}$ and any functions $a,b\in B(\Sigma)$
              holds $a|_{A^\bullet}=m_{A^\bullet}(a)$ and
              $m_{A^\bullet}(ab)=m_{A^\bullet}(a)m_{A^\bullet}(b)$;

        \item for any $a\in B(\Sigma)$ there is a function $b\in B(\Sigma)$ such
              that $b|_{X^\bullet}=0$ and a pointwise equality holds
              $a=\sum_{A\in\mathcal{A}} m_{A^\bullet}(a)\chi_{A^\bullet} + b$;

        \item for any $[f]\in L_p(X,\mu)$ holds $[f]=[\sum_{A\in\mathcal{A}}
                  m_{A^\bullet}(f)\chi_{A^\bullet}]$.
    \end{enumerate}
\end{proposition}
\begin{proof} $(i)$ Since $\mathcal{A}$ is a decomposition of $X$, then
    $(X,\mu)$ is decomposable and $X^\bullet$ is measurable. Note that disjoint
    sets $A\setminus A^\bullet$ for $A\in\mathcal{A}$ are of measure zero, hence
    so is their union $X\setminus X^\bullet$.

    $(iii)$ Fix $a\in B(\Sigma)$ and $A\in \mathcal{A}$. Since $A^\bullet$ has
    only two measurable subsets, then $a$ is constant on $A^\bullet$. Therefore
    $a|_{A^\bullet}=m_{A^\bullet}(a)$. As a consequence for the measurable
    function $b=a-\sum_{A\in\mathcal{A}}m_{A^\bullet}(a)\chi_{A^\bullet}$ we
    have $b|_{X^\bullet}=0$.

    $(iv)$ The result immediately follows from paragraph $(iii)$.
\end{proof}

\begin{proposition}\label{LpBSigmaModSuffCond} Let $1\leq p\leq +\infty$ and
    $(X,\mu)$ be a disjoint union of finite hard atoms. Then the
    $B(\Sigma)$-module $L_p(X,\mu)$ is relatively projective.
\end{proposition}
\begin{proof} Let $\mathcal{A}$ denote the set of hard atoms of $X$.

    Consider the case $p=+\infty$. Define a bounded linear operator
    $$
        \rho:L_\infty(X,\mu)\to B(\Sigma):
        [f]\mapsto\sum_{A\in\mathcal{A}}m_{A^\bullet}(f)\chi_{A^\bullet}.
    $$
    From paragraph $(ii)$ of proposition~\ref{HardAtomicMeasProp} it follows
    that $\rho$ is a $B(\Sigma)$-morphism. Therefore $\sigma=\rho\projtens
        1_{L_\infty(X,\mu)}$ is a $B(\Sigma)$-morphism too. From paragraph $(iv)$ of
    proposition~\ref{HardAtomicMeasProp} we get that
    $\pi_{L_\infty(X,\mu)}\sigma=1_{L_\infty(X,\mu)}$. Since $L_\infty(X,\mu)$
    is a unital $B(\Sigma)$-module, then by [\cite{HelHomolBanTopAlg},
            proposition IV.1.2] it is relatively projective.

    Consider the case $1\leq p<+\infty$. Let $[f]\in L_p(X,\mu)$, then from
    paragraph $(iv)$ of proposition~\ref{HardAtomicMeasProp} we get
    $[f]=[\sum_{A\in\mathcal{A}}m_{A^\bullet}(f)\chi_{A^\bullet}]$. Even more,
    since $p<+\infty$ we have
    $[f]=\sum_{A\in\mathcal{A}}m_{A^\bullet}(f)[\chi_{A^\bullet}]$ in
    $L_p(X,\mu)$. Note that the latter sum contains only countably many non-zero
    summands. We denote indices of these summands as $\mathcal{A}_f$. Consider
    arbitrary finite subset $\mathcal{F}=\{A_1,\ldots,A_n\}\subset\mathcal{A}_f$
    and denote $x_k=\chi_{A_k^\bullet}$,
    $y_k=m_{A_k^\bullet}(f)[\chi_{A_k^\bullet}]$ for $k\in \{1,\ldots,n\}$. Let
    $\omega\in\mathbb{C}$ be any $n$-th root of $1$. Since $\mathcal{F}$ is a
    disjoint family $\Vert\sum_{k=1}^n \omega^k x_k\Vert_{B(\Sigma)}\leq 1$ and
    $\Vert \sum_{k=1}^n\omega^k y_k\Vert_{L_p(X,\mu)}\leq\Vert
        f\Vert_{L_p(X,\mu)}$. Therefore by [\cite{HelHomolBanTopAlg}, proposition
            II.2.44] for any $f\in L_p(X,\mu)$ we have a well defined element
    $\sigma_f=\sum_{A\in\mathcal{A}_f} x_k\projtens y_k=\sum_{A\in\mathcal{A}}
        x_k\projtens y_k\in B(\Sigma)\projtens L_p(X,\mu)$ of norm not greater than
    $\Vert f\Vert_{L_p(X,\mu)}$. Now using paragraph $(ii)$ of
    proposition~\ref{HardAtomicMeasProp} it is easy to check that the mapping
    $$
        \sigma: L_p(X,\mu)\to B(\Sigma)\projtens L_p(X,\mu):
        [f]\mapsto \sum_{A\in\mathcal{A}}
        m_{A^\bullet}(f)\chi_{A^\bullet}\projtens[\chi_{A^\bullet}]
    $$
    is a well defined $B(\Sigma)$-morphism of norm at most $1$. From paragraph
    $(iv)$ of proposition~\ref{HardAtomicMeasProp} we get that
    $\pi_{L_p(X,\mu)}\sigma=1_{L_p(X,\mu)}$. Since $L_p(X,\mu)$ is a unital
    $B(\Sigma)$-module, then by [\cite{HelHomolBanTopAlg}, proposition IV.1.2]
    it is relatively projective.
\end{proof}

\begin{theorem}\label{LpBSigmaModCrit} Let $(X,\mu)$ be a decomposable measure
    space and $1\leq p\leq +\infty$. Then the following are equivalent:
    \begin{enumerate}[label = (\roman*)]
        \item $L_p(X,\mu)$ is a relatively projective $B(\Sigma)$-module;

        \item $(X,\mu)$ is a disjoint union of hard atoms of finite measure.
    \end{enumerate}
\end{theorem}
\begin{proof} The result follows from propositions~\ref{LpBSigmaModNecessCond}
    and~\ref{LpBSigmaModSuffCond}.
\end{proof}

%-------------------------------------------------------------------------------
%   Preliminaries on topological measure theory
%-------------------------------------------------------------------------------

\section{Preliminaries on topological measure
  theory}\label{SectionPreliminariesOnTopologicalMeasureTheory}

A detailed discussion of topological measures can be found
in~\cite{FremMeasTh4.1}. From now we shall consider measures $\mu$ defined on
the $\sigma$-algebra $Bor(S)$ of Borel sets of a topological space $S$. By
$\operatorname{supp}(\mu)$ we shall denote the support of $\mu$. The measure
$\mu$ is called:
\begin{enumerate}[label = (\roman*)]
    \item strictly positive if $\operatorname{supp}(\mu)=S$;

    \item fully supported if $\mu(S\setminus\operatorname{supp}(\mu))=0$;

    \item locally finite if every point in $S$ has an open neighbourhood of
          finite measure;

    \item inner compact regular if $\mu(E)=\sup \{\mu(K): K\subset E, K\mbox{ is
                  compact}\}$ for any $E\in Bor(S)$;

    \item outer open regular if $\mu(E)=\inf \{\mu(U): E\subset U, U\mbox{ is
                  open}\}$ for any $E\in Bor(S)$;

    \item inner open regular if $\mu(E)=\sup \{\mu(U): U\subset E, U\mbox{ is
                  open}\}$ for any $E\in Bor(S)$;

    \item residual if $\mu(E)=0$ for every Borel nowhere dense set $E$;

    \item normal if it is residual and fully supported.
\end{enumerate}

If $\mu$ is locally finite, then all compact sets have finite
measure~[\cite{FremMeasTh4.1}, proposition 411G (a)]. Any finite inner compact
regular measure is outer open regular [\cite{FremMeasTh4.1}, proposition 411X
        (a)]. Clearly, $\mu^B$ is inner compact regular for any $B\in Bor(S)$ whenever
$\mu$ is inner compact regular.

\begin{proposition}\label{InnerOpenRegMeasCharac} Let $S$ be a locally compact
    Hausdorff space and $\mu$ be a Borel measure on $S$. Then
    \begin{enumerate}[label = (\roman*)]
        \item $\mu$ is inner open regular if and only if
              $\mu(E)=\mu(\operatorname{int}_S(E))$ for $E\in Bor(S)$;

        \item if $\mu$ is finite and inner open regular, then
              $\mu(E)=\mu(\operatorname{int}_S(E))=\mu(\operatorname{cl}_S(E))$
              and $\mu$ is residual;

        \item if $\mu$ is finite, inner compact regular and inner open regular,
              then $\mu$ is normal;
    \end{enumerate}
\end{proposition}
\begin{proof} $(i)$ It is enough to note that the supremum in the definition of
    inner open regular measure is attained at maximal open subset of $E$, which
    is $\operatorname{int}_S(E)$.

    $(ii)$ The first equality was proved in the previous paragraph. Since $\mu$
    is finite for all $E\in Bor(S)$ we have
    $\mu(E)=\mu(S)-\mu(\operatorname{int}_S(S\setminus E))=\mu(S)-\mu(S\setminus
        \operatorname{cl}_S(E))=\mu(\operatorname{cl}_S(E))$. Now consider a nowhere
    dense Borel set $E\subset S$, then $\mu(E)=\mu(\operatorname{cl}_S(E))
        =\mu(\operatorname{cl}_S(\operatorname{int}_S(E)))=\mu(\varnothing)=0$.
    Since $E$ is arbitrary, then $\mu$ is residual.

    $(iii)$ Any inner compact regular measure has support and is fully supported
        [\cite{FremMeasTh4.1}, propositions 411C, 411N (d)]. The rest follows
    from paragraphs $(i)$ and $(ii)$.
\end{proof}

\begin{proposition}\label{NormalMeasCharac} Let $S$ be a locally compact
    Hausdorff space and $\mu$ be a Borel measure on $S$. Suppose that for any
    compact set $K\subset S$ with $\mu(K)>0$ there exists an open set $U\subset
        K$ with $\mu(U)>0$. Then
    \begin{enumerate}[label = (\roman*)]
        \item $\mu(K)=\mu(\operatorname{int}_S(K))$ for any compact set
              $K\subset S$;
        \item if $\mu$ is inner compact regular then $\mu$ is inner open
              regular.
    \end{enumerate}
\end{proposition}
\begin{proof} Denote $K'=K\setminus \operatorname{int}_S(K)$. This is a closed
    subset of the compact set $K$, hence it is compact. Suppose $\mu(K')>0$,
    then there exists an open set $U\subset K'$ with $\mu(U)>0$. As a
    consequence $U\subset K$ is a non-empty open set disjoint from
    $\operatorname{int}_S(K)$. Contradiction, hence $\mu(K')=0$ and
    $\mu(K)=\mu(\operatorname{int}_S(K))$.

    $(ii)$ Fix $c<\mu(B)$. Since $\mu$ is inner compact regular, then there
    exists a compact set $K\subset B$ with $c<\mu(K)$. From previous paragraph
    we get
    $c<\mu(K)=\mu(\operatorname{int}_S(K))\leq\mu(\operatorname{int}_S(B))$. As
    $c<\mu(B)$ is arbitrary we conclude that
    $\mu(B)\leq\mu(\operatorname{int}_S(B))$. The inverse inequality is obvious.
\end{proof}

\begin{proposition}\label{MeasAtomCharac} Let $S$ be a locally compact Hausdorff
    space and $\mu$ be a locally finite inner compact regular Borel measure on
    $S$. Let $A$ be an atom for $\mu$. Then

    $(i)$ $\mu(A)$ is finite;

    $(ii)$ there exists a point $s\in A$ such that $\mu(A)=\mu(\{s\})$.
\end{proposition}
\begin{proof} $(i)$ Since $\mu$ is inner compact regular there exists a compact
    set $K\subset A$ of positive measure. Since $\mu$ is locally finite $\mu(K)$
    is finite. As $A$ is an atom and $\mu(K)>0$, we get $\mu(A)=\mu(K)<+\infty$.

    $(ii)$ By previous paragraph $0<\mu(A)<+\infty$. Let $\mathcal{K}$ denote
    the compact subsets of $A$ that have the same measure as $A$. Since $\mu$ is
    inner compact regular there exists a compact set $K\subset A$ of positive
    measure. Since $A$ is an atom we immediately get $\mu(K)=\mu(A)$, that is
    $K\in\mathcal{K}$. Thus $\mathcal{K}$ is not empty. Now consider two
    arbitrary sets $K',K''\in\mathcal{K}$. Clearly, $C:=K'\cap K''$ is a compact
    subset of $A$. Suppose that $\mu(C)=0$, and consider $L'=K'\setminus C$,
    $L''=K''\setminus C$. These are two disjoint subsets of $A$, such that
    $\mu(L')=\mu(L'')=\mu(A)$, so $\mu(A)\geq \mu(L'\cup L'')=2\mu(A)$.
    Contradiction, hence $\mu(C)>0$ and therefore $C\in\mathcal{K}$. As $K',
        K''\in \mathcal{K}$ are arbitrary we have shown that $\mathcal{K}$ is a
    family of compact sets with the finite intersection property. Therefore
    $K^*=\bigcap\mathcal{K}$ is not empty. Clearly, $K^*$ is compact as
    intersection of compact sets. Suppose $K^*$ has two distinct points $s'$ and
    $s''$. Consider singletons $C'=\{s'\}$ and $C''=\{s''\}$. Suppose
    $\mu(C')>0$, then $\mu(C')=\mu(A)$ and $C'\in\mathcal{K}$ as $A$ is an atom.
    This contradicts minimality of $K^*$ as $C'$ is a proper subset of $K^*$, so
    $\mu(C')=0$. Similarly, $\mu(C'')=0$. Consider $L=K^*\setminus (C'\cup
        C'')$, then $\mu(L)=\mu(K^*)=\mu(A)$. As $\mu$ is inner compact regular
    there exists a compact set $\hat{K}\subset L\subset A$ of positive measure,
    therefore $\mu(\hat{K})=\mu(A)$. By construction $\hat{K}\in\mathcal{K}$ is
    a proper subset of $\mathcal{K}$. This contradicts minimality of $K^*$,
    therefore $K^*$ is a non-empty set without two distinct points, hence a
    singleton. Thus $\mu(A)=\mu(K^*)=\mu(\{s\})$ for some $s\in A$.
\end{proof}

%-------------------------------------------------------------------------------
%	Projectivity of $C_0(S)$-modules L_p(S,\mu)
%-------------------------------------------------------------------------------

\section{Relative projectivity of \texorpdfstring{$C_0(S)$}{C0S}-modules
  \texorpdfstring{$L_p(S,\mu)$}{LpSmu}
 }\label{SectionRelativeProjectivityOfC0SModulesLpSmu}

Results of this section are somewhat similar to the case of modules over the
algebra of bounded measurable functions, but the case $p=+\infty$ doesn't seem
to have a simple criterion.

\begin{proposition}\label{LpEssC0ModCharac} Let $S$ be a locally compact
    Hausdorff space, $\mu$ be a locally finite Borel measure on $S$ and $1\leq
        p\leq+\infty$. Then
    \begin{enumerate}[label = (\roman*)]
        \item $[f]\in {L_p(S,\mu)}_{ess}$ if and only if for any $\varepsilon
                  >0$ there exists a compact set $K\subset S$ such that $\Vert
                  [f\chi_{S\setminus K}]\Vert_{L_p(S,\mu)}< \varepsilon$;
        \item if $p<+\infty$ and $\mu$ is inner compact regular, then
              ${L_p(S,\mu)}_{ess}=L_p(S,\mu)$.
    \end{enumerate}
    In particular, for any compact set $K\subset S$ and $[f]\in L_p(S,\mu)$
    holds $[f]\chi_K\in {L_p(S,\mu)}_{ess}$.
\end{proposition}
\begin{proof} The proof is a standard density argument.
\end{proof}


\begin{proposition}\label{MorphLpEssC0Prop} Let $S$ be a locally compact
    Hausdorff space and $\mu$ be a locally finite Borel measure on $S$. Assume
    we are given a $C_0(S)$-morphism $\psi:{L_p(S,\mu)}_{ess}\to C_0(S)$ with
    $1\leq p\leq+\infty$, a function $[f]\in L_p(S,\mu)$ and a compact set
    $K\subset S$. Then
    \begin{enumerate}[label = (\roman*)]
        \item if $[f]=[f]\chi_K$, then $\psi(f)|_{S\setminus K}=0$;

        \item if $[f]=[f]\chi_K$ and $\psi(f)\neq 0$, then there is an open set
              $U\subset K$ of positive measure.
    \end{enumerate}
\end{proposition}
\begin{proof} $(i)$ By paragraph $(i)$ of proposition~\ref{LpEssC0ModCharac} we
    have $[f]=[f]\chi_K\in {L_p(S,\mu)}_{ess}$, so we can speak of the function
    $a=\psi(f)\in C_0(S)$. Let $V$ be an open set containing $K$, then there
    exists a continuous function $b\in C_0(S)$ such that $b|_K=1$ and
    $b|_{S\setminus V}=0$ [\cite{DalesBanSpContFunDualSp}, theorem 1.4.25]. By
    construction $\chi_K=b\chi_K$, so
    $a=\psi([f])=\psi([f]\chi_K)=\psi(b[f]\chi_K)=b\psi([f]\chi_K)=b\psi([f])=ba$.
    As $b|_{S\setminus V}=0$ we get $a|_{S\setminus V}=0$. Since $S$ is
    Hausdorff and $V$ is an arbitrary open set containing $K$, then
    $a|_{S\setminus K}=0$.

    $(ii)$ Using notation of the previous paragraph we have $a\neq 0$ and
    $a|_{S\setminus K}=0$. Consider non-negative continuous function $c=|a|$,
    then $c\neq 0$ and $c|_{S\setminus K}=0$. Since $c\neq 0$, then the open set
    $U=c^{-1}((0, +\infty))$ is non-empty. Moreover, $U\subset K$ as
    $c|_{S\setminus K}=0$. Now peek any $s\in U$. By construction $a(s)\neq 0$.
    Since $\{s\}$ is a compact set there exists a continuous function $e\in
        C_0(S)$ such that $e(s)=1$ and $e|_{S\setminus U}=0$
    [\cite{DalesBanSpContFunDualSp}, theorem 1.4.25]. Consider function
    $[g]=e[f]\in {L_p(S,\mu)}_{ess}$, then $\psi([g])\neq 0$ as
    $\psi([g])(s)=\psi(e[f])(s)=(e\psi([f]))(s)=e(s)\psi([f])(s)=a(s)\neq 0$.
    Since $\psi([g])\neq 0$, we have $[g]\neq 0$ in ${L_p(S,\mu)}_{ess}$.
    Therefore $\mu(U)>0$ because by construction $[g]\chi_{S\setminus U}=0$.
\end{proof}

\begin{proposition}\label{LpC0ModNecessCond} Let $S$ be a locally compact
    Hausdorff space and $\mu$ be a locally finite inner compact regular Borel
    measure on $S$. Let $1\leq p\leq +\infty$ and $L_p(S,\mu)$ is a relatively
    projective $C_0(S)$-module. Then,

    $(i)$ $\mu$ is inner open regular;

    $(ii)$ any atom of $\mu$ is an isolated point in $S$;

    $(iii)$ if $p<+\infty$ and $\mu$ is outer open regular then $\mu$ is purely
    atomic.
\end{proposition}
\begin{proof} $(i)$ Let $K\subset S$ be a compact set with of positive measure.
    Then by paragraph $(i)$ of proposition~\ref{LpEssC0ModCharac} the function
    $[f]:=[\chi_K]$ is non-zero in ${L_p(S,\mu)}_{ess}$. Since $C_0(S)$-module
    $L_p(S,\mu)$ is relatively projective, then by paragraph $(ii)$ of
    proposition~\ref{RelProjNecesCond} there exists a $C_0(S)$-morphism
    $\psi:L_p(S,\mu)\to C_0(S)$ such that $\psi([f])\neq 0$. Now from paragraph
    $(ii)$ of proposition~\ref{MorphLpEssC0Prop} we get that there exists an
    open set $U\subset K$ with $\mu(U)>0$. Since $K\subset S$ is arbitrary we
    are in position to apply paragraph $(ii)$ of
    proposition~\ref{NormalMeasCharac}. Hence
    $\mu(B)=\mu(\operatorname{int}_S(B))$ for any Borel set $B\subset S$. It
    remains to apply proposition~\ref{InnerOpenRegMeasCharac}.

    $(ii)$ Let $A$ be an atom of $\mu$. By paragraph $(ii)$ of
    proposition~\ref{MeasAtomCharac} there exists a point $s\in A$ such that
    $\mu(\{s\})=\mu(A)>0$. From paragraph $(i)$ it follows that
    $\mu(\operatorname{int}_S(\{s\}))=\mu(\{s\})>0$. Therefore $\{s\}$ is an
    open set, i.e. $s$ is an isolated point.

    $(iii)$ Let $S_a^\mu$ be the set of singleton atoms of $\mu$ and
    $S_c^\mu=S\setminus S_a^\mu$. By paragraph $(ii)$ all atoms are isolated, so
    $S_c^\mu$ is closed and hence Borel. Consider arbitrary compact subset
    $K\subset S_c^\mu$. Suppose $\mu(K)>0$, then by paragraph $(i)$ of
    proposition~\ref{LpEssC0ModCharac} the function $[f]:=[\chi_K]$ is non-zero
    in ${L_p(S,\mu)}_{ess}$. As $C_0(S)$-module $L_p(S, \mu)$ is relatively
    projective by paragraph $(ii)$ of propositions~\ref{RelProjNecesCond} and
    proposition~\ref{MorphLpEssC0Prop} we have a $C_0(S)$-morphism
    $\psi:{L_p(S,\mu)}_{ess}\to C_0(S)$ such that $\psi([f])\neq 0$ and
    $\psi([f])|_{S\setminus K}=0$. Denote $a:=\psi([f])\neq 0$. Since
    $a|_{S\setminus K}=0$, there exists a point $s\in K$ such that $a(s)\neq 0$.
    Fix $\varepsilon > 0$. Note that $s$ is not an atom because $s\in K\subset
        S_c^\mu$, hence from outer open regularity of $\mu$ we get an open set
    $W\subset S$ such that $s\in W$ and $\mu(W)<\varepsilon$. Since $\{s\}$ is
    compact, then there exists a continuous function $b\in C_0(S)$ such that
    $b(s)=1$, $0\leq b\leq 1$ and $b|_{S\setminus W}=0$
    [\cite{DalesBanSpContFunDualSp}, theorem 1.4.25]. As $p<+\infty$ we get
    $\Vert b[f]\Vert_{L_p(S,\mu)} \leq {\mu(W\cap K)}^{1/p}<\varepsilon^{1/p}$.
    Finally,
    $$
        |a(s)|=|a(s)b(s)|=|(ba)(s)|=|(b\psi([f]))(s)|=|\psi(b[f])(s)|
        \leq\Vert \psi (b[f])\Vert_{C_0(S)}\leq
    $$
    $$
        \leq\Vert\psi\Vert\Vert b[f]\Vert_{L_p(S,\mu)}
        \leq\Vert\psi\Vert\varepsilon^{1/p}
    $$
    Since $\varepsilon>0$ is arbitrary $|a(s)|=0$, but $a(s)\neq 0$ by choice of
    $s$. Contradiction, hence $\mu(K)=0$. As $K\subset S_c^\mu$ is arbitrary
    from inner compact regularity of $\mu$ we get $\mu(S_c^\mu)=0$. In other
    words $\mu$ is purely atomic.
\end{proof}

\begin{proposition}\label{C0ModLpRetrProj} Let $S$ be a locally compact
    Hausdorff space and $\mu$ be a Borel measure on $S$. Let $1\leq p\leq
        +\infty$ and $L_p(S,\mu)$ be a relatively projective $C_0(S)$-module. Then
    for any Borel set $B\subset S$ the $C_0(S)$-module $L_p(S,\mu^B)$ is
    relatively projective.
\end{proposition}
\begin{proof} The proof is the same as in proposition~\ref{BSigmaModLpRetrProj}.
\end{proof}

\begin{theorem}\label{ReflLpC0ModCrit} Let $S$ be a locally compact Hausdorff
    space and $\mu$ be a decomposable inner compact regular Borel measure on
    $S$. Let $1\leq p< +\infty$. Then the following are equivalent:
    \begin{enumerate}[label = (\roman*)]
        \item $L_p(S,\mu)$ is a relatively projective $C_0(S)$-module;
        \item $\mu$ is purely atomic and all atoms are isolated points.
    \end{enumerate}
\end{theorem}
\begin{proof} $(i)\implies (ii)$ Let $\mathcal{D}$ be a decomposition of $S$
    onto Borel subsets of finite measure. Fix $D\in\mathcal{D}$ and consider
    $C_0(S)$-module $L_p(S,\mu^D)$. Since $D$ has finite measure then $\mu^D$ is
    finite, inner compact regular and outer open regular. By
    proposition~\ref{C0ModLpRetrProj} the $C_0(S)$-module $L_p(S,\mu^D)$ is
    relatively projective. So from paragraph $(iii)$ of
    proposition~\ref{LpC0ModNecessCond} we get that $\mu^D$ (and a fortiori
    $\mu_D$) is purely atomic and all atoms are isolated points. Since
    $D\in\mathcal{D}$ is arbitrary, by proposition [\cite{FremMeasTh2}, exercise
            214X (i)] the measure $\mu$ is purely atomic and all atoms are isolated
    points.

    $(ii)\implies (i)$ Let $S_a^\mu$ denote the set of singleton atoms of $\mu$.
    Since all points in $S_a^\mu$ are isolated, then $S_a^\mu$ is discrete and
    $C_0(S_a^\mu)$ is biprojective [\cite{HelHomolBanTopAlg}, theorem 4.5.26].
    Since $p<+\infty$, $\mu$ is purely atomic and all atoms are isolated points
    the $C_0(S_a^\mu)$-module $L_p(S,\mu)$ is  essential. Bearing all this in
    mind from [\cite{HelBanLocConvAlg}, proposition VII.1.60(II)] we get that
    the $C_0(S_a^\mu)$-module $L_p(S,\mu)$ is relatively projective. Since
    $S_a^\mu$ is open in $S$, then $C_0(S_a^\mu)$ is a two-sided closed ideal of
    $C_0(S)$. Now by [\cite{RamsHomPropSemgroupAlg}, proposition 2.3.3(i)] the
    $C_0(S)$-module $L_p(S,\mu)$ is relatively projective.
\end{proof}

The case of $C_0(S)$-module $L_\infty(S,\mu)$ is much harder. We shall only give
two necessary but quite restrictive conditions for relative projectivity.

\begin{definition}\label{WideFamilyDef} Let $S$ be a locally compact Hausdorff
    space and $\mu$ be a Borel measure on $S$. A family $\mathcal{F}$ of Borel
    subsets of $S$ is called wide if
    \begin{enumerate}[label = (\roman*)]
        \item each element of $\mathcal{F}$ has finite positive measure and is
              contained in some compact subset;
        \item  any compact subset of $S$ intersects only finitely many sets of
              $\mathcal{F}$;
        \item  any two distinct sets in $\mathcal{F}$ do not intersect.
    \end{enumerate}
\end{definition}

\begin{proposition}\label{LInfEssNotCompl} Let $S$ be a locally compact
    Hausdorff space and $\mu$ be a Borel measure on $S$. If $S$ admits an
    infinite wide family $\mathcal{F}$ then the essential part of
    $C_0(S)$-module $L_\infty(S,\mu)$ is not complemented in $L_\infty(S,\mu)$.
\end{proposition}
\begin{proof} Assume that ${L_\infty(S,\mu)}_{ess}$ is complemented in
    $L_\infty(S,\mu)$, then there exists a bounded linear operator
    $P:L_\infty(S,\mu)\to {L_\infty(S,\mu)}_{ess}$ such that $P([f])=[f]$ for
    all $[f]\in {L_\infty(S,\mu)}_{ess}$. Now given a wide family
    $\mathcal{F}={(F_\lambda)}_{\lambda\in\Lambda}$ we define a bounded linear
    operator
    $$
        I:\ell_\infty(\Lambda)\to L_\infty(S,\mu):
        x\mapsto\biggl[\sum_{\lambda\in\Lambda}x_\lambda \chi_{F_\lambda}\biggr]
    $$
    which is a well because $\mathcal{F}$ is a disjoint famliy. Consider $x\in
        c_0(\Lambda)$. Fix $\varepsilon > 0$, then there exists a finite subset
    $\Lambda_0\subset\Lambda$ such that $|x_\lambda|<\varepsilon$ for all
    $\lambda\in\Lambda\setminus\Lambda_0$. Let $K_\lambda$ denote the
    compact set containing $F_\lambda$ for $\lambda\in\Lambda$. Then
    $K_0=\bigcup_{\lambda\in\Lambda_0}K_\lambda$ is a compact set. If $s\in
        S\setminus K$, then $\chi_{F_\lambda}(s)=0$ for all
    $\lambda\in\Lambda\setminus\Lambda_0$. Therefore $\Vert
        I(x)\chi_{S\setminus K}\Vert_{L_\infty(S,\mu)}
        =\Vert[\sum_{\lambda\in\Lambda\setminus\Lambda_0}x_\lambda\chi_{F_\lambda}]
        \Vert_{L_\infty(S,\mu)}
        =\sup_{\lambda\in\Lambda\setminus\Lambda_0}|x_\lambda|<\varepsilon$. Now
    by paragraph $(i)$ of proposition~\ref{LpEssC0ModCharac} we get that
    $I(x)\in {L_\infty(S,\mu)}_{ess}$. Next we define a bounded linear
    operator
    $$
        R:L_\infty(S,\mu)\to c_0(\Lambda):
        [f]\mapsto \biggl(
        \lambda\mapsto{\mu(F_\lambda)}^{-1}\int_{F_\lambda} f(s)d\mu(s)
        \biggr)
    $$
    The only thing that needs clarification is the fact that $R$ has range in
    $c_0(\Lambda)$. Fix $[f]\in {L_\infty(S,\mu)}_{ess}$. Let $\varepsilon>0$.
    By paragraph $(i)$ of proposition~\ref{LpEssC0ModCharac} there exists a
    compact set $K\subset S$ such that $\Vert
        [f]\chi_{K}\Vert_{L_\infty(S,\mu)}<\varepsilon$. Consider set
    $\Lambda_K=\{\lambda\in\Lambda: K\cap F_\lambda\neq\varnothing \}$. By
    definition of $\mathcal{F}$ the set $\Lambda_K$ is finite. For any
    $\lambda\in\Lambda\setminus \Lambda_K$ holds $F_\lambda\cap K=\varnothing$,
    so $|{R(f)}_\lambda|\leq\varepsilon$. Since $\varepsilon>0$ is arbitrary
    $R([f])\in c_0(\Lambda)$. Now we define a bounded linear operator $Q=RPI$.
    Recall that $P([f])=[f]$ for all $[f]\in {L_\infty(S,\mu)}_{ess}$. Then it
    is easy to check that for all $x\in c_0(\Lambda)$ and $\lambda\in\Lambda$
    holds ${Q(x)}_\lambda=x_\lambda$. Thus $Q:\ell_\infty(\Lambda)\to
        c_0(\Lambda)$ is a bounded linear operator such that $Q(x)=x$ for all $x\in
        c_0(\Lambda)$. Since $\Lambda$ is infinite we get a contradiction with
    Phillips' theorem~\cite{PhilOnLinTran}. Therefore, ${L_\infty(S,\mu)}_{ess}$
    is not complemented in $L_\infty(S,\mu)$.
\end{proof}

Now we need to remind some notions from general topology. A family $\mathcal{F}$
of subsets of a topological space $S$ is called locally finite if every point of
$S$ has an open neighbourhood that intersects only finitely many sets in
$\mathcal{F}$. A topological space $S$ is called pseudocompact if every locally
finite family of non-empty open sets is finite.

\begin{proposition}\label{LInfRelProjSuppCond} Let $S$ be an locally compact
    Hausdorff space and $\mu$ be a locally finite Borel measure on $S$. If
    $L_\infty(S,\mu)$ is relatively projective as $C_0(S)$-module then
    $\operatorname{supp}(\mu)$ is pseudocompact.
\end{proposition}
\begin{proof} Denote $M:=\operatorname{supp}(\mu)$. Assume that $M$ is not
    pseudocompact, then there is an infinite disjoint family $\mathcal{U}$ of
    non-empty open sets in $M$ which is locally finite. Since $S$ is locally
    compact for each $U\in\mathcal{U}$ we can choose a non-empty open set $V_U$
    and a compact set $K_U$ such that $V_U\subset K_U\subset  U$. We may choose
    $V$ such that $\mu(V)$ is finite as $\mu$ is locally finite. Even more,
    $\mu(V)>0$ since $V$ is an open subset of $M$. Clearly, the family
    $\mathcal{V}=\{V_U:U\in\mathcal{U}\}$ is infinite disjoint and locally
    finite. Hence for any $s\in S$ there exists on open set $W_s$ such that
    $s\in W_s$ and the set $\{V\in\mathcal{V}: V\cap W_s\neq\varnothing \}$ is
    finite.

    By construction $\mathcal{V}$ is a disjoint family of sets of positive
    finite measure each contained in some compact set. Let $K\subset S$ be an
    arbitrary compact set. Then $\{W_s:s\in K\}$ is an open cover of $K$. Since
    $K$ is compact there exists a finite set $S_0$ such that $\{W_s:s\in S_0\}$
    is a cover for $K$. Since each $W_s$ intersects only finitely many sets of
    $\mathcal{V}$ then so does $\bigcup_{s\in S_0}W_s$ and a fortiori so does
    $K$. Thus $\mathcal{V}$ is a wide family. By
    proposition~\ref{LInfEssNotCompl} the essential part of the $C_0(S)$-module
    $L_\infty(S,\mu)$ is not complemented in $L_\infty(S,\mu)$. Now from
    proposition~\ref{ProjModEssPartCompl} it follows that $L_\infty(S,\mu)$ is
    not a relatively projective $C_0(S)$-module. Contradiction, so $M$ is
    pseudocompact.
\end{proof}

\begin{theorem}\label{LInfReProjNecessCond} Let $S$ be a locally compact
    Hausdorff space and $\mu$ be a locally finite inner compact regular Borel
    measure on $S$. If $L_\infty(S,\mu)$ is relatively projective as
    $C_0(S)$-module then $\mu$ is inner open regular and
    $\operatorname{supp}(\mu)$ is pseudocompact.

\end{theorem}
\begin{proof} The result follows from propositions~\ref{LpC0ModNecessCond}
    and~\ref{LInfRelProjSuppCond}.
\end{proof}

Even though the last theorem is not a criterion we shall say a few words on how
that hypothetic criterion may look like. The last theorem puts restrictions on
the topology of underlying space $S$, but it cannot determine it completely.
Indeed, consider arbitrary locally compact Hausdorff space $S$ with at least one
isolated points $\{s\}$. Let $\mu$ be a point mass measure at $\{s\}$. It is
easy to check that the resulting $C_0(S)$-module $L_\infty(S,\mu)$ is relatively
projective. Thus we shall restrict our attention to strictly positive measures.

If $\mu$ is a strictly positive measure, then under assumptions of
proposition~\ref{LInfReProjNecessCond} the space $S$ is pseudocompact. Recall
that every continuous function on a pseudocompact space is bounded
    [\cite{HrusPseudCompTopSp}, theorem 1.1.3(3)]. Now note that any finite inner
open regular measure is residual, then by result of
    [\cite{ZindResMeasLocCompSp}, corollary 2.7] every measurable function on $S$ is
continuous at open dense set. These facts suggest that $S$ must have a peculiar
topology. Indeed, if a space $S$ has no isolated points and admits a non-zero
finite normal measure then $S$ cannot be separable locally compact Hausdorff
    [\cite{DalesBanSpContFunDualSp}, proposition 4.7.20], locally connected locally
compact Hausdorff [\cite{DalesBanSpContFunDualSp}, proposition 4.7.23],
connected locally compact Hausdorff $F$-space [\cite{DalesBanSpContFunDualSp},
proposition 4.7.24], separable metrizable [\cite{FlachNormMeasTopSp}, example
        1].

From previous discussion it is tempting to say that $C_0(S)$ ``looks like''
$L_\infty(S,\mu)$ whenever $\mu$ is strictly positive and $L_\infty(S,\mu)$ is
relatively projective. In this direction we have the following result.

\begin{proposition}\label{LInfReProjSuffCond} Let $S$ be a hyper-Stonean space
    and $\mu$ be a finite strictly positive normal inner compact regular Borel
    measure on $S$. Then $C_0(S)$-module $L_\infty(S,\mu)$ is relatively
    projective.
\end{proposition}
\begin{proof} By [\cite{DalesBanSpContFunDualSp}, corollary 4.7.6] spaces
    $L_\infty(S,\mu)$ and $C_0(S)$ are isomorphic as $C^*$-algebras. In
    particular $L_\infty(S,\mu)$ is isomorphic to $C_0(S)$ as $C_0(S)$-module.
    Since $S$ is compact then $C_0(S)$ is a unital algebra, so it is relatively
    projective as $C_0(S)$-module [\cite{HelBanLocConvAlg}, example VII.1.1].
\end{proof}

For strictly positive measures previous proposition is the only known example of
relatively projective $C_0(S)$-module $L_\infty(S,\mu)$.

\section{Funding}\label{SectionFunding} This research was supported by the
Russian Foundation for Basic Research (grant no. 19--01--00447).

\begin{thebibliography}{999}
    %
    \bibitem{HelHomolBanTopAlg}
    \textit{A. Ya. Helemskii} The homology of Banach and topological algebras,
    Springer, 41 (1989)
    %
    %
    \bibitem{SelivBiprojBanAlg}
    \textit{Yu. V. Selivanov} Biprojective Banach algebras, Math. USSR-Izv.,
    15:2 (1980), 387–399
    %
    %
    \bibitem{GrothProdTenTopNucl}
    \textit{A. Grothendieck} Produits tensoriels topologiques et espaces
    nucl{\'e}aires, Mem. Amer. Math. Soc., №16 (1955)
    %
    %
    \bibitem{FremMeasTh2}, \textit{D. H. Fremlin} Measure Theory, Vol. 2,
    {2003}, Torres Fremlin
    %
    %
    \bibitem{FremMeasTh4.1}, \textit{D. H. Fremlin} Measure Theory, Vol. 4(1),
    {2003}, Torres Fremlin
    %
    %
    \bibitem{DalesBanSpContFunDualSp}
    \textit{H. G. Dales, F. K. Dashiel Jr., A.T.-M. Lau, D. Strauss} Banach
    spaces of continuous functions as dual spaces, Berlin, Springer (2016)
    %
    %
    \bibitem{HelBanLocConvAlg}
    \textit{A.Ya. Helemskii}, Banach and locally convex algebras. Oxford
    University Press, (1993)
    %
    %
    \bibitem{RamsHomPropSemgroupAlg}
    \textit{P. Ramsden} Homological properties of semigroup algebras, The
    University of Leeds, PhD thesis (2009)
    %
    %
    \bibitem{PhilOnLinTran}
    \textit{R. S. Phillips} On linear transformations, Trans. Amer. Math. Soc.
    48 (1940), 516--541
    %
    %
    \bibitem{HrusPseudCompTopSp}
    \textit{M. Hrusak, A. Tamariz-Mascarua, M. Tkachenko} Pseudocompact
    topological spaces. Springer (2018)
    %
    %
    \bibitem{ZindResMeasLocCompSp}
    \textit{O. Zindulka} Residual measures in locally compact spaces. Topology
    and its Applications 108, no. 3 (2000), 253--265.
    %
    %
    \bibitem{FlachNormMeasTopSp}
    \textit{J. Flachsmeyer} Normal and category measures on topological spaces.
    General Topology and its Relations to Modern Analysis and Algebra (1972),
    109--116.
    %
\end{thebibliography}

Norbert Nemesh, Faculty of Mechanics and Mathematics, Moscow State University,
Moscow 119991 Russia

\textit{E-mail address:} nemeshnorbert@yandex.ru


\end{document}