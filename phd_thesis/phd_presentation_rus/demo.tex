% chktex-file 1
% chktex-file 19
\documentclass[9pt,pdf,utf8,russian]{beamer}

\usetheme[progressbar=frametitle]{metropolis}
\usepackage{appendixnumberbeamer}

\usepackage{booktabs}
\usepackage[scale=2]{ccicons}

\usepackage{pgfplots}
\usepgfplotslibrary{dateplot}

\usepackage{xspace}
\newcommand{\themename}{\textbf{\textsc{metropolis}}\xspace}

\usepackage[T2A]{fontenc}
\usepackage[utf8]{inputenc}
\usepackage[russian]{babel}

\usepackage{amssymb,amsmath}
\usepackage[utf8]{inputenc}
\usepackage{mathrsfs}
\usepackage[matrix,arrow,curve]{xy}

\title{Метрическая и топологическая проективность, инъективность и плоскость
банаховых модулей}
\author{Немеш Норберт Тиборович}
\date{23 декабря 2016 г.}
\institute{МГУ имени М.В. Ломоносова}
\begin{document}

\maketitle

\section{Определения}

\begin{frame}[fragile]{Проективность}
    \begin{block}{}
        Банахов $A$-модуль $P$ называется \textit{проективным}, если для
        любого \alert{допустимого} эпиморфизма $\xi:X\to Y$ и любого морфизма
        $\phi:P\to Y$ существует морфизм $\psi:P\to X$  делающий диаграмму
        \begin{table}
            \begin{tabular}{cc}
                $\xymatrix{
                                                       &
                {X} \ar[d]^{\xi}                       \\  % chktex 3
                {P} \ar@{-->}[ur]^{\psi} \ar[r]^{\phi} & % chktex 3
                {Y}}$                                  &
                \alert<2>{\only<2>{
                    $\Vert\phi\Vert=\Vert\psi\Vert$}}  \\
            \end{tabular}
        \end{table}
        коммутативной.
    \end{block}

    \pause

    Какие эпиморфизмы считать допустимыми?

    \begin{itemize}[<+- | alert@+>]
        \item Метрическая теория: $\xi$ --- строгая коизометрия,
              т.е. $\xi(B_X)=B_Y$
        \item Топологическая теория: $\xi$ --- открытое отображение
        \item Относительная теория: $\xi$ имеет дополняемое ядро
    \end{itemize}
\end{frame}

\begin{frame}[fragile]{Инъективность}
    \begin{block}{}
        Банахов $A$-модуль $J$ называется \textit{инъективным}, если для
        любого \alert{допустимого}
        мономорфизма $\xi:Y\to X$ и любого морфизма $\phi:Y\to J$ существует
        морфизм $\psi:X\to J$  делающий диаграмму
        \begin{table}
            \begin{tabular}{cc}
                $\xymatrix{
                    & {X} \ar@{-->}[dl]_{\psi}           \\% chktex 3
                {J} & {Y} \ar[l]_{\phi} \ar[u]_{\xi}}$ & % chktex 3
                \alert<2>{\only<2>{
                    $\Vert\phi\Vert=\Vert\psi\Vert$}}    \\
            \end{tabular}
        \end{table}
        коммутативной.
    \end{block}
    \pause

    Какие мономорфизмы считать допустимыми?

    \begin{itemize}[<+- | alert@+>]
        \item Метрическая теория: $\xi$ --- изометрия
        \item Топологическая теория: $\xi$ --- вложение с замкнутым образом
        \item Относительная теория: $\xi$ имеет дополняемый образ
    \end{itemize}
\end{frame}

\begin{frame}{Плоскость}
    \begin{block}{}
        Банахов $A$-модуль $F$ называется \textit{плоским}, если модуль
        $F^*$ инъективен.
    \end{block}
\end{frame}

\begin{frame}{Гомологическая теория банаховых пространств}
    \begin{itemize}
        \item Метрическая инъективность
              (Kelly, Nachbin, Goodner, Hasumi 1950--1958)
        \item Метрическая плоскость (Grothendieck, 1955)
        \item Топологическая проективность (Köthe, 1966)
        \item Топологическая плоскость (Retherford, 1972)
    \end{itemize}
\end{frame}

\section{Основные результаты}

\begin{frame}{Проективные идеалы}
    \begin{alertblock}{Теорема}
        Замкнутый идеал коммутативной банаховой алгебры,
        обладающий ограниченной аппроксимативной
        единицей топологически проективен тогда и только тогда,
        когда он обладает единицей.
    \end{alertblock}
    \pause
    \begin{alertblock}{Теорема}
        Замкнутый идеал коммутативной банаховой алгебры, обладающий
        \alert{сжимающей} аппроксимативной
        единицей \alert{метрически} проективен тогда и только тогда, когда
        он обладает единицей \alert{нормы 1}.
    \end{alertblock}
    \pause
    \begin{alertblock}{Теорема}
        Замкнутый левый идеал \alert{$C^*$-алгебры} метрически или
        топологически проективен тогда и только тогда, когда он обладает
        самосопряженной правой единицей.
    \end{alertblock}
\end{frame}

\begin{frame}{Плоские модули}
    \only<1>{
        \begin{block}{Определение (Lindenstrauss-Pełczyński, 1968)}
            Пространство $E$ называется $\mathscr{L}_p$-пространством если
            существует константа $C>0$ такая, что для любого конечномерного
            подпространства $F$ в $E$ существует конечномерное подпространство
            $G$ в $E$ $C$-изоморфное конечномерному $\ell_p$ пространству
            и содержащее $F$.
        \end{block}
        \begin{exampleblock}{Пример}
            \begin{itemize}
                \item $L_1\in\mathscr{L}_1$
                \item $C(K)\in\mathscr{L}_\infty$
            \end{itemize}
        \end{exampleblock}
    }
    \pause
    \begin{block}{Определение}
        Банахова алгебра $A$ называется аменабельной, если все её правые,
        левые и двусторонние	 модули \textit{относительно} плоские.
    \end{block}
    \pause
    \begin{alertblock}{Теорема}
        Над аменабельной банаховой алгеброй всякий банахов модуль,
        являющийся $\mathscr{L}_1$-пространством, топологически плоский.
    \end{alertblock}
    \pause
    \begin{block}{Теорема (J.R. Retherford, 1972)}
        $\mathscr{L}_1$-пространства --- это в точности топологически плоские
        банаховы \textit{пространства}.
    \end{block}
\end{frame}

\begin{frame}{Инъективные $C^*$-алгебры}
    \begin{block}{Определение (Dubinsky-Pełczyński-Rosenthal, 1972)}
        Говорят, что банахово пространство $E$ имеет свойство l.u.st. если
        $E^{**}$ изоморфно дополняемому подпространству некоторой
        банаховой решетки.
    \end{block}
    \pause
    \begin{alertblock}{Теорема}
        Если $C^*$-алгебра топологически инъективна как правый модуль над
        собой, то
        \begin{itemize}
            \item $A$ имеет свойство l.u.st;
            \item $A$ --- субоднородная $C^*$-алгебра;
            \item $A$ --- есть $*$-подалгебра в $M_n(C(K))$
        \end{itemize}
    \end{alertblock}
\end{frame}

\begin{frame}{Инъективные $AW^*$-алгебры}
    \begin{block}{Определение}
        $AW^*$ алгебра --- это $C^*$-алгебра в которой у любого подмножества
        правый алгебраический аннулятор порожден некоторой проекцией.
    \end{block}
    \pause
    \[
        W^*\subset AW^*\subset C^*
    \]
    \pause
    \begin{alertblock}{Теорема}
        $AW^*$-алгебра $A$ топологически инъективна как правый модуль над
        собой тогда и только тогда, когда
        \[
            A=\bigoplus_{i=1}^N M_{n_i}(C(K_i)),
        \]
        где $K_i$ --- стоуновы пространства.
    \end{alertblock}
\end{frame}

\begin{frame}{Свойство Данфорда-Петтиса}
    \begin{block}{Определение (Grothendieck, 1953)}
        Говорят, что банахово пространство $E$ имеет свойство Данфорда-Петтиса,
        если для любого банахова пространства $F$
        всякий слабо компактный оператор $T:E\to F$ будет вполне непрерывным.
    \end{block}
    \pause
    \begin{exampleblock}{Пример}
        \begin{table}
            \begin{tabular}{lr}
                Обладают      & Не обладают               \\
                \midrule
                $L_1$, $C(K)$ & рефлексивные пространства \\
            \end{tabular}
        \end{table}
    \end{exampleblock}
    \pause
    \begin{alertblock}{Теорема}
        Если банахова алгебра является $\mathscr{L}_1$- или
        $\mathscr{L}_\infty$-пространством,
        то все её топологически проективные, инъективные и плоские модули
        имеют свойство Данфорда-Петтиса.
    \end{alertblock}
\end{frame}

\begin{frame}{Маленькая категория}
    \alert{Итог}: большинство модулей гомологически нетривиальны.
    \pause

    \alert{Причина}: категория банаховых модулей очень большая.
    \pause
    \begin{exampleblock}{Пример маленькой категории}
        \[
            A=B(\Omega,\Sigma)
        \]
        \pause
        \[
            \operatorname{Ob}(\mathbf{C})=\{L_p(\Omega,\mu):
            1\leq p\leq\infty, \mu \mbox{ --- }\sigma\mbox{-аддитивная мера}\}
        \]
        \pause
        \[
            \operatorname{Hom}(\mathbf{C})=\{M_g:
            L_p(\Omega,\mu)\to L_q(\Omega,\nu):f\mapsto gf\}
        \]
    \end{exampleblock}
    \pause
    \begin{alertblock}{Теорема}
        В категории $\mathbf{C}$ все модули являются проективными,
        инъективными и плоскими в смысле метрической, топологической
        и относительной теории.
    \end{alertblock}
\end{frame}


\begin{frame}{Ссылки}
    \begin{itemize}
        \item Немеш Н. Метрически и топологически проективные идеалы банаховых
              алгебр // Матем. Заметки.  --- 2016.
              --- Т. 99, № 4. --- С. 526--533.
        \item Немеш Н. Топологически инъективные $C^*$-алгебры //
              Функц. анал. и прил. --- 2016. --- Т. 50, № 2. --- С. 88--91.
        \item Немеш Н. Гомологическая тривиальность категории модулей $L_p$ //
              Вест. Моск. ун-та. Сер. 1. Математика. Механика.
              --- 2016. Т. 71, № 4. --- С. 3--12.
    \end{itemize}
\end{frame}

%  Тема моей диссертации --- ``Метрическая и топологическая проективность,
%  инъективность и плоскость банаховых модулей''. Я провел исследование этих
%  двух версий банаховой гомологии и сравнил их с классической (так называемой
%  относительной) банаховой гомологией. Во всех трех теориях я изучал
%  проективные инъективные и плоские банаховы модули. Мы будем называть их
%  гомологически тривиальными.

%  ---------

%  Определяются они так

%  ---------

%  Модуль $P$ называется проективным если для каждого допустимого эпиморфизма
%  $\xi$ и любого морфизма $\phi$ должен существовать морфизм $\psi$ делающий
%  диаграмму коммутативной

%  ---------

%  Какие можно взять допустимые эпиморфизмы? В метрической теории это строгие
%  коизометрии, то есть операторы отображающие замкнутый единичный шар на
%  замкнутый единичный шар. Кстати, для метрической теории мы дополнительно
%  требуем равенство норм морфизмов $\phi$ и $\psi$.

%  ---------

%  В топологической это открытые отображения.

%  ---------

%  В относительной - отображения у которых ядро есть подпространство дополняемое
%  в смысле банаховой геометрии.

%  ---------

%  В случае с инъективностью мы рассматриваем допустимые мономорфизмы и все
%  стрелки в диаграмме меняют направление

%  ---------

%  Допустимые мономорфизмы для трех теорий соответственно изометрии,

%  ---------

%  вложения с замкнутым образом

%  ---------

%  и вложения с дополняемым образом.

%  ---------

%  Во всех трех теориях плоскость мы определяем через инъективность: модуль
%  называется плоским если сопряженный к нему инъективен

%  Как видно, метрическая и топологическая банахова гомология представляют
%  интерес потому что в них рассматриваются модули для которых разрешимы
%  максимально широкие классы задач подъема и продолжения морфизмов.

%  ---------

%  Для случая нулевой алгебры метрическая и топологическая теория вырождается в
%  гомологическую теорию банаховых пространств. Основные задачи в этой области
%  были решены в середине прошлого века Гротендиком, Кётэ, Келли, Нахбиным,
%  Резерфордом.

%  ---------

%  Перейдем к главным результатам диссертации.

%  ---------

%  Поиск проективных модулей естественно начать с идеалов. Здесь я доказал
%  следующую теорему

%  Замкнутый идеал {\it коммутативной} банаховой алгебры, обладающий
%  ограниченной аппроксимативной единицей топологически проективен тогда и
%  только тогда, когда он обладает единицей.

%  ---------

%  В метрическом случае надо потребовать чтобы аппроксимативная единица была
%  сжимающая, тогда идеал будет иметь единицу нормы 1:

%  Отмечу, что в относительной теории есть лишь необходимое условие
%  проективности идеала, а именно паракомпактность его спектра.

%  ---------

%  В некоммутативном случае удалось получить результат для $C^*$-алгебр, а
%  именно:

%  Левый замкнутый идеал $C^*$-алгебры метрически или топологически проективен
%  тогда и только тогда, когда он обладает самосопряженной правой единицей.

%  Опять же в относительной теории известно лишь, что все идеалы сепарабельных
%  $C^*$-алгебр проективны.

%  ---------

%  Чтобы сформулировать следующий результат мне нужно дать определение
%  пространств Линденштраусса-Пельчинского. Пространство $E$ называется
%  $\mathscr{L}_p$-пространством если существует константа $C>0$ такая что для
%  любого конечномерного подпространства $F$ в $E$ существует конечномерное
%  подпространство $G$ в $E$ $C$-изоморфное конечномерному $\ell_p$-пространству
%  и содержащее $F$.

%  Грубо говоря это пространства локально устроенны как $\ell_p$-пространства.

%  ---------

%  Всякое пространство непрерывных функций на компакте есть
%  $\mathscr{L}_\infty$-пространство, Всякое пространство функций интегрируемых
%  по Лебегу является $\mathscr{L}_1$ пространством

%  ---------

%  Еще одно определение, теперь из банаховой гомологии:

%  Банахова алгебра $A$ называется аменабельной, если все ее правые, левые и
%  двусторонние модули относительно плоские

%  В некотором смысле это гомологически лучшие алгебры, и класс таких алгебр
%  достаточно широк.

%  ---------

%  Итак, следующий мой результат это необходимое условие топологической
%  плоскости банахова модуля

%  Над аменабельной банаховой алгеброй всякий банахов модуль, являющийся
%  $\mathscr{L}_1$-пространством, топологически плоский.

%  ---------

%  Тут, кстати, стоит напомнить результат Резерфорда:

%  Все топологически плоские банаховы пространства это в точности
%  $\mathscr{L}_1$-пространства.

%  ---------

%  Перейдем к следующему примеру подтверждающему важную роль банаховой геометрии
%  для изучения инъективности.

%  Говорят, что банахово пространство $E$ имеет свойство l.u.st. если $E^{**}$
%  изоморфно дополняемому подпространству некоторой банаховой решетки.

%  ---------

%  Для меня это свойство важно потому что все $C^*$-алгебры топологически
%  инъективные над собой как правые модули имеют свойство l.u.st. Как следствие
%  они являются субоднородными, то есть могут быть представлены как инволютивные
%  подалгебры в матричных алгебрах с коэффициентами в алгебре непрерывных
%  функций.

%  ---------

%  Это было необходимое условие. Критерий топологической инъективности удалось
%  получить для $AW^*$-алгебр. $AW^*$ алгебра --- это $C^*$-алгебра в которой у
%  любого подмножества правый алгебраический аннулятор порожден некоторой
%  проекцией. Эти алгебры образуют промежуточный класс между $C^*$-алгебрами и
%  алгебрами фон Нойманна.

%  Теперь критерий:

%  $AW^*$-алгбера топологически инъективна как правый модуль над собой тогда и
%  только тогда, когда она является прямой суммой матричных алгебр с
%  коэффициентами в коммутативных $AW^*$-алгебрах

%  ---------

%  Еще один пример связан со свойством Данфорда-Петтиса.

%  Говорят, что банахово пространство $E$ имеет свойство Данфорда-Петтиса, если
%  для любого банахова пространства $F$ всякий слабо компактный оператор $T:E\to
%  F$ будет вполне непрерывным.

%  ----------

%  Это свойство определил Гротендик и доказал, что им обладают все $L_1$ и
%  $C(K)$-пространства, но им не обладает ни одно бесконечномерное рефлексивное
%  пространство. Я же доказал, что

%  ---------

%  Если банахова алгебра, является $\mathscr{L}_1$- или
%  $\mathscr{L}_\infty$-пространством, то все ее топологически проективные,
%  инъективные и плоские модули имеют свойство Данфорда-Петтиса.

%  Отмечу, что в относительной банаховой гомологии такое необходимое условие не
%  выполняется.

%  -----------

%  Из полученных результатов можно вывести (и это сделано в моей работе), что
%  большинство классических модулей анализа не являются гомологически
%  тривиальными ни в метрической ни в топологической теории.

%  -----------

%  Причина состоит в том что эти модули рассматриваются в очень большой
%  категории --- категории всех модулей

%  -----------

%  Но ситуация может кардинально измениться для маленьких категории. Рассмотрим
%  алгебру ограниченных измеримых функций на измеримом пространстве.

%  -----------

%  Рассмотрим категорию $C$ состоящую из $L_p$-пространств рассмотренных как
%  модули над этой алгеброй.

%  -----------

%  В этом случае морфизмы этой категории будут операторами умножения. Мне
%  удалось описать все допустимые операторы умножения в этой категории и в итоге
%  получился следующий результат:

%  -----------

%  В категории $\mathscr{C}$ все модули являются проективными, инъективными и
%  плоскими в смысле метрической, топологической и относительной теории.

\end{document}
