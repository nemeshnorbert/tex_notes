% chktex-file 19
% chktex-file 1

%%% Проверка используемого TeX-движка %%%
\usepackage{ifxetex}

%%% Поля и разметка страницы %%%
\usepackage{lscape}                                % Для включения альбомных страниц
\usepackage{geometry}                              % Для последующего задания полей
\usepackage{float}

%%% Математические пакеты %%%
%\usepackage{amsthm,amsfonts,amsmath,amssymb,amscd}  % Математические дополнения от AMS
\usepackage{amssymb,amsmath,amsthm}
\usepackage[utf8]{inputenc}
\usepackage{mathrsfs}
\usepackage[matrix,arrow,curve]{xy}

%%% Кодировки и шрифты %%%
\ifxetex  % chktex 1
    \usepackage{polyglossia}                         % Поддержка многоязычности
    \usepackage{fontspec}                            % TrueType-шрифты
\else
    \usepackage{cmap}                                % Улучшенный поиск русских слов в полученном pdf-файле
    \usepackage[T2A]{fontenc}                        % Поддержка русских букв
    \usepackage[utf8]{inputenc}                      % Кодировка utf8
    \usepackage[english, russian]{babel}             % Языки: русский, английский
    \IfFileExists{pscyr.sty}{\usepackage{pscyr}}{}   % Красивые русские шрифты
\fi

%%% Оформление абзацев %%%
\usepackage{indentfirst}                           % Красная строка
\usepackage{enumitem}

%%% Цвета %%%
\usepackage[usenames]{color}
\usepackage{color}
\usepackage{colortbl}

%%% Таблицы %%%
\usepackage{longtable}                             % Длинные таблицы
\usepackage{multirow,makecell,array}               % Улучшенное форматирование таблиц

%%% Общее форматирование
\usepackage[singlelinecheck=off,center]{caption}   % Многострочные подписи
\usepackage{soul}                                  % Поддержка переносоустойчивых подчёркиваний и зачёркиваний
\usepackage{icomma}                                % Запятая в десятичных дробях

%%% Библиография %%%
%\usepackage{cite}
\usepackage[backend=biber,style=gost-numeric,sorting=none,defernumbers=true]{biblatex}
\usepackage[autostyle]{csquotes}
\usepackage{fnpct}
\AdaptNoteOpt\footfullcite\multfootcite

%%% Гиперссылки %%%
\usepackage[linktocpage=true,plainpages=false,pdfpagelabels=false]{hyperref}

%%% Изображения %%%
\usepackage{graphicx}                              % Подключаем пакет работы с графикой

%%% Опционально %%%
% Следующий пакет может быть полезен, если надо ужать текст, чтобы сам текст не править, но чтобы места он занимал поменьше
%\usepackage{savetrees}

% Этот пакет может быть полезен для печати текста брошюрой
%\usepackage[print]{booklet}