% chktex-file 19 chktex-file 26
\subsection*{\Large Общая характеристика работы}
\fontsize{14pt}{15pt}\selectfont

\textbf{Актуальность темы.} В любой математической теории, где рассматриваются
множества с определенными структурами и отображения, должным образом реагирующие
на эти структуры, рано или поздно возникают задачи подъема и продолжения таких
отображений. На современном языке это задачи подъема и продолжения морфизмов в
тех или иных категориях. Объекты, из которых любой морфизм можно поднять вдоль
любого эпиморфизма, называются проективными, а объекты, у которых любой морфизм
действующий в них можно продолжить вдоль любого мономорфизма, называются
инъективными. В категориях, где есть разумные аналоги тензорного произведения,
например, в замкнутых моноидальных категориях, можно определить понятие плоского
объекта, то есть объекта, у которого соответствующий функтор тензорного
произведения сохраняет мономорфизмы. Понятия проективности, инъективности и
плоскости являются тремя китами, на которых покоится здание гомологической
алгебры. С момента своего возникновения в 50-х годах XX века эта ветвь
математики широко развилась и стала незаменимым инструментом в алгебраической
теории чисел, теории представлений, комплексном и функциональном анализе. Цель
данной диссертации --- изучение двух специальных типов гомологически тривиальных
объектов в категории банаховых модулей, точнее изучение метрически и
топологически проективных, инъективных и плоских банаховых модулей.

При попытке перенести понятия проективности, инъективности и плоскости из чистой
алгебры в функциональный анализ естественно возникли два подхода. В первом из
них, назовем его топологическим, от подъемов операторов и продолжений операторов
требовалась лишь непрерывность, а в роли эпиморфизмов и мономорфизмов
рассматривались открытые отображения и вложения с замкнутым образом
соответственно. Второй подход, назовем его метрическим, учитывал не только
топологическую структуру банаховых пространств, но и точное значение нормы: в
качестве эпиморфизмов рассматривались строгие фактор-отображения, а в качестве
мономорфизмов брали изометрические вложения.

Первый нетривиальный результат в этой области был получен в конце 20-хх годов XX
века. Это теорема Хана-Банаха
\footfullcite{HahnLinSystInLinSp}
\footfullcite{BanachOnLinFuncI}
\footfullcite{BanachOnLinFuncII},
которая утверждает метрическую инъективность $\mathbb{C}$ как банахова
пространства. Конечно, в то время гомологической алгебры не было и в помине.
Затем в 1950 году Нахбин \footfullcite{NachThOfHahnBanachType} и Гуднер
\footfullcite{GooProjInNorLinSp} , в предположении существования хотя бы одной
крайней точки в единичном шаре пространства, доказали, что все метрически
инъективные действительные банаховы пространства суть $C(K)$-пространства, для
некоторого стоунова пространства $K$. В 1952 году Келли
\footfullcite{KellBanSpWithExtProp} смог избавиться от этого предположения.
Наконец, в 1958 году Хасуми \footfullcite{HasumiExtPropComplBanSp} получил
описание метрически инъективных комплексных банаховых пространств. В этот
промежуток в 1955 году Гротендик описал \footfullcite{GrothMetrProjFlatBanSp}
метрически плоские банаховы пространства, они оказались изометрически изоморфны
$L_1$-пространствам. В 1966 Кёте доказал \footfullcite{KotheTopProjBanSp} , что
все топологически проективные банаховы пространства топологически изоморфны
$\ell_1$-пространствам. В 1972 году Стигал и Резерфорд
\footfullcite{StegRethNucOpL1LInfSp} показали, что топологически плоские
пространства --- это в точности $\mathscr{L}_1$-пространства, то есть
пространства локально устроенные так же как и $\ell_1$-пространства. Наконец, в
2013 Хелемский заметил \footfullcite{HelMetrFrQMod} , что из результатов
Гротендика легко получается описание метрически проективных банаховых
пространств --- они изометрически изоморфны $\ell_1$-пространствам. Единственная
до сих пор нерешенная задача --- это описание топологически инъективных
банаховых пространств. Некоторые специалисты оценивают ее как безнадежную
\footfullcite{OnSepInjBanSp} .

Параллельно с этим шло становление банаховой гомологии. В 1954 году Данфорд
\footfullcite{DunfSpecOp} показал взаимосвязь между расширениями банаховых
алгебр и спектральными операторами. Используя банахов аналог комплекса
Хохшильда, в 1962 году Камовиц \footfullcite{KamohomGrComBanAlg} определил
группы когомологий банаховой алгебры с коэффициентами в банаховом бимодуле.
Вскоре эта конструкция нашла применения в теории расширений и дифференцирований
банаховых алгебр
\footfullcite{GuichardetHomolCohomolBanAlg}
\footfullcite{JohnsonWeddebDecompBanAlgWithFinDimRad}.

В 1970 году Хелемским был предложен общий подход к гомологии банаховых алгебр
\footfullcite{HelemHomolDimNorModBanAlg} . Его идея состояла в построении
варианта относительной гомологической алгебры для категорий банаховых модулей.
Здесь под относительной гомологической алгеброй мы понимаем конструкцию
Эйленберга и Мура \footfullcite{EilenbergMooreFoundOfRelHomolAlg} , в которой
рассматриваются не все эпиморфизмы и мономорфизмы, а только некоторый выделенный
класс так называемых допустимых морфизмов. В случае банаховых модулей Хелемский
определил допустимые морфизмы как морфизмы банаховых модулей, обладающие
дополняемым (в смысле банаховой геометрии) ядром и образом. Как следствие,
возникли понятия относительно проективного, инъективного и плоского банахова
модуля. На категорию банаховых модулей были перенесены многие конструкции
гомологической алгебры, такие как резольвента, производный функтор, группы
когомологий. Построенную теорию стали называть относительной банаховой
гомологией.

Методы относительной банаховой гомологии позволили получить ряд результатов о
наличии аналитической структуры в спектре коммутативной полупростой банаховой
алгебры
\footfullcite{PugachProjAndFlatIdealOfBanAlg}
\footfullcite{PugachHmolPropFuncAlgAndAnalytPolyDiscs},
дать гомологическую интерпретацию таким топологическим понятиям как
дискретность, паракомпактность \footfullcite{HelemDescRelProjIdealCOmega} и
метризуемость
\footfullcite{KurmakDependStrctHomolDimOfCOmegaOnOmega}
\footfullcite{SelivanovHomolDimOfCyclMod}. В теории операторных алгебр были 
получены структурные теоремы о строении $C^*$- и $W^*$-алгебр
\footfullcite{HelemHomolEssenceConnAmen}
\footfullcite{HelemProjHomolClassifOfCStarAlg}
\footfullcite{HelemWedderburnTypeThForOpAlgAndMod}
и некоторых несамосопряженных операторных алгебр
\footfullcite{GolovibHomolPropHilbModOverNestOpAlg}
\footfullcite{GolovinSpatProjPropInClOfCSLAlg}. 
Пожалуй, самый известный результат здесь --- это теорема Джонсона
\footfullcite{JohnsonCohomolBanAlg}: локально компактная группа аменабельна
тогда и только тогда, когда ее сверточная алгебра относительно аменабельна.

Относительная банахова гомология стремительно развивалась, и вопросов всегда
было больше чем ответов. Поэтому другие варианты банаховой гомологий почти не
рассматривались. Под другими вариантами мы понимаем варианты относительной
гомологии банаховых модулей с иными классами допустимых морфизмов. Можно
рассматривать как более узкие, так и более широкие классы, чем класс
относительно допустимых морфизмов. Первые два кандидата --- это, конечно же,
допустимые морфизмы из метрической и топологической теории банаховых
пространств. Метрической банаховой гомологией назовем вариант относительной
гомологии банаховых модулей, где допустимые мономорфизмы --- это изометрические
морфизмы модулей, а допустимые эпиморфизмы --- это морфизмы модулей, являющиеся
строгими фактор-отображениями. Аналогично в топологической банаховой гомологии в
качестве допустимых мономорфизмов берутся морфизмы модулей, являющиеся
вложениями с замкнутым образом, а в качестве допустимых эпиморфизмов
рассматриваются морфизмы модулей, являющиеся открытыми отображениями.
Устоявшихся названий для этих версий банаховой гомологии еще нет.

Первые упоминания метрической банаховой гомологии можно найти уже в 1967 году в
работе Хаманы \footfullcite{HamInjEnvBanMod} . Точнее, Хамана исследовал только
метрическую инъективность банаховых модулей. Он дал определение инъективной
оболочки банахова модуля, доказал ее существование и единственность. Также он
доказал следующий критерий: унитальная $C^*$-алгебра метрически инъективна как
модуль над собой тогда и только тогда, когда она является коммутативной
$AW^*$-алгеброй. Более основательно метрическая банахова гомология была впервые
рассмотрена в 1979 году в работе Гравена \footfullcite{GravInjProjBanMod} . Он
определил понятия метрически проективного, инъективного и плоского банахова
модуля и описал простейшие свойства таких модулей. В качестве приложений он дал
критерии метрической проективности, инъективности и плоскости классических
модулей гармонического анализа над сверточной алгеброй. Уже в этой работе были
первые намеки на взаимосвязь между метрической банаховой гомологией и геометрией
банаховых пространств. Подход Гравена во многом схож с методами относительной
банаховой гомологи, но судя по аннотации к статье, сам Гравен, скорее всего, не
знал о существовании этого направления.

История топологической банаховой гомологии также началась с исследования
инъективности. В 1984 году Хелемский и Шейнберг
\footfullcite{HelemSheinbergFlatBanModAndAmenBanAlg} при изучении относительной
аменабельности банаховых алгебр дали определение топологически инъективного и
плоского банахова модуля. Они доказали критерий топологической плоскости
циклических модулей и дали достаточное условие топологической плоскости идеалов.
В следующий раз об этом направлении банаховой гомологии напомнил Уайт в 1995
году \footfullcite{WhiteInjmoduAlg} . Его определения строго проективного,
инъективного и плоского модуля учитывали нормы морфизмов, но по своей сути они
были эквиваленты определениям топологической проективности, инъективности и
плоскости. Уайт доказал базовые свойства этих модулей по аналогии с
относительной банаховой гомологией, дал количественный аналог теоремы Шейнберга
о топологической плоскости циклических модулей. Также Уайт провел исследование
некоторых гомологических свойств модулей над равномерными алгебрами.

В 2008 году Хелемский начал систематическое исследование гомологически
тривиальных объектов в метрической теории. В работах
\footfullcite{HelemNonMatrVersnExtPrincplByArvsnWttstck}
\footfullcite{HelMetrFlatNorMod}
\footfullcite{HelemExtrmVersnProjNorModOverSeqAlgs}
он дал описания метрически проективных и плоских объектов для некоторых
специальных категорий банаховых модулей. После этого, в работе
\footfullcite{HelMetrFrQMod} он предложил новый подход к доказательству базовых
теорем для различных версий банаховой гомологии. Идея состояла в определении
понятия проективности и свободы для так называемых оснащенных категорий. В
последствии метрическая, топологическая и относительная банахова гомология были
вписаны в эту общекатегорную схему.

В данной работе предпринята попытка собрать воедино все важные результаты и
провести полное исследование метрических и топологических гомологических свойств
трех типов модулей анализа: классических модулей над алгебрами ограниченных и
компактных операторов  на гильбертовом пространстве, модулей над алгебрами
ограниченных и исчезающих на бесконечности функций и модулей над сверточной
алгеброй и алгеброй мер локально компактной группы.

\textbf{Цель работы.} Целью данной работы является:
\begin{enumerate}
    \item Построение общей теории для изучения проективных, инъективных и
          плоских модулей в метрической и топологической банаховой гомологии.
    \item Исследование гомологических свойств классических модулей анализа.
    \item Сравнение метрической и топологической теории с уже хорошо изученной
          относительной банаховой гомологией.
\end{enumerate}

\textbf{Основные положения выносимые на защиту.} В диссертации получены
следующие результаты:
\begin{enumerate}
    \item Доказано, что замкнутый идеал коммутативной банаховой алгебры
          обладающий ограниченной аппроксимативной единицей топологически
          проективен тогда и только тогда, когда он обладает единицей.
          Аналогичное утверждение получено и для метрической проективности.
    \item Доказано, что для банаховой алгебры, являющейся $\mathscr{L}_1$- или
          $\mathscr{L}_\infty$-пространством, все ее топологически проективные,
          инъективные и плоские модули имеют свойство Данфорда-Петтиса.
    \item Доказано, что над относительно аменабельной банаховой алгеброй всякий
          банахов модуль, являющийся $\mathscr{L}_1$-пространством,
          топологически плоский.
    \item Дан критерий проективности идеалов $C^*$-алгебр, а именно, доказано,
          что левый замкнутый идеал $C^*$-алгебры метрически или топологически
          проективен тогда и только тогда, когда он обладает самосопряженной
          правой единицей.
    \item Получено описание $AW^*$-алгебр топологически инъективных над собой
          как правые модули. Все такие алгебры являются произведением конечного
          числа матричных алгебр с коэффициентами в коммутативных
          $AW^*$-алгебрах.
    \item Получено описание топологически инъективных, топологически
          сюръективных, изометрических и  коизометрических операторов умножения
          действующих между $L_p$-пространствами. Доказано, что топологически
          инъективные и изометрические операторы умножения обладают
          соответственно ограниченными и сжимающими левыми обратными операторами
          умножения, а топологически сюръективные и коизометрические обладают
          соответственно ограниченными и сжимающим правыми обратными операторами
          умножения.
\end{enumerate}

\textbf{Апробация работы.} Результаты диссертации докладывались на следующих
научных семинарах механико-математического факультета МГУ:

\begin{itemize}
    \item Научный семинар ``Алгебры в анализе'' под руководством профессора А.Я.
          Хелемского  (2014 — 2016 гг., неоднократно)
    \item Научный семинар ``Теория групп'' под руководством профессора А.Ю.
          Ольшанского и доцента А.А. Клячко (2016 г.)
\end{itemize}

\textbf{Основные методы исследования.} В диссертации используются методы
локальной теории банаховых пространств, теории операторных и банаховых алгебр,
гармонического анализа и теории меры. Помимо этого, применяются специфические
методы гомологической теории банаховых алгебр.

\textbf{Теоретическая и практическая ценность.} Работа носит теоретический
характер. Результаты и методы настоящей работы могут найти применение в
гомологической теории банаховых алгебр, геометрии банаховых пространств,
абстрактном гармоническом анализе и теории операторных алгебр.

\textbf{Публикации.} Основные результаты диссертации опубликованы в статьях
~\cite{NemMetTopProjIdBanAlg},~\cite{NemTopInjCStarAlg},
~\cite{NemHomolTrivCatModLp} в журналах из перечня рекомендованного ВАК. Работ в
соавторстве нет.

\textbf{Структура и объем работы.} Диссертация состоит из введения, 3 глав,
заключения и списка литературы из 101 наименования. Общий объем диссертации
составляет 118 страниц.

\subsection*{\Large Краткое содержание диссертации}
Во введении обсуждается история возникновения различных версий банаховой
гомологии, формулируются основные задачи, исследуемые в работе и перечисляются
главные результаты.

Глава 1 содержит предварительные сведения. В параграфе 1.1 перечислены все
необходимые результаты из геометрии банаховых пространств. Параграф 1.2 содержит
краткое введение в теорию банаховых алгебр и их модулей. В параграфе 1.3 дано
краткое введение в относительную банахову гомологию и перечислены главные
определения и факты из теории оснащенных категорий.

В главе 2 обсуждаются общие свойства метрически и топологически проективных,
инъективных и плоских модулей. В некоторых случаях получаются полные описания
таких модулей. Все эти результаты активно используются в следующей главе, где
речь идет о классических модулях анализа. Более детально структура главы 2
следующая.

Основываясь на общих теоремах теории оснащенных категорий, в параграфе 2.1
доказываются базовые свойства метрически и топологически проективных,
инъективных и плоских модулей. Изучаются различные категорные конструкции (такие
как произведение, копроизведение и тензорное произведение), сохраняющие эти три
свойства. Эти результаты используются для исследования метрической и
топологической проективности левых идеалов банаховых алгебр. Для произвольных
банаховых алгебр даются необходимые условия. Для случая коммутативных банаховых
алгебр, получается следующий критерий.

\begin{theorem*}[2.1.16] Если замкнутый идеал коммутативной банаховой алгебры
    обладает ограниченной аппроксимативной единицей, то он топологически
    проективен тогда и только тогда, когда он обладает единицей.
\end{theorem*}

В единые доказательства объединяются известные результаты о метрической и
топологической проективности и плоскости циклических модулей.

Параграф 2.2 посвящен банахово-геометрическим свойствам гомологически
тривиальных модулей в метрической и топологической теории. Здесь даются критерии
метрической и топологической проективности, инъективности и плоскости
аннуляторных модулей и устанавливается их тесная связь с метрически и
топологически проективными, инъективными и плоскими банаховыми пространствами.
Далее в работе дается несколько примеров, подтверждающих тезис: ``метрически и
топологически гомологически тривиальные модули над банаховой алгеброй схожи по
банахово-геометрическими свойствами со своей алгеброй''. Примеры включают
свойство быть $\mathscr{L}_1$-пространством, свойство Данфорда-Петтиса и
свойство l.u.st. Здесь следует выделить следующий результат.

\begin{theorem*}[2.2.13] Если банахова алгебра, является, как банахово
    пространство, $\mathscr{L}_1$- или $\mathscr{L}_\infty$-пространством, то
    все ее топологически проективные, инъективные и плоские модули имеют
    свойство Данфорда-Петтиса.
\end{theorem*}

В параграфе 2.3 перечисляются условия, при которых метрическая и топологическая
проективность, инъективность и плоскость сохраняются при переходе между модулем
над алгеброй и модулем над идеалом этой алгебры. В конце параграфа даются
необходимые и достаточные условия топологической плоскости банаховых модулей и
необходимые условия метрической и топологической инъективности двусторонних
идеалов банаховых алгебр. Главный результат параграфа звучит так.

\begin{theorem*}[2.3.9] Пусть $A$ --- относительно аменабельная банахова
    алгебра, и $F$ --- левый банахов $A$-модуль, являющийся, как банахово
    пространство, $\mathscr{L}_1$-пространством. Тогда $F$ --- топологически
    плоский $A$-модуль.
\end{theorem*}

В главе 3 общие результаты главы 2 применяются к классическим модулям анализа.

В параграфе 3.1 исследуется метрическая и топологическая проективность,
инъективность и плоскость идеалов $C^*$-алгебр. В этой части, по-видимому,
наиболее важны два следующих критерия.

\begin{theorem*}[3.1.4] Левый замкнутый идеал $C^*$-алгебры топологически
    проективен тогда и только тогда, когда он обладает самосопряженной правой
    единицей.
\end{theorem*}

\begin{theorem*}[3.1.11] $AW^*$-алгебра топологически инъективна как правый
    модуль над собой тогда и только тогда, когда она является произведением
    конечного числа матричных алгебр с коэффициентами в коммутативных
    $AW^*$-алгебрах.
\end{theorem*}

Также доказывается критерий метрической и топологической плоскости $C^*$-алгебры
как модуля над своим двусторонним идеалом. Эти критерии оказываются незаменимыми
средствами для описания в метрической и топологической теории некоторых
гомологически тривиальных модулей над алгебрами ограниченных и компактных
операторов на гильбертовом пространстве. Аналогичная классификация дается и в
коммутативном случае, то есть для алгебр ограниченных и исчезающих на
бесконечности функций заданных на некотором индексном множестве.

В параграфе 3.2 исследуется проективность, инъективность и плоскость
классических модулей гармонического анализа. Доказывается критерий
топологической плоскости левых идеалов сверточной алгебры вида
$L_1(G)\convol\mu$ для некоторой идемпотентной меры $\mu$. Благодаря
специфической банахово-геометрической структуре сверточной алгебры и алгебры мер
локально компактной группы устанавливается, что большинство классических модулей
гармонического анализа не являются гомологически тривиальными.

В параграфе 3.3 строится пример категории модулей ведущей себя совершенно иначе,
чем категории рассмотренные ранее. Эта категория состоит из пространств Лебега
со структурой банахова модуля над алгеброй измеримых ограниченных функций. Все
эти модули оказываются гомологически тривиальными по отношению к своей
категории. В основе этого результата лежит следующая теорема.

\begin{theorem*}[3.3.12, 3.3.16, 3.3.19, 3.3.24] Пусть $(\Omega,\Sigma,\mu)$ и
    $(\Omega,\Sigma,\nu)$ --- два $\sigma$-конечных пространства с мерой и пусть
    $1\leq p, q\leq+\infty$. Тогда ограниченный оператор умножения
    $M_g:L_p(\Omega,\mu)\to L_q(\Omega,\nu):f\mapsto gf$ является
    \begin{enumerate}[label = (\roman*)]
        \item топологически инъективным тогда и только тогда, когда он обладает
              левым обратным ограниченным оператором умножения;

        \item изометрическим тогда и только тогда, когда он обладает левым
              обратным сжимающим оператором умножения;

        \item топологически сюръективным тогда и только тогда, когда он обладает
              правым обратным ограниченным оператором умножения;

        \item коизометрическим тогда и только тогда, когда он обладает правым
              обратным изометрическим оператором умножения.
    \end{enumerate}
\end{theorem*}

\textbf{Заключение.} В диссертации была построена общая теория метрически и
топологически проективных, инъективных и плоских модулей. Было дано описание
широкого класса топологически проективных идеалов банаховых алгебр. На примере
свойств Данфорда-Петтиса и свойства l.u.st. была указана тесная взаимосвязь
банаховой геометрии и банаховой гомологии. В работе получены необходимые условия
для топологической плоскости банаховых модулей над относительно аменабельными
алгебрами и дан критерий топологической инъективности $AW^*$-алгебр. На основе
доказанных теорем были изучены гомологические свойства классических модулей над
алгебрами последовательностей, алгебрами операторов и групповыми алгебрами.

Скажем несколько слов о нерешенных проблемах и возможных направлениях
дальнейшего исследования. Среди нерешенных задач стоит отметить следующие две.

\begin{problem*} Существует ли необходимое условие топологической проективности
    идеала коммутативной банаховой алгебры более сильное, чем паракомпактность
    спектра?
\end{problem*}

Всякий топологически проективный идеал коммутативной банаховой алгебры
относительно проективен и поэтому обладает паракомпактным спектром. Как
показывает пример алгебры $A_0(\mathbb{D})$, компактность необходимым условием
не является.

\begin{problem*} Верно ли, что $C^*$-алгебры топологически инъективные над собой
    как правые модули являются $AW^*$-алгебрами?
\end{problem*}

В случае метрической инъективности это было доказано Хаманой. В случае
топологической инъективности это, похоже, будет сложной задачей даже в
коммутативном случае. На данный момент не известно ни одного примера
содержательной категории функционального анализа, где было бы получено полное
описание топологически инъективных объектов.

Направлений для дальнейшего исследования достаточно много. Например, в работе не
рассматриваются гомологические свойства бимодулей. Данное исследование
ограничивается только случаем банаховых модулей, а более общие нормированные
модули не рассматриваются. Наконец, отметим, что в работе рассматриваются только
гомологически тривиальные модули. Было бы интересно увидеть как резольвенты,
когомологии и глобальные размерности в топологической теории отличаются от своих
``относительных'' аналогов. Для метрической теории вместо стандартной техники
резольвент, скорее всего, придется применять расширения Йонеды.

\textbf{Благодарности.} Автор выражает глубокую благодарность своему научному
руководителю профессору Александру Яковлевичу Хелемскому за постановку
интересных задач и полезные обсуждения, и доценту Алексею Юльевичу Пирковскому
за ценные советы и замечания.

%\newpage
\renewcommand{\refname}{\Large Публикации автора по теме диссертации}
\printbibliography[keyword=phdresult]