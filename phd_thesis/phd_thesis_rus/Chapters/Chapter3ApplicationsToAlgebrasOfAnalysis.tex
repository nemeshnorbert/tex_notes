% Chapter Template

\chapter{Приложения к алгебрам анализа} % Main chapter title

\label{ChapterApplicationsToAlgebrasOfAnalysis} % Change X to a consecutive number; for referencing this chapter elsewhere, use \ref{ChapterX}

Грубо говоря, среди конкретных банаховых модулей существует три типа: модули с поточечным умножением, модули с композицией операторов в роли внешнего умножения и модули со сверткой. Мы исследуем главные примеры модулей этих типов. Следуя подходу Дэйлса и Полякова из \cite{DalPolHomolPropGrAlg}, мы систематизируем все результаты о классических модулях анализа, но в этот раз для метрической и топологической теории. Мы рассмотрим модули над $C^*$-алгебрами, модули над алгебрами последовательностей и, наконец, классические модули гармонического анализа.

%----------------------------------------------------------------------------------------
%	Applications to operator algebras
%----------------------------------------------------------------------------------------


\section{Приложения к модулям над \texorpdfstring{$C^*$}{C*}-алгебрами}
\label{SectionApplicationsToCStarAlgebras}

%----------------------------------------------------------------------------------------
%	Spatial modules
%----------------------------------------------------------------------------------------

\subsection{Пространственные модули}
\label{SubSectionSpatialModules}

Мы начнем с простейшего примера модулей над операторными алгебрами. По теореме Гельфанда-Наймарка [\cite{HelBanLocConvAlg}, теорема 4.7.57] для любой $C^*$-алгебры $A$ существует гильбертово пространство $H$ и изометрический ${}^*$-гомоморфизм $\varrho:A\to\mathcal{B}(H)$. Для гильбертовых пространств $H$, для которых существует такой гомоморфизм, мы можем рассмотреть  левый $A$-модуль $H_\varrho$ с внешним умножением определенным равенством $a\cdot x=\varrho(a)(x)$. Автоматически мы получаем структуру правого $A$-модуля на пространстве $H^*$, которое по теореме Рисса изометрически изоморфно $H^{cc}$. Этот изоморфизм позволяет определить структуру правого $A$-модуля на $H^{cc}$ с помощью равенства $\overline{x}\cdot a=\overline{\varrho(a^*)(x)}$. Такие модули мы будем называть пространственными. Их подробное исследование можно найти в работах Головина \cite{GolovibHomolPropHilbModOverNestOpAlg}, \cite{GolovinSpatProjPropInClOfCSLAlg}. В дальнейшем, для фиксированных $x_1,x_2\in H$ через $x_1\bigcirc x_2$ мы будем обозначать одномерный оператор $x_1\bigcirc x_2:H\to H:x\mapsto \langle x, x_2\rangle x_1$. 

\begin{proposition}\label{SpatModOverCStarAlgProp} Пусть $A$ --- $C^*$-алгебра и пусть $\varrho:A\to\mathcal{B}(H)$ --- изометрический ${}^*$-гомоморфизм такой, что его образ содержит подпространство одномерных операторов вида $\{x\bigcirc x_0:x\in H\}$ для некоторого ненулевого вектора $x_0\in H$. Тогда левый $A$-модуль $H_\varrho$ --- метрически проективный и плоский, а правый $A$-модуль $H_\varrho^{cc}$ --- метрически инъективный.
\end{proposition}
\begin{proof} Не теряя общности, мы можем считать, что $\Vert x_0\Vert=1$. Рассмотрим линейные операторы $\pi:A_+\to H_\varrho:a\oplus_1 z\mapsto \varrho(a)(x_0)+zx_0$ и $\sigma:H_\varrho\to A_+:x\mapsto \varrho^{-1}(x\bigcirc x_0)$. Прямая проверка показывает, что $\pi$ и $\sigma$ --- сжимающие $A$-морфизмы причем $\pi\sigma=1_{H_\varrho}$. Следовательно, $H_\varrho$ есть ретракт $A_+$ в $A-\mathbf{mod}_1$. Из предложений \ref{UnitalAlgIsMetTopProj} и \ref{RetrMetTopProjIsMetTopProj} следует, что $H_\varrho$ --- метрически проективный $A$-модуль. По предложению \ref{MetTopProjIsMetTopFlat} он также метрически плоский. Так как $H_\varrho^{cc}\isom{\mathbf{mod}_1-A}H_\varrho^*$, предложение \ref{DualMetTopProjIsMetrInj} дает метрическую инъективность $H_\varrho^{cc}$.
\end{proof}

В дальнейшем нам понадобится следующее простое следствие предыдущего предложения.

\begin{proposition}\label{FinDimNHModTopProjFlat} Пусть $H$ --- конечномерное гильбертово пространство. Тогда $\mathcal{N}(H)$ топологически проективный и плоский $\mathcal{B}(H)$-модуль.
\end{proposition}
\begin{proof} Из теоремы Шаттена-фон Нойманна [\cite{HelBanLocConvAlg}, предложение 0.3.38] мы знаем, что $\mathcal{N}(H)\isom{\mathbf{Ban}_1}H\projtens H^*$. Пусть $\varrho=1_{\mathcal{B}(H)}$, тогда можно утверждать чуть больше: $\mathcal{N}(H)\isom{\mathcal{B}(H)-\mathbf{mod}_1} H_\varrho\projtens H^*$. Так как $H^*$ конечномерно, то $H^*\isom{\mathbf{Ban}}\ell_1(\mathbb{N}_n)$ для $n=\dim(H)$ и, как следствие, $\mathcal{N}(H)\isom{\mathcal{B}(H)-\mathbf{mod}} H_\varrho\projtens\ell_1(\mathbb{N}_n)$. По предложению \ref{SpatModOverCStarAlgProp} модуль $H_\varrho$ топологически проективен, поэтому из следствия \ref{MetTopProjTensProdWithl1} мы получаем, что $\mathcal{N}(H)$ топологически проективен как $\mathcal{B}(H)$-модуль. Утверждение о топологической плоскости следует из предложения \ref{MetTopProjIsMetTopFlat}.
\end{proof}

%----------------------------------------------------------------------------------------
%	Projective ideals of C^*-algebras
%----------------------------------------------------------------------------------------

\subsection{Проективные идеалы \texorpdfstring{$C^*$}{C*}-алгебр}
\label{SubSectionProjectiveIdealsOfCStarAlgebras}

Изучение гомологически тривиальных идеалов $C^*$-алгебр мы начнем с проективности, но перед тем, как сформулировать главный результат, нам нужна подготовительная лемма.

\begin{lemma}\label{ContFuncCalcOnIdealOfCStarAlg} Пусть $I$ --- левый идеал унитальной $C^*$-алгебры $A$. Пусть $a\in I$ --- самосопряженный элемент, и пусть $E$ --- действительное подпространство исчезающих в нуле действительнозначных функций из $C(\operatorname{sp}_A(a))$. Тогда существует изометрический гомоморфизм $\operatorname{RCont}_a^0:E\to I$ корректно определенный равенством $\operatorname{RCont}_a^0(f)=a$, где $f:\operatorname{sp}_A(a)\to\mathbb{C}:t\mapsto t$.
\end{lemma}
\begin{proof} Через $\mathbb{R}_0[t]$ мы обозначим действительное линейное подпространство в $E$, состоящее из многочленов исчезающих в нуле. Так как $I$ --- левый идеал в $A$ и многочлен $p\in\mathbb{R}_0[t]$ не имеет свободного члена, то $p(a)\in I$. Следовательно, корректно определен $\mathbb{R}$-линейный гомоморфизм алгебр $\operatorname{RPol}_a^0:\mathbb{R}_0[t]\to I:p\mapsto p(a)$. Из непрерывного функционального исчисления для любого многочлена $p$ выполнено $\Vert p(a)\Vert=\Vert p|_{\operatorname{sp}_A(a)}\Vert_\infty$, поэтому $\Vert\operatorname{RPol}_a^0(p)\Vert=\Vert p|_{\operatorname{sp}_A(a)}\Vert_\infty$. Значит, $\operatorname{RPol}_a^0$ изометричен. Так как $\mathbb{R}_0[t]$ плотно в $E$ и $I$ полно, то $\operatorname{RPol}_a^0$ имеет изометрическое продолжение $\operatorname{RCont}_a^0:E\to I$, которое является $\mathbb{R}$-линейным гомоморфизмом. 
\end{proof}

Следующее доказательство основано на идеях Блечера и Каниа. В [\cite{BleKanFinGenCStarAlgHilbMod}, лемма 2.1] они доказали, что любой алгебраически конечно порожденный левый идеал $C^*$-алгебры является главным.  

\begin{theorem}\label{LeftIdealOfCStarAlgMetTopProjCharac} Пусть $I$ --- левый идеал $C^*$-алгебры $A$. Тогда следующие условия эквивалентны:

$i)$ $I=Ap$ для некоторого самосопряженного идемпотента $p\in I$;

$ii)$ $I$ --- метрически проективный $A$-модуль;

$iii)$ $I$ --- топологически проективный $A$-модуль.
\end{theorem}
\begin{proof} $i)$ $\implies$ $ii)$ Так как $p$ --- самосопряженный идемпотент, то $\Vert p\Vert=1$, поэтому из пункта $i)$ предложения \ref{UnIdeallIsMetTopProj} следует, что идеал $I$ метрически проективен как $A$-модуль.

$ii)$ $\implies$ $iii)$ Импликация следует из предложения \ref{MetProjIsTopProjAndTopProjIsRelProj}.

$iii)$ $\implies$ $i)$ По теореме 4.7.79 из \cite{HelBanLocConvAlg} мы знаем, что $I$ обладает некоторой правой сжимающей аппроксимативной единицей $(e_\nu)_{\nu\in N}$. Так как идеал $I$ имеет правую аппроксимативную единицу, то он является существенным левым $I$-модулем, и тем более существенным левым $A$-модулем. По предложению \ref{NonDegenMetTopProjCharac} морфизм $\pi_I$ имеет правый обратный $A$-морфизм $\sigma:I\to A\projtens \ell_1(B_I)$. Для каждого $d\in B_I$ рассмотрим $A$-морфизмы $p_d:A\projtens \ell_1(B_I)\to A:a\projtens \delta_x\mapsto \delta_x(d)a$ и $\sigma_d=p_d\sigma$. Тогда $\sigma(x)=\sum_{d\in B_I}\sigma_d(x)\projtens \delta_d$ для всех $x\in I$. Из отождествления $A\projtens\ell_1(B_I)\isom{\mathbf{Ban}_1}\bigoplus_1\{ A:d\in B_I\}$, для всех $x\in I$ мы имеем $\Vert\sigma(x)\Vert=\sum_{d\in B_I} \Vert\sigma_d(x)\Vert$. Так как $\sigma$ есть правый обратный морфизм к $\pi_I$, то $x=\pi_I(\sigma(x))=\sum_{d\in B_I}\sigma_d(x)d$ для всех $x\in I$. 

Для всех $x\in I$ мы имеем
$\Vert\sigma_d(x)\Vert=\Vert\sigma_d(\lim_\nu xe_\nu)\Vert=\lim_\nu\Vert x\sigma_d(e_\nu)\Vert \leq\Vert x\Vert\liminf_\nu\Vert\sigma_d(e_\nu)\Vert$, поэтому $\Vert\sigma_d\Vert\leq \liminf_\nu\Vert\sigma_d(e_\nu)\Vert$. Тогда для любого множества $S\in\mathcal{P}_0(B_I)$ выполнено
$$
\sum_{d\in S}\Vert \sigma_d\Vert
\leq \sum_{d\in S}\liminf_\nu\Vert \sigma_d(e_\nu)\Vert
\leq \liminf_\nu\sum_{d\in S}\Vert \sigma_d(e_\nu)\Vert
\leq \liminf_\nu\sum_{d\in B_I}\Vert \sigma_d(e_\nu) \Vert
$$
$$
=\liminf_{\nu}\Vert\sigma(e_\nu)\Vert
\leq \Vert\sigma\Vert\liminf_{\nu}\Vert e_\nu\Vert
\leq \Vert\sigma\Vert.
$$
Так как $S\in \mathcal{P}_0(B_I)$ произвольно, то сумма $\sum_{d\in B_I}\Vert\sigma_d\Vert$ конечна. Как следствие, сумма $\sum_{d\in B_I}\Vert\sigma_d\Vert^2$ тоже конечна. 

Теперь будем рассматривать алгебру $A$ как идеал в своей унитизации $A_\#$. Тогда $I$ также идеал в $A_\#$. Зафиксируем натуральное число $m\in\mathbb{N}$ и действительное число $\epsilon>0$. Тогда существует множество $S\in\mathcal{P}_0(B_I)$ такое, что $\sum_{d\in B_I\setminus S}\Vert\sigma_d\Vert<\epsilon$. Обозначим мощность этого множества через $N$. Рассмотрим положительный элемент $b=\sum_{d\in B_I}\Vert\sigma_d\Vert^2 d^*d\in I$. Из леммы \ref{ContFuncCalcOnIdealOfCStarAlg} мы знаем, что $b^{1/m}\in I$, поэтому $b^{1/m}=\sum_{d\in B_I}\sigma_d(b^{1/m})d$. Из непрерывного функционального исчисления следует, что $\Vert b^{1/m}\Vert=\sup_{t\in\operatorname{sp}_{A_\#}(b)} t^{1/m}\leq\Vert b\Vert^{1/m}$, тогда $\limsup_{m\to\infty}\Vert b^{1/m}\Vert\leq 1$. Следовательно, $\Vert b^{1/m}\Vert\leq 2$ для достаточно больших $m$. Положим $\varsigma_d:=\sigma_d(b^{1/m})$, $u:=\sum_{d\in S}\varsigma_d d$ и $v:=\sum_{d\in B_I\setminus S}\varsigma_d d$. Тогда 
$$
b^{2/m}=(b^{1/m})^*b^{1/m}=u^*u+u^*v+v^*u+v^*v.
$$
Ясно, что $\varsigma_d^*\varsigma_d\leq \Vert \varsigma_d\Vert^2 e_{A_\#}\leq \Vert \sigma_d\Vert^2\Vert b^{1/m}\Vert^2 e_{A_\#}\leq 4\Vert \sigma_d\Vert^2 e_{A_\#}$. Для любых $x,y\in A$ всегда выполнено $x^*x+y^*y-(x^*y+y^*x)=(x-y)^*(x-y)\geq 0$, и поэтому 
$$
d^*\varsigma_d^* \varsigma_c c+c^*\varsigma_c^* \varsigma_d d
\leq d^*\varsigma_d^*\varsigma_d d + c^*\varsigma_c^*\varsigma_c c
\leq 4\Vert \sigma_d\Vert^2 d^*d+4\Vert \sigma_c\Vert^2 c^*c
$$
для всех $c,d\in B_I$. Просуммируем эти неравенства по $c\in S$ и $d\in S$, тогда
$$
\begin{aligned}
\sum_{c\in S}\sum_{d\in S}c^*\varsigma_c^* \varsigma_d d
&=\frac{1}{2}\left(\sum_{c\in S}\sum_{d\in S}d^*\varsigma_d^* \varsigma_c c+\sum_{c\in S}\sum_{d\in S}c^*\varsigma_c^* \varsigma_d d\right)\\
&\leq\frac{1}{2}\left(4 N\sum_{d\in S} \Vert \sigma_d\Vert^2 d^*d+
4 N\sum_{c\in S} \Vert \sigma_c\Vert^2 c^*c\right)\\
&=4 N\sum_{d\in S} \Vert \sigma_d\Vert^2 d^*d.
\end{aligned}
$$
Следовательно,
$$
u^*u
=\left(\sum_{c\in S}\varsigma_c c\right)^*\left(\sum_{d\in S}\varsigma_d d\right)
=\sum_{c\in S}\sum_{d\in S}c^*\varsigma_c^* \varsigma_d d
\leq N\sum_{d\in S} 4\Vert \sigma_d\Vert^2 d^*d
\leq 4N b.
$$
Заметим, что
$$
\Vert u\Vert
\leq \sum_{d\in S}\Vert\varsigma_d\Vert\Vert d\Vert
\leq \sum_{d\in S}2\Vert\sigma_d\Vert
\leq 2\Vert\sigma\Vert,
\qquad
\Vert v\Vert
\leq \sum_{d\in B_I\setminus S}\Vert\varsigma_d\Vert\Vert d\Vert
\leq \sum_{d\in B_I\setminus S}2\Vert\sigma_d\Vert
\leq 2\epsilon;
$$
поэтому $\Vert u^*v+v^*u\Vert\leq 8\Vert\sigma\Vert\epsilon$ и $\Vert v^*v\Vert\leq 4\epsilon^2$. Так как $u^*v+v^*u$ и $v^*v$ ---  самосопряженные элементы, то $u^*v+v^*u\leq 8\Vert\sigma\Vert\epsilon e_{A_\#}$ и $v^*v\leq 4\epsilon^2 e_{A_\#}$
Таким образом, для любого $\epsilon>0$ и достаточно большого $m$ выполнено 
$$
b^{2/m}
=u^*u+u^*v+v^*u+v^*v
\leq 4Nb+\epsilon(8\Vert\sigma\Vert+4\epsilon)e_{A_\#}.
$$

Другими словами, $g_m(b)\geq 0$ для непрерывной функции $g_m:\mathbb{R}_+\to\mathbb{R}:t\mapsto 4Nt+\epsilon(8\Vert\sigma\Vert+4\epsilon)-t^{2/m}$. Теперь выберем $\epsilon>0$ так, чтобы $M:=\epsilon(8\Vert\sigma\Vert+4\epsilon)<1$. По теореме об отображении спектра [\cite{HelLectAndExOnFuncAn}, теорема 6.4.2] мы получаем $g_m(\operatorname{sp}_{A_\#}(b))=\operatorname{sp}_{A_\#}(g_m(b))\subset\mathbb{R}_+$. Легко проверить, что $g_m$ имеет только одну точку экстремума $t_{0,m}=(2Nm)^{\frac{m}{2-m}}$, где она достигает минимума. Так как $\lim_{m\to\infty} g_m(t_{0,m})=M-1<0$, $g_m(0)=M>0$ и $\lim_{t\to\infty} g_m(t)=+\infty$, то для достаточно больших $m$ функция $g_m$ имеет ровно два корня: $t_{1,m}\in(0,t_{0,m})$ и $t_{2,m}\in(t_{0,m},+\infty)$. Следовательно, решением неравенства $g_m(t)\geq 0$ будет $t\in[0,t_{1,m}]\cup[t_{2,m},+\infty)$. Значит, $\operatorname{sp}_{A_\#}(b)\subset[0,t_{1,m}]\cup[t_{2,m},+\infty)$ для всех достаточно больших $m$. Так как $\lim_m t_{0,m}=0$, то так же $\lim_m t_{1,m}=0$. Заметим, что $g_m(1)=4N+M-1>0$ для достаточно больших $m$, и поэтому $t_{2,m}\leq 1$. Следовательно, $\operatorname{sp}_{A_\#}(b)\subset\{0\}\cup[1,+\infty)$.

Рассмотрим непрерывную функцию $h:\mathbb{R}_+\to\mathbb{R}:t\mapsto\min(1, t)$. Тогда по лемме \ref{ContFuncCalcOnIdealOfCStarAlg} мы получаем идемпотент $p=h(b)=\operatorname{RCont}_b^0(h)\in I$, такой, что $\Vert p\Vert=\sup_{t\in\operatorname{sp}_{A_\#}(b)}|h(t)|\leq 1$. Следовательно, $p$ --- самосопряженный идемпотент. Так как $h(t)t=th(t)=t$ для всех $t\in \operatorname{sp}_{A_\#}(b)$, то $bp=pb=b$. Последнее равенство влечет
$$
0=(e_{A_\#}-p)b(e_{A_\#}-p)=\sum_{d\in B_I}(\Vert\sigma_d\Vert d(e_{A_\#}-p))^*\Vert\sigma_d\Vert d(e_{A_\#}-p).
$$
Так как правая часть этого равенства неотрицательна, то $d=dp$ для всех $d\in B_I$, для которых $\sigma_d\neq 0$. Наконец, для всех $x\in I$ мы получаем $xp=\sum_{d\in B_I}\sigma_d(x)dp=\sum_{d\in B_I}\sigma_d(x)d=x$, то есть $I=Ap$ для некоторого самосопряженного идемпотента $p\in I$.
\end{proof}

Следует отметить, что в относительной теории нет аналогичного описания относительной проективности левых идеалов $C^*$-алгебр. Правда, известно, что для случая сепарабельных $C^*$-алгебр (то есть для $C^*$-алгебр сепарабельных как банахово пространство) все левые идеалы относительно проективны. В [\cite{LykProjOfBanAndCStarAlgsOfContFld}, параграф 6] можно найти хороший обзор последних результатов на эту тему.

\begin{corollary}\label{BiIdealOfCStarAlgMetTopProjCharac} Пусть $I$ --- двусторонний идеал $C^*$-алгебры $A$. Тогда следующие условия эквивалентны:

$i)$ $I$ унитален;

$ii)$ $I$ метрически проективен как $A$-модуль;

$iii)$ $I$ топологически проективен как $A$-модуль.
\end{corollary}
\begin{proof} Идеал $I$ имеет сжимающую аппроксимативную единицу [\cite{HelBanLocConvAlg}, теорема 4.7.79]. Следовательно, $I$ имеет правую единицу тогда и только тогда, когда он унитален. Теперь все эквивалентности следуют из теоремы \ref{LeftIdealOfCStarAlgMetTopProjCharac}. 
\end{proof}

\begin{corollary}\label{IdealofCommCStarAlgMetTopProjCharac} Пусть $L$ --- хаусдорфово локально компактное пространство, и пусть $I$ --- идеал в $C_0(L)$. Тогда следующие условия эквивалентны:

$i)$ $\operatorname{Spec}(I)$ компактен;

$ii)$ $I$ метрически проективный $C_0(L)$-модуль;

$iii)$ $I$ топологически проективный $C_0(L)$-модуль.
\end{corollary}
\begin{proof} По теореме Гельфанда-Наймарка $I\isom{\mathbf{Ban}_1}C_0(\operatorname{Spec}(I))$; следовательно, идеал $I$ полупрост. Отсюда, в силу теоремы Шилова об идемпотентах, идеал $I$ унитален тогда и только тогда, когда $\operatorname{Spec}(I)$ компактен. Осталось применить следствие \ref{BiIdealOfCStarAlgMetTopProjCharac}. 
\end{proof}

Отметим, что класс относительно проективных идеалов в $C_0(L)$ намного шире. Известно, что идеал $I$ в алгебре $C_0(L)$ относительно проективен тогда и только тогда, когда $\operatorname{Spec}(I)$ паракомпактен [\cite{HelHomolBanTopAlg}, глава IV,\S\S 2-3].

%----------------------------------------------------------------------------------------
%	Injective ideals of C^*-algebras
%----------------------------------------------------------------------------------------

\subsection{Инъективные идеалы \texorpdfstring{$C^*$}{C*}-алгебр}
\label{SubSectionInjectiveIdealsOfCStarAlgebras}

Перейдем к обсуждению инъективности двусторонних идеалов $C^*$-алгебр. К сожалению, мы не получим полного их описания, но приведем много примеров и некоторые необходимые условия. 

Заметим, что двусторонний идеал $I$ в $C^*$-алгебре $A$ сам является $C^*$-алгеброй с сжимающей аппроксимативной единицей [\cite{HelBanLocConvAlg}, теорема 4.7.79]. Следовательно, $I$ верен как $I$-модуль. Теперь из предложения \ref{ReduceInjIdToInjAlg} мы получаем, что $I$ топологически инъективен как $A$-модуль тогда и только тогда, когда $I$ топологически инъективен как $I$-модуль. Следовательно, при рассмотрении идеалов мы можем ограничиться рассмотрением $C^*$-алгебр $\langle$~метрически / топологически~$\rangle$ инъективных над собой как правые модули.

Нам необходимо напомнить несколько фактов об $AW^*$-алгебрах, так как в этом параграфе они играют ключевую роль. В попытках дать чисто алгебраическое определение для $W^*$-алгебр в \cite{KaplProjInBanAlg} Капланский определил этот класс $C^*$-алгебр. Алгебра $A$ называется $AW^*$-алгеброй, если это $C^*$-алгебра, такая, что для любого подмножества $S\subset A$ правый алгебраический аннулятор $\operatorname{r.ann}_A(S)=\{y\in A: Sy=\{0\}\}$ имеет вид $pA$ для некоторого самосопряженного идемпотента $p\in A$. Этот класс содержит все $W^*$-алгебры, но он строго больше. Заметим, что в случае коммутативных $C^*$-алгебр свойство быть $AW^*$-алгеброй эквивалентно тому, что $\operatorname{Spec}(A)$ является стоуновым пространством [\cite{BerbBaerStarRings}, теорема 1.7.1]. Главные результаты об $AW^*$-алгебрах и более общих бэровских ${}^*$-кольцах можно найти в \cite{BerbBaerStarRings}. 

Следующее предложение есть комбинация результатов Хаманы и Такесаки.

\begin{proposition}[Хамана, Такесаки]\label{MetInjCStarAlgCharac} $C^*$-алгебра метрически инъективна как правый модуль над собой тогда и только тогда, когда она является коммутативной $AW^*$-алгеброй.
\end{proposition}
\begin{proof} Если алгебра $A$ метрически инъективна как $A$-модуль, то по предложению \ref{MetTopInjOfId} она имеет левую единицу нормы $1$. Так как $A$ также обладает сжимающей аппроксимативной единицей  [\cite{HelBanLocConvAlg}, теорема 4.7.79], то $A$ унитальна. Теперь из результата Хаманы  [\cite{HamInjEnvBanMod}, предложение 2] алгебра $A$ есть коммутативная $AW^*$-алгебра. Хотя Хамана доказал этот факт для левых модулей, его докзательство легко модифицировать и для случая правых модулей.

Обратную импликацию доказал Такесаки в [\cite{TakHanBanThAndJordDecomOfModMap}, теорема 2]. Хотя в работе рассматривались двусторонние модули, рассуждение для правых модулей точно такое же.
\end{proof}

Осталось рассмотреть топологическую инъективность. Как показывает следующее предложение, топологически инъективные $C^*$-алгебры, говоря нестрого, не так уж далеки от коммутативных. Это предложение использует банахово-геометрическое свойство l.u.st. Его определение можно найти в параграфе \ref{SubSectionHomologicallyTrivialModulesOverBanachAlgebrasWithSpecificGeometry}.

\begin{proposition}\label{TopInjIdHaveLUST} Пусть $A$ --- $C^*$-алгебра, топологически инъективная как $A$-модуль. Тогда $A$ обладает свойством l.u.st и, как следствие, не может содержать $\mathcal{B}(\ell_2(\mathbb{N}_n))$ как ${}^*$-подалгебру для достаточно большого $n\in\mathbb{N}$.
\end{proposition}
\begin{proof} По теореме Гельфанда-Наймарка [\cite{HelBanLocConvAlg}, теорема 4.7.57] существует гильбертово пространство $H$ и изометрический ${}^*$-гомоморфизм $\varrho:A\to\mathcal{B}(H)$. Обозначим $\Lambda:=B_{H_\varrho^{cc}}$. Легко проверить, что оператор 
$$
\rho:A\to\bigoplus\nolimits_\infty\{H_\varrho^{cc}:\overline{x}\in \Lambda\}:a\mapsto \bigoplus\nolimits_\infty\{\overline{x}\cdot a:\overline{x}\in \Lambda\}
$$
является изометрическим морфизмом правых $A$-модулей. Так как $A$ топологически инъективна как $A$-модуль, то $\rho$ имеет правый обратный $A$-морфизм $\tau$. Следовательно, $A$ дополняемо в $E:=\bigoplus_\infty\{H_\varrho^{cc}:\overline{x}\in \Lambda\}$ посредством проектора $\rho\tau$. Заметим, что $H_{\varrho}^{cc}$, как любое гильбертово пространство, является банаховой решеткой, поэтому $E$ тоже банахова решетка. Как любая банахова решетка $E$ имеет свойство l.u.st [\cite{DiestAbsSumOps}, теорема 17.1], и, следовательно, это свойство имеет $A$, так как свойство l.u.st наследуется дополняемыми подпространствами.

Допустим, $A$ содержит $\mathcal{B}(\ell_2(\mathbb{N}_n))$ как ${}^*$-подалгебру для произвольного $n\in\mathbb{N}$. На самом деле, такая копия алгебры $\mathcal{B}(\ell_2(\mathbb{N}_n))$ необходимо $1$-дополняема в $A$ [\cite{LauLoyWillisAmnblOfBanAndCStarAlgsOfLCG}, лемма 2.1]. Следовательно, для локальных безусловных констант выполнено неравенство $\kappa_u(\mathcal{B}(\ell_2(\mathbb{N}_n)))\leq \kappa_u(A)$. По теореме 5.1 из \cite{GorLewAbsSmOpAndLocUncondStrct} мы знаем, что $\lim_n \kappa_u(\mathcal{B}(\ell_2(\mathbb{N}_n)))=+\infty$, поэтому $\kappa_u(A)=+\infty$. Это противоречит тому, что $A$ обладает свойством l.u.st. Значит, $A$ не может содержать $\mathcal{B}(\ell_2(\mathbb{N}_n))$ как ${}^*$-подалгебру для достаточно большого $n\in\mathbb{N}$.
\end{proof}

Напомним важную в теории $C^*$-алгебр конструкцию матричной алгебры.
 Для заданной $C^*$-алгебры $A$ через $M_n(A)$ мы обозначим линейное пространство матриц размера $n\times n$ со значениями в $A$. Это линейное пространство можно наделить структурой ${}^*$-алгебры с инволюцией и умножением с помощью равенств
$$
(ab)_{i,j}=\sum_{k=1}^n a_{i,k}b_{k,j},
\qquad\qquad
(a^*)_{i,j}=(a_{j,i}^*)
$$ 
для всех $a,b\in M_n(A)$ и $i,j\in\mathbb{N}_n$. Существует единственная норма на $M_n(A)$, которая превращает ее в $C^*$-алгебру [\cite{MurphyCStarAlgsAndOpTh}, теорема 3.4.2]. Очевидно, $M_n(\mathbb{C})$ изометрически изоморфна как ${}^*$-алгебра алгебре $\mathcal{B}(\ell_2(\mathbb{N}_n))$. Из [\cite{MurphyCStarAlgsAndOpTh}, замечание 3.4.1] следует, что естественные вложения
$i_{k,l}:A\to M_n(A):a\mapsto(a\delta_{i,k}\delta_{j,l})_{i,j\in\mathbb{N}_n}$ и проекции $\pi_{k,l}:M_n(A)\to A:a\mapsto a_{k,l}$ непрерывны. Следовательно, для заданного непрерывного оператора $\phi:A\to B$ между $C^*$-алгебрами $A$ и $B$ линейный оператор 
$$
M_n(\phi):M_n(A)\to M_n(B):a\mapsto (\phi(a_{i,j}))_{i,j\in\mathbb{N}_n}
$$ 
также непрерывен. Более того, если $\phi$ --- $A$-морфизм, то $M_n(\phi)$ есть $M_n(A)$-морфизм. Наконец, упомянем два изометрических изоморфизма связанных с матричными алгебрами:
$$
M_n\left(\bigoplus\nolimits_\infty\{A_\lambda:\lambda\in\Lambda\}\right)
\isom{\mathbf{Ban}_1}
\bigoplus\nolimits_\infty\{M_n\left(A_\lambda\right):\lambda\in\Lambda\},
\qquad
M_n(C(K))\isom{\mathbf{Ban}_1}C(K,M_n(\mathbb{C})).
$$

Как показывает предложение \ref{TopInjIdHaveLUST}, топологически инъективные над собой $C^*$-алгебры не могут содержать $\mathcal{B}(\ell_2(\mathbb{N}_n))$ как ${}^*$-подалгебру для достаточно большого $n\in\mathbb{N}$. $C^*$-алгебры не содержащие $\mathcal{B}(\ell_2(\mathbb{N}_n))$ как ${}^*$-подалгебру для достаточно большого $n\in\mathbb{N}$ называются субоднородными. Они являются замкнутыми ${}^*$-подалгебрами [\cite{BlackadarOpAlg}, предложение IV.1.4.3] алгебр $M_n(C(K))$ для некоторого компактного хаусдорфова пространства $K$ и некоторого натурального числа $n$. Больше подробностей о субоднородных $C^*$-алгебрах можно найти в [\cite{BlackadarOpAlg}, параграф IV.1.4]. 

Приведем два важных примера некоммутативных $C^*$-алгебр топологически инъективных над собой.

\begin{proposition}\label{FinDimBHModTopInj} Пусть $H$ --- конечномерное гильбертово пространство. Тогда пространство $\mathcal{B}(H)$ --- топологически инъективный $\mathcal{B}(H)$-модуль. 
\end{proposition}
\begin{proof} Напомним, что $\mathcal{B}(H)\isom{\mathbf{mod}_1-\mathcal{B}(H)}\mathcal{N}(H)^*$. Теперь результат немедленно следует из предложений \ref{FinDimNHModTopProjFlat} и \ref{DualMetTopProjIsMetrInj}.
\end{proof}

\begin{proposition}\label{CKMatrixModTopInj} Пусть $K$ --- стоуново пространство и $n\in\mathbb{N}$, тогда пространство $M_n(C(K))$ --- топологически инъективный $M_n(C(K))$-модуль.
\end{proposition}
\begin{proof} Для заданного $s\in K$ через $\mathbb{C}_s$ мы будем обозначать правый $C(K)$-модуль $\mathbb{C}$ с внешним умножением определенным равенством $z\cdot a=a(s)z$ для всех $a\in C(K)$ и $z\in\mathbb{C}_s$. Аналогично, через $M_n(\mathbb{C}_s)$ мы обозначим правый банахов $M_n(C(K))$-модуль $M_n(\mathbb{C})$ с внешним умножением определенным равенством
$$
(x\cdot a)_{i,j}=\sum_{k=1}^n x_{i,k}a_{k,j}
$$
для каждого $a\in M_n(C(K))$ и $x\in M_n(\mathbb{C}_s)$. Известно, что $C^*$-алгебра $M_n(C(K))$ ядерна [\cite{BroOzaCStarAlgFinDimApprox}, следствие 2.4.4], тогда из [\cite{HaaNucCStarAlgAmen}, теорема 3.1] следует, что она относительно аменабельна и даже $1$-аменабельна [\cite{RundeAmenConstFour}, пример 2]. Так как пространство $M_n(\mathbb{C}_s)$ конечномерно, то оно является $\mathscr{L}_{1,C}$-пространством для некоторой константы $C\geq 1$ не зависящей от $s$. Тогда по предложению \ref{MetTopEssL1FlatModAoverAmenBanAlg} банахов $M_n(C(K))$-модуль $M_n(\mathbb{C}_s)^*$ $C$-топологически плоский. Так как этот модуль существенный, то по предложению \ref{CTopFlatCharac} правый $M_n(C(K))$-модуль $M_n(\mathbb{C}_s)^{**}$ $C$-топологически инъективен. Так как $M_n(\mathbb{C}_s)^{**}$ изометрически изоморфен $M_n(\mathbb{C}_s)$ как правый $M_n(C(K))$-модуль, то из предложения \ref{MetTopInjModProd} мы получаем, что $\bigoplus_\infty\{M_n(\mathbb{C}_s):s\in K\}$ --- топологически инъективный $M_n(C(K))$-модуль.

Заметим, что по предложению \ref{MetInjCStarAlgCharac} банахов $C(K)$-модуль $C(K)$ метрически инъективен, поэтому изометрический $C(K)$-морфизм $\widetilde{\rho}:C(K)\to\bigoplus_\infty\{ \mathbb{C}_s:s\in K\}:x\mapsto \bigoplus_\infty\{x(s):s\in K\}$ имеет левый обратный сжимающий $C(K)$-морфизм $\widetilde{\tau}:\bigoplus_\infty\{ \mathbb{C}_s:s\in K\} \to C(K)$. Теперь легко проверить, что линеные операторы
$$
\rho:M_n(C(K))\to\bigoplus\nolimits_\infty\{M_n(\mathbb{C}_s):s\in K\}:x\mapsto \bigoplus\nolimits_\infty\{(x_{i,j}(s))_{i,j\in\mathbb{N}_n}:s\in K\}
$$
$$
\tau:\bigoplus\nolimits_\infty\{M_n(\mathbb{C}_s):s\in K\}\to M_n(C(K)):y\mapsto \left(\widetilde{\tau}\left(\bigoplus\nolimits_\infty\{y_{s,i,j}:s\in K\}\right)\right)_{i,j\in\mathbb{N}_n}
$$
являются $M_n(C(K))$-морфизмами, причем $\tau \rho=1_{M_n(C(K))}$. Следовательно, $M_n(C(K))$ является ретрактом топологически инъективного $M_n(C(K))$-модуля $\bigoplus_\infty\{M_n(\mathbb{C}_s):s\in K\}$ в $\mathbf{mod}_1-M_n(C(K))$. Наконец, из предложения \ref{RetrMetTopInjIsMetTopInj} мы получаем, что $M_n(C(K))$ топологически инъективен как $M_n(C(K))$-модуль.
\end{proof}

\begin{theorem}\label{TopInjAWStarAlgCharac} Пусть $A$ --- $C^*$-алгебра. Тогда следующие условия эквивалентны:

$i)$ $A$ --- топологически инъективная как $A$-модуль $AW^*$-алгебра;

$ii)$ $A$ изоморфна как $C^*$-алгебра алгебре $\bigoplus_\infty\{M_{n_\lambda}(C(K_\lambda)):\lambda\in\Lambda\}$ для некоторого конечного набора стоуновых пространств $(K_\lambda)_{\lambda\in\Lambda}$ и натуральных чисел $(n_\lambda)_{\lambda\in\Lambda}$.
\end{theorem}
\begin{proof}$i)\implies ii)$ Из предложения 6.6 в \cite{SmithDecompPropCStarAlg} мы знаем, что $AW^*$-алгебра либо изоморфна как $C^*$-алгебра алгебре $\bigoplus_\infty\{M_{n_\lambda}(C(K_\lambda)):\lambda\in\Lambda\}$ для некоторого конечного набора натуральных чисел $(n_\lambda)_{\lambda\in\Lambda}$ и стоуновых пространств $(K_\lambda)_{\lambda\in\Lambda}$, либо содержит $\bigoplus_\infty\{ \mathcal{B}(\ell_2(\mathbb{N}_n)):n\in\mathbb{N}\}$ как ${}^*$-подалгебру. Последняя возможность исключается предложением \ref{TopInjIdHaveLUST}.

$ii)\implies i)$ Для каждого $\lambda\in\Lambda$ алгебра $M_{n_\lambda}(C(K_\lambda))$ унитальна, так как $K_\lambda$ компактно. Следовательно, $M_{n_\lambda}(C(K_\lambda))$ --- верный $M_{n_\lambda}(C(K_\lambda))$-модуль. По предложению \ref{CKMatrixModTopInj} 
это еще и топологически инъективный $M_{n_\lambda}(C(K_\lambda))$-модуль. Теперь топологическая инъективность $A$ как $A$-модуля следует из пункта $ii)$ предложения \ref{MetTopProjInjFlatUnderSumOfAlg}. Достаточно положить $p=\infty$ и $X_\lambda=A_\lambda=M_{n_\lambda}(C(K_\lambda))$ для всех $\lambda\in\Lambda$. 

Для всех $\lambda\in\Lambda$ алгебра $C(K_\lambda)$ является $AW^*$-алгеброй потому, что $K_\lambda$ --- стоуново пространство [\cite{BerbBaerStarRings}, теорема 1.7.1]. Следовательно, $M_{n_\lambda}(C(K_\lambda))$ также является $AW^*$-алгеброй [\cite{BerbBaerStarRings}, следствие 9.62.1]. Наконец, $A$ есть $AW^*$-алгебра как $\bigoplus_\infty$-сумма $AW^*$-алгебр [\cite{BerbBaerStarRings}, предложение 1.10.1].
\end{proof}

Хотелось бы доказать, что топологически инъективные над собой $C^*$-алгебры являются $AW^*$-алгебрами, но, похоже, это очень сложная задача даже в коммутативном случае.

%----------------------------------------------------------------------------------------
%	Flat ideals of C^*-algebras
%----------------------------------------------------------------------------------------

\subsection{Плоские идеалы \texorpdfstring{$C^*$}{C*}-алгебр}
\label{SubSectionFlatIdealsOfCStarAlgebras}

Рассмотрением метрической и топологической плоскости мы завершим это длинное изучение идеалов $C^*$-алгебр.

\begin{proposition}\label{IdealofCstarAlgisMetTopFlat} Пусть $I$ --- левый идеал $C^*$-алгебры $A$. Тогда $I$ --- метрически и топологически плоский $A$-модуль.
\end{proposition}
\begin{proof} По предложению 4.7.78 из \cite{HelBanLocConvAlg} идеал $I$ имеет сжимающую аппроксимативную единицу. Остается применить предложение \ref{MetTopFlatIdealsInUnitalAlg}.
\end{proof}

Для доказательства следующего предложения нам понадобится определение слабо секвенциально полного банахова пространства. Будем говорить, что банахово пространство $E$ слабо секвенциально полно, если для любой последовательности $(x_n)_{n\in\mathbb{N}}\subset E$ такой, что последовательность $(f(x_n))_{n\in\mathbb{N}}\subset\mathbb{C}$ фундаментальна для всех $f\in E^*$, существует вектор $x\in E$ такой, что $\lim_n f(x_n)=f(x)$ для всех $f\in E^*$. Другими словами: всякая последовательность, фундаментальная в слабой топологии, сходится в слабой топологии. Стандартный пример слабо секвенциально полного банахова пространства --- это любое $L_1$-пространство [\cite{WojBanSpForAnalysts}, следствие III.C.14]. Это свойство наследуется замкнутыми подпространствами. Стандартный пример банахова пространства, не являющегося слабо секвенциально полным, --- это $c_0(\mathbb{N})$. Чтобы в этом убедиться, достаточно рассмотреть последовательность $(\sum_{k=1}^n \delta_k)_{n\in\mathbb{N}}$. 

\begin{proposition}\label{CStarAlgIsTopFlatOverItsIdeal} Пусть $I$ --- собственный двусторонний идеал $C^*$-алгебры $A$. Тогда следующие условия эквивалентны: 

$i)$ $A$ является $\langle$~метрически / топологически~$\rangle$ плоским $I$-модулем;

$ii)$ $\langle$~$\operatorname{dim}(A)=1$, $I=\{0\}$ / факторалгебра $A/I$ конечномерна~$\rangle$.

\end{proposition}
\begin{proof} Мы можем рассматривать $I$ как идеал в унитизации $A_\#$ алгебры $A$. Так как $I$ --- двусторонний идеал, то он имеет сжимающую двустороннюю аппроксимативную единицу $(e_\nu)_{\nu\in N}$ такую, что $0\leq e_\nu\leq e_{A_\#}$ [\cite{HelBanLocConvAlg}, предложение 4.7.79]. Как следствие, $\sup_{\nu\in N}\Vert e_{A_\#}-e_\nu\Vert\leq 1$. Снова из-за наличия аппроксимативной единицы в $I$ мы имеем $A_{ess}:=\operatorname{cl}_A(IA)=I$.

Для начала рассмотрим случай топологической плоскости. По предложению \ref{TopFlatModCharac} банахов $I$-модуль $A$ будет топологически плоским тогда и только тогда, когда $A_{ess}=I$ и $A/A_{ess}=A/I$ будут топологически плоскими $I$-модулями. По предложению \ref{IdealofCstarAlgisMetTopFlat} идеал $I$ топологически плоский как $I$-модуль. По предложению \ref{MetTopFlatAnnihModCharac} аннуляторный $I$-модуль $A/I$ топологически плоский тогда и только тогда, когда он является $\mathscr{L}_1$-пространством. Мы утверждаем, что модуль $A/I$ есть $\mathscr{L}_1$-пространство тогда и только тогда, когда он конечномерен. Допустим, модуль $A/I$ является $\mathscr{L}_1$-пространством, тогда он слабо секвенциально полон [\cite{BourgNewClOfLpSp}, предложение 1.29]. Так как $I$ --- двусторонний идеал, то $A/I$ есть $C^*$-алгебра [\cite{HelBanLocConvAlg}, теорема 4.7.81]. По предложению 2 из \cite{SakWeakCompOpOnOpAlg} каждая слабо секвенциально полная $C^*$-алгебра конечномерна, поэтому $A/I$ конечномерно. Обратно, если алгебра $A/I$ конечномерна, то она является $\mathscr{L}_1$-пространством, как и любое конечномерное банахово пространство.

Перейдем к рассмотрению метрической плоскости. Допустим, $A$ --- метрически плоский $I$-модуль. Из предложения \ref{MetFlatIsTopFlatAndTopFlatIsRelFlat} следует, что $A$ --- топологически плоский $I$-модуль, поэтому из рассуждений предыдущего абзаца мы знаем, что $A/I$ --- конечномерная $C^*$-алгебра. Как мы сказали ранее, $\sup_{\nu\in N}\Vert e_{A_\#}-e_\nu\Vert\leq 1$, поэтому из пункта $iii)$ предложения \ref{DualBanModDecomp} следует, что $(A/A_{ess})^*=(A/I)^*$ есть ретракт $A^*$ в $\mathbf{mod}_1-I$. Теперь из предложений \ref{MetTopFlatCharac} и \ref{RetrMetTopInjIsMetTopInj} следует, что $A/I$ есть метрически плоский $I$-модуль. Так как это аннуляторный $I$-модуль, то из предложения \ref{MetTopFlatAnnihModCharac} следует, что $I=\{0\}$ и $A/I$ есть $L_1$-пространство. Как мы показали, ранее пространство $A/I$ конечномерно, поэтому $A/I\isom{\mathbf{Ban}_1}\ell_1(\mathbb{N}_n)$ для  $n=\operatorname{dim}(A/I)$. С другой стороны, $A/I$ --- конечномерная $C^*$-алгебра, поэтому она изометрически изоморфна пространству $\bigoplus_\infty\{ \mathcal{B}(\ell_2(\mathbb{N}_{n_k})):k\in\mathbb{N}_m\}$ для некоторых натуральных чисел $(n_k)_{k\in\mathbb{N}_m}$ [\cite{DavCSatrAlgByExmpl}, теорема III.1.1]. Допустим, что $\operatorname{dim}(A/I)>1$, тогда $A$ содержит изометрическую копию $\ell_\infty(\mathbb{N}_2)$. Следовательно, имеется изометрическое вложение $\ell_\infty(\mathbb{N}_2)$ в $\ell_1(\mathbb{N}_n)$. Это невозможно по теореме $1$ из \cite{LyubIsomEmdbFinDimLp}. Следовательно, $\operatorname{dim}(A/I)=1$. Так как $I=\{0\}$, то $\operatorname{dim}(A)=1$. Обратно, если $I=\{0\}$ и $\operatorname{dim}(A)=1$, то мы имеем аннуляторный $I$-модуль $A$ который изометрически изоморфен $\ell_1(\mathbb{N}_1)$. По предложению \ref{MetTopFlatAnnihModCharac} он является метрически плоским. 
\end{proof}

%----------------------------------------------------------------------------------------
%	\mathcal{K}(H)- and \mathcal{B}(H)-modules
%----------------------------------------------------------------------------------------

\subsection{\texorpdfstring{$\mathcal{K}(H)$}{K(H)}- и \texorpdfstring{$\mathcal{B}(H)$}{B(H)}-модули}
\label{SubSectionKHAndBHModules}

В этом параграфе мы применим результаты об идеалах $C^*$-алгебр к изучению классических модулей над алгеброй компактных и алгеброй ограниченных операторов на гильбертовом пространстве. Для произвольного гильбертова пространства $H$ рассмотрим $\mathcal{B}(H)$, $\mathcal{K}(H)$ и $\mathcal{N}(H)$ как левые и правые банаховы модули над $\mathcal{B}(H)$ и $\mathcal{K}(H)$. Для всех этих модулей внешнее умножение --- это композиция операторов. Нам также пригодятся изоморфизмы Шаттена-фон Нойманна $\mathcal{N}(H)\isom{\mathbf{Ban}_1}\mathcal{K}(H)^*$, $\mathcal{B}(H)\isom{\mathbf{Ban}_1}\mathcal{N}(H)^*$ [\cite{TakThOpAlgVol1}, теоремы II.1.6, II.1.8]. На самом деле, это изоморфизмы левых и правых $\mathcal{B}(H)$- и, тем более, $\mathcal{K}(H)$-модулей.

\begin{proposition}\label{KHAndBHModBH} Пусть $H$ --- гильбертово пространство. Тогда:

$i)$ $\mathcal{B}(H)$ метрически и топологически проективный и плоский как $\mathcal{B}(H)$-модуль;

$ii)$ $\mathcal{B}(H)$ метрически или топологически проективный или плоский как $\mathcal{K}(H)$-модуль тогда и только тогда, когда $H$ конечномерно;

$iii)$ $\mathcal{B}(H)$ топологически инъективен как $\mathcal{B}(H)$- или $\mathcal{K}(H)$-модуль тогда и только тогда, когда $H$ конечномерно;

$iv)$ $\mathcal{B}(H)$ метрически инъективен как $\mathcal{B}(H)$- или $\mathcal{K}(H)$-модуль тогда и только тогда, когда $\dim(H)\leq 1$.
\end{proposition}
\begin{proof} $i)$ Так как $\mathcal{B}(H)$ унитальная алгебра, то по предложению \ref{UnitalAlgIsMetTopProj} она метрически и топологически проективна как $\mathcal{B}(H)$-модуль. Оба результата о плоскости следуют из предложения \ref{MetTopProjIsMetTopFlat}.

$ii)$ Для бесконечномерного $H$ банахово пространство $\mathcal{B}(H)/\mathcal{K}(H)$ бесконечномерно, поэтому из предложения  \ref{CStarAlgIsTopFlatOverItsIdeal} следует, что модуль $\mathcal{B}(H)$ не является ни метрически ни топологически плоским как $\mathcal{K}(H)$-модуль. Оба утверждения о проективности следуют из предложения \ref{MetTopProjIsMetTopFlat}. Если $H$ конечномерно, то $\mathcal{K}(H)=\mathcal{B}(H)$, поэтому утверждение следует из пункта $i)$.

$iii)$ Если $H$ бесконечномерно, то $\mathcal{B}(H)$ содержит $\mathcal{B}(\ell_2(\mathbb{N}_n))$ как ${}^*$-подалгебру для всех $n\in\mathbb{N}$. Тогда из предложения \ref{TopInjIdHaveLUST} следует, что $\mathcal{B}(H)$ не является топологически инъективным $\mathcal{B}(H)$-модулем. Все остальное следует из пункта $i)$ предложения \ref{MetTopInjUnderChangeOfAlg}. Если $H$ конечномерно, то $\mathcal{K}(H)=\mathcal{B}(H)$, поэтому утверждение следует из предложения \ref{FinDimBHModTopInj}.

$iv)$ Если $\dim(H)>1$, то $\mathcal{B}(H)$ некоммутативная $C^*$-алгебра. По предложению \ref{MetInjCStarAlgCharac} она не будет метрически инъективной как $\mathcal{B}(H)$-модуль. Теперь из пункта $i)$ предложения \ref{MetTopInjUnderChangeOfAlg} мы получаем, что $\mathcal{B}(H)$ не является метрически инъективным как $\mathcal{K}(H)$-модуль. Если $\dim(H)\leq 1$, оба утверждения, очевидно, следуют из предложения \ref{MetInjCStarAlgCharac}.
\end{proof}

\begin{proposition}\label{KHAndBHModKH} Пусть $H$ --- гильбертово пространство. Тогда:

$i)$ $\mathcal{K}(H)$ метрически и топологически плоский как $\mathcal{B}(H)$- или $\mathcal{K}(H)$-модуль;

$ii)$ $\mathcal{K}(H)$ метрически или топологически проективный как $\mathcal{B}(H)$- или $\mathcal{K}(H)$-модуль тогда и только тогда, когда $H$ конечномерно;

$iii)$ $\mathcal{K}(H)$ топологически инъективный как $\mathcal{B}(H)$- или $\mathcal{K}(H)$-модуль тогда и только тогда, когда $H$ конечномерно;

$iv)$ $\mathcal{K}(H)$ метрически инъективный как $\mathcal{B}(H)$- или $\mathcal{K}(H)$-модуль тогда и только тогда, когда $\dim(H)\leq 1$.
\end{proposition}
\begin{proof} Через $A$ мы обозначим одну из алгебр $\mathcal{B}(H)$ или $\mathcal{K}(H)$. Отметим, что $\mathcal{K}(H)$ --- это двусторонний идеал в $A$. 

$i)$ Напомним, что $\mathcal{K}(H)$ имеет сжимающую аппроксимативную единицу состоящую из конечномерных проекторов на все конечномерные подпространства в $H$. Так как $\mathcal{K}(H)$ --- это двусторонний идеал в $A$, то утверждение следует из предложения \ref{IdealofCstarAlgisMetTopFlat}.

$ii)$, $iii)$, $iv)$ Если $H$ бесконечномерно, то $\mathcal{K}(H)$ не является унитальной банаховой алгеброй. Из следствия \ref{BiIdealOfCStarAlgMetTopProjCharac} и предложения \ref{MetTopInjOfId} банахов $A$-модуль $\mathcal{K}(H)$ не является ни метрически ни топологически проективным или инъективным. Если $H$ конечномерно, то $\mathcal{K}(H)=\mathcal{B}(H)$, поэтому оба результата следуют из пунктов $i)$, $iii)$ и $iv)$ предложения \ref{KHAndBHModBH}.
\end{proof}

\begin{proposition}\label{KHAndBHModNH} Пусть $H$ --- гильбертово пространство. Тогда:

$i)$ $\mathcal{N}(H)$ метрически и топологически инъективен как $\mathcal{B}(H)$- или $\mathcal{K}(H)$-модуль;

$ii)$ $\mathcal{N}(H)$ топологически проективный или плоский $\mathcal{B}(H)$- или $\mathcal{K}(H)$-модуль тогда и только тогда, когда $H$ конечномерно;

$iii)$ $\mathcal{N}(H)$ метрически проективный или плоский как $\mathcal{B}(H)$- или $\mathcal{K}(H)$-модуль тогда и только тогда, когда $\dim(H)\leq 1$.
\end{proposition}
\begin{proof} Через $A$ мы обозначим одну из алгебр $\mathcal{B}(H)$ или $\mathcal{K}(H)$.

$i)$ Заметим, что $\mathcal{N}(H)\isom{\mathbf{mod}_1-A}\mathcal{K}(H)^*$, поэтому утверждение следует из предложения \ref{MetTopFlatCharac} и пункта $i)$ предложения \ref{KHAndBHModKH}.

$ii)$ Допустим, $H$ бесконечномерно. Так как $\mathcal{B}(H)\isom{\mathbf{mod}_1-A}\mathcal{N}(H)^*$, то из предложения \ref{DualMetTopProjIsMetrInj} и пункта $iii)$ предложения \ref{KHAndBHModBH} мы получаем, что $\mathcal{N}(H)$ не является топологически проективным как $A$-модуль. Оба результата о плоскости следуют из предложения \ref{MetTopProjIsMetTopFlat}. Если $H$ конечномерно, то результат следует из предложения \ref{FinDimNHModTopProjFlat}.

$iii)$ Допустим, что $\dim(H)>1$, тогда из пункта $iv)$ предложения \ref{KHAndBHModBH} банахов $A$-модуль $\mathcal{B}(H)$ не является метрически инъективным. Так как $\mathcal{B}(H)\isom{\mathbf{mod}_1-A}\mathcal{N}(H)^*$, то из предложения \ref{MetTopFlatCharac} мы получаем, что $\mathcal{N}(H)$ не является метрическим плоским $A$-модулем. По предложению \ref{MetTopProjIsMetTopFlat}, он не будет метрически проективным $A$-модулем. Если $\dim(H)\leq 1$, тогда $\mathcal{N}(H)=\mathcal{K}(H)=\mathcal{B}(H)$, поэтому утверждение следует из пункта $i)$ предложения \ref{KHAndBHModBH}.
\end{proof}

\begin{proposition}\label{KHAndBHModsRelTh} Пусть $H$ --- гильбертово пространство. Тогда:

$i)$ как $\mathcal{K}(H)$-модуль $\mathcal{N}(H)$ является относительно проективным, инъективным и плоским, $\mathcal{K}(H)$ является относительно проективным и плоским, но относительно инъективным только для конечномерного $H$, $\mathcal{B}(H)$ является относительно инъективным и плоским, но относительно проективным только для конечномерного $H$;

$ii)$ как $\mathcal{B}(H)$-модуль $\mathcal{N}(H)$ является относительно проективным, инъективным и плоским, $\mathcal{K}(H)$ является относительно проективным и плоским, $\mathcal{B}(H)$ является относительно проективным, инъективным и плоским.
\end{proposition}
\begin{proof} $i)$ Заметим, что $H$ есть относительно проективный $\mathcal{K}(H)$-модуль [\cite{HelBanLocConvAlg}, теорема 7.1.27], поэтому из предложения 7.1.13 в \cite{HelBanLocConvAlg} мы получаем, что $\mathcal{N}(H)\isom{\mathcal{K}(H)-\mathbf{mod}_1}H\projtens H^*$ относительно проективен как $\mathcal{K}(H)$-модуль. По теореме IV.2.16 из \cite{HelHomolBanTopAlg} банахов $\mathcal{K}(H)$-модуль $\mathcal{K}(H)$ относительно проективен. Тем более $\mathcal{N}(H)$ и $\mathcal{K}(H)$ относительно плоские $\mathcal{K}(H)$-модули [\cite{HelBanLocConvAlg}, предложение 7.1.40], поэтому $\mathcal{N}(H)\isom{\mathbf{mod}_1-\mathcal{K}(H)}\mathcal{K}(H)^*$ и $\mathcal{B}(H)\isom{\mathbf{mod}_1-\mathcal{K}(H)}\mathcal{N}(H)^*$ есть относительно инъективные $\mathcal{K}(H)$-модули. Из [\cite{RamsHomPropSemgroupAlg}, предложение 2.2.8  (i)] мы знаем, что банахова алгебра, относительно инъективная над собой, как правый модуль, обязана иметь левую единицу. Следовательно, $\mathcal{K}(H)$ не является относительно инъективным $\mathcal{K}(H)$-модулем для бесконечномерного $H$. Если $H$ конечномерно, то $\mathcal{K}(H)$-модуль $\mathcal{K}(H)$ относительно инъективен потому, что $\mathcal{K}(H)=\mathcal{B}(H)$, и как мы показали ранее, $\mathcal{B}(H)$ относительно инъективен как $\mathcal{K}(H)$-модуль. По следствию 5.5.64 из \cite{DalBanAlgAutCont} алгебра $\mathcal{K}(H)$ относительно аменабельна, поэтому все ее левые модули относительно плоские [\cite{HelBanLocConvAlg}, теорема 7.1.60]. В частности, $\mathcal{B}(H)$ относительно плоский $\mathcal{K}(H)$-модуль. Из [\cite{HelHomolBanTopAlg}, упражнение V.2.20] мы знаем, что $\mathcal{B}(H)$ не является относительно проективным $\mathcal{K}(H)$-модулем для бесконечномерного $H$. Если $H$ конечномерно, то $\mathcal{B}(H)$ относительно проективен как $\mathcal{K}(H)$-модуль потому, что $\mathcal{B}(H)=\mathcal{K}(H)$, и как мы показали ранее, $\mathcal{K}(H)$ относительно проективный $\mathcal{K}(H)$-модуль.

$ii)$ Из пункта $i)$ предложения \ref{KHAndBHModBH} и предложения \ref{MetProjIsTopProjAndTopProjIsRelProj} следует, что $\mathcal{B}(H)$ --- относительно проективный $\mathcal{B}(H)$-модуль. Из [\cite{RamsHomPropSemgroupAlg}, предложения 2.3.3, 2.3.4] мы знаем, что $\langle$~существенный относительно проективный / верный относительно инъективный~$\rangle$ модуль над идеалом банаховой алгебры будет $\langle$~относительно проективным / относительно инъективным~$\rangle$ над самой алгеброй. Так как $\mathcal{K}(H)$ и $\mathcal{N}(H)$ --- существенные и верные $\mathcal{K}(H)$-модули, то из результатов предыдущего пункта мы получаем, что $\mathcal{N}(H)$ является относительно проективным и плоским, а $\mathcal{K}(H)$ является относительно проективным  $\mathcal{B}(H)$-модулем. Теперь, из [\cite{HelBanLocConvAlg}, предложение 7.1.40] следует, что все вышеупомянутые модули относительно плоские как $\mathcal{B}(H)$-модули. В частности, $\mathcal{B}(H)\isom{\mathbf{mod}_1-\mathcal{B}(H)}\mathcal{N}(H)^*$ относительно инъективен как $\mathcal{B}(H)$-модуль.
\end{proof}

Результаты этого параграфа собраны в следующих трех таблицах. Каждая ячейка таблицы содержит условие, при котором соответствующий модуль имеет соответствующее свойство, и предложение, в котором это доказано. Мы используем символ ??? для случаев где ответ нам не известен. Из таблиц видно, что для некоммутативных алгебр гомологическая тривиальность встречается в метрической и топологической теории очень редко. Проще указать случаи, в которых метрические и топологические свойства совпадают с относительными: плоскость $\mathcal{K}(H)$ как $\mathcal{B}(H)$- или $\mathcal{K}(H)$-модуля, инъективность $\mathcal{N}(H)$ как $\mathcal{B}(H)$- или $\mathcal{K}(H)$-модуля, проективность и плоскость $\mathcal{B}(H)$-модуля $\mathcal{B}(H)$. В остальных случаях $H$ должно быть хотя бы конечномерно, чтобы эти свойства оказались эквивалентны в метрической, топологической и относительной теории.

\begin{scriptsize}
\begin{longtable}{|c|c|c|c|c|c|c|} 
\multicolumn{7}{c}{\mbox{Гомологически тривиальные $\mathcal{K}(H)$- и $\mathcal{B}(H)$-модули в метрической теории}}                                                                                                                                                                                                                                                                                                                                                                                                                                               \\
				 
\hline          & \multicolumn{3}{c|}{$\mathcal{K}(H)$-модули}                                                                                                                                                                                                                     & \multicolumn{3}{c|}{$\mathcal{B}(H)$-модули}                                                                                                                                                                                                                         \\
\hline
                & Проективность                                                                  & Инъективность                                                                  & Плоскость                                                                       & \mbox{Проективность}                                                                   & Инъективность                                                                   & Плоскость                                                                       \\ 
\hline
$\mathcal{N}(H)$  & \begin{tabular}{@{}c@{}}$\dim(H)\leq 1$ \\ \ref{KHAndBHModNH}\end{tabular}            & \begin{tabular}{@{}c@{}}$H$ любое   \\ \ref{KHAndBHModNH}\end{tabular}         & \begin{tabular}{@{}c@{}}$\dim(H)\leq 1$ \\ \ref{KHAndBHModNH}\end{tabular}             & \begin{tabular}{@{}c@{}}$\dim(H)\leq 1$ \\ \ref{KHAndBHModNH}\end{tabular}             & \begin{tabular}{@{}c@{}}$H$\mbox{ любое }  \\ \ref{KHAndBHModNH}\end{tabular}          & \begin{tabular}{@{}c@{}}$\dim(H)\leq 1$ \\ \ref{KHAndBHModNH}\end{tabular}             \\
\hline
$\mathcal{B}(H)$  & \begin{tabular}{@{}c@{}}$\dim(H)<\aleph_0$ \\ \ref{KHAndBHModBH}\end{tabular}         & \begin{tabular}{@{}c@{}}$\dim(H)\leq 1$ \\ \ref{KHAndBHModBH}\end{tabular}            & \begin{tabular}{@{}c@{}}$\dim(H)<\aleph_0$ \\ \ref{KHAndBHModBH}\end{tabular}          & \begin{tabular}{@{}c@{}}$H$\mbox{ любое } \\ \ref{KHAndBHModBH}\end{tabular}           & \begin{tabular}{@{}c@{}}$\dim(H)\leq 1$ \\ \ref{KHAndBHModBH}\end{tabular}             & \begin{tabular}{@{}c@{}}$H$\mbox{ любое } \\ \ref{KHAndBHModBH}\end{tabular}           \\ 
\hline
$\mathcal{K}(H)$  & \begin{tabular}{@{}c@{}}$\dim(H)<\aleph_0$ \\ \ref{KHAndBHModKH}\end{tabular}         & \begin{tabular}{@{}c@{}}$\dim(H)\leq 1$ \\ \ref{KHAndBHModKH}\end{tabular}            & \begin{tabular}{@{}c@{}}$H$\mbox{ любое } \\ \ref{KHAndBHModKH}\end{tabular}           & \begin{tabular}{@{}c@{}}$\dim(H)<\aleph_0$ \\ \ref{KHAndBHModKH}\end{tabular}          & \begin{tabular}{@{}c@{}}$\dim(H)\leq 1$ \\ \ref{KHAndBHModKH}\end{tabular}             & \begin{tabular}{@{}c@{}}$H$\mbox{ любое } \\ \ref{KHAndBHModKH}\end{tabular}           \\ 
\hline

\multicolumn{7}{c}{\mbox{Гомологически тривиальные $\mathcal{K}(H)$- и $\mathcal{B}(H)$-модули в топологической теории}}                                                                                                                                                                                                                                                                                                                                                                                                                                            \\
					 
\hline          & \multicolumn{3}{c|}{$\mathcal{K}(H)$-модули}                                                                                                                                                                                                                     & \multicolumn{3}{c|}{$\mathcal{B}(H)$-модули}                                                                                                                                                                                                                         \\
\hline
                & \mbox{Проективность}                                                                  & Инъективность                                                                  & Плоскость                                                                       & \mbox{Проективность}                                                                   & Инъективность                                                                   & Плоскость                                                                       \\ 
\hline
$\mathcal{N}(H)$  & \begin{tabular}{@{}c@{}}$\dim(H)<\aleph_0$ \\ \ref{KHAndBHModNH}\end{tabular}         & \begin{tabular}{@{}c@{}}$H$ любое  \\ \ref{KHAndBHModNH}\end{tabular}          & \begin{tabular}{@{}c@{}}$\dim(H)<\aleph_0$ \\ \ref{KHAndBHModNH}\end{tabular}          & \begin{tabular}{@{}c@{}}$\dim(H)<\aleph_0$ \\ \ref{KHAndBHModNH}\end{tabular}          & \begin{tabular}{@{}c@{}}$H$ любое  \\ \ref{KHAndBHModNH}\end{tabular}           & \begin{tabular}{@{}c@{}}$\dim(H)<\aleph_0$ \\ \ref{KHAndBHModNH}\end{tabular}          \\
\hline
$\mathcal{B}(H)$  & \begin{tabular}{@{}c@{}}$\dim(H)<\aleph_0$ \\ \ref{KHAndBHModBH}\end{tabular}         & \begin{tabular}{@{}c@{}}$\dim(H)<\aleph_0$ \\ \ref{KHAndBHModBH}\end{tabular}         & \begin{tabular}{@{}c@{}}$\dim(H)<\aleph_0$ \\ \ref{KHAndBHModBH}\end{tabular}          & \begin{tabular}{@{}c@{}}$H$ любое  \\ \ref{KHAndBHModBH}\end{tabular}           & \begin{tabular}{@{}c@{}}$\dim(H)<\aleph_0$ \\ \ref{KHAndBHModBH}\end{tabular}          & \begin{tabular}{@{}c@{}}$H$ любое  \\ \ref{KHAndBHModBH}\end{tabular}           \\ 
\hline
$\mathcal{K}(H)$  & \begin{tabular}{@{}c@{}}$\dim(H)<\aleph_0$ \\ \ref{KHAndBHModKH}\end{tabular}         & \begin{tabular}{@{}c@{}}$\dim(H)<\aleph_0$ \\ \ref{KHAndBHModKH}\end{tabular}         & \begin{tabular}{@{}c@{}}$H$ любое  \\ \ref{KHAndBHModKH}\end{tabular}           & \begin{tabular}{@{}c@{}}$\dim(H)<\aleph_0$ \\ \ref{KHAndBHModKH}\end{tabular}          & \begin{tabular}{@{}c@{}}$\dim(H)<\aleph_0$ \\ \ref{KHAndBHModKH}\end{tabular}          & \begin{tabular}{@{}c@{}}$H$ любое  \\ \ref{KHAndBHModKH}\end{tabular}           \\ 
\hline

\multicolumn{7}{c}{\mbox{Гомологически тривиальные $\mathcal{K}(H)$- и $\mathcal{B}(H)$-модули в относительной теории}}                                                                                                                                                                                                                                                                                                                                                                                                                                             \\

\hline          & \multicolumn{3}{c|}{$\mathcal{K}(H)$-модули}                                                                                                                                                                                                                     & \multicolumn{3}{c|}{$\mathcal{B}(H)$-модули}                                                                                                                                                                                                                         \\
\hline
                & \mbox{Проективность}                                                                  & Инъективность                                                                  & Плоскость                                                                       & \mbox{Проективность}                                                                   & Инъективность                                                                   & Плоскость                                                                       \\ 
\hline
$\mathcal{N}(H)$  & \begin{tabular}{@{}c@{}}$H$ любое  \\ \ref{KHAndBHModsRelTh}, i)\end{tabular}  & \begin{tabular}{@{}c@{}}$H$ любое  \\ \ref{KHAndBHModsRelTh}, i)\end{tabular}  & \begin{tabular}{@{}c@{}}$H$ любое  \\ \ref{KHAndBHModsRelTh}, i)\end{tabular}   & \begin{tabular}{@{}c@{}}$H$ любое  \\ \ref{KHAndBHModsRelTh}, ii)\end{tabular}  & \begin{tabular}{@{}c@{}}$H$ любое  \\ \ref{KHAndBHModsRelTh}, ii)\end{tabular}  & \begin{tabular}{@{}c@{}}$H$ любое  \\ \ref{KHAndBHModsRelTh}, ii)\end{tabular}  \\
\hline
$\mathcal{B}(H)$  & \begin{tabular}{@{}c@{}}$\dim(H)<\aleph_0$ \\ \ref{KHAndBHModsRelTh}, i)\end{tabular} & \begin{tabular}{@{}c@{}}$H$ любое  \\ \ref{KHAndBHModsRelTh}, i)\end{tabular}  & \begin{tabular}{@{}c@{}}$H$ любое  \\ \ref{KHAndBHModsRelTh}, i)\end{tabular}   & \begin{tabular}{@{}c@{}}$H$ любое  \\ \ref{KHAndBHModsRelTh}, ii)\end{tabular}  & \begin{tabular}{@{}c@{}}$H$ любое  \\ \ref{KHAndBHModsRelTh}, ii)\end{tabular}  & \begin{tabular}{@{}c@{}}$H$ любое  \\ \ref{KHAndBHModsRelTh}, ii)\end{tabular}  \\
\hline
$\mathcal{K}(H)$  & \begin{tabular}{@{}c@{}}$H$ любое  \\ \ref{KHAndBHModsRelTh}, i)\end{tabular}  & \begin{tabular}{@{}c@{}}$\dim(H)<\aleph_0$ \\ \ref{KHAndBHModsRelTh}, i)\end{tabular} & \begin{tabular}{@{}c@{}}$H$ любое  \\ \ref{KHAndBHModsRelTh}, i)\end{tabular}   & \begin{tabular}{@{}c@{}}$H$ любое  \\ \ref{KHAndBHModsRelTh}, ii)\end{tabular}  & \begin{tabular}{@{}c@{}} ??? \end{tabular}                                             & \begin{tabular}{@{}c@{}}$H$ любое  \\ \ref{KHAndBHModsRelTh}, ii)\end{tabular}  \\
\hline
\end{longtable}
\end{scriptsize}




%----------------------------------------------------------------------------------------
%	c_0(\Lambda)- and l_infty(\Lambda)-modules
%----------------------------------------------------------------------------------------

\subsection{\texorpdfstring{$c_0(\Lambda)$}{c0(Lambda)}- и \texorpdfstring{$\ell_\infty(\Lambda)$}{lInfty(Lambda)}-модули}
\label{SubSectionc0AndlInftyModules}

Мы продолжим наше изучение модулей над $C^*$-алгебрами и перейдем к коммутативным примерам. Для заданного индексного множества $\Lambda$ мы рассмотрим пространства $c_0(\Lambda)$ и $\ell_p(\Lambda)$ при $1\leq p\leq+\infty$ как левые и правые модули над алгебрами $c_0(\Lambda)$ и $\ell_\infty(\Lambda)$. Для всех этих модулей внешнее умножение --- это поточечное умножение. Хорошо известно, что $c_0(\Lambda)^*\isom{\mathbf{Ban}_1}\ell_1(\Lambda)$ и $\ell_p(\Lambda)^*\isom{\mathbf{Ban}_1}\ell_{p^*}(\Lambda)$ для $1\leq p<+\infty$. На самом деле, эти изоморфизмы являются изоморфизмами $\ell_\infty(\Lambda)$- и $c_0(\Lambda)$-модулей. 

Для заданного $\lambda\in\Lambda$ мы определим $\mathbb{C}_\lambda$ как левый или правый $\ell_\infty(\Lambda)$- или $c_0(\Lambda)$-модуль $\mathbb{C}$ с внешним умножением определенным равенствами
$$
a\cdot_\lambda z=a(\lambda)z,\qquad z\cdot_\lambda a=a(\lambda) z.
$$

\begin{proposition}\label{OneDimlInftyc0ModMetTopProjIngFlat} Пусть $\Lambda$ --- произвольное множество и $\lambda\in\Lambda$. Тогда $\mathbb{C}_\lambda$ метрически и топологически проективный, инъективный и плоский $\ell_\infty(\Lambda)$- или $c_0(\Lambda)$-модуль.
\end{proposition}
\begin{proof} Пусть $A$ обозначает одну из алгебр $\ell_\infty(\Lambda)$ или $c_0(\Lambda)$. Легко проверить, что отображения $\pi:A_+\to\mathbb{C}_\lambda:a\oplus_1 z\mapsto a(\lambda)+z$ и $\sigma:\mathbb{C}_\lambda\to A_+:z\mapsto z\delta_\lambda\oplus_1 0$ являются сжимающими $A$-морфизмами левых $A$-модулей. Так как $\pi\sigma=1_{\mathbb{C}_\lambda}$, то $\mathbb{C}_\lambda$ есть ретракт $A_+$ в $A-\mathbf{mod}_1$. Из предложений \ref{UnitalAlgIsMetTopProj} и \ref{RetrMetTopProjIsMetTopProj} следует, что $\mathbb{C}_\lambda$ метрически и топологически проективен как $A$-модуль и, тем более, метрически и топологически плоский по предложению \ref{MetTopProjIsMetTopFlat}. Из предложения \ref{DualMetTopProjIsMetrInj} мы знаем, что $\mathbb{C}_\lambda^*$ метрически и топологически инъективен как $A$-модуль. Теперь метрическая и топологическая инъективность $\mathbb{C}_\lambda$ следует из изоморфизма $\mathbb{C}_\lambda\isom{\mathbf{mod}_1-A}\mathbb{C}_\lambda^*$.
\end{proof}

\begin{proposition}\label{c0AndlInftyModlIfty} Пусть $\Lambda$ --- произвольное множество. Тогда:

$i)$ $\ell_\infty(\Lambda)$ метрически и топологически плоский $\ell_\infty(\Lambda)$-модуль;

$ii)$ $\ell_\infty(\Lambda)$ метрически или топологически проективный или плоский как $c_0(\Lambda)$-модуль тогда и только тогда, когда $\Lambda$ конечно;

$iii)$ $\ell_\infty(\Lambda)$ метрически и топологически инъективен как $\ell_\infty(\Lambda)$- и $c_0(\Lambda)$-модуль.
\end{proposition}
\begin{proof} $i)$ Так как $\ell_\infty(\Lambda)$ --- унитальная алгебра, то она метрически и топологически проективна как $\ell_\infty(\Lambda)$-модуль по предложению \ref{UnitalAlgIsMetTopProj}. Оба результата о плоскости следуют из предложения \ref{MetTopProjIsMetTopFlat}.

$ii)$ Для бесконечного $\Lambda$ банахово пространство $\ell_\infty(\Lambda)/c_0(\Lambda)$ бесконечномерно, поэтому по предложению \ref{CStarAlgIsTopFlatOverItsIdeal} модуль $\ell_\infty(\Lambda)$ не является ни метрически, ни топологически плоским как $c_0(\Lambda)$-модуль. Оба утверждения о проективности следуют из предложения \ref{MetTopProjIsMetTopFlat}. Если $\Lambda$ конечно, то $c_0(\Lambda)=\ell_\infty(\Lambda)$, поэтому результат следует из пункта $i)$.

$iii)$ Пусть $A$ обозначает одну из алгебр $\ell_\infty(\Lambda)$ или $c_0(\Lambda)$. Заметим, что $\ell_\infty(\Lambda)\isom{A-\mathbf{mod}_1}\bigoplus_\infty\{\mathbb{C}_\lambda:\lambda\in\Lambda\}$, поэтому из предложений \ref{OneDimlInftyc0ModMetTopProjIngFlat} и \ref{MetTopInjModProd} следует, что $\ell_\infty(\Lambda)$ метрически инъективный $A$-модуль. Утверждение о топологической инъективности следует из предложения \ref{MetInjIsTopInjAndTopInjIsRelInj}.
\end{proof}

\begin{proposition}\label{c0AndlInftyModc0} Пусть $\Lambda$ --- произвольное множество. Тогда:

$i)$ $c_0(\Lambda)$ метрически и топологически плоский $\ell_\infty(\Lambda)$- и $c_0(\Lambda)$-модуль;

$ii)$ $c_0(\Lambda)$ метрически или топологически проективный $\ell_\infty(\Lambda)$- или $c_0(\Lambda)$-модуль тогда и только тогда, когда $\Lambda$ конечно;

$iii)$ $c_0(\Lambda)$ метрически или топологически инъективный $\ell_\infty(\Lambda)$- или $c_0(\Lambda)$-модуль тогда и только тогда, когда $\Lambda$ конечно.
\end{proposition}
\begin{proof} Пусть $A$ обозначает одну из алгебр $\ell_\infty(\Lambda)$ или $c_0(\Lambda)$. Отметим, что $c_0(\Lambda)$ --- двусторонний идеал в $A$. 

$i)$ Напомним, что $c_0(\Lambda)$ имеет сжимающую аппроксимативную единицу вида $(\sum_{\lambda\in S}\delta_\lambda)_{S\in\mathcal{P}_0(\Lambda)}$. Так как $c_0(\Lambda)$ есть двусторонний идеал в $A$, то результат следует из предложения \ref{IdealofCstarAlgisMetTopFlat}.

$ii)$, $iii)$ Если $\Lambda$ бесконечно, то $c_0(\Lambda)$ не унитальная банахова алгебра. Из следствия \ref{BiIdealOfCStarAlgMetTopProjCharac} и предложения \ref{MetTopInjOfId} банахов $A$-модуль $c_0(\Lambda)$ не является ни метрически ни топологически проективным или инъективным. Если $\Lambda$ конечно, то $c_0(\Lambda)=\ell_\infty(\Lambda)$, поэтому оба результата следуют из пунктов $i)$ и $iii)$ предложения \ref{c0AndlInftyModlIfty}.
\end{proof}

\begin{proposition}\label{c0AndlInftyModl1} Пусть $\Lambda$ --- произвольное множество. Тогда:

$i)$ $\ell_1(\Lambda)$ метрически и топологически инъективный $\ell_\infty(\Lambda)$- или $c_0(\Lambda)$-модуль;

$ii)$ $\ell_1(\Lambda)$ метрически и топологически проективный и плоский $\ell_\infty(\Lambda)$- или $c_0(\Lambda)$-модуль;
\end{proposition}
\begin{proof} Пусть $A$ обозначает одну из алгебр $\ell_\infty(\Lambda)$ или $c_0(\Lambda)$.

$i)$ Заметим, что $\ell_1(\Lambda)\isom{\mathbf{mod}_1-A}c_0(\Lambda)^*$, поэтому утверждение следует из предложения \ref{MetTopFlatCharac} и пункта $i)$ предложения \ref{c0AndlInftyModc0}.

$ii)$ Так как $\ell_1(\Lambda)\isom{A-\mathbf{mod}_1}\bigoplus_1\{\mathbb{C}_\lambda:\lambda\in\Lambda\}$, то из предложений \ref{OneDimlInftyc0ModMetTopProjIngFlat} и \ref{MetTopProjModCoprod} следует, что $\ell_1(\Lambda)$ метрически проективен как $A$-модуль. Топологическая проективность следует из предложения \ref{MetProjIsTopProjAndTopProjIsRelProj}. Утверждение о метрической и топологической плоскости теперь следуют из предложения \ref{MetTopProjIsMetTopFlat}.
\end{proof}

\begin{proposition}\label{c0AndlInftyModlp} Пусть $\Lambda$ --- произвольное множество и $1<p<+\infty$. Тогда:

$i)$ $\ell_p(\Lambda)$ топологически проективный, инъективный и плоский $\ell_\infty(\Lambda)$- или $c_0(\Lambda)$-модуль тогда и только тогда, когда $\Lambda$ конечно;

$ii)$ если $\ell_p(\Lambda)$ метрически проективный, инъективный или плоский $\ell_\infty(\Lambda)$- или $c_0(\Lambda)$-модуль, то $\Lambda$ конечно.
\end{proposition}
\begin{proof} Пусть $A$ обозначает одну из алгебр $\ell_\infty(\Lambda)$ или $c_0(\Lambda)$, тогда $A$ есть $\mathscr{L}_\infty$-пространство. 

$i)$, $ii)$ Так как банахово пространство $\ell_p(\Lambda)$ рефлексивно при $1<p<+\infty$, то из следствия \ref{NoInfDimRefMetTopProjInjFlatModOverMthscrL1OrLInfty} следует, что пространство $\ell_p(\Lambda)$ необходимо конечномерно если оно является метрически или топологически проективным, инъективным или плоским $\ell_\infty(\Lambda)$- или $c_0(\Lambda)$-модулем. Это эквивалентно тому, что $\Lambda$ конечно. Если $\Lambda$ конечно, то $\ell_p(\Lambda)\isom{A-\mathbf{mod}}\ell_1(\Lambda)$ и $\ell_p(\Lambda)\isom{\mathbf{mod}-A}\ell_1(\Lambda)$, поэтому топологическая проективность, инъективность и плоскость следуют из предложения \ref{c0AndlInftyModl1}.
\end{proof}

\begin{proposition}\label{c0AndlInftyModsRelTh} Пусть $\Lambda$ --- произвольное множество. Тогда:

$i)$ как $c_0(\Lambda)$-модули $\ell_p(\Lambda)$ для $1\leq p<+\infty$ и $\mathbb{C}_\lambda$ для $\lambda\in\Lambda$ являются относительно проективными, инъективными и плоскими, $c_0(\Lambda)$ является относительно проективным и плоским, но относительно инъективным только для конечного $\Lambda$, $\ell_\infty(\Lambda)$ является относительно инъективным и плоским, но относительно проективным только для конечного $\Lambda$;

$ii)$ как $\ell_\infty(\Lambda)$-модули $\ell_p(\Lambda)$ для $1\leq p\leq+\infty$ и $\mathbb{C}_\lambda$ для $\lambda\in\Lambda$ являются относительно проективными, инъективными и плоскими, $c_0(\Lambda)$ является относительно проективным и плоским.
\end{proposition}
\begin{proof} $i)$ Алгебра $c_0(\Lambda)$ относительно бипроективна [\cite{HelHomolBanTopAlg}, теорема IV.5.26] и имеет сжимающую аппроксимативную единицу, поэтому по теореме 7.1.60 из \cite{HelBanLocConvAlg} все существенные $c_0(\Lambda)$-модули проективны. Тогда $c_0(\Lambda)$ и $\ell_p(\Lambda)$ при $1\leq p<+\infty$ являются относительно проективными $c_0(\Lambda)$-модулями. Тем более, они являются относительно плоскими $c_0(\Lambda)$-модулями [\cite{HelBanLocConvAlg}, предложение 7.1.40]. Из того же предложения $c_0(\Lambda)$-модули $\ell_1(\Lambda)\isom{\mathbf{mod}_1-c_0(\Lambda)}c_0(\Lambda)^*$ и $\ell_{p^*}(\Lambda)\isom{\mathbf{mod}_1-c_0(\Lambda)}\ell_p(\Lambda)^*$ для $1\leq p<+\infty$ относительно инъективны. Из [\cite{RamsHomPropSemgroupAlg}, предложение 2.2.8 (i)] мы знаем, что банахова алгебра относительно инъективная над собой как правый модуль, обязана иметь левую единицу. Следовательно, $c_0(\Lambda)$ не является относительно инъективным $c_0(\Lambda)$-модулем для бесконечного $\Lambda$. Если $\Lambda$ конечно, то $c_0(\Lambda)$-модуль $c_0(\Lambda)$  относительно инъективен потому, что $c_0(\Lambda)=\ell_\infty(\Lambda)$, и, как было доказано выше, $\ell_\infty(\Lambda)$ есть относительно инъективный $c_0(\Lambda)$-модуль. Из [\cite{HelHomolBanTopAlg}, следствие V.2.16(II)] мы знаем, что $\ell_\infty(\Lambda)$ не может быть относительно проективным $c_0(\Lambda)$-модулем для бесконечного $\Lambda$. Если же $\Lambda$ конечно, то $\ell_\infty(\Lambda)$ относительно проективен как $c_0(\Lambda)$-модуль потому, что  $\ell_\infty(\Lambda)=c_0(\Lambda)$ и как было показано выше $c_0(\Lambda)$ есть относительно проективный $c_0(\Lambda)$-модуль. Из предложений \ref{OneDimlInftyc0ModMetTopProjIngFlat}, \ref{MetProjIsTopProjAndTopProjIsRelProj}, \ref{MetInjIsTopInjAndTopInjIsRelInj} и \ref{MetFlatIsTopFlatAndTopFlatIsRelFlat} следуют утверждения о модулях $\mathbb{C}_\lambda$, где $\lambda\in\Lambda$ произвольно.

$ii)$ Пункт $i)$ предложения \ref{c0AndlInftyModlIfty} и предложение \ref{MetProjIsTopProjAndTopProjIsRelProj} показывают, что $\ell_\infty(\Lambda)$ есть относительно проективный $\ell_\infty(\Lambda)$-модуль. Из [\cite{RamsHomPropSemgroupAlg}, предложения 2.3.3, 2.3.4] мы знаем, что $\langle$~существенный относительно проективный / верный относительно инъективный~$\rangle$ модуль над идеалом банаховой алгебры $\langle$~относительно проективен / относительно инъективен~$\rangle$ над самой алгеброй. Так как $c_0(\Lambda)$ и $\ell_p(\Lambda)$ для $1\leq p<+\infty$ есть существенные и верные $c_0(\Lambda)$-модули, то из результатов предыдущего пункта мы получаем, что $\ell_p(\Lambda)$ для $1\leq p<+\infty$ относительно проективны и инъективны как $\ell_\infty(\Lambda)$-модули. Также мы получаем, что $c_0(\Lambda)$ относительно проективен как $\ell_\infty(\Lambda)$-модуль. Таким образом, все эти $\ell_\infty(\Lambda)$-модули также будут относительно плоскими [\cite{HelBanLocConvAlg}, предложение 7.1.40]. Как следствие, $\ell_\infty(\Lambda)\isom{\mathbf{mod}_1-\ell_\infty(\Lambda)}\ell_1(\Lambda)^*$ является относительно инъективным $\ell_\infty(\Lambda)$-модулем. Из предложений \ref{OneDimlInftyc0ModMetTopProjIngFlat}, \ref{MetProjIsTopProjAndTopProjIsRelProj}, \ref{MetInjIsTopInjAndTopInjIsRelInj} и \ref{MetFlatIsTopFlatAndTopFlatIsRelFlat} следуют утверждения о модулях $\mathbb{C}_\lambda$,  где $\lambda\in\Lambda$ произвольно.
\end{proof}

Результаты этого параграфа собраны в следующих трех таблицах. Каждая ячейка таблицы содержит условие, при котором соответствующий модуль имеет соответствующее свойство, и предложение, в котором это доказано. Для случая  $\ell_\infty(\Lambda)$- и $c_0(\Lambda)$-модулей $\ell_p(\Lambda)$ при $1<p<+\infty$ мы не имеем критерия гомологической тривиальности в метрической теории, только необходимое условие. Чтобы это подчеркнуть, мы используем символ $\implies$. Сомнительно, что $\ell_p(\Lambda)$ при $1<p<+\infty$ будет метрически проективным инъективным или плоским при $\operatorname{Card}(\Lambda)>1$. К сожалению, мы также не знаем является ли $\ell_\infty(\Lambda)$-модуль $c_0(\Lambda)$ относительно инъективным, поэтому мы пишем $???$ в соответствующей ячейке. 

\begin{scriptsize}
\begin{longtable}{|c|c|c|c|c|c|c|} 
\multicolumn{7}{c}{\mbox{Гомологическая тривиальность $c_0(\Lambda)$- и $\ell_\infty(\Lambda)$-модулей в метрической теории}}                                                                                                                                                                                                                                                                                                                                                                                                                                                                                                                                                                                                                            \\
				 
\hline               & \multicolumn{3}{c|}{$c_0(\Lambda)$-модули}                                                                                                                                                                                                                                                                                                                & \multicolumn{3}{c|}{$\ell_\infty(\Lambda)$-модули}                                                                                                                                                                                                                                                                                                        \\
\hline
                     & Проективность                                                                                                & Инъективность                                                                                                &  Плоскость                                                                                                     & Проективность                                                                                                & Инъективность                                                                                                &  Плоскость                                                                                                     \\ 
\hline
$\ell_1(\Lambda)$      & \begin{tabular}{@{}c@{}}$\Lambda$ любое  \\ \ref{c0AndlInftyModl1}\end{tabular}                              & \begin{tabular}{@{}c@{}}$\Lambda$ любое   \\ \ref{c0AndlInftyModl1}\end{tabular}                             & \begin{tabular}{@{}c@{}}$\Lambda$ любое  \\ \ref{c0AndlInftyModl1}\end{tabular}                              & \begin{tabular}{@{}c@{}}$\Lambda$ любое  \\ \ref{c0AndlInftyModl1}\end{tabular}                              & \begin{tabular}{@{}c@{}}$\Lambda$ любое   \\ \ref{c0AndlInftyModl1}\end{tabular}                             & \begin{tabular}{@{}c@{}}$\Lambda$ любое  \\ \ref{c0AndlInftyModl1}\end{tabular}                              \\
\hline
$\ell_p(\Lambda)$      & \begin{tabular}{@{}c@{}}$\operatorname{Card}(\Lambda)<\aleph_0$ \\ ($\implies$) \ref{c0AndlInftyModlp}\end{tabular} & \begin{tabular}{@{}c@{}}$\operatorname{Card}(\Lambda)<\aleph_0$ \\ ($\implies$) \ref{c0AndlInftyModlp}\end{tabular} & \begin{tabular}{@{}c@{}}$\operatorname{Card}(\Lambda)<\aleph_0$ \\ ($\implies$) \ref{c0AndlInftyModlp}\end{tabular} & \begin{tabular}{@{}c@{}}$\operatorname{Card}(\Lambda)<\aleph_0$ \\ ($\implies$) \ref{c0AndlInftyModlp}\end{tabular} & \begin{tabular}{@{}c@{}}$\operatorname{Card}(\Lambda)<\aleph_0$ \\ ($\implies$) \ref{c0AndlInftyModlp}\end{tabular} & \begin{tabular}{@{}c@{}}$\operatorname{Card}(\Lambda)<\aleph_0$ \\ ($\implies$) \ref{c0AndlInftyModlp}\end{tabular} \\
\hline
$\ell_\infty(\Lambda)$ & \begin{tabular}{@{}c@{}}$\operatorname{Card}(\Lambda)<\aleph_0$ \\ \ref{c0AndlInftyModlIfty}\end{tabular}           & \begin{tabular}{@{}c@{}}$\Lambda$ любое  \\ \ref{c0AndlInftyModlIfty}\end{tabular}                           & \begin{tabular}{@{}c@{}}$\operatorname{Card}(\Lambda)<\aleph_0$ \\ \ref{c0AndlInftyModlIfty}\end{tabular}           & \begin{tabular}{@{}c@{}}$\Lambda$ любое  \\ \ref{c0AndlInftyModlIfty}\end{tabular}                           & \begin{tabular}{@{}c@{}}$\Lambda$ любое  \\ \ref{c0AndlInftyModlIfty}\end{tabular}                           & \begin{tabular}{@{}c@{}}$\Lambda$ любое  \\ \ref{c0AndlInftyModlIfty}\end{tabular}                           \\ 
\hline
$c_0(\Lambda)$         & \begin{tabular}{@{}c@{}}$\operatorname{Card}(\Lambda)<\aleph_0$ \\ \ref{c0AndlInftyModc0}\end{tabular}              & \begin{tabular}{@{}c@{}}$\operatorname{Card}(\Lambda)< \aleph_0$ \\ \ref{c0AndlInftyModc0}\end{tabular}             & \begin{tabular}{@{}c@{}}$\Lambda$ любое  \\ \ref{c0AndlInftyModc0}\end{tabular}                              & \begin{tabular}{@{}c@{}}$\operatorname{Card}(\Lambda)<\aleph_0$ \\ \ref{c0AndlInftyModc0}\end{tabular}              & \begin{tabular}{@{}c@{}}$\operatorname{Card}(\Lambda)< \aleph_0$ \\ \ref{c0AndlInftyModc0}\end{tabular}             & \begin{tabular}{@{}c@{}}$\Lambda$ любое  \\ \ref{c0AndlInftyModc0}\end{tabular}                              \\ 
\hline
$\mathbb{C}_\lambda$   & \begin{tabular}{@{}c@{}}$\lambda$ любое  \\ \ref{OneDimlInftyc0ModMetTopProjIngFlat}\end{tabular}            & \begin{tabular}{@{}c@{}}$\lambda$ любое  \\ \ref{OneDimlInftyc0ModMetTopProjIngFlat}\end{tabular}            & \begin{tabular}{@{}c@{}}$\lambda$ любое  \\ \ref{OneDimlInftyc0ModMetTopProjIngFlat}\end{tabular}            & \begin{tabular}{@{}c@{}}$\lambda$ любое  \\ \ref{OneDimlInftyc0ModMetTopProjIngFlat}\end{tabular}            & \begin{tabular}{@{}c@{}}$\lambda$ любое  \\ \ref{OneDimlInftyc0ModMetTopProjIngFlat}\end{tabular}            & \begin{tabular}{@{}c@{}}$\lambda$ любое  \\ \ref{OneDimlInftyc0ModMetTopProjIngFlat}\end{tabular}            \\
\hline

\multicolumn{7}{c}{\mbox{Гомологическая тривиальность $c_0(\Lambda)$- и $\ell_\infty(\Lambda)$-модулей в топологической теории}}                                                                                                                                                                                                                                                                                                                                                                                                                                                                                                                                                                                                                         \\
					 
\hline               & \multicolumn{3}{c|}{$c_0(\Lambda)$-модули}                                                                                                                                                                                                                                                                                                                & \multicolumn{3}{c|}{$\ell_\infty(\Lambda)$-модули}                                                                                                                                                                                                                                                                                                        \\
\hline
                     & Проективность                                                                                                & Инъективность                                                                                                &  Плоскость                                                                                                     & Проективность                                                                                                & Инъективность                                                                                                &  Плоскость                                                                                                     \\ 
\hline
$\ell_1(\Lambda)$      & \begin{tabular}{@{}c@{}}$\Lambda$ любое   \\ \ref{c0AndlInftyModl1}\end{tabular}                             & \begin{tabular}{@{}c@{}}$\Lambda$ любое  \\ \ref{c0AndlInftyModl1}\end{tabular}                              & \begin{tabular}{@{}c@{}}$\Lambda$ любое   \\ \ref{c0AndlInftyModl1}\end{tabular}                             & \begin{tabular}{@{}c@{}}$\Lambda$ любое   \\ \ref{c0AndlInftyModl1}\end{tabular}                             & \begin{tabular}{@{}c@{}}$\Lambda$ любое  \\ \ref{c0AndlInftyModl1}\end{tabular}                              & \begin{tabular}{@{}c@{}}$\Lambda$ любое   \\ \ref{c0AndlInftyModl1}\end{tabular}                             \\
\hline
$\ell_p(\Lambda)$      & \begin{tabular}{@{}c@{}}$\operatorname{Card}(\Lambda)<\aleph_0$ \\ \ref{c0AndlInftyModlp}\end{tabular}              & \begin{tabular}{@{}c@{}}$\operatorname{Card}(\Lambda)<\aleph_0$ \\ \ref{c0AndlInftyModlp}\end{tabular}              & \begin{tabular}{@{}c@{}}$\operatorname{Card}(\Lambda)<\aleph_0$ \\ \ref{c0AndlInftyModlp}\end{tabular}              & \begin{tabular}{@{}c@{}}$\operatorname{Card}(\Lambda)<\aleph_0$ \\ \ref{c0AndlInftyModlp}\end{tabular}              & \begin{tabular}{@{}c@{}}$\operatorname{Card}(\Lambda)<\aleph_0$ \\ \ref{c0AndlInftyModlp}\end{tabular}              & \begin{tabular}{@{}c@{}}$\operatorname{Card}(\Lambda)<\aleph_0$ \\ \ref{c0AndlInftyModlp}\end{tabular}              \\
\hline
$\ell_\infty(\Lambda)$ & \begin{tabular}{@{}c@{}}$\operatorname{Card}(\Lambda)<\aleph_0$ \\ \ref{c0AndlInftyModlIfty}\end{tabular}           & \begin{tabular}{@{}c@{}}$\Lambda$ любое  \\ \ref{c0AndlInftyModlIfty}\end{tabular}                           & \begin{tabular}{@{}c@{}}$\operatorname{Card}(\Lambda)<\aleph_0$ \\ \ref{c0AndlInftyModlIfty}\end{tabular}           & \begin{tabular}{@{}c@{}}$\Lambda$ любое  \\ \ref{c0AndlInftyModlIfty}\end{tabular}                           & \begin{tabular}{@{}c@{}}$\Lambda$ любое  \\ \ref{c0AndlInftyModlIfty}\end{tabular}                           & \begin{tabular}{@{}c@{}}$\Lambda$ любое  \\ \ref{c0AndlInftyModlIfty}\end{tabular}                           \\ 
\hline
$c_0(\Lambda)$         & \begin{tabular}{@{}c@{}}$\operatorname{Card}(\Lambda)<\aleph_0$ \\ \ref{c0AndlInftyModc0}\end{tabular}              & \begin{tabular}{@{}c@{}}$\operatorname{Card}(\Lambda)<\aleph_0$ \\ \ref{c0AndlInftyModc0}\end{tabular}              & \begin{tabular}{@{}c@{}}$\Lambda$ любое  \\ \ref{c0AndlInftyModc0}\end{tabular}                              & \begin{tabular}{@{}c@{}}$\operatorname{Card}(\Lambda)<\aleph_0$ \\ \ref{c0AndlInftyModc0}\end{tabular}              & \begin{tabular}{@{}c@{}}$\operatorname{Card}(\Lambda)<\aleph_0$ \\ \ref{c0AndlInftyModc0}\end{tabular}              & \begin{tabular}{@{}c@{}}$\Lambda$ любое  \\ \ref{c0AndlInftyModc0}\end{tabular}                              \\ 
\hline
$\mathbb{C}_\lambda$   & \begin{tabular}{@{}c@{}}$\lambda$ любое  \\ \ref{OneDimlInftyc0ModMetTopProjIngFlat}\end{tabular}            & \begin{tabular}{@{}c@{}}$\lambda$ любое  \\ \ref{OneDimlInftyc0ModMetTopProjIngFlat}\end{tabular}            & \begin{tabular}{@{}c@{}}$\lambda$ любое  \\ \ref{OneDimlInftyc0ModMetTopProjIngFlat}\end{tabular}            & \begin{tabular}{@{}c@{}}$\lambda$ любое  \\ \ref{OneDimlInftyc0ModMetTopProjIngFlat}\end{tabular}            & \begin{tabular}{@{}c@{}}$\lambda$ любое  \\ \ref{OneDimlInftyc0ModMetTopProjIngFlat}\end{tabular}            & \begin{tabular}{@{}c@{}}$\lambda$ любое  \\ \ref{OneDimlInftyc0ModMetTopProjIngFlat}\end{tabular}            \\
\hline

\multicolumn{7}{c}{\mbox{Гомологическая тривиальность $c_0(\Lambda)$- и $\ell_\infty(\Lambda)$-модулей в относительной теории}}                                                                                                                                                                                                                                                                                                                                                                                                                                                                                                                                                                                                                          \\

\hline               & \multicolumn{3}{c|}{$c_0(\Lambda)$-модули}                                                                                                                                                                                                                                                                                                                & \multicolumn{3}{c|}{$\ell_\infty(\Lambda)$-модули}                                                                                                                                                                                                                                                                                                        \\
\hline
                     & Проективность                                                                                                & Инъективность                                                                                                &  Плоскость                                                                                                     & Проективность                                                                                                & Инъективность                                                                                                &  Плоскость                                                                                                     \\ 
\hline
$\ell_1(\Lambda)$      & \begin{tabular}{@{}c@{}}$\Lambda$ любое  \\ \ref{c0AndlInftyModsRelTh}, i)\end{tabular}                      & \begin{tabular}{@{}c@{}}$\Lambda$ любое   \\ \ref{c0AndlInftyModsRelTh}, i)\end{tabular}                     & \begin{tabular}{@{}c@{}}$\Lambda$ любое  \\ \ref{c0AndlInftyModsRelTh}, i)\end{tabular}                      & \begin{tabular}{@{}c@{}}$\Lambda$ любое   \\ \ref{c0AndlInftyModsRelTh}, ii)\end{tabular}                    & \begin{tabular}{@{}c@{}}$\Lambda$ любое  \\ \ref{c0AndlInftyModsRelTh}, ii)\end{tabular}                     & \begin{tabular}{@{}c@{}}$\Lambda$ любое   \\ \ref{c0AndlInftyModlIfty}, ii)\end{tabular}                     \\
\hline
$\ell_p(\Lambda)$      & \begin{tabular}{@{}c@{}}$\Lambda$ любое  \\ \ref{c0AndlInftyModsRelTh}, i)\end{tabular}                      & \begin{tabular}{@{}c@{}}$\Lambda$ любое   \\ \ref{c0AndlInftyModsRelTh}, i)\end{tabular}                     & \begin{tabular}{@{}c@{}}$\Lambda$ любое  \\ \ref{c0AndlInftyModsRelTh}, i)\end{tabular}                      & \begin{tabular}{@{}c@{}}$\Lambda$ любое   \\ \ref{c0AndlInftyModsRelTh}, ii)\end{tabular}                    & \begin{tabular}{@{}c@{}}$\Lambda$ любое  \\ \ref{c0AndlInftyModsRelTh}, ii)\end{tabular}                     & \begin{tabular}{@{}c@{}}$\Lambda$ любое   \\ \ref{c0AndlInftyModlIfty}, ii)\end{tabular}                     \\
\hline
$\ell_\infty(\Lambda)$ & \begin{tabular}{@{}c@{}}$\operatorname{Card}(\Lambda)<\aleph_0$ \\ \ref{c0AndlInftyModsRelTh}, i)\end{tabular}      & \begin{tabular}{@{}c@{}}$\Lambda$ любое   \\ \ref{c0AndlInftyModsRelTh}, i)\end{tabular}                     & \begin{tabular}{@{}c@{}}$\Lambda$ любое  \\ \ref{c0AndlInftyModsRelTh}, i)\end{tabular}                      & \begin{tabular}{@{}c@{}}$\Lambda$ любое   \\ \ref{c0AndlInftyModsRelTh}, ii)\end{tabular}                    & \begin{tabular}{@{}c@{}}$\Lambda$ любое  \\ \ref{c0AndlInftyModsRelTh}, ii)\end{tabular}                     & \begin{tabular}{@{}c@{}}$\Lambda$ любое   \\ \ref{c0AndlInftyModlIfty}, ii)\end{tabular}                     \\
\hline
$c_0(\Lambda)$         & \begin{tabular}{@{}c@{}}$\Lambda$ любое  \\ \ref{c0AndlInftyModsRelTh}, i)\end{tabular}                      & \begin{tabular}{@{}c@{}}$\operatorname{Card}(\Lambda)<\aleph_0$  \\ \ref{c0AndlInftyModsRelTh}, i) \end{tabular}    & \begin{tabular}{@{}c@{}}$\Lambda$ любое  \\ \ref{c0AndlInftyModsRelTh}, i)\end{tabular}                      & \begin{tabular}{@{}c@{}}$\Lambda$ любое   \\ \ref{c0AndlInftyModsRelTh}, ii)\end{tabular}                    & \begin{tabular}{@{}c@{}}\mbox{ ??? } \end{tabular}                                                                  & \begin{tabular}{@{}c@{}}$\Lambda$ любое   \\ \ref{c0AndlInftyModlIfty}, ii)\end{tabular}                     \\
\hline
$\mathbb{C}_\lambda$   & \begin{tabular}{@{}c@{}}$\lambda$ любое  \\ \ref{c0AndlInftyModsRelTh}, i)\end{tabular}                      & \begin{tabular}{@{}c@{}}$\lambda$ любое   \\ \ref{c0AndlInftyModsRelTh}, i)\end{tabular}                     & \begin{tabular}{@{}c@{}}$\lambda$ любое  \\ \ref{c0AndlInftyModsRelTh}, i)\end{tabular}                      & \begin{tabular}{@{}c@{}}$\lambda$ любое   \\ \ref{c0AndlInftyModsRelTh}, ii)\end{tabular}                    & \begin{tabular}{@{}c@{}}$\lambda$ любое  \\ \ref{c0AndlInftyModsRelTh}, ii)\end{tabular}                     & \begin{tabular}{@{}c@{}}$\lambda$ любое   \\ \ref{c0AndlInftyModlIfty}, ii)\end{tabular}                     \\
\hline
\end{longtable}
\end{scriptsize}
Из этих таблиц легко видеть, что для модулей над коммутативными $C^*$-алгебрами, есть много общего между относительной, метрической и топологической теориями. Например, $\ell_1(\Lambda)$ --- проективный, инъективный и плоский $\ell_\infty(\Lambda)$- или $c_0(\Lambda)$-модуль во всех трех теориях. 


%----------------------------------------------------------------------------------------
%	Applications to harmonic analysis
%----------------------------------------------------------------------------------------

\section{Приложения к гармоническому анализу}
\label{SectionApplicationsToHarmonicAnalysis}


%----------------------------------------------------------------------------------------
%	Preliminaries on harmonic analysis
%----------------------------------------------------------------------------------------

\subsection{Предварительные сведения по гармоническому анализу}
\label{SubSectionPreliminariesOnHarmonicAnalysis}

Пусть $G$ --- локально компактная группа. Ее единицу мы будем обозначать через $e_G$. По хорошо известной теореме Хаара [\cite{HewRossAbstrHarmAnalVol1}, параграф 15.8] существует единственная с точностью до положительной константы регулярная по Борелю мера $m_G$, которая конечна на всех компактных множествах, положительна на всех открытых множествах и инвариантна относительно левых сдвигов, то есть $m_G(sE)=m_G(E)$ для всех $s\in G$ и $E\in Bor(G)$. Здесь, через $Bor(G)$ мы обозначаем борелевскую $\sigma$-алгебру открытых множеств в $G$. Мера обладающая вышеперечисленными свойствами, называется левой мерой Хаара $G$. Если $G$ компактно, то мы предполагаем, что $m_G(G)=1$. Если $G$ бесконечно и дискретно, то в роли $m_G$ мы берем считающую меру. Для каждого $s\in G$ отображение $m:Bor(G)\to[0,+\infty]:E\mapsto m_G(Es)$ также является левой мерой Хаара, поэтому из единственности мы получаем, что $m(E)=\Delta(s)m_G(E)$ для некоторого $\Delta(s)>0$. Функция $\Delta:G\to(0,+\infty)$ называется модулярной функцией группы $G$. Ясно, что $\Delta(st)=\Delta(s)\Delta(t)$ для всех $s,t\in G$. Если модулярная функция тождественно равна единице, то соответствующая группа называется модулярной. В частности, модулярными являются компактные, коммутативные и дискретные группы. В дальнейшем мы будем использовать обозначение $L_p(G)$ для $L_p(G,m_G)$ при всех $1\leq p\leq+\infty$. Для фиксированного $s\in G$ мы определим оператор левого сдвига $L_s:L_1(G)\to L_1(G):f\mapsto(t\mapsto f(s^{-1}t))$ и оператор правого сдвига $R_s:L_1(G)\to L_1(G):f\mapsto (t\mapsto f(ts))$. 

Структура группы на $G$ позволяет задать на $L_1(G)$ структуру банаховой алгебры. Для заданных $f,g\in L_1(G)$ мы определим их свертку по формуле
$$
(f\convol g)(s)=\int_G f(t)g(t^{-1}s)dm_G(t)=\int_G f(st)g(t^{-1})dm_G(t)=\int_G f(st^{-1})g(t)\Delta(t^{-1})dm_G(t)
$$
для почти всех $s\in G$. В этом случае $L_1(G)$ с операцией свертки в качестве умножения становится банаховой алгеброй. Банахова алгебра $L_1(G)$ имеет сжимающую двустороннюю аппроксимативную единицу состоящую из положительных непрерывных функций с компактным носителем. Банахова алгебра $L_1(G)$ унитальна тогда и только тогда, когда $G$ дискретно, и в этом случае $\delta_{e_G}$ есть единица в $L_1(G)$. Структура группы на $G$ также позволяет сделать банахово пространство комплексных конечных борелевских мер $M(G)$ банаховой алгеброй. Свертку двух мер $\mu,\nu\in M(G)$ мы определим по формуле
$$
(\mu\convol \nu)(E)=\int_G\nu(s^{-1}E)d\mu(s)=\int_G\mu(Es^{-1})d\nu(s)
$$
для всех $E\in Bor(G)$. Банахово пространство $M(G)$ вместе с такой сверткой есть унитальная банахова алгебра. Роль единицы играет  мера Дирака $\delta_{e_G}$ сосредоточенная в $e_G$. На самом деле, $M(G)$ есть копроизведение в $L_1(G)-\mathbf{mod}_1$ (но не в $M(G)-\mathbf{mod}_1$) своего двустороннего идеала $M_a(G)$ мер, абсолютно непрерывных по отношению к $m_G$, и подалгебры $M_s(G)$, состоящей из мер, сингулярных по отношению к $m_G$. Напомним, что $M_a(G)\isom{M(G)-\mathbf{mod}_1}L_1(G)$ и $M_s(G)$ есть аннуляторный $L_1(G)$-модуль. Наконец, $M(G)=M_a(G)$ тогда и только тогда, когда $G$ дискретна. 

Перейдем к обсуждению классических левых и правых модулей над алгебрами $L_1(G)$ и $M(G)$. Так как $L_1(G)$ можно рассматривать как двусторонний идеал в $M(G)$ с помощью изометрического $M(G)$-морфизма $i:L_1(G)\to M(G):f\mapsto f m_G$, то нам будет достаточно задавать структуру $M(G)$-модуля. Для $1\leq p<+\infty$ и любых $f\in L_p(G)$, $\mu\in M(G)$ положим по определению
$$
(\mu\convol_p f)(s)=\int_G f(t^{-1}s)d\mu(t),
\qquad\qquad
(f\convol_p \mu)(s)=\int_G f(st^{-1})\Delta(t^{-1})^{1/p}d\mu(t).
$$
Эти внешние умножения превращают банаховы пространства $L_p(G)$ для $1\leq p<+\infty$ в левые и правые $M(G)$-модули. Заметим, что при $p=1$ и $\mu\in M_a(G)$ мы получаем обычное определение свертки. Для $1<p\leq +\infty$ и любых $f\in L_p(G)$, $\mu\in M(G)$ положим по определению,
$$
(\mu\cdot_p f)(s)=\int_G \Delta(t)^{1/p}f(st)d\mu(t),
\qquad\qquad
(f\cdot_p \mu)(s)=\int_G f(ts)d\mu(t).
$$
Эти внешние умножения также превращают банаховы пространства $L_p(G)$ для $1<p\leq+\infty$ в левые и правые $M(G)$-модули. Этот специальный выбор внешних умножений хорошо взаимодействует с двойственностью. Действительно, $(L_p(G),\convol_p)^*\isom{\mathbf{mod}_1-M(G)}(L_{p^*}(G),\cdot_{p^*})$ для всех $1\leq p<+\infty$. Более того, банахово пространство $C_0(G)$ также можно наделить структурой левого и правого $M(G)$-модуля с помощью внешнего умножения $\cdot_\infty$. Более того, $(C_0(G),\cdot_\infty)^*\isom{M(G)-\mathbf{mod}_1}(M(G),\convol)$ и также $C_0(G)$ есть замкнутый левый и правый $M(G)$-подмодуль в $L_\infty(G)$.

Характером локально компактно группы $G$ называется непрерывный гомоморфизм из $G$ в $\mathbb{T}$. Множество характеров группы $G$ есть группа. Следуя Понтрягину, мы будем ее обозначать $\widehat{G}$. Она становится локально компактной группой, если на ней рассмотреть компактно-открытую топологию. Для любого характера $\gamma\in\widehat{G}$ можно определить непрерывный характер $\varkappa_\gamma^L:L_1(G)\to\mathbb{C}:f\mapsto \int_G f(s)\overline{\gamma(s)}dm_G(s)$ на $L_1(G)$. Все характеры на $L_1(G)$ устроены таким образом. Этот результат доказан Гельфандом [\cite{KaniBanAlg}, теоремы 2.7.2, 2.7.5]. Аналогично, для каждого $\gamma\in\widehat{G}$ можно таким же образом $\varkappa_\gamma^M:M(G)\to\mathbb{C}:\mu\mapsto\int_{G} \overline{\gamma(s)}d\mu(s)$ определить характер $M(G)$. Через $\mathbb{C}_\gamma$ мы будем обозначать так называемый аугментационный левый и правый $L_1(G)$- или $M(G)$-модуль. Его внешние умножения определяются по формуле
$$
f\cdot_{\gamma}z=z\cdot_{\gamma}f=\varkappa_\gamma^L(f)z,
\qquad\qquad
\mu\cdot_{\gamma}z=z\cdot_{\gamma}\mu=\varkappa_\gamma^M(\mu)z
$$
для всех $f\in L_1(G)$, $\mu\in M(G)$ и $z\in\mathbb{C}$. 

Одно из многочисленных эквивалентных определений аменабельной группы говорит, что локально компактная группа $G$ аменабельна, если существует $L_1(G)$-морфизм правых модулей $M:L_\infty(G)\to\mathbb{C}_{e_{\widehat{G}}}$ такой, что $M(\chi_G)=1$ [\cite{HelBanLocConvAlg}, параграф VII.2.5]. Можно даже предполагать, что $M$ --- сжимающий [\cite{HelBanLocConvAlg}, замечание 7.1.54].

Большинство результатов перечисленных в этом параграфе, но не снабженных ссылкой, можно найти в [\cite{DalBanAlgAutCont}, параграф 3.3].

%----------------------------------------------------------------------------------------
%	L_1(G)-modules
%----------------------------------------------------------------------------------------

\subsection{\texorpdfstring{$L_1(G)$}{L1(G)}-модули}
\label{SubSectionL1GModules}

В метрической теории гомологически тривиальные $L_1(G)$-модули гармонического анализа были изучены в \cite{GravInjProjBanMod}. Мы используем идеи этой работы, чтобы объединить подходы к изучению гомологически тривиальных модулей в метрической и топологической теории. 

\begin{proposition}\label{LInfIsL1MetrInj} Пусть $G$ --- локально компактная группа. Тогда $L_1(G)$ метрически и топологически плоский $L_1(G)$-модуль, то есть $L_1(G)$-модуль $L_\infty(G)$ метрически и топологически инъективен.
\end{proposition} 
\begin{proof} Так как $L_1(G)$ имеет сжимающую аппроксимативную единицу, то $L_1(G)$ метрически и топологически плоский $L_1(G)$-модуль по предложению \ref{MetTopFlatIdealsInUnitalAlg}. Так как $L_\infty(G)\isom{\mathbf{mod}_1-L_1(G)}L_1(G)^*$, то по предложению \ref{MetTopFlatCharac} этот $L_1(G)$-модуль метрически и топологически инъективен.
\end{proof}

\begin{proposition}\label{OneDimL1ModMetTopProjCharac} Пусть $G$ --- локально компактная группа, и $\gamma\in\widehat{G}$. Тогда следующие условия эквивалентны: 

$i)$ $G$ компактна;

$ii)$ $\mathbb{C}_\gamma$ метрически проективный $L_1(G)$-модуль;

$iii)$ $\mathbb{C}_\gamma$ топологически проективный $L_1(G)$-модуль.
\end{proposition}
\begin{proof} $i)$$\implies$$ ii)$ Рассмотрим $L_1(G)$-морфизмы $\sigma^+:\mathbb{C}_\gamma\to L_1(G)_+:z\mapsto z\gamma \oplus_1 0$ и $\pi^+:L_1(G)_+\to\mathbb{C}_\gamma: f\oplus_1 w\to f\cdot_{\gamma}1+w$. Легко проверить, что $\Vert\pi^+\Vert=\Vert\sigma^+\Vert=1$ и $\pi^+\sigma^+=1_{\mathbb{C}_\gamma}$. Следовательно, $\mathbb{C}_\gamma$ есть ретракт $L_1(G)_+$ в $L_1(G)-\mathbf{mod}_1$. Из предложений \ref{UnitalAlgIsMetTopProj} и \ref{RetrMetTopProjIsMetTopProj} следует, что $\mathbb{C}_\gamma$ метрически проективен.

$ii)$$\implies$$ iii)$ Импликация следует из предложения \ref{MetProjIsTopProjAndTopProjIsRelProj}.

$iii)$$\implies$$i)$ Рассмотрим $L_1(G)$-морфизм $\pi:L_1(G)\to\mathbb{C}_\gamma:f\mapsto f\cdot_{\gamma} 1$. Легко видеть, что $\pi$ строго коизометричен. Так как $\mathbb{C}_\gamma$ топологически проективен, то существует $L_1(G)$-морфизм $\sigma:\mathbb{C}_\gamma\to L_1(G)$ такой, что $\pi\sigma=1_{\mathbb{C}_\gamma}$. Пусть $f=\sigma(1)\in L_1(G)$ и $(e_\nu)_{\nu\in N}$ --- стандартная аппроксимативная единица $L_1(G)$. Так как $\sigma$ является $L_1(G)$-морфизмом, то для всех $s,t\in G$ выполнено 
$$
f(s^{-1}t)
=L_s(f)(t)
=\lim_\nu L_s(e_\nu\convol \sigma(1))(t)
=\lim_\nu((\delta_s\convol e_\nu)\convol \sigma(1))(t)
=\lim_\nu\sigma((\delta_s\convol e_\nu)\cdot_{\gamma} 1)(t)
$$
$$
=\lim_\nu\sigma(\varkappa_\gamma^L(\delta_s\convol e_\nu))(t)
=\lim_\nu\varkappa_\gamma^L(\delta_s\convol e_\nu)\sigma(1)(t)
=\lim_\nu(e_\nu\convol\gamma)(s^{-1})f(t)
=\gamma(s^{-1})f(t).
$$
Значит, для функции $g(t):=\gamma(t^{-1})f(t)$ из $L_1(G)$ выполнено $g(st)=g(t)$ для всех $s,t\in G$. Тогда $g$ является константной функцией в $L_1(G)$. Последнее возможно тогда и только тогда, когда $G$ компактна.
\end{proof}

\begin{proposition}\label{OneDimL1ModMetTopInjFlatCharac} Пусть $G$ --- локально компактная группа, и $\gamma\in\widehat{G}$. Тогда следующие условия эквивалентны: 

$i)$ $G$ аменабельна;

$ii)$ $\mathbb{C}_\gamma$ метрически инъективный $L_1(G)$-модуль;

$iii)$ $\mathbb{C}_\gamma$ топологически инъективный и плоский $L_1(G)$-модуль.

$iv)$ $\mathbb{C}_\gamma$ метрически плоский $L_1(G)$-модуль;

$v)$ $\mathbb{C}_\gamma$ топологически плоский $L_1(G)$-модуль;

\end{proposition}
\begin{proof} 
$i)$$\implies$$ ii)$ Так как $G$ аменабельна, то существует сжимающий $L_1(G)$-морфизм $M:L_\infty(G)\to\mathbb{C}_{e_{\widehat{G}}}$ со свойством $M(\chi_G)=1$. Рассмотрим линейные операторы $\rho:\mathbb{C}_\gamma\to L_\infty(G):z\mapsto z\overline{\gamma}$ и $\tau:L_\infty(G)\to\mathbb{C}_\gamma:f\mapsto M(f\gamma)$. Это $L_1(G)$-морфизмы  правых $L_1(G)$-модулей. Проверим это для оператора $\tau$: для всех $f\in L_\infty(G)$ и $g\in L_1(G)$ выполнено
$$
\tau(f\cdot_\infty g)
=M((f\cdot_\infty g)\gamma)
=M(f\gamma\cdot_\infty g\overline{\gamma})
=M(f\gamma)\cdot_{e_{\widehat{G}}} g\overline{\gamma}
=M(f\gamma)\varkappa_\gamma^L(g)
=\tau(f)\cdot_{\gamma} g.
$$  
Легко проверить, что $\rho$ и $\tau$ сжимающие и $\tau\rho=1_{\mathbb{C}_\gamma}$. Следовательно, $\mathbb{C}_\gamma$ --- ретракт $L_\infty(G)$ в $\mathbf{mod}_1-L_1(G)$. Из предложений \ref{LInfIsL1MetrInj} и \ref{RetrMetTopInjIsMetTopInj} следует, что $\mathbb{C}_\gamma$ метрически инъективен как $L_1(G)$-модуль.

$ii)$$\implies$$ iii)$ Импликация следует из предложения \ref{MetInjIsTopInjAndTopInjIsRelInj}.

$iii)$$ \implies$$ i)$ Так как $\rho$ есть изометрический $L_1(G)$-морфизм правых $L_1(G)$-модулей, и $\mathbb{C}_\gamma$ топологически инъективен как $L_1(G)$-модуль, то $\rho$ является коретракцией в $\mathbf{mod}-L_1(G)$. Обозначим его левый обратный морфизм через $\pi$, тогда $\pi(\overline{\gamma})=\pi(\rho(1))=1$. Рассмотрим ограниченный линейный функционал $M:L_\infty(G)\to\mathbb{C}_\gamma:f\mapsto \pi(f\overline{\gamma})$. Для всех $f\in L_\infty(G)$ и $g\in L_1(G)$ выполнено
$$
M(f\cdot_\infty g)
=\pi((f\cdot_\infty g)\overline{\gamma})
=\pi(f\overline{\gamma}\cdot_\infty g\gamma)
=\pi(f\overline{\gamma})\cdot_{\gamma} g\gamma
=M(f)\varkappa_\gamma^L(g\gamma)
=M(f)\cdot_{e_{\widehat{G}}}g.
$$
Следовательно, $M$ является $L_1(G)$-морфизмом, причем $M(\chi_G)=\pi(\overline{\gamma})=1$. Значит, $G$ аменабельна.

$ii)$$\Longleftrightarrow$$iv)$, $iii)$$\Longleftrightarrow$$v)$ Заметим, что $\mathbb{C}_\gamma^*\isom{\mathbf{mod}_1-L_1(G)}\mathbb{C}_\gamma$, поэтому все эквивалентности следуют из трех предыдущих пунктов и предложения \ref{MetTopFlatCharac}.
\end{proof}

В следующем предложении мы рассмотрим некоторый специфический тип идеалов алгебры $L_1(G)$. Они имеют вид $L_1(G)\convol \mu$ для некоторой идемпотентной меры $\mu$. На самом деле, этот тип идеалов, в случае коммутативной компактной группы $G$, совпадает с классом левых идеалов в $L_1(G)$, имеющих правую ограниченную аппроксимативную единицу [\cite{KaniBanAlg}, следствие 5.6.2].

\begin{proposition}\label{CommIdealByIdemMeasL1MetTopProjCharac} Пусть $G$ --- локально компактная группа, и $\mu\in M(G)$ --- идемпотентная мера, то есть $\mu\convol\mu=\mu$. Если левый идеал $I=L_1(G)\convol\mu$ банаховой алгебры $L_1(G)$ топологически проективен как $L_1(G)$-модуль, то $\mu=p m_G$, для некоторого $p\in I$.
\end{proposition}
\begin{proof} Пусть $\phi:I\to L_1(G)$ --- произвольный морфизм левых $L_1(G)$-модулей. Рассмотрим $L_1(G)$-морфизм $\phi':L_1(G)\to L_1(G):x\mapsto\phi(x\convol\mu)$. По теореме Венделя [\cite{WendLeftCentrzrs}, теорема 1], существует мера $\nu\in M(G)$ такая, что $\phi'(x)=x\convol\nu$ для всех $x\in L_1(G)$. В частности, $\phi(x)=\phi(x\convol\mu)=\phi'(x)=x\convol\nu$ для всех $x\in I$. Теперь ясно, что оператор $\psi:I\to I:x\mapsto\nu\convol x$ является морфизмом правых $I$-модулей, причем $\phi(x)y=x\psi(y)$ для всех $x,y\in I$. Теперь мы видим, что выполнено условие $(**)$ леммы \ref{GoodIdealMetTopProjIsUnital}, поэтому $I$ имеет правую единицу, назовем ее $e\in I$. Тогда $x\convol\mu=x\convol\mu\convol e$ для всех $x\in L_1(G)$. Две меры равны, если их свертки со всеми функциями из $L_1(G)$ совпадают [\cite{DalBanAlgAutCont}, следствие 3.3.24], поэтому $\mu=\mu\convol e m_G$. Так как $e\in I\subset L_1(G)$, то $\mu=\mu\convol e m_G\in M_a(G)$. Положим $p=\mu\convol e\in I$, тогда $\mu=p m_G$.
\end{proof}

Мы предполагаем, что левый идеал $L_1(G)\convol \mu$ для идемпотентной меры $\mu$ метрически проективен как $L_1(G)$-модуль тогда и только тогда, когда $\mu=p m_G$ для $p\in I$ нормы $1$.

\begin{theorem}\label{L1ModL1MetTopProjCharac} Пусть $G$ --- локально компактная группа. Тогда следующие условия эквивалентны:

$i)$ $G$ дискретно;

$ii)$ $L_1(G)$ метрически проективный $L_1(G)$-модуль;

$iii)$ $L_1(G)$ топологически проективный $L_1(G)$-модуль.
\end{theorem}
\begin{proof} $i)$$\implies$$ ii)$ Если $G$ дискретно, то $L_1(G)$ --- унитальная банахова алгебра с единицей нормы $1$. Из предложения \ref{UnIdeallIsMetTopProj} следует, что $L_1(G)$ метрически проективен как $L_1(G)$-модуль.

$ii)$$\implies$$ iii)$ Импликация следует из предложения  \ref{MetProjIsTopProjAndTopProjIsRelProj}.

$iii)$$ \implies$$ i)$ Очевидно, что $\delta_{e_G}$ --- идемпотентная мера. Так как идеал $L_1(G)=L_1(G)\convol \delta_{e_G}$ топологически проективен как $L_1(G)$-модуль, то из предложения \ref{CommIdealByIdemMeasL1MetTopProjCharac} мы получаем, что $\delta_{e_G}=f m_G$ для некоторой функции $f\in L_1(G)$. Это возможно только если $G$ дискретна.
\end{proof}

Стоит отметить, что $L_1(G)$-модуль $L_1(G)$ относительно проективен для любой локально компактной группы $G$ [\cite{HelBanLocConvAlg}, упражнение 7.1.17].

\begin{proposition}\label{L1MetTopProjAndMetrFlatOfMeasAlg} Пусть $G$ --- локально компактная группа. Тогда следующие свойства эквивалентны:

$i)$ $G$ дискретна;

$ii)$ $M(G)$ метрически проективный $L_1(G)$-модуль;

$iii)$ $M(G)$ топологически проективный $L_1(G)$-модуль;

$iv)$ $M(G)$ метрически плоский $L_1(G)$-модуль.
\end{proposition}
\begin{proof} 
$i)$$\implies$$ ii)$ Так как $M(G)\isom{L_1(G)-\mathbf{mod}_1}L_1(G)$ для дискретной группы $G$, то утверждение следует из теоремы \ref{L1ModL1MetTopProjCharac}. 

$ii)$$\implies$$ iii)$ Импликация следует из предложения \ref{MetProjIsTopProjAndTopProjIsRelProj}.

$ii)$$\implies$$ iv)$ Импликация следует из предложения \ref{MetTopProjIsMetTopFlat}.

$iii)$$\implies$$ i)$ Напомним, что $M(G)\isom{L_1(G)-\mathbf{mod}_1} L_1(G)\bigoplus_1 M_s(G)$, поэтому по предложению \ref{MetTopProjModCoprod} банахов $L_1(G)$-модуль $M_s(G)$ топологически проективен. Так как $M_s(G)$ есть аннуляторный $L_1(G)$-модуль, то из предложения \ref{MetTopProjOfAnnihModCharac} мы получаем, что $L_1(G)$ имеет правую единицу. Так как $L_1(G)$ также имеет двустороннюю ограниченную аппроксимативную единицу, то $L_1(G)$ унитальна. Последнее возможно только если $G$ дискретно.

$iv)$$\implies$$ i)$ Так как $M(G)\isom{L_1(G)-\mathbf{mod}_1} L_1(G)\bigoplus_1 M_s(G)$, то из предложения \ref{MetTopFlatModCoProd} банахов $L_1(G)$-модуль $M_s(G)$ является метрически плоским. Так как $M_s(G)$ есть аннуляторный $L_1(G)$-модуль, то из предложения \ref{MetTopFlatAnnihModCharac} следует, что $M_s(G)=\{0\}$. Последнее означает, что $G$ дискретно.
\end{proof}

\begin{proposition}\label{MeasAlgIsL1TopFlat} Пусть $G$ --- локально компактна группа. Тогда $M(G)$ топологически плоский $L_1(G)$-модуль.
\end{proposition}
\begin{proof} Так как $M(G)$ есть $L_1$-пространство, то, тем более, это  $\mathscr{L}_1$-пространство. Так как $M_s(G)$ дополняемо в $M(G)$, то $M_s(G)$ есть $\mathscr{L}_1$-пространство [\cite{BourgNewClOfLpSp}, предложение 1.28]. Так как $M_s(G)$ --- аннуляторный $L_1(G)$-модуль, то из предложения \ref{MetTopFlatAnnihModCharac} мы имеем, что $L_1(G)$-модуль $M_s(G)$ топологически плоский. Напомним, что по предложению \ref{LInfIsL1MetrInj} банахов $L_1(G)$-модуль $L_1(G)$ является топологически плоским. Так как $M(G)\isom{L_1(G)-\mathbf{mod}_1}L_1(G)\bigoplus_1 M_s(G)$, то из предложения \ref{MetTopFlatModCoProd}, мы получаем, что $L_1(G)$-модуль $M(G)$ топологически плоский.
\end{proof}

%----------------------------------------------------------------------------------------
%	M(G)-modules
%----------------------------------------------------------------------------------------

\subsection{\texorpdfstring{$M(G)$}{M(G)}-модули}
\label{SubSectionMGModules}

Мы переходим к обсуждению классических $M(G)$-модулей в гармоническом анализе. Как мы сейчас увидим, большинство результатов можно вывести из результатов об $L_1(G)$-модулях.

\begin{proposition}\label{MGMetTopProjInjFlatRedToL1} Пусть $G$ --- локально компактная группа, и пусть $X$ --- $\langle$~существенный / верный / существенный~$\rangle$ $L_1(G)$-модуль. Тогда:

$i)$ $X$ --- метрически $\langle$~проективный / инъективный / плоский~$\rangle$ $M(G)$-модуль тогда и только тогда, когда он метрически $\langle$~проективный / инъективный / плоский~$\rangle$ $L_1(G)$-модуль;

$ii)$ $X$ --- топологически $\langle$~проективный / инъективный / плоский~$\rangle$ $M(G)$-модуль тогда и только тогда, когда он топологически $\langle$~проективный / инъективный / плоский~$\rangle$ $L_1(G)$-модуль.
\end{proposition}
\begin{proof} Напомним, что $L_1(G)\isom{L_1(G)-\mathbf{mod}_1}M_a(G)$ является двусторонним $1$-дополняемым идеалом алгебры $M(G)$. Теперь пункты $i)$ и $ii)$ следуют из предложения $\langle$~\ref{MetTopProjUnderChangeOfAlg} / \ref{MetTopInjUnderChangeOfAlg}  / \ref{MetTopFlatUnderChangeOfAlg}~$\rangle$.
\end{proof} 

Следует напомнить, что $L_1(G)$-модули $C_0(G)$, $L_p(G)$ для $1\leq p<\infty$ и $\mathbb{C}_\gamma$ для $\gamma\in\widehat{G}$ существенны, и $L_1(G)$-модули $C_0(G)$, $M(G)$, $L_p(G)$ для $1\leq p\leq \infty$ и $\mathbb{C}_\gamma$ для $\gamma\in\widehat{G}$ верны. 

\begin{proposition}\label{MGModMGMetTopProjFlatCharac} Пусть $G$ --- локально компактная группа. Тогда $M(G)$ метрически и топологически проективный $M(G)$-модуль. Как следствие, он метрически и топологически плоский $M(G)$-модуль.
\end{proposition} 
\begin{proof} Так как $M(G)$ --- унитальная алгебра, то метрическая и топологическая проективность следуют из предложения \ref{UnitalAlgIsMetTopProj}. Теперь остается применить предложение \ref{MetTopProjIsMetTopFlat}.
\end{proof}

%----------------------------------------------------------------------------------------
%	Banach geometric restrictions
%----------------------------------------------------------------------------------------

\subsection{Банахово-геометрические ограничения}
\label{SubSectionBanachGeometricRestriction}

В этом параграфе мы покажем, что многие модули гармонического анализа не являются ни метрически, ни топологически проективными, инъективными или плоскими по причинам ``плохой'' банаховой геометрии. 

\begin{proposition}\label{StdModAreNotRetrOfL1LInf} Пусть $G$ --- бесконечная локально компактная группа. Тогда:

$i)$ $L_1(G)$, $C_0(G)$, $M(G)$, $L_\infty(G)^*$ не являются топологически инъективными банаховыми пространствами;

$ii)$ $C_0(G)$, $L_\infty(G)$ не дополняемы ни в одном $L_1$-пространстве.
\end{proposition}
\begin{proof}
Так как $G$ бесконечно, то все рассматриваемые модули бесконечномерны.

$i)$ Если бесконечномерное банахово пространство топологически инъективно,  то оно содержит копию $\ell_\infty(\mathbb{N})$ [\cite{RosOnRelDisjFamOfMeas}, следствие 1.1.4], и, следовательно, копию $c_0(\mathbb{N})$. Банахово пространство $L_1(G)$ слабо секвенциально полно [\cite{WojBanSpForAnalysts}, следствие III.C.14], поэтому по следствию 5.2.11 из \cite{KalAlbTopicsBanSpTh} оно не может содержать копию $c_0(\mathbb{N})$. Значит, $L_1(G)$ не может быть топологически инъективным банаховым пространством. Если $M(G)$ топологически инъективно, то таково же и его дополняемое пространство $M_a(G)\isom{\mathbf{Ban}_1}L_1(G)$. Из рассуждений выше мы знаем, что это невозможно, значит $M(G)$ не может быть топологически инъективным банаховым пространством. Из следствия 3 в \cite{LauMingComplSubspInLInfOfG} пространство $C_0(G)$ не дополняемо в $L_\infty(G)$. Значит, и $C_0(G)$ не может быть топологически инъективным банаховым пространством. Далее, банахово пространство $L_1(G)$ дополняемо в $L_\infty(G)^*\isom{\mathbf{Ban}_1}L_1(G)^{**}$ [\cite{DefFloTensNorOpId}, предложение B10]. Значит, если бы банахово пространство $L_\infty(G)^*$ было бы топологически инъективно, то таково же было бы и $L_1(G)$. Как мы показали ранее, это невозможно, значит, $L_\infty(G)^*$ не является топологически инъективным банаховым пространством. 

$ii)$ Если $C_0(G)$ --- ретракт $L_1$-пространства, то $M(G)\isom{\mathbf{Ban}_1}C_0(G)^*$ будет ретрактом $L_\infty$-пространства, и, как следствие, будет топологически инъективным банаховым пространством. Это противоречит пункту $i)$, поэтому $C_0(G)$ не может быть ретрактом $L_1$-пространства. Заметим, что $\ell_\infty(\mathbb{N})$ вкладывается в $L_\infty(G)$, и, как следствие, мы имеем вложение $c_0(\mathbb{N})$ в $L_\infty(G)$. Если бы $L_\infty(G)$ было бы ретрактом $L_1$-пространства, то нашлось бы $L_1$-пространство содержащее копию $c_0(\mathbb{N})$. Как мы показали в пункте $i)$ это невозможно.
\end{proof}

Начиная с этого момента и до конца параграфа, через $A$ мы будем обозначать одну из алгебр $L_1(G)$ или $M(G)$. Напомним, что $L_1(G)$ и $M(G)$ являются  $L_1$-пространствами.

\begin{proposition}\label{StdModAreNotL1MGMetTopProj} Пусть $G$ --- бесконечная локально компактная группа. Тогда:

$i)$ $C_0(G)$, $L_\infty(G)$ не являются ни метрически, ни топологически проективными $A$-модулями;

$ii)$ $L_1(G)$, $C_0(G)$, $M(G)$, $L_\infty(G)^*$ не являются ни метрически, ни топологически проективными $A$-модулями;

$iii)$ $L_\infty(G)$, $C_0(G)$ не являются ни метрически, ни топологически плоскими $A$-модулями.
\end{proposition}
\begin{proof} $i)$ Утверждение следует из пункта $i)$ предложения \ref{TopProjInjFlatModOverL1Charac} и предложения  \ref{StdModAreNotRetrOfL1LInf}.

$ii)$ Утверждение следует из пункта $ii)$ предложения \ref{TopProjInjFlatModOverL1Charac} и предложения \ref{StdModAreNotRetrOfL1LInf}.

$iii)$ Напомним, что $C_0(G)^*\isom{\mathbf{mod}_1-A}M(G)$. Значит утверждение следует из пункта $i)$ и предложения \ref{MetTopFlatCharac}.
\end{proof}

Осталось рассмотреть гомологическую тривиальность $A$-модулей в метрической и топологической теории для конечной группы $G$.

\begin{proposition}\label{LpFinGrL1MGMetrInjProjCharac} Пусть $G$ --- нетривиальная конечная группа и пусть $1\leq p\leq \infty$. Тогда $A$-модуль $L_p(G)$ метрически $\langle$~проективен / инъективен~$\rangle$ тогда и только тогда, когда $\langle$~$p=1$ / $p=\infty$~$\rangle$
\end{proposition}
\begin{proof} Допустим, $L_p(G)$ метрически $\langle$~проективен / инъективен~$\rangle$ как $A$-модуль. Так как $L_p(G)$ конечномерно, то из пунктов $i)$ и $ii)$ предложения \ref{TopProjInjFlatModOverL1Charac} мы получаем, что $\langle$~$L_p(G)\isom{\mathbf{Ban}_1}\ell_1(\mathbb{N}_n)$ / $L_p(G)\isom{\mathbf{Ban}_1}C(\mathbb{N}_n)\isom{\mathbf{Ban}_1}\ell_\infty(\mathbb{N}_n)$ ~$\rangle$, для $n=\operatorname{Card}(G)>1$. Теперь мы воспользуемся теоремой $1$ из \cite{LyubIsomEmdbFinDimLp} для банаховых пространств над полем $\mathbb{C}$. Она гласит, что если для $2\leq m\leq k$ и $1\leq r,s\leq \infty$ существует изометрическое вложение из $\ell_r(\mathbb{N}_m)$ в $\ell_s(\mathbb{N}_k)$, то либо $r=2$, $s\in 2\mathbb{N}$ либо $r=s$. Следовательно, $\langle$~$p=1$ / $p=\infty$~$\rangle$. Обратная импликация легко следует из $\langle$~теоремы \ref{L1ModL1MetTopProjCharac} / предложения \ref{LInfIsL1MetrInj}~$\rangle$
\end{proof}

\begin{proposition}\label{StdModFinGrL1MGMetrInjProjFlatCharac} Пусть $G$ --- конечная группа. Тогда:

$i)$ $C_0(G)$, $L_\infty(G)$ метрически инъективны как $A$-модули;

$ii)$ $C_0(G)$, $L_p(G)$ для $1<p\leq\infty$ метрически проективны как $A$-модули тогда и только тогда, когда $G$ тривиальна;

$iii)$ $M(G)$, $L_p(G)$ для $1\leq p<\infty$ метрически инъективны как $A$-модули тогда и только тогда, когда $G$ тривиальна;

$iv)$ $C_0(G)$, $L_p(G)$ для $1<p\leq\infty$ метрически плоские как $A$-модули тогда и только тогда, когда $G$ тривиальна.
\end{proposition}
\begin{proof}
$i)$ Так как $G$ конечна,то $C_0(G)=L_\infty(G)$. Теперь утверждение следует из предложения \ref{LInfIsL1MetrInj}.

$ii)$ Если $G$ тривиальна, то есть $G=\{e_G\}$, то $L_p(G)=C_0(G)=L_1(G)$ и утверждение следует из пункта $i)$. Если $G$ нетривиальна, то заметим, что $C_0(G)=L_\infty(G)$ и воспользуемся предложением \ref{LpFinGrL1MGMetrInjProjCharac}.

$iii)$ Если $G=\{e_G\}$, то $M(G)=L_p(G)=L_\infty(G)$ и тогда утверждение следует из пункта $i)$. Если $G$ нетривиальна, то заметим, что $M(G)=L_1(G)$ и воспользуемся предложением \ref{LpFinGrL1MGMetrInjProjCharac}.

$iv)$ Из пункта $iii)$ следует, что $L_p(G)$ для $1\leq p<\infty$ является метрически инъективным $A$-модулей тогда и только тогда, когда $G$ тривиальна. Теперь утверждение следует из предложения \ref{MetTopFlatCharac} и изоморфизмов $C_0(G)^*\isom{\mathbf{mod}_1-L_1(G)}M(G)\isom{\mathbf{mod}_1-L_1(G)}L_1(G)$, $L_p(G)^*\isom{\mathbf{mod}_1-L_1(G)}L_{p^*}(G)$ при $1\leq p^*<\infty$.
\end{proof}

Здесь следует сказать, что если бы мы рассматривали банаховы пространства над полем действительных чисел, то $L_\infty(G)$ и $L_1(G)$ были бы, соответственно, метрически проективны и инъективны, еще в одном случае, когда $G$ состоит из двух элементов. Причина этого эффекта в том, что 
$L_\infty(\mathbb{Z}_2)\isom{L_1(\mathbb{Z}_2)-\mathbf{mod}_1}\mathbb{R}_{\gamma_0}\bigoplus\nolimits_1\mathbb{R}_{\gamma_1}
$ и $L_1(\mathbb{Z}_2)\isom{L_1(\mathbb{Z}_2)-\mathbf{mod}_1}\mathbb{R}_{\gamma_0}\bigoplus\nolimits_\infty\mathbb{R}_{\gamma_1}$ для $\gamma_0,\gamma_1\in\widehat{\mathbb{Z}_2}$ определенных равенствами $\gamma_0(0)=\gamma_0(1)=\gamma_1(0)=-\gamma_1(1)=1$.

\begin{proposition}\label{StdModFinGrL1MGTopInjProjFlatCharac} Пусть $G$ --- конечная группа. Тогда $A$-модули $C_0(G)$, $M(G)$, $L_p(G)$ для $1\leq p\leq \infty$ являются топологически проективными, инъективными и плоскими.
\end{proposition} 
\begin{proof}
Так как $G$ конечна, то $M(G)=L_1(G)$ и $C_0(G)=L_\infty(G)$ и, как следствие, эти модули не требуют отдельного рассмотрения. Так как $M(G)=L_1(G)$, мы ограничимся случаем $A=L_1(G)$. Тождественное отображение $i:L_1(G)\to L_p(G):f\mapsto f$ является топологическим изоморфизмом банаховых пространств потому, что $L_1(G)$ и $L_p(G)$ для $1\leq p<\infty$ имеют одинаковую конечную размерность. Так как $G$ конечна, то она унимодулярна. Следовательно, внешние умножения в модулях $(L_1(G),\convol)$ и $(L_p(G),\convol_p)$ совпадают при $1\leq p<\infty$. Значит, $i$ --- изоморфизм в $L_1(G)-\mathbf{mod}$ и $\mathbf{mod}-L_1(G)$. Аналогично можно показать, что $(L_\infty(G),\cdot_\infty)$ и $(L_p(G),\cdot_p)$ при $1<p\leq\infty$ изоморфны в $L_1(G)-\mathbf{mod}$ и $\mathbf{mod}-L_1(G)$. Наконец, легко проверить, что $(L_1(G),\convol)$ и $(L_\infty(G),\cdot_\infty)$ изоморфны в $L_1(G)-\mathbf{mod}$ и $\mathbf{mod}-L_1(G)$ посредством морфизма $j:L_1(G)\to L_\infty(G):f\mapsto(s\mapsto f(s^{-1}))$. Таким образом, все рассматриваемые модули изоморфны. Осталось заметить, что по предложению \ref{LInfIsL1MetrInj} банахов $L_1(G)$-модуль $L_1(G)$ топологически проективный и плоский по теореме \ref{L1ModL1MetTopProjCharac} и предложению \ref{LInfIsL1MetrInj}, а $L_\infty(G)$ топологически инъективный $A$-модуль по предложению \ref{LInfIsL1MetrInj}.
\end{proof}

Результаты этого параграфа собраны в первых двух таблицах. В третьей таблице мы приводим известные результаты из относительной теории. Каждая ячейка таблицы содержит условие, при котором соответствующий модуль имеет соответствующее свойство, и предложения в которых это доказано. Формулировки и доказательства теорем, описывающих гомологически тривиальные модули $\mathbb{C}_\gamma$ в относительной теории такие же, как и в предложениях \ref{OneDimL1ModMetTopProjCharac}, \ref{OneDimL1ModMetTopInjFlatCharac} и \ref{OneDimL1ModMetTopInjFlatCharac}. Как обычно, стрелка $\implies$ обозначает, что известно только необходимое условие. 

\begin{scriptsize}
\begin{longtable}{|c|c|c|c|c|c|c|} 
\multicolumn{7}{c}{\mbox{Гомологически тривиальные $L_1(G)$- и $M(G)$-модули в метрической теории}}                                                                                                                                                                                                                                                                                                                                                                                                                                                                                                                                                                                                                                                                                                                                                                                                                                                                                                                                             \\
				 
\hline            & \multicolumn{3}{c|}{$L_1(G)$-модули}                                                                                                                                                                                                                                                                                                                                                                                                                                                                 & \multicolumn{3}{c|}{$M(G)$-модули}                                                                                                                                                                                                                                                                                                                                                                                                                                                                  \\
\hline
                  & Проективность                                                                                                                                               & Инъективность                                                                                                                                                & Плоскость                                                                                                                                                    & Проективность                                                                                                                                               & Инъективность                                                                                                                                                & Плоскость                                                                                                                                                   \\ 
\hline
 $L_1(G)$           & \begin{tabular}{@{}c@{}}$G$ дискретна  \\ \ref{L1ModL1MetTopProjCharac}\end{tabular}                                                                     & \begin{tabular}{@{}c@{}}$G=\{e_G\}$ \\ \ref{StdModAreNotL1MGMetTopProj}, \ref{StdModFinGrL1MGMetrInjProjFlatCharac}\end{tabular}                                  & \begin{tabular}{@{}c@{}}$G$ любая  \\ \ref{LInfIsL1MetrInj}\end{tabular}                                                                                   & \begin{tabular}{@{}c@{}}$G$ дискретна  \\ \ref{L1ModL1MetTopProjCharac},\ref{MGMetTopProjInjFlatRedToL1}\end{tabular}                                   & \begin{tabular}{@{}c@{}}$G=\{e_G\}$ \\ \ref{StdModAreNotL1MGMetTopProj}, \ref{StdModFinGrL1MGMetrInjProjFlatCharac}\end{tabular}                                   & \begin{tabular}{@{}c@{}}$G$ любая  \\ \ref{LInfIsL1MetrInj},\ref{MGMetTopProjInjFlatRedToL1}\end{tabular}                                                 \\ 
\hline
 $L_p(G)$           & \begin{tabular}{@{}c@{}}$G=\{e_G\}$ \\ \ref{NoInfDimRefMetTopProjInjFlatModOverMthscrL1OrLInfty},\ref{LpFinGrL1MGMetrInjProjCharac}\end{tabular}                  & \begin{tabular}{@{}c@{}}$G=\{e_G\}$ \\ \ref{NoInfDimRefMetTopProjInjFlatModOverMthscrL1OrLInfty},\ref{LpFinGrL1MGMetrInjProjCharac}\end{tabular}                  & \begin{tabular}{@{}c@{}}$G=\{e_G\}$ \\ \ref{NoInfDimRefMetTopProjInjFlatModOverMthscrL1OrLInfty},\ref{StdModFinGrL1MGMetrInjProjFlatCharac}\end{tabular}           & \begin{tabular}{@{}c@{}}$G=\{e_G\}$ \\ \ref{NoInfDimRefMetTopProjInjFlatModOverMthscrL1OrLInfty},\ref{LpFinGrL1MGMetrInjProjCharac}\end{tabular}                 & \begin{tabular}{@{}c@{}}$G=\{e_G\}$ \\ \ref{NoInfDimRefMetTopProjInjFlatModOverMthscrL1OrLInfty},\ref{LpFinGrL1MGMetrInjProjCharac}\end{tabular}                   & \begin{tabular}{@{}c@{}}$G=\{e_G\}$ \\ \ref{NoInfDimRefMetTopProjInjFlatModOverMthscrL1OrLInfty},\ref{StdModFinGrL1MGMetrInjProjFlatCharac}\end{tabular}          \\
\hline
 $L_\infty(G)$      & \begin{tabular}{@{}c@{}}$G=\{e_G\}$ \\ \ref{StdModAreNotL1MGMetTopProj},\ref{LpFinGrL1MGMetrInjProjCharac}\end{tabular}                                           & \begin{tabular}{@{}c@{}}$G$ любая  \\ \ref{LInfIsL1MetrInj}\end{tabular}                                                                                  & \begin{tabular}{@{}c@{}}$G=\{e_G\}$ \\ \ref{StdModAreNotL1MGMetTopProj},\ref{StdModFinGrL1MGMetrInjProjFlatCharac}\end{tabular}                                    & \begin{tabular}{@{}c@{}}$G=\{e_G\}$ \\ \ref{StdModAreNotL1MGMetTopProj},\ref{LpFinGrL1MGMetrInjProjCharac}\end{tabular}                                          & \begin{tabular}{@{}c@{}}$G$ любая  \\ \ref{LInfIsL1MetrInj},\ref{MGMetTopProjInjFlatRedToL1}\end{tabular}                                                  & \begin{tabular}{@{}c@{}}$G=\{e_G\}$ \\ \ref{StdModAreNotL1MGMetTopProj},\ref{StdModFinGrL1MGMetrInjProjFlatCharac}\end{tabular}                                   \\ 
\hline
$M(G)$              & \begin{tabular}{@{}c@{}}$G$ дискретна  \\ \ref{L1MetTopProjAndMetrFlatOfMeasAlg}\end{tabular}                                                            & \begin{tabular}{@{}c@{}}$G=\{e_G\}$ \\ \ref{StdModAreNotL1MGMetTopProj},\ref{StdModFinGrL1MGMetrInjProjFlatCharac}\end{tabular}                                   & \begin{tabular}{@{}c@{}}$G$ дискретна  \\ \ref{MeasAlgIsL1TopFlat}\end{tabular}                                                                           & \begin{tabular}{@{}c@{}}$G$ любая  \\ \ref{MGModMGMetTopProjFlatCharac}\end{tabular}                                                                     & \begin{tabular}{@{}c@{}}$G=\{e_G\}$ \\ \ref{StdModAreNotL1MGMetTopProj},\ref{StdModFinGrL1MGMetrInjProjFlatCharac}\end{tabular}                                    & \begin{tabular}{@{}c@{}}$G$ любая  \\ \ref{MGModMGMetTopProjFlatCharac}\end{tabular}                                                                      \\ 
\hline
$C_0(G)$            & \begin{tabular}{@{}c@{}}$G=\{e_G\}$ \\ \ref{StdModAreNotL1MGMetTopProj},\ref{StdModFinGrL1MGMetrInjProjFlatCharac}\end{tabular}                                   & \begin{tabular}{@{}c@{}}$G$ конечна  \\ \ref{StdModAreNotL1MGMetTopProj},\ref{StdModFinGrL1MGMetrInjProjFlatCharac}\end{tabular}                         & \begin{tabular}{@{}c@{}}$G=\{e_G\}$ \\ \ref{StdModAreNotL1MGMetTopProj},\ref{StdModFinGrL1MGMetrInjProjFlatCharac}\end{tabular}                                    & \begin{tabular}{@{}c@{}}$G=\{e_G\}$ \\ \ref{StdModAreNotL1MGMetTopProj},\ref{StdModFinGrL1MGMetrInjProjFlatCharac}\end{tabular}                                  & \begin{tabular}{@{}c@{}}$G$ конечна  \\ \ref{StdModAreNotL1MGMetTopProj},\ref{StdModFinGrL1MGMetrInjProjFlatCharac}\end{tabular}                          & \begin{tabular}{@{}c@{}}$G=\{e_G\}$ \\ \ref{StdModAreNotL1MGMetTopProj},\ref{StdModFinGrL1MGMetrInjProjFlatCharac}\end{tabular}                                   \\ 
\hline          
$\mathbb{C}_\gamma$ & \begin{tabular}{@{}c@{}}$G$ компактна  \\ \ref{OneDimL1ModMetTopProjCharac}\end{tabular}                                                                  & \begin{tabular}{@{}c@{}}$G$ аменабельна  \\ \ref{OneDimL1ModMetTopInjFlatCharac}\end{tabular}                                                              & \begin{tabular}{@{}c@{}}$G$ аменабельна  \\ \ref{OneDimL1ModMetTopInjFlatCharac}\end{tabular}                                                               & \begin{tabular}{@{}c@{}}$G$ компактна  \\ \ref{OneDimL1ModMetTopProjCharac},\ref{MGMetTopProjInjFlatRedToL1}\end{tabular}                                & \begin{tabular}{@{}c@{}}$G$ аменабельна  \\ \ref{OneDimL1ModMetTopInjFlatCharac},\ref{MGMetTopProjInjFlatRedToL1}\end{tabular}                              & \begin{tabular}{@{}c@{}}$G$ аменабельна  \\ \ref{OneDimL1ModMetTopInjFlatCharac},\ref{MGMetTopProjInjFlatRedToL1}\end{tabular}                             \\ 
\hline
\multicolumn{7}{c}{\mbox{Гомологически тривиальные $L_1(G)$- и $M(G)$-модули в топологической теории}}                                                                                                                                                                                                                                                                                                                                                                                                                                                                                                                                                                                                                                                                                                                                                                                                                                                                                                                                        \\
					 
\hline            & \multicolumn{3}{c|}{$L_1(G)$-модули}                                                                                                                                                                                                                                                                                                                                                                                                                                                                 & \multicolumn{3}{c|}{$M(G)$-модули}                                                                                                                                                                                                                                                                                                                                                                                                                                                                  \\
\hline
                  & Проективность                                                                                                                                               & Инъективность                                                                                                                                                & Плоскость                                                                                                                                                    & Проективность                                                                                                                                               & Инъективность                                                                                                                                                & Плоскость                                                                                                                                                   \\ 
\hline
$L_1(G)$            & \begin{tabular}{@{}c@{}}$G$ дискретна  \\ \ref{L1ModL1MetTopProjCharac}\end{tabular}                                                                     & \begin{tabular}{@{}c@{}}$G$ конечна  \\ \ref{StdModAreNotL1MGMetTopProj}, \ref{StdModFinGrL1MGTopInjProjFlatCharac}\end{tabular}                         & \begin{tabular}{@{}c@{}}$G$ любая  \\ \ref{LInfIsL1MetrInj}\end{tabular}                                                                                   & \begin{tabular}{@{}c@{}}$G$ дискретна  \\ \ref{L1ModL1MetTopProjCharac},\ref{MGMetTopProjInjFlatRedToL1}\end{tabular}                                    & \begin{tabular}{@{}c@{}}$G$ конечна  \\ \ref{StdModAreNotL1MGMetTopProj}, \ref{StdModFinGrL1MGTopInjProjFlatCharac}\end{tabular}                         & \begin{tabular}{@{}c@{}}$G$ любая  \\ \ref{LInfIsL1MetrInj},\ref{MGMetTopProjInjFlatRedToL1}\end{tabular}                                                 \\ 
\hline
 $L_p(G)$           & \begin{tabular}{@{}c@{}}$G$ конечна  \\ \ref{NoInfDimRefMetTopProjInjFlatModOverMthscrL1OrLInfty},\ref{StdModFinGrL1MGTopInjProjFlatCharac}\end{tabular} & \begin{tabular}{@{}c@{}}$G$ конечна  \\ \ref{NoInfDimRefMetTopProjInjFlatModOverMthscrL1OrLInfty},\ref{StdModFinGrL1MGTopInjProjFlatCharac}\end{tabular} & \begin{tabular}{@{}c@{}}$G$ конечна  \\ \ref{NoInfDimRefMetTopProjInjFlatModOverMthscrL1OrLInfty},\ref{StdModFinGrL1MGTopInjProjFlatCharac}\end{tabular}  & \begin{tabular}{@{}c@{}}$G$ конечна  \\ \ref{NoInfDimRefMetTopProjInjFlatModOverMthscrL1OrLInfty},\ref{StdModFinGrL1MGTopInjProjFlatCharac}\end{tabular} & \begin{tabular}{@{}c@{}}$G$ конечна  \\ \ref{NoInfDimRefMetTopProjInjFlatModOverMthscrL1OrLInfty},\ref{StdModFinGrL1MGTopInjProjFlatCharac}\end{tabular} & \begin{tabular}{@{}c@{}}$G$ конечна  \\ \ref{NoInfDimRefMetTopProjInjFlatModOverMthscrL1OrLInfty},\ref{StdModFinGrL1MGTopInjProjFlatCharac}\end{tabular} \\ 
\hline
 $L_\infty(G)$      & \begin{tabular}{@{}c@{}}$G$ конечна  \\ \ref{StdModAreNotL1MGMetTopProj},\ref{StdModFinGrL1MGTopInjProjFlatCharac}\end{tabular}                          & \begin{tabular}{@{}c@{}}$G$ любая  \\ \ref{LInfIsL1MetrInj}\end{tabular}                                                                                  & \begin{tabular}{@{}c@{}}$G$ конечна  \\ \ref{StdModAreNotL1MGMetTopProj},\ref{StdModFinGrL1MGTopInjProjFlatCharac}\end{tabular}                           & \begin{tabular}{@{}c@{}}$G$ конечна  \\ \ref{StdModAreNotL1MGMetTopProj},\ref{StdModFinGrL1MGTopInjProjFlatCharac}\end{tabular}                          & \begin{tabular}{@{}c@{}}$G$ любая  \\ \ref{LInfIsL1MetrInj},\ref{MGMetTopProjInjFlatRedToL1}\end{tabular}                                                 & \begin{tabular}{@{}c@{}}$G$ конечна  \\ \ref{StdModAreNotL1MGMetTopProj},\ref{StdModFinGrL1MGTopInjProjFlatCharac}\end{tabular}                          \\ 
\hline
$M(G)$              & \begin{tabular}{@{}c@{}}$G$ дискретна  \\ \ref{L1MetTopProjAndMetrFlatOfMeasAlg}\end{tabular}                                                            & \begin{tabular}{@{}c@{}}$G$ конечна  \\ \ref{NoInfDimRefMetTopProjInjFlatModOverMthscrL1OrLInfty},\ref{StdModFinGrL1MGTopInjProjFlatCharac}\end{tabular} & \begin{tabular}{@{}c@{}}$G$ любая  \\ \ref{MeasAlgIsL1TopFlat}\end{tabular}                                                                                & \begin{tabular}{@{}c@{}}$G$ любая  \\ \ref{MGModMGMetTopProjFlatCharac}\end{tabular}                                                                      & \begin{tabular}{@{}c@{}}$G$ конечна  \\ \ref{NoInfDimRefMetTopProjInjFlatModOverMthscrL1OrLInfty},\ref{StdModFinGrL1MGTopInjProjFlatCharac}\end{tabular} & \begin{tabular}{@{}c@{}}$G$ любая  \\ \ref{MGModMGMetTopProjFlatCharac}\end{tabular}                                                                      \\ 
\hline
$C_0(G)$            & \begin{tabular}{@{}c@{}}$G$ конечна  \\ \ref{StdModAreNotL1MGMetTopProj},\ref{StdModFinGrL1MGTopInjProjFlatCharac}\end{tabular}                          & \begin{tabular}{@{}c@{}}$G$ конечна  \\ \ref{StdModAreNotL1MGMetTopProj},\ref{StdModFinGrL1MGTopInjProjFlatCharac}\end{tabular}                          & \begin{tabular}{@{}c@{}}$G$ конечна  \\ \ref{StdModAreNotL1MGMetTopProj},\ref{StdModFinGrL1MGTopInjProjFlatCharac}\end{tabular}                           & \begin{tabular}{@{}c@{}}$G$ конечна  \\ \ref{StdModAreNotL1MGMetTopProj},\ref{StdModFinGrL1MGTopInjProjFlatCharac}\end{tabular}                          & \begin{tabular}{@{}c@{}}$G$ конечна  \\ \ref{StdModAreNotL1MGMetTopProj},\ref{StdModFinGrL1MGTopInjProjFlatCharac}\end{tabular}                          & \begin{tabular}{@{}c@{}}$G$ конечна  \\ \ref{StdModAreNotL1MGMetTopProj},\ref{StdModFinGrL1MGTopInjProjFlatCharac}\end{tabular}                          \\ 
\hline          
$\mathbb{C}_\gamma$ & \begin{tabular}{@{}c@{}}$G$ компактна  \\ \ref{OneDimL1ModMetTopProjCharac}\end{tabular}                                                                  & \begin{tabular}{@{}c@{}}$G$ аменабельна  \\ \ref{OneDimL1ModMetTopInjFlatCharac}\end{tabular}                                                              & \begin{tabular}{@{}c@{}}$G$ аменабельна  \\ \ref{OneDimL1ModMetTopInjFlatCharac}\end{tabular}                                                               & \begin{tabular}{@{}c@{}}$G$ компактна  \\ \ref{OneDimL1ModMetTopProjCharac},\ref{MGMetTopProjInjFlatRedToL1}\end{tabular}                                 & \begin{tabular}{@{}c@{}}$G$ аменабельна  \\ \ref{OneDimL1ModMetTopInjFlatCharac},\ref{MGMetTopProjInjFlatRedToL1}\end{tabular}                             & \begin{tabular}{@{}c@{}}$G$ аменабельна  \\ \ref{OneDimL1ModMetTopInjFlatCharac},\ref{MGMetTopProjInjFlatRedToL1}\end{tabular}                             \\ 
\hline
\multicolumn{7}{c}{\mbox{Гомологически тривиальные $L_1(G)$- и $M(G)$-модули в относительной теории}}                                                                                                                                                                                                                                                                                                                                                                                                                                                                                                                                                                                                                                                                                                                                                                                                                                                                                                                                           \\

\hline            & \multicolumn{3}{c|}{$L_1(G)$-модули}                                                                                                                                                                                                                                                                                                                                                                                                                                                                 & \multicolumn{3}{c|}{$M(G)$-модули}                                                                                                                                                                                                                                                                                                                                                                                                                                                                  \\
\hline
                  & Проективность                                                                                                                                               & Инъективность                                                                                                                                                & Плоскость                                                                                                                                                    & Проективность                                                                                                                                               & Инъективность                                                                                                                                                & Плоскость                                                                                                                                                   \\ 
\hline
$L_1(G)$            & \begin{tabular}{@{}c@{}}$G$ любая  \\ \mbox{\cite{DalPolHomolPropGrAlg}, \S 6}\end{tabular}                                                               & \begin{tabular}{@{}c@{}}$G$ аменабельна  \\ и дискретна \\ \mbox{\cite{DalPolHomolPropGrAlg}, \S 6}\end{tabular}                                 & \begin{tabular}{@{}c@{}}$G$ любая  \\ \mbox{\cite{DalPolHomolPropGrAlg}, \S 6}\end{tabular}                                                                & \begin{tabular}{@{}c@{}}$G$ любая  \\ \mbox{\cite{RamsHomPropSemgroupAlg}, \S 3.5}\end{tabular}                                                           & \begin{tabular}{@{}c@{}}$G$ аменабельна  \\ и дискретна \\ \mbox{\cite{RamsHomPropSemgroupAlg}, \S 3.5}\end{tabular}                             & \begin{tabular}{@{}c@{}}$G$ любая  \\ \mbox{\cite{RamsHomPropSemgroupAlg}, \S 3.5}\end{tabular}                                                           \\ 
\hline
 $L_p(G)$           & \begin{tabular}{@{}c@{}}$G$ компактна  \\ \mbox{\cite{DalPolHomolPropGrAlg}, \S 6}\end{tabular}                                                           & \begin{tabular}{@{}c@{}}$G$ аменабельна  \\ \cite{RachInjModAndAmenGr}\end{tabular}                                                                        & \begin{tabular}{@{}c@{}}$G$ аменабельна  \\ \cite{RachInjModAndAmenGr}\end{tabular}                                                                         & \begin{tabular}{@{}c@{}}$G$ компактна  \\ \mbox{\cite{RamsHomPropSemgroupAlg}, \S 3.5}\end{tabular}                                                       & \begin{tabular}{@{}c@{}}$G$ аменабельна  \\ \mbox{\cite{RamsHomPropSemgroupAlg}, \S 3.5}, \cite{RachInjModAndAmenGr}\end{tabular}                          & \begin{tabular}{@{}c@{}}$G$ аменабельна  \\ \mbox{\cite{RamsHomPropSemgroupAlg}, \S 3.5}\end{tabular}                                                      \\
\hline
 $L_\infty(G)$      & \begin{tabular}{@{}c@{}}$G$ конечна  \\ \mbox{\cite{DalPolHomolPropGrAlg}, \S 6}\end{tabular}                                                            & \begin{tabular}{@{}c@{}}$G$ любая  \\ \mbox{\cite{DalPolHomolPropGrAlg}, \S 6}\end{tabular}                                                               & \begin{tabular}{@{}c@{}}$G$ аменабельна  \\ \mbox{\cite{DalPolHomolPropGrAlg}, \S 6}\end{tabular}                                                           & \begin{tabular}{@{}c@{}}$G$ конечна  \\ \mbox{\cite{RamsHomPropSemgroupAlg}, \S 3.5}\end{tabular}                                                        & \begin{tabular}{@{}c@{}}$G$ любая  \\ \mbox{\cite{RamsHomPropSemgroupAlg}, \S 3.5}\end{tabular}                                                           & \begin{tabular}{@{}c@{}}$G$ аменабельна  \\ ($\implies$)\mbox{\cite{RamsHomPropSemgroupAlg}, \S 3.5}\end{tabular}                                          \\ 
\hline
$M(G)$              & \begin{tabular}{@{}c@{}}$G$ дискретна  \\ \mbox{\cite{DalPolHomolPropGrAlg}, \S 6}\end{tabular}                                                          & \begin{tabular}{@{}c@{}}$G$ аменабельна \\ \mbox{\cite{DalPolHomolPropGrAlg}, \S 6}\end{tabular}                                                           & \begin{tabular}{@{}c@{}}$G$ любая  \\ \mbox{\cite{RamsHomPropSemgroupAlg}, \S 3.5}\end{tabular}                                                            & \begin{tabular}{@{}c@{}}$G$ любая  \\ \mbox{\cite{RamsHomPropSemgroupAlg}, \S 3.5}\end{tabular}                                                           & \begin{tabular}{@{}c@{}}$G$ аменабельна  \\ \mbox{\cite{RamsHomPropSemgroupAlg}, \S 3.5}\end{tabular}                                                      & \begin{tabular}{@{}c@{}}$G$ любая  \\ \mbox{\cite{RamsHomPropSemgroupAlg}, \S 3.5}\end{tabular}                                                           \\ 
\hline
$C_0(G)$            & \begin{tabular}{@{}c@{}}$G$ компактна  \\ \mbox{\cite{DalPolHomolPropGrAlg}, \S 6}\end{tabular}                                                           & \begin{tabular}{@{}c@{}}$G$ конечна  \\ \mbox{\cite{DalPolHomolPropGrAlg}, \S 6}\end{tabular}                                                            & \begin{tabular}{@{}c@{}}$G$ аменабельна  \\ \mbox{\cite{DalPolHomolPropGrAlg}, \S 6}\end{tabular}                                                           & \begin{tabular}{@{}c@{}}$G$ компактна  \\ \mbox{\cite{RamsHomPropSemgroupAlg}, \S 3.5}\end{tabular}                                                       & \begin{tabular}{@{}c@{}}$G$ конечна  \\ \mbox{\cite{RamsHomPropSemgroupAlg}, \S 3.5}\end{tabular}                                                        & \begin{tabular}{@{}c@{}}$G$ аменабельна  \\ \mbox{\cite{RamsHomPropSemgroupAlg}, \S 3.5}\end{tabular}                                                      \\ 
\hline          
$\mathbb{C}_\gamma$ & \begin{tabular}{@{}c@{}}$G$ компактна  \\ \ref{OneDimL1ModMetTopProjCharac}\end{tabular}                                                                  & \begin{tabular}{@{}c@{}}$G$ аменабельна  \\ \ref{OneDimL1ModMetTopInjFlatCharac}\end{tabular}                                                              & \begin{tabular}{@{}c@{}}$G$ аменабельна  \\ \ref{OneDimL1ModMetTopInjFlatCharac}\end{tabular}                                                               & \begin{tabular}{@{}c@{}}$G$ компактна  \\ \ref{OneDimL1ModMetTopProjCharac},\ref{MGMetTopProjInjFlatRedToL1}\end{tabular}                                 & \begin{tabular}{@{}c@{}}$G$ аменабельна  \\ \ref{OneDimL1ModMetTopInjFlatCharac},\ref{MGMetTopProjInjFlatRedToL1}\end{tabular}                             & \begin{tabular}{@{}c@{}}$G$ аменабельна  \\ \ref{OneDimL1ModMetTopInjFlatCharac},\ref{MGMetTopProjInjFlatRedToL1}\end{tabular}                             \\                   
\hline
\end{longtable}
\end{scriptsize}

Следует сказать, что результаты о модулях $L_p(G)$ верны для обоих видов внешнего умножения $\convol_p$ и $\cdot_p$. Наиболее интересный результат параграфа состоит в том, что $L_1(G)$-модуль $L_1(G)$ метрически или топологически проективен только для дискретной группы $G$. Для метрического случая это было доказано в [\cite{GravInjProjBanMod}, теорема 4.14(ii)]. 

%----------------------------------------------------------------------------------------
%	An example of small category.
%----------------------------------------------------------------------------------------

\section{Пример ``маленькой'' категории}
\label{SectionAnExampleOfSmallCategory}

Как мы видели по многим примерам, большинство стандартных модулей анализа оказываются гомологически нетривиальными по отношению к большим категориям, таким, например, как категория всех банаховых модулей. Ситуация может кардинально измениться для сравнительно ``маленьких'' категорий. Данный параграф посвящен построению содержательного примера такого рода.

%----------------------------------------------------------------------------------------
%	Preliminaries on measure theory and L_p-spaces
%----------------------------------------------------------------------------------------

\subsection{Предварительные сведения по теории меры и \texorpdfstring{$L_p$}{Lp}-пространствам}
\label{SubSectionPreliminariesOnMeasureTheoryAndLpSpaces}

Прежде чем переходить к основной теме, мы напомним некоторые основные факты из теории меры и договоримся об обозначениях. Все эти предварительные сведения можно найти в первых двух томах \cite{FremMeasTh}. Пусть $(\Omega, \Sigma, \mu)$ --- пространство с мерой. Будем говорить, что измеримое множество $E$ является атомом, если $\mu(E)>0$ и для любого измеримого подмножества $F$ в $E$ либо $F$ либо $E\setminus F$ имеют меру $0$. Непосредственно из определения следует, что атомы в строго локализуемых пространствах с мерой имеют конечную меру. Вообще говоря, атом может и не быть одноточечным множеством.

Пространство с мерой называется неатомическим, если его мера не имеет ни одного атома. Пространство с мерой называется атомическим, если каждое измеримое множество положительной меры содержит атом. С помощью леммы Цорна легко показать, что атомическое пространство с мерой можно представить в виде объединения непересекающихся атомов.  Это семейство счетно, если пространство с мерой $\sigma$-конечно. Эти факты позволяют сказать, что структура атомических пространств с мерой полностью изучена. 

Структура строго локализуемых неатомических пространств с мерой описана Дороти Махарам [\cite{FremMeasTh}, теорема 332B], но нам понадобится лишь следующее свойство таких пространств с мерой [\cite{FremMeasTh}, предложение 215D]: если $E$ --- измеримое множество положительной меры в неатомическом пространстве с мерой, то для всех $0\leq t\leq \mu(E)$ существует измеримое подмножество $F$ в $E$ такое, что $\mu(F)=t$.

Допустим, $(\Omega,\Sigma,\mu)$ является $\sigma$-конечным пространством с мерой, тогда существуют атомическая мера $\mu_1:\Sigma\to[0,+\infty]$ и неатомическая мера $\mu_2:\Sigma\to[0,+\infty]$ такие, что $\mu=\mu_1+\mu_2$. Более того, существуют измеримые множества $\Omega_a^{\mu}$ и $\Omega_{na}^{\mu}=\Omega\setminus \Omega_a^{\mu}$ такие, что $\mu_1(\Omega_{na}^{\mu})=\mu_2(\Omega_a^{\mu})=0$. Множества $\Omega_a^{\mu}$ и $\Omega_{na}^{\mu}$ называются соответственно атомической и неатомической частью пространства с мерой $(\Omega,\Sigma,\mu)$.

Вообще говоря, две меры могут быть ни абсолютно непрерывны, ни сингулярны по отношению друг к другу. На этот случай есть теорема Лебега о разложении мер. Для двух заданных $\sigma$-конечных мер $\mu$ и $\nu$, заданных на измеримом пространстве $(\Omega,\Sigma)$, существует измеримая функция $\rho_{\nu,\mu}:\Omega\to\mathbb{R}$, $\sigma$-конечная мера $\nu_s:\Sigma\to[0,+\infty]$ и два измеримых множества $\Omega_s^{\nu,\mu}$, $\Omega_c^{\nu,\mu}=\Omega\setminus\Omega_s^{\nu,\mu}$ таких, что
$\nu=\rho_{\nu,\mu}\mu+\nu_s$ и $\mu(\Omega_s^{\nu,\mu})=\nu_s(\Omega_c^{\nu,\mu})=0$, то есть $\mu\perp\nu_s$.

Перейдем к обсуждению $L_p$-пространств. Пусть $1\leq p\leq+\infty$ и $(\Omega,\Sigma,\mu)$ --- пространство с мерой. Если все пространство $\Omega$ является атомом, то его $L_p$-пространство одномерно и существует изометрический изоморфизм
$$
J_p:L_p(\Omega,\mu)\to \ell_p(\mathbb{N}_1):f\mapsto\left(1\mapsto \mu(\Omega)^{1/p-1}\int_{\Omega} f(\omega)d\mu(\omega)\right).
$$
Если $\Omega=\bigcup_{\lambda\in\Lambda}\Omega_\lambda$ есть представление $\Omega$ как объединения непересекающихся измеримых множеств, то для всех $1\leq p\leq+\infty$ мы имеем изометрический изоморфизм
$$
I_p:L_p(\Omega,\mu)\to \bigoplus\nolimits_p\{ L_p(\Omega_\lambda,\mu|_{\Omega_\lambda}):\lambda\in\Lambda\}: f\mapsto (\lambda\mapsto f|_{\Omega_\lambda}).
$$
Если каждое множество $\Omega_\lambda$ является атомом, то $\Omega$ --- атомическое пространство с мерой, и мы получаем еще один изометрический изоморфизм
$$
\widetilde{I}_p:L_p(\Omega,\mu)\to \ell_p(\Lambda):f\mapsto\left (\lambda\mapsto \mu(\Omega_\lambda)^{1/p-1}\int_{\Omega_\lambda} f(\omega)d\mu(\omega)\right).
$$
Замечание: работая с индексами $p$, мы будем полагать по определению, что $1/0=\infty$ и $1/\infty=0$. Еще один полезный прием в изучении $L_p$-пространств --- это так называемая замена плотности: если $\rho:\Omega\to(0,+\infty)$ --- измеримая функция, то оператор
$$
\bar{I}_p:L_p(\Omega,\mu)\to L_p(\Omega,\rho\mu): f\mapsto\rho^{-1/p} f
$$
является изометрическим изоморфизмом. Для различных значений $p$ бесконечномерные $L_p$-пространства не топологически изоморфны. Это можно доказать с использованием понятий котипа и типа [\cite{KalAlbTopicsBanSpTh}, теорема 6.2.14]. Очевидно, в конечномерном случае изоморфизм существует только для пространств одинаковой размерности. Более точно: если $\Lambda$ --- конечное множество и $1\leq p,q\leq +\infty$, то существует константа $C_{p,q}>0$ такая, что $\Vert x\Vert_{\ell_p(\Lambda)}\leq C_{p,q}\Vert x\Vert_{\ell_q(\Lambda)}$ для всех $x\in\mathbb{C}^\Lambda$.

%----------------------------------------------------------------------------------------
%	The category of B(\Omega,\Sigma)-module L_p
%----------------------------------------------------------------------------------------

\subsection{Категория \texorpdfstring{$B(\Omega,\Sigma)$}{B(Omega,Sigma)}-модулей \texorpdfstring{$L_p$}{Lp}}
\label{SubSectionTheCategoryOfBOmegaSigmaModulesLp}

``Маленькая'' категория, которую мы будем изучать --- это категория $B(\Omega,\Sigma)$-модулей вида $L_p(\Omega,\mu)$ на некотором измеримом пространстве $(\Omega,\Sigma)$ для различных $\sigma$-конечных положительных мер $\mu$ и различных $1\leq p\leq +\infty$. Мы обозначим эту категорию как $B(\Omega,\Sigma)-\mathbf{mod(L)}$. Мы покажем, что все ее модули являются метрически и топологически проективными, инъективными и плоскими. По пути мы дадим полное описание топологически сюръективных, топологически инъективных, коизометрических и изометрических операторов умножения между $L_p$-пространствами. В \cite{HelTensProdAndMultModLp} Хелемский полностью описал морфизмы $B(\Omega,\Sigma)-\mathbf{mod(L)}$, но только для случая локально компактного пространства $\Omega$ с борелевской $\sigma$-алгеброй. Внимательное изучение его доказательства показывает, что это описание верно для всех $\sigma$-конечных пространств с мерой. Чтобы правильно сформулировать результат, нам понадобится несколько обозначений. Через $L_0(\Omega,\Sigma)$ мы обозначим линейное пространство измеримых функций на $\Omega$. Для $1\leq p,q\leq +\infty$ и положительных $\sigma$-конечных мер $\mu,\nu$ на измеримом пространстве $(\Omega,\Sigma)$ мы обозначим $\Omega_+:=\{\omega\in\Omega_c^{\nu,\mu}:\rho_{\nu,\mu}(\omega)>0\}$ и
$$
L_{p,q,\mu,\nu}(\Omega):=
\begin{cases}
\{g\in L_0(\Omega,\Sigma):g\in L_{pq/(p-q)}(\Omega,\rho_{\nu,\mu}^{p/(p-q)}\mu),\quad g|_{\Omega\setminus\Omega_+}=0\}&\text{ если }\quad p>q\\
\{g\in L_0(\Omega,\Sigma):g\rho_{\nu,\mu}^{1/p}\in L_{\infty}(\Omega,\mu),\quad g|_{\Omega\setminus\Omega_+}=0\}&\text{ если }\quad p=q\\
\{g\in L_0(\Omega,\Sigma):g\rho_{\nu,\mu}^{1/p}\mu^{pq/(p-q)}\in L_{\infty}(\Omega,\mu),\quad g|_{\Omega\setminus\Omega_a^{\mu}}=0\}&\text{ если }\quad p<q,\\
\end{cases}
$$
$$
\Vert g\Vert_{L_{p,q,\mu,\nu}(\Omega)}:=
\begin{cases}
\Vert g\Vert_{L_{pq/(p-q)}(\Omega,\rho_{\nu,\mu}^{p/(p-q)}\mu)}&\text{ если }\quad p>q\\
\Vert g\rho_{\nu,\mu}^{1/p}\Vert_{L_{\infty}(\Omega,\mu)}&\text{ если }\quad p=q\\
\Vert g\rho_{\nu,\mu}^{1/p}\mu^{pq/(p-q)}\Vert_{L_{\infty}(\Omega,\mu)}&\text{ если }\quad p<q.\\
\end{cases}
$$

\begin{theorem}[\cite{HelTensProdAndMultModLp}, теорема 4.1]\label{LpModMorphCharac}
Пусть $(\Omega,\Sigma)$ --- измеримое пространство, $1\leq p,q\leq +\infty$ и $\mu,\nu$ --- две $\sigma$-конечные меры на $(\Omega, \Sigma)$. Тогда существует изометрический изоморфизм
$$
\mathcal{I}_{p,q,\mu,\nu}:L_{p,q,\mu,\nu}(\Omega)\to\operatorname{Hom}_{B(\Omega,\Sigma)-\mathbf{mod(L)}}(L_p(\Omega,\mu),L_q(\Omega,\nu)):g\mapsto (f\mapsto g f).
$$
\end{theorem}

Проще говоря, все морфизмы в $B(\Omega,\Sigma)-\mathbf{mod(L)}$ являются операторами умножения.

%----------------------------------------------------------------------------------------
%	Multiplication operators
%----------------------------------------------------------------------------------------

\subsection{Операторы умножения}
\label{SubSectionMultiplicationOperators}

Пусть $(\Omega,\Sigma,\mu)$ и $(\Omega,\Sigma,\nu)$ --- два пространства с мерой с одной и той же $\sigma$-алгеброй измеримых множеств. Для заданной функции $g\in L_0(\Omega,\Sigma)$ и чисел $1\leq p,q\leq +\infty$ мы определим оператор умножения
$$
M_g:L_p(\Omega,\mu)\to L_q(\Omega,\nu): f\mapsto g f.
$$ 
Конечно, требуются некоторые ограничения на $g$, $\mu$ и $\nu$ чтобы оператор $M_g$ был корректно определен. Для заданного множества $E\in\Sigma$ через $M_g^E$ мы обозначим линейный оператор
$$
M_g^E:L_p(E,\mu|_E)\to L_q(E,\nu|_E):f\mapsto g|_E f.
$$
Он корректно определен, так как равенство $f|_{\Omega\setminus E}=0$ влечет $M_g(f)|_{\Omega\setminus E}=0$. 

\begin{proposition}\label{MultpOpSurjInjDesc} Пусть $(\Omega,\Sigma,\mu)$ --- пространство с мерой и $g\in L_0(\Omega,\Sigma)$. Обозначим $Z_g:=g^{-1}(\{0\})$. Тогда для оператора $M_g:L_p(\Omega,\mu)\to L_q(\Omega,\mu)$ выполнено:

$i)$ $\operatorname{Ker}(M_g)=\{f\in L_p(\Omega,\mu):f|_{\Omega\setminus {Z_g}}=0\}$, то есть оператор $M_g$ инъективен тогда и только тогда, когда $\mu(Z_g)=0$;

$ii)$ $\operatorname{Im}(M_g)\subset\{h\in L_q(\Omega,\mu): h|_{Z_g}=0\}$, поэтому если $M_g$ сюръективен, то $\mu(Z_g)=0$.

\end{proposition}
\begin{proof}
$i)$ Желаемое равенство следует из цепочки эквивалентностей:
$$
f\in\operatorname{Ker}(M_g)
\Longleftrightarrow g f=0
\Longleftrightarrow f|_{\Omega\setminus Z_g}=0.
$$
$ii)$ Так как $g|_{Z_g}=0$, то для всех функций $f\in L_p(\Omega,\mu)$ выполнено $M_g(f)|_{Z_g}=(g f)|_{Z_g}=0$ и мы получаем нужное включение. Если оператор $M_g$ сюръективен, то, очевидно, $\mu(Z_g)=0$.
\end{proof} 

Для заданного измеримого множества $E\in\Sigma$ и функции $f\in L_0(E,\Sigma|_{E})$ через $\widetilde{f}$ мы будем обозначать функцию из $L_0(\Omega, \Sigma)$ такую, что $\widetilde{f}|_E=f$ и $\widetilde{f}|_{\Omega\setminus E}=0$.

\begin{proposition}\label{MultOpDecompDecomp} Пусть $(\Omega,\Sigma,\mu)$, $(\Omega,\Sigma,\nu)$ --- два пространства с мерой и $1\leq p,q\leq +\infty$. Допустим, имеется представление $\Omega=\bigcup_{\lambda\in\Lambda}\Omega_\lambda$ множества $\Omega$ в виде конечного объединения непересекающихся измеримых множеств. Тогда 

$i)$ оператор $M_g$ топологически инъективен тогда и только тогда, когда операторы $M_g^{\Omega_\lambda}$ топологически инъективны для всех $\lambda\in\Lambda$;

$ii)$ оператор $M_g$ топологически сюръективен тогда и только тогда, когда операторы $M_g^{\Omega_\lambda}$ топологически сюръективны для всех $\lambda\in\Lambda$;

$iii)$ если оператор $M_g$ изометричен, то таковы и операторы $M_g^{\Omega_\lambda}$ для всех $\lambda\in\Lambda$;

$iv)$ если оператор $M_g$ коизометричен, то таковы и операторы $M_g^{\Omega_\lambda}$ для всех $\lambda\in\Lambda$.
\end{proposition}
\begin{proof}
$i)$ Пусть оператор $M_g$ $c$-топологически инъективен. Зафиксируем индекс $\lambda\in\Lambda$ и функцию $f\in L_p(\Omega_\lambda,\mu|_{\Omega_\lambda})$. Тогда 
$$
\Vert M_g^{\Omega_\lambda}(f)\Vert_{L_q(\Omega_\lambda,\nu|_{\Omega_\lambda})}
=\Vert g \widetilde{f}\Vert_{L_q(\Omega,\nu)}
\geq c^{-1}\Vert\widetilde{f}\Vert_{L_p(\Omega,\mu)}
=c^{-1}\Vert f\Vert_{L_p(\Omega_\lambda,\mu|_{\Omega_\lambda})}.
$$
Следовательно, операторы $M_g^{\Omega_\lambda}$ $c$-топологически инъективны для всех $\lambda\in\Lambda$. 

Обратно, допустим, что операторы $M_g^{\Omega_\lambda}$ $c'$-топологически инъективны для всех $\lambda\in\Lambda$. Тогда для любой функции $f\in L_p(\Omega,\mu)$ мы имеем
$$
\Vert M_g(f)\Vert_{L_q(\Omega,\nu)}
=\left\Vert\left(\Vert M_g^{\Omega_\lambda}(f|_{\Omega_\lambda})\Vert_{L_q(\Omega_\lambda,\nu|_{\Omega_\lambda})}:\lambda\in\Lambda\right)\right\Vert_{\ell_q(\Lambda)}
\geq (c')^{-1}\left\Vert\left(\Vert f|_{\Omega_\lambda}\Vert_{L_p(\Omega_\lambda,\mu|_{\Omega_\lambda})}:\lambda\in\Lambda\right)\right\Vert_{\ell_q(\Lambda)}
$$
$$
\geq (c')^{-1} C_{p,q}^{-1}\left\Vert\left(\Vert f|_{\Omega_\lambda}\Vert_{L_p(\Omega_\lambda,\mu|_{\Omega_\lambda})}:\lambda\in\Lambda\right)\right\Vert_{\ell_p(\Lambda)}
=(c')^{-1}C_{p,q}^{-1}\Vert f\Vert_{L_p(\Omega,\mu)}.
$$
Следовательно, оператор $M_g$ $c$-топологически инъективен для $c=c'C_{p,q}>0$.

$ii)$ Допустим, оператор $M_g$ $c$-топологически сюръективен. Зафиксируем индекс $\lambda\in\Lambda$ и функцию $h\in L_q(\Omega_\lambda,\nu|_{\Omega_\lambda})$. Тогда существует функция $f\in L_p(\Omega,\mu)$ такая, что $M_g(f)=\widetilde{h}$ и $\Vert f\Vert_{L_p(\Omega,\mu)}\leq c\Vert \widetilde{h}\Vert_{L_q(\Omega,\nu)}$. Следовательно, $M_g^{\Omega_\lambda}(f|_{\Omega_\lambda})=\widetilde{h}|_{\Omega_\lambda}=h$ и $\Vert f|_{\Omega_\lambda}\Vert_{L_p(\Omega_\lambda,\mu|_{\Omega_\lambda})}\leq \Vert f\Vert_{L_p(\Omega,\mu)}\leq c\Vert\widetilde{h}\Vert_{L_q(\Omega,\nu)}=c\Vert h\Vert_{L_q(\Omega_\lambda,\nu|_{\Omega_\lambda})}$. Так как функция $h$ произвольна, то операторы $M_g^{\Omega_\lambda}$ $c$-топологически сюръективны.

Обратно, допустим операторы $M_g^{\Omega_\lambda}$ $c'$-топологически сюръективны для всех $\lambda\in\Lambda$. Пусть $h$ --- произвольная функция из $L_q(\Omega,\nu)$. Из предположения мы получаем, что для каждого $\lambda\in\Lambda$ существует функция $f_\lambda\in L_p(\Omega_\lambda,\mu|_{\Omega_\lambda})$ такая, что $M_g^{\Omega_\lambda}(f_\lambda)=h|_{\Omega_\lambda}$ и $\Vert f_\lambda\Vert_{L_p(\Omega_\lambda,\mu|_{\Omega_\lambda})}\leq c'\Vert h|_{\Omega_\lambda}\Vert_{L_q(\Omega_\lambda,\nu|_{\Omega_\lambda})}$. Определим функцию $f\in L_0(\Omega,\Sigma)$ так, что $f|_{\Omega_\lambda}=f_\lambda$. Тогда
$$
\Vert f\Vert_{L_p(\Omega,\mu)}
=\left\Vert\left(\Vert f_\lambda\Vert_{L_p(\Omega_\lambda,\mu|_{\Omega_\lambda})}:\lambda\in\Lambda\right)\right\Vert_{\ell_p(\Lambda)}
\leq c'\left\Vert\left(\Vert h|_{\Omega_\lambda}\Vert_{L_q(\Omega_\lambda,\nu|_{\Omega_\lambda})}:\lambda\in\Lambda\right)\right\Vert_{\ell_p(\Lambda)}
$$
$$
\leq c'C_{p,q}\left\Vert\left(\Vert h|_{\Omega_\lambda}\Vert_{L_q(\Omega_\lambda,\nu|_{\Omega_\lambda})}:\lambda\in\Lambda\right)\right\Vert_{\ell_q(\Lambda)}
=c'C_{p,q}\Vert h\Vert_{L_q(\Omega,\nu)}.
$$
Очевидно, $M_g(f)=h$. Так как функция $h$ произвольна, то оператор $M_g$ $c$-топологически сюръективен для $c=c'C_{p,q}>0$.

$iii)$ Зафиксируем индекс $\lambda\in\Lambda$ и функцию $f\in L_p(\Omega_\lambda,\mu|_{\Omega_\lambda})$. Тогда 
$$
\Vert M_g^{\Omega_\lambda}(f)\Vert_{L_q(\Omega_\lambda,\nu|_{\Omega_\lambda})}
=\Vert g \widetilde{f}\Vert_{L_q(\Omega,\nu)}
=\Vert\widetilde{f}\Vert_{L_p(\Omega,\mu)}
=\Vert f\Vert_{L_p(\Omega_\lambda,\mu|_{\Omega_\lambda})}.
$$
Значит, операторы $M_g^{\Omega_\lambda}$ изометричны для всех $\lambda\in\Lambda$.

$iv)$ Зафиксируем индекс $\lambda\in\Lambda$. Так как оператор $M_g$ коизометричен, то он сжимающий и $1$-топологически сюръективный. Из доказательства пункта $ii)$ мы получаем, что оператор $M_g^{\Omega_\lambda}$ $1$-топологически сюръективен. Пусть $f$ --- произвольная функция из $ L_p(\Omega_\lambda,\mu|_{\Omega_\lambda})$. Так как $M_g$ сжимающий, то
$$
\Vert M_g^{\Omega_\lambda}(f)\Vert_{L_q(\Omega_\lambda,\nu|_{\Omega_\lambda})}
=\Vert M_g(\widetilde{f})\chi_{\Omega_\lambda}\Vert_{L_q(\Omega,\nu)}
=\Vert M_g(\widetilde{f}\chi_{\Omega_\lambda})\Vert_{L_q(\Omega,\nu)}
\leq \Vert\widetilde{f}\chi_{\Omega_\lambda}\Vert_{L_p(\Omega,\mu)}
=\Vert f\Vert_{L_p(\Omega_{\lambda},\mu|_{\Omega_\lambda})}.
$$
Следовательно, оператор $M_g^{\Omega_\lambda}$ сжимающий и $1$-топологически сюръективный, то есть коизометрический.
\end{proof}

\begin{proposition}\label{MultOpCharacBtwnTwoSingMeasSp} Пусть $(\Omega,\Sigma,\mu)$ и $(\Omega,\Sigma,\nu)$ --- два $\sigma$-конечных пространства с мерой. Пусть $1\leq p,q\leq +\infty$ и $g\in L_0(\Omega,\Sigma)$. Если $\mu\perp\nu$, то оператор $M_g:L_p(\Omega,\mu)\to L_q(\Omega,\nu)$ нулевой.
\end{proposition}
\begin{proof} В силу того, что $\mu\perp\nu$ существует множество $\Omega_s^{\nu,\mu}\in\Sigma$ такое, что $\mu(\Omega_s^{\nu,\mu})=\nu(\Omega_c^{\nu,\mu})=0$, где $\Omega_c^{\nu,\mu}=\Omega\setminus\Omega_s^{\nu,\mu}$. Так как $\mu(\Omega_s^{\nu,\mu})=0$, то $\chi_{\Omega_c^{\nu,\mu}}=\chi_{\Omega}$ в $L_p(\Omega,\mu)$ и $\chi_{\Omega_c^{\nu,\mu}}=0$ в $L_q(\Omega,\nu)$. Теперь для всех функций $f\in L_p(\Omega,\mu)$ мы имеем $M_g(f)=M_g(f \chi_{\Omega})=M_g(f \chi_{\Omega_c^{\nu,\mu}})=g f\chi_{\Omega_c^{\nu,\mu}}=0$. Следовательно, $M_g=0$.
\end{proof}

С этого момента начинается главная часть нашего исследования операторов умножения. Мы покажем, что $\langle$~изометрические / топологически инъективные~$\rangle$ морфизмы в $B(\Omega,\Sigma)-\mathbf{mod(L)}$ являются коретракциями. Аналогично, $\langle$~строго коизометрические / топологически сюръективные~$\rangle$ морфизмы в $B(\Omega,\Sigma)-\mathbf{mod(L)}$ являются ретракциями. Позже, используя эти описания, мы легко опишем, все метрически и топологически проективные, инъективные и плоские модули категории $B(\Omega,\Sigma)-\mathbf{mod(L)}$.

\begin{proposition}\label{MultpOpPropIfPeqqualsQ} Пусть $(\Omega,\Sigma,\mu)$ --- пространство с мерой и $g\in L_0(\Omega,\Sigma)$. Тогда 

$i)$ оператор $M_g:L_p(\Omega,\mu)\to L_p(\Omega,\mu)$ ограничен тогда и только тогда, когда $g\in L_\infty(\Omega,\mu)$;

$ii)$оператор $M_g$ --- топологический изоморфизм тогда и только тогда, когда $c\leq |g|\leq C$ для некоторых $C,c>0$.
\end{proposition}
\begin{proof}
$i)$ Допустим существует множество $E\in\Sigma$ положительной меры такое, что $|g|_E|>\Vert M_g\Vert$. Тогда
$$
\Vert M_g(\chi_E)\Vert_{L_p(\Omega,\mu)}
=\Vert g\chi_E\Vert_{L_p(\Omega,\mu)}
>\Vert M_g\Vert\Vert\chi_E\Vert_{L_p(\Omega,\mu)}.
$$
Противоречие, значит, для всех множеств $E\in\Sigma$ положительной меры выполнено $|g|_E|\leq \Vert M_g\Vert$, то есть $|g|\leq \Vert M_g\Vert$ и $g\in L_\infty(\Omega,\mu)$. Обратно, пусть $g\in L_\infty(\Omega,\mu)$, тогда $|g|\leq C$ для некоторого $C>0$. Для произвольной функции $f\in L_p(\Omega,\mu)$ мы имеем
$$
\Vert M_g(f)\Vert_{L_p(\Omega,\mu)}
=\Vert g  f\Vert_{L_p(\Omega,\mu)}
\leq C\Vert f\Vert_{L_p(\Omega,\mu)}.
$$
Значит, $M_g\in\mathcal{B}(L_p(\Omega,\mu))$.

$ii)$ Заметим, что $M_g^{-1}=M_{1/g}$ при условии, что функция $1/g$ корректно определена. Оператор $M_g$ является топологическим изоморфизмом тогда и только тогда, когда $M_g$ и $M_g^{-1}$ суть ограниченные операторы. Из предыдущего пункта и равенства $M_g^{-1}=M_{1/g}$ мы видим, что это эквивалентно ограниченности $g$ и $1/g$. Значит, $c\leq|g|\leq C$ для некоторых $C,c>0$.
\end{proof}

\begin{proposition}\label{EquivMultOp} Пусть $(\Omega,\Sigma,\mu)$ --- $\sigma$-конечное атомическое пространство с мерой, $1\leq p,q\leq +\infty$ и $g\in L_0(\Omega,\Sigma)$. Тогда оператор $\widetilde{M}_{\widetilde{g}}:=\widetilde{I}_q M_g\widetilde{I}_p^{-1}\in\mathcal{B}(\ell_p(\Lambda),\ell_q(\Lambda))$ является оператором умножения на функцию $\widetilde{g}:\Lambda\to\mathbb{C}:\lambda\mapsto \mu(\Omega_\lambda)^{1/q-1/p-1}\int_{\Omega_\lambda}f(\omega)d\mu(\omega)$, где $\{\Omega_\lambda:\lambda\in\Lambda\}$ --- не более чем счетное разложение $\Omega$ на семейство непересекающихся атомов.
\end{proposition}
\begin{proof} Пусть $1\leq p,q\leq +\infty$. Для любого $x\in\ell_p(\Lambda)$ мы имеем
$$
\widetilde{M}_{\widetilde{g}}(x)(\lambda)
=(\widetilde{I}_q((M_g\widetilde{I}_p^{-1})(x))(\lambda)
=J_q(M_g(\widetilde{I}_p^{-1}(x))|_{\Omega_\lambda})(1)
$$
$$
=J_q((g \widetilde{I}_p^{-1}(x))|_{\Omega_\lambda})(1)
=\mu(\Omega_\lambda)^{1/q-1}\int_{\Omega_\lambda}(g|_{\Omega_\lambda} \widetilde{I}_p^{-1}(x)|_{\Omega_\lambda})(\omega)d\mu(\omega)
$$
$$
=\mu(\Omega_\lambda)^{1/q-1}\int_{\Omega_\lambda}(g \mu(\Omega)^{-1/p}x(\lambda)\chi_{\Omega_{\lambda}})(\omega)d\mu(\omega)
=x(\lambda)\mu(\Omega_\lambda)^{1/q-1/p-1}\int_{\Omega_\lambda} g(\omega)d\mu(\omega).
$$
Значит, $\widetilde{M}_{\widetilde{g}}$ --- оператор умножения, причем $\widetilde{g}(\lambda)=\mu(\Omega_\lambda)^{1/q-1/p-1}\int_{\Omega_\lambda} g(\omega)d\mu(\omega)$.
\end{proof}

Так как $\widetilde{I}_p$ и $\widetilde{I}_q$ --- изометрические изоморфизмы, то оператор $M_g$ топологически инъективен тогда и только тогда, когда оператор $\widetilde{M}_{\widetilde{g}}$ топологически инъективен. 

\begin{proposition}\label{TopInjMultOpCharacOnPureAtomMeasSp} Пусть $(\Omega,\Sigma,\mu)$ --- $\sigma$-конечное атомическое пространство с мерой, $1\leq p,q\leq +\infty$ и $g\in L_0(\Omega,\Sigma)$. Тогда следующие условия эквивалентны:

$i)$ $M_g\in\mathcal{B}(L_p(\Omega,\mu),L_q(\Omega,\mu))$ --- топологически инъективный оператор;

$ii)$ $|g|\geq c$ для некоторого $c>0$, при этом если $p\neq q$, то пространство $(\Omega,\Sigma,\mu)$ состоит из конечного числа атомов.
\end{proposition}
\begin{proof}
$i)$$\implies$$ ii)$ По предположению, оператор $M_g$ топологически инъективен, тогда таков же и $\widetilde{M}_{\widetilde{g}}$, то есть $\Vert\widetilde{M}_{\widetilde{g}}(x)\Vert_{\ell_q(\Lambda)}\geq c'\Vert x\Vert_{\ell_p(\Lambda)}$ для некоторого $c'>0$ и всех $x\in\ell_p(\Lambda)$. Через $\{\Omega_\lambda:\lambda\in\Lambda\}$ мы обозначим не более, чем счетное разбиение $\Omega$ на семейство непересекающихся атомов. Мы рассмотрим два случая.

Пусть $p\neq q$. Допустим, что множество $\Lambda$ счетно. Если $p,q<+\infty$, то мы приходим к противоречию так как по теореме Питта [\cite{KalAlbTopicsBanSpTh}, предложение 2.1.6] не существует вложения $\ell_p(\Lambda)$ в $\ell_q(\Lambda)$ для счетного $\Lambda$ и $1\leq p,q< +\infty$, $p\neq q$. 

Если $1\leq p<+\infty$ и $q=+\infty$, то рассмотрим произвольное конечное множество $F\in\mathcal{P}_0(\Lambda)$. Тогда 
$$
\sup_{\lambda\in\Lambda}|\widetilde{g}(\lambda)|
\geq\max_{\lambda\in F}|\widetilde{g}(\lambda)|
=\left\Vert\widetilde{M}_{\widetilde{g}}\left(\sum_{\lambda\in F}\delta_\lambda\right)\right\Vert_{\ell_\infty(\Lambda)}
\geq c'\left\Vert\sum_{\lambda\in F}\delta_\lambda\right\Vert_{\ell_p(\Lambda)}
=c'\operatorname{Card}(F)^{1/p}.
$$
Так как множество $\Lambda$ счетно, то $\sup_{\lambda\in\Lambda}|\widetilde{g}(\lambda)|\geq c'\sup_{F\in\mathcal{P}_0(\Lambda)}\operatorname{Card}(F)^{1/p}=+\infty$. С другой стороны, поскольку $\widetilde{M}_{\widetilde{g}}$ --- ограниченный оператор, мы получаем, что 
$$
\sup_{\lambda\in\Lambda}|\widetilde{g}(\lambda)|
=\sup_{\lambda\in\Lambda}\Vert\widetilde{M}_{\widetilde{g}}(\delta_\lambda)\Vert_{\ell_\infty(\Lambda)}
\leq\Vert\widetilde{M}_{\widetilde{g}}\Vert\Vert \delta_\lambda\Vert_{\ell_p(\Lambda)}
=\Vert\widetilde{M}_{\widetilde{g}}\Vert<+\infty.
$$
Противоречие.

Если $1\leq q<+\infty$ и $p=+\infty$ то мы снова приходим к противоречию. Действительно, так как множество $\Lambda$ счетно, то пространство $\ell_\infty(\Lambda)$ несепарабельно, а $\ell_q(\Lambda)$ сепарабельно. Поскольку оператор $\widetilde{M}_{\widetilde{g}}$ топологически инъективен, то $\operatorname{Im}(\widetilde{M}_{\widetilde{g}})$ --- несепарабельное подпространство в $\ell_q(\Lambda)$. Противоречие.

Во всех случаях мы пришли к противоречию, значит, множество $\Lambda$ конечно и пространство $(\Omega,\Sigma,\mu)$ состоит из конечного числа атомов. Очевидно, функция
$g$ однозначно определяется своими значениями $k_\lambda\in\mathbb{C}$ на атомах $\{\Omega_\lambda:\lambda\in\Lambda\}$. По предложению \ref{MultpOpSurjInjDesc} функция $g$ равна нулю только на множествах меры $0$, поэтому $k_\lambda\neq 0$ для всех $\lambda\in\Lambda$. Поскольку множество $\Lambda$ конечно, то $|g|\geq c$ для $c=\min_{\lambda\in\Lambda}|k_\lambda|>0$. 

Пусть $p=q$. Для любого $\lambda\in\Lambda$ мы имеем
$$
|\widetilde{g}(\lambda)|
=\Vert \widetilde{g} \delta_\lambda\Vert_{\ell_q(\Lambda)}
=\Vert \widetilde{M}_{\widetilde{g}}(\delta_\lambda)\Vert_{\ell_q(\Lambda)}
\geq c'\Vert \delta_\lambda\Vert_{\ell_p(\Lambda)}
=c'.
$$
Поэтому для всех $\omega\in\Omega_\lambda$ выполнено
$$
|g(\omega)|
=\left|\mu(\Omega_\lambda)^{-1}\int_{\Omega_\lambda}g(\omega)d\mu(\omega)\right|
=\left|\mu(\Omega_\lambda)^{-1}\mu(\Omega_\lambda)^{1+1/p-1/p}\widetilde{g}(\lambda)\right|
=|\widetilde{g}(\lambda)|\geq c'.
$$
Так как $\Omega=\bigcup_{\lambda\in\Lambda}\Omega_\lambda$, то в итоге мы получаем, что $|g|\geq c'$.

$ii)$$\implies$$ i)$ Допустим $|g|\geq c$ для $c>0$. Тогда из предложения \ref{EquivMultOp} следует, что $|\widetilde{g}|\geq c$. Если $p\neq q$, то мы дополнительно предполагаем, что пространство $(\Omega,\Sigma,\mu)$ состоит из конечного числа атомов. В этом случае пространство $L_p(\Omega,\mu)$ конечномерно. По предположению функция $g$ не принимает нулевых значений, поэтому оператор $M_g$ топологически инъективен. Если же $p=q$, то для всех $x\in\ell_p(\Lambda)$ мы имеем
$$
\Vert \widetilde{M}_{\widetilde{g}}(x)\Vert_{\ell_p(\Lambda)}=\Vert g x\Vert_{\ell_p(\Lambda)}\geq c\Vert x\Vert_{\ell_p(\Lambda)},
$$
поэтому оператор $\widetilde{M}_{\widetilde{g}}$ топологически инъективен, а вместе с ним и $M_g$.
\end{proof}

\begin{proposition}\label{TopInjMultOpCharacOnNonAtomMeasSp} Пусть $(\Omega,\Sigma,\mu)$ --- неатомическое пространство с мерой, $1\leq p,q\leq +\infty$ и $g\in L_0(\Omega,\Sigma)$. Тогда следующие условия эквивалентны:

$i)$ $M_g\in\mathcal{B}(L_p(\Omega,\mu),L_q(\Omega,\mu))$ --- топологически инъективный оператор;

$ii)$ $|g|\geq c$ для некоторого $c>0$, и $p=q$.
\end{proposition}
\begin{proof}
$i)$$\implies$$ ii)$ Согласно условию $\Vert M_g(f)\Vert_{L_q(\Omega,\mu)}\geq c\Vert f\Vert_{L_p(\Omega,\mu)}$ для некоторого $c>0$ и всех функций $f\in L_p(\Omega,\mu)$. Мы рассмотрим три случая.

Пусть $p>q$. Существуют константа $C>0$ и множество $E\in\Sigma$ положительной меры такие, что $|g|_E|\leq C$, иначе $M_g$ не определен корректно. Так как $(\Omega,\Sigma,\mu)$ --- неатомическое пространство с мерой, то существует последовательность $\{E_n:n\in\mathbb{N}\}\subset\Sigma$ подмножеств $E$ такая, что $\mu(E_n)=2^{-n}$. Поскольку $p>q$, мы получаем
$$
c
\leq\frac{\Vert M_g(\chi_{E_n})\Vert_{L_q(\Omega,\mu)}}{\Vert \chi_{E_n}\Vert_{L_p(\Omega,\mu)}}
\leq\frac{C\Vert\chi_{E_n}\Vert_{L_q(\Omega,\mu)}}{\Vert \chi_{E_n}\Vert_{L_p(\Omega,\mu)}}
\leq C\mu(E_n)^{1/q-1/p},
$$
$$
c
\leq\inf_{n\in\mathbb{N}}C\mu(E_n)^{1/q-1/p}
=C\inf_{n\in\mathbb{N}} 2^{n(1/p-1/q)}=0.
$$
Противоречие.

Пусть $p=q$. Зафиксируем число $c'<c$. Допустим, найдется множество $E\in\Sigma$ положительной меры такое, что $|g|_{E}|<c'$, тогда
$$
\Vert M_g(\chi_{E})\Vert_{L_p(\Omega,\mu)}
=\Vert g \chi_{E}\Vert_{L_p(\Omega,\mu)}
\leq c' \Vert \chi_{E}\Vert_{L_p(\Omega,\mu)}
<c\Vert \chi_{E}\Vert_{L_p(\Omega,\mu)}.
$$
Противоречие. Так как $c'<c$ произвольно, то мы заключаем, что $|g|_E|\geq c$ для всех множеств $E\in\Sigma$ положительной меры. Следовательно, $|g|\geq c$.

Теперь пусть $p<q$. Допустим, нашлись число $c'>0$ и множество $E\in\Sigma$ положительной меры такие, что $|g|_E|>c'$. Снова рассмотрим последовательность $\{E_n:n\in\mathbb{N}\}\subset\Sigma$ подмножеств в $E$ такую, что $\mu(E_n)=2^{-n}$. Из неравенства $p<q$ следует, что
$$
\Vert M_g\Vert
\geq\frac{\Vert M_g(\chi_{E_n})\Vert_{L_q(\Omega,\mu)}}{\Vert \chi_{E_n}\Vert_{L_p(\Omega,\mu)}}
\geq\frac{c'\Vert\chi_{E_n}\Vert_{L_q(\Omega,\mu)}}{\Vert \chi_{E_n}\Vert_{L_p(\Omega,\mu)}}
\geq c'\mu(E_n)^{1/q-1/p}
$$
$$
\Vert M_g\Vert
\geq\sup_{n\in\mathbb{N}}c'\mu(E_n)^{1/q-1/p}
\geq c'\sup_{n\in\mathbb{N}}2^{n(1/p-1/q)}
=+\infty.
$$
Противоречие, значит, $g=0$. В этом случае по предложению \ref{MultpOpSurjInjDesc} оператор $M_g$ не топологически инъективен.

$ii)$$\implies$$ i)$ Обратно, допустим $|g|\geq c>0$ и пусть $p=q$. Тогда для любой функции $f\in L_p(\Omega,\mu)$ выполнено
$$
\Vert M_g(f)\Vert_{L_p(\Omega,\mu)}
=\Vert g f\Vert_{L_p(\Omega,\mu)}
\geq c\Vert f\Vert_{L_p(\Omega,\mu)}.
$$
Значит, оператор $M_g$ топологически инъективен.
\end{proof}

\begin{proposition}\label{TopInjMultOpCharacOnMeasSp} Пусть $(\Omega,\Sigma,\mu)$ --- $\sigma$-конечное пространство с мерой, $1\leq p,q\leq +\infty$ и $g\in L_0(\Omega,\Sigma)$. Тогда следующие условия эквивалентны:

$i)$ $M_g\in\mathcal{B}(L_p(\Omega,\mu),L_q(\Omega,\mu))$ --- топологически инъективный оператор;

$ii)$ $M_g$ --- топологический изоморфизм;

$iii)$ $|g|\geq c$ для некоторого $c>0$, при этом если $p\neq q$ то пространство $(\Omega,\Sigma,\mu)$ состоит из конечного числа атомов.
\end{proposition}
\begin{proof} $i)\Longleftrightarrow iii)$ Рассмотрим разложение
$\Omega=\Omega_a^{\mu}\cup\Omega_{na}^{\mu}$, где $(\Omega_{na}^{\mu},\Sigma|_{\Omega_{na}^{\mu}},\mu|_{\Omega_{na}^{\mu}})$ --- неатомическое пространство с мерой и $(\Omega_a^{\mu},\Sigma|_{\Omega_a^{\mu}},\mu|_{\Omega_a^{\mu}})$ --- атомическое пространство с мерой. По предложению \ref{MultOpDecompDecomp} оператор $M_g$ топологически инъективен тогда и только тогда, когда таковы $M_g^{\Omega_a^{\mu}}$ и $M_g^{\Omega_{na}^{\mu}}$. Предложения \ref{TopInjMultOpCharacOnPureAtomMeasSp}, \ref{TopInjMultOpCharacOnNonAtomMeasSp} дают для этого необходимые и достаточные условия.

$i)$$\implies$$ ii)$ Допустим, оператор $M_g$ топологически инъективен. Если $p=q$, то из рассуждений выше следует, что $|g|\geq c$ для некоторого $c>0$. Поскольку оператор $M_g$ ограничен, то из предложения \ref{MultpOpPropIfPeqqualsQ} мы также получаем, что $C\geq |g|$ для некоторого $C>0$. Теперь из того же самого предложения мы заключаем, что $M_g$ есть топологический изоморфизм. Если $p\neq q$, то из предыдущего пункта известно, что пространство $(\Omega,\Sigma,\mu)$ состоит из конечного числа атомов и функция $g$ не равна нулю. Следовательно, $\operatorname{dim}(L_p(\Omega,\Sigma,\mu))=\operatorname{dim}(\ell_p(\Lambda))=\operatorname{Card}(\Lambda)<+\infty$. Аналогично, $\operatorname{dim}(L_q(\Omega,\Sigma,\mu))=\operatorname{Card}(\Lambda)<+\infty$. Так как функция $g$ не равна нулю, то по предложению \ref{MultpOpSurjInjDesc} оператор $M_g$ инъективен. Итак, $M_g$ --- инъективный оператор между конечномерными пространствами равной размерности. Следовательно, он является  топологическим изоморфизмом.

$ii)$$\implies$$ i)$ Обратно, если $M_g$ --- топологический изоморфизм, то, очевидно, он топологически инъективен.
\end{proof}

\begin{proposition}\label{TopInjMultOpCharacBtwnTwoContMeasSp} Пусть $(\Omega,\Sigma,\mu)$ --- $\sigma$-конечное пространство с мерой, $1\leq p,q\leq+\infty$. Допустим, $g,\rho\in L_0(\Omega,\Sigma)$ и функция $\rho$ неотрицательна. Тогда следующие условия эквивалентны:

$i)$ $M_g\in\mathcal{B}(L_p(\Omega,\mu),L_q(\Omega,\rho\mu))$ --- топологически инъективный оператор;

$ii)$ $M_g$ --- топологический изоморфизм;

$iii)$ функция $\rho$ положительна, $|g \rho^{1/q}|\geq c$ для некоторого $c>0$, при этом если $p\neq q$, то пространство $(\Omega,\Sigma,\mu)$ состоит из конечного числа атомов.
\end{proposition}
\begin{proof} $i)$$\implies$$ iii)$ Рассмотрим множество $E=\rho^{-1}(\{0\})$. Предположим, что $\mu(E)>0$, тогда $\chi_E\neq 0$ в $L_p(\Omega,\mu)$. С другой стороны, $(\rho\mu)(E)=\int_E\rho(\omega)d\mu(\omega)=0$, поэтому $\chi_E=0$ в $L_q(\Omega,\rho\mu)$ и $M_g(\chi_E)=g\chi_E=0$ в $L_q(\Omega,\rho \mu)$. Таким образом, мы получили, что оператор $M_g$ не инъективен и, как следствие, не топологически инъективен. Противоречие, поэтому $\mu(E)=0$ и функция $\rho$ положительна. Следовательно, корректно определен изометрический изоморфизм $\bar{I}_q:L_q(\Omega,\mu)\to L_q(\Omega,\rho\mu):f\mapsto \rho^{-1/q} f$. Очевидно, $M_{g\rho^{1/q}}=\bar{I}_q^{-1} M_g\in\mathcal{B}(L_p(\Omega,\mu),L_q(\Omega,\mu))$. Поскольку $\bar{I}_q$ --- изометрический изоморфизм и оператор $M_g$ топологически инъективен, то $M_{g \rho^{1/q}}$ также топологически инъективен. Из предложения \ref{TopInjMultOpCharacOnMeasSp} мы получаем, что $|g\rho^{1/q}|\geq c$ для некоторого $c>0$ и если $p\neq q$, то пространство $(\Omega,\Sigma,\mu)$ состоит из конечного числа атомов.

$iii)$$\implies$$ i)$ Из предложения \ref{TopInjMultOpCharacOnMeasSp} следует, что оператор $M_{g \rho^{1/q}}$ топологически инъективен. Так как функция $\rho$ положительна, то мы имеем изометрический изоморфизм $\bar{I}_q$. Теперь из равенства $M_g=\bar{I}_q M_{g \rho^{1/q}}$ следует, что оператор $M_g$ также топологически инъективен.

$i)$$\implies$$ ii)$ Как мы доказали ранее оператор $M_{g \rho^{1/q}}$ топологически инъективен и $\bar{I}_q$ --- изометрический изоморфизм. По предложению \ref{TopInjMultOpCharacOnMeasSp} оператор $M_{g \rho^{1/q}}$ является топологическим изоморфизмом. Так как $M_g=\bar{I}_q M_{g \rho^{1/q}}$ и $\bar{I}_q$ есть изометрический изоморфизм, то $M_g$  является топологическим изоморфизмом.

$ii)$$\implies$$ i)$ Если $M_g$ --- топологический изоморфизм, тогда, очевидно, он топологически инъективен.
\end{proof}
\begin{proposition}\label{TopInjMultOpCharacBtwnTwoMeasSp} Пусть $(\Omega,\Sigma,\mu)$, $(\Omega,\Sigma,\nu)$ --- два $\sigma$-конечных пространства с мерой, $1\leq p,q\leq +\infty$ и $g\in L_0(\Omega,\Sigma)$. Тогда следующие условия эквивалентны:

$i)$ $M_g\in\mathcal{B}(L_p(\Omega,\mu), L_q(\Omega,\nu))$ --- топологически инъективный оператор;

$ii)$ $M_g^{\Omega_c^{\nu,\mu}}$ --- топологический изоморфизм;

$iii)$ функция $\rho_{\nu,\mu}|_{\Omega_c^{\nu,\mu}}$ положительна, $|g \rho_{\nu,\mu}^{1/q}|_{\Omega_c^{\nu,\mu}}|\geq c$ для некоторого $c>0$, при этом если $p\neq q$, то пространство $(\Omega,\Sigma,\mu)$ состоит из конечного числа атомов.
\end{proposition}
\begin{proof}
По предложению \ref{MultOpDecompDecomp} оператор $M_g$ топологически инъективен тогда и только тогда, когда операторы $M_g^{\Omega_c^{\nu,\mu}}:L_p(\Omega_c^{\nu,\mu},\mu|_{\Omega_c^{\nu,\mu}})\to L_q(\Omega_c^{\nu,\mu},\rho_{\nu,\mu} \mu|_{\Omega_c^{\nu,\mu}})$ и $M_g^{\Omega_s^{\nu,\mu}}:L_p(\Omega_s^{\nu,\mu},\mu|_{\Omega_s^{\nu,\mu}})\to L_q(\Omega_s^{\nu,\mu},\nu_s|_{\Omega_s^{\nu,\mu}})$  топологически инъективны. По предложению \ref{MultOpCharacBtwnTwoSingMeasSp} оператор $M_g^{\Omega_s^{\nu,\mu}}$ нулевой. Так как $\mu(\Omega_s^{\nu,\mu})=0$, то $L_p(\Omega_s^{\nu,\mu},\mu|_{\Omega_s^{\nu,\mu}})=\{0\}$. Отсюда мы заключаем, что оператор $M_g^{\Omega_s^{\nu,\mu}}$ топологически инъективен. Значит, топологическая инъективность $M_g$ эквивалентна топологической инъективности  $M_g^{\Omega_c^{\nu,\mu}}$. Остается применить предложение \ref{TopInjMultOpCharacBtwnTwoContMeasSp}.
\end{proof}

\begin{proposition}\label{TopInjMultOpDescBtwnTwoMeasSp} Пусть $(\Omega,\Sigma,\mu)$, $(\Omega,\Sigma,\nu)$ --- два $\sigma$-конечных пространства с мерой, $1\leq p,q\leq +\infty$ и $g\in L_0(\Omega,\Sigma)$. Тогда следующие условия эквивалентны:

$i)$ $M_g\in\mathcal{B}(L_p(\Omega,\mu),L_q(\Omega,\nu))$ --- топологически инъективный оператор;

$ii)$ $M_{\chi_{\Omega_c^{\nu,\mu}}/g}\in\mathcal{B}(L_q(\Omega,\nu), L_p(\Omega,\mu))$ --- топологически сюръективный левый обратный оператор к $M_g$.
\end{proposition}
\begin{proof}
$i)$$\implies$$ ii)$ По предложению \ref{MultOpDecompDecomp} оператор  $M_g^{\Omega_c^{\nu,\mu}}$ топологически инъективен. По предложению \ref{TopInjMultOpCharacBtwnTwoContMeasSp} оператор $M_g^{\Omega_c^{\nu,\mu}}$ обратим и $(M_g^{\Omega_c^{\nu,\mu}})^{-1}=M_{1/g}^{\Omega_c^{\nu,\mu}}$. Тогда для любой функции $h\in L_q(\Omega,\nu)$ выполнено
$$
\Vert M_{\chi_{\Omega_c^{\nu,\mu}}/g}(h)\Vert_{L_p(\Omega,\mu)}=
\Vert M_{1/g}(h)\chi_{\Omega_c^{\nu,\mu}}\Vert_{L_p(\Omega,\mu)}=
\Vert M_{1/g}^{\Omega_c^{\nu,\mu}}(h|_{\Omega_c^{\nu,\mu}})\Vert_{L_p(\Omega_c^{\nu,\mu},\mu|_{\Omega_c^{\nu,\mu}})}
$$
$$
\leq\Vert M_{1/g}^{\Omega_c^{\nu,\mu}}\Vert\Vert h|_{\Omega_c^{\nu,\mu}}\Vert_{L_q(\Omega_c^{\nu,\mu},\nu|_{\Omega_c^{\nu,\mu}})}
\leq\Vert M_{1/g}^{\Omega_c^{\nu,\mu}}\Vert\Vert h\Vert_{L_q(\Omega,\nu)}.
$$ 
Поэтому оператор $M_{\chi_{\Omega_c^{\nu,\mu}}/g}$ ограничен. Теперь заметим, что для любой функции $f\in L_p(\Omega,\mu)$ выполнено
$$
M_{\chi_{\Omega_c^{\nu,\mu}}/g}(M_g(f))
=M_{\chi_{\Omega_c^{\nu,\mu}}/g}(g  f)
=(\chi_{\Omega_c^{\nu,\mu}}/g)  g  f
=f \chi_{\Omega_c^{\nu,\mu}}.
$$
Так как $\mu(\Omega\setminus\Omega_c^{\nu,\mu})=0$, то $\chi_{\Omega_c^{\nu,\mu}}=\chi_{\Omega}$, поэтому $M_{\chi_{\Omega_c^{\nu,\mu}}/g}(M_g(f))=f \chi_{\Omega_c^{\nu,\mu}}=f \chi_{\Omega}=f$. Это означает, что $M_{\chi_{\Omega_c^{\nu,\mu}}/g}$ --- левый обратный оператор умножения к $M_g$. Рассмотрим произвольную функцию $f\in L_p(\Omega,\mu)$, тогда для $h=M_g(f)$ выполнено $M_{\chi_{\Omega_c^{\nu,\mu}}/g}(h)=f$ и $\Vert h\Vert_{L_q(\Omega,\nu)}\leq\Vert M_g\Vert\Vert f\Vert_{L_p(\Omega,\mu)}$. Так как функция $f$ произвольна, то оператор $M_{\chi_{\Omega_c^{\nu,\mu}}/g}$ топологически сюръективен.

Обратно, если $M_g$ имеет левый обратный оператор $M_{\chi_{\Omega_c^{\nu,\mu}}/g}$, то для всех $f\in L_p(\Omega,\mu)$ выполнено
$$
\Vert M_g(f)\Vert_{L_q(\Omega,\nu)}
\geq\Vert M_{\chi_{\Omega_c^{\nu,\mu}}/g}\Vert^{-1}\Vert M_{\chi_{\Omega_c^{\nu,\mu}}/g}(M_g(f))\vert_{L_p(\Omega,\mu)}
\geq\Vert M_{\chi_{\Omega_c^{\nu,\mu}}/g}\Vert^{-1}\Vert f\Vert_{L_p(\Omega,\mu)}.
$$
Следовательно, оператор $M_g$ топологически инъективен.
\end{proof}

\begin{proposition}\label{IsomMultOpCharacOnMeasSp} Пусть $(\Omega,\Sigma,\mu)$ --- $\sigma$-конечное пространство с мерой, $1\leq p,q\leq +\infty$ и $g\in L_0(\Omega,\Sigma)$. Тогда следующие условия эквивалентны:

$i)$ $M_g\in\mathcal{B}(L_p(\Omega,\mu),L_q(\Omega,\mu))$ --- изометрический оператор;

$ii)$ $|g|=\mu(\Omega)^{1/p-1/q}$, при этом если $p\neq q$, то пространство $(\Omega,\Sigma,\mu)$ состоит из одного атома.

\end{proposition}
\begin{proof} $i)$$\implies$$ ii)$ Пусть $p=q$. Допустим существует множество $E\in\Sigma$ положительной меры такое, что $|g|_E|<1$, тогда
$$
\Vert M_g(\chi_E)\Vert_{L_p(\Omega,\mu)}
=\Vert g \chi_E\Vert_{L_p(\Omega,\mu)}
<\Vert\chi_E\Vert_{L_p(\Omega,\mu)}
=\Vert M_g(\chi_E)\Vert_{L_p(\Omega,\mu)}.
$$
Противоречие, значит для любого множества  $E\in\Sigma$ положительной меры выполнено $|g|_E|\geq 1$, то есть $|g|\geq 1$. Аналогично доказывается, что $|g|\leq 1$. Следовательно, $|g|=1=\mu(\Omega)^{1/p-1/q}$. Пусть $p\neq q$, тогда так как оператор $M_g$ --- изометрия, то он топологически инъективен. По предложению \ref{TopInjMultOpCharacOnMeasSp} пространство $(\Omega,\Sigma,\mu)$ состоит из конечного числа атомов. Допустим, что имеется хотя бы два атома, назовем их $\Omega_1$ и $\Omega_2$. Они оба конечной меры, поэтому мы можем рассмотреть функции $h_k=\Vert\chi_{\Omega_k}\Vert_{L_p(\Omega,\mu)}^{-1}\chi_{\Omega_k}$ для $k\in\mathbb{N}_2$. Так как эти атомы не пересекаются, то $h_1h_2=0$ и как результат $M_g(h_1)M_g(h_2)=0$. Заметим, что для любого $1\leq r\leq +\infty$ и любых функций $f_1,f_2\in L_r(\Omega,\mu)$ со свойством $f_1f_2=0$ верно
$$
\Vert f_1+f_2\Vert_{L_r(\Omega,\mu)}
=\left\Vert\left(\Vert f_\lambda\Vert_{L_r(\Omega,\mu)}:\lambda\in\mathbb{N}_2\right)\right\Vert_{\ell_r(\mathbb{N}_2)}.
$$
Значит,
$$
\Vert M_g(h_1+h_2)\Vert_{L_q(\Omega,\mu)}
=\Vert h_1+h_2\Vert_{L_p(\Omega,\mu)}
=\left\Vert\left( 1 :\lambda\in\mathbb{N}_2\right)\right\Vert_{\ell_p(\mathbb{N}_2)}
=2^{1/p}.
$$
С другой стороны
$$
\Vert M_g(h_1+h_2)\Vert_{L_q(\Omega,\mu)}
=\Vert M_g(h_1)+M_g(h_2)\Vert_{L_q(\Omega,\mu)}
$$
$$
=\left\Vert\left(\Vert M_g(h_\lambda)\Vert_{L_q(\Omega,\mu)}:\lambda\in\mathbb{N}_2\right)\right\Vert_{\ell_q(\mathbb{N}_2)}
=\left\Vert\left(\Vert h_\lambda\Vert_{L_p(\Omega,\mu)}:\lambda\in\mathbb{N}_2\right)\right\Vert_{\ell_q(\mathbb{N}_2)}
=2^{1/q}.
$$
Поэтому $2^{1/p}=2^{1/q}$. Противоречие, значит, $(\Omega,\Sigma,\mu)$ состоит из одного атома. В этом случае для любой функции $f\in L_p(\Omega,\mu)$ выполнено
$$
\Vert M_g(f)\Vert_{L_q(\Omega,\mu)}
=\Vert J_q(M_g(f))\Vert_{\ell_q(\mathbb{N}_1)}
=\Vert J_q(g  f)\Vert_{\ell_q(\mathbb{N}_1)}
=\mu(\Omega)^{1/q-1}\left|\int_\Omega g(\omega) f(\omega)d\mu(\omega)\right|
$$
$$
\Vert f\Vert_{L_p(\Omega,\mu)}
=\Vert J_p(f)\Vert_{\ell_p(\mathbb{N}_1)}
=\mu(\Omega)^{1/p-1}\left|\int_\Omega f(\omega)d\mu(\omega)\right|.
$$
Через $c$ мы обозначим константное значение функции $g$, тогда
$$
\Vert M_g(f)\Vert_{L_q(\Omega,\mu)}
=\mu(\Omega)^{1/q-1}\left|\int_\Omega g(\omega) f(\omega)d\mu(\omega)\right|
=\mu(\Omega)^{1/q-1}|c|\left|\int_\Omega f(\omega)d\mu(\omega)\right|.
$$
Из этого равенства следует, что оператор $M_g$ будет изометрией, если $|g|=|c|=\mu(\Omega)^{1/p-1/q}$.

$ii)$$\implies$$ i)$. Пусть $p=q$, тогда $|g|=1$. Поэтому для произвольной функции $f\in L_p(\Omega,\mu)$ выполнено
$$
\Vert M_g(f)\Vert_{L_p(\Omega,\mu)}
=\Vert g  f\Vert_{L_p(\Omega,\mu)}
=\Vert |g|  f\Vert_{L_p(\Omega,\mu)}
=\Vert f\Vert_{L_p(\Omega,\mu)},
$$
значит, оператор $M_g$ --- изометрия. Пусть $p\neq q$, тогда по предположению $(\Omega,\Sigma,\mu)$ состоит из одного атома, и тогда
$$
\Vert M_g(f)\Vert_{L_q(\Omega,\mu)}
=\mu(\Omega)^{1/q-1}\left|\int_\Omega g(\omega) f(\omega)d\mu(\omega)\right|
=\mu(\Omega)^{1/q-1}|c|\left|\int_\Omega f(\omega)d\mu(\omega)\right|
$$
$$
=\mu(\Omega)^{1/p-1}\left|\int_\Omega f(\omega)d\mu(\omega)\right|
=\Vert f\Vert_{L_p(\Omega,\mu)}.
$$
Следовательно, оператор $M_g$ изометричен.
\end{proof}

\begin{proposition}\label{IsomMultOpCharacBtwnTwoContMeasSp} Пусть $(\Omega,\Sigma,\mu)$ --- $\sigma$-конечное пространство с мерой и $1\leq p,q\leq +\infty$. Допустим, $g,\rho\in L_0(\Omega,\Sigma)$ и функция $\rho$ неотрицательна. Тогда следующие условия эквивалентны:

$i)$ оператор $M_g\in\mathcal{B}(L_p(\Omega,\mu), L_q(\Omega,\rho \mu))$ изометричен;

$ii)$ $M_g$ --- изометрический изоморфизм;

$iii)$ функция $\rho$ положительна, $|g  \rho^{1/q}|=\mu(\Omega)^{1/p-1/q}$, при этом если $p\neq q$, то пространство $(\Omega,\Sigma,\mu)$ состоит из одного атома.
\end{proposition}
\begin{proof} $i)$$\implies$$ iii)$ Так как оператор $M_g$ изометричен, то он топологически инъективен и из предложения \ref{TopInjMultOpCharacBtwnTwoMeasSp} следует, что функция $\rho$ положительна. Таким образом, мы имеем изометрический изоморфизм $\bar{I}_q:L_q(\Omega,\mu)\to L_q(\Omega,\rho \mu):f\mapsto \rho^{-1/q}  f$. Очевидно, $M_{g \rho^{1/q}}=\bar{I}_q^{-1} M_g\in\mathcal{B}(L_p(\Omega,\mu),L_q(\Omega,\mu))$. Так как $\bar{I}_q$ --- изометрический изоморфизм и оператор $M_g$ изометричен, то таков же и $M_{g  \rho^{1/q}}$. Осталось применить предложение \ref{IsomMultOpCharacOnMeasSp}.

$iii)$$\implies$$ i)$ По предложению \ref{IsomMultOpCharacOnMeasSp} оператор $M_{g \rho^{1/q}}$ изометричен. Так как функция $\rho$ положительна, то корректно определен изометрически изоморфизм $\bar{I}_q$. Тогда из равенства $M_g=\bar{I}_q M_{g \rho^{1/q}}$ следует, что оператор $M_g$ также изометричен.

$i)$$\implies$$ ii)$ Поскольку оператор $M_g$ изометричен, он топологически инъективен, и по предложению \ref{TopInjMultOpCharacBtwnTwoContMeasSp} он является изоморфизмом, причем, по предположению, изометрическим.

$ii)$$\implies$$ i)$ Очевидно.
\end{proof}

\begin{proposition}\label{IsomMultOpCharacBtwnTwoMeasSp} Пусть $(\Omega,\Sigma,\mu)$, $(\Omega,\Sigma,\nu)$ --- два $\sigma$-конечных пространства с мерой, $1\leq p,q\leq +\infty$ и $g\in L_0(\Omega,\Sigma)$. Тогда следующие условия эквивалентны:

$i)$ $M_g$ --- изометрический оператор;

$ii)$ $M_g^{\Omega_c^{\nu,\mu}}$ --- изометрический оператор;

$iii)$ функция $\rho_{\nu,\mu}|_{\Omega_c^{\nu,\mu}}$ положительна, $|g  \rho_{\nu,\mu}^{1/q}|_{\Omega_c^{\nu,\mu}}|=\mu(\Omega_c^{\nu,\mu})^{1/p-1/q}$, при этом если $p\neq q$ то пространство $(\Omega,\Sigma,\mu)$ состоит из одного атома.
\end{proposition}
\begin{proof} $i)$$\implies$$ ii)$$\implies$$ iii)$ Так как оператор $M_g$ изометричен, то по предложению \ref{MultOpDecompDecomp} оператор $M_g^{\Omega_c^{\nu,\mu}}$ также изометричен. Осталось применить предложение \ref{IsomMultOpCharacBtwnTwoContMeasSp}.

$iii)$$\implies$$ i)$ По предложению \ref{IsomMultOpCharacBtwnTwoContMeasSp} оператор $M_g^{\Omega_c^{\nu,\mu}}$ изометричен. Теперь рассмотрим произвольную функцию $f\in L_p(\Omega,\mu)$. Так как $\mu(\Omega\setminus\Omega_c^{\nu,\mu})=0$, то $\chi_{\Omega_c^{\nu,\mu}}=\chi_{\Omega}$ в $L_p(\Omega,\mu)$. Как следствие, $f=f\chi_{\Omega}=f\chi_{\Omega_c^{\nu,\mu}}=f\chi_{\Omega_c^{\nu,\mu}}\chi_{\Omega_c^{\nu,\mu}}$ в $L_p(\Omega,\mu)$ и $M_g(f)=M_g(f\chi_{\Omega_c^{\nu,\mu}})\chi_{\Omega_c^{\nu,\mu}}$. Учитывая, что оператор $M_g^{\Omega_c^{\nu,\mu}}$ изометричен, мы получаем
$$
\Vert M_g(f)\Vert_{L_q(\Omega,\nu)}
=\Vert M_g(f\chi_{\Omega_c^{\nu,\mu}})\chi_{\Omega_c^{\nu,\mu}}\Vert_{L_q(\Omega,\nu)}
=\Vert M_g(f\chi_{\Omega_c^{\nu,\mu}})\Vert_{L_q(\Omega_c^{\nu,\mu},\nu|_{\Omega_c^{\nu,\mu}})}
$$
$$
=\Vert M_g^{\Omega_c^{\nu,\mu}}(f|_{\Omega_c^{\nu,\mu}})\Vert_{L_q(\Omega_c^{\nu,\mu},\nu|_{\Omega_c^{\nu,\mu}})}
=\Vert f|_{\Omega_c^{\nu,\mu}}\Vert_{L_p(\Omega_c^{\nu,\mu},\mu|_{\Omega_c^{\nu,\mu}})}.
$$
Поскольку $\mu(\Omega\setminus\Omega_c^{\nu,\mu})=0$, то $\Vert f|_{\Omega_c^{\nu,\mu}}\Vert_{L_p(\Omega_c^{\nu,\mu},\mu|_{\Omega_c^{\nu,\mu}})}=\Vert f\Vert_{L_p(\Omega,\mu)}$, и поэтому $\Vert M_g(f)\Vert_{L_q(\Omega,\nu)}=\Vert f\Vert_{L_p(\Omega,\mu)}$. Значит, оператор $M_g$ изометричен.
\end{proof}

\begin{proposition}\label{IsomMultOpDescBtwnTwoMeasSp} Пусть $(\Omega,\Sigma,\mu)$, $(\Omega,\Sigma,\nu)$ --- два $\sigma$-конечных пространства с мерой, $1\leq p,q\leq +\infty$ и $g\in L_0(\Omega,\Sigma)$. Тогда следующие условия эквивалентны:

$i)$ $M_g\in\mathcal{B}(L_p(\Omega,\mu),L_q(\Omega,\nu))$ --- изометрический оператор;

$ii)$ $M_{\chi_{\Omega_c^{\nu,\mu}}/g}\in\mathcal{B}(L_q(\Omega,\nu), L_p(\Omega,\mu))$ --- строго коизометрический левый обратный оператор к $M_g$.
\end{proposition}
\begin{proof} $i)$$\implies$$ ii)$ По предложению \ref{MultOpDecompDecomp} оператор $M_g^{\Omega_c^{\nu,\mu}}$ изометричен, и по предложению \ref{IsomMultOpCharacBtwnTwoContMeasSp} он обратим, причем, очевидно, что $(M_g^{\Omega_c^{\nu,\mu}})^{-1}=M_{1/g}^{\Omega_c^{\nu,\mu}}$. Так как оператор $M_g^{\Omega_c^{\nu,\mu}}$ изометричен, то таков же и его левый обратный. Тогда для любой функции $h\in L_q(\Omega,\nu)$ выполнено
$$
\Vert M_{\chi_{\Omega_c^{\nu,\mu}}/g}(h)\Vert_{L_p(\Omega,\mu)}=
\Vert M_{1/g}(h|_{\Omega_c^{\nu,\mu}})\Vert_{L_p(\Omega_c^{\nu,\mu},\mu|_{\Omega_c^{\nu,\mu}})}=
\Vert M_{1/g}^{\Omega_c^{\nu,\mu}}(h|_{\Omega_c^{\nu,\mu}})\Vert_{L_p(\Omega_c^{\nu,\mu},\mu|_{\Omega_c^{\nu,\mu}})}
$$
$$
=\Vert h|_{\Omega_c^{\nu,\mu}}\Vert_{L_q(\Omega_c^{\nu,\mu},\nu|_{\Omega_c^{\nu,\mu}})}
\leq \Vert h \Vert_{L_q(\Omega,\nu)}.
$$
Поэтому оператор $M_{\chi_{\Omega_c^{\nu,\mu}}/g}$ сжимающий. Далее для всех функций $f\in L_p(\Omega,\mu)$ мы имеем
$$
M_{\chi_{\Omega_c^{\nu,\mu}}/g}(M_g(f))
=M_{\chi_{\Omega_c^{\nu,\mu}}/g}(g  f)
=(\chi_{\Omega_c^{\nu,\mu}}/g)  g  f
=f \chi_{\Omega_c^{\nu,\mu}}.
$$
Так как $\mu(\Omega\setminus\Omega_c^{\nu,\mu})=0$, то $\chi_{\Omega_c^{\nu,\mu}}=\chi_{\Omega}$, и поэтому $M_{\chi_{\Omega_c^{\nu,\mu}}/g}(M_g(f))=f \chi_{\Omega_c^{\nu,\mu}}=f \chi_{\Omega}=f$. Последнее означает, что оператор $M_{\chi_{\Omega_c^{\nu,\mu}}/g}$ является левым обратным оператором умножения к $M_g$. Рассмотрим произвольную функцию $f\in L_p(\Omega,\mu)$, тогда для $h=M_g(f)$ выполнено $M_{\chi_{\Omega_c^{\nu,\mu}}/g}(h)=f$ и $\Vert h\Vert_{L_q(\Omega,\nu)}\leq\Vert f\Vert_{L_p(\Omega,\mu)}$. Следовательно, $M_{\chi_{\Omega_c^{\nu,\mu}}}/g$ есть строго $1$-топологически сюръективный оператор, но он еще и сжимающий, а, значит, строго коизометрический.

$ii)$$\implies$$ i)$ Рассмотрим произвольную функцию $f\in L_p(\Omega,\mu)$, тогда существует функция $h\in L_q(\Omega,\nu)$ такая, что $M_{\chi_{\Omega_c^{\nu,\mu}}/g}(h)=f$ и $\Vert h\Vert_{L_q(\Omega,\nu)}\leq \Vert f\Vert_{L_p(\Omega,\mu)}$. Следовательно,
$$
\Vert M_g(f)\Vert_{L_q(\Omega,\nu)}
=\Vert M_g(M_{\chi_{\Omega_c^{\nu,\mu}}/g}(h))\Vert_{L_q(\Omega,\nu)}
=\Vert \chi_{\Omega_c^{\nu,\mu}}h\Vert_{L_q(\Omega,\nu)}
\leq\Vert h\Vert_{L_q(\Omega,\nu|)}
\leq\Vert f\Vert_{L_p(\Omega,\mu)}.
$$
Поскольку оператор $M_{\chi_{\Omega_c^{\nu,\mu}}/g}$ сжимающий и левый обратный к $M_g$, то
$$
\Vert f\Vert_{L_p(\Omega,\mu)}
=\Vert M_{\chi_{\Omega_c^{\nu,\mu}}/g}(M_g(f))\Vert_{L_p(\Omega,\mu)}
\leq\Vert M_g(f)\Vert_{L_q(\Omega,\nu)}
$$
и, значит, $\Vert M_g(f)\Vert_{L_q(\Omega,\nu)}=\Vert f\Vert_{L_p(\Omega,\mu)}$. Так как функция $f$ произвольна, то оператор $M_g$ изометричен.
\end{proof}

Теперь мы обсудим топологически сюръективные и коизометрические операторы умножения. Их описание получить проще и большинство доказательств схожи (но не идентичны) с рассуждениями для топологически инъективных и изометрических операторов.

\begin{proposition}\label{TopSurMultOpCharacOnMeasSp} Пусть $(\Omega,\Sigma,\mu)$ --- $\sigma$-конечное пространство с мерой, $1\leq p,q\leq +\infty$ и $g\in L_0(\Omega,\Sigma)$. Тогда следующие условия эквивалентны:

$i)$ $M_g\in\mathcal{B}(L_p(\Omega,\mu),L_q(\Omega,\mu))$ --- топологически сюръективный оператор;

$ii)$ $M_g$ --- топологический изоморфизм;

$iii)$ $|g|\geq c$ для некоторого $c>0$, при этом если $p\neq q$, то пространство $(\Omega,\Sigma,\mu)$ состоит из конечного числа атомов.
\end{proposition}
\begin{proof} $i)$$\implies$$ ii)$ Так как оператор $M_g$ топологически сюръективен, то он сюръективен и по предложению \ref{MultpOpSurjInjDesc} он инъективен. Следовательно, оператор $M_g$ биективен. Так как $L_p$ пространства полны, то из теоремы об открытом отображении мы получаем, что $M_g$ --- топологический изоморфизм. 

$ii)$$\implies$$ i)$ Если $M_g$ --- топологический изоморфизм, то он, очевидно, топологически сюръективен.

$ii)\Longleftrightarrow iii)$ Эквивалентность следует из предложения \ref{TopInjMultOpCharacOnMeasSp}.
\end{proof}
 
\begin{proposition}\label{TopSurMultOpCharacBtwnTwoContMeasSp} Пусть $(\Omega,\Sigma,\nu)$ --- $\sigma$-конечное пространство с мерой, $1\leq p,q\leq +\infty$ и $g,\rho\in L_0(\Omega,\Sigma)$ причем функция $\rho$  неотрицательна. Тогда следующие условия эквивалентны:

$i)$ $M_g\in\mathcal{B}(L_p(\Omega,\rho \nu),L_q(\Omega,\nu))$ --- топологически сюръективный оператор;

$ii)$ $M_g$ --- топологический изоморфизм;

$iii)$ функция $\rho$ положительна, $|g  \rho^{-1/p}|\geq c$ для некоторого $c>0$, при этом если $p\neq q$, то пространство $(\Omega,\Sigma,\mu)$ состоит из конечного числа атомов.
\end{proposition}
\begin{proof} $i)$$\implies$$ iii)$ Рассмотрим множество $E=\rho^{-1}(\{0\})$. Допустим, $\nu(E)>0$, тогда $\chi_E\neq 0$ в $L_q(\Omega,\nu)$. С другой стороны, $(\rho \nu)(E)=\int_E\rho(\omega)d\nu(\omega)=0$, поэтому $\chi_E=0$ в $L_p(\Omega,\rho \nu)$. Тогда для всех функций $f\in L_p(\Omega,\rho \nu)$ выполнено $M_g(f)\chi_E=M_g(f \chi_E)=M_g(0)=0$ в $L_q(\Omega,\nu)$. Последнее равенство означает, что $\operatorname{Im}(M_g)\subset\{h\in L_q(\Omega,\nu): h|_E=0\}$. Поскольку $\nu(E)\neq 0$, оператор $M_g$ не сюръективен и, как следствие, не является топологически сюръективным. Противоречие, значит, $\nu(E)=0$ и $\rho$ --- положительная функция. Значит, корректно определен изометрический изоморфизм $\bar{I}_p:L_p(\Omega,\nu)\to L_p(\Omega,\rho \nu):f\mapsto \rho^{-1/p}  f$. Очевидно, $M_{g \rho^{-1/p}}=M_g \bar{I}_p\in\mathcal{B}(L_p(\Omega,\nu),L_q(\Omega,\nu))$. Так как $\bar{I}_p$ --- изометрический изоморфизм и оператор $M_g$ топологически сюръективен, то таков же и $M_{g  \rho^{-1/p}}$. Осталось применить предложение \ref{TopSurMultOpCharacOnMeasSp}.

$iii)$$\implies$$ i)$ По предложению \ref{TopSurMultOpCharacOnMeasSp} оператор $M_{g \rho^{-1/p}}$ топологически сюръективен. Так как функция $\rho$ положительна, то корректно определен изометрический изоморфизм $\bar{I}_p$. Из равенства $M_g= M_{g \rho^{-1/p}}\bar{I}_p^{-1}$ следует, что оператор $M_g$ также топологически сюръективен.

$i)$$\implies$$ ii)$ Как мы доказали выше, оператор $M_{g \rho^{1/q}}$ топологически инъективен и $\bar{I}_q$ --- изометрический изоморфизм. Тогда по предложению \ref{TopSurMultOpCharacOnMeasSp} оператор $M_{g \rho^{1/q}}$ является топологическим изоморфизмом. Так как $M_g=\bar{I}_q M_{g \rho^{1/q}}$ то оператор $M_g$ тоже является топологическим изоморфизмом.

$ii)$$\implies$$ i)$. Если $M_g$ --- топологический изоморфизм, то он, очевидно, топологически сюръективен.
\end{proof}

\begin{proposition}\label{TopSurMultOpCharacBtwnTwoMeasSp} Пусть $(\Omega,\Sigma,\mu)$, $(\Omega,\Sigma,\nu)$ --- два $\sigma$-конечных пространства с мерой, $1\leq p,q\leq +\infty$ и $g\in L_0(\Omega,\Sigma)$. Тогда следующие условия эквивалентны:

$i)$ $M_g\in\mathcal{B}(L_p(\Omega,\mu), L_q(\Omega,\nu))$ --- топологически сюръективный оператор;

$ii)$ $M_g^{\Omega_c^{\mu,\nu}}$ --- топологически сюръективный оператор;

$iii)$ функция $\rho_{\mu,\nu}|_{\Omega_c^{\mu,\nu}}$ положительна, $|g \rho_{\mu,\nu}^{-1/p}|_{\Omega_c^{\mu,\nu}}|\geq c$ для некоторого $c>0$, при этом если $p\neq q$, то пространство $(\Omega,\Sigma,\mu)$ состоит из конечного числа атомов.
\end{proposition}
\begin{proof} По предложению \ref{MultOpDecompDecomp} оператор $M_g$ топологически сюръективен тогда и только тогда, когда таковы же операторы $M_g^{\Omega_c^{\mu,\nu}}:L_p(\Omega_c^{\mu,\nu},\rho_{\mu,\nu} \nu|_{\Omega_c^{\mu,\nu}})\to L_q(\Omega_c^{\mu,\nu},\nu|_{\Omega_c^{\mu,\nu}})$ и $M_g^{\Omega_s^{\mu,\nu}}:L_p(\Omega_s^{\mu,\nu},\mu_s|_{\Omega_s^{\mu,\nu}})\to L_q(\Omega_s^{\mu,\nu},\nu|_{\Omega_s^{\mu,\nu}})$. По предложению \ref{MultOpCharacBtwnTwoSingMeasSp} оператор $M_g^{\Omega_s^{\mu,\nu}}$ нулевой. Так как $\nu(\Omega_s^{\mu,\nu})=0$, то $L_p(\Omega_s^{\mu,\nu},\nu|_{\Omega_s^{\mu,\nu}})=\{0\}$. Отсюда мы заключаем, что $M_g^{\Omega_s^{\mu,\nu}}$ топологически сюръективен. Значит, топологическая сюръективность оператора $M_g$ эквивалентна топологической сюръективности оператора $M_g^{\Omega_c^{\mu,\nu}}$. Осталось применить предложение \ref{TopSurMultOpCharacBtwnTwoContMeasSp}.
\end{proof}

\begin{proposition}\label{TopSurMultOpDescBtwnTwoMeasSp} Пусть $(\Omega,\Sigma,\mu)$, $(\Omega,\Sigma,\nu)$ --- два $\sigma$-конечных пространства с мерой, $1\leq p,q\leq +\infty$ и $g\in L_0(\Omega,\Sigma)$. Тогда следующие условия эквивалентны:

$i)$ $M_g\in\mathcal{B}(L_p(\Omega,\mu),L_q(\Omega,\nu))$ --- топологически сюръективный оператор;

$ii)$ $M_{\chi_{\Omega_c^{\mu,\nu}}/g}\in\mathcal{B}(L_q(\Omega,\nu), L_p(\Omega,\mu))$ --- топологически инъективный правый обратный оператор к $M_g$.
\end{proposition}
\begin{proof}
$i)$$\implies$$ ii)$ По предложению \ref{MultOpDecompDecomp} оператор $M_g^{\Omega_c^{\mu,\nu}}$ топологически сюръективен. По предложению \ref{TopSurMultOpCharacBtwnTwoContMeasSp} он обратим, причем, очевидно, $(M_g^{\Omega_c^{\mu,\nu}})^{-1}=M_{1/g}^{\Omega_c^{\mu,\nu}}$. Тогда для любой функции $h\in L_q(\Omega,\nu)$ выполнено
$$
\Vert M_{\chi_{\Omega_c^{\mu,\nu}}/g}(h)\Vert_{L_p(\Omega,\mu)}=
\Vert M_{1/g}(h|_{\Omega_c^{\mu,\nu}})\Vert_{L_p(\Omega_c^{\mu,\nu},\mu|_{\Omega_c^{\mu,\nu}})}=
\Vert M_{1/g}^{\Omega_c^{\mu,\nu}}(h|_{\Omega_c^{\mu,\nu}})\Vert_{L_p(\Omega_c^{\mu,\nu},\mu|_{\Omega_c^{\mu,\nu}})}
$$
$$
\leq\Vert M_{1/g}^{\Omega_c^{\mu,\nu}}\Vert\Vert h|_{\Omega_c^{\mu,\nu}}\Vert_{L_q(\Omega_c^{\mu,\nu},\nu|_{\Omega_c^{\mu,\nu}})}
\leq\Vert M_{1/g}^{\Omega_c^{\mu,\nu}}\Vert\Vert h\Vert_{L_q(\Omega,\nu)}.
$$ 
Следовательно, оператор $M_{\chi_{\Omega_c^{\mu,\nu}}/g}$ ограничен. Теперь заметим, что для любой функции $h\in L_q(\Omega,\nu)$ также выполнено
$$
M_g(M_{\chi_{\Omega_c^{\mu,\nu}}/g}(h))
=M_g(\chi_{\Omega_c^{\mu,\nu}}/g  h)
=g (\chi_{\Omega_c^{\mu,\nu}}/g)   h
=h \chi_{\Omega_c^{\mu,\nu}}.
$$
Так как $\nu(\Omega\setminus\Omega_c^{\mu,\nu})=0$, то $\chi_{\Omega_c^{\mu,\nu}}=\chi_{\Omega}$ и поэтому $M_g(M_{\chi_{\Omega_c^{\mu,\nu}}/g}(h))=h \chi_{\Omega_c^{\mu,\nu}}=h \chi_{\Omega}=h$. Последнее означает, что $M_{\chi_{\Omega_c^{\mu,\nu}}/g}$ является правым обратным оператором умножения к $M_g$. Наконец, заметим, что
$$
\Vert M_{\chi_{\Omega_c^{\mu,\nu}}/g}(h)\Vert_{L_p(\Omega,\mu)}
\geq\Vert M_g\Vert\Vert M_g(M_{\chi_{\Omega_c^{\mu,\nu}}/g}(h))\Vert_{L_q(\Omega,\nu)}
\geq\Vert M_g\Vert\Vert h\Vert_{L_q(\Omega,\nu)}.
$$
для всех функций $h\in L_q(\Omega,\nu)$. Значит, оператор $M_{\chi_{\Omega_c^{\mu,\nu}}/g}$ топологически инъективен.

$ii)$$\implies$$ i)$ Для произвольной функции $h\in L_q(\Omega,\nu)$ рассмотрим функцию $f=M_{\chi_{\Omega_c^{\mu,\nu}}/g}(h)$. Тогда $M_g(f)=M_g(M_{\chi_{\Omega_c^{\mu,\nu}}/g}(h))=h$ и $\Vert f\Vert_{L_p(\Omega,\mu)}\leq\Vert M_{\chi_{\Omega_c^{\mu,\nu}}/g}\Vert\Vert h\Vert_{L_q(\Omega,\nu)}$. Так как $h$ произвольна, то оператор $M_g$ топологически сюръективен.
\end{proof}

\begin{proposition}\label{CoisomMultOpCharacOnMeasSp} Пусть $(\Omega,\Sigma,\mu)$ --- $\sigma$-конечное пространство с мерой, $1\leq p,q\leq +\infty$ и $g\in L_0(\Omega,\Sigma)$. Тогда следующие условия эквивалентны:

$i)$ $M_g\in\mathcal{B}(L_p(\Omega,\mu),L_q(\Omega,\mu))$ --- коизометричеcкий оператор;

$ii)$ $M_g$ --- изометрический изоморфизм;

$iii)$ $|g|=\mu(\Omega)^{1/q-1/p}$, при этом если $p\neq q$, то пространство $(\Omega,\Sigma,\mu)$ состоит из одного атома.
\end{proposition}
\begin{proof} Так как оператор $M_g$ коизометричен, то он топологически сюръективен и поэтому из предложения \ref{TopSurMultOpCharacOnMeasSp} мы получаем, что он топологический изоморфизм. Как следствие, он инъективен, но инъективный коизометрический оператор есть изометрический изоморфизм. Осталось заметить, что всякий изометрический изоморфизм является строгой коизометрией. Таким образом, оператор $M_g$ коизометричен тогда и только тогда, когда он строго коизометричен тогда и только тогда, когда он изометрический изоморфизм. Осталось применить предложение \ref{IsomMultOpCharacOnMeasSp}.
\end{proof}

\begin{proposition}\label{CoisomMultOpCharacBtwnTwoContMeasSp} Пусть $(\Omega,\Sigma,\nu)$ --- $\sigma$-конечное пространство с мерой, $1\leq p,q\leq +\infty$ и $g,\rho\in L_0(\Omega,\Sigma)$ причем функция $\rho$ неотрицательна. Тогда следующие условия эквивалентны:

$i)$ $M_g\in\mathcal{B}(L_p(\Omega,\rho \nu),L_q(\Omega,\nu))$ --- коизометрический оператор; 

$ii)$ $M_g$ --- изометрический изоморфизм;

$iii)$ функция $\rho$ положительна, $|g  \rho^{-1/p}|=\mu(\Omega)^{1/p-1/q}$, при этом если $p\neq q$, то пространство $(\Omega,\Sigma,\mu)$ состоит из одного атома.
\end{proposition}
\begin{proof} $i)$$\implies$$ ii)$ Допустим, оператор $M_g$ коизометричен, тогда он топологически сюръективен. По предложению \ref{TopSurMultOpCharacBtwnTwoContMeasSp} оператор $M_g$ является топологический изоморфизмом, и, как следствие, он биективен. Осталось заметить, что биективная коизометрия есть изометрический изоморфизм.

$ii)$$\implies$$ i)$ Если оператор $M_g$ --- изометрический изоморфизм, то он, конечно, коизометрия и даже строгая коизометрия.

$ii)\Longleftrightarrow iii)$ Эквивалентность следует из предложения \ref{IsomMultOpCharacBtwnTwoContMeasSp}.
\end{proof}

\begin{proposition}\label{CoisomMultOpCharacBtwnTwoMeasSp} Пусть $(\Omega,\Sigma,\mu)$, $(\Omega,\Sigma,\nu)$ --- два $\sigma$-конечных пространства с мерой, $1\leq p,q\leq +\infty$ и $g\in L_0(\Omega,\Sigma)$. Тогда следующие условия эквивалентны: 

$i)$ $M_g\in\mathcal{B}(L_p(\Omega,\mu), L_q(\Omega,\nu))$ --- коизометрический оператор;

$ii)$ $M_g^{\Omega_c^{\mu,\nu}}$ --- изометрический изоморфизм;

$iii)$ функция $\rho_{\mu,\nu}|_{\Omega_c^{\mu,\nu}}$ положительна, $|g \rho_{\mu,\nu}^{-1/p}|_{\Omega_c^{\mu,\nu}}|=\mu(\Omega_c^{\mu,\nu})^{1/p-1/q}$, при этом если $p\neq q$, то пространство $(\Omega,\Sigma,\mu)$ состоит из одного атома.
\end{proposition}
\begin{proof} $i)$$\implies$$ ii)$ Так как оператор $M_g$ коизометричен, то из предложения \ref{MultOpDecompDecomp} мы знаем, что оператор $M_g^{\Omega_c^{\mu,\nu}}$ также коизометричен. Из предложения \ref{CoisomMultOpCharacBtwnTwoContMeasSp} мы получаем, что он также является изометрическим изоморфизмом. 

$ii)$$\implies$$ i)$ Рассмотрим произвольную функцию $h\in L_q(\Omega,\nu)$, тогда существует функция $f\in L_p(\Omega_c^{\mu,\nu},\mu|_{\Omega_c^{\mu,\nu}})$ такая, что $M_g^{\Omega_c^{\mu,\nu}}(f)=h|_{\Omega_c^{\mu,\nu}}$. По предложению \ref{MultOpCharacBtwnTwoSingMeasSp} оператор $M_g^{\Omega_s^{\mu,\nu}}$ нулевой, поэтому
$$
M_g(\widetilde{f})
=\widetilde{M_g^{\Omega_c^{\mu,\nu}}(\widetilde{f}|_{\Omega_c^{\mu,\nu}})}+\widetilde{M_g^{\Omega_s^{\mu,\nu}}(\widetilde{f}|_{\Omega_s^{\mu,\nu}})}
=\widetilde{h|_{\Omega_c^{\mu,\nu}}}.
$$
Так как $\nu(\Omega_s^{\mu,\nu})=0$, то $\Vert h-\widetilde{h|_{\Omega_c^{\mu,\nu}}}\Vert_{L_q(\Omega,\nu)}=\Vert h\chi_{\Omega_s^{\mu,\nu}}\Vert_{L_q(\Omega,\nu)}=0$, и мы заключаем, что $h=\widetilde{h|_{\Omega_c^{\mu,\nu}}}$. Итак, мы построили функцию $\widetilde{f}\in L_p(\Omega,\mu)$ такую, что $M_g(\widetilde{f})=h$ и $\Vert \widetilde{f}\Vert_{L_p(\Omega,\mu)}=\Vert f\Vert_{L_p(\Omega_c^{\mu,\nu},\mu|_{\Omega_c^{\mu,\nu}})}=\Vert h|_{\Omega_c^{\mu,\nu}}\Vert_{L_q(\Omega_c^{\mu,\nu},\nu|_{\Omega_c^{\mu,\nu}})}\leq\Vert h\Vert_{L_q(\Omega,\nu)}$. Так как функция $h$ произвольна, то оператор $M_g$ $1$-топологически сюръективен. Заметим, что
$$
\Vert M_g(f)\Vert_{L_q(\Omega,\nu)}
=\Vert\widetilde{M_g^{\Omega_c^{\mu,\nu}}(f|_{\Omega_c^{\mu,\nu}})}+\widetilde{M_g^{\Omega_s^{\mu,\nu}}(f|_{\Omega_s^{\mu,\nu}})}\Vert_{L_q(\Omega,\nu)}
=\Vert\widetilde{M_g^{\Omega_c^{\mu,\nu}}(f|_{\Omega_c^{\mu,\nu}})}\Vert_{L_q(\Omega,\nu)}
$$
$$
=\Vert M_g^{\Omega_c^{\mu,\nu}}(f|_{\Omega_c^{\mu,\nu}})\Vert_{L_q(\Omega_c^{\mu,\nu},\nu|_{\Omega_c^{\mu,\nu}})}
=\Vert f|_{\Omega_c^{\mu,\nu}}\Vert_{L_p(\Omega_c^{\mu,\nu},\mu|_{\Omega_c^{\mu,\nu}})}
\leq\Vert f \Vert_{L_p(\Omega,\mu)}.
$$
для всех функций $f\in L_p(\Omega,\mu)$, значит оператор $M_g$ сжимающий, но от также $1$-топологически сюръективный. Следовательно, оператор $M_g$ коизометрический.

$ii)\Longleftrightarrow iii)$ Эквивалентность следует из предложения \ref{CoisomMultOpCharacBtwnTwoContMeasSp}.
\end{proof}

\begin{proposition}\label{CoisomMultOpDescBtwnTwoMeasSp} Пусть $(\Omega,\Sigma,\mu)$, $(\Omega,\Sigma,\nu)$ --- два $\sigma$-конечных пространства с мерой, $1\leq p,q\leq +\infty$ и $g\in L_0(\Omega,\Sigma)$. Тогда следующие условия эквивалентны: 

$i)$ $M_g\in\mathcal{B}(L_p(\Omega,\mu),L_q(\Omega,\nu))$ --- коизометрический оператор;

$ii)$ $M_{\chi_{\Omega_c^{\mu,\nu}}/g}\in\mathcal{B}(L_q(\Omega,\nu), L_p(\Omega,\mu))$ изометрический  правый обратный оператор к $M_g$.
\end{proposition}
\begin{proof}
$i)$$\implies$$ ii)$ По предложению \ref{MultOpDecompDecomp} оператор $M_g^{\Omega_c^{\mu,\nu}}$ коизометричен и по предложению \ref{CoisomMultOpCharacBtwnTwoContMeasSp} он изометричен и обратим. При этом, очевидно, $(M_g^{\Omega_c^{\mu,\nu}})^{-1}=M_{1/g}^{\Omega_c^{\mu,\nu}}$. Тогда для любой функции $h\in L_q(\Omega,\nu)$ мы имеем
$$
\Vert M_{\chi_{\Omega_c^{\mu,\nu}}/g}(h)\Vert_{L_p(\Omega,\mu)}=
\Vert M_{1/g}(h)\chi_{\Omega_c^{\mu,\nu}}\Vert_{L_p(\Omega,\mu)}=
\Vert M_{1/g}(h|_{\Omega_c^{\mu,\nu}})\Vert_{L_p(\Omega_c^{\mu,\nu},\mu|_{\Omega_c^{\mu,\nu}})}$$
$$
=
\Vert M_{1/g}^{\Omega_c^{\mu,\nu}}(h|_{\Omega_c^{\mu,\nu}})\Vert_{L_p(\Omega_c^{\mu,\nu},\mu|_{\Omega_c^{\mu,\nu}})}
=\Vert h|_{\Omega_c^{\mu,\nu}}\Vert_{L_q(\Omega_c^{\mu,\nu},\nu|_{\Omega_c^{\mu,\nu}})}
\leq\Vert h\Vert_{L_q(\Omega,\nu)}.
$$ 
Значит, $M_{\chi_{\Omega_c^{\mu,\nu}}/g}$ --- сжимающий оператор. Теперь заметим, что для всех функций $h\in L_q(\Omega,\nu)$ выполнено
$$
M_g(M_{\chi_{\Omega_c^{\mu,\nu}}/g}(h))
=M_g(\chi_{\Omega_c^{\mu,\nu}}/g  h)
=g (\chi_{\Omega_c^{\mu,\nu}}/g)   h
=h \chi_{\Omega_c^{\mu,\nu}}.
$$
Так как $\nu(\Omega\setminus\Omega_c^{\mu,\nu})=0$, то $\chi_{\Omega_c^{\mu,\nu}}=\chi_{\Omega}$ и поэтому $M_g(M_{\chi_{\Omega_c^{\mu,\nu}}/g}(h))=h \chi_{\Omega_c^{\mu,\nu}}=h \chi_{\Omega}=h$. Последнее означает, что $M_{\chi_{\Omega_c^{\mu,\nu}}/g}$ правый обратный оператор умножения к $M_g$. Наконец, для произвольной функции $h\in L_q(\Omega,\nu)$ мы имеем
$$
\Vert M_{\chi_{\Omega_c^{\mu,\nu}}/g}(h)\Vert_{L_p(\Omega,\mu)}
\geq\Vert M_g\Vert\Vert M_g(M_{\chi_{\Omega_c^{\mu,\nu}}/g}(h))\Vert_{L_q(\Omega,\nu)}
\geq\Vert h\Vert_{L_q(\Omega,\nu)}.
$$
Значит, оператор $M_{\chi_{\Omega_c^{\mu,\nu}}/g}$ $1$-топологически инъективен, но также сжимающий. Следовательно, оператор $M_{\chi_{\Omega_c^{\mu,\nu}}/g}$ изометрический.

$ii)$$\implies$$ i)$ Рассмотрим произвольную функцию $h\in L_q(\Omega,\nu)$ и функцию $f=M_{\chi_{\Omega_c^{\mu,\nu}}/g}(h)$. Тогда $M_g(f)=M_g(M_{\chi_{\Omega_c^{\mu,\nu}}/g}(h))=h$ и $\Vert f\Vert_{L_p(\Omega,\mu)}\leq\Vert h\Vert_{L_q(\Omega,\nu)}$. Так как $h$ произвольна, то оператор $M_g$ строго $1$-топологически сюръективен. Пусть $f\in L_p(\Omega,\mu)$. По предположению оператор $M_{\chi_{\Omega_c^{\mu,\nu}}/g}$ изометричен, поэтому 
$$
\Vert M_g(f)\Vert_{L_q(\Omega,\nu)}
=\Vert M_{\chi_{\Omega_c^{\mu,\nu}}/g}(M_g(f))\Vert_{L_p(\Omega,\mu)}
=\Vert f\chi_{\Omega_c^{\mu,\nu}}\Vert_{L_p(\Omega,\mu)}
\leq\Vert f\Vert_{L_p(\Omega,\mu)}.
$$
Так как $f$ произвольна, то оператор $M_g$ сжимающий, но он еще и $1$-топологически сюръективен, а значит строго коизометричен.
\end{proof}

Это доказательство также показывает, что каждый коизометрический оператор умножения строго коизометричен.



%----------------------------------------------------------------------------------------
%	Homological triviality of the category B(Omega,Sigma)-modules L_p
%----------------------------------------------------------------------------------------

\subsection{Гомологическая тривиальность категории \texorpdfstring{$B(\Omega,\Sigma)$-модулей $L_p$}{B(Omega)-модулей Lp}}
\label{SubSectionHomologicalTrivialityOfTheCategoryBOmegaSigmaModulesLp}

Теперь мы готовы доказать, что категория рассмотренная в начале параграфа гомологически тривиальна, то есть все ее модули являются проективными, инъективными и плоскими.

\begin{proposition}\label{HomTrivlOfLpCat} Пусть $(\Omega,\Sigma)$ измеримое пространство и $\mu$ --- $\sigma$-конечная мера на $\Omega$. Тогда $B(\Omega,\Sigma)$-модуль $L_p(\Omega,\mu)$ $\langle$~метрически / топологически~$\rangle$ проективный, инъективный и плоский по отношению к категории $\langle$~$B(\Omega,\Sigma)-\mathbf{mod(L)}_1$ / $B(\Omega,\Sigma)-\mathbf{mod(L)}$~$\rangle$.
\end{proposition}
\begin{proof} Обозначим $X:=L_p(\Omega,\mu)$ и $\langle$~$\mathbf{C}:=B(\Omega,\Sigma)-\mathbf{mod(L)}_1$ / $\mathbf{C}:=B(\Omega,\Sigma)-\mathbf{mod(L)}$~$\rangle$. 

Рассмотрим ковариантный функтор $\langle$~$F_{proj}:=\operatorname{Hom}_{\mathbf{C}}(X,-):\mathbf{C}\to\mathbf{Ban}_1$ / $F_{proj}:=\operatorname{Hom}_{\mathbf{C}}(X,-):\mathbf{C}\to\mathbf{Ban}$~$\rangle$. По предложению $\langle$~\ref{CoisomMultOpDescBtwnTwoMeasSp} / \ref{TopSurMultOpCharacBtwnTwoMeasSp}~$\rangle$ любой $\langle$~коизометрический / топологически сюръективный~$\rangle$ морфизм $\xi$ в $\mathbf{C}$ есть ретракция, значит, оператор $F_{proj}(\xi)$ --- ретракция в $\langle$~$\mathbf{Ban}_1$ / $\mathbf{Ban}$~$\rangle$, и, как следствие, $\langle$~строго коизометричен / сюръективен~$\rangle$. Так как морфизм $\xi$ произволен, то модуль $X$ $\langle$~метрически / топологически~$\rangle$ проективен.

Рассмотрим контравариантный функтор $\langle$~$F_{inj}=\operatorname{Hom}_{\mathbf{C}}(-,X):\mathbf{C}\to\mathbf{Ban}_1$ / $F_{inj}=\operatorname{Hom}_{\mathbf{C}}(-,X):\mathbf{C}\to\mathbf{Ban}$~$\rangle$. Из предложения $\langle$~\ref{IsomMultOpDescBtwnTwoMeasSp} / \ref{TopInjMultOpDescBtwnTwoMeasSp}~$\rangle$ любой $\langle$~изометрический / топологически инъективный~$\rangle$ морфизм $\xi$ из $\mathbf{C}$ является коретракцией, значит, оператор $F_{inj}(\xi)$ --- ретракция в $\langle$~$\mathbf{Ban}_1$ / $\mathbf{Ban}$~$\rangle$, и, как следствие, $\langle$~строго коизометричен / сюръективен~$\rangle$. Так как морфизм $\xi$ произволен, то модуль $X$ $\langle$~метрически / топологически~$\rangle$ инъективен.

Рассмотрим ковариантный функтор $\langle$~$F_{flat}=-\projmodtens{B(\Omega,\Sigma)}X:\mathbf{C}\to\mathbf{Ban}_1$ / $F_{flat}=-\projmodtens{B(\Omega,\Sigma)}X:\mathbf{C}\to\mathbf{Ban}$~$\rangle$. Снова, по предложению $\langle$~\ref{IsomMultOpDescBtwnTwoMeasSp} / \ref{TopInjMultOpDescBtwnTwoMeasSp}~$\rangle$ любой $\langle$~изометрический / топологически инъективный~$\rangle$ морфизм $\xi$ в $\mathbf{C}$ является коретракцией, значит, оператор $F_{flat}(\xi)$ --- коретракция в $\langle$~$\mathbf{Ban}_1$ / $\mathbf{Ban}$~$\rangle$, и, как следствие, он $\langle$~изометричен / топологически инъективен~$\rangle$. Так как морфизм $\xi$ произволен, то модуль $X$ $\langle$~метрически / топологически~$\rangle$ плоский.
\end{proof}