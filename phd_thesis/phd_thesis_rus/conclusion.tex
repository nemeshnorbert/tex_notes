\chapter*{Заключение}						% Заголовок
\addcontentsline{toc}{chapter}{Заключение}	% Добавляем его в оглавление

Подведем итоги. В диссертации была построена общая теория метрически и топологически проективных, инъективных и плоских модулей. Было дано описание широкого класса топологически проективных идеалов банаховых алгебр. На примере свойств Данфорда-Петтиса и свойства l.u.st. была указана тесная взаимосвязь банаховой геометрии и гомологической алгебры. В работе получены необходимые условия для топологической плоскости банаховых модулей над относительно аменабельными алгебрами и дан критерий топологической инъективности $AW^*$-алгебр. На основе доказанных теорем были изучены гомологические свойства классических модулей над алгебрами последовательностей, алгебрами операторов и групповыми алгебрами.

Скажем несколько слов о нерешенных проблемах и возможных направлениях дальнейшего исследования. Среди нерешенных задач стоит отметить следующие две.

\begin{problem*} Существует ли необходимое условие топологической проективности  идеала коммутативной банаховой алгебры более сильное, чем паракомпактность спектра?
\end{problem*}

Всякий топологически проективный идеал коммутативной банаховой алгебры относительно проективен и поэтому обладает паракомпактным спектром. Как показывает пример алгебры $A_0(\mathbb{D})$, компактность необходимым условием не является.

\begin{problem*} Верно ли, что $C^*$-алгебры топологически инъективные над собой как правые модули являются $AW^*$-алгебрами?
\end{problem*}

В случае метрической инъективности это было доказано Хаманой в \cite{HamInjEnvBanMod}. В случае топологической инъективности это, похоже, будет сложной задачей даже в коммутативном случае. На данный момент не известно ни одного примера содержательной категории функционального анализа, где было бы получено полное описание топологически инъективных объектов.

Направлений для дальнейшего исследования достаточно много. Например, в работе не рассматриваются гомологические свойства бимодулей. Данное исследование ограничивается только случаем банаховых модулей, а более общие нормированные модули не рассматриваются. Похоже, что утверждение о важной роли банаховой геометрии для метрической и топологической гомологии нормированных модулей останется верным, но подтвердить его будет намного сложнее, так как результатов о геометрии нормированных пространств не так уж много. Здесь следует упомянуть две работы. В \cite{HelMetrFrQMod} Хелемский доказал, что все метрически проективные нормированные пространства изометрически изоморфны $\ell_1^0$-пространствам, то есть пространствам финитных функции с $\ell_1$-нормой. После этого в \cite{GronbeakLiftProblmNorSp} Грюнбек доказал, что все топологически проективные нормированные пространства топологически изоморфны $\ell_1^0$-пространствам. 

Наконец, отметим, что в работе рассматриваются только гомологически тривиальные модули. Было бы интересно увидеть как резольвенты, когомологии и глобальные размерности в топологической теории отличаются от своих ``относительных'' аналогов. Первые шаги в этом направлении для модулей над нулевой алгеброй, то есть для банаховых пространств, можно найти в [\cite{HavLinComplAnalPrblmBook}, проблема 1.14]. Для метрической теории вместо стандартной техники резольвент, скорее всего, придется применять расширения Йонеды. 

\clearpage