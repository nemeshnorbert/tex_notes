%%% Макет страницы %%%
% Выставляем значения полей (ГОСТ 7.0.11-2011, 5.3.7)
\geometry{a4paper,top=2cm,bottom=2cm,left=2.5cm,right=1cm}

%%% Кодировки и шрифты %%%
\ifxetex{}
  \setmainlanguage{russian}
  \setotherlanguage{english}
  \defaultfontfeatures{Ligatures=TeX,Mapping=tex-text}
  \setmainfont{Times New Roman}
  \newfontfamily\cyrillicfont{Times New Roman}
  \setsansfont{Arial}
  \newfontfamily\cyrillicfontsf{Arial}
  \setmonofont{Courier New}
  \newfontfamily\cyrillicfonttt{Courier New}
\else
  \IfFileExists{pscyr.sty}{\renewcommand{\rmdefault}{ftm}}{}
\fi

%%% Интервалы %%%
% Полуторный интервал (ГОСТ Р 7.0.11-2011, 5.3.6)
\linespread{1.3}                    

%%% Выравнивание и переносы %%%
% Избавляемся от переполнений
\sloppy                             
% Запрещаем разрыв страницы после первой строки абзаца
\clubpenalty=10000                  
% Запрещаем разрыв страницы после последней строки абзаца
\widowpenalty=10000                 

%%% Библиография %%%
%\makeatletter
% % Оформляем библиографию по ГОСТ 7.1 (ГОСТ Р 7.0.11-2011, 5.6.7)
%\bibliographystyle{utf8gost71u}     
% % Заменяем библиографию с квадратных скобок на точку
%\renewcommand{\@biblabel}[1]{#1.}   
%\makeatother

%%% Изображения %%%
%\graphicspath{{images/}}            % Пути к изображениям

%%% Цвета гиперссылок %%%
\definecolor{linkcolor}{rgb}{0,0,1}
\definecolor{citecolor}{rgb}{0,0,1}
\definecolor{urlcolor}{rgb}{0,0,1}
\hypersetup{
    colorlinks, linkcolor={linkcolor},
    citecolor={citecolor}, urlcolor={urlcolor}
}

%%% Оглавление %%%
\renewcommand{\cftchapdotsep}{\cftdotsep}

%%% Шаблон %%%
\newcommand{\todo}[1]{\textcolor{blue}{#1}}

%%% Списки %%%
% Используем дефис для ненумерованных списков (ГОСТ 2.105-95, 4.1.7)
\renewcommand{\labelitemi}{\normalfont\bfseries{--}} 

%%% Колонтитулы %%%
% Порядковый номер страницы печатают на середине 
% верхнего поля страницы (ГОСТ Р 7.0.11-2011, 5.3.8)
\makeatletter
% Подчиняем первые страницы каждой главы общим правилам
\let\ps@plain\ps@fancy{}              
\makeatother
% Меняем стиль оформления страниц
\pagestyle{fancy}                   
% Очищаем текущие значения
\fancyhf{}                          
% Печатаем номер страницы на середине верхнего поля
\fancyhead[C]{\thepage}             
% Убираем разделительную линию
\renewcommand{\headrulewidth}{0pt}  

\setlength{\headheight}{15pt}