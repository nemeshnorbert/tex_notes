 %\title{PhD thesis}
%%%%%%%%%%%%%%%%%%%%%%%%%%%%%%%%%%%%%%%%%
% Thesis 
% LaTeX Template
% Version 1.3 (21/12/12)
%
% This template has been downloaded from:
% http://www.latextemplates.com
%
% Original authors:
% Steven Gunn 
% http://users.ecs.soton.ac.uk/srg/softwaretools/document/templates/
% and
% Sunil Patel
% http://www.sunilpatel.co.uk/thesis-template/
%
% License:
% CC BY-NC-SA 3.0 (http://creativecommons.org/licenses/by-nc-sa/3.0/)
%
% Note:
% Make sure to edit document variables in the Thesis.cls file
%
%%%%%%%%%%%%%%%%%%%%%%%%%%%%%%%%%%%%%%%%%

% TODO(norberrt)
% Get rid of the section on the "small category"
% Get rid of excessively detailed introduction on approximation property
% 

%----------------------------------------------------------------------------------------
%	PACKAGES AND OTHER DOCUMENT CONFIGURATIONS
%----------------------------------------------------------------------------------------

\documentclass[11pt, a4paper, oneside]{Thesis} % Paper size, default font size and one-sided paper 

\graphicspath{{./Pictures/}} % Specifies the directory where pictures are stored

\usepackage[square, numbers, comma, sort&compress]{natbib} % Use the natbib reference package - read up on this to edit the reference style; if you want text (e.g. Smith et al., 2012) for the in-text references (instead of numbers), remove 'numbers' 
\hypersetup{urlcolor=blue, colorlinks=true} % Colors hyperlinks in blue - change to black if annoying
\title{\ttitle} % Defines the thesis title - don't touch this

\usepackage{amssymb,amsmath}
\usepackage[utf8]{inputenc} 
\usepackage{mathrsfs}
\usepackage[matrix,arrow,curve]{xy}


%----------------------------------------------------------------------------------------
%	MY COMMANDS AND DEFINITIONS
%----------------------------------------------------------------------------------------

\newcommand{\projtens}{\mathbin{\widehat{\otimes}}}
\newcommand{\convol}{\ast}
\newcommand{\projmodtens}[1]{\mathbin{\widehat{\otimes}}_{#1}}
\newcommand{\isom}[1]{\mathop{\mathbin{\cong}}\limits_{#1}}

\begin{document}

\frontmatter % Use roman page numbering style (i, ii, iii, iv...) for the pre-content pages

\setstretch{1.3} % Line spacing of 1.3

% Define the page headers using the FancyHdr package and set up for one-sided printing
\fancyhead{} % Clears all page headers and footers
\rhead{\thepage} % Sets the right side header to show the page number
\lhead{} % Clears the left side page header

\pagestyle{fancy} % Finally, use the "fancy" page style to implement the FancyHdr headers

\newcommand{\HRule}{\rule{\linewidth}{0.5mm}} % New command to make the lines in the title page

% PDF meta-data
\hypersetup{pdftitle={\ttitle}}
\hypersetup{pdfsubject=\subjectname}
\hypersetup{pdfauthor=\authornames}
\hypersetup{pdfkeywords=\keywordnames}

%----------------------------------------------------------------------------------------
%	TITLE PAGE
%----------------------------------------------------------------------------------------

\begin{titlepage}

\begin{center}

\textsc{\LARGE \univname}\\[1.5cm] % University name
\textsc{\Large Doctoral Thesis}\\[0.5cm] % Thesis type

\HRule \\[0.4cm] % Horizontal line
{\huge \bfseries \ttitle}\\[0.4cm] % Thesis title
\HRule \\[1.5cm] % Horizontal line
 
\begin{minipage}{0.4\textwidth}
\begin{flushleft} \large
\emph{Author:}\\

{\authornames} % Author name. Full example \href{http://www.johnsmith.com}{\authornames}. Remove the \href bracket to remove the link 
\end{flushleft}
\end{minipage}
\begin{minipage}{0.4\textwidth}
\begin{flushright} \large
\emph{Supervisor:} \\
{\supname} % Supervisor name. Full example \href{http://www.jamessmith.com}{\supname}. Remove the \href bracket to remove the link
\end{flushright}
\end{minipage}\\[3cm]
 
\large \textit{A thesis submitted in fulfilment of the requirements\\ for the degree of \degreename}\\[0.3cm] % University requirement text
\textit{in the}\\[0.4cm]
\groupname\\\deptname\\[2cm] % Research group name and department name
 
{\large \today}\\[4cm] % Date
%\includegraphics{Logo} % University/department logo - uncomment to place it
 
\vfill
\end{center}

\end{titlepage}

%----------------------------------------------------------------------------------------
%	DECLARATION PAGE
%	Your institution may give you a different text to place here
%----------------------------------------------------------------------------------------

\Declaration{

\addtocontents{toc}{\vspace{1em}} % Add a gap in the Contents, for aesthetics

I, \authornames, declare that this thesis titled, '\ttitle' and the work presented in it are my own. I confirm that:

\begin{itemize} 
\item[\tiny{$\blacksquare$}] This work was done wholly or mainly while in candidature for a research degree at this University.
\item[\tiny{$\blacksquare$}] Where any part of this thesis has previously been submitted for a degree or any other qualification at this University or any other institution, this has been clearly stated.
\item[\tiny{$\blacksquare$}] Where I have consulted the published work of others, this is always clearly attributed.
\item[\tiny{$\blacksquare$}] Where I have quoted from the work of others, the source is always given. With the exception of such quotations, this thesis is entirely my own work.
\item[\tiny{$\blacksquare$}] I have acknowledged all main sources of help.
\item[\tiny{$\blacksquare$}] Where the thesis is based on work done by myself jointly with others, I have made clear exactly what was done by others and what I have contributed myself.\\
\end{itemize}
 
Signed:\\
\rule[1em]{25em}{0.5pt} % This prints a line for the signature
 
Date:\\
\rule[1em]{25em}{0.5pt} % This prints a line to write the date
}

\clearpage % Start a new page

%----------------------------------------------------------------------------------------
%	QUOTATION PAGE
%----------------------------------------------------------------------------------------

\pagestyle{empty} % No headers or footers for the following pages

\null\vfill % Add some space to move the quote down the page a bit

\textit{Never trust any result proved after 11 PM. }

\begin{flushright}
Professional secret
\end{flushright}

\vfill\vfill\vfill\vfill\vfill\vfill\null % Add some space at the bottom to position the quote just right

\clearpage % Start a new page

%----------------------------------------------------------------------------------------
%	ABSTRACT PAGE
%----------------------------------------------------------------------------------------

\addtotoc{Abstract} % Add the "Abstract" page entry to the Contents

\abstract{\addtocontents{toc}{\vspace{1em}}} % Add a gap in the Contents, for aesthetics

In this thesis we study metric and topological versions of projectivity injectivity and flatness of Banach modules over Banach algebras. These two non standard versions of Banach homology theories are studied in parallel under unified approach.

Chapter 1 gives background required for our studies. In paragraph 1.1 we collect all necessary facts on category theory, topology and measure theory. Paragraph 1.2 contains a brief introduction into Banach structures. Here we discuss Banach spaces, Banach algebras and Banach modules. In paragraph 1.3 we give a short introduction into relative Banach homology. Then we list main results from the theory of rigged categories. This theory is a common ground for many known versions of projectivity and injectivity, in particular of metric and topological ones.

In chapter 2 we establish general properties of projective injective and flat modules. In some cases we give complete characterizations of such modules. Results of this chapter are extensively used in chapter 3 when dealing with specific modules of analysis. Let us discuss contents of this chapter in more detail. In paragraph 2.1 we derive basic properties of projective injective and flat modules from the theory of rigged categories. We also study different constructions that preserve homological triviality of Banach modules. These results are used to characterize projectivity and flatness of cyclic modules. We also give necessary conditions for projectivity of left ideals of Banach algebras. As the consequence we describe projective ideals of commutative Banach algebras that admit bounded approximate identities. Paragraph 2.2 is devoted to Banach geometric properties of homologically trivial modules. We characterize projective injective and flat annihilator modules and establish their strong relation to projective injective and flat Banach spaces. Then we give several examples confirming that homologically trivial module and its Banach algebra have similar Banach geometric properties. Examples include the property of being an $\mathscr{L}_1^g$-space, the Dunford-Pettis property and the l.u.st. property. Paragraph 2.3 is quite short. Here we list conditions under which projectivity injectivity and flatness are preserved under transition between modules over algebra to modules over ideal. Finally, we give a necessary and sufficient conditions of topological flatness of a Banach module and necessary condition of injectivity of two-sided ideals.

In chapter 3 we apply general results to specific modules of analysis. In paragraph 3.1 we investigate projectivity injectivity and flatness of ideals of $C^*$-algebras. We describe projective left ideals of $C^*$-algebras and give a criterion for injectivity of $AW^*$-algebras. These characterizations are indispensable in description of homologically trivial modules over algebras of bounded and compact operators on a Hilbert space. We perform similar research for commutative case of algebras of bounded and vanishing functions on discrete sets and locally compact Hausdorff spaces. In paragraph 3.2 we proceed to study of standard modules of harmonic analysis. Due to specific Banach geometric structure of convolution algebra and measure algebra we easily show that most of standard modules of harmonic analysis are homologically non trivial. The most intriguing result of the paragraph is non-projectivity of convolution algebra of a non discrete group. In paragraph 3.3 we construct an example of category which behaves essentially different than the standard categories of analysis. This category consist of $L_p$-spaces with the structure of Banach module over algebra of bounded measurable functions. All modules of this category turn out to be homologically trivial. To show this we characterize topologically injective, topologically surjective, isometric and coisometric multiplication operators between $L_p$-spaces.

\clearpage % Start a new page

%----------------------------------------------------------------------------------------
%	ACKNOWLEDGEMENTS
%----------------------------------------------------------------------------------------

\setstretch{1.3} % Reset the line-spacing to 1.3 for body text (if it has changed)

\acknowledgements{\addtocontents{toc}{\vspace{1em}} % Add a gap in the Contents, for aesthetics

Firstly, I would like to thank my supervisor Professor Alexander Helemskii for formulation of interesting problems and continuous encouragement to finish this work.  I'm grateful to Narutaka Ozawa, Leonel Robert and Alexei Pirkovskii for answering my silly questions. Also I would like to thank Tomasz Kania for bringing my attention to the l.u.st. property. Lastly, but definitely not least, I want to thank Viktoria for continued support and love, without her this project would not have been possible.
}
\clearpage % Start a new page

%----------------------------------------------------------------------------------------
%	LIST OF CONTENTS/FIGURES/TABLES PAGES
%----------------------------------------------------------------------------------------

\pagestyle{fancy} % The page style headers have been "empty" all this time, now use the "fancy" headers as defined before to bring them back

\lhead{\emph{Contents}} % Set the left side page header to "Contents"
\tableofcontents % Write out the Table of Contents

%\lhead{\emph{List of Figures}} % Set the left side page header to "List of Figures"
%\listoffigures % Write out the List of Figures

%\lhead{\emph{List of Tables}} % Set the left side page header to "List of Tables"
%\listoftables % Write out the List of Tables

%----------------------------------------------------------------------------------------
%	ABBREVIATIONS
%----------------------------------------------------------------------------------------

%\clearpage % Start a new page

%\setstretch{1.5} % Set the line spacing to 1.5, this makes the following tables easier to read

%\lhead{\emph{Abbreviations}} % Set the left side page header to "Abbreviations"
%\listofsymbols{ll} % Include a list of Abbreviations (a table of two columns)
%{
%\textbf{LAH} & \textbf{L}ist \textbf{A}bbreviations \textbf{H}ere \\
%\textbf{Acronym} & \textbf{W}hat (it) \textbf{S}tands \textbf{F}or \\
%}

%----------------------------------------------------------------------------------------
%	PHYSICAL CONSTANTS/OTHER DEFINITIONS
%----------------------------------------------------------------------------------------

%\clearpage % Start a new page

%\lhead{\emph{Physical Constants}} % Set the left side page header to "Physical Constants"

%\listofconstants{lrcl} % Include a list of Physical Constants (a four column table)
%{
%Speed of Light & $c$ & $=$ & $2.997\ 924\ 58\times10^{8}\ \mbox{ms}^{-\mbox{s}}$ (exact)\\
% Constant Name & Symbol & = & Constant Value (with units) \\
%}

%----------------------------------------------------------------------------------------
%	DEDICATION
%----------------------------------------------------------------------------------------

%\setstretch{1.3} % Return the line spacing back to 1.3

%\pagestyle{empty} % Page style needs to be empty for this page

%\dedicatory{For/Dedicated to/To my\ldots} % Dedication text

%\addtocontents{toc}{\vspace{2em}} % Add a gap in the Contents, for aesthetics

%----------------------------------------------------------------------------------------
%	THESIS CONTENT - CHAPTERS
%----------------------------------------------------------------------------------------

\mainmatter % Begin numeric (1,2,3...) page numbering

\pagestyle{fancy} % Return the page headers back to the "fancy" style

% Include the chapters of the thesis as separate files from the Chapters folder
% Uncomment the lines as you write the chapters

% Chapter Template

\chapter{Preliminaries} % Main chapter title

\label{ChapterPreliminaries} % Change X to a consecutive number; for referencing this chapter elsewhere, use \ref{ChapterX}

\lhead{Chapter 1. \emph{Preliminaries}} % Change X to a consecutive number; this is for the header on each page - perhaps a shortened title

In what follows, we present some parts in parallel fashion by listing the respective options in order, enclosed and separate like this: $\langle$~.../...~$\rangle$. For example: a real number $x$ is $\langle$~positive / non negative~$\rangle$ if $\langle$~$x>0$ / $x\geq 0$~$\rangle$. Sometimes one of the parts might be empty. We use symbol $:=$ for equality by definition.

We use the following standard notation for some commonly used sets of numbers: $\mathbb{C}$ denotes the complex numbers, $\mathbb{R}$ denotes the real numbers, $\mathbb{Z}$ denotes the integers, $\mathbb{N}$ denotes the natural numbers, $\mathbb{N}_n$ denotes the set of first $n$ natural numbers, $\mathbb{R}_+$ denotes the set of non negative real numbers, $\mathbb{T}$ denotes the set of complex numbers of modulus $1$, finally, $\mathbb{D}$ denotes the set of complex numbers with modulus less than $1$. For $z\in\mathbb{C}$ the symbol $\overline{z}$ stands for the complex conjugate number.

For a given map $f:M\to M'$ and subset $\langle$~$N\subset M$ / $N'\subset M'$ such that $\operatorname{Im}(f)\subset N'$~$\rangle$ by $\langle$~$f|_N$ / $f|^{N'}$~$\rangle$ we denote the  $\langle$~restriction of $f$ onto $N$ / corestriction of $f$ onto $N'$~$\rangle$, that is $\langle$~$f|_N:N\to M':x\mapsto f(x)$ / $f|^{N'}:M\to N':x\mapsto f(x)$~$\rangle$. The indicator function of a subset $N$ of the set $M$ is denoted by $\chi_{N}$, so that $\chi_N(x)=1$ for $x\in N$ and $\chi_N(x)=0$ for $x\in M\setminus N$. We also use the shortcut $\delta_x=\chi_{\{x\}}$ where $x\in M$. By $\mathcal{P}(M)$ we denote the set of all subsets of $M$, and $\mathcal{P}_0(M)$ stands for the set of all finite subsets of $M$. The symbol $M^N$ stands for the set of all functions from $N$ to $M$. By $\operatorname{Card}(M)$ we denote the cardinality of $M$. By $\aleph_0$ we denote the cardinality of $\mathbb{N}$.

%----------------------------------------------------------------------------------------
%	Broad foundations
%----------------------------------------------------------------------------------------

\section{Broad foundations}

\label{SectionBroadFoundations} 

%----------------------------------------------------------------------------------------
%	Categorical language
%----------------------------------------------------------------------------------------

\subsection{Categorical language}
\label{SubSectionCategoricalLanguage}

Here we recall some basic facts and definitions from category theory and fix notation we shall use. We assume that our reader is familiar with such basics of category theory as category, functor, morphism. Otherwise see [\cite{HelLectAndExOnFuncAn}, chapter 0] for a quick introduction or [\cite{KashivShapCatsAndSheavs}, chapter 1] for more details.

For a given category $\mathbf{C}$ by $\operatorname{Ob}(\mathbf{C})$ we denote the class of its objects. The symbol $\mathbf{C}^o$ stands for the opposite category. For a given objects $X$ and $Y$ by $\operatorname{Hom}_{\mathbf{C}}(X, Y)$ we denote the set of morphisms from $X$ to $Y$. Often we shall write $\phi:X\to Y$ instead of $\phi\in\operatorname{Hom}_{\mathbf{C}}(X,Y)$. A morphism $\phi:X\to Y$ is called $\langle$~retraction / coretraction~$\rangle$ if it has a $\langle$~right / left~$\rangle$ inverse morphism. Morphism $\phi$ is called an isomorphism if it is retraction and coretraction. Usually we shall express existence of isomorphism between $X$ and $Y$ as $X\isom{\mathbf{C}} Y$. We say that two morphisms $\phi:X_1\to Y_1$ and $\psi:X_2\to Y_2$ are equivalent in $\mathbf{C}$ if there exist isomorphisms $\alpha:X_1\to X_2$ and $\beta:Y_1\to Y_2$ such that $\beta\phi=\psi\alpha$.

The first obvious example of the category that comes to mind is the category of all sets and all maps between them. We denote this category by $\mathbf{Set}$. Other examples will be given later. Two main examples of functors that any category has are functors of morphisms. For a fixed $X\in\operatorname{Ob}(\mathbf{C})$ we define covariant and a contravariant functors
$$
\operatorname{Hom}_{\mathbf{C}}(X,-):\mathbf{C}\to\mathbf{Set}:Y\mapsto \operatorname{Hom}_{\mathbf{C}}(X,Y), \phi\mapsto(\psi\mapsto \phi\circ\psi),
$$
$$
\operatorname{Hom}_{\mathbf{C}}(-,X):\mathbf{C}\to\mathbf{Set}:Y\mapsto \operatorname{Hom}_{\mathbf{C}}(Y,X), \phi\mapsto(\psi\mapsto \psi\circ\phi).
$$
This construction has its reminiscent analogs in many categories of mathematics with slight modification of categories between which these functors act.

We say that two covariant functors $F:\mathbf{C}\to\mathbf{D}$, $G:\mathbf{C}\to\mathbf{D}$ are isomorphic if there exists a class of isomorphisms $\{\eta_X:X\in\operatorname{Ob}(\mathbf{C})\}$ in $\mathbf{D}$ (called natural isomorphisms), such that $G(f)\eta_X=\eta_Y F(f)$ for all $f:X\to Y$. In this case we simply write $F\cong G$. A $\langle$~covariant / contravariant~$\rangle$ functor $F:\mathbf{C}\to\mathbf{D}$ is called representable by object $X$ if $\langle$~$F\cong\operatorname{Hom}_{\mathbf{C}}(X,-)$ / $F\cong\operatorname{Hom}_{\mathbf{C}}(-,X)$~$\rangle$. If functor is representable, then its representing object is unique up to isomorphism in $\mathbf{C}$.

Constructions of categorical product and coproduct shall play an important role in this thesis. We say that $X$ is the the $\langle$~product / coproduct~$\rangle$ of the family of objects $\{X_\lambda:\lambda\in\Lambda\}$ if the functor $\langle$~$\prod_{\lambda\in\Lambda}\operatorname{Hom}_{\mathbf{C}}(-,X_{\lambda}):\mathbf{C}\to\mathbf{Set}$ / $\prod_{\lambda\in\Lambda}\operatorname{Hom}_{\mathbf{C}}(X_{\lambda},-):\mathbf{C}\to\mathbf{Set}$~$\rangle$ is representable by object $X$. As the consequence we get that a $\langle$~product / coproduct~$\rangle$, if it exists, is unique up to an isomorphism. Later we shall give examples of the $\langle$~products / coproducts~$\rangle$ in different categories of functional analysis. 


%----------------------------------------------------------------------------------------
%	Topology
%----------------------------------------------------------------------------------------

\subsection{Topology}
\label{SubSectionTopology}

Let $(S,\tau)$ be a topological space. Elements of $\tau$ are called open sets, and their complements are called closed sets. Let $E$ be an arbitrary subset of $S$. By $\operatorname{cl}_S(E)$ we denote the closure of $E$ in $S$, that is the smallest closed set that contains $E$. Similarly, by $\operatorname{int}_S(E)$ we denote the interior of $E$, that is the largest open set that contained in $E$. We say that $E$ is a neighborhood of point $s\in S$ if $s\in \operatorname{int}_S(E)$. We say that a set $E$ is dense in a set $F$ if $F\subset\operatorname{cl}_S(E)$. Note that $E$ may be regarded as topological space, if we endow it with subspace topology which equals $\{U\cap E:U\in\tau\}$. Topological space is called Hausdorff if any two distinct points have disjoint open neighborhoods. In this thesis we shall work with Hausdorff spaces only.

A map $f:X\to Y$ between topological spaces is called continuous if preimage under $f$ of any open set is open. By $\mathbf{Top}$ we denote the category of topological spaces with continuous maps in the role of morphisms. Isomorphisms in $\mathbf{Top}$ are called homeomorphisms. The category $\mathbf{Top}$ admits  products. For a given family of topological spaces $\{S_\lambda:\lambda\in\Lambda\}$ their product is the Tychonoff product $\prod_{\lambda\in\Lambda}S_\lambda$, that is the Cartesian product of the family $\{S_\lambda:\lambda\in\Lambda\}$ with the coarsest topology making all natural projections $p_\lambda:\prod_{\lambda\in\Lambda}S_\lambda\to S_\lambda$ continuous.

In this thesis we shall encounter four types of topological spaces: compact spaces, paracompact spaces, locally compact spaces and extremely disconnected spaces. Before giving their definitions we need to recall the notion of cover. Let $\mathcal{E}$ be a family of subsets of topological space $S$. We say that $\mathcal{E}$ is a cover if its union equals $S$. We say that cover is open if all its elements are open sets. A cover is called locally finite if any point of $S$ has a neighborhood that intersects only finitely many elements of the cover. We say that cover $\mathcal{E}_1$ is inscribed into cover $\mathcal{E}_2$ if any element of $\mathcal{E}_1$ is a subset of some element of $\mathcal{E}_2$. A cover $\mathcal{E}_1$ is called a subcover of $\mathcal{E}_2$ if $\mathcal{E}_1\subset\mathcal{E}_2$. Finally, a topological space is called 

$i)$ compact if any its open cover admits a finite open subcover; 

$ii)$ paracompact if any its open cover is inscribed into some locally finite open cover; 

$iii)$ locally compact if any its point has a compact neighborhood;

$iv)$ extremely disconnected spaces if the closure of any its open set is open;

$v)$ Stonean if it is an extremely disconnected Hausdorff compact space.

The property of being $\langle$~compact / paracompact / locally compact~$\rangle$ space is preserved by $\langle$~closed / closed / open and closed~$\rangle$ subspaces. Any non compact locally compact Hausdorff space $S$ can be regarded as dense subspace of some compact Hausdorff space. There is the smallest and the largest such compactification. The smallest one is called the Alexandroff compactification $\alpha S$.  By definition $\alpha S:=S\cup \{S\}$. A subset of $\alpha S$ is called open if it is an open subset of $S$ or has the form $\{S\}\cup S\setminus K$ for some compact set $K\subset S$. The largest compactification $\beta S$ is called the Stone-Cech compactification. It may be represented as the image of the embedding $j:S\to\prod_{f\in C}[0,1]:s\mapsto \prod_{f\in C}f(s)$, where $C$ is a set of all continuous maps from $S$ to $[0,1]$. Stone-Cech compactification is highly non constructive. Even $\beta\mathbb{N}$ has no explicit description, though it is known that $\beta\mathbb{N}$ is an extremely disconnected Hausdorff compact.

Occasionally we shall apply the Urysohn's lemma to locally compact Hausdorff spaces. It states that for any compact subset $K$ of open set $V$ in a locally compact Hausdorff space $S$ there exists a continuous function $f:S\to [0,1]$ such that $f|_K=1$ and $f|_{S\setminus V}=0$. 

For more details on topological spaces see comprehensive treatise \cite{EngelGenTop}. 

%----------------------------------------------------------------------------------------
%	Filters, nets and limits
%----------------------------------------------------------------------------------------

\subsection{Filters, nets and limits}

\label{SubSectionFiltersNetsAndLimits} 

We will use two generalizations of the notion of the sequence and the limit of the sequence.

A family $\mathfrak{F}$ of subsets of the set $M$ is called a filter on a set $M$ if $\mathfrak{F}$ doesn't contain the empty set, $\mathfrak{F}$ is closed under finite intersections and $\mathfrak{F}$ contains all supersets of its elements. In general filters are too large to be described explicitly. To overcome this difficulty we shall use filterbases. A non empty family $\mathfrak{B}$ of subset of a set $M$ is called a filterbase on a set $M$ if $\mathfrak{B}$ doesn't contain empty set and the intersection of any two elements of $\mathfrak{B}$ contains some element of $\mathfrak{B}$. Given a filterbase $\mathfrak{B}$ we can construct a filter by adding to $\mathfrak{B}$ all supersets of elements of $\mathfrak{B}$.

We say that filter $\mathfrak{F}_1$ dominates filter $\mathfrak{F}_2$ if $\mathfrak{F}_2\subset\mathfrak{F}_1$. Therefore the set of all filters on a given set is partially ordered set. Filters that are maximal with respect to this order are called ultrafilters. An easy application of Zorn's lemma gives that any filter is dominated by some ultrafilter.

Let $\mathfrak{F}$ be a filter on  a set $M$, and $\phi:M\to S$ be a map from $M$ to the Hausdorff topological space $S$. We say that $x$ is a limit of $\phi$ along $\mathfrak{F}$ and write $x=\lim_{\mathfrak{F}} \phi(m)$ if for every open neighborhood $U$ of $x$ holds $\phi^{-1}(U)\in\mathfrak{F}$. Directly from the definition it follows that if $\phi$ has a limit along $\mathfrak{F}$ then it has the same limit along any filter that dominates $\mathfrak{F}$. 

Limit along filter preserve order structure of $\mathbb{R}$. More precisely: if two functions $\phi:M\to\mathbb{R}$ and $\psi:M\to\mathbb{R}$ have limits along filter $\mathfrak{F}$ and $\phi\leq\psi$, then 
$$
\lim_{\mathfrak{F}}\phi(m)\leq\lim_{\mathfrak{F}}\psi(m).
$$

Limit along filter respects continuous functions. Rigorously this formulates as follows. Assume for each $\lambda\in\Lambda$ a function $\phi_\lambda:M\to S_\lambda$ has a limit along filter $\mathfrak{F}$, then for any continuous function $g:\prod_{\lambda\in\Lambda}S_\lambda\to Y$ holds
$$
\lim_{\mathfrak{F}}g\left(\prod_{\lambda\in\Lambda}\phi_\lambda(m)\right)=g\left(\prod_{\lambda\in\Lambda}\lim_{\mathfrak{F}}\phi_\lambda(m)\right).
$$
In particular limit along filter is linear and multiplicative. Just like ordinary sequences.

The most important feature of filters and the reason of our interest is the following: if $\mathfrak{U}$ is an ultrafilter on the set $M$ and $\phi:M\to K$ is a function with values in the compact Hausdorff space $K$, then $\lim_{\mathfrak{U}}\phi(m)$ exists. In particular, we always can speak of limits along ultrafilters of bounded scalar valued functions.

Another approach to the generalization of the notion of the limit is a limit of the net. A directed set is a partially ordered set $(N,\leq)$ in which any two elements have upper bound. Every directed set gives rise to the so called section filter, whose filterbase consist of so called sections $\{\nu':\nu\leq\nu'\}$ for some $\nu\in N$.
Any function $x:N\to X$ from the directed set $(N,\leq)$ into the topological space $X$ is called a net. Usually it is denoted as $(x_\nu)_{\nu\in N}$ to allude to sequences. A limit of the net $x:N\to X$ is a limit of the function $x$ along section filter of the directed set $N$. It is denoted $\lim_\nu x_\nu$. We shall exploit both notions of the limit.

More on this matters can be found in [\cite{BourbElemMathGenTopLivIII}, section 7].


%----------------------------------------------------------------------------------------
%	Measure theory basics
%----------------------------------------------------------------------------------------

\subsection{Measure theory}
\label{SubSectionMeasureTheory}

A family $\Sigma$ of subsets of the set $\Omega$ is called a $\sigma$-algebra if it contains an empty set, contains complements of all its elements and closed under countable unions. If $\Sigma$ is a $\sigma$-algebra of subsets of $\Omega$ we call $(\Omega,\Sigma)$ a measurable subspace. Elements of $\Sigma$ are called measurable sets. 

A function $\mu:\Sigma\to[0,+\infty]$ such that:  

$i)$ $\mu(\varnothing)=0$; 

$ii)$ $\mu\left(\bigcup\limits_{n\in\mathbb{N}} E_n\right)=\sum\limits_{n\in\mathbb{N}}\mu(E_n)$ for any family of disjoint sets $(E_n)_{n\in\mathbb{N}}$ in $\Sigma$; 

is called a measure. The triple $(\Omega,\Sigma,\mu)$ is called a measure space. If $\mu$ attains only finite values we may drop the first condition. The second condition is essential and called the $\sigma$-additivity. The simplest example of measure space is an  arbitrary set $\Lambda$ with $\sigma$-algebra of all subsets and so called counting measure $\mu_c:\mathcal{P}(\Lambda)\to[0,+\infty]$. By definition $\mu_c(E)$ equals $\operatorname{Card}(E)$ if $E$ is finite and $+\infty$ otherwise. If $E$ is a measurable set, by $\Sigma|_E$ we denote the $\sigma$-algebra $\{F\cap E:F\in\Sigma\}$ and by $\mu|_E$ we denote the restriction of $\mu$ to $\Sigma|_E$. A set $E$ in $\Omega$ is called negligible if there exists a measurable set $F$ of measure $0$ that contains $E$. Similarly, a set $E$ in $\Omega$ is conegligible if $\Omega\setminus E$ is negligible. Let $P$ be some property that depends on points of $\Omega$. We say that $P$ holds almost everywhere if the set where $P$ is violated is negligible. A measure space is called $\sigma$-finite if there exists a countable family of measurable sets of finite measure whose union is the whole space. The class of $\sigma$-finite spaces is enough for most applications but we shall encounter a more generic measure spaces.

A measurable space $(\Omega,\Sigma,\mu)$ is called strictly localizable if there exists a  family of disjoint measurable subsets $\{E_\lambda:\lambda\in\Lambda\}$ of finite measure such that: 

$i)$ $\bigcup_{\lambda\in\Lambda}E_\lambda=\Omega$;

$ii)$ $E$ is measurable iff $E\cap E_\lambda$ is measurable for all $\lambda\in\Lambda$;

$iii)$ for any measurable $E$ holds $\mu(E)=\sum_{\lambda\in\Lambda}\mu(E\cap E_\lambda)$. 
  
  The class of strictly localizable measure spaces is huge. It includes all $\sigma$-finite measure spaces, their arbitrary unions, Haar measures of locally compact groups, counting measures and much more. In what follows we shall consider only strictly localizable measure spaces.

We shall exploit a more detailed classification of measure spaces. We say that a measurable set $E$ is an atom if $\mu(E)>0$ and for any measurable subset $F$ of $E$ either $F$ or $E\setminus F$ is negligible. Directly from the definition it follows that that all atoms of strictly localizable measure spaces are of finite measure. In general an atom may not be a mere singleton.

We say that a measure space is non atomic if there is no atoms for its measure. A measure space is called purely atomic if every measurable set of positive measure contains an atom. A straightforward application of Zorn's lemma gives that a purely atomic measure space can be represented as disjoint union of some family of atoms. This family is countable if measure space is $\sigma$-finite. These facts allow us to say that the structure of purely atomic measure space is well understood. 

The structure of strictly localizable non atomic measure spaces is given by Maharam's theorem [\cite{FremMeasTh}, 332B]. We shall exploit only the following property of non atomic measures [\cite{FremMeasTh}, proposition 215D]: if $E$ is a measurable set of positive measure in a non atomic measure space, then for all $0\leq c\leq \mu(E)$ there exists a measurable subset $F$ of $E$ such that $\mu(F)=c$.

For completeness we shall say a few words on constructions with measures. The product measure of two measure spaces $(\Omega_1,\Sigma_1,\mu_1)$ and $(\Omega_2,\Sigma_2,\mu_2)$ we denote by $\mu_1\times \mu_2$. The definition of product measure for localizable measure spaces is rather involved [\cite{FremMeasTh}, definition 251F] and we don't give it here.  For our purposes it is enough to know that the product of two strictly localizable measure spaces is again strictly localizable [\cite{FremMeasTh}, proposition 251N]. By direct sum of measure spaces $\{(\Omega_\lambda, \Sigma_\lambda, \mu_\lambda):\lambda\in\Lambda\}$ we denote disjoint union of set $\{\Omega_\lambda:\lambda\in\Lambda\}$ with $\sigma$-algebra defined as $\Sigma=\{E\subset \Omega: E\cap E_\lambda\in\Sigma_\lambda\mbox{ for all }\lambda\in\Lambda\}$ and measure given by the formula $\mu(E)=\sum_{\lambda\in\Lambda}\mu_\lambda(E\cap E_\lambda)$. It is clear now that strictly localizable measure space are exactly direct sums of finite measure spaces.

Assume $(\Omega,\Sigma,\mu)$ is a $\sigma$-finite measure space, then there exists a purely atomic measure $\mu_1:\Sigma\to[0,+\infty]$ and a non atomic measure $\mu_2:\Sigma\to[0,+\infty]$ such that $\mu=\mu_1+\mu_2$. Even more there exist measurable sets $\Omega_a^{\mu}$ and $\Omega_{na}^{\mu}=\Omega\setminus \Omega_a^{\mu}$ such that $\mu_1(\Omega_{na}^{\mu})=\mu_2(\Omega_a^{\mu})=0$. The sets $\Omega_a^{\mu}$ and $\Omega_{na}^{\mu}$ are called respectively the atomic and the non atomic parts of measure space $(\Omega,\Sigma,\mu)$.

By measurable function we always mean a complex or real valued function on measurable space, with the property that preimage of every open set is measurable. We say that two measurable functions are equivalent if the set where they are different is negligible. If $f:\Omega\to\mathbb{R}$ is an  integrable function on $(\Omega,\Sigma,\mu)$, then we may define a new measure 
$$
f\mu:\Sigma\to[0,+\infty]:E\mapsto\int_{E}f(\omega)d\mu(\omega).
$$

The notion of measure can be extended by changing the range of values that the measure can attain. Any $\sigma$-additive function $\mu:\Sigma\to\mathbb{C}$ on a measurable space $(\Omega,\Sigma)$ is called a complex measure. Any complex measure $\mu$ can be represented as $\mu=\mu_1-\mu_2+i(\mu_3-\mu_4)$, where $\mu_1,\mu_2,\mu_3,\mu_4$ --- are finite measures. As the consequence every complex measure is finite and therefore we have a well defined total variation measure:
$$
|\mu|:\Sigma\to[0,+\infty):E\mapsto\sup\left\{\sum_{n\in\mathbb{N}}|\mu(E_n)|:\{E_n:n\in\mathbb{N}\}\subset\Sigma -\mbox{partition of }E\right\}
$$

Let $\mu$ and $\nu$ be two measures on a measurable space $(\Omega,\Sigma)$. We say that $\mu$ and $\nu$ are mutually singular and write $\mu\perp\nu$ if there exists a measurable set $E$ such that $\mu(E)=\nu(\Omega\setminus E)=0$. The opposite property is absolute continuity. We say that $\nu$ is absolutely continuous with respect to $\mu$ and write $\nu\ll\mu$ if $\nu(E)=0$ for every measurable set $E$ with $\mu(E)=0$. In general, two measures may neither be absolutely continuous nor singular with respect to each other. We have a Lebesgue decomposition theorem for this case. For a given two $\sigma$-finite measures $\mu$ and $\nu$ on a measurable space $(\Omega,\Sigma)$ there exists a measurable function $\rho_{\nu,\mu}:\Omega\to\mathbb{C}$, a $\sigma$-finite measure $\nu_s:\Sigma\to[0,+\infty]$ and two measurable sets $\Omega_s^{\nu,\mu}$, $\Omega_c^{\nu,\mu}=\Omega\setminus\Omega_s^{\nu,\mu}$ such that
$\nu=\rho_{\nu,\mu}\mu+\nu_s$ and $\mu(\Omega_s^{\nu,\mu})=\nu_s(\Omega_c^{\nu,\mu})=0$, i.e. $\mu\perp\nu_s$.

Finally we shall say a few words on measures defined on topological spaces. Given a topological space $S$ we may consider the minimal $\sigma$-algebra containing all open subsets of $S$. It is called the Borel $\sigma$-algebra of $S$ and denoted by $Bor(S)$. Measures and complex measures defined on Borel $\sigma$-algebras are supported with adjective Borel. We shall not consider measures on general topological spaces and stick with locally compact Hausdorff spaces. This significantly simplifies further considerations. We say that a complex Borel measure $\mu:Bor(S)\to\mathbb{C}$ defined on a locally compact Hausdorff space $S$ is regular if for any Borel set $E$ and any $\epsilon>0$ there exists a compact set $K\subset E$ such that $|\mu|(E\setminus K)<\epsilon$. The support of complex Borel measure $\mu$ is a set of all points $s\in S$ for which every open neighborhood of $s$ has positive measure. We denote the set of such points by $\operatorname{supp}(\mu)$. The support is always closed. 

Most of the results and definitions in this section can be found in the first, second and the fourth volumes of \cite{FremMeasTh}.

%----------------------------------------------------------------------------------------
%	Banach spaces, algebras and modules
%----------------------------------------------------------------------------------------

\section{Banach structures}

\label{SectionBanachStructures}

%----------------------------------------------------------------------------------------
%	Banach spaces
%----------------------------------------------------------------------------------------

\subsection{Banach spaces}
\label{SubSectionBanachSpaces}

We assume that our reader is familiar with fundamentals of functional analysis and its constructions, otherwise consult \cite{HelLectAndExOnFuncAn} or \cite{ConwACoursInFuncAn}. In this thesis we will highly rely on results about geometry of Banach spaces. See \cite{CarothShortCourseBanSp}, \cite{KalAlbTopicsBanSpTh} or \cite{FabHabBanSpTh} for a quick introduction. All Banach spaces are considered over complex field, unless otherwise stated. 

By $\langle$~$B_E$ / $B_E^\circ$~$\rangle$ we denote the $\langle$~closed / open~$\rangle$ unit ball of Banach space $E$ with center at zero. If $F$ is a closed subspace of $E$, then $E/F$ stands for the quotient Banach space. By $E^{cc}$ we denote a Banach space with the same set of vectors as in $E$, the same addition but with new multiplication by conjugate scalars:  $\alpha \overline{x}:=\overline{\overline{\alpha}x}$ for $\alpha\in\mathbb{C}$ and $x\in E$. Note: elements of $E^{cc}$ we denote by $\overline{x}$. Clearly, $(E^{cc})^{cc}=E$.

Now fix two Banach spaces $E$ and $F$. A map $T:E\to F$ is called conjugate linear if the respective map $T:E^{cc}\to F$ is linear. A linear operator $T:E\to F$ is called:

$i)$ bounded if its norm $\Vert T\Vert:=\sup\{\Vert T(x)\Vert:x\in B_E\}$ is finite.

$ii)$ contractive if its norm is at most $1$;

$iii)$ compact if $T(B_E)$ is relatively compact in $F$;

$iv)$ nuclear if it can be represented as an absolutely convergent series of rank one operators.

Is well known that any nuclear operator is compact. any compact operator is bounded, any bounded operator is continuous. By $\langle$~$\mathcal{B}(E,F)$ / $\mathcal{K}(E,F)$ / $\mathcal{N}(E,F)$~$\rangle$ we denote the Banach space of $\langle$~bounded / compact / nuclear~$\rangle$ linear operators from $E$ to $F$. If $F=E$ we use the shortcut $\langle$~$\mathcal{B}(E)$ / $\mathcal{K}(E)$ / $\mathcal{N}(E)$~$\rangle$ for this space. The norms in $\mathcal{B}(E,F)$ and $\mathcal{K}(E,F)$ are just the usual operator norm. The norm of a nuclear operator $T$ is defined by equality
$$
\Vert T\Vert:=\inf\left\{\sum_{n=1}^\infty\Vert S_n\Vert: T=\sum_{n=1}^\infty S_n,\quad (S_n)_{n\in\mathbb{N}} - \mbox{ rank one operators}\right\}.
$$

By $\mathbf{Ban}$ we shall denote the category of Banach spaces with bounded linear operators in the role of morphisms, while $\mathbf{Ban}_1$ stands for the category of Banach spaces with contractive operators in the role of morphisms. As the consequence, $\operatorname{Hom}_{\mathbf{Ban}}(E,F)$ is just another name for $\mathcal{B}(E,F)$.

Let $E$, $F$ and $G$ be three Banach spaces, then a bilinear operator $\Phi:E\times F\to G$ is called bounded if its norm $\Vert \Phi\Vert:=\sup\{\Vert \Phi(x,y)\Vert:x\in B_E, y\in B_F\}$ is finite. The Banach space of all bounded bilinear operators on $E\times F$ with values in $G$ is denoted by $\mathcal{B}(E\times F,G)$.

A few words on classification of bounded linear operators. A bounded linear operator $T:E\to F$ is called:

$i)$ topologically injective if it performs homeomorphism on its image;

$ii)$ topologically surjective if it is an open map;

$iii)$ coisometric if it maps open unit ball onto open unit ball;

$iv)$ strictly coisometric if it maps closed unit ball onto closed unit ball. 

But we shall refine these definitions. We say a bounded linear operator $T:E\to F$ is:

$v)$ $c$-topologically injective, if $\Vert x\Vert\leq c\Vert  T(x)\Vert$ for all $x\in E$;

$vi)$ $c$-topologically surjective, if $cT(B_E^\circ)\supset B_F^\circ$;

$vii)$ strictly $c$-topologically surjective, if $cT(B_E)\supset B_F$; 

Note that $T$ is topologically $\langle$~injective / surjective~$\rangle$ iff it is $c$-topologically $\langle$~injective / surjective~$\rangle$ for some $c>0$. Obviously $\langle$~coisometric / strictly coisometric~$\rangle$ operators are exactly contractive $\langle$~$1$-topologically surjective / strictly $1$-topologically surjective~$\rangle$ operators.

Two Banach spaces $E$ and $F$ are $\langle$~isometrically isomorphic / topologically isomorphic~$\rangle$ as Banach spaces if there exists a bounded linear operator $T:E\to F$ which is both $\langle$~isometric and surjective / topologically injective and topologically surjective~$\rangle$. The fact that $E$ and $F$ are $\langle$~isometrically isomorphic / topologically isomorphic~$\rangle$ Banach spaces means that $\langle$~$E\isom{\mathbf{Ban}_1}F$ / $E\isom{\mathbf{Ban}}F$~$\rangle$. The Banach-Mazur distance between $E$ and $F$ is defined by the formula 
$$
d_{BM}(E,F):=\inf\{\Vert T\Vert\Vert T^{-1}\Vert: T \in \mathcal{B}(E,F) \mbox{ --- a topological isomorphism}\}.
$$ 
If $E$ and $F$ are not topologically isomorphic the Banach-Mazur distance between them is infinite.

One more important class of operators is the class of bounded projections. A bounded linear operator $P:E\to E$ is called a projection if $P^2=P$. If $F=P(E)$, then we say that $P$ is a projection from $E$ onto $F$ and $F$ is complemented in $E$. If $\Vert P\Vert\leq c$ we say that $F$ is $c$-complemented in $E$. Finally, we say that $F$ is contractively complemented in $E$ if it is $1$-complemented in $E$. Another equivalent characterization says that $F$ is a complemented subspace of Banach space $E$ if there exists a closed subspace $G$ in $E$ such that $E\isom{\mathbf{Ban}}F\bigoplus G$. All finite dimensional subspaces are complemented, but not necessarily contractively complemented. An example of contractively complemented subspace is the following: consider arbitrary Banach space $E$, then $E^*$ is contractively complemented in $E^{***}$ via Dixmier projection $P=\iota_{E^*}(\iota_E)^*$, where $\iota_E$ is the natural embedding of $E$ into its second dual $E^{**}$. A canonical example of uncomplemented subspace is $c_0(\mathbb{N})$ in $\mathbb{\ell_\infty}(\mathbb{N})$  [\cite{KalAlbTopicsBanSpTh}, theorem 2.5.5].

Now consider the algebraic tensor product $E\otimes F$ of Banach spaces $E$ and $F$. This linear space can be endowed with different norms, but the most important is the projective norm. For $u\in E\otimes F$ we define its projective norm as
$$
\Vert u\Vert:=\inf\left\{\sum_{i=1}^n \Vert x_i\Vert\Vert y_i\Vert: u=\sum_{i=1}^n x_i\otimes y_i, (x_i)_{i\in\mathbb{N}_n}\subset E, (y_i)_{i\in\mathbb{N}_n}\subset F\right\}
$$
It is indeed a norm, but not complete in general. The symbol $E\projtens F$ stands for the completion of $E\otimes F$ under projective norm. We call the resulting completion the projective tensor product of Banach spaces $E$ and $F$. Let $T:E_1\to E_2$ and $S:F_1\to F_2$ be two bounded linear operators between Banach spaces, then there exists a unique bounded linear operator $T\projtens S:E_1\projtens F_1\to E_2\projtens F_2$ such that $(T\projtens S)(x\projtens y)=T(x)\projtens S(y)$ for all $x\in E_1$ and $y\in F_1$. Even more $\Vert T\projtens S\Vert=\Vert T\Vert\Vert S\Vert$. The main feature of projective tensor product which makes it so important is the following universal property: for any Banach spaces $E$, $F$ and $G$ there is a natural isometric isomorphism:
$$
\mathcal{B}(E\projtens F,G)\isom{\mathbf{Ban}_1}\mathcal{B}(E\times F,G)
$$
In other words, projective tensor product linearizes bounded bilinear operators. Also we have the following two (natural in $E$, $F$ and $G$) isometric isomorphisms:
$$
\mathcal{B}(E\projtens F,G)
\isom{\mathbf{Ban}_1}\mathcal{B}(E,\mathcal{B}(F,G))
\isom{\mathbf{Ban}_1}\mathcal{B}(F,\mathcal{B}(E,G))
$$
The last isomorphism is called the law of adjoint associativity. There are many other tensor norms on the algebraic tensor product of Banach spaces. Their thorough treatment can be found in \cite{DiestMetTheoryOfTensProd}.

Now we are able to craft four very important functors:
$$
\mathcal{B}(-,E):\mathbf{Ban}\to\mathbf{Ban}
\qquad\qquad
\mathcal{B}(E,-):\mathbf{Ban}\to\mathbf{Ban}
$$
$$
-\projtens E:\mathbf{Ban}\to\mathbf{Ban}
\qquad\qquad
E\projtens -:\mathbf{Ban}\to\mathbf{Ban}.
$$
We shall often encounter them. For example, the well known adjoint functor ${}^*$ is nothing more than $\mathcal{B}(-,\mathbb{C})$. All these functors have their obvious analogs on $\mathbf{Ban}_1$.

Now we  proceed to classical examples of Banach spaces. 

An important source of examples of Banach spaces are $L_p$-spaces, also known as Lebesgue spaces. A detailed discussion of basic properties of $L_p$-spaces can be found in \cite{CarothShortCourseBanSp}.  Let $(\Omega,\Sigma,\mu)$ be a measure space.  For $1\leq p<\infty$, as usually, the symbol $L_p(\Omega,\mu)$ stands for the Banach space of equivalence classes of functions $f:\Omega\to\mathbb{C}$ such that $|f|^p$ is Lebesgue integrable with respect to measure $\mu$. The norm of such function is defined as
$$
\Vert f\Vert:=\left(\int\limits_{\Omega}|f(\omega)|^pd\mu(\omega)\right)^{1/p}.
$$ 
By $L_\infty(\Omega,\mu)$ we denote the Banach space of equivalence classes of bounded measurable functions with norm defined as 
$$
\Vert f\Vert:=\inf\left\{\sup_{\omega\in\Omega\setminus N}|f(\omega)|:N\subset\Omega - \mbox{is negligible}\right\}.
$$
For simplicity we shall speak of functions in $L_p(\Omega,\mu)$ instead of their equivalence classes. All equalities and inequalities about functions of $L_p$-spaces are understood up to negligible sets. It is well know that $L_p(\Omega,\mu)^*\isom{\mathbf{Ban}_1}L_{p^*}(\Omega,\mu)$ for $1\leq p<+\infty$ [\cite{FremMeasTh}, theorems 243G, 244K]. One more well known fact is that, $L_p$-spaces are reflexive for $1<p<+\infty$. Here we exploited the standard notation $p^*=+\infty$ if $p=1$ and $p^*=p/(p-1)$ if $1<p<+\infty$. Clearly, $p^{**}=p$ for $1<p<+\infty$.

The most well known classes of Banach spaces are related to continuous functions. Let $S$ be a locally compact Hausdorff space. We say that a function $f:S\to\mathbb{C}$ vanishes at infinity  if for any $\epsilon>0$ there exists a compact $K\subset S$ such that $|f(s)|\leq\epsilon$ for all $s\in S\setminus K$. The linear space of continuous functions on $S$ vanishing at infinity is denoted by $C_0(S)$. When endowed with $\sup$-norm $C_0(S)$ becomes a Banach space. Any set $\Lambda$ with discrete topology may be regarded as a locally compact space and following the traditional notation we shall write $c_0(\Lambda)$ instead of $C_0(\Lambda)$. If $K$ is a compact Hausdorff space then all functions on $K$ vanish at infinity. We use the notation $C(K)$ for $C_0(K)$ to indicate that $K$ is compact. Some Banach spaces in fact are $C(K)$-spaces in disguise. For example, if we are given a measure space $(\Omega,\Sigma,\mu)$, then $B(\Omega,\Sigma)$ --- the space of bounded measurable functions with $\sup$-norm or $L_\infty(\Omega,\mu)$ are $C(K)$ spaces for some compact Hausdorff space $K$ [\cite{KalAlbTopicsBanSpTh}, remark 4.2.8]. If $S$ is a locally compact Hausdorff space, then $M(S)$ stands for the Banach space of complex finite Borel regular measures on $S$. The norm of measure $\mu\in M(S)$ is defined by equality $\Vert\mu\Vert=|\mu|(S)$, where $|\mu|$ is a total variation measure of measure $\mu$. By Riesz-Markov-Kakutani theorem  [\cite{ConwACoursInFuncAn}, section C.18] we have $C_0(S)^*\isom{\mathbf{Ban}_1}M(S)$. In fact $M(S)$ is an $L_1$-space, see discussion after [\cite{DalLauSecondDualOfMeasAlg}, proposition 2.14]. 

We shall also mention one important specific case of $L_p$-spaces. For a given index set $\Lambda$ and a counting measure $\mu_c:\mathcal{P}(\Lambda)\to[0,+\infty]$ the respective $L_p$-space is denoted by $\ell_p(\Lambda)$. For this type of measure spaces we have one more important isomorphism $c_0(\Lambda)^*\isom{\mathbf{Ban}_1}\ell_1(\Lambda)$. For convenience we define $c_0(\varnothing)=\ell_p(\varnothing)=\{0\}$ for $1\leq p\leq+\infty$. This example motivates the following construction.

Let $\{E_\lambda:\lambda\in\Lambda\}$ be an arbitrary family of Banach spaces. For each $x\in \prod_{\lambda\in\Lambda} E_\lambda$ we define
$\Vert x\Vert_p=\Vert(\Vert x_\lambda\Vert)_{\lambda\in\Lambda}\Vert_{\ell_p(\Lambda)}$ for $1\leq p\leq +\infty$ and $\Vert x\Vert_0=\Vert(\Vert x_\lambda\Vert)_{\lambda\in\Lambda}\Vert_{c_0(\Lambda)}$. Then the Banach space $\left\{x\in \prod_{\lambda\in\Lambda} E_\lambda: \Vert x\Vert_p<+\infty\right\}$ with the norm $\Vert\cdot\Vert_p$ is denoted by $\bigoplus_p\{E_\lambda:\lambda\in\Lambda\}$. We call these objects $\bigoplus_p$-sums of Banach spaces $\{E_\lambda:\lambda\in\Lambda\}$. It is almost tautological that the Banach space $\ell_p(\Lambda)$ is the $\bigoplus_p$-sum of the family $\{\mathbb{C}:\lambda\in\Lambda\}$. A nice property of $\bigoplus_p$-sums is their interrelation with duality:
$$
\left(\bigoplus\nolimits_p\{E_\lambda:\lambda\in\Lambda\}\right)^*\isom{\mathbf{Ban}_1}
\bigoplus\nolimits_{p^*}\{E_\lambda^*:\lambda\in\Lambda\}
$$
for all $1\leq p<+\infty$ and 
$$
\left(\bigoplus\nolimits_0\{E_\lambda:\lambda\in\Lambda\}\right)^*\isom{\mathbf{Ban}_1}
\bigoplus\nolimits_1\{E_\lambda^*:\lambda\in\Lambda\}
$$
If $\{T_\lambda\in\mathcal{B}(E_\lambda, F_\lambda):\lambda\in\Lambda\}$ is a family of bounded linear operators, then for all $1\leq p\leq+\infty$ and $p=0$ we have a well defined linear operator
$$
T:\bigoplus\nolimits_p\{E_\lambda:\lambda\in\Lambda\}\to \bigoplus\nolimits_p\{ F_\lambda:\lambda\in\Lambda\}:x\mapsto \bigoplus\nolimits_p\{ T_\lambda(x_\lambda):\lambda\in\Lambda\}
$$
which we shall denote by $\bigoplus_p\{T_\lambda:\lambda\in\Lambda\}$. Its norm equals $\sup_{\lambda\in\Lambda}\Vert T_\lambda\Vert$.

Among different $\bigoplus_p$-sums the $\langle$~$\bigoplus_1$-sums / $\bigoplus_\infty$-sums~$\rangle$ play a special role in Banach space theory. The reason is that any family of Banach spaces admit $\langle$~product / coproduct~$\rangle$ in $\mathbf{Ban}_1$ which in fact is their $\langle$~$\bigoplus_1$-sum / $\bigoplus_\infty$-sum~$\rangle$. The same statement holds for $\mathbf{Ban}$ if we restrict ourselves to finite families of objects [\cite{HelLectAndExOnFuncAn}, chapter 2, section 5].

We proceed to advanced topics of Banach space theory. Below we shall discuss several geometric properties of Banach spaces such as the property of being an $\mathscr{L}_p^g$-space, weak sequential completeness, the Dunford-Pettis property, the l.u.st. property and the approximation property. In what follows, imitating Banach space geometers, we shall say that a Banach space $E$ contains $\langle$~an isometric copy / a copy~$\rangle$ of Banach space $F$ if $F$ is $\langle$~isometrically isomorphic / topologically isomorphic~$\rangle$ to some closed subspace of $E$.

Let $1\leq p\leq +\infty$. We say that $E$ is an $\mathscr{L}_{p,C}^g$-space if for any $\epsilon>0$ and any finite dimensional subspace $F$ of $E$ there exists a finite dimensional $\ell_p$-space $G$ and two bounded linear operators $S:F\to G$, $T:G\to E$ such that $TS|^F=1_F$ and $\Vert T\Vert\Vert S\Vert\leq C+\epsilon$. If $E$ is an $\mathscr{L}_{p,C}^g$-space for some $C\geq 1$ we simply say, that $E$ is an $\mathscr{L}_p^g$-space. This definition [\cite{DefFloTensNorOpId}, definition 23.1] is an improvement of the definition of $\mathscr{L}_p$-spaces given by Lindenstrauss and Pelczynski in their pioneering work \cite{LinPelAbsSumOpInLpSpAndApp}. Clearly, any finite dimensional Banach space is an $\mathscr{L}_p^g$-space for all $1\leq p\leq +\infty$. Any $L_p$-space is an $\mathscr{L}_{p,1}^g$-space  [\cite{DefFloTensNorOpId}, exercise 4.7], but the converse is not true. Any $c$-complemented subspace of $\mathscr{L}_{p,C}^g$-space is an $\mathscr{L}_{p,cC}^g$-space [\cite{DefFloTensNorOpId}, corollary 23.2.1(2)]. A Banach space is an $\mathscr{L}_{p,C}^g$-space iff its dual is an $\mathscr{L}_{p^*,C}^g$-space [\cite{DefFloTensNorOpId}, corollary 23.2.1(1)]. All $C(K)$-spaces are $\mathscr{L}_{\infty, 1}^g$-spaces [\cite{DefFloTensNorOpId}, lemma 4.4]. Note that, for a given locally compact Hausdorff space $S$ the Banach space $C_0(S)$ is complemented in $C(\alpha S)$. Therefore $C_0(S)$-spaces are $\mathscr{L}_\infty^g$-spaces too. We will mainly concern in $\mathscr{L}_1^g$- and $\mathscr{L}_\infty^g$-spaces.

We say that a Banach space $E$ is weakly sequentially complete if for any sequence $(x_n)_{n\in\mathbb{N}}\subset E$ such that $(f(x_n))_{n\in\mathbb{N}}\subset\mathbb{C}$ is a Cauchy sequence for any $f\in E^*$ there exists a vector $x\in E$ such that $\lim_n f(x_n)=f(x)$ for all $f\in E^*$. That is any weakly Cauchy sequence converges in the weak topology. A typical example of weakly sequentially complete Banach space is any $L_1$-space [\cite{WojBanSpForAnalysts}, corollary III.C.14]. This property is preserved by closed subspaces. A typical example of Banach space that is not weakly sequentially complete is $c_0(\mathbb{N})$, just consider the sequence $(\sum_{k=1}^n \delta_k)_{n\in\mathbb{N}}$.

Now we proceed to the discussion of the Dunford-Pettis property. A bounded linear operator $T:E\to F$ is called weakly compact if it maps the unit ball of $E$ into a relatively weakly compact subset of $F$. A bounded linear operator is called completely continuous if the image of any weakly compact subset of $E$ is norm compact in $F$. A Banach space $E$ is said to have the Dunford-Pettis property if any weakly compact operator from $E$ to any Banach space $F$ is completely continuous. There is a simple internal characterization [\cite{KalAlbTopicsBanSpTh}, theorem 5.4.4]: a Banach space $E$ has the Dunford-Pettis property if $\lim_n f_n(x_n)=0$ for all sequences $(x_n)_{n\in\mathbb{N}}\subset E$ and $(f_n)_{n\in\mathbb{N}}\subset E^*$, that both weakly converge to $0$. Now it is easy to deduce, that if a Banach space $E^*$ has the Dunford-Pettis property, then so does $E$. In his seminal work \cite{GrothApllFaiblCompSpCK} Grothendieck showed that all $L_1$-spaces and $C(K)$-spaces have this property. The Dunford-Pettis property passes to complemented subspaces [\cite{FabHabBanSpTh}, proposition 13.44]. This property behaves badly with reflexive spaces: since the unit ball of a reflexive space is weakly compact [\cite{MeggIntroBanSpTh}, theorem 2.8.2], then reflexive Banach space with the Dunford-Pettis property has norm compact unit ball and therefore this space is finite dimensional. 

To introduce the next Banach geometric property we need definitions of Banach lattice and unconditional Schauder basis. 

A real Riesz space $E$ is a vector space over $\mathbb{R}$ with the structure of partially ordered set such that $x\leq y$ implies $x+z\leq y+z$ for every $x,y,z\in E$ and $ax\geq 0$ for every $x\geq 0$, $a\in\mathbb{R}_+$. A partially ordered set is a lattice if any two elements ${x,y}$ have the least upper bound $x\vee y$ and the greatest lower bound $x\wedge y$. A real vector lattice is real Riesz space which is lattice as partially ordered set. If $E$ is a real vector lattice, then for every $x\in E$ we define its absolute value by equality $|x|:=x\vee(-x)$. A complex vector lattice $E$ is a vector space over $\mathbb{C}$ such that there exists a real vector subspace $\operatorname{Re}(E)$ which is real vector lattice and

$i)$ for any $x\in E$ there are unique $\operatorname{Re}(x),\operatorname{Im}(x)\in \operatorname{Re}(E)$ such that $x=\operatorname{Re}(x)+i\operatorname{Im}(x)$;

$ii)$ for any $x\in E$ there exist an absolute value $|x|:=\sup\{\operatorname{Re}(e^{i\theta}x):\theta\in\mathbb{R}\}$.

A Banach lattice is a Banach space with the structure of the complex vector lattice such that $\Vert x\Vert\leq \Vert y\Vert$ whenever $|x|\leq |y|$. A classical example of Banach lattice $E$ is an $L_p$-space or a $C(K)$-space. In both cases $\operatorname{Re}(E)$ consist of real valued functions in $E$. If $\{E_\lambda:\lambda\in\Lambda\}$ is a family of Banach lattices then for any $1\leq p\leq +\infty$ or $p=0$ their $\bigoplus_p$-sum is a Banach lattice with lattice operation defined as $x\leq y$ if $x_\lambda\leq y_\lambda$ for all $\lambda\in\Lambda$, where $x,y\in\bigoplus_p\{ E_\lambda:\lambda\in\Lambda\}$. The dual space $E^*$ of a Banach lattice $E$ is again a Banach lattice with lattice operation defined by $f\leq g$ if $f(x)\leq g(x)$ for all $x\geq 0$, where $f,g\in  E^*$. A very nice account of Banach lattices can be found in [\cite{LaceyIsomThOfClassicBanSp}, section 1].

The property of being a Banach lattice puts some restrictions on the geometry of the space \cite{SherOrderInOpAlg}, \cite{KadOrderPropOfBoundSAOps}. To explain the Banach geometric reason of this phenomena we need the definition of an unconditional Schauder basis. Let $E$ be a Banach space. A collection of functionals $(f_\lambda)_{\lambda\in\Lambda}$ in $E^*$ is called a biorthogonal system for vectors $(x_\lambda)_{\lambda\in\Lambda}$  from $E$ if $f_\lambda(x_{\lambda'})=1$ whenever $\lambda=\lambda'$ and $0$ otherwise. A collection $(x_\lambda)_{\lambda\in\Lambda}$ in $E$ is called an unconditional Schauder basis if there exists a biorthogonal system $(f_\lambda)_{\lambda\in\Lambda}$ in $E^*$ for it such that
the series $\sum_{\lambda\in\Lambda} f_\lambda(x)x_\lambda$ unconditionally converges to $x$ for any $x\in E$. All $\ell_p$-spaces with $1\leq p<+\infty$ have an unconditional Schauder basis, for example, it is $(\delta_\lambda)_{\lambda\in\Lambda}$. A typical example of space without unconditional basis is $C([0,1])$. Even more this Banach space can not even be a subspace of the space with unconditional basis [\cite{KalAlbTopicsBanSpTh}, proposition 3.5.4].  Any unconditional Schauder basis $(x_\lambda)_{\lambda\in\Lambda}$ in $E$ satisfy the following property  [\cite{DiestAbsSumOps}, proposition 1.6]: there exists a constant $\kappa\geq 1$ such that
$$
\left\Vert \sum_{\lambda\in\Lambda}t_\lambda f_\lambda(x)x_\lambda\right\Vert
\leq
\kappa\left\Vert \sum_{\lambda\in\Lambda}f_\lambda(x)x_\lambda\right\Vert
$$
for all $x\in E$ and $t\in\ell_\infty(\Lambda)$. The least such constant $\kappa$ among all unconditional Schauder bases of $E$ is denoted by $\kappa(E)$. Similar constant could be defined for Banach spaces without unconditional Schauder bases. The local unconditional constant $\kappa_u(E)$ of Banach space $E$ is defined to be the infimum of all scalars $c$ with the following property: given any finite dimensional subspace $F$ of $E$ there exists a Banach space $G$ with unconditional Schauder basis and two bounded linear operators $S:F\to G$, $T:G\to E$ such that $TS|^{F}=1_F$ and $\Vert T\Vert\Vert S\Vert\kappa(G)\leq c$. We say that a Banach space $E$ has the local unconditional structure property (the l.u.st. property for short) if $\kappa_u(E)$ is finite. Clearly any Banach space $E$ with unconditional Schauder basis has the l.u.st. property with $\kappa_u(E)=\kappa(E)$. In particular, all finite dimensional Banach spaces have the l.u.st. property. Though a general Banach lattice $E$ may not have an unconditional Schauder basis it is still has the l.u.st. property with $\kappa_u(E)=1$  [\cite{DiestAbsSumOps}, theorem 17.1]. Directly from the definition it follows that the l.u.st. property is preserved by complemented subspaces. More precisely: if $F$ is a $c$-complemented subspace of $E$, then $\kappa_u(F)\leq c\kappa_u(E)$. Therefore all complemented subspaces of Banach lattices have the l.u.st. property. This sufficient condition is not far from criterion [\cite{DiestAbsSumOps}, theorem 17.5]: a Banach space $E$ has the l.u.st. property iff $E^{**}$ is topologically isomorphic to a complemented subspace of some Banach lattice. As the corollary of this criterion we get that $E$ has the l.u.st. property iff so does $E^*$ [\cite{DiestAbsSumOps}, corollary 17.6].

The last property we shall discuss is a well known approximation property introduced by Grothendieck in \cite{GrothProdTenTopNucl}. We say that a Banach space $E$ has the approximation property if for any compact set $K\subset E$ and any $\epsilon>0$ there exists a finite rank operator $T:E\to E$ such that $\Vert T(x)-x\Vert<\epsilon$ for all $x\in K$. If $T$ can be chosen with $\Vert T\Vert\leq c$, then $E$ is said to have the $c$-bounded approximation property. The metric approximation property is another name for $1$-bounded approximation property. We say that $E$ has the bounded approximation property if $E$ has the $c$-bounded approximation property for some $c\geq 1$. None of these properties are preserved by subspaces or quotient spaces, but the approximation property and the bounded approximation properties are inherited by complemented subspaces [\cite{DefFloTensNorOpId}, exercise 5.5]. All $L_p$-spaces and $C(K)$-spaces have the metric approximation property [\cite{DefFloTensNorOpId}, section 5.2(3)], but their subspaces may fail the approximation property [\cite{DefFloTensNorOpId}, section 5.2(1)]. Any Banach space with unconditional Schauder basis has the approximation property [\cite{RyanIntroTensNormsBanSp}, example 4.4]. If $E^*$ has the approximation property, then so does $E$ [\cite{DefFloTensNorOpId}, corollary 5.7.2]. The reason why approximation property is so important is rather simple --- it has a lot of equivalent reformulations that involve many nice properties of Banach spaces. For example, the following properties of Banach space $E$ are equivalent [\cite{DefFloTensNorOpId}, sections 5.3, 5.6]:

$i)$ $E$ has the approximation property;

$ii)$ the natural mapping $Gr:E^*\projtens E\to\mathcal{N}(E)$ is an isometric isomorphism;

$iii)$ for any Banach space $F$ every compact operator $T:F\to E$ can be approximated in the operator norm by finite rank operators.

There is much more to list, but we confine ourselves with these three properties.

%----------------------------------------------------------------------------------------
%	Banach algebras and their modules
%----------------------------------------------------------------------------------------

\subsection{Banach algebras and their modules}
\label{SubSectionBanachAlgebrasAndTheirModules}

A thorough treatment of Banach algebras and Banach modules can be found in \cite{HelBanLocConvAlg} or \cite{HelHomolBanTopAlg} or \cite{DalBanAlgAutCont}. We shall describe only the bare minimum required for us.

A Banach algebra $A$ is an associative algebra over $\mathbb{C}$ which is a Banach space and the multiplication bilinear operator $\cdot:A\times A\to A:(a,b)\mapsto ab$ is of the norm at most $1$. A typical example of commutative Banach algebra is the algebra of continuous functions on a compact Hausdorff space with pointwise multiplication. A typical non commutative example is the algebra of bounded linear operators on the Hilbert space with composition in the role of multiplication. Both examples belong to a very important class of $C^*$-algebras to be discussed below. By $\langle$~left / right / two-sided~$\rangle$ ideal $I$ of a Banach algebra $A$ we always mean a closed subalgebra of $A$ such that $\langle$~$ax$ / $xa$ / $ax$ and $xa$~$\rangle$ belong to $I$ for all $a\in A$ and $x\in I$.

We say that an element $p$ of a Banach algebra $A$ is a $\langle$~left / right~$\rangle$ identity of $A$ if $\langle$~$pa=a$ / $ap=a$~$\rangle$ for all $a\in A$. The element which is both left and right identity is called the identity and denoted by $e_A$. In general we do not assume that Banach algebras are unital, i.e. has an identity. Even if a Banach algebra $A$ is unital we do not require its identity to be of norm $1$. We use notation $A_+=A\bigoplus_1\mathbb{C}$ for the standard unitization of Banach algebras. The multiplication in $A_+$ is defined as $(a\oplus_1 z)(b\oplus_1 w)=(ab+wa+zb)\oplus_1 zw$, for $a,b\in A$ and $z,w\in\mathbb{C}$. Clearly $(0,1)$ is the identity of $A_+$. By $A_\times$ we denote the conditional unitization of $A$, i.e. $A_\times=A$ if $A$ has identity of norm one and $A_\times=A_+$ otherwise. Even in the absence of identity in case of Banach algebras there are good substitutes for it which are called approximate identities. We say that a net $(e_\nu)_{\nu\in N}$ in $A$ is a $\langle$~left / right / two-sided~$\rangle$ approximate identity if $\langle$~$\lim_\nu e_\nu a=a$ / $\lim_\nu ae_\nu=a$ / $\lim_\nu e_\nu a=\lim_\nu ae_\nu=a$~$\rangle$ for all $a\in A$. In all these three definitions convergence of nets is understood in the norm topology. If we will consider weak topology, we shall get definitions of left, right and two-sided weak approximate identities. We say that an approximate identity $(e_\nu)_{\nu\in N}$ is $c$-bounded if $\sup_\nu\Vert e_\nu\Vert\leq c$. An approximate identity $(e_\nu)_{\nu\in N}$ is called $\langle$~bounded / contractive~$\rangle$ if it is $\langle$~$1$-bounded / $c$-bounded for some $c\geq 1$~$\rangle$. Occasionally we will use the following simple fact: if $A$ is a Banach algebra with $\langle$~left / right~$\rangle$ identity $p$ and $\langle$~right / left~$\rangle$ approximate identity $(e_\nu)_{\nu\in N}$, then $A$ is unital with identity $p$ of norm $\lim_\nu\Vert e_\nu\Vert$. 

If $A$ is a unital Banach algebra we define the spectrum $\operatorname{sp}_A(a)$ of element $a$ in $A$ as the set of all complex numbers $z$ such that $a-ze_A$ is not invertible in $A$. For Banach algebras the spectrum of any element is a non empty compact subset of $\mathbb{C}$ [\cite{HelBanLocConvAlg}, corollary 2.1.16].

A character on a Banach algebra $A$ is a non zero linear homomorphism $\varkappa:A\to\mathbb{C}$. All characters are continuous and are contained in the unit ball of $A^*$  [\cite{HelBanLocConvAlg}, theorem 1.2.6]. Therefore we may consider the set of characters with the induced weak$^*$ topology. The resulting topological space is Hausdorff and locally compact. It is called the spectrum of Banach algebra $A$ and denoted by $\operatorname{Spec}(A)$. If $A$ is unital then its spectrum is compact [\cite{HelBanLocConvAlg}, theorem 1.2.50]. Now for a given Banach algebra $A$ with non empty spectrum we can construct a contractive homomorphism $\Gamma_A:A\to C_0(\operatorname{Spec}(A)):a\mapsto(\varkappa\mapsto \varkappa(a))$ called the Gelfand transform of $A$ [\cite{HelBanLocConvAlg}, theorem 4.2.11]. The kernel of this homomorphism is called the Jacobson's radical and denoted by $\operatorname{Rad}(A)$. For Banach algebra $A$ with empty spectrum we define $\operatorname{Rad}(A)=A$. If $\operatorname{Rad}(A)=\{0\}$, then $A$ is called semisimple. By Shilov's idempotent theorem [\cite{KaniBanAlg}, section 3.5] any semisimple Banach algebra with compact spectrum is unital.

Most of standard constructions for Banach spaces have their counterparts for Banach algebras. For example $\bigoplus_p$-sum of Banach algebras endowed with componentwise multiplication is a Banach algebra, the quotient of a given Banach algebra by its two sided ideal is a Banach algebra. Even the projective tensor product of two Banach algebras is a Banach algebra with multiplication defined on elementary tensors the same way as in pure algebra.

We shall proceed to the discussion of the most important class of Banach algebras. Let $A$ be an associative algebra over $\mathbb{C}$, then a conjugate linear operator ${}^*:A\to A$ is called an involution if $(ab)^*=b^*a^*$ and $a^{**}=a$ for all $a,b\in A$. Algebras with involution are called ${}^*$-algebras. Homomorphisms between ${}^*$-algebras that preserve involution are called ${}^*$-homomorphisms. A Banach algebra with isometric involution is called a ${}^*$-Banach algebra. An example of such algebra is the Banach algebra of bounded linear operators on Hilbert space with operation of taking the Hilbert adjoint operator in the role of involution. In fact there is much more to this algebra than one could expect. We say that a ${}^*$-Banach algebra $A$ is a $C^*$-algebra if it satisfies $\Vert a^*a\Vert=\Vert a\Vert^2$ for all $a\in A$. One of the biggest advantages of $C^*$-algebras is their celebrated representation theorems by Gelfand and Naimark. The first representation theorem [\cite{HelBanLocConvAlg}, theorem 4.7.13] states that any commutative $C^*$-algebra $A$ is isometrically isomorphic as ${}^*$-algebra to $C_0(\operatorname{Spec}(A))$. The second theorem [\cite{HelBanLocConvAlg}, theorem 4.7.57] gives a description of generic $C^*$-algebras as closed ${}^*$-Banach subalgebras of $\mathcal{B}(H)$ for some Hilbert space $H$. Such representation is not unique, but a norm (if it exists) that turn a ${}^*$-algebra into a $C^*$-algebra is always unique. If a ${}^*$-subalgebra of $\mathcal{B}(H)$ is weak${}^*$ closed it is called a von Neumann algebra. If a $C^*$-algebra is isomorphic as ${}^*$-algebra to a von Neumann algebra it is called a $W^*$-algebra. By well known Sakai's theorem [\cite{BlackadarOpAlg}, theorem III.2.4.2] each $C^*$-algebra which is dual as Banach space is a $W^*$-algebra, but beware a $W^*$-algebra may be represented as non weak${}^*$ closed ${}^*$-subalgebra in $\mathcal{B}(H)$ for some Hilbert space $H$. 

A lot of standard constructions pass to $C^*$-algebras from Banach algebras, but not all. For example a $\bigoplus_\infty$-sum of $C^*$-algebras is again a $C^*$-algebra. A quotient of $C^*$-algebra by closed two-sided ideal is a $C^*$-algebra too. Meanwhile the projective tensor product of $C^*$-algebras is rarely a $C^*$-algebra, though there a lot of norms that may turn their algebraic tensor product into a $C^*$-algebra. In this thesis we shall exploit one specific and highly
important for $C^*$-algebras construction of matrix algebras. For a given $C^*$-algebra $A$ by $M_n(A)$ we denote the linear space of $n\times n$ matrices with entries in $A$. In fact $M_n(A)$ is ${}^*$-algebra with involution and multiplication defined by equalities 
$$
(ab)_{i,j}=\sum_{k=1}^n a_{i,k}b_{k,j}
\qquad\qquad
(a^*)_{i,j}=(a_{j,i}^*)
$$ 
for all $a,b\in M_n(A)$ and $i,j\in\mathbb{N}_n$. There is a unique norm on $M_n(A)$ that makes it a $C^*$-algebra [\cite{MurphyCStarAlgsAndOpTh}, theorem 3.4.2]. Obviously, $M_n(\mathbb{C})$ is isometrically isomorphic as ${}^*$-algebra to $\mathcal{B}(\ell_2(\mathbb{N}_n))$. From [\cite{MurphyCStarAlgsAndOpTh}, remark 3.4.1] it follows that the natural embeddings $i_{k,l}:A\to M_n(A):a\mapsto(a\delta_{i,k}\delta_{j,l})_{i,j\in\mathbb{N}_n}$ and projections $\pi_{k,l}:M_n(A)\to A:a\mapsto a_{k,l}$ are continuous. Therefore for a given bounded linear operator $\phi:A\to B$ between $C^*$-algebras $A$ and $B$ the linear operator 
$$
M_n(\phi):M_n(A)\to M_n(B):a\mapsto (\phi(a_{i,j}))_{i,j\in\mathbb{N}_n}
$$ 
is also bounded. Even more if $\phi$ is an $A$-morphism or ${}^*$-homomorphism, then so does $M_n(\phi)$. Finally we shall mention two isometric isomorphisms that will be of use:
$$
M_n\left(\bigoplus\nolimits_\infty\{A_\lambda:\lambda\in\Lambda\}\right)
\isom{\mathbf{Ban}_1}
\bigoplus\nolimits_\infty\{M_n\left(A_\lambda\right):\lambda\in\Lambda\},
$$
$$
M_n(C(K))\isom{\mathbf{Ban}_1}C(K,M_n(\mathbb{C}))
$$

Now a few facts on approximate identities and identities of $C^*$-algebras and their ideals. Any two-sided closed ideal of $C^*$-algebra has a two-sided contractive positive approximate identity [\cite{HelBanLocConvAlg}, theorem 4.7.79], and any left ideal has a right contractive positive approximate identity. In some cases even an approximate identity is not enough, so for this situation there is a procedure to endow a $C^*$-algebra with identity and preserve $C^*$-algebraic structure [\cite{HelBanLocConvAlg}, proposition 4.7.6]. This type of unitization we shall denote as $A_\#$. Till the end of this paragraph we assume that $A$ is a unital $C^*$-algebra. An element $a\in A$ is called a projection (or an orthogonal projection) if $a=a^*=a^2$, self-adjoint if $a=a^*$, positive if $a=b^*b$ for some $b\in A$, unitary if $a^*a=aa^*=e_A$. The set $A_{pos}$ of all positive elements of $A$ is a closed cone in $A$. If an element $a\in A$ is $\langle$~self-adjoint / positive~$\rangle$, then $\langle$~$\operatorname{sp}_A(a)\subset[-\Vert a\Vert, \Vert a\Vert]$ / $\operatorname{sp}_A(a)\subset[0,\Vert a\Vert]$~$\rangle$. For a given self-adjoint element $a\in A$, there always exists the isometric ${}^*$-homomorphism $\operatorname{Cont}_a:C(\operatorname{sp}_A(a))\to A$ such that $\operatorname{Cont}_a(f)=a$, where $f:\operatorname{sp}_A(a)\to\mathbb{C}:t\mapsto t$. It is called the continuous functional calculus [\cite{HelBanLocConvAlg}, theorem 4.7.24]. Loosely speaking it allows to take continuous functions of self-adjoint elements of $C^*$-algebras, so following standard convention we shall write $f(a)$ instead of $\operatorname{Cont}_a(f)$. Another related result called the spectral mapping theorem allows to compute the spectrum of elements given by continuous functional calculus: $\operatorname{sp}_A(f(a))=f(\operatorname{sp}_A(a))$.

We proceed to the discussion of more general objects --- Banach modules. Let $A$ be a Banach algebra, we say that $X$ is a $\langle$~left / right~$\rangle$ Banach $A$-module if $X$ is a Banach space endowed with bilinear operator $\langle$~$\cdot:A\times X\to X$ / $\cdot: X\times A\to X$~$\rangle$ of norm at most $1$ (called a module action), such that $\langle$~$a\cdot(b\cdot x)=ab\cdot x$ / $(x\cdot a)\cdot b=x\cdot ab$~$\rangle$ for all $a,b\in A$ and $x\in X$. Any Banach space $E$ can be turned into a $\langle$~left / right~$\rangle$ Banach $A$-module be defining $\langle$~$a\cdot x=0$ / $x\cdot a=0$~$\rangle$ for all $a\in A$ and $x\in E$. Any Banach algebra $A$ can be regarded as a left and right Banach $A$-module --- the module action coincides with algebra multiplication. Of course, there are more meaningful examples too.  Usually we shall discuss only left Banach modules since for their right sided counterparts all definitions and results are similar. We call a left Banach module $X$ over unital Banach algebra $A$ unital if $e_A\cdot x=x$ for all $x\in X$. For a given left Banach $A$-module $X$ and $S\subset A$, $M\subset X$ we define their products $S\cdot M=\{a\cdot x:a\in S, x\in M\}$, $SM=\operatorname{span} (S\cdot M)$ and annihilators $S^{\perp M}=\{a\in S:a\cdot M=\{0\}\}$, ${}^{S\perp}M=\{x\in M: S\cdot x=\{0\}\}$. The essential and annihilator parts of $X$ are defined as $X_{ess}=\operatorname{cl}_X(A X)$, $X_{ann}={}^{A\perp}X$. The module $X$ is called $\langle$~faithful / annihilator / essential~$\rangle$ if $\langle$~${}^{A\perp}X=\{0\}$ / $X=X_{ann}$ / $X=X_{ess}$~$\rangle$. An obvious application of Hahn-Banach theorem shows that $X$ is an essential $A$-module iff $X^*$ is a faithful $A$-module.

Let $X$ and $Y$ be $\langle$~left / right~$\rangle$ Banach $A$-modules. We say that a linear operator $\phi:X\to Y$ is an $A$-module map of $\langle$~left / right~$\rangle$ modules if $\langle$~$\phi(a\cdot x)=a\cdot \phi(x)$ / $\phi(x\cdot a)=\phi(x)\cdot a$~$\rangle$ for all $a\in A$ and $x\in X$. A bounded $A$-module map is called an $A$-morphism. The set of $A$-morphisms between $\langle$~left / right~$\rangle$ $A$-modules $X$ and $Y$ we denote as $\langle$~${}_A\mathcal{B}(X,Y)$ / $\mathcal{B}_A(X,Y)$~$\rangle$. Note that if $X$ and $Y$ are $\langle$~left / right~$\rangle$ annihilator $A$-modules, then $\langle$~${}_A\mathcal{B}(X,Y)=\mathcal{B}(X,Y)$ / $\mathcal{B}_A(X,Y)=\mathcal{B}(X,Y)$~$\rangle$.

By $\langle A-\mathbf{mod}$ / $\mathbf{mod}-A\rangle$ we shall denote the category of $\langle$~left / right~$\rangle$ $A$-modules with continuous $A$-module maps in the role of morphisms. By $\langle$~$A-\mathbf{mod}_1$ / $\mathbf{mod}_1-A$~$\rangle$ we denote its subcategory of $\langle$~$A-\mathbf{mod}$ / $\mathbf{mod}-A$~$\rangle$ with the same objects and contractive morphisms only. Therefore $\langle$~ $\operatorname{Hom}_{A-\mathbf{mod}}(X,Y)={}_A\mathcal{B}(X,Y)$ /  $\operatorname{Hom}_{\mathbf{mod}-A}(X,Y)=\mathcal{B}_A(X,Y)$~$\rangle$. 

As in any category we can speak of retraction and	 coretractions in the category of Banach modules. But for this particular case we have several refinements for the standard definitions. An $A$-morphism $\xi:X\to Y$ is called a $\langle$~$c$-retraction / $c$-coretraction~$\rangle$ if there exist an $A$-morphism $\eta:Y\to X$ such that $\langle$~$\xi\eta=1_Y$ / $\eta\xi=1_X$~$\rangle$ and $\Vert\xi\Vert\Vert\eta\Vert\leq c$. From the definition it follows that composition of $\langle$~$c_1$- and $c_2$-retraction / $c_1$- and $c_2$-coretraction~$\rangle$ gives a $\langle$~$c_1c_2$-retraction / $c_1c_2$-coretraction~$\rangle$. Clearly, the adjoint of $\langle$~$c$-retraction / $c$-coretraction~$\rangle$ is a $\langle$~$c$-coretraction / $c$-retraction~$\rangle$. Finally, an $A$-morphism $\xi:X\to Y$ is called a $c$-isomorphism if there exists an $A$-morphism $\eta:Y\to X$ such that $\xi\eta=1_Y$, $\eta\xi=1_X$ and $\Vert\xi\Vert\Vert\eta\Vert\leq c$. In this case we say that $A$-modules $X$ and $Y$ are $c$-isomorphic.

Now we mention several constructions over Banach modules that we will encounter in this thesis.  Any left Banach $A$-module can be regarded as unital Banach module over $A_+$, and we put by definition $(a\oplus_1 z)\cdot x=a\cdot x+zx$ for all $a\in A$, $x\in X$ and $z\in\mathbb{C}$. Most constructions used for Banach spaces transfer to Banach modules.  We say that a linear subspace $ Y$ of a left Banach $A$-module $X$ is a left $A$-submodule of $X$ if $A\cdot Y\subset Y$. For example, any left ideal $I$ of a Banach algebra $A$ is a left $A$-submodule of $A$. If $Y$ is a closed left $A$-submodule of the left Banach $A$-module $X$, then the Banach space $X/Y$ can be endowed with the structure of the left Banach $A$-module, just put by definition $a\cdot(x+Y)=a\cdot x+Y$ for all $a\in A$ and $x+Y\in X/Y$. This object is called the quotient $A$-module.  Quotient modules of the form $A/I$, where $I$ is a left ideal of $A$, are called cyclic modules. For motivation for this term see [\cite{HelBanLocConvAlg}, proposition 6.2.2]. Clearly, $X/X_{ess}$ is an annihilator $A$-module. If $X$ is a left Banach $A$-module and $E$ is a Banach space, then $\langle$~$\mathcal{B}(X,E)$ / $\mathcal{B}(E,X)$~$\rangle$ is a $\langle$~right / left~$\rangle$ Banach $A$-module with module action defined by $\langle$~$(T\cdot a)(x)=T(a\cdot x)$ for all $a\in A$, $x\in X$ and $T\in\mathcal{B}(X, E)$ / $(a\cdot T)(x)=a\cdot T(x)$ for all $a\in A$, $x\in E$ and $T\in\mathcal{B}(E, X)$~$\rangle$. In particular, $X^*$ is a right Banach $A$-module. If $\{X_\lambda:\lambda\in\Lambda\}$ is a family of left Banach $A$-modules and $1\leq p\leq +\infty$ or $p=0$, then their $\bigoplus_p$-sum is a left Banach $A$-module with module action defined by $a\cdot x=\bigoplus_p\{ a\cdot x_\lambda:\lambda\in\Lambda\}$, where $a\in A$, $x\in\bigoplus_p\{ X_\lambda:\lambda\in\Lambda\}$. Again, as in Banach space theory, any family of $A$-modules admits the $\langle$~product / coproduct~$\rangle$ in $A-\mathbf{mod}_1$ which in fact is their $\langle$~$\bigoplus_1$-sum / $\bigoplus_\infty$-sum~$\rangle$. The category $A-\mathbf{mod}$ admits $\langle$~products / coproducts~$\rangle$ only for finite families of objects. Similar statements are valid for $\mathbf{mod}-A$ and $\mathbf{mod}_1-A$.

Projective tensor product of Banach spaces also has its module version, it is called the projective module tensor product. Assume $X$ is a right and $Y$ is a left Banach $A$-module. Their projective module tensor product $X\projmodtens{A}Y$ is defined as quotient space $X\projtens Y / N$ where $N=\operatorname{cl}_{X\projtens Y}(\operatorname{span}\{x\cdot a\projtens y-x\projtens a\cdot y:x\in X,y\in Y,a\in A\})$. Let $\phi\in\mathcal{B}_A(X_1,X_2)$ and $\psi\in{}_A\mathcal{B}(Y_1,Y_2)$ for right Banach $A$-modules $X_1$, $X_2$ and left Banach $A$-modules $Y_1$, $Y_2$, then there exists a unique bounded linear operator $\phi\projmodtens{A} \psi:X_1\projmodtens{A} Y_1\to X_2\projmodtens{A} Y_2$ such that $(\phi\projmodtens{A} \psi)(x\projmodtens{A} y)=\phi(x)\projmodtens{A} \psi(y)$ for all $x\in X_1$ and $y\in Y_1$. Even more $\Vert \phi\projmodtens{A} \psi\Vert\leq\Vert \phi\Vert\Vert \psi\Vert$. The projective module tensor product has its own universal property: for any right Banach $A$-module $X$, any left Banach $A$-module $Y$ and any Banach space $E$ there exists an isometric isomorphism:
$$
\mathcal{B}(X\projmodtens{A}Y,E)\isom{\mathbf{Ban}_1}\mathcal{B}_{bal}(X\times Y, E)
$$
where $\mathcal{B}_{bal}(X\times Y, E)$ stands for the Banach space of bilinear operators $\Phi:X\times Y\to E$ satisfying $\Phi(x\cdot a,y)=\Phi(x,a\cdot y)$ for all $x\in X$, $y\in Y$ and $a\in A$. Such bilinear operators are called balanced.
Furthermore we have two (natural in $X$, $Y$ and $E$) isometric isomorphisms:
$$
\mathcal{B}(X\projmodtens{A}Y,E)
\isom{\mathbf{Ban}_1}
{}_A\mathcal{B}(Y,\mathcal{B}(X,E))
\isom{\mathbf{Ban}_1}
\mathcal{B}_A(X,\mathcal{B}(Y,E))
$$
Analogously to Banach space theory we may define the following functors:
$$
\mathcal{B}(-,E):A-\mathbf{mod}\to \mathbf{mod}-A
\qquad\qquad
\mathcal{B}(E,-):\mathbf{mod}-A\to \mathbf{mod}-A
$$
$$
-\projmodtens{A} Y:\mathbf{mod}-A\to\mathbf{Ban}
\qquad\qquad
X\projmodtens{A} -:A-\mathbf{mod}\to\mathbf{Ban}
$$
where $E$ is a Banach space, $X$ is a right $A$-module and $Y$ is a left $A$-module. All these functors have their counterparts for categories $A-\mathbf{mod}_1$, $\mathbf{mod}_1-A$. 

In some cases it is possible to explicitly compute the projective module tensor product. For example [\cite{HelBanLocConvAlg}, proposition 6.3.24] if $I$ is a left closed ideal of $A_+$ with left $\langle$~contractive / bounded~$\rangle$ approximate identity, and $X$ is a left Banach module then the linear operator 
$$
i_{I,X}:I\projmodtens{A}X \to \operatorname{cl}_X(IX):a\projmodtens{A} x\mapsto a\cdot x
$$
is $\langle$~a topological isomorphism / an isometric isomorphism~$\rangle$ of Banach spaces. If $I$ is a two-sided ideal, then $i_{I,X}$ is a morphism of left $A$-modules. We call reduced all left Banach modules of the form $A\projmodtens{A}X$. 

Most of what have been said here can be generalized to Banach bimodules, but in this thesis we shall not exploit them much. In those rare case when we shall encounter bimodules, the respective definitions and results are easily recoverable from their one sided counterparts.

%----------------------------------------------------------------------------------------
%	Banach homology
%----------------------------------------------------------------------------------------

\section{Banach homology}
\label{SectionBanachHomology}

%----------------------------------------------------------------------------------------
%	Relative homology
%----------------------------------------------------------------------------------------

\subsection{Relative homology}
\label{SubSectionRelativeHomology}

Further we briefly discuss ABCs of relative homology introduced and intensively studied by Helemskii. Fix an arbitrary Banach algebra $A$. We say that a morphism $\xi:X\to Y$ of left $A$-modules $X$ and $Y$ is a relatively admissible epimorphism if it admits a right inverse bounded linear operator. A left $A$-module $P$ is called relatively projective if for any relatively admissible  epimorphism $\xi:X\to Y$ and for any $A$-morphism $\phi:P\to Y$ there exists an $A$-morphism $\psi:P\to X$ such that the diagram
$$
\xymatrix{
& {X} \ar[d]^{\xi}\\
{P} \ar@{-->}[ur]^{\psi} \ar[r]^{\phi} &{Y}}
$$
is commutative. Such $A$-morphism $\psi$ is called a lifting of $\phi$ and it is not unique in general. Similarly,  we say that a morphism $\xi:Y\to X$ of right $A$-modules $X$ and $Y$ is a relatively admissible monomorphism if it admits a left inverse bounded linear operator. A right $A$-module $J$ is called relatively injective if for any relatively admissible  monomorphism $\xi:Y\to X$ and for any $A$-morphism $\phi:Y\to J$ there exists an $A$-morphism $\psi:X\to J$ such that the diagram
$$
\xymatrix{
& {X} \ar@{-->}[dl]_{\psi} \\
{J} &{Y} \ar[l]_{\phi} \ar[u]_{\xi}}
$$
is commutative. Such $A$-morphism $\psi$ is called an extension of $\phi$ and it is not unique in general.

The reason for considering relatively admissible morphisms in these definitions is the intention of separation Banach geometric and algebraic motives that may prevent an $A$-module to be relatively projective or injective. A straightforward check shows that any retract of relatively $\langle$~projective / injective~$\rangle$ $A$-module is again relatively $\langle$~projective / injective~$\rangle$. Obviously, any relatively admissible $\langle$~epimorphism / monomorphism~$\rangle$ $\langle$~onto / from~$\rangle$ a relatively $\langle$~projective / injective~$\rangle$ $A$-module is a $\langle$~retraction / coretraction~$\rangle$.

A special class of relatively $\langle$~projective / injective~$\rangle$ $A$-modules is the so-called relatively $\langle$~free / cofree~$\rangle$ modules. These are modules of the form $\langle$~$A_+\projtens E$ / $\mathcal{B}(A_+,E)$~$\rangle$ for some Banach space $E$. Their main feature is the following: for any $A$-module $X$ there exists a relatively $\langle$~free / cofree~$\rangle$ $A$-module $F$, which in fact is $\langle$~$A_+\projtens X$ / $\mathcal{B}(A_+,X)$~$\rangle$ and a relatively admissible $\langle$~epimorphism $\xi:F\to X$ / monomorphism $\xi:X\to F$~$\rangle$. If $X$ is relatively $\langle$~projective / injective~$\rangle$ we immediately get that $\xi$ is a $\langle$~retraction / coretraction~$\rangle$. Therefore an $A$-module is relatively $\langle$~projective / injective~$\rangle$ iff it is a retract of relatively $\langle$~free / cofree~$\rangle$ $A$-module. 

It is worth to emphasize one more time that major nuance of relative Banach homology is deliberate balance between algebra and topology in choice of admissible morphisms. This choice allowed one to build homological theory with some interesting phenomena with no analogs in pure algebra. We demonstrate one example related to Banach algebras. Consider morphism of $A$-bimodules  $\Pi_A:A\projtens A\to A:a\projtens b\mapsto ab$. We say that a Banach algebra $A$ is 

$i)$ relatively $c$-biprojective if $\Pi_A$ is a $c$-retraction of $A$-bimodules;

$ii)$ relatively $c$-biflat if $\Pi_A^*$ is a $c$-coretraction of $A$-bimodules;

$iii)$ relatively $c$-contractible if $\Pi_{A_+}$ is a $c$-retraction of $A$-bimodules;

$iv)$ relatively $c$-amenable if $\Pi_{A_+}^*$ is a $c$-coretraction of $A$-bimodules.

We say that $A$ is relatively $\langle$~biprojective / biflat / contractive / amenable~$\rangle$ if it is relatively $\langle$~$c$-biprojective / $c$-biflat / $c$-contractive / $c$-amenable~$\rangle$ for some $c\geq 1$. The infimum of the constants $c$ is called the $\langle$~biprojectivity / biflatness / contractivity / amenability~$\rangle$ constant.  With slight modifications of [\cite{HelBanLocConvAlg}, proposition 7.1.72] one can show that $A$ is relatively $\langle$~$c$-contractible / $c$-amenable~$\rangle$ iff there exists $\langle$~an element $d\in A\projtens A$ / a net $(d_\nu)_{\nu\in N}\subset A\projtens A$~$\rangle$ with norm not greater than $c$ such that for all $a\in A$ holds $\langle$~$a\cdot d-d\cdot a=0$ and $a\Pi_A(d)=a$ / $\lim_\nu(a\cdot d_\nu-d_\nu\cdot a)=0$ and $\lim_\nu a\Pi_A(d_\nu)=a$~$\rangle$. Note that $\langle$~such element $d$ / such net $(d_\nu)_{\nu\in N}$~$\rangle$ is called $\langle$~a diagonal / an approximate diagonal~$\rangle$. From homological point of view, the main advantage of relatively $\langle$~biprojective / biflat / contractible / amenable~$\rangle$ Banach algebras is that $\langle$~any reduced / any reduced / any / any~$\rangle$ left and right Banach $A$-module is relatively $\langle$~projective / flat / projective / flat~$\rangle$ [\cite{HelBanLocConvAlg}, theorem 7.1.60]. As for flatness such phenomena is typical for relative Banach homology, but not for the purely algebraic one.

%----------------------------------------------------------------------------------------
%	Rigged categories
%----------------------------------------------------------------------------------------

\subsection{Rigged categories}
\label{SubSectionRiggedCategories}

Claims on projectivity and injectivity from previous section have their analogs for a lot of other types of projectivity and injectivity in other categories of mathematics \cite{SemadeniProjInjDual}. Even more, one may easily see that injectivity and projectivity are somewhat dual to each other. All these observations suggest that there is a general categorical approach to study basic properties of homologically trivial objects. Such approach has been promoted by Helemskii in \cite{HelMetrFrQMod}. As we shall see it covers relative theory while results given above are obvious consequences of more general facts. 

Let $\mathbf{C}$ and $\mathbf{D}$ be two fixed categories. An ordered pair ($\mathbf{C}, \square:\mathbf{C}\to\mathbf{D}$), where $\square$ is a faithful covariant functor, is called a rigged category. We say that a morphism $\xi$ in $\mathbf{C}$ is $\square$-admissible epimorphism if $\square (\xi)$ is a retraction in $\mathbf{D}$. An object $P$ in $\mathbf{C}$ is called $\square$-projective, if for every $\square$-admissible epimorphism $\xi$ in $\mathbf{C}$ the map $\operatorname{Hom}_{\mathbf{C}}(P,\xi)$ is surjective. An object $F$ in $\mathbf{C}$ is called $\square$-free with base $M$ in  $\mathbf{D}$, if there exists an isomorphism of functors $\operatorname{Hom}_{\mathbf{D}}(M,\square(-))\cong\operatorname{Hom}_{\mathbf{C}}(F,-)$. A rigged category $(\mathbf{C},\square)$ is called  freedom-loving [\cite{HelMetrFrQMod}, definition 2.10], if every object in $\mathbf{D}$ is a base of some $\square$-free object in $\mathbf{C}$. We may summarize results of propositions 2.3, 2.11  and 2.12 in \cite{HelMetrFrQMod} as follows:

$i)$ any retract of $\square$-projective object is $\square$-projective;

$ii)$ any $\square$-admissible epimorphism into $\square$-projective object is a retraction;

$iii)$ any $\square$-free object is $\square$-projective;

$iv)$ if $(\mathbf{C},\square)$ is freedom-loving rigged category, then any object is $\square$-projective iff it is a retract of $\square$-free object;

$v)$ coproduct of the family of $\square$-projective objects is $\square$-projective.

The opposite rigged category of $(\mathbf{C}, \square)$ 
is a rigged category $(\mathbf{C}^{o},\square^{o}:\mathbf{C}^{o}\to\mathbf{D}^{o})$. 
Thus by passing to the opposite rigged category we may define admissible monomorphisms, injectivity and cofreedom. A morphism $\xi$ in called $\square$-admissible monomorphism if it is $\square^o$-admissible epimorphism. An object $J$ in $\mathbf{C}$ is called $\square$-injective if it is $\square^o$-projective. Finally, an object $F$ in $\mathbf{C}$ is called $\square$-cofree if it is $\square^o$-free. This gives us analogs of results as above for injectivity and cofreedom.

Now consider faithful functor $\square_{rel}:A-\mathbf{mod}\to\mathbf{Ban}$ that just `forgets'' the module structure. One can easily see that $(A-\mathbf{mod},\square_{rel})$ is a rigged category whose $\square_{rel}$-admissible $\langle$~epimorphisms / monomorphisms~$\rangle$ are exactly relatively admissible $\langle$~epimorphisms / monomorphisms~$\rangle$ and $\langle$~$\square_{rel}$-projective / $\square_{rel}$-injective~$\rangle$ objects are exactly relatively $\langle$~projective / injective~$\rangle$ $A$-modules. Even more all $\langle$~$\square_{rel}$-free / $\square_{rel}$-cofree~$\rangle$ objects are isomorphic in $A-\mathbf{mod}$ to $\langle$~$A_+\projtens E$ / $\mathcal{B}(A_+,E)$ ~$\rangle$ for some Banach space $E$. This example shows, that relative theory perfectly fits into the realm of rigged categories.

We shall apply this scheme for metric and topological theory in the next chapter. These two theories put much weaker restrictions on their admissible morphisms. The proverb ``all covet, all lose'' perfectly explains what will happen next.
	
% chktex-file 19 
% chktex-file 35 
% Chapter Template
% Main chapter title
% Change X to a consecutive number; for referencing 
% this chapter elsewhere, use~\ref{ChapterX}
\chapter{Общая теория}\label{ChapterGeneralTheory} 

%-------------------------------------------------------------------------------
%	Projectivity, injectivity and flatness
%-------------------------------------------------------------------------------

\section{
    Проективность, инъективность и плоскость
}\label{
    SectionProjectivityInjectivityAndFlatness
}


%-------------------------------------------------------------------------------
%	Metric and topological projectivity
%-------------------------------------------------------------------------------

\subsection{
    Метрическая и топологическая проективность
}\label{
    SubSectionMetricAndTopologicalProjectivity
}

В дальнейшем $A$ обозначает необязательно унитальную банахову алгебру. Мы сразу
же приступим к формулировке, пожалуй, двух самых важных определений в этой
работе.

\begin{definition}[\cite{HelMetrFrQMod}, определение 1.4]\label{MetProjMod}
$A$-модуль $P$ называется метрически проективным, если для любого строго
коизометрического $A$-морфизма $\xi:X\to Y$ и любого $A$-морфизма $\phi:P\to Y$
существует $A$-морфизм $\psi:P\to X$ такой, что $\xi\psi=\phi$ и
$\Vert\psi\Vert=\Vert\phi\Vert$.
\end{definition}

\begin{definition}[\cite{HelMetrFrQMod}, определение 1.2]\label{TopProjMod}
$A$-модуль $P$ называется топологически проективным, если для любого
топологически сюръективного $A$-морфизма $\xi:X\to Y$ и любого $A$-морфизма
$\phi:P\to Y$ существует $A$-морфизм $\psi:P\to X$ такой, что $\xi\psi=\phi$.
\end{definition}

Эквивалентные и более короткие определения звучат так: $A$-модуль $P$ называется
$\langle$~метрически / топологически~$\rangle$ проективным, если функтор
$\langle$~$\operatorname{Hom}_{A-\mathbf{mod}_1}(P,-)
    :A-\mathbf{mod}_1\to\mathbf{Ban}_1$
/
$\operatorname{Hom}_{A-\mathbf{mod}}(P,-)
    :A-\mathbf{mod}\to\mathbf{Ban}$~$\rangle$
переводит $\langle$~строго коизометрические / топологически
сюръективные~$\rangle$ $A$-морфизмы в $\langle$~строго коизометрические /
сюръективные~$\rangle$ операторы.

Теперь мы нацелены применить аппарат оснащенных категорий к метрической и
топологической проективности. Прежде чем это сделать, нам нужно описать
категории так называемых полунормированных пространств определенных Штейнером
в~\cite{ShtTopFrClassicQuantMod}. Полулинейное пространство над полем
$\mathbb{C}$ это множество $E$, элементы которого называются векторами, с
бинарной операцией $\cdot:\mathbb{C}\times E\to E$, удовлетворяющей трем
аксиомам:
\begin{enumerate}[label = (\roman*)]
    \item $\alpha\cdot(\beta\cdot x)=\alpha\beta\cdot x$ для всех
    $\alpha,\beta\in\mathbb{C}$ и $x\in E$; 

    \item $1\cdot x=x$ для всех $x\in E$; 

    \item существует нулевой вектор $0\in E$ такой, что $0\cdot x=0$ для всех 
    $x\in E$. 
\end{enumerate}

Отображение $T:E\to F$ между полулинейными пространствами называется
полулинейным оператором, если $T(\alpha\cdot x)=\alpha\cdot T(x)$ для всех
$\alpha\in\mathbb{C}$ и $x\in E$. Полунормированное пространство $E$ --- это
полулинейное пространство вместе с функцией  $\Vert\cdot\Vert:E\to\mathbb{R}_+$
(называемой нормой) такой, что
\begin{enumerate}[label = (\roman*)]
    \item $\Vert x\Vert=0$ тогда и только тогда, когда $x=0$;

    \item $\Vert\alpha\cdot x\Vert=|\alpha|\Vert x\Vert$ для всех
    $\alpha\in\mathbb{C}$ и $x\in E$. 
\end{enumerate}

Полулинейный оператор $T:E\to F$ между полунормированными пространствами $E$ и
$F$ называется ограниченным, если существует константа $C\geq 0$ такая, что
$\Vert\phi(x)\Vert\leq C\Vert x\Vert$ для всех $x\in E$. Наименьшая такая
константа называется нормой $\phi$ и обозначается как $\Vert \phi\Vert$.
Наконец, мы определим категорию полунормированных пространств $\mathbf{HNor}$:
ее объекты --- полунормированные пространства, ее морфизмы --- полулинейные
операторы. На самом деле, полунормированные пространства и ограниченные
полулинейные операторы это то, что останется от нормированных пространств и
ограниченных линейных операторов, если убрать из их определений все упоминания
операции сложения векторов. Вот типичный пример  полунормированного
пространства. Для заданного непустого множества $\Lambda$ рассмотрим букет
$\mathbb{C}^\Lambda:=\bigvee  \{\mathbb{C}:\lambda\in\Lambda \}$ копий
$\mathbb{C}$ с общим нулевым вектором. Умножение на скаляры и норма в
$\mathbb{C}^\Lambda$ определяются очевидным образом. Также положим по
определению $\mathbb{C}^\varnothing= \{0 \}$. Любое полунормированное
пространство изоморфно в $\mathbf{HNor}$ пространству $\mathbb{C}^\Lambda$ для
некоторого множества $\Lambda$ [\cite{ShtTopFrClassicQuantMod}, предложение
1.1.9].

В~\cite{HelMetrFrQMod} и~\cite{ShtTopFrClassicQuantMod} были построены два
верных функтора: 
$$
\square_{met}:A-\mathbf{mod}_1\to\mathbf{Set}
:X\mapsto B_X,\phi\mapsto\phi|_{B_X}^{B_Y},
$$
$$
\square_{top}:A-\mathbf{mod}\to\mathbf{HNor}:X\mapsto X,\phi\mapsto\phi.
$$
Первый из них отправляет банахов $A$-модуль в его единичный шар, а всякий
сжимающий $A$-морфизм в соответствующее биограничение. Второй функтор
``забывает'' о модульной и аддитивной структуре. В тех же статьях было доказано,
что, во-первых, $A$-морфизм $\xi$ $\langle$~строго коизометричен / топологически
сюръективен~$\rangle$ тогда и только тогда, когда он
$\langle$~$\square_{met}$-допустимый / $\square_{top}$-допустимый~$\rangle$
эпиморфизм и, во-вторых, $A$-модуль $P$ является $\langle$~метрически /
топологически~$\rangle$ проективным тогда и только тогда, когда он
$\langle$~$\square_{met}$-проективен / $\square_{top}$-проективен~$\rangle$.
Таким образом, мы немедленно получаем следующее предложение.

\begin{proposition}\label{RetrMetTopProjIsMetTopProj} Всякий ретракт
$\langle$~метрически / топологически~$\rangle$ проективного модуля в
$\langle$~$A-\mathbf{mod}_1$ / $A-\mathbf{mod}$~$\rangle$ снова
$\langle$~метрически / топологически~$\rangle$ проективен.
\end{proposition}

Также было доказано, что оснащенная категория
$\langle$~$(A-\mathbf{mod}_1,\square_{met})$ /
$(A-\mathbf{mod},\square_{top})$~$\rangle$ свободолюбива, и что
$\langle$~$\square_{met}$-свободные / $\square_{top}$-свободные~$\rangle$ модули
изоморфны в $\langle$~$A-\mathbf{mod}_1$ / $A-\mathbf{mod}$~$\rangle$ модулям
вида $A_+\projtens \ell_1(\Lambda)$ для некоторого множества $\Lambda$. Более
того, для любого $A$-модуля $X$ существует $\langle$~$\square_{met}$-допустимый
/ $\square_{top}$-допустимый~$\rangle$ эпиморфизм
$$
\pi_X^+:A_+\projtens \ell_1(B_X):a\projtens \delta_x\mapsto a\cdot x.
$$
Как следствие общих результатов об оснащенных категориях мы получаем следующее
предложение.

\begin{proposition}\label{MetTopProjModViaCanonicMorph} $A$-модуль $P$
$\langle$~метрически / топологически~$\rangle$ проективен тогда и только тогда,
когда  $\pi_P^+$ --- ретракция в $\langle$~$A-\mathbf{mod}_1$ /
$A-\mathbf{mod}$~$\rangle$.
\end{proposition}

Так как $\langle$~$\square_{met}$-свободные /
$\square_{top}$-свободные~$\rangle$ модули совпадают с точностью до изоморфизма
в $A-\mathbf{mod}$ и всякая ретракция в $A-\mathbf{mod}_1$ есть ретракция в
$A-\mathbf{mod}$, то из предложения~\ref{RetrMetTopProjIsMetTopProj} мы видим,
что любой метрически проективный $A$-модуль топологически проективен. Напомним,
что каждый относительно проективный модуль есть ретракт в $A-\mathbf{mod}$
модуля вида $A_+\projtens E$ для некоторого банахова пространства $E$.
Следовательно, каждый топологически проективный $A$-модуль будет относительно
проективным. Мы резюмируем эти результаты в следующем предложении.

\begin{proposition}\label{MetProjIsTopProjAndTopProjIsRelProj} Каждый метрически
проективный модуль топологически проективен, и каждый топологически проективный
модуль относительно проективен.
\end{proposition}

Количественный аналог определения топологической проективности был дан Уайтом.

\begin{definition}[\cite{WhiteInjmoduAlg}, определение 2.4]\label{CTopProjMod}
$A$-модуль $P$ называется $C$-топологически проективным, если для любого строго
$c$-топологически сюръективного $A$-морфизма $\xi:X\to Y$ и любого $A$-морфизма
$\phi:P\to Y$ существует $A$-морфизм $\psi:P\to X$ такой, что $\xi\psi=\phi$ и
$\Vert\psi\Vert\leq cC\Vert\phi\Vert$.
\end{definition}

Нам понадобятся следующие два факта об этом типе проективности.

\begin{proposition}[\cite{WhiteInjmoduAlg}, лемма
2.7]\label{RetrCTopProjIsCTopProj} Всякий $C_1$-ретракт $C_2$-топологически
проективного модуля является $C_1C_2$-топологически проективным.
\end{proposition}

\begin{proposition}[\cite{WhiteInjmoduAlg}, предложение
2.10]\label{CTopProjModViaCanonicMorph} $A$-модуль $P$ является
$C$-топологически проективным тогда и только тогда, когда $\pi_P^+$ ---
$C$-ретракция в $A-\mathbf{mod}$.
\end{proposition}

Как следствие, банахов модуль топологически проективен тогда и только тогда,
когда он $C$-топологически проективен для некоторого $C$. Далее мы будем
использовать определения~\ref{TopProjMod} и~\ref{CTopProjMod} без ссылки на их
эквивалентность.

Теперь перейдем к обсуждению примеров. Заметим, что категория банаховых
пространств может рассматриваться как категория левых банаховых модулей над
нулевой алгеброй. Как следствие, мы получаем определение $\langle$~метрически /
топологически~$\rangle$ проективного банахова пространства. Все результаты
полученные выше верны для этого типа проективности. Оба типа проективных
банаховых пространств уже описаны. В~\cite{KotheTopProjBanSp} Кёте доказал, что
все топологически проективные банаховы пространства топологически изоморфны
$\ell_1(\Lambda)$ для некоторого множества $\Lambda$. Используя результат
Гротендика из~\cite{GrothMetrProjFlatBanSp}, Хелемский показал, что метрически
проективные банаховы пространства изометрически изоморфны $\ell_1(\Lambda)$ для
некоторого индексного множества $\Lambda$ [\cite{HelMetrFrQMod}, предложение
3.2].

\begin{proposition}\label{UnitalAlgIsMetTopProj} $A$-модуль $A_\times$
метрически и топологически проективен.
\end{proposition} 
\begin{proof} Рассмотрим произвольный $A$-морфизм $\phi:A_\times\to Y$ и строго
коизометрический $A$-морфизм $\xi:X\to Y$. Из определения строго
коизометрического оператора следует, что существует $x_0\in X$ такой, что
$\xi(x_0)=\phi(e_{A_\times})$ и $\Vert x_0\Vert=\Vert\phi(e_{A_\times})\Vert$.
Рассмотрим $A$-морфизм $\psi:A_\times\to X:a\mapsto a\cdot x_0$. Очевидно,
$\Vert\psi\Vert\leq\Vert x_0\Vert\leq\Vert\phi\Vert\Vert
e_{A_\times}\Vert=\Vert\phi\Vert$. С другой стороны, $\xi\psi=\phi$, так что
$\Vert\phi\Vert\leq\Vert\xi\Vert\Vert\psi\Vert=\Vert\psi\Vert$. Следовательно,
$\Vert\phi\Vert=\Vert\psi\Vert$. Итак, мы доказали по определению, что
$A_\times$ --- метрически проективный $A$-модуль. По
предложению~\ref{MetProjIsTopProjAndTopProjIsRelProj} он также топологически
проективен.
\end{proof}

\begin{proposition}\label{NonDegenMetTopProjCharac} Пусть $P$ --- существенный
$A$-модуль. Тогда $P$ $\langle$~метрически / $C$-топологически~$\rangle$
проективен тогда и только тогда, когда отображение
$\pi_P:A\projtens\ell_1(B_P):a\projtens\delta_x\mapsto a\cdot x$ есть
$\langle$~$1$-ретракция / $C$-ретракция~$\rangle$ в $A-\mathbf{mod}$.
\end{proposition} 
\begin{proof}
Если $P$ $\langle$~метрически / $C$-топологически~$\rangle$ проективен, то по
предложению  $\langle$~\ref{MetTopProjModViaCanonicMorph}
/~\ref{CTopProjModViaCanonicMorph}~$\rangle$ морфизм $\pi_P^+$ имеет правый
обратный морфизм $\sigma^+$ c нормой $\langle$~не более $1$ / не более
$C$~$\rangle$. Тогда 
$$
\sigma^+(P)
=\sigma^+(\operatorname{cl}_{A_+\projtens\ell_1(B_P)}(AP))
\subset \operatorname{cl}_{A_+\projtens\ell_1(B_P)}(A\cdot\sigma(P))
$$
$$
=\operatorname{cl}_{A_+\projtens\ell_1(B_P)}(A\cdot(A_+\projtens\ell_1(B_P)))
=A\projtens\ell_1(B_P).
$$
Поэтому корректно определено коограничение $\sigma:P\to A\projtens\ell_1(B_P)$,
которое также есть $A$-морфизм c нормой $\langle$~не более $1$ / не более
$C$~$\rangle$. Ясно, что $\pi_P\sigma=1_P$, поэтому $\pi_P$ является
$\langle$~$1$-ретракцией / $C$-ретракцией~$\rangle$ в $A-\mathbf{mod}$.

Обратно, допустим $\pi_P$ имеет правый обратный морфизм $\sigma$ c нормой
$\langle$~не более $1$ / не более $C$~$\rangle$. Тогда его копродолжение
$\sigma^+$ также является правым обратным морфизмом к $\pi_P^+$ с той же нормой.
Снова, по предложению $\langle$~\ref{MetTopProjModViaCanonicMorph}
/~\ref{CTopProjModViaCanonicMorph}~$\rangle$ модуль $P$ $\langle$~метрически /
$C$-топологически~$\rangle$ проективен. 
\end{proof}

Следует напомнить, что $\langle$~произвольное / лишь конечное~$\rangle$
семейство объектов в $\langle$~$A-\mathbf{mod}_1$ / $A-\mathbf{mod}$~$\rangle$
обладает категорным копроизведением, которое на самом деле есть их
$\bigoplus_1$-сумма. В этом и состоит причина почему мы делаем дополнительное
предположение во втором пункте следующего предложения.

\begin{proposition}\label{MetTopProjModCoprod} Пусть
${(P_\lambda)}_{\lambda\in\Lambda}$ --- семейство банаховых $A$-модулей. Тогда 
\begin{enumerate}[label = (\roman*)] 
    \item $A$-модуль $\bigoplus_1 \{P_\lambda:\lambda\in\Lambda \}$ метрически
    проективен тогда и только тогда, когда для всех $\lambda\in\Lambda$ банахов
    $A$-модуль $P_\lambda$ метрически проективен;

    \item $A$-модуль $\bigoplus_1 \{P_\lambda:\lambda\in\Lambda \}$
    $C$-топологически проективен тогда и только тогда, когда для всех
    $\lambda\in\Lambda$ банахов $A$-модуль $P_\lambda$ $C$-топологически
    проективен.
\end{enumerate}
\end{proposition}
\begin{proof} Обозначим $P:=\bigoplus_1 \{P_\lambda:\lambda\in\Lambda \}$.

$(i)$ Доказательство аналогично доказательству из пункта $(ii)$.

$(ii)$ Допустим, что $P$ $C$-топологически проективен. Заметим, что для каждого
$\lambda\in\Lambda$ модуль $P_\lambda$ является $1$-ретрактом $P$ посредством
канонической проекции $p_\lambda:P\to P_\lambda$. По
предложению~\ref{RetrCTopProjIsCTopProj} модуль $P_\lambda$ $C$-топологически
проективен.

Обратно, допустим, что для каждого $\lambda\in\Lambda$ модуль $P_\lambda$
$C$-топологически проективен. По предложению~\ref{CTopProjModViaCanonicMorph} мы
имеем семейство $C$-ретракций $\pi_\lambda:A_+\projtens\ell_1(S_\lambda)\to
P_\lambda$. Следовательно, $\bigoplus_1 \{\pi_{P_\lambda}^+:\lambda\in\Lambda
\}$ является $C$-ретракцией в $A-\mathbf{mod}$. Значит, $P$ есть $C$-ретракт 
$$
\bigoplus\nolimits_1\left \{
    A_+\projtens \ell_1(S_\lambda):\lambda\in\Lambda\right \}
\isom{A-\mathbf{mod}_1}
\bigoplus\nolimits_1\left \{\bigoplus\nolimits_1 \{
    A_+:s\in S_\lambda \}:\lambda\in\Lambda\right \}
\isom{A-\mathbf{mod}_1}
\bigoplus\nolimits_1 \{A_+:s\in S \}
$$
в $A-\mathbf{mod}$, где $S=\bigsqcup_{\lambda\in\Lambda}S_\lambda$. Ясно, что
последний модуль $1$-топологически проективен, поэтому из
предложения~\ref{RetrCTopProjIsCTopProj} следует, что $A$-модуль $P$
$C$-топологически проективен.
\end{proof}

\begin{corollary}\label{MetTopProjTensProdWithl1} Пусть $P$ --- банахов
$A$-модуль и $\Lambda$ --- произвольное множество. Тогда $A$-модуль $P\projtens
\ell_1(\Lambda)$ $\langle$~метрически / топологически~$\rangle$ проективен тогда
и только тогда, когда $P$ $\langle$~метрически / топологически~$\rangle$
проективен.
\end{corollary}
\begin{proof} Заметим, что $P\projtens
\ell_1(\Lambda)\isom{A-\mathbf{mod}_1}\bigoplus_1 \{P:\lambda\in\Lambda \}$.
Теперь достаточно применить предложение~\ref{MetTopProjModCoprod} с
$P_\lambda=P$ для всех $\lambda\in\Lambda$.
\end{proof}

%-------------------------------------------------------------------------------
%	Metric and topological projectivity of ideals and cyclic modules
%-------------------------------------------------------------------------------

\subsection{
    Метрическая и топологическая проективность идеалов и циклических модулей
}\label{
    SubSectionMetricAndTopologicalProjectivityOfIdealsAndCyclicModules
}

Как мы вскоре увидим, идемпотенты играют основную роль в изучении метрической и
топологической проективности. Поэтому нам следует напомнить одно из следствий
теоремы Шилова об идемпотентах [\cite{KaniBanAlg}, параграф 3.5]: каждая
полупростая коммутативная банахова алгебра с компактным спектром имеет единицу,
но не обязательно нормы $1$. 

\begin{proposition}\label{UnIdeallIsMetTopProj} Пусть $I$ --- левый идеал
банаховой алгебры $A$. Тогда
\begin{enumerate}[label = (\roman*)]
    \item если $I=Ap$ для некоторого $\langle$~идемпотента $p\in I$ нормы $1$ /
    идемпотента $p\in I$~$\rangle$, то $I$ $\langle$~метрически /
    топологически~$\rangle$ проективен как $A$-модуль;

    \item если $I$ --- коммутативная полупростая алгебра и
    $\operatorname{Spec}(I)$ компактен, то $I$ топологически проективен как
    $A$-модуль.
\end{enumerate}
\end{proposition}
\begin{proof} 
$(i)$ Очевидно, что $A$-модульные операторы $\pi:A_\times\to I:x\mapsto xp$ и
$\sigma:I\to A_\times:x\mapsto x$ $\langle$~сжимающие / ограниченные~$\rangle$ и
$\pi\sigma=1_I$. Тогда $I$ есть ретракт $A_\times$ в
$\langle$~$A-\mathbf{mod}_1$ / $A-\mathbf{mod}$~$\rangle$. Теперь результат
следует из предложений~\ref{RetrMetTopProjIsMetTopProj}
и~\ref{UnitalAlgIsMetTopProj}.

$(ii)$ По теореме Шилова об идемпотентах идеал $I$ унитален. Вообще говоря, норма
его единицы не меньше $1$. Из пункта $(i)$ следует, что идеал $I$ топологически
проективен.
\end{proof}

Предположение полупростоты в~\ref{UnIdeallIsMetTopProj} не обязательно. В
[\cite{DalesIntroBanAlgOpHarmAnal}, упражнение 2.3.7] дан пример коммутативной
унитальной банаховой алгебры $A$, которая не является полупростой. По
предложению~\ref{UnIdeallIsMetTopProj} она топологически проективна как
$A$-модуль. Чтобы доказать главный результат этого параграфа нам нужны две
подготовительные леммы.

\begin{lemma}\label{ImgOfAMorphFromBiIdToA} Пусть $I$ --- двусторонний идеал
банаховой алгебры $A$, существенный как левый $I$-модуль и пусть задан
$A$-морфизм $\phi:I\to A$. Тогда $\operatorname{Im}(\phi)\subset I$.
\end{lemma}
\begin{proof} Так как $I$ --- правый идеал, то $\phi(ab)=a\phi(b)\in I$ для всех
$a,b\in I$. Поэтому $\phi(I\cdot I)\subset I$. Так как $I$ --- существенный
левый $I$-модуль, то $I=\operatorname{cl}_A(\operatorname{span}(I\cdot I))$ и
$\operatorname{Im}(\phi)\subset\operatorname{cl}_A(\operatorname{span}\phi(I\cdot
I))=\operatorname{cl}_A(\operatorname{span}I)=I$.
\end{proof}

\begin{lemma}\label{GoodIdealMetTopProjIsUnital} Пусть $I$ --- левый идеал
банаховой алгебры $A$. Допустим, выполнено одно из следующих условий:

$(*)$ $I$ имеет левую $\langle$~сжимающую / ограниченную~$\rangle$
аппроксимативную единицу, и для любого морфизма $\phi:I\to A$ левых $A$-модулей
найдется морфизм $\psi:I\to I$ правых $I$-модулей со свойством
$\phi(x)y=x\psi(y)$ для всех $x,y\in I$.

$(**)$ $I$ имеет правую $\langle$~сжимающую / ограниченную~$\rangle$
аппроксимативную единицу, и существует $\langle$~$C=1$ / $C\geq 1$~$\rangle$
такое, что для любого морфизма $\phi:I\to A$ левых $A$-модулей найдется морфизм
$\psi:I\to I$ правых $I$-модулей со свойствами $\Vert\psi\Vert\leq
C\Vert\phi\Vert$ и $\phi(x)y=x\psi(y)$ для всех $x,y\in I$.

Тогда следующие условия эквивалентны:
\begin{enumerate}[label = (\roman*)]
    \item $I$ $\langle$~метрически / топологически~$\rangle$ проективен как
    $A$-модуль;

    \item $I$ обладает $\langle$~правой единицей нормы $1$ / правой
    единицей~$\rangle$.
\end{enumerate}
\end{lemma} 
\begin{proof} $(i) \implies (ii)$ Если выполнено $(*)$ или $(**)$, то $I$
обладает односторонней аппроксимативной единицей. Следовательно, $I$ ---
существенный левый $I$-модуль и тем более существенный $A$-модуль. По
предложению~\ref{NonDegenMetTopProjCharac}, существует правый обратный
$A$-морфизм $\sigma:I\to A\projtens \ell_1(B_I)$ к $\pi_I$ в
$\langle$~$A-\mathbf{mod}_1$ / $A-\mathbf{mod}$~$\rangle$. Для каждого $d\in
B_I$ рассмотрим $A$-морфизм $p_d:A\projtens \ell_1(B_I)\to A:a\projtens
\delta_x\mapsto \delta_x(d)a$ и $\sigma_d=p_d\sigma$. Тогда
$\sigma(x)=\sum_{d\in B_I}\sigma_d(x)\projtens \delta_d$ для всех $x\in I$. Из
отождествления $A\projtens\ell_1(B_I)\isom{\mathbf{Ban}_1}\bigoplus_1 \{A:d\in
B_I \}$ мы имеем $\Vert\sigma(x)\Vert=\sum_{d\in B_I} \Vert\sigma_d(x)\Vert$ для
всех $x\in I$. Так как $\sigma$ --- правый обратный морфизм к $\pi_I$ то
$x=\pi_I(\sigma(x))=\sum_{d\in B_I}\sigma_d(x)d$ для всех $x\in I$. 

Предположим, выполнено условие $(*)$. Тогда для каждого $d\in B_I$ существует
морфизм правых $I$-модулей $\tau_d:I\to I$ такой, что $\sigma_d(x)d=x\tau_d(d)$
для всех $x\in I$.  Пусть ${(e_\nu)}_{\nu\in N}$ --- левая $\langle$~сжимающая /
ограниченная~$\rangle$ аппроксимативная единица в $I$ ограниченная по норме
константой $D$. Поскольку $\tau_d(d)\in I$ для всех $d\in B_I$, то для любого
множества $S\in\mathcal{P}_0(B_I)$ выполнено
$$
\sum_{d\in S}\Vert \tau_d(d)\Vert
=\sum_{d\in S}\lim_{\nu}\Vert e_\nu \tau_d(d) \Vert
=\lim_{\nu}\sum_{d\in S}\Vert e_\nu \tau_d(d)\Vert
=\lim_{\nu}\sum_{d\in S}\Vert \sigma_d(e_\nu)d \Vert
$$
$$
\leq\liminf_{\nu}\sum_{d\in S}\Vert\sigma_d(e_\nu)\Vert\Vert d\Vert 
\leq\liminf_{\nu}\sum_{d\in S}\Vert\sigma_d(e_\nu)\Vert
\leq\liminf_{\nu}\sum_{d\in B_I}\Vert\sigma_d(e_\nu)\Vert
$$
$$
=\liminf_{\nu}\Vert\sigma(e_\nu)\Vert
\leq\Vert\sigma\Vert\liminf_{\nu}\Vert e_\nu\Vert
\leq D\Vert\sigma\Vert.
$$

Теперь предположим что, выполнено условие $(**)$. Из предположения, для каждого
$d\in B_I$ существует морфизм правых $I$-модулей $\tau_d:I\to I$ такой, что
$\sigma_d(x)d=x\tau_d(d)$ для всех $x\in I$ и $\Vert\tau_d\Vert\leq
C\Vert\sigma_d\Vert$. Пусть ${(e_\nu)}_{\nu\in N}$ --- правая $\langle$~сжимающая
/ ограниченная~$\rangle$ аппроксимативная единица в $I$ ограниченная по норме
некоторой константой $D$. Для всех $x\in I$ выполнено
$$
\Vert\sigma_d(x)\Vert
=\Vert\sigma_d(\lim_\nu x e_\nu)\Vert
=\lim_\nu\Vert x\sigma_d(e_\nu)\Vert
\leq\Vert x\Vert\liminf_\nu\Vert\sigma_d(e_\nu)\Vert,
$$
поэтому $\Vert\sigma_d\Vert\leq \liminf_\nu\Vert\sigma_d(e_\nu)\Vert$. Тогда для
всех $S\in\mathcal{P}_0(B_I)$ выполнено
$$
\sum_{d\in S}\Vert \tau_d(d)\Vert
\leq \sum_{d\in S}\Vert \tau_d\Vert\Vert d\Vert
\leq C\sum_{d\in S}\Vert \sigma_d\Vert
\leq C\sum_{d\in S}\liminf_\nu \Vert \sigma_d(e_\nu)\Vert
\leq C\liminf_{\nu}\sum_{d\in S}\Vert \sigma_d(e_\nu) \Vert
$$
$$
\leq C\liminf_{\nu}\sum_{d\in B_I}\Vert \sigma_d(e_\nu) \Vert
=C\liminf_{\nu}\Vert\sigma(e_\nu)\Vert
\leq C\Vert\sigma\Vert\liminf_{\nu}\Vert e_\nu\Vert
\leq CD\Vert\sigma\Vert.
$$

Для обоих предположений $(*)$ и $(**)$ мы доказали, что число $\sum_{d\in
S}\Vert \tau_d(d)\Vert$ ограничено $\langle$~единицей / некоторой
константой~$\rangle$ для любого $S\in \mathcal{P}_0(B_I)$. Следовательно,
существует $p=\sum_{d\in B_I}\tau_d(d)\in I$ со свойством $\langle$~$\Vert
p\Vert\leq 1$ / $\Vert p\Vert< \infty$~$\rangle$. Более того, для всех $x\in I$
выполнено $x=\sum_{d\in B_I}\sigma_d(x)d=\sum_{d\in B_I}x\tau_d(d)=xp$, то есть
$p$ --- правая единица в $I$. 

$(ii) \implies(i)$ Пусть $p\in I$  --- правая единица для $I$, тогда $I=Ap$.
Теперь из предложения~\ref{UnIdeallIsMetTopProj} мы получаем, что идеал $I$
$\langle$~метрически / топологически~$\rangle$ проективен как $A$-модуль.
\end{proof}

Условие $(*)$ предыдущей леммы будет использовано в следующей теореме. Что
касается условия $(**)$, оно будет использовано значительно позже.

\begin{theorem}\label{GoodCommIdealMetTopProjIsUnital} Пусть $I$ --- идеал
коммутативной банаховой алгебры $A$ и $I$ имеет $\langle$~сжимающую /
ограниченную~$\rangle$ аппроксимативную единицу. Тогда $I$ $\langle$~метрически
/ топологически~$\rangle$ проективен как $A$-модуль тогда и только тогда, когда
$I$ имеет $\langle$~единицу нормы $1$ / единицу~$\rangle$.
\end{theorem} 
\begin{proof} Поскольку $A$ коммутативна, то для любого $A$-морфизма $\phi:I\to
A$ и любых $x,y\in I$ выполнено $\phi(x)y=x\phi(y)$. Так как $I$ имеет
ограниченную аппроксимативную единицу и $I$ коммутативен, то мы можем применить
лемму~\ref{ImgOfAMorphFromBiIdToA}, чтобы заключить $\phi(y)\in I$. Теперь
выполнено условие $(*)$ леммы~\ref{GoodIdealMetTopProjIsUnital}, и мы получаем
желаемую равносильность.
\end{proof}

В относительной теории нет аналогичного критерия проективности идеалов. Наиболее
общий результат такого типа дает лишь необходимое условие: если идеал $I$
коммутативной банаховой алгебры $A$ относительно проективен как $A$-модуль, то
$I$ имеет паракомпактный спектр. Этот результат получен Хелемским
[\cite{HelHomolBanTopAlg}, теорема IV.3.6]. 

Отметим, что существование ограниченной аппроксимативной единицы не является
необходимым условием для топологической проективности идеала коммутативной
банаховой алгебры. Действительно, рассмотрим банахову алгебру  $A_0(\mathbb{D})$
--- идеал алгебры на диске состоящий из функций исчезающих в нуле. Комбинируя
предложения 4.3.5 и 4.3.13 параграф $(iii)$ из~\cite{DalBanAlgAutCont} мы
заключаем, что $A_0(\mathbb{D})$ не имеет ограниченных аппроксимативных единиц.
С другой стороны, из [\cite{HelBanLocConvAlg}, пример IV.2.2] мы знаем, что
$A_0(\mathbb{D})\isom{A_0(\mathbb{D})-\mathbf{mod}} {A_0(\mathbb{D})}_+$, поэтому
согласно предложению~\ref{UnitalAlgIsMetTopProj}, $A_0(\mathbb{D})$ ---
топологически проективный $A_0(\mathbb{D})$-модуль.

Следующее предложение является очевидной модификацией описания алгебраически
проективных циклических модулей. Оно схоже с [\cite{WhiteInjmoduAlg},
предложение 2.11].

\begin{proposition}\label{MetTopProjCycModCharac} Пусть $I$ --- левый идеал в
$A_\times $. Тогда следующие условия эквивалентны:
\begin{enumerate}[label = (\roman*)]
    \item $A$-модуль $A_\times /I$ $\langle$~метрически /
    топологически~$\rangle$ проективен $\langle$~и естественное
    фактор-отображение $\pi:A_\times \to A_\times /I$ является строгой
    коизометрией /~$\rangle$;

    \item существует идемпотент $p\in I$ такой, что $I=A_\times  p$ $\langle$~и
    $\Vert e_{A_\times }-p\Vert= 1$ /~$\rangle$
\end{enumerate}
\end{proposition}
\begin{proof} $(i) \implies(ii)$ Поскольку отображение $\pi$ $\langle$~строго
коизометрично / топологически сюръективно~$\rangle$ и модуль $A_\times /I$
$\langle$~метрически / топологически~$\rangle$ проективен, то $\pi$ имеет правый
обратный морфизм $\sigma$, который $\langle$~изометричен / топологически
инъективен~$\rangle$. Положим, $e_{A_\times }-p=(\sigma\pi)(e_{A_\times })$,
тогда $(\sigma\pi)(a)=a(e_{A_\times }-p)$. По построению, 
$\pi\sigma=1_{A_\times }$, поэтому  
$$
e_{A_\times }-p
=(\sigma\pi)(e_{A_\times })
=(\sigma\pi)(\sigma\pi)(e_{A_\times })
=(\sigma\pi)(e_{A_\times }-p)
=(e_{A_\times }-p)(\sigma\pi)(e_{A_\times })
={(e_{A_\times }-p)}^2.
$$
Откуда следует, что $p^2=p$. Значит, $A_\times p=\operatorname{Ker}(\sigma\pi)$
так как $(\sigma\pi)(a)=a-ap$. Поскольку $\sigma$ инъективен, мы получаем
$A_\times p=\operatorname{Ker}(\pi)=I$. Наконец, заметим, что 
$\Vert e_{A_\times}-p\Vert
=\Vert(\sigma\pi)(e_{A_\times})\Vert
\leq\Vert\sigma\Vert\Vert\pi\Vert\Vert e_{A_\times }\Vert
=\Vert\sigma\Vert$.

$(ii) \implies(i)$ Пусть $p^2=p$, рассмотрим левый идеал $I=A_\times p$ и
морфизм $A$-модулей $\sigma:A_\times /I\to A_\times:a+I\mapsto a-ap$. Легко
проверить, что $\pi\sigma=1_{A_\times/I}$ и $\Vert\sigma\Vert\leq\Vert
e_{A_\times }-p\Vert$. Это означает, что $\pi:A_\times \to A_\times /I$ ---
ретракция в $\langle$~$A-\mathbf{mod}_1$ / $A-\mathbf{mod}$~$\rangle$ и, в
частности, $\langle$~строго коизометрический / топологически
сюръективный~$\rangle$ оператор. Теперь из
предложений~\ref{UnitalAlgIsMetTopProj} и~\ref{RetrMetTopProjIsMetTopProj}
следует, что $A$-модуль $A_\times /I$ $\langle$~метрически /
топологически~$\rangle$ проективен.
\end{proof} 

В отличие от топологической теории, в относительной теории нет полного описания
относительно проективных циклических модулей. Есть частичные ответы при
дополнительных предположениях. Например, если идеал $I$ дополняем в $A_\times$
как банахово пространство, то в относительной теории имеет место почти такой же
критерий [\cite{HelBanLocConvAlg}, предложение 7.1.29]. Существуют другие
описания относительно проективных циклических модулей при менее ограничительных
требованиях на банахову геометрию. Например, Селиванов доказал, что если $I$ ---
двусторонний идеал и либо $A/I$ имеет свойство аппроксимации либо все
неприводимые $A$-модули имеют свойство аппроксимации, то $A/I$  относительно
проективен тогда и только тогда, когда
$A_\times\isom{A-\mathbf{mod}}I\bigoplus_1 I'$ для некоторого левого идеала $I'$
в $A$. Подробности можно найти в [\cite{HelHomolBanTopAlg}, глава IV, \S 4].



%-------------------------------------------------------------------------------
%	Metric and topological injectivity
%-------------------------------------------------------------------------------

\subsection{
    Метрическая и топологическая инъективность
}\label{
    SubSectionMetricAndTopologicalInjectivity
}

В этом параграфе, если не оговорено иначе, мы будем считать все модули правыми.

\begin{definition}[\cite{HelMetrFrQMod}, определение 4.3]\label{MetInjMod}
$A$-модуль $J$ называется метрически инъективным, если для любого
изометрического $A$-морфизма $\xi:Y\to X$ и любого $A$-морфизма $\phi:Y\to J$
существует $A$-морфизм $\psi:X\to J$ такой, что $\psi\xi=\phi$  и
$\Vert\psi\Vert=\Vert\phi\Vert$.
\end{definition}

\begin{definition}[\cite{HelMetrFrQMod}, определение 4.3]\label{TopInjMod}
$A$-модуль $J$ называется топологически инъективным, если для любого
топологически инъективного $A$-морфизма $\xi:Y\to X$ и любого $A$-морфизма
$\phi:Y\to J$ существует $A$-морфизм $\psi:X\to J$ такой, что $\psi\xi=\phi$.
\end{definition}

Эквивалентные и более короткие определения звучат так: $A$-модуль $J$ называется
$\langle$~метрически / топологически~$\rangle$ инъективным, если функтор
$\langle$~$\operatorname{Hom}_{\mathbf{mod}_1-A}(-,J)
:\mathbf{mod}_1-A\to\mathbf{Ban}_1$
/
$\operatorname{Hom}_{\mathbf{mod}-A}(-,J):\mathbf{mod}-A\to\mathbf{Ban}$~$\rangle$
переводит $\langle$~изометрические / топологически инъективные~$\rangle$
$A$-морфизмы в $\langle$~строго коизометрические / сюръективные~$\rangle$
операторы.

В~\cite{HelMetrFrQMod} и~\cite{ShtTopFrClassicQuantMod} были построены два
верных функтора:
$$
\square_{met}^d:\mathbf{mod}_1-A\to\mathbf{Set}
:X\mapsto B_{X^*},\phi\mapsto\phi^*|_{B_{Y^*}}^{B_{X^*}},
$$
$$
\square_{top}^d:\mathbf{mod}-A\to\mathbf{HNor}:X\mapsto X^*,\phi\mapsto\phi^*.
$$
Первый из них отправляет банахов $A$-модуль в единичный шар своего сопряженного
пространства, а всякий сжимающий $A$-морфизм в соответствующее биограничение
своего сопряженного. Второй функтор ``забывает'' о модульной и аддитивной
структуре сопряженного пространства и сопряженного $A$-морфизма. В тех же
статьях было доказано, что, во-первых, $A$-морфизм $\xi$ $\langle$~изометричен /
топологически инъективен~$\rangle$ тогда и только тогда, когда он
$\langle$~$\square_{met}^d$-допустимый / $\square_{top}^d$-допустимый~$\rangle$
мономорфизм и, во-вторых, $A$-модуль $J$ $\langle$~метрически /
топологически~$\rangle$ инъективен тогда и только тогда, когда он
$\langle$~$\square_{met}^d$-инъективен / $\square_{top}^d$-инъективен~$\rangle$.
Таким образом, мы немедленно получаем следующее утверждение.

\begin{proposition}\label{RetrMetTopInjIsMetTopInj} Всякий ретракт
$\langle$~метрически / топологически~$\rangle$ инъективного модуля в
$\langle$~$\mathbf{mod}_1-A$ / $\mathbf{mod}-A$~$\rangle$ снова
$\langle$~метрически / топологически~$\rangle$ инъективен.
\end{proposition}

Также было доказано, что оснащенная категория
$\langle$~$(\mathbf{mod}_1-A,\square_{met}^d)$ /
$(\mathbf{mod}-A,\square_{top}^d)$~$\rangle$ косовободолюбива, и что
$\langle$~$\square_{met}^d$-косвободные /
$\square_{top}^d$-косвободные~$\rangle$ модули изоморфны в
$\langle$~$\mathbf{mod}_1-A$ / $\mathbf{mod}-A$~$\rangle$ модулям вида
$\mathcal{B}(A_+, \ell_\infty(\Lambda))$ для некоторого множества $\Lambda$.
Более того, для любого $A$-модуля $X$ существует
$\langle$~$\square_{met}^d$-допустимый / $\square_{top}^d$-допустимый~$\rangle$
мономорфизм
$$
\rho_X^+:X\to\mathcal{B}(A_+,\ell_\infty(B_{X^*}))
:x\mapsto(a\mapsto(f\mapsto f(x\cdot a))).
$$
Как следствие общих результатов об оснащенных категориях мы получаем следующее
предложение.

\begin{proposition}\label{MetTopInjModViaCanonicMorph} $A$-модуль $J$
$\langle$~метрически / топологически~$\rangle$ инъективен тогда и только тогда,
когда $\rho_J^+$ --- коретракция в $\langle$~$\mathbf{mod}_1-A$ /
$\mathbf{mod}-A$~$\rangle$.
\end{proposition}

Так как $\langle$~$\square_{met}^d$-косвободные /
$\square_{top}^d$-косвободные~$\rangle$ модули совпадают с точностью до
изоморфизма в $\mathbf{mod}-A$ и всякая ретракция в $\mathbf{mod}_1-A$ есть
ретракция в $\mathbf{mod}-A$, то из предложения~\ref{RetrMetTopInjIsMetTopInj}
мы видим, что любой метрически инъективный $A$-модуль топологически инъективен.
Напомним, что каждый относительно инъективный модуль есть ретракт в
$\mathbf{mod}-A$ модуля вида $\mathcal{B}(A_+,E)$ для некоторого банахова
пространства $E$. Следовательно, каждый топологически инъективный $A$-модуль
будет относительно инъективным. Мы резюмируем эти результаты в следующем
предложении.

\begin{proposition}\label{MetInjIsTopInjAndTopInjIsRelInj} Каждый метрически
инъективный модуль топологически инъективен, и каждый топологически инъективный
модуль относительно инъективен.
\end{proposition}

Количественный аналог определения топологической инъективности был дан Уайтом.

\begin{definition}[\cite{WhiteInjmoduAlg}, определение 3.4]\label{CTopInjMod}
$A$-модуль $J$ называется $C$-топологически инъективным, если для любого
$c$-топологически инъективного $A$-морфизма $\xi:Y\to X$ и любого $A$-морфизма
$\phi:Y\to J$ существует $A$-морфизм $\psi:X\to J$ такой, что $\psi\xi=\phi$ и
$\Vert\psi\Vert\leq cC\Vert\phi\Vert$.
\end{definition}

Нам понадобятся следующие два факта об этом типе инъективности.

\begin{proposition}[\cite{WhiteInjmoduAlg}, лемма
3.7]\label{RetrCTopInjIsCTopInj} Всякий $C_1$-ретракт $C_2$-топологически
инъективного модуля является $C_1C_2$-топологически инъективным.
\end{proposition}

\begin{proposition}[\cite{WhiteInjmoduAlg}, предложение
3.10]\label{CTopInjModViaCanonicMorph} $A$-модуль $J$ является $C$-топологически
инъективным тогда и только тогда, когда $\rho_J^+$ --- $C$-коретракция в
$\mathbf{mod}-A$.
\end{proposition}

Как следствие, банахов модуль топологически инъективен тогда и только тогда,
когда он $C$-топологически инъективен для некоторого $C$. Далее мы будем
использовать определения~\ref{TopInjMod} и~\ref{CTopInjMod} без ссылки на их
эквивалентность.

Теперь перейдем к обсуждению примеров. Заметим, что категория банаховых
пространств может рассматриваться как категория правых банаховых модулей над
нулевой алгеброй. Как следствие, мы получаем определение $\langle$~метрически /
топологически~$\rangle$ инъективного банахова пространства. Все результаты,
полученные выше, верны для этого типа инъективности. Эквивалентное определение
говорит, что банахово пространство $\langle$~метрически /
топологически~$\rangle$ инъективно, если оно $\langle$~$1$-дополняемо /
дополняемо~$\rangle$ в любом объемлющем банаховом пространстве. Стандартный
пример метрически инъективного банахова пространства это
$L_\infty$-пространство. На данный момент полностью описаны только метрически
инъективные банаховы пространства --- эти пространства изометрически изоморфны
$C(K)$-пространствам для некоторого экстремально несвязного компактного
хаусдорфова пространства $K$ [\cite{LaceyIsomThOfClassicBanSp}, теорема 3.11.6].
Обычно такие топологические пространства называются стоуновыми. Самые последние
достижения в изучении топологически инъективных банаховых пространств можно
найти в [\cite{JohnLinHandbookGeomBanSp}, глава 40].

\begin{proposition}\label{DualOfUnitalAlgIsMetTopInj} $A$-модуль $A_\times^*$
метрически и топологически инъективен. 
\end{proposition}
\begin{proof} Рассмотрим произвольный $A$-морфизм $\phi:Y\to A_\times^*$ и
изометрический $A$-морфизм $\xi:Y\to X$. Определим ограниченный линейный
функционал $f:Y\to\mathbb{C}:y\mapsto \phi(y)(e_{A_\times})$. Так как $\xi$ ---
изометрия, то по теореме Хана-Банаха мы может продолжить $f$ до некоторого
ограниченного линейного функционала $g:X\to\mathbb{C}$ с той же самой нормой,
что у $f$. Рассмотрим $A$-морфизм $\psi:X\to A_\times^*:x\mapsto (a\mapsto
g(x\cdot a))$. Очевидно, $\Vert\psi\Vert\leq\Vert g\Vert=\Vert
f\Vert\leq\Vert\phi\Vert$. С другой стороны $\psi\xi=\phi$, поэтому
$\Vert\phi\Vert\leq\Vert\psi\Vert\Vert\xi\Vert=\Vert\psi\Vert$. Таким образом,
$\Vert\phi\Vert=\Vert\psi\Vert$. Итак, мы доказали по определению, что
$A_\times^*$ --- метрически инъективный $A$-модуль. По
предложению~\ref{MetInjIsTopInjAndTopInjIsRelInj} он также топологически
инъективен.
\end{proof}

\begin{proposition}\label{NonDegenMetTopInjCharac}  Пусть $J$ --- верный
$A$-модуль. Тогда $J$ $\langle$~метрически / $C$-топологически~$\rangle$
инъективен тогда и только тогда, когда отображение
$\rho_J:J\to\mathcal{B}(A,\ell_\infty(B_{J^*})):x\mapsto(a\mapsto(f\mapsto
f(x\cdot a)))$ есть $\langle$~$1$-коретракция / $C$-коретракция~$\rangle$ в
$\mathbf{mod}-A$.
\end{proposition} 
\begin{proof}
Если $J$ $\langle$~метрически / $C$-топологически~$\rangle$ инъективен, то по
предложению $\langle$~\ref{MetTopInjModViaCanonicMorph}
/~\ref{CTopInjModViaCanonicMorph}~$\rangle$ $A$-морфизм $\rho_J^+$ имеет правый
обратный морфизм $\tau^+$ с нормой $\langle$~не более $1$ / не более
$C$~$\rangle$. Допустим нам задан оператор $T\in
\mathcal{B}(A_+,\ell_\infty(B_{J^*}))$ такой, что $T|_A=0$. Зафиксируем $a\in
A$, тогда $T\cdot a=0$, и поэтому $\tau^+(T)\cdot a=\tau^+(T\cdot a)=0$.
Поскольку $J$ --- верный модуль и $a\in A$ произвольно, то $\tau^+(T)=0$.
Рассмотрим естественную проекцию $p:A_+\to A$ и определим $A$-морфизм
$j=\mathcal{B}(p,\ell_\infty(B_{J^*}))$ и ограниченный линейный оператор
$\tau=\tau^+ j$. Для любого $a\in A$ и $T\in\mathcal{B}(A,\ell_\infty(B_{J^*}))$
мы имеем $\tau(T\cdot a)-\tau(T)\cdot a=\tau^+(j(T\cdot a)-j(T)\cdot a)=0$,
потому что $j(T\cdot a)-j(T)\cdot a|_A=0$. Значит $\tau$ --- $A$-морфизм.
Заметим, что $\Vert\tau\Vert\leq\Vert\tau^+\Vert\Vert j\Vert\leq
\Vert\tau^+\Vert$. Следовательно, $\tau$ имеет норму $\langle$~не более $1$ / не
более $C$~$\rangle$. Очевидно, для всех $x\in J$ выполнено
$\rho_J^+(x)-j(\rho_J(x))|_A=0$, поэтому $\tau^+(\rho_J^+(x)-j(\rho_J(x)))=0$.
Как следствие, $\tau(\rho_J(x))=\tau^+(j(\rho_J(x)))=\tau^+(\rho_J^+(x))=x$ для
всех $x\in J$. Так как $\tau\rho_J=1_J$, то $\rho_J$ ---
$\langle$~$1$-коретракция / $C$-коретракция~$\rangle$ в $\mathbf{mod}-A$.

Обратно, допустим $\rho_J$ --- $\langle$~$1$-коретракция /
$C$-коретракция~$\rangle$ в $\mathbf{mod}-A$, то есть $\rho_J$ имеет правый
обратный морфизм $\tau$ с нормой $\langle$~не более $1$ / не более
$C$~$\rangle$. Рассмотрим естественное вложение $i:A\to A_+$ и определим
$A$-морфизм $q=\mathcal{B}(i,\ell_\infty(B_{J^*}))$. Очевидно,
$\rho_J=q\rho_J^+$. Рассмотрим $A$-морфизм $\tau^+=\tau q$. Заметим, что
$\Vert\tau^+\Vert\leq\Vert\tau\Vert\Vert q\Vert\leq \Vert\tau\Vert$.
Следовательно, $\tau^+$ имеет норму $\langle$~не более $1$ / не более
$C$~$\rangle$. Очевидно, $\tau^+\rho_J^+=\tau q\rho_J^+=\tau\rho_J=1_J$. Поэтому
$\rho_J^+$ --- $\langle$~$1$-коретракция / $C$-коретракция~$\rangle$ в
$\mathbf{mod}-A$ и тогда по предложению
$\langle$~\ref{MetTopInjModViaCanonicMorph}
/~\ref{CTopInjModViaCanonicMorph}~$\rangle$ банахов $A$-модуль $J$
$\langle$~метрически / $C$-топологически~$\rangle$ инъективен.
\end{proof}

Следует напомнить, что $\langle$~произвольное / лишь конечное~$\rangle$
семейство объектов в $\langle$~$\mathbf{mod}_1-A$ / $\mathbf{mod}-A$~$\rangle$
обладает категорным произведением, которое на самом деле есть их
$\bigoplus_\infty$-сумма. В этом и состоит причина почему мы делаем
дополнительное предположение во втором пункте следующего предложения.

\begin{proposition}\label{MetTopInjModProd} Пусть
${(J_\lambda)}_{\lambda\in\Lambda}$ --- семейство банаховых $A$-модулей. Тогда 
\begin{enumerate}[label = (\roman*)]
    \item $A$-модуль $\bigoplus_\infty \{J_\lambda:\lambda\in\Lambda \}$
    метрически инъективен тогда и только тогда, когда для всех
    $\lambda\in\Lambda$ банахов $A$-модуль $J_\lambda$ метрически инъективен;

    \item $A$-модуль $\bigoplus_\infty \{J_\lambda:\lambda\in\Lambda \}$
    $C$-топологически инъективен тогда и только тогда, когда для всех
    $\lambda\in\Lambda$ банахов $A$-модуль $J_\lambda$ $C$-топологически
    инъективен.
\end{enumerate}
\end{proposition}
\begin{proof} Обозначим $J:=\bigoplus_\infty \{J_\lambda:\lambda\in\Lambda \}$.

$(i)$ Доказательство аналогично доказательству из пункта $(ii)$.

$(ii)$ Допустим, что $J$ $C$-топологически инъективен. Заметим, что для каждого
$\lambda\in\Lambda$ модуль $J_\lambda$ является $1$-ретрактом $J$ посредством
канонической проекции $p_\lambda:J\to J_\lambda$. По
предложению~\ref{RetrCTopInjIsCTopInj} модуль $J_\lambda$ $C$-топологически
инъективен.

Обратно, допустим, что для каждого $\lambda\in\Lambda$ модуль $J_\lambda$
$C$-топологически инъективен. По предложению~\ref{CTopInjModViaCanonicMorph} мы
имеем семейство $C$-коретракций
$\rho_\lambda:J_\lambda\to\mathcal{B}(A_+,\ell_\infty(S_\lambda))$.
Следовательно, $\bigoplus_\infty \{\rho_\lambda:\lambda\in\Lambda \}$ является
$C$-коретракцией в $A-\mathbf{mod}$. Значит, $J$ есть $C$-ретракт 
$$
\bigoplus\nolimits_\infty \{
    \mathcal{B}(A_+,\ell_\infty(S_\lambda)):\lambda\in\Lambda \}
\isom{\mathbf{mod}_1-A}
\bigoplus\nolimits_\infty\left \{
    \bigoplus\nolimits_\infty \{ 
        A_+^*:s\in S_\lambda \}:\lambda\in\Lambda\right \}
\isom{\mathbf{mod}_1-A}
$$
$$
\bigoplus\nolimits_\infty \{A_+^*:s\in S \}
\isom{\mathbf{mod}_1-A}
\mathcal{B}(A_+,\ell_\infty(S))
$$
в $\mathbf{mod}-A$, где $S=\bigsqcup_{\lambda\in\Lambda}S_\lambda$. Ясно, что
последний модуль $1$-топологически инъективен, поэтому из
предложения~\ref{RetrCTopInjIsCTopInj} следует, что $A$-модуль $J$
$C$-топологически инъективен.
\end{proof}

\begin{corollary}\label{MetTopInjlInftySum} Пусть $J$ --- банахов $A$-модуль и
$\Lambda$ --- произвольное множество. Тогда $A$-модуль  $\bigoplus_\infty
\{J:\lambda\in\Lambda \}$ $\langle$~метрически / топологически~$\rangle$
инъективен тогда и только тогда, когда $J$ $\langle$~метрически /
топологически~$\rangle$ инъективен.
\end{corollary}
\begin{proof} Для доказательства достаточно применить
предложение~\ref{MetTopInjModProd} c $J_\lambda=J$ для всех $\lambda\in\Lambda$.
\end{proof}

\begin{proposition}\label{MapsFroml1toMetTopInj} Пусть $J$ --- банахов
$A$-модуль и $\Lambda$ --- произвольное множество. Тогда $A$-модуль
$\mathcal{B}(\ell_1(\Lambda),J)$ $\langle$~метрически / топологически~$\rangle$
инъективен тогда и только тогда, когда $J$ $\langle$~метрически /
топологически~$\rangle$ инъективен.
\end{proposition}
\begin{proof} 
Допустим, $A$-модуль $\mathcal{B}(\ell_1(\Lambda), J)$  $\langle$~метрически /
топологически~$\rangle$ инъективен. Зафиксируем $\lambda\in\Lambda$ и рассмотрим
сжимающие $A$-морфизмы
$i_\lambda:J\to\mathcal{B}(\ell_1(\Lambda),J):x\mapsto(f\mapsto f(\lambda)x)$ и
$p_\lambda:\mathcal{B}(\ell_1(\Lambda),J)\to J:T\mapsto T(\delta_\lambda)$.
Очевидно, $p_\lambda i_\lambda=1_J$, то есть $J$ есть ретракт
$\mathcal{B}(\ell_1(\Lambda),J)$ в $\langle$~$\mathbf{mod}_1-A$ /
$\mathbf{mod}-A$~$\rangle$. Из предложения~\ref{RetrMetTopInjIsMetTopInj}
следует, что $A$-модуль $J$ $\langle$~метрически / топологически~$\rangle$
инъективен.

Обратно, поскольку $J$ $\langle$~метрически / топологически~$\rangle$
инъективен, то по предложению~\ref{MetTopInjModViaCanonicMorph} морфизм
$\rho_J^+$ является коретракцией в $\langle$~$\mathbf{mod}_1-A$ /
$\mathbf{mod}-A$~$\rangle$. Применим функтор $\mathcal{B}(\ell_1(\Lambda),-)$ к
этой коретракции, чтобы получить другую коретракцию
$\mathcal{B}(\ell_1(\Lambda),\rho_J^+)$. Заметим, что 
$$
\mathcal{B}(\ell_1(\Lambda),\ell_\infty(B_{J^*}))
\isom{\mathbf{Ban}_1}{(\ell_1(\Lambda)\projtens \ell_1(B_{J^*}))}^*
\isom{\mathbf{Ban}_1}{\ell_1(\Lambda\times B_{J^*})}^*
\isom{\mathbf{Ban}_1}\ell_\infty(\Lambda\times B_{J^*}),
$$ 
поэтому существует изометрический изоморфизм банаховых модулей:
$$
\mathcal{B}(\ell_1(\Lambda),\mathcal{B}(A_+,\ell_\infty(B_{J^*})))
\isom{\mathbf{mod}_1-A}\mathcal{B}(
    A_+,\mathcal{B}(\ell_1(\Lambda),\ell_\infty(B_{J^*})
)
\isom{\mathbf{mod}_1-A}\mathcal{B}(A_+,\ell_\infty(\Lambda\times B_{J^*})).
$$ 
Значит $\mathcal{B}(\ell_1(\Lambda),J)$ --- ретракт
$\mathcal{B}(A_+,\ell_\infty(\Lambda\times B_{J^*}))$ в
$\langle$~$\mathbf{mod}_1-A$ / $\mathbf{mod}-A$~$\rangle$, то есть ретракт
$\langle$~метрически / топологически~$\rangle$ инъективного $A$-модуля. По
предложению~\ref{RetrMetTopInjIsMetTopInj} $A$-модуль
$\mathcal{B}(\ell_1(\Lambda), J)$ $\langle$~метрически / топологически~$\rangle$
инъективен.
\end{proof}

%-------------------------------------------------------------------------------
%	Metric and topological flatness
%-------------------------------------------------------------------------------

\subsection{
    Метрическая и топологическая плоскость
}\label{
    SubSectionMetricAndTopologicalFlatness
}

Чтобы сохранить единый стиль обозначений мы будем называть метрически плоскими
$A$-модули статьи~\cite{HelMetrFlatNorMod}, где они были названы экстремально
плоскими.

\begin{definition}[\cite{HelMetrFlatNorMod}, I]\label{MetFlatMod} $A$-модуль $F$
называется метрически плоским, если для каждого изометрического $A$-морфизма
$\xi:X\to Y$ правых $A$-модулей оператор $\xi\projmodtens{A}
1_F:X\projmodtens{A} F\to Y\projmodtens{A} F$ изометричен.
\end{definition}

\begin{definition}[\cite{HelMetrFlatNorMod}, определение I]\label{TopFlatMod}
$A$-модуль $F$ называется топологически плоским, если для каждого топологически
инъективного $A$-морфизма $\xi:X\to Y$ правых $A$-модулей оператор
$\xi\projmodtens{A} 1_F:X\projmodtens{A} F\to Y\projmodtens{A} F$ топологически
инъективен.
\end{definition}

Эквивалентные и более короткие определения звучат так: $A$-модуль $J$ называется
$\langle$~метрически / топологически~$\rangle$ плоским, если функтор
$\langle$~$-\projmodtens{A} F:A-\mathbf{mod}_1\to\mathbf{Ban}_1$ /
$-\projmodtens{A} F:A-\mathbf{mod}\to\mathbf{Ban}$~$\rangle$ переводит
$\langle$~изометрические / топологически инъективные~$\rangle$ $A$-морфизмы в
$\langle$~изометрические / топологически инъективные~$\rangle$ операторы.

Снова рассмотрим категорию банаховых пространств как категорию левых банаховых
модулей над нулевой алгеброй, тогда мы получим определения $\langle$~метрически
/ топологически~$\rangle$ плоского банахова пространства. Из работы
Гротендика~\cite{GrothMetrProjFlatBanSp} следует, что любое метрически плоское
банахово пространство изометрически изоморфно $L_1(\Omega,\mu)$ для некоторого
пространства с мерой $(\Omega,\Sigma,\mu)$. Для топологически плоских банаховых
пространств, в отличие от топологически инъективных, мы также имеем критерий
[\cite{StegRethNucOpL1LInfSp}, теорема V.1]: банахово пространство топологически
плоское тогда и только тогда, когда оно является $\mathscr{L}_1$-пространством.

Хорошо известно, что $A$-модуль $F$ относительно плоский тогда и только тогда,
когда $F^*$ относительно инъективный [\cite{HelBanLocConvAlg}, теорема 7.1.42].
Следующее предложение есть очевидный аналог данного результата.

\begin{proposition}\label{MetTopFlatCharac} $A$-модуль $F$ $\langle$~метрически
/ топологически~$\rangle$ плоский тогда и только тогда, когда $F^*$
$\langle$~метрически / топологически~$\rangle$ инъективен.
\end{proposition}
\begin{proof} Рассмотрим произвольный $\langle$~изометрический / топологически
инъективный~$\rangle$ морфизм правых $A$-модулей, скажем, $\xi:X\to Y$. Оператор
$\xi\projmodtens{A} 1_F$ $\langle$~изометричен / топологически
инъективен~$\rangle$ тогда и только тогда, когда его сопряженный оператор
${(\xi\projmodtens{A} 1_F)}^*$ $\langle$~строго коизометричен / топологически
сюръективен~$\rangle$  [\cite{HelLectAndExOnFuncAn}, упражнения 4.4.6, 4.4.7].
Так как операторы ${(\xi\projmodtens{A} 1_F)}^*$ и $\mathcal{B}_A(\xi,F^*)$
эквивалентны в $\mathbf{Ban}_1$ посредством универсального свойства модульного
проективного тензорного произведения, то $\xi\projmodtens{A} 1_F$
$\langle$~изометричен / топологически инъективен~$\rangle$ тогда и только тогда,
когда $\mathcal{B}_A(\xi,F^*)$ $\langle$~строго коизометричен / топологически
сюръективен~$\rangle$. Поскольку морфизм $\xi$ произволен, мы видим, что $F$
$\langle$~метрически / топологически~$\rangle$ плоский тогда и только тогда,
когда $F^*$  $\langle$~метрически / топологически~$\rangle$ инъективный.
\end{proof}

Комбинируя предложение~\ref{MetTopFlatCharac} с
предложениями~\ref{RetrMetTopInjIsMetTopInj}
и~\ref{MetInjIsTopInjAndTopInjIsRelInj}, мы получаем следующие два факта.

\begin{proposition}\label{RetrMetTopFlatIsMetTopFlat} Всякий ретракт
$\langle$~метрически / топологически~$\rangle$ плоского модуля в
$\langle$~$A-\mathbf{mod}_1$ / $A-\mathbf{mod}$~$\rangle$ снова
$\langle$~метрически / топологически~$\rangle$ плоский.
\end{proposition}

\begin{proposition}\label{MetFlatIsTopFlatAndTopFlatIsRelFlat} Каждый метрически
плоский модуль топологический плоский, и каждый топологически плоский модуль
относительно плоский.
\end{proposition}

Количественный аналог определения топологической плоскости был дан Уайтом. Его
определение содержало ошибку, к счастью, не повлиявшую на основные результаты.
Мы берем на себя ответственность исправить эту ошибку.

\begin{definition}[\cite{WhiteInjmoduAlg}, определение 4.8]\label{CTopFlatMod}
$A$-модуль $F$ называется $C$-топологически плоским, если для каждого
$c$-топологически инъективного $A$-морфизма $\xi:X\to Y$ правых $A$-модулей
оператор $\xi\projmodtens{A} 1_F:X\projmodtens{A} F\to Y\projmodtens{A} F$
$cC$-топологически инъективен.
\end{definition}

Ключевым для нас будет следующий факт.

\begin{proposition}[\cite{WhiteInjmoduAlg}, лемма 4.10]\label{CTopFlatCharac}
$A$-модуль $F$ является $C$-топологически плоским тогда и только тогда, когда
$F^*$ $C$-топологически инъективен.
\end{proposition}

Как следствие, банахов модуль топологически плоский тогда и только тогда, когда
он $C$-топологически плоский для некоторого $C$. Далее мы будем использовать
определения~\ref{TopFlatMod} и~\ref{CTopFlatMod} без ссылки на их
эквивалентность. Из предложений~\ref{CTopFlatCharac}
и~\ref{RetrCTopInjIsCTopInj} мы получаем еще одно полезное предложение.

\begin{proposition}\label{RetrCTopFlatIsCTopFlat} Всякий $C_1$-ретракт
$C_2$-топологически плоского модуля является $C_1C_2$-топологически плоским.
\end{proposition}

\begin{proposition}\label{DualMetTopProjIsMetrInj} Пусть $P$ ---
$\langle$~метрически / топологически~$\rangle$ проективный $A$-модуль, и
$\Lambda$ --- произвольное множество. Тогда $A$-модуль
$\mathcal{B}(P,\ell_\infty(\Lambda))$ $\langle$~метрически /
топологически~$\rangle$ инъективен как $A$-модуль. В частности, $P^*$
$\langle$~метрически / топологически~$\rangle$ инъективен как $A$-модуль.
\end{proposition}
\begin{proof} Из предложения~\ref{MetTopProjModViaCanonicMorph} мы знаем, что
$\pi_P^+$ --- ретракция в $\langle$~$A-\mathbf{mod}_1$ /
$A-\mathbf{mod}$~$\rangle$. Тогда $A$-морфизм
$\rho^+=\mathcal{B}(\pi_P^+,\ell_\infty(\Lambda))$ есть коретракция в
$\langle$~$\mathbf{mod}_1-A$ / $\mathbf{mod}-A$~$\rangle$. Заметим, что
$\mathcal{B}(A_+\projtens\ell_1(B_P),\ell_\infty(\Lambda))
\isom{\mathbf{mod}_1-A}
\mathcal{B}(A_+,\mathcal{B}(\ell_1(B_P),\ell_\infty(\Lambda)))
\isom{\mathbf{mod}_1-A}
\mathcal{B}(A_+,\ell_\infty(B_P\times\Lambda))$.
Итак, мы показали, что существует коретракция из
$\mathcal{B}(P,\ell_\infty(\Lambda))$ в $\langle$~метрически /
топологически~$\rangle$ инъективный $A$-модуль. По
предложению~\ref{RetrMetTopInjIsMetTopInj} банахов $A$-модуль
$\mathcal{B}(P,\ell_\infty(\Lambda))$ является $\langle$~метрически /
топологически~$\rangle$ инъективным. Чтобы доказать последнее утверждение
достаточно положить $\Lambda=\mathbb{N}_1$.
\end{proof}

Как следствие предложений~\ref{MetTopFlatCharac}
и~\ref{DualMetTopProjIsMetrInj}, мы получаем следующее.

\begin{proposition}\label{MetTopProjIsMetTopFlat} Каждый $\langle$~метрически /
топологически~$\rangle$ проективный модуль является $\langle$~метрически /
топологически~$\rangle$ плоским.
\end{proposition}

Позже мы убедимся, что $\langle$~метрическая / топологическая~$\rangle$
плоскость --- это более слабое свойство, чем $\langle$~метрическая /
топологическая~$\rangle$ проективность.

\begin{proposition}\label{MetTopFlatModCoProd} Пусть
${(F_\lambda)}_{\lambda\in\Lambda}$ --- семейство банаховых $A$-модулей. Тогда: 

\begin{enumerate}[label = (\roman*)]
    \item $A$-модуль $\bigoplus_1 \{F_\lambda:\lambda\in\Lambda \}$ метрически
    плоский тогда и только тогда, когда для всех $\lambda\in\Lambda$ банахов
    $A$-модуль $F_\lambda$ метрически плоский;
    \item $A$-модуль $\bigoplus_1 \{F_\lambda:\lambda\in\Lambda \}$
    $C$-топологически плоский тогда и только тогда, когда для всех
    $\lambda\in\Lambda$ банахов $A$-модуль $F_\lambda$ $C$-топологически
    плоский.
\end{enumerate}
\end{proposition}
\begin{proof} По предложению  $\langle$~\ref{MetTopFlatCharac}
/~\ref{CTopFlatCharac}~$\rangle$ $A$-модуль $F$ $\langle$~метрически /
$C$-топологически~$\rangle$ плоский тогда и только тогда, когда $F^*$
$\langle$~метрически / $C$-топологически~$\rangle$ инъективен. Осталось
применить предложение~\ref{MetTopInjModProd} с $J_\lambda=F_\lambda^*$ для всех
$\lambda\in\Lambda$ и вспомнить, что 
${\left(\bigoplus_1 \{
    F_\lambda:\lambda\in\Lambda \}
\right)}^*
\isom{\mathbf{mod}_1-A}\bigoplus_\infty \{ F_\lambda^*:\lambda\in\Lambda \}$.
\end{proof}

%-------------------------------------------------------------------------------
%	Metric and topological flatness of ideals and cyclic modules
%-------------------------------------------------------------------------------

\subsection{
    Метрическая и топологическая плоскость идеалов и циклических модулей
}\label{
    SubSectionMetricAndTopologicalFlatnessOfIdealsAndCyclicModules
}

В этом параграфе мы обсудим условия, при которых идеалы и циклические модули
будут метрически и топологически плоскими. Доказательства во многом схожи с
подходами использованными при изучении относительной плоскости идеалов и
циклических модулей.

\begin{proposition}\label{MetTopFlatIdealsInUnitalAlg} Пусть $I$ --- левый идеал
в $A_\times $ и $I$ имеет правую $\langle$~сжимающую / ограниченную~$\rangle$
аппроксимативную единицу. Тогда $A$-модуль $I$ $\langle$~метрически /
топологически~$\rangle$ плоский.
\end{proposition}
\begin{proof} Пусть $\xi:X\to Y$ --- $\langle$~изометрический / топологически
инъективный~$\rangle$ морфизм правых $A$-модулей. Так как $I$ имеет правую
$\langle$~сжимающую / ограниченную~$\rangle$ аппроксимативную единицу, то из
[\cite{HelBanLocConvAlg}, предложение 6.3.24] следует, что линейные операторы
$i_{X,I}:X\projmodtens{A} I\to \operatorname{cl}_X(XI):x\projmodtens{A} a\mapsto
x\cdot a$, $i_{Y,I}:Y\projmodtens{A} I\to
\operatorname{cl}_Y(YI):y\projmodtens{A} a\mapsto y\cdot a$ суть
$\langle$~изометрические изоморфизмы / топологические изоморфизмы~$\rangle$
банаховых пространств. Очевидно, оператор $i_0=i_{Y,I}(\xi\projmodtens{A}
1_I)i_{X,I}^{-1}$, поточечно совпадает с $\xi$. Следовательно, $i_0$ ---
$\langle$~изометрически / топологически инъективный~$\rangle$ оператор и таковым
будет $\xi\projmodtens{A} 1_I$, потому что он изометрически эквивалентен $i_0$.
Поскольку морфизм $\xi$ произволен, $A$-модуль $I$ $\langle$~метрически /
топологически~$\rangle$ плоский. 
\end{proof}

Отметим, что такое же достаточное условие относительной плоскости идеалов есть и
в относительной теории [\cite{HelBanLocConvAlg}, предложение 7.1.45]. Теперь мы
можем дать пример метрически плоского модуля, который не является даже
топологически проективным. Очевидно, $\ell_\infty(\mathbb{N})$-модуль
$c_0(\mathbb{N})$ не унитален как идеал алгебры $\ell_\infty(\mathbb{N})$, но
имеет сжимающую аппроксимативную единицу. По
теореме~\ref{GoodCommIdealMetTopProjIsUnital} этот модуль не является
топологически проективным, но он метрически плоский по
предложению~\ref{MetTopFlatIdealsInUnitalAlg}.

``Метрическая'' часть следующего предложения есть небольшая модификация и
восполнение пробелов в [\cite{WhiteInjmoduAlg}, предложение 4.11]. Случай
топологической плоскости идеалов был изучен Хелемским в
[\cite{HelHomolBanTopAlg}, теорема VI.1.20].

\begin{proposition}\label{MetTopFlatCycModCharac} Пусть $I$ --- левый
собственный идеал в $A_\times $. Тогда следующие условия эквивалентны:
\begin{enumerate}[label = (\roman*)]
    \item $A$-модуль $A_\times /I$ $\langle$~метрически /
    топологически~$\rangle$ плоский;

    \item $I$ имеет правую ограниченную аппроксимативную единицу
    ${(e_\nu)}_{\nu\in N}$ $\langle$~такую, что 
    $\sup_{\nu\in N}\Vert 
        e_{A_\times}-e_\nu
    \Vert\leq 1$ /~$\rangle$.
\end{enumerate}
\end{proposition}
\begin{proof} $(i) \implies(ii)$  Так как $A_\times /I$ $\langle$~метрически /
топологически~$\rangle$ плоский, то по предложению~\ref{MetTopFlatCharac} правый
$A$-модуль ${(A_\times /I)}^*$ является $\langle$~метрически /
топологически~$\rangle$ инъективным. Пусть $\pi:A_\times \to A_\times /I$ ---
естественная проекция, тогда $\pi^*:{(A_\times /I)}^*\to A_\times ^*$ является
изометрией. Поскольку ${(A_\times /I)}^*$ $\langle$~метрически /
топологически~$\rangle$ инъективен, оператор $\pi^*$ --- коретракция, то есть
существует $\langle$~строго коизометрический / топологически
сюръективный~$\rangle$ $A$-морфизм $\tau:A_\times ^*\to {(A_\times /I)}^*$ такой,
что $\tau\pi^*=1_{{(A_\times /I)}^*}$. Рассмотрим элемент $p\in A^{**}$ такой, что
$\iota_{A_\times }(e_{A_\times })-p=\tau^*(\pi^{**}(\iota_{A_\times
}(e_{A_\times })))$. Рассмотрим произвольный $f\in I^\perp$. Так как
$I^\perp=\pi^*({(A_\times /I)}^*)$, то существует $g\in {(A_\times /I)}^*$ со
свойством $f=\pi^*(g)$. Значит,
$$
(\iota_{A_\times }(e_{A_\times })-p)(f)
=\tau^*(\pi^{**}(\iota_{A_\times }(e_{A_\times })))(\pi^*(g))
=\pi^{**}(\iota_{A_\times }(e_{A_\times }))(\tau(\pi^*(g)))
$$
$$
=\pi^{**}(\iota_{A_\times }(e_{A_\times }))(g)
=\iota_{A_\times }(e_{A_\times })(\pi^*(g))
=\iota_{A_\times }(e_{A_\times })(f).
$$
Следовательно, $p(f)=0$ для всех $f\in I^\perp$, то есть $p\in I^{\perp\perp}$.
Напомним, что $I^{\perp\perp}$ есть слабое${}^*$ замыкание $I$ в $A^{**}$,
поэтому существует направленность ${(e_\nu'')}_{\nu\in N''}\subset I$ такая, что
${(\iota_I(e_\nu''))}_{\nu\in N''}$ сходится к $p$ в слабой${}^*$ топологии.
Очевидно, ${(\iota_{A_\times }(e_{A_\times }-e_\nu''))}_{\nu\in N''}$ сходится к
$\iota_{A_\times }(e_{A_\times })-p$ в этой же топологии. Из
[\cite{PosAndApproxIdinBanAlg}, лемма 1.1] следует, что существует
направленность в выпуклой оболочке 
$\operatorname{conv}{(\iota_{A_\times}(e_{A_\times }-e_\nu''))}_{\nu\in N''}
=\iota_{A_\times }(e_{A_\times})
-\operatorname{conv}{(\iota_{A_\times }(e_\nu''))}_{\nu\in N''}$ которая
слабо${}^*$ сходится к $\iota_{A_\times }(e_{A_\times })-p$ и ограничена по
норме числом $\Vert \iota_{A_\times }(e_{A_\times })-p\Vert$. Обозначим эту
направленность как 
${(\iota_{A_\times }(e_{A_\times })-\iota_{A_\times}(e_\nu'))}_{\nu\in N'}$, 
тогда ${(\iota_{A_\times }(e_\nu'))}_{\nu\in N'}$
слабо${}^*$ сходится к $p$. Для любого $a\in I$ и $f\in I^*$ мы имеем
$$
\lim_{\nu}f(ae_\nu')
=\lim_{\nu}\iota_{A_\times }(e_\nu')(f\cdot a)
=p(f\cdot a)
=\iota_{A_\times }(e_{A_\times })(f\cdot a)
-\tau^*(\pi^{**}(\iota_{A_\times }(e_{A_\times })))(f\cdot a)
$$
$$
=f(a)-\iota_{A_\times }(e_{A_\times })(\pi^*(\tau(f\cdot a)))
=f(a)-\pi^*(\tau(f)\cdot a)(e_{A_\times })
=f(a)-\tau(f)(\pi(a))
=f(a),
$$
поэтому ${(e_\nu')}_{\nu\in N'}$ --- правая слабая ограниченная аппроксимативная
единица для $I$. Из [\cite{AppIdAndFactorInBanAlg}, предложение 33.2] мы
получаем, что существует направленность 
${(e_\nu)}_{\nu\in N}\subset\operatorname{conv}{(e_\nu')}_{\nu\in N'}$ 
которая является правой
ограниченной аппроксимативной единицей для $I$. Для любого $\nu\in N$ мы имеем
$e_{A_\times }-e_\nu\in\operatorname{conv}{(e_{A_\times }-e_\nu')}_{\nu\in N'}$,
поэтому учитывая ограничение на нормы элементов направленности 
${(\iota_{A_\times}(e_{A_\times }-e_\nu'))}_{\nu\in N'}$ мы получаем
$$
\sup_{\nu\in N}\Vert e_{A_\times }-e_\nu\Vert
\leq\Vert \iota_{A_\times }(e_{A_\times })-p\Vert
\leq\Vert\tau^*(\pi^{**}(\iota_{A_\times }(e_{A_\times })))\Vert
\leq\Vert\tau^*\Vert\Vert\pi^{**}\Vert\Vert\iota_{A_\times }(e_{A_\times })\Vert
=\Vert\tau\Vert.
$$
Так как $\tau$ $\langle$~сжимающий / ограниченный~$\rangle$ морфизм, то мы
получаем желаемую оценку. По построению, ${(e_\nu)}_{\nu\in N}$ --- правая
ограниченная аппроксимативная единица для $I$.

$(ii) \implies (i)$ Обозначим $C=\sup_{\nu\in N}\Vert e_{A_\times
}-e_\nu\Vert$. Пусть $\mathfrak{F}$ --- фильтр сечений на $N$ и пусть
$\mathfrak{U}$ --- ультрафильтр содержащий $\mathfrak{F}$. Для фиксированного
$f\in A_\times ^*$ и $a\in A_\times $ выполнено $|f(a-a e_\nu)|=|f(a(e_{A_\times
}-e_\nu))|\leq\Vert f\Vert\Vert a\Vert\Vert e_{A_\times }-e_\nu\Vert\leq C\Vert
f\Vert\Vert a\Vert$, то есть ${(f(a-ae_\nu))}_{\nu\in N}$ --- ограниченная
направленность комплексных чисел. Следовательно, корректно определен предел
$\lim_{\mathfrak{U}}f(a-ae_\nu)$ по ультрафильтру $\mathfrak{U}$. Так как
${(e_\nu)}_{\nu\in N}$ --- правая аппроксимативная единица для $I$ и
$\mathfrak{U}$ содержит фильтр сечений, то для всех $a\in I$ выполнено
$\lim_{\mathfrak{U}}f(a-ae_\nu)=\lim_{\nu}f(a-ae_\nu)=0$. Таким образом, для
каждого $f\in A_\times ^*$ корректно определено отображение $\tau(f):A_\times
/I\to \mathbb{C}:a+I\mapsto \lim_{\mathfrak{U}} f(a-ae_\nu)$. Очевидно, это
линейный функционал и из неравенств доказанных выше следует, что его норма не
превосходит $C\Vert f\Vert$. Теперь легко проверить, что 
$\tau:A_\times ^*\to {(A_\times /I)}^*
:f\mapsto \tau(f)$ есть $\langle$~сжимающий /
ограниченный~$\rangle$ $A$-морфизм. Для всех 
$g\in{(A_\times /I)}^*$ и $a+I\in A_\times /I$ имеем
$$
\tau(\pi^*(g))(a+I)
=\lim_{\mathfrak{U}}\pi^*(g)(a-ae_\nu)
=\lim_{\mathfrak{U}} g(\pi(a-ae_\nu))
=\lim_{\mathfrak{U}} g(a+I)
=g(a+I),
$$
то есть $\tau:A_\times ^*\to {(A_\times /I)}^*$ --- ретракция. Правый $A$-модуль
$A_\times ^*$ $\langle$~метрически / топологически~$\rangle$ инъективен по
предложению~\ref{DualOfUnitalAlgIsMetTopInj}, поэтому его ретракт 
${(A_\times /I)}^*$ также 
$\langle$~метрически / топологически~$\rangle$ инъективен. Теперь
предложение~\ref{MetTopFlatCharac} гарантирует $\langle$~метрическую /
топологическую~$\rangle$ плоскость $A$-модуля $A_\times /I$.
\end{proof}

Следует сказать, что всякая операторная алгебра $A$ (не обязательно
самосопряженная) обладающая сжимающей аппроксимативной единицей имеет сжимающую
аппроксимативную единицу ${(e_\nu)}_{\nu\in N}$ со свойством $\sup_{\nu\in N}\Vert
e_{A_\#}-e_\nu\Vert\leq 1$ и даже $\sup_{\nu\in N}\Vert e_{A_\#}-2e_\nu\Vert\leq
1$. Здесь $A_\#$ --- унитизация $A$ как операторной алгебры. Подробности можно
найти в~\cite{PosAndApproxIdinBanAlg},~\cite{BleContrAppIdInOpAlg}.

Снова мы попробуем сравнить наши результаты о метрической и топологической
плоскости циклических модулей с их относительными аналогами. Хелемский и
Шейнберг показали [\cite{HelHomolBanTopAlg}, теорема VII.1.20], что циклический
модуль будет относительно плоским если $I$ имеет правую ограниченную
аппроксимативную единицу. В случае когда $I^\perp$ дополняемо в $A_\times^*$
верна и обратная импликация. В топологической теории это требование излишне,
поэтому удается получить критерий. Метрическая плоскость циклических модулей
слишком сильное свойство из-за специфических ограничений на норму
аппроксимативной единицы. Как мы увидим в следующем параграфе, оно настолько
ограничительное, что не позволяет построить ни одного ненулевого аннуляторного
метрически проективного, инъективного или плоского модуля над ненулевой
банаховой алгеброй.

%-------------------------------------------------------------------------------
%	The impact of Banach geometry
%-------------------------------------------------------------------------------

\section{Влияние банаховой геометрии}\label{SectionTheImpactOfBanachGeometry}


%-------------------------------------------------------------------------------
%	Homologically trivial annihilator modules
%-------------------------------------------------------------------------------

\subsection{
    Гомологически тривиальные аннуляторные модули
}\label{
    SubSectionHomoligicallyTrivialAnnihilatorModules
}

В этом параграфе мы сконцентрируем наше внимание на метрической и топологической
проективности, инъективности и плоскости аннуляторных модулей. Если не оговорено
иначе, все банаховы пространства в этом параграфе рассматриваются как
аннуляторные модули. Отметим очевидный факт: всякий ограниченный линейный
оператор между аннуляторными $A$-модулями является $A$-морфизмом.

\begin{proposition}\label{AnnihCModIsRetAnnihMod} Пусть $X$ --- ненулевой
аннуляторный $A$-модуль. Тогда $\mathbb{C}$ есть ретракт $X$ в
$A-\mathbf{mod}_1$.
\end{proposition}
\begin{proof} Рассмотрим произвольный вектор $x_0\in X$ нормы $1$. Используя
теорему Хана-Банаха выберем функционал $f_0\in X^*$ так, чтобы $\Vert
f_0\Vert=f_0(x_0)=1$. Рассмотрим линейные операторы $\pi:X\to
\mathbb{C}:x\mapsto f_0(x)$, $\sigma:\mathbb{C}\to X:z\mapsto zx_0$. Легко
проверить, что $\pi$ и $\sigma$ суть сжимающие $A$-морфизмы и, более того,
$\pi\sigma=1_\mathbb{C}$. Другими словами, $\mathbb{C}$ есть ретракт $X$ в
$A-\mathbf{mod}_1$.
\end{proof}

Пришло время вспомнить, что любая банахова алгебра $A$ может рассматриваться как
собственный максимальный идеал в $A_+$, причем
$\mathbb{C}\isom{A-\mathbf{mod}_1} A_+/A$. Если рассматривать $\mathbb{C}$ как
правый аннуляторный $A$-модуль, то имеет место еще один изоморфизм
$\mathbb{C}\isom{\mathbf{mod}_1-A}{(A_+/A)}^*$. 

\begin{proposition}\label{MetTopProjModCCharac} Аннуляторный $A$-модуль
$\mathbb{C}$ $\langle$~метрически / топологически~$\rangle$ проективен тогда и
только тогда, когда $\langle$~$A= \{0 \}$ / $A$ имеет правую единицу~$\rangle$.
\end{proposition}
\begin{proof} 
Достаточно исследовать $\langle$~метрическую / топологическую~$\rangle$
проективность модуля $A_+/A$. Естественное фактор-отображение $\pi:A_+\to A_+/A$
является строгой коизометрией, поэтому по
предложению~\ref{MetTopProjCycModCharac} $\langle$~метрическая /
топологическая~$\rangle$ проективность $A_+/A$ эквивалентна существованию $p\in
A$ такого, что $A=A_+p$ $\langle$~и $\Vert e_{A_+}-p\Vert=1$ /~$\rangle$.
$\langle$~Осталось заметить, что $\Vert e_{A_+}-p\Vert=1$ тогда и только тогда,
когда $p=0$, что эквивалентно $A=A_+p= \{0 \}$ /~$\rangle$.
\end{proof}

\begin{proposition}\label{MetTopProjOfAnnihModCharac} Пусть $P$ --- ненулевой
аннуляторный $A$-модуль. Тогда следующие условия эквивалентны:
\begin{enumerate}[label = (\roman*)]
    \item $P$ --- $\langle$~метрически / топологически~$\rangle$ проективный
    $A$-модуль;

    \item $\langle$~$A= \{0 \}$ / $A$ имеет правую единицу~$\rangle$ и $P$ ---
    $\langle$~метрически / топологически~$\rangle$ проективное банахово
    пространство, то есть $\langle$~$P\isom{\mathbf{Ban}_1}\ell_1(\Lambda)$ /
    $P\isom{\mathbf{Ban}}\ell_1(\Lambda)$~$\rangle$ для некоторого множества
    $\Lambda$.
\end{enumerate}
\end{proposition}
\begin{proof} $(i) \implies (ii)$ Из
предложений~\ref{RetrMetTopProjIsMetTopProj} и~\ref{AnnihCModIsRetAnnihMod}
следует, что $A$-модуль $\mathbb{C}$ $\langle$~метрически /
топологически~$\rangle$ проективен как ретракт $\langle$~метрически /
топологически~$\rangle$ проективного модуля $P$.
Предложение~\ref{MetTopProjModCCharac} дает, что $\langle$~$A= \{0 \}$ / $A$
имеет правую единицу~$\rangle$.  По следствию~\ref{MetTopProjTensProdWithl1}
аннуляторный $A$-модуль
$\mathbb{C}\projtens\ell_1(B_P)\isom{A-\mathbf{mod}_1}\ell_1(B_P)$
$\langle$~метрически / топологически~$\rangle$ проективен. Рассмотрим строгую
коизометрию $\pi:\ell_1(B_P)\to P$ корректно определенную равенством
$\pi(\delta_x)=x$. Поскольку $P$ и $\ell_1(B_P)$ --- аннуляторные модули, то
$\pi$ --- $A$-морфизм. Так как $P$ $\langle$~метрически /
топологически~$\rangle$ проективен, то $A$-морфизм $\pi$ имеет правый обратный
морфизм $\sigma$ в $\langle$~$A-\mathbf{mod}_1$ / $A-\mathbf{mod}$~$\rangle$.
Таким образом, $\sigma\pi$ есть $\langle$~сжимающий / ограниченный~$\rangle$
проектор из $\langle$~метрически / топологически~$\rangle$ проективного банахова
пространства $\ell_1(B_P)$ на $P$, то есть $P$ --- $\langle$~метрически /
топологически~$\rangle$ проективное банахово пространство. Теперь из
$\langle$~[\cite{HelMetrFrQMod}, предложение 3.2] /
результатов~\cite{KotheTopProjBanSp}~$\rangle$ следует, что пространство $P$
$\langle$~метрически / топологически~$\rangle$ изоморфно $\ell_1(\Lambda)$ для
некоторого множества $\Lambda$. 

$(ii) \implies (i)$ По предложению~\ref{MetTopProjModCCharac} аннуляторный
$A$-модуль $\mathbb{C}$ $\langle$~метрически / топологически~$\rangle$
проективен. По следствию~\ref{MetTopProjTensProdWithl1} аннуляторный $A$-модуль
$\mathbb{C}\projtens\ell_1(\Lambda)\isom{A-\mathbf{mod}_1}\ell_1(\Lambda)$ также
$\langle$~метрически / топологически~$\rangle$ проективен.
\end{proof}

\begin{proposition}\label{MetTopInjModCCharac} Правый аннуляторный $A$-модуль
$\mathbb{C}$ $\langle$~метрически / топологически~$\rangle$ инъективен тогда и
только тогда, когда $\langle$~$A= \{0 \}$ / $A$  имеет правую ограниченную
аппроксимативную единицу~$\rangle$.
\end{proposition}
\begin{proof} Благодаря предложению~\ref{MetTopFlatCharac} достаточно изучить
$\langle$~метрическую / топологическую~$\rangle$ плоскость модуля $A_+/A$. По
предложению~\ref{MetTopFlatCycModCharac} это эквивалентно существованию правой
ограниченной аппроксимативной единицы ${(e_\nu)}_{\nu\in N}$ в $A$ $\langle$~со
свойством $\sup_{\nu\in N}\Vert e_{A_+}-e_\nu\Vert\leq 1$ /~$\rangle$.
$\langle$~Осталось заметить, что $\Vert e_{A_+}-e_\nu\Vert\leq 1$ тогда и только
тогда, когда $e_\nu=0$, что эквивалентно $A= \{0 \}$ /~$\rangle$.
\end{proof}

\begin{proposition}\label{MetTopInjOfAnnihModCharac} Пусть $J$ --- ненулевой
правый аннуляторный $A$-модуль. Тогда следующие условия эквивалентны:
\begin{enumerate}[label = (\roman*)]
    \item $J$ --- $\langle$~метрически / топологически~$\rangle$ инъективный
    $A$-модуль;

    \item $\langle$~$A= \{0 \}$ / $A$ имеет правую ограниченную аппроксимативную
    единицу~$\rangle$ и $J$ ---  $\langle$~метрически / топологически~$\rangle$
    инъективное банахово пространство $\langle$~то есть
    $J\isom{\mathbf{Ban}_1}C(K)$ для некоторого для стоунова пространства $K$
    /~$\rangle$.
\end{enumerate}
\end{proposition}
\begin{proof} $(i) \implies (ii)$  Из
предложений~\ref{RetrMetTopInjIsMetTopInj} и~\ref{AnnihCModIsRetAnnihMod} мы
получаем, что $A$-модуль $\mathbb{C}$ $\langle$~метрически /
топологически~$\rangle$ инъективен как ретракт $\langle$~метрически /
топологически~$\rangle$ инъективного модуля $J$.
Предложение~\ref{MetTopInjModCCharac} дает нам, что $\langle$~$A= \{0 \}$ / $A$
имеет правую ограниченную аппроксимативную единицу~$\rangle$. По
предложению~\ref{MapsFroml1toMetTopInj} аннуляторный $A$-модуль
$\mathcal{B}(\ell_1(B_{J^*}),\mathbb{C})\isom{\mathbf{mod}_1-A}\ell_\infty(B_{J^*})$
$\langle$~метрически / топологически~$\rangle$ инъективен. Рассмотрим изометрию
$\rho:J\to\ell_\infty(B_{J^*})$ корректно определенную равенством
$\rho(x)(f)=f(x)$. Так как $J$ и $\ell_\infty(B_{J^*})$ --- аннуляторные модули,
то $\rho$ является $A$-морфизмом. Поскольку $J$ $\langle$~метрически /
топологически~$\rangle$ инъективен, $\rho$ имеет левый обратный морфизм $\tau$ в
$\langle$~$\mathbf{mod}_1-A$ / $\mathbf{mod}-A$~$\rangle$. Тогда $\rho\tau$ ---
$\langle$~сжимающий / ограниченный~$\rangle$ проектор из $\langle$~метрически /
топологически~$\rangle$ инъективного банахова пространства
$\ell_\infty(B_{J^*})$ на $J$, поэтому $J$ также является $\langle$~метрически /
топологически~$\rangle$ инъективным банаховым пространством. $\langle$~Из
[\cite{LaceyIsomThOfClassicBanSp}, теорема 3.11.6] мы знаем, что $J$
изометрически изоморфно $C(K)$ для некоторого стоунова пространства $K$.
/~$\rangle$ 

$(ii) \implies(i)$ По предложению~\ref{MetTopInjModCCharac} аннуляторный
$A$-модуль $\mathbb{C}$ $\langle$~метрически / топологически~$\rangle$
инъективен. По предложению~\ref{MapsFroml1toMetTopInj} аннуляторный $A$-модуль
$\mathcal{B}(\ell_1(B_{J^*}),\mathbb{C})\isom{\mathbf{mod}_1-A}\ell_\infty(B_{J^*})$
также $\langle$~метрически / топологически~$\rangle$ инъективен. Так как $J$ ---
$\langle$~метрически / топологически~$\rangle$ инъективное банахово пространство
и существует изометрическое вложение $\rho:J\to \ell_\infty(B_{J^*})$, то $J$
является $\langle$~$1$-ретрактом / ретрактом~$\rangle$ пространства
$\ell_\infty(B_{J^*})$. Напомним, что $J$ и $\ell_\infty(B_{J^*})$ аннуляторные
модули, поэтому данная ретракция также является ретракцией в
$\langle$~$\mathbf{mod}_1-A$ / $\mathbf{mod}-A$~$\rangle$. По
предложению~\ref{RetrMetTopInjIsMetTopInj} $A$-модуль $J$ $\langle$~метрически /
топологически~$\rangle$ инъективен.
\end{proof}

\begin{proposition}\label{MetTopFlatAnnihModCharac} Пусть $F$ --- ненулевой
аннуляторный $A$-модуль. Тогда следующие условия эквивалентны:
\begin{enumerate}[label = (\roman*)]
    \item $F$ --- $\langle$~метрически / топологически~$\rangle$ плоский
    $A$-модуль;

    \item $\langle$~$A= \{0 \}$ / $A$ имеет правую ограниченную аппроксимативную
    единицу~$\rangle$ и $F$ --- $\langle$~метрически / топологически~$\rangle$
    плоское банахово пространство, то есть
    $\langle$~$F\isom{\mathbf{Ban}_1}L_1(\Omega,\mu)$ для некоторого
    пространства с мерой $(\Omega, \Sigma, \mu)$ / $F$ есть
    $\mathscr{L}_1$-пространство~$\rangle$.
\end{enumerate}
\end{proposition}
\begin{proof} Из $\langle$~[\cite{GrothMetrProjFlatBanSp}, теорема 1] /
[\cite{StegRethNucOpL1LInfSp}, теорема VI.6]~$\rangle$ мы знаем, что банахово
пространство $F^*$ $\langle$~метрически / топологически~$\rangle$ инъективно
тогда и только тогда, когда $\langle$~$F\isom{\mathbf{Ban}_1}L_1(\Omega,\mu)$
для некоторого пространства с мерой $(\Omega, \Sigma, \mu)$ / $F$ есть
$\mathscr{L}_1$-пространство~$\rangle$. Теперь эквивалентность следует из
предложений~\ref{MetTopInjOfAnnihModCharac} и~\ref{MetTopFlatCharac}.
\end{proof}

Следует сравнить эти результаты с аналогичными в относительной теории. Из
$\langle$~[\cite{RamsHomPropSemgroupAlg}, предложение 2.1.7] /
[\cite{RamsHomPropSemgroupAlg}, предложение 2.1.10]~$\rangle$ мы знаем, что
аннуляторный модуль над банаховой алгеброй $A$  относительно
$\langle$~проективный / плоский~$\rangle$ тогда и только тогда, когда $A$ имеет
$\langle$~правую единицу / правую ограниченную аппроксимативную
единицу~$\rangle$. В метрической и топологической теории, в отличие от
относительной, гомологическая тривиальность аннуляторных модулей налагает
ограничения не только на алгебру, но и на геометрию самого модуля. Эти
геометрические ограничения запрещают существование некоторых гомологически
лучших банаховых алгебр. Одно из важных свойств относительно
$\langle$~стягиваемых / аменабельных~$\rangle$ банаховых алгебр --- это
$\langle$~проективность / плоскость~$\rangle$ всех (и в частности аннуляторных)
левых банаховых модулей над ней. Резкое отличие метрической и топологической
теории в том, что в них подобных алгебр не может быть.

\begin{proposition} Не существует банаховой алгебры $A$ такой, что все
$A$-модули $\langle$~метрически / топологически~$\rangle$ плоские. Тем более, не
существует банаховых алгебр таких, что все $A$-модули $\langle$~метрически /
топологически~$\rangle$ проективны.
\end{proposition}
\begin{proof} Рассмотрим бесконечномерное $\mathscr{L}_\infty$-пространство $X$
(например $\ell_\infty(\mathbb{N})$) как аннуляторный $A$-модуль. Из
[\cite{DefFloTensNorOpId}, параграф 23.3] мы знаем, что $X$ не является
$\mathscr{L}_1$-пространством. Следовательно, по
предложению~\ref{MetTopFlatAnnihModCharac} модуль $X$ не является топологически
плоским. По предложению~\ref{MetFlatIsTopFlatAndTopFlatIsRelFlat} он также и не
метрически плоский. Наконец, из предложения~\ref{MetTopProjIsMetTopFlat}
следует, что $X$ не является ни метрически, ни топологически проективным.
\end{proof}

%-------------------------------------------------------------------------------
%	Homologically trivial modules over Banach algebras with specific geometry
%-------------------------------------------------------------------------------

\subsection{
    Гомологически тривиальные модули над банаховыми 
    алгебрами со специальной геометрией
}\label{
    SubSectionHomologicallyTrivialModulesOverBanachAlgebrasWithSpecificGeometry
}

Цель данного параграфа --- убедиться в том, что гомологически тривиальные модули
над некоторыми банаховыми алгебрами имеют с этими алгебрами схожие
геометрические свойства.

Снова напомним, что под пространством с мерой мы понимаем строго локализуемое
пространство с мерой. В следующем предложении мы несколько раз встретимся с
произведением пространств с мерой. Произведение двух пространств с мерой
$(\Omega_1,\Sigma_1,\mu_1)$ и $(\Omega_2,\Sigma_2,\mu_2)$ мы будем обозначать
$\mu_1\times \mu_2$. Произведение двух строго локализуемых пространств с мерой
есть  строго локализуемое пространство с мерой [\cite{FremMeasTh}, предложение
251N]. 

Результаты следующего предложения в случае метрической теории были получены
Гравеном в~\cite{GravInjProjBanMod}.

\begin{proposition}\label{TopProjInjFlatModOverL1Charac} Пусть $A$ --- банахова
алгебра изометрически изоморфная, как банахово пространство, пространству
$L_1(\Theta,\nu)$ для некоторого пространства с мерой $(\Theta,\Sigma,\nu)$.
Тогда:
\begin{enumerate}[label = (\roman*)]
    \item если $P$ --- $\langle$~метрически / топологически~$\rangle$
    проективный $A$-модуль, то $P$ --- $\langle$~$L_1$-пространство / ретракт
    $L_1$-пространства~$\rangle$;

    \item если $J$ --- $\langle$~метрически / топологически~$\rangle$
    инъективный $A$-модуль, то $J$ --- $\langle$~$C(K)$-пространство для
    некоторого стоунова пространства $K$ / топологически инъективное банахово
    пространство~$\rangle$;

    \item если $F$ --- $\langle$~метрически / топологически~$\rangle$ плоский
    $A$-модуль, то $F$ --- $\langle$~$L_1$-пространство /
    $\mathscr{L}_1$-пространство~$\rangle$.
\end{enumerate}
\end{proposition}
\begin{proof} Через $(\Theta',\Sigma', \nu')$ обозначим пространство с мерой
$(\Theta,\Sigma, \nu)$ с одним добавленным атомом, тогда
$A_+\isom{\mathbf{Ban}_1} L_1(\Theta',\nu')$.

$(i)$ Так как $P$ $\langle$~метрически / топологически~$\rangle$ проективен как
$A$-модуль, то по предложению~\ref{MetTopProjModViaCanonicMorph} он является
ретрактом $A_+\projtens \ell_1(B_P)$ в $\langle$~$A-\mathbf{mod}_1$ /
$A-\mathbf{mod}$~$\rangle$. Пусть $\mu_c$ --- считающая мера на $B_P$, тогда по
теореме Гротендика [\cite{HelLectAndExOnFuncAn}, теорема 2.7.5]
$$
A_+\projtens\ell_1(B_P)
\isom{\mathbf{Ban}_1}L_1(\Theta',\nu')\projtens L_1(B_P,\mu_c)
\isom{\mathbf{Ban}_1}L_1(\Theta'\times B_P,\nu'\times \mu_c).
$$
Следовательно, $P$ --- $\langle$~$1$-ретракт / ретракт~$\rangle$
$L_1$-пространства. Осталось заметить, что любой $1$-ретракт $L_1$-пространства
есть снова $L_1$-пространство [\cite{LaceyIsomThOfClassicBanSp}, теорема
6.17.3].

$(ii)$ Так как $J$ $\langle$~метрически / топологически~$\rangle$ инъективный
$A$-модуль, то по предложению~\ref{MetTopInjModViaCanonicMorph} он является
ретрактом $\mathcal{B}(A_+,\ell_\infty(B_{J^*}))$ в $\langle$~$\mathbf{mod}_1-A$
/ $\mathbf{mod}-A$~$\rangle$. Пусть $\mu_c$ --- считающая мера на $B_{J^*}$,
тогда по теореме Гротендика [\cite{HelLectAndExOnFuncAn}, теорема 2.7.5]
$$
\mathcal{B}(A_+,\ell_\infty(B_{J^*}))
\isom{\mathbf{Ban}_1}{(A_+\projtens \ell_1(B_{J^*}))}^*
\isom{\mathbf{Ban}_1}{(L_1(\Theta',\nu')\projtens L_1(B_P,\mu_c))}^*
$$
$$
\isom{\mathbf{Ban}_1}{L_1(\Theta'\times B_P,\nu'\times \mu_c)}^*
\isom{\mathbf{Ban}_1}L_\infty(\Theta'\times B_P,\nu'\times \mu_c).
$$
Следовательно, $J$ --- $\langle$~$1$-ретракт / ретракт~$\rangle$
$L_\infty$-пространства. Поскольку $L_\infty$-пространства $\langle$~метрически
/ топологически~$\rangle$ инъективны, то таковы же и их ретракты $J$. Осталось
напомнить, что каждое метрически инъективное банахово пространство суть
$C(K)$-пространство для некоторого стоунова пространства $K$
[\cite{LaceyIsomThOfClassicBanSp}, теорема 3.11.6].

$iii)$ Из $\langle$~[\cite{GrothMetrProjFlatBanSp}, теорема 1] /
[\cite{StegRethNucOpL1LInfSp}, теорема VI.6]~$\rangle$ мы знаем, что банахово
пространство $F^*$ $\langle$~метрически / топологически~$\rangle$ инъективно
тогда и только тогда, когда $F$ является $\langle$~$L_1$-пространством /
$\mathscr{L}_1$-пространством~$\rangle$. Остается применить результаты пункта
$(ii)$ и предложение~\ref{MetTopFlatCharac}.
\end{proof}

\begin{proposition}\label{TopProjInjFlatModOverMthscrL1SpCharac} Пусть $A$ ---
банахова алгебра изоморфная, как банахово пространство, некоторому
$\mathscr{L}_1$-пространству. Тогда любой топологически $\langle$~проективный /
инъективный / плоский~$\rangle$ $A$-модуль является
$\langle$~$\mathscr{L}_1$-пространством / $\mathscr{L}_\infty$-пространством /
$\mathscr{L}_1$-пространством~$\rangle$.
\end{proposition}
\begin{proof} Если алгебра $A$ есть $\mathscr{L}_1$-пространство, то такова же и
$A_+$. 

Пусть $P$ --- топологически проективный $A$-модуль. Тогда по
предложению~\ref{MetTopProjModViaCanonicMorph} он есть ретракт $A_+\projtens
\ell_1(B_P)$ в $A-\mathbf{mod}$ и тем более в $\mathbf{Ban}$. Поскольку
$\ell_1(B_P)$ есть $\mathscr{L}_1$-пространство, то таково же
$A_+\projtens\ell_1(B_P)$ как проективное тензорное произведение
$\mathscr{L}_1$-пространств [\cite{GonzDPPInTensProd}, предложение 1].
Следовательно, $P$ есть $\mathscr{L}_1$-пространство как ретракт
$\mathscr{L}_1$-пространства [\cite{BourgNewClOfLpSp}, предложение 1.28].

Пусть $J$ --- топологически инъективный $A$-модуль. Тогда по
предложению~\ref{MetTopInjModViaCanonicMorph} он есть ретракт
$\mathcal{B}(A_+,\ell_\infty(B_{J^*}))
\isom{\mathbf{mod}_1-A}{(A_+\projtens\ell_1(B_{J^*}))}^*$
в $\mathbf{mod}-A$ и тем более в $\mathbf{Ban}$. Как мы показали выше,
пространство $A_+\projtens\ell_1(B_{J^*})$ является
$\mathscr{L}_1$-пространством, тогда его сопряженное
$\mathcal{B}(A_+,\ell_\infty(B_{J^*}))$ есть $\mathscr{L}_\infty$-пространство
[\cite{BourgNewClOfLpSp}, предложение 1.27]. Осталось заметить, что любой
ретракт $\mathscr{L}_\infty$-пространства есть $\mathscr{L}_\infty$-пространство
[\cite{BourgNewClOfLpSp}, предложение 1.28].

Наконец, пусть $F$ --- топологически плоский $A$-модуль, тогда $F^*$
топологически инъективен по предложению~\ref{MetTopFlatCharac}. Из предыдущего
абзаца следует, что $F^*$ --- это $\mathscr{L}_\infty$-пространство. По теореме
VI.6 из~\cite{StegRethNucOpL1LInfSp} пространство $F$ является
$\mathscr{L}_1$-пространством.
\end{proof}

Перейдем к обсуждению свойства Данфорда-Петтиса для гомологически тривиальных
банаховых модулей. Прежде напомним определение и перечислим основные факты о
свойстве Данфорда-Петтиса. Ограниченный линейный оператор $T:E\to F$ называется
слабо компактным, если он отправляет единичный шар пространства $E$ в
относительно слабо компактное подмножество в $F$. Ограниченный линейный оператор
называется вполне непрерывным, если образ любого слабо компактного подмножества
в $E$ компактен в нормовой топологии $F$. Говорят, что банахово пространство $E$
имеет свойство Данфорда-Петтиса, если любой слабо компактный оператор из $E$ в
произвольное банахово пространство $F$ вполне непрерывен. Существует простое
внутреннее описание этого свойства [\cite{KalAlbTopicsBanSpTh}, теорема 5.4.4]:
банахово пространство $E$ обладает свойством Данфорда-Петтиса если $\lim_n
f_n(x_n)=0$ для любых слабо сходящихся к $0$ последовательностей
${(x_n)}_{n\in\mathbb{N}}\subset E$ и ${(f_n)}_{n\in\mathbb{N}}\subset E^*$. Теперь
легко доказать, что если банахово пространство $E^*$ имеет свойство
Данфорда-Петтиса, то его имеет и $E$. В своей новаторской
работе~\cite{GrothApllFaiblCompSpCK} Гротендик показал, что все
$L_1$-пространства и $C(K)$-пространства имеют это свойство. Более того любое
$\mathscr{L}_1$-пространство и любое $\mathscr{L}_\infty$-пространство имеет
свойство Данфорда-Петтиса  [\cite{BourgNewClOfLpSp}, предложение 1.30].
Поскольку единичный шар рефлексивного банахова пространства слабо компактен
[\cite{MeggIntroBanSpTh}, теорема 2.8.2], то у таких пространств со свойством
Данфорда-Петтиса единичный шар компактен в нормовой топологии, и следовательно,
такие пространства конечномерны. Свойство Данфорда-Петтиса наследуется
дополняемыми подпространствами [\cite{FabHabBanSpTh}, предложение 13.44]. 

Ключевым для нас будет результат Бургейна о банаховых пространствах со
специальной локальной структурой. В [\cite{BourgOnTheDPP}, теорема 5] он
доказал, что первое, второе и так далее сопряженное пространство банахова
пространства с $E_p$-локальной структурой обладает свойством Данфорда-Петтиса.
Здесь, $E_p$ обозначает класс всех $\bigoplus_\infty$-сумм $p$ копий $p$-мерных
$\ell_1$-пространств для некоторого натурального $p$. 

\begin{proposition}\label{C0SumOfL1SpHaveDPP} Пусть $
\{L_1(\Omega_\lambda,\mu_\lambda):\lambda\in\Lambda \}$ --- семейство
бесконечномерныx $L_1$-пространств. Тогда банахово пространство $\bigoplus_0
\{L_1(\Omega_\lambda,\mu_\lambda):\lambda\in\Lambda \}$ имеет
$(E_p,1+\epsilon)$-локальную структуру для всех $\epsilon>0$.
\end{proposition}
\begin{proof} Для каждого $\lambda\in\Lambda$ через
$L_1^0(\Omega_\lambda,\mu_\lambda)$ обозначим плотное подпространство в
$L_1(\Omega_\lambda,\mu_\lambda)$ натянутое на характеристические функции
измеримых множеств из $\Sigma_\lambda$. Обозначим $E:=\bigoplus_0
\{L_1(\Omega_\lambda,\mu_\lambda):\lambda\in\Lambda \}$, пусть
$E_0:=\bigoplus_{00} \{L_1(\Omega_\lambda,\mu_\lambda):\lambda\in\Lambda \}$ ---
не обязательно замкнутое подпространство в $E$, состоящее из векторов с конечным
числом ненулевых координат.

Зафиксируем $\epsilon>0$ и конечномерное подпространство $F$ в $E$. Так как $F$
конечномерно, то существует ограниченный проектор $Q:E\to E$ на $F$. Выберем
$\delta>0$ так, чтобы $\delta\Vert Q\Vert<1$ и $(1+\delta\Vert
Q\Vert){(1-\delta\Vert Q\Vert)}^{-1}<1+\epsilon$. Заметим, что $B_F$ компактно,
потому что $F$ конечномерно. Следовательно, существует конечная $\delta/2$-сеть
${(x_k)}_{k\in\mathbb{N}_n}\subset E_0$ для $B_F$. Для каждого $k\in\mathbb{N}_n$
имеем $x_k=\bigoplus_0 \{x_{k,\lambda}:\lambda\in\Lambda \}$, где
$x_{k,\lambda}=\sum_{j=1}^{m_{k,\lambda}}d_{k,j,\lambda}\chi_{D_{j,k,\lambda}}$
для некоторых комплексных чисел
${(d_{j,k,\lambda})}_{j\in\mathbb{N}_{m_{k,\lambda}}}$ и измеримых множеств
${(D_{j,k,\lambda})}_{j\in\mathbb{N}_{m_{k,\lambda}}}$ конечной меры. Пусть
${(C_{i,\lambda})}_{i\in\mathbb{N}_{m_\lambda}}$ --- множество всех попарных
пересечений элементов из ${(D_{j,k,\lambda})}_{j\in\mathbb{N}_{m_{k,\lambda}}}$
исключая множества меры $0$. Тогда $x_{k,\lambda}=\sum_{i=1}^{m_\lambda}
c_{i,k,\lambda}\chi_{C_{i,\lambda}}$ для некоторых комплексных чисел
${(c_{j,k,\lambda})}_{j\in\mathbb{N}_{m_{\lambda}}}$. Обозначим $\Lambda_k=
\{\lambda\in\lambda:x_{k,\lambda}\neq 0 \}$. По определению пространства $E_0$
множество $\Lambda_k$ конечно для каждого $k\in\mathbb{N}_n$. Рассмотрим
конечное множество $\Lambda_0=\bigcup_{k\in\mathbb{N}_n}\Lambda_k$. Так как
пространство $L_1(\Omega_\lambda, \mu_\lambda)$ бесконечномерно, то мы можем
добавить к семейству $ \{\chi_{C_{i,\lambda}}:i\in\mathbb{N}_{m_\lambda} \}$
любое конечное число дизъюнктных множеств положительной конечной меры. Поэтому,
далее считаем, что $m_\lambda=\operatorname{Card}(\Lambda_0)$ для всех
$\lambda\in\Lambda_0$. Для каждого $\lambda\in\Lambda_0$ корректно определен
проектор 
$$
P_\lambda
:L_1(\Omega_\lambda,\mu_\lambda)\to L_1(\Omega_\lambda,\mu_\lambda)
:x_\lambda\mapsto \sum_{i=1}^{m_\lambda}\left( 
    {\mu(C_{i,\lambda})}^{-1}
    \int_{C_{i,\lambda}}x_\lambda(\omega)d\mu_\lambda(\omega)
\right)\chi_{C_{i,\lambda}}.
$$
Легко проверить, что $P_\lambda(\chi_{C_{i,\lambda}})=\chi_{C_{i,\lambda}}$ для
всех $i\in\mathbb{N}_{m_\lambda}$. Следовательно,
$P_\lambda(x_{k,\lambda})=x_{k,\lambda}$ для всех $k\in\mathbb{N}_n$. Так как
множества ${(C_{i,\lambda})}_{i\in\mathbb{N}_{m_\lambda}}$ не пересекаются и имеют
положительную меру, то
$\operatorname{Im}(P_\lambda)\isom{\mathbf{Ban}_1}\ell_1(\mathbb{N}_{m_\lambda})$.
Для $\lambda\in\Lambda\setminus\Lambda_0$ мы положим $P_\lambda=0$ и рассмотрим
проектор $P:=\bigoplus_0 \{P_\lambda:\lambda\in\Lambda \}$. По построению он
сжимающий с образом $\operatorname{Im}(P)\isom{\mathbf{Ban}_1}\bigoplus_0
\{\ell_1(\mathbb{N}_{m_\lambda}):\lambda\in\Lambda_0 \}\in E_{p}$. Рассмотрим
произвольный вектор $x\in B_F$, тогда существует номер $k\in\mathbb{N}_n$ такой,
что $\Vert x-x_k\Vert\leq \delta/2$. Тогда $\Vert P(x)-x\Vert=\Vert
P(x)-P(x_k)+x_k-x\Vert\leq\Vert P\Vert\Vert x-x_k\Vert+\Vert
x_k-x\Vert\leq\delta$.

Построив проекторы $P$ и $Q$, рассмотрим оператор $I:=1_E+PQ-Q$. Очевидно,
$\Vert 1_E-I\Vert=\Vert PQ-Q\Vert\leq \delta\Vert Q\Vert$. Следовательно $I$ ---
изоморфизм по стандартному трюку с рядами фон Нойманна
[\cite{KalAlbTopicsBanSpTh}, предложение A.2]. Более того,
$I^{-1}=\sum_{p=0}^\infty{(1_E-I)}^p$, поэтому
$$
\Vert I^{-1}\Vert
\leq\sum_{p=0}^\infty {\Vert 1_E-I\Vert}^p
\leq\sum_{p=0}^\infty{(\delta\Vert Q\Vert)}^p
={(1-\delta\Vert Q\Vert)}^{-1},
\quad
\Vert I\Vert\leq\Vert 1_E\Vert+\Vert I-1_E\Vert\leq 1+\delta\Vert Q\Vert.
$$
Заметим, что $PI=P+P^2Q-PQ=P+PQ-PQ=P$, поэтому для всех $x\in F$ выполнено 
$$
I(x)=x+P(Q(x))-Q(x)=x+P(x)-x=P(x)=P(P(x))=P(I(x))
$$
и $x=(I^{-1}PI)(x)$. Последнее означает, что $F$ содержится в образе
ограниченного проектора $R=I^{-1}PI$. Обозначим этот образ через $F_0$ и
рассмотрим биограничение  $I_0=I|_{F_0}^{\operatorname{Im}(P)}$ изоморфизма $I$.
Так как 
$\Vert I_0\Vert\Vert I_0^{-1}\Vert
\leq\Vert I\Vert
\Vert I^{-1}\Vert
\leq(1+\delta\Vert Q\Vert){(1-\delta\Vert Q\Vert)}^{-1}
<1+\epsilon$, то $d_{BM}(F_0,\operatorname{Im}(P))<1+\epsilon$. 
Таким образом, мы показали, что для любого конечномерного подпространства в 
$E$ существует подпространство $F_0$ в $E$ содержащее $F$ такое, что 
$d_{BM}(F_0,U)<1+\epsilon$ для некоторого $U\in E_{p}$. Это значит, что 
$E$ имеет $(E_{p}, 1+\epsilon)$-локальную структуру.
\end{proof}

\begin{proposition}\label{ProdOfL1SpHaveDPP} Пусть $
\{(\Omega_\lambda,\Sigma_\lambda,\mu_\lambda):\lambda\in\Lambda \}$ ---
семейство пространств с мерой. Тогда банахово пространство $\bigoplus_\infty
\{L_1(\Omega_\lambda,\mu_\lambda):\lambda\in\Lambda \}$ обладает свойством
Данфорда-Петтиса.
\end{proposition}
\begin{proof} Сначала предположим, что пространства $L_1(\Omega_\lambda,
\mu_\lambda)$ бесконечномерны для всех $\lambda\in\Lambda$. Из
предложения~\ref{C0SumOfL1SpHaveDPP} мы знаем, что банахово пространство
$F:=\bigoplus_0 \{L_1(\Omega_\lambda,\mu_\lambda):\lambda\in\Lambda \}$ имеет
$E_{p}$-локальную структуру. Тогда по теореме 5 из~\cite{BourgOnTheDPP} первое,
второе и так далее сопряженное пространство пространства $F$ обладают свойством
Данфорда-Петтиса. Как следствие, мы получаем, что 
$F^{**}={\left(
    \bigoplus_0 \{L_1(\Omega_\lambda,\mu_\lambda):\lambda\in\Lambda \}
\right)}^{**}$
$\isom{\mathbf{Ban}_1}
\bigoplus_\infty \{{L_1(\Omega_\lambda,\mu_\lambda)}^{**}:\lambda\in\Lambda \}$ 
имеет свойство Данфорда-Петтиса. 
Из [\cite{DefFloTensNorOpId}, предложение B10] мы знаем, что
каждое $L_1$-пространство 1-дополняемо в своем втором сопряженном. Для каждого
$\lambda\in\Lambda$ через $P_\lambda$ обозначим соответствующий проектор в
${L_1(\Omega_\lambda,\mu_\lambda)}^{**}$. Таким образом $\bigoplus_\infty
\{L_1(\Omega_\lambda,\mu_\lambda):\lambda\in\Lambda \}$ 1-дополняемо в
$F^{**}
\isom{\mathbf{Ban}_1}
\bigoplus_\infty \{{L_1(\Omega_\lambda,\mu_\lambda)}^{**}:\lambda\in\Lambda \}$ 
посредством проектора 
$\bigoplus_\infty  \{P_\lambda:\lambda\in\Lambda \}$. Так как $F^{**}$
имеет свойство Данфорда-Петтиса, то из [\cite{FabHabBanSpTh}, предложение 13.44]
следует, что это свойство имеет и дополняемое в $F^{**}$ подпространство
$\bigoplus_\infty \{L_1(\Omega_\lambda,\mu_\lambda):\lambda\in\Lambda \}$.

Теперь рассмотрим общий случай. Любое $L_1$-пространство можно рассматривать как
$1$-дополняемое подпространство некоторого бесконечномерного $L_1$-пространства.
Как следствие, пространство $\bigoplus_\infty
\{L_1(\Omega_\lambda,\mu_\lambda):\lambda\in\Lambda \}$ будет $1$-дополняемо в
$\bigoplus_\infty$-сумме бесконечномерных $L_1$-пространств. Как было показано
выше, такая сумма обладает свойством Данфорда-Петтиса, а значит,  Осталось
вспомнить, что это свойство Данфорда-Петтиса наследуется дополняемыми
подпространствами [\cite{FabHabBanSpTh}, предложение 13.44].
\end{proof}

\begin{proposition}\label{ProdOfDualsOfMthscrLInftySpHaveDPP} Пусть $E$ ---
$\mathscr{L}_\infty$-пространство и $\Lambda$ --- произвольное множество. Тогда
банахово пространство $\bigoplus_\infty \{E^*:\lambda\in\Lambda \}$ имеет
свойство Данфорда-Петтиса.
\end{proposition}
\begin{proof} Поскольку $E$ --- это $\mathscr{L}_\infty$-пространство, то $E^*$
дополняемо в некотором $L_1$-пространстве [\cite{LinPelAbsSumOpInLpSpAndApp},
предложение 7.4]. То есть существует ограниченный линейный проектор
$P:L_1(\Omega,\mu)\to L_1(\Omega,\mu)$ с образом топологически изоморфным
пространству $E^*$. В этом случае $\bigoplus_\infty \{ E^*:\lambda\in\Lambda \}$
дополняемо в $\bigoplus_\infty \{ L_1(\Omega,\mu):\lambda\in\Lambda \}$
посредством проектора $\bigoplus_\infty \{P:\lambda\in\Lambda \}$. Пространство
$\bigoplus_\infty \{ L_1(\Omega,\mu):\lambda\in\Lambda \}$ имеет свойство
Данфорда-Петтиса по предложению~\ref{ProdOfL1SpHaveDPP}. Тогда из
[\cite{FabHabBanSpTh}, предложение 13.44] следует, что этим свойством обладает и
его дополняемое подпространство $\bigoplus_\infty \{ E^*:\lambda\in\Lambda \}$.
\end{proof}

\begin{theorem}\label{TopProjInjFlatModOverMthscrL1OrLInftySpHaveDPP} Пусть $A$
--- банахова алгебра, являющаяся, как банахово пространство, $\mathscr{L}_1$-
или $\mathscr{L}_\infty$-пространством. Тогда топологически проективные,
инъективные и плоские $A$-модули имеют свойство Данфорда-Петтиса.
\end{theorem}
\begin{proof} Предположим $A$ является $\mathscr{L}_1$-пространством. Заметим,
что $\mathscr{L}_1$- и $\mathscr{L}_\infty$-пространства имеют свойство
Данфорда-Петтиса [\cite{BourgNewClOfLpSp}, предложение 1.30]. Теперь результат
следует из предложения~\ref{TopProjInjFlatModOverMthscrL1SpCharac}.

Предположим $A$ является $\mathscr{L}_\infty$-пространством, тогда такова же и
$A_+$. Пусть $J$ --- топологически инъективный $A$-модуль, тогда по
предложению~\ref{MetTopInjModViaCanonicMorph} он ретракт
$$
\mathcal{B}(A_+,\ell_\infty(B_{J^*}))
\isom{\mathbf{mod}_1-A}{(A_+\projtens\ell_1(B_{J^*}))}^*
\isom{\mathbf{mod}_1-A}
{\left(\bigoplus\nolimits_1 \{ A_+:\lambda\in B_{J^*} \}\right)}^*
\isom{\mathbf{mod}_1-A}
\bigoplus\nolimits_\infty \{ A_+^*:\lambda\in B_{J^*} \}
$$ 
в $\mathbf{mod}-A$ и тем более в $\mathbf{Ban}$. По
предложению~\ref{ProdOfDualsOfMthscrLInftySpHaveDPP} последний модуль имеет
свойство Данфорда-Петтиса. Так как $J$ его ретракт, то он тоже обладает этим
свойством [\cite{FabHabBanSpTh}, предложение 13.44]. 

Если $F$ топологически плоский $A$-модуль, то $F^*$ топологически инъективен по
предложению~\ref{MetTopFlatCharac}. Из предыдущего абзаца мы знаем, что тогда
$F^*$ имеет свойство Данфорда-Петтиса и, как следствие, этим свойством обладает
сам модуль $F$.

Пусть $P$ --- топологически проективный $A$-модуль. По
предположению~\ref{MetTopProjIsMetTopFlat} он топологически плоский и тогда из
предыдущего абзаца мы видим, что $P$ имеет свойство Данфорда-Петтиса.
\end{proof}

\begin{corollary}\label{NoInfDimRefMetTopProjInjFlatModOverMthscrL1OrLInfty}
Пусть $A$ --- банахова алгебра, являющаяся, как банахово пространство,
$\mathscr{L}_1$- или $\mathscr{L}_\infty$-пространством. Тогда не существует
топологически проективного, инъективного или плоского бесконечномерного
рефлексивного $A$-модуля. Тем более не существует метрически проективного,
инъективного или плоского бесконечномерного рефлексивного $A$-модуля.
\end{corollary}
\begin{proof} Из теоремы~\ref{TopProjInjFlatModOverMthscrL1OrLInftySpHaveDPP} мы
знаем, что любой топологически инъективный $A$-модуль имеет свойство
Данфорда-Петтиса. С другой стороны не существует бесконечномерного рефлексивного
банахова пространства с этим свойством. Итак, мы получили желаемый результат в
контексте топологической инъективности. Так как пространство, сопряженное к
рефлексивному снова рефлексивно, то из предложения~\ref{MetTopFlatCharac}
следует результат для топологической плоскости. Осталось вспомнить что по
предложению~\ref{MetTopProjIsMetTopFlat} каждый топологически проективный модуль
является топологически плоским. Чтобы доказать последнее утверждение вспомним,
что по предложению $\langle$~\ref{MetProjIsTopProjAndTopProjIsRelProj}
/~\ref{MetInjIsTopInjAndTopInjIsRelInj}
/~\ref{MetFlatIsTopFlatAndTopFlatIsRelFlat}~$\rangle$ метрическая
$\langle$~проективность / инъективность / плоскость~$\rangle$ влечет
топологическую $\langle$~проективность / инъективность / плоскость~$\rangle$.
\end{proof}

Стоит сказать, что в относительной теории существуют примеры рефлексивных
бесконечномерных относительно проективных, инъективных и плоских модулей над
банаховыми алгебрами, являющимися $\mathscr{L}_1$- или
$\mathscr{L}_\infty$-пространствами. Приведем два примера. Первый связан с
сверточной алгеброй $L_1(G)$ локально компактной группы $G$ с мерой Хаара. Эта
алгебра --- $\mathscr{L}_1$-пространство. В [\cite{DalPolHomolPropGrAlg}, \S6]
и~\cite{RachInjModAndAmenGr} было доказано, что для для $1<p<+\infty$ банахов
$L_1(G)$-модуль $L_p(G)$ является относительно $\langle$~проективным /
инъективным / плоским~$\rangle$ тогда и только тогда, когда группа $G$
$\langle$~компактна / аменабельна / аменабельна~$\rangle$. Заметим, что любая
компактная группа аменабельна [\cite{PierAmenLCA}, предложение 3.12.1], и
поэтому для компактной группы $G$ модуль $L_p(G)$ будет относительно проективным
инъективным и плоским для всех $1<p<+\infty$. Второй пример будет про алгебры
$c_0(\Lambda)$ и $\ell_\infty(\Lambda)$ для бесконечного множества $\Lambda$.
Это $\mathscr{L}_\infty$-пространства. Как мы покажем в
предложении~\ref{c0AndlInftyModsRelTh}, над этими алгебрами модули
$\ell_p(\Lambda)$ для $1<p<\infty$ всегда являются относительно проективными,
инъективными и плоскими. 

Чтобы обсудить еще одно банахово-геометрическое свойство, нам понадобятся долгие
приготовления, а именно, определение банаховой решетки и безусловного базиса
Шаудера. 

Действительное пространство Риса $E$ --- это линейное пространство над полем
$\mathbb{R}$ со структурой частично упорядоченного множества, такой, что  $x\leq
y$ влечет $x+z\leq y+z$ для всех $x,y,z\in E$ и $ax\geq 0$ для всех $x\geq 0$,
$a\in\mathbb{R}_+$. Частично упорядоченное множество называется решеткой, если
любые два элемента ${x,y}$ имеют точную верхнюю грань $x\vee y$ и точную нижнюю
грань $x\wedge y$. Действительная векторная решетка --- это действительное
пространство Риса, которое, как частично упорядоченное множество, является
решеткой. Если $E$ --- действительная векторная решетка, то для каждого вектора
$x\in E$ мы определяем его модуль по формуле $|x|:=x\vee(-x)$. Комплексная
векторная решетка $E$ --- это линейное пространство над полем $\mathbb{C}$,
такое, что существует действительное линейное подпространство
$\operatorname{Re}(E)$, являющееся действительной векторной решеткой, причем:
\begin{enumerate}[label = (\roman*)]
    \item для каждого $x\in E$ существуют единственные
    $\operatorname{Re}(x),\operatorname{Im}(x)\in \operatorname{Re}(E)$ такие,
    что $x=\operatorname{Re}(x)+i\operatorname{Im}(x)$;

    \item для каждого $x\in E$ определено абсолютное значение $|x|:=\sup
    \{\operatorname{Re}(e^{i\theta}x):\theta\in\mathbb{R} \}$.
\end{enumerate}

Банахова решетка --- это банахово пространство со структурой комплексной
векторной решетки такой, что $|x|\leq |y|$ влечет $\Vert x\Vert\leq \Vert
y\Vert$. Классический пример банаховой решетки $E$ --- это $L_p$-пространство
или $C(K)$-пространство. В обоих случаях $\operatorname{Re}(E)$ состоит из
действительнозначных функций в $E$. Если $ \{E_\lambda:\lambda\in\Lambda \}$ ---
семейство банаховых решеток, то для любого $1\leq p\leq +\infty$ или $p=0$ их
$\bigoplus_p$-сумма есть банахова решетка, причем для $x,y\in\bigoplus_p \{
E_\lambda:\lambda\in\Lambda \}$ выполнено $x\leq y$ если $x_\lambda\leq
y_\lambda$ для всех $\lambda\in\Lambda$. Сопряженное пространство $E^*$
банаховой решетки $E$ есть снова банахова решетка, причем для $f,g\in  E^*$
выполнено $f\leq g$ если $f(x)\leq g(x)$ для всех $x\geq 0$. Хорошее введение в
теорию банаховых решеток можно найти в [\cite{LaceyIsomThOfClassicBanSp},
параграф 1].

Свойство быть банаховой решеткой накладывает определенные ограничения на
геометрию банахова
пространства~\cite{SherOrderInOpAlg},~\cite{KadOrderPropOfBoundSAOps}. Чтобы
объяснить этот эффект, нам понадобится определение безусловного базиса Шаудера.
Пусть $E$ --- банахово пространство. Набор функционалов
${(f_\lambda)}_{\lambda\in\Lambda}$ в $E^*$ называется биортогональной системой
для векторов ${(x_\lambda)}_{\lambda\in\Lambda}$ из $E$, если
$f_\lambda(x_{\lambda'})=1$ при $\lambda=\lambda'$, иначе $0$. Набор векторов
${(x_\lambda)}_{\lambda\in\Lambda}$ в $E$ называется безусловным базисом Шаудера,
если в $E^*$ существует биортогональная система
${(f_\lambda)}_{\lambda\in\Lambda}$ для ${(x_\lambda)}_{\lambda\in\Lambda}$ такая,
что ряд $\sum_{\lambda\in\Lambda} f_\lambda(x)x_\lambda$ безусловно сходится к
$x$ для любого $x\in E$. Все $\ell_p$-пространства при $1\leq p<+\infty$ имеют
безусловный базис Шаудера, например, это ${(\delta_\lambda)}_{\lambda\in\Lambda}$.
Классический пример пространства без безусловного базиса Шаудера --- это
$C([0,1])$. Более того, это пространство не может быть даже подпространством
банахова пространства с безусловным базисом Шаудера [\cite{KalAlbTopicsBanSpTh},
предложение 3.5.4]. Любой безусловный базис Шаудера
${(x_\lambda)}_{\lambda\in\Lambda}$ в $E$ удовлетворяет следующему свойству
[\cite{DiestAbsSumOps}, предложение 1.6]: существует константа $\kappa\geq 1$
такая, что
$$
\left\Vert \sum_{\lambda\in\Lambda}t_\lambda f_\lambda(x)x_\lambda\right\Vert
\leq
\kappa\left\Vert \sum_{\lambda\in\Lambda}f_\lambda(x)x_\lambda\right\Vert
$$
для всех $x\in E$ и $t\in\ell_\infty(\Lambda)$. Наименьшая из таких констант
$\kappa$ по всем безусловным базисам Шаудера пространства $E$ обозначается через
$\kappa(E)$. Аналогичная характеристика может быть определена и для банаховых
пространств без безусловного базиса Шаудера. Локальная безусловная константа
$\kappa_u(E)$ банахова пространства $E$ определяется как инфимум всех чисел $C$
со следующим свойством: для каждого конечномерного подпространства $F$ в $E$
существует банахово пространство $G$ с безусловным базисом Шаудера и
ограниченные линейные операторы $S:F\to G$, $T:G\to E$ такие, что $TS|^{F}=1_F$
и $\Vert T\Vert\Vert S\Vert\kappa(G)\leq C$. Говорят, что банахово пространство
$E$ имеет локально безусловную структуру если $\kappa_u(E)$ конечно. Изначально
это свойство было определено на английском и называлось the \textbf{l}ocal
\textbf{u}nconditional \textbf{st}ructure property. Для него использовали
аббревиатуру l.u.st, и мы поступим так же. Очевидно, любое банахово пространство
$E$ с безусловным базисом Шаудера имеет свойство l.u.st, причем
$\kappa_u(E)=\kappa(E)$. В частности, все конечномерные банаховы пространства
имеют свойство l.u.st. Хотя произвольная банахова решетка $E$ может и не
обладать безусловным базисом Шаудера, она все равно имеет свойство l.u.st, и при
этом $\kappa_u(E)=1$  [\cite{DiestAbsSumOps}, теорема 17.1]. Непосредственно из
определения следует, что свойство l.u.st наследуется дополняемыми
подпространствами. Точнее, если $F$ --- $C$-дополняемое подпространство в $E$,
то $\kappa_u(F)\leq C\kappa_u(E)$. Следовательно, все дополняемые
подпространства банаховых решеток обладают свойством l.u.st. Это необходимое
условие не так уж далеко от критерия [\cite{DiestAbsSumOps}, теорема 17.5]:
банахово пространство $E$ имеет свойство l.u.st тогда и только тогда, когда
$E^{**}$ топологически изоморфно, как банахово пространство, дополняемому
подпространству некоторой банаховой решетки. Как следствие, этого критерия мы
получаем, что банахово пространство $E$ имеет свойство l.u.st тогда и только
тогда, когда его имеет $E^*$ [\cite{DiestAbsSumOps}, следствие 17.6].

\begin{proposition} Пусть $A$ --- банахова алгебра, обладающая свойством l.u.st.
Тогда всякий топологически проективный, инъективный и плоский $A$-модуль тоже
обладает этим свойством.
\end{proposition}
\begin{proof} Если $J$ --- топологически инъективный $A$-модуль, то по
предложению~\ref{MetTopInjModViaCanonicMorph} он является ретрактом
$\mathcal{B}(A_+,\ell_\infty(B_{J^*}))\isom{\mathbf{mod}_1-A}\bigoplus_\infty
\{ A_+^*:\lambda\in B_{J^*} \}$ в $\mathbf{mod}-A$ и тем более в $\mathbf{Ban}$.
Если $A$ обладает свойством l.u.st, то $A^{**}$ дополняемо в некоторой банаховой
решетке $E$ [\cite{DiestAbsSumOps}, теорема 17.5]. Как следствие $A_+^{***}$
дополняемо в банаховой решетке $F:={\left(E\bigoplus_1\mathbb{C}\right)}^*$
посредством некоторого ограниченного проектора $P:F\to F$. Следовательно,
$\bigoplus_\infty \{A_+^{***}:\lambda\in B_{J^*} \}$ дополняемо в банаховой
решетке $\bigoplus_\infty \{F:\lambda\in B_{J^*} \}$ посредством ограниченного
проектора $\bigoplus_\infty \{ P:\lambda\in B_{J^*} \}$. Любая банахова решетка
имеет свойство l.u.st [\cite{DiestAbsSumOps}, теорема 17.1]. Это свойство
наследуется дополняемыми подпространствами, поэтому $\bigoplus_\infty
\{A_+^{***}:\lambda\in B_{J^*} \}$ тоже имеет свойство l.u.st. Заметим, что
$A_+^*$ $1$-дополняемо в $A_+^{***}$ посредством проектора Диксмье $Q$, значит
$\bigoplus_\infty \{A_+^*:\lambda\in B_{J^*} \}$ $1$-дополняемо в
$\bigoplus_\infty \{A_+^{***}:\lambda\in B_{J^*} \}$ посредством проектора
$\bigoplus_\infty \{Q:\lambda\in B_{J^*} \}$. Так как последнее пространство
имеет свойство l.u.st, то им обладает и ретракт $\bigoplus_\infty
\{A_+^*:\lambda\in B_{J^*} \}$. Наконец, $J$ --- это ретракт $\bigoplus_\infty
\{A_+^*:\lambda\in B_{J^*} \}$, поэтому он тоже обладает этим свойством.

Если $F$ --- топологически плоский $A$-модуль, то $F^*$ топологически инъективен
по предложению~\ref{MetTopFlatCharac}. Из рассуждений предыдущего абзаца
следует, что $F^*$ имеет свойство l.u.st. Из [\cite{DiestAbsSumOps}, следствие
17.6] мы заключаем, что этим свойством обладает и сам модуль $F$.

Наконец, если $P$ --- топологически проективный $A$-модуль, то он топологически
плоский по предложению~\ref{MetTopProjIsMetTopFlat}. Из  предыдущего абзаца мы
знаем, что в этом случае $P$ имеет свойство l.u.st.
\end{proof}

%-------------------------------------------------------------------------------
%	Further properties of projective injective and flat modules
%-------------------------------------------------------------------------------

\section{
    Дальнейшие свойства проективных, инъективных и плоских модулей
}\label{
    SectionFurtherPropertiesOfProjectiveInjectiveAndFlatModules
}

%-------------------------------------------------------------------------------
%	Homological triviality of modules under change of algebra
%-------------------------------------------------------------------------------

\subsection{
    Гомологическая тривиальность модулей при замене алгебры
}\label{
    SubSectionHomologicalTrivialityOfModulesUnderChangeOfAlgebra
}

В дальнейшем при исследовании метрически и топологически гомологически
тривиальных модулей над различными алгебрами анализа нам очень пригодятся
следующие предложения. Они суть метрически и топологические версии предложений
2.3.2, 2.3.3 и 2.3.4 из~\cite{RamsHomPropSemgroupAlg}.

\begin{proposition}\label{MorphCoincide} Пусть $X$ и $Y$ --- $\langle$~левые /
правые~$\rangle$ банаховы $A$-модули. Допустим, что выполнено одно из условий:
\begin{enumerate}[label = (\roman*)]
    \item $I$ --- $\langle$~левый / правый~$\rangle$ идеал в $A$, и $X$ ---
    существенный $I$-модуль;

    \item $I$ --- $\langle$~правый / левый~$\rangle$ идеал в $A$, и $Y$ ---
    верный $I$-модуль.
\end{enumerate}

Тогда $\langle$~${}_A\mathcal{B}(X,Y)={}_I\mathcal{B}(X,Y)$ /
$\mathcal{B}_A(X,Y)=\mathcal{B}_I(X,Y)$~$\rangle$.
\end{proposition}
\begin{proof} Мы рассмотрим случай только левых модулей, так как для правых
модулей доказательства аналогичны. Возьмем произвольный морфизм $\phi\in
{}_I\mathcal{B}(X,Y)$.

$(i)$ Рассмотрим $x\in I\cdot X$, тогда $x=a'\cdot x'$ для некоторых $a'\in I$,
$x'\in X$. Для любого $a\in A$ выполнено $\phi(a\cdot x)=\phi(aa'\cdot
x')=aa'\cdot\phi(x')=a\cdot\phi(a'\cdot x')=a\cdot\phi(x)$. Следовательно,
$\phi(a\cdot x)=a\cdot\phi(x)$ для всех $a\in A$ и $x\in
\operatorname{cl}_X(IX)=X$. Значит, $\phi\in {}_A\mathcal{B}(X,Y)$.

$(ii)$ Для любых $a\in I$, $a'\in A$ и $x\in X$ выполнено $a\cdot(\phi(a'\cdot
x)-a'\cdot\phi(x))=\phi(aa'\cdot x)-aa'\cdot\phi(x)=0$. Так как $Y$ --- верный
$I$-модуль, то $\phi(a'\cdot x)=a'\cdot \phi(x)$ для всех $x\in X$, $a'\in A$.
Значит, $\phi\in{}_A\mathcal{B}(X,Y)$.

В обоих случаях мы доказали, что $\phi\in{}_A\mathcal{B}(X,Y)$ для любого
$\phi\in{}_I\mathcal{B}(X,Y)$, следовательно, ${}_I\mathcal{B}(X,Y)\subset
{}_A\mathcal{B}(X,Y)$. Обратное включение очевидно.
\end{proof}

\begin{proposition}\label{MetTopProjUnderChangeOfAlg} Пусть $I$ --- замкнутая
подалгебра в $A$, и $P$ --- банахов $A$-модуль, существенный как $I$-модуль.
Тогда:
\begin{enumerate}[label = (\roman*)]
    \item если $I$ --- левый идеал в $A$ и $P$ $\langle$~метрически /
    $C$-топологически~$\rangle$ проективен как $I$-модуль, то $P$
    $\langle$~метрически / $C$-топологически~$\rangle$ проективен как
    $A$-модуль;

    \item если $I$ --- $\langle$~$1$-дополняемый / $c$-дополняемый~$\rangle$
    правый идеал $A$ и $P$ $\langle$~метрически / $C$-топологически~$\rangle$
    проективен как $A$-модуль, то $P$ $\langle$~метрически /
    $cC$-топологически~$\rangle$ проективен как $I$-модуль.
\end{enumerate}
\end{proposition}
\begin{proof} Через $\widetilde{\pi}_P: I\projtens \ell_1(B_P)\to P$ и
$\pi_P:A\projtens \ell_1(B_P)\to P$ мы обозначим стандартные эпиморфизмы.

$(i)$ По предложению~\ref{NonDegenMetTopProjCharac} морфизм $\widetilde{\pi}_P$
имеет в $\langle$~$I-\mathbf{mod}_1$ / $I-\mathbf{mod}$~$\rangle$ правый
обратный морфизм  $\widetilde{\sigma}$ нормы $\langle$~не более $1$ / не более
$C$~$\rangle$. Пусть $i:I\to A$ --- естественное вложение, тогда рассмотрим
$\langle$~сжимающий / ограниченный~$\rangle$ $I$-морфизм $\sigma=(i\projtens
1_{\ell_1(B_P)})\widetilde{\sigma}$. По предложению~\ref{MorphCoincide} оператор
$\sigma$ является $A$-морфизмом. Очевидно, $\sigma$ имеет норму $\langle$~не
более $1$ / не более $C$~$\rangle$. Для $\pi_P:A\projtens \ell_1(B_P)\to P$
выполнено $\pi_P(i\projtens 1_{\ell_1(B_P)})=\widetilde{\pi}_P$, следовательно,
$\pi_P\sigma=\pi_P(i\projtens
1_{\ell_1(B_P)})\widetilde{\sigma}=\widetilde{\pi}_P\widetilde{\sigma}=1_P$.
Таким образом, $\pi_P$ есть $\langle$~$1$-ретракция / $C$-ретракция~$\rangle$ в
$\langle$~$A-\mathbf{mod}_1$ / $A-\mathbf{mod}$~$\rangle$. По
предложению~\ref{NonDegenMetTopProjCharac} банахов $A$-модуль $P$
$\langle$~метрически / $C$-топологически~$\rangle$ проективен.

$(ii)$ Поскольку $P$ существенный $I$-модуль, он тем более будет существенным
$A$-модулем. По предложению~\ref{NonDegenMetTopProjCharac} морфизм $\pi_P$ имеет
правый обратный морфизм $\sigma$ в $\langle$~$A-\mathbf{mod}_1$ /
$A-\mathbf{mod}$~$\rangle$ нормы $\langle$~не более $1$ / не более
$C$~$\rangle$. Ясно,что $\sigma$ является правым обратным морфизмом в $\pi_P$ и
в $\langle$~$I-\mathbf{mod}_1$ / $I-\mathbf{mod}$~$\rangle$. Через $i:I\to A$ мы
обозначим естественное вложение, а через $r:A\to I$ $\langle$~сжимающий /
ограниченный~$\rangle$ левый обратный оператор. По предположению $\Vert
r\Vert\leq c$. Рассмотрим $\langle$~сжимающий / ограниченный~$\rangle$ линейный
оператор $\widetilde{\sigma}=(r\projtens 1_{\ell_1(B_P)})\sigma$. Очевидно, его
норма $\langle$~не превосходит $1$ / не превосходит $cC$~$\rangle$. Так как $I$
--- правый идеал в $A$ и $P$ является существенным $I$-модулем, то
$\sigma(P)=\sigma(\operatorname{cl}_P(IP))=\operatorname{cl}_{A\projtens
\ell_1(B_P)}(I\cdot (A\projtens \ell_1(B_P)))=I\projtens \ell_1(B_P)$, поэтому
$\sigma=(ir\projtens 1_{\ell_1(B_P)})\sigma$. Более того, так как
$\sigma(P)\subset I\projtens\ell_1(B_P)$ и $r|_I=1_I$, то $\sigma$ является
$I$-морфизмом. Очевидно, $\pi_P(i\projtens 1_{\ell_1(B_P)})=\widetilde{\pi}_P$,
поэтому $\widetilde{\pi}_P\widetilde{\sigma}=\pi_P(i\projtens
1_{\ell_1(B_P)})(r\projtens 1_{\ell_1(B_P)})\sigma=\pi_P(ir\projtens
1_{\ell_1(B_P)})\sigma=\pi_P\sigma=1_P$. Таким образом, $\widetilde{\pi}_P$ ---
$\langle$~$1$-ретракция / $cC$-ретракция~$\rangle$ в
$\langle$~$I-\mathbf{mod}_1$ / $I-\mathbf{mod}$~$\rangle$, поэтому из
предложения~\ref{NonDegenMetTopProjCharac} следует, что $I$-модуль $P$
$\langle$~метрически / $cC$-топологически~$\rangle$ проективен.
\end{proof}

\begin{proposition}\label{MetTopInjUnderChangeOfAlg} Пусть $I$ --- замкнутая
подалгебра в $A$, и $J$ --- правый банахов $A$-модуль верный как $I$-модуль.
Тогда:
\begin{enumerate}[label = (\roman*)]
    \item если $I$ --- левый идеал в $A$ и $J$ $\langle$~метрически /
    $C$-топологически~$\rangle$ инъективный $I$-модуль, то $J$
    $\langle$~метрически / $C$-топологически~$\rangle$ инъективен как
    $A$-модуль;

    \item если $I$ --- $\langle$~$1$-дополняемый / $c$-дополняемый~$\rangle$
    правый идеал $A$ и $J$ $\langle$~метрически / $C$-топологически~$\rangle$
    инъективен как $A$-модуль, то $J$ $\langle$~метрически /
    $cC$-топологически~$\rangle$ инъективен как $I$-модуль.
\end{enumerate}
\end{proposition}
\begin{proof} Через $\widetilde{\rho}_J:J\to\mathcal{B}(I,\ell_\infty(B_{J^*}))$
и $\rho_J:J\to\mathcal{B}(A,\ell_\infty(B_{J^*}))$ мы обозначим стандартные
мономорфизмы.

$(i)$ По предложению~\ref{NonDegenMetTopInjCharac} морфизм $\widetilde{\rho}_J:
J\to\mathcal{B}(I,\ell_\infty(B_{J^*}))$ имеет левый обратный морфизм в
$\langle$~$\mathbf{mod}_1-I$ / $\mathbf{mod}-I$~$\rangle$, скажем,
$\widetilde{\tau}$ нормы $\langle$~не более $1$ / не более $C$~$\rangle$. Пусть
$i:I\to A$ --- естественное вложение, тогда рассмотрим $I$-морфизм
$q=\mathcal{B}(i,\ell_\infty(B_{J^*}))$. Очевидно $\widetilde{\rho}_J=q\rho_J$.
Рассмотрим $I$-морфизм $\tau =\widetilde{\tau} q$. По
предложению~\ref{MorphCoincide} он также является $A$-морфизмом. Заметим, что
$\Vert\tau \Vert\leq\Vert\widetilde{\tau}\Vert\Vert
q\Vert\leq\Vert\widetilde{\tau}\Vert$, поэтому и $\tau$ имеет норму $\langle$~не
более $1$ / не более $C$~$\rangle$. Ясно, что $\tau \rho_J=\widetilde{\tau}
q\rho_J=\widetilde{\tau}\widetilde{\rho}_J=1_J$. Таким образом, $\rho_J$ есть
$\langle$~$1$-коретракция / $C$-коретракция~$\rangle$ в
$\langle$~$\mathbf{mod}_1-A$ / $\mathbf{mod}-A$~$\rangle$, поэтому из
предложения~\ref{NonDegenMetTopInjCharac} следует, что $A$-модуль $J$
$\langle$~метрически / $C$-топологически~$\rangle$ инъективен.

$(ii)$ Поскольку $J$ $\langle$~метрически / $C$-топологически~$\rangle$
инъективен как $A$-модуль, то из предложения~\ref{NonDegenMetTopInjCharac}
морфизм $\rho_J$ имеет левый обратный морфизм в $\langle$~$\mathbf{mod}_1-A$ /
$\mathbf{mod}-A$~$\rangle$, скажем, $\tau $ нормы $\langle$~не более $1$ / не
более $C$~$\rangle$. Допустим, нам дан оператор $T\in
\mathcal{B}(A,\ell_\infty(B_{J^*}))$ такой, что $T|_I=0$. Зафиксируем $a\in I$,
тогда $T\cdot a=0$, и поэтому $\tau (T)\cdot a=\tau (T\cdot a)=0$. Так как $J$
является верным $I$-модулем и $a\in I$ произволен, то $\tau (T)=0$. Через
$r:A\to I$  мы обозначим оператор левый обратный к $i$. Он существует по
предположению и его норма не превосходит $c$. Определим ограниченные линейные
операторы $j=\mathcal{B}(r,\ell_\infty(B_{J^*}))$ и $\widetilde{\tau}=\tau  j$.
Для любого $a\in I$ и $T\in\mathcal{B}(I,\ell_\infty(B_{J^*}))$ выполнено
$\widetilde{\tau}(T\cdot a)-\widetilde{\tau}(T)\cdot a=\tau (j(T\cdot
a)-j(T)\cdot a)=0$ потому, что $j(T\cdot a)-j(T)\cdot a|_I=0$. Следовательно,
$\widetilde{\tau}$ есть $I$-морфизм. Заметим, что
$\Vert\widetilde{\tau}\Vert\leq\Vert\tau \Vert\Vert j\Vert$, поэтому
$\widetilde{\tau}$ имеет норму $\langle$~не более $1$ / не более $cC$~$\rangle$.
Очевидно, для всех $x\in J$ выполнено $\rho_J(x)-j(\widetilde{\rho}_J(x))|_I=0$,
поэтому $\tau (\rho_J(x)-j(\widetilde{\rho}_J(x)))=0$. Как следствие,
$\widetilde{\tau}(\widetilde{\rho}_J(x))=\tau (j(\widetilde{\rho}_J(x)))=\tau
(\rho_J(x))=x$ для всех $x\in J$. Так как
$\widetilde{\tau}\widetilde{\rho}_J=1_J$, то $\widetilde{\rho}_J$ ---
$\langle$~$1$-коретракция / $cC$-коретракция~$\rangle$ в
$\langle$~$\mathbf{mod}_1-I$ / $\mathbf{mod}-I$~$\rangle$, поэтому из
предложения~\ref{NonDegenMetTopInjCharac} следует, что $I$-модуль $J$
$\langle$~метрически / $cC$-топологически~$\rangle$ инъективен.
\end{proof}

\begin{proposition}\label{MetTopFlatUnderChangeOfAlg} Пусть $I$ --- замкнутая
подалгебра в $A$, и $F$ --- банахов $A$-модуль существенный как $I$-модуль.
Тогда:
\begin{enumerate}[label = (\roman*)]
    \item если $I$ --- левый идеал в $A$ и $F$ $\langle$~метрически /
    $C$-топологически~$\rangle$ плоский $I$-модуль, то $F$ $\langle$~метрически
    / $C$-топологически~$\rangle$ плоский $A$-модуль;

    \item если $I$ --- $\langle$~$1$-дополняемый / $c$-дополняемый~$\rangle$
    правый идеал $A$ и $F$ есть $\langle$~метрически /
    $C$-топологически~$\rangle$ плоский $A$-модуль, то $F$ $\langle$~метрически
    / $cC$-топологически~$\rangle$ плоский $I$-модуль.
\end{enumerate}
\end{proposition}
\begin{proof} Заметим, что модуль, сопряженный к существенному модулю, будет
верным. Теперь все результаты следуют из предложений~\ref{MetTopFlatCharac}
и~\ref{MetTopInjUnderChangeOfAlg}.
\end{proof}

\begin{proposition}\label{MetTopProjInjFlatUnderSumOfAlg} Пусть
${(A_\lambda)}_{\lambda\in\Lambda}$ --- семейство банаховых алгебр и для каждого
$\lambda\in\Lambda$ пусть $X_\lambda$ ---  $\langle$~существенный / верный /
существенный~$\rangle$ банахов $A_\lambda$-модуль. Обозначим $A=\bigoplus_p
\{A_\lambda:\lambda\in\Lambda \}$, где $1\leq p\leq +\infty$ или $p=0$. Пусть
$X$ обозначает $\langle$~$\bigoplus_1 \{X_\lambda:\lambda\in\Lambda \}$ /
$\bigoplus_\infty \{X_\lambda:\lambda\in\Lambda \}$ / $\bigoplus_1
\{X_\lambda:\lambda\in\Lambda \}$~$\rangle$. Тогда:
\begin{enumerate}[label = (\roman*)]
    \item $X$ метрически $\langle$~проективный / инъективный / плоский~$\rangle$
    $A$-модуль тогда и только тогда, когда для всех $\lambda\in\Lambda$ банахов
    $A_\lambda$-модуль $X_\lambda$ метрически $\langle$~проективный /
    инъективный / плоский~$\rangle$;

    \item $X$ $C$-топологически $\langle$~проективный / инъективный /
    плоский~$\rangle$ $A$-модуль тогда и только тогда, когда для всех
    $\lambda\in\Lambda$ банахов $A_\lambda$-модуль $X_\lambda$ $C$-топологически
    $\langle$~проективный / инъективный / плоский~$\rangle$.
\end{enumerate}
\end{proposition}
\begin{proof} Заметим, что для каждого $\lambda\in\Lambda$ естественное вложение
$i_\lambda:A_\lambda\to A$ позволяет рассматривать $A_\lambda$ как
$1$-дополняемый  двусторонний идеал в $A$.

$(i)$ Доказательство дословно повторяет рассуждения из пункта $(ii)$.

$(ii)$ Допустим $X_\lambda$ $C$-топологически $\langle$~проективный / инъективный
/ плоский~$\rangle$ банахов $A_\lambda$-модуль для всех $\lambda\in\Lambda$,
тогда из пункта $(i)$ предложения $\langle$~\ref{MetTopProjUnderChangeOfAlg}
/~\ref{MetTopInjUnderChangeOfAlg} /~\ref{MetTopFlatUnderChangeOfAlg}~$\rangle$
этот модуль $C$-топологически $\langle$~проективный / инъективный /
плоский~$\rangle$ как $A$-модуль. Осталось применить предложение
$\langle$~\ref{MetTopProjModCoprod} /~\ref{MetTopInjModProd}
/~\ref{MetTopFlatModCoProd}~$\rangle$. 

Обратно, допустим, что $X$ $C$-топологически $\langle$~проективный / инъективный
/ плоский~$\rangle$ $A$-модуль. Зафиксируем произвольный $\lambda\in\Lambda$.
Очевидно, мы можем рассматривать $X_\lambda$ как $A$-модуль и более того
$X_\lambda$, очевидно, является $1$-ретрактом $X$ в $\langle$~$A-\mathbf{mod}_1$
/ $\mathbf{mod}_1-A$ / $A-\mathbf{mod}_1$~$\rangle$. По предложению
$\langle$~\ref{RetrCTopProjIsCTopProj} /~\ref{RetrCTopInjIsCTopInj}
/~\ref{RetrCTopFlatIsCTopFlat}~$\rangle$ модуль $X_\lambda$ $C$-топологически
$\langle$~проективный / инъективный / плоский~$\rangle$ как $A$-модуль. Осталось
применить пункт  $(ii)$ предложения $\langle$~\ref{MetTopProjUnderChangeOfAlg}
/~\ref{MetTopInjUnderChangeOfAlg} /~\ref{MetTopFlatUnderChangeOfAlg}~$\rangle$.
\end{proof} 

%-------------------------------------------------------------------------------
%	Flat modules and injective ideals
%-------------------------------------------------------------------------------

\subsection{
    Плоские модули и инъективные идеалы
}\label{
    SubSectionFlatModulesAndInjectiveIdeals
}

Используя результаты предыдущих параграфов, мы докажем еще несколько полезных
фактов о метрической и топологической инъективности и плоскости банаховых
модулей.

\begin{proposition}\label{DualBanModDecomp} Пусть $B$ --- унитальная банахова
алгебра, $A$ --- ее подалгебра с двусторонней ограниченной аппроксимативной
единицей ${(e_\nu)}_{\nu\in N}$ и пусть $X$ --- унитальный левый $B$-модуль.
Тогда:
\begin{enumerate}[label = (\roman*)]
    \item $X^*\isom{\mathbf{mod}-A}X_{ess}^*\bigoplus_\infty {(X/X_{ess})}^*$,
    где $X_{ess}:=\operatorname{cl}_X(AX)$;

    \item $X_{ess}^*$ есть $C_1$-дополняемый $A$-подмодуль в $X^*$ для
    $C_1=\sup_{\nu\in N}\Vert e_\nu\Vert$;

    \item ${(X/X_{ess})}^*$ есть $C_2$-дополняемый $A$-подмодуль в $X^*$ для
    $C_2=\sup_{\nu\in N}\Vert e_B - e_\nu\Vert$;

    \item если $X$ является $\mathscr{L}_1$-пространством, то $X_{ess}$ и
    $X/X_{ess}$ тоже $\mathscr{L}_1$-пространства.
\end{enumerate}
\end{proposition}
\begin{proof} Рассмотрим естественное вложение $\rho:X_{ess}\to X:x\mapsto x$ и
фактор-отображение $\pi:X\to X/X_{ess}:x\mapsto x+X_{ess}$. Пусть $\mathfrak{F}$
--- фильтр сечений на $N$ и пусть $\mathfrak{U}$ ультрафильтр содержащий
$\mathfrak{F}$. Для заданного $f\in X ^*$ и $x\in X $ выполнено $|f(x-e_\nu\cdot
x)|\leq\Vert f\Vert\Vert e_B - e_\nu\Vert\Vert x\Vert\leq C_2\Vert f\Vert\Vert
x\Vert$, то есть ${(f(x-e_\nu\cdot x))}_{\nu\in N}$ --- ограниченная
направленность комплексных чисел. Следовательно, корректно определен предел
$\lim_{\mathfrak{U}}f(x-e_\nu\cdot x)$ вдоль ультрафильтра $\mathfrak{U}$.
Поскольку ${(e_\nu)}_{\nu\in N}$ есть двусторонняя аппроксимативная единица для
$A$ и $\mathfrak{U}$ содержит фильтр сечений, то для всех $x\in X_{ess}$
выполнено $\lim_{\mathfrak{U}}f(x-e_\nu\cdot x)=\lim_{\nu}f(x-e_\nu\cdot x)=0$.
Следовательно, для каждого $f\in X ^*$ мы имеем корректно определенное
отображение $\tau(f):X /X_{ess}\to \mathbb{C}:x+X_{ess}\mapsto
\lim_{\mathfrak{U}} f(x-e_\nu\cdot x)$. Очевидно, это линейный функционал и из
неравенств выше мы видим, что его норма ограничена сверху константой $C_2\Vert
f\Vert$. Теперь легко проверить, что 
$\tau:X^*\to {(X/ X_{ess})}^*:f\mapsto \tau(f)$ 
есть $A$-морфизм нормы не более $C_2$. Аналогично можно показать, что
$\sigma:X_{ess}^*\to X^*:h\mapsto(x\mapsto \lim_{\mathfrak{U}}h(e_\nu\cdot x))$
есть $A$-морфизм с нормой не превосходящей $C_1$. Для любых $f\in X^*$, 
$g\in {(X/X_{ess})}^*$, $h\in X_{ess}^*$ и $x\in X$, $y\in X_{ess}$ мы имеем
$$
\sigma(h)(y)
=\lim_{\mathfrak{U}}h(e_\nu\cdot y)
=\lim_{\nu}h(e_\nu\cdot y)
=h(y),
\qquad
(\rho^*\sigma)(h)(y)
=\sigma(h)(\rho(y))
\sigma(h)(y)
=h(y),
$$
$$
(\tau\pi^*)(g)(x+X_{ess})
=\lim_{\mathfrak{U}}\pi^*(g)(x-e_\nu\cdot x)
=\lim_{\mathfrak{U}}g(\pi(x-e_\nu\cdot x))
=\lim_{\mathfrak{U}}g(x+X_{ess})
=g(x+X_{ess}),
$$
$$
(\tau\sigma)(h)(x+X_{ess})
=\lim_{\mathfrak{U}}\sigma(h)(x-e_\nu\cdot x)
=\lim_{\mathfrak{U}}(\sigma(h)(x)-h(e_\nu\cdot x))
=\sigma(h)(x)-\lim_{\mathfrak{U}}h(e_\nu\cdot x)=0,
$$
$$
(\pi^*\tau + \sigma\rho^*)(f)(x)
=\tau(f)(x+X_{ess})+\lim_{\mathfrak{U}}\rho^*(f)(e_\nu\cdot x)
=\lim_{\mathfrak{U}}f(x - e_\nu\cdot x)+\lim_{\mathfrak{U}}f(e_\nu\cdot x)
=f(x).
$$
Следовательно, $\tau \pi^*=1_{{(X/X_{ess})}^*}$, $\rho^*\sigma=1_{X_{ess}^*}$,
$\rho^*\pi^*=0$, $\tau\sigma=0$ и $\pi^*\tau+\sigma\rho^*=1_{X^*}$. Это
эквивалентно тому, что 
$X^*\isom{\mathbf{mod}-A}X_{ess}^*\bigoplus_\infty {(X/X_{ess})}^*$.

Теперь рассмотрим $A$-морфизмы $P_1=\sigma\rho^*$, $P_2=\pi^*\tau$. Их равенств
выше следует, что $P_1^2=P_1$, $P_2^2=P_2$ и
$\operatorname{Im}(P_1)\isom{\mathbf{mod}-A}X_{ess}^*$,
$\operatorname{Im}(P_2)\isom{\mathbf{mod}-A} {(X/X_{ess})}^*$. Нормы этих
проекторов легко оценить: $\Vert P_1\Vert\leq\Vert \sigma\Vert\Vert
\rho^*\Vert=C_1$ и $\Vert P_2\Vert\leq \Vert \pi^*\Vert\Vert\tau\Vert\leq C_2$.

Теперь рассмотрим случай, когда $X$ является $\mathscr{L}_1$-пространством.
Тогда $X^*$ есть $\mathscr{L}_\infty$-пространство [\cite{BourgNewClOfLpSp},
предложение 1.27]. Так как $X_{ess}^*$ и ${(X/X_{ess})}^*$ дополняемы в $X^*$, то
они тоже являются $\mathscr{L}_\infty$-пространствами [\cite{BourgNewClOfLpSp},
предложение 1.28]. Снова из [\cite{BourgNewClOfLpSp}, предложение 1.27] мы
получаем, что $X_{ess}$ и $X/X_{ess}$ являются $\mathscr{L}_1$-пространствами.
\end{proof}

Следующее предложение является аналогом [\cite{RamsHomPropSemgroupAlg},
предложение 2.1.11] для топологической теории.

\begin{proposition}\label{TopFlatModCharac} Пусть $A$ --- банахова алгебра с
двусторонней ограниченной аппроксимативной единицей, и пусть $F$ --- левый
$A$-модуль. Тогда следующие условия эквивалентны:
\begin{enumerate}[label = (\roman*)]
    \item $F$ --- топологически плоский $A$-модуль;

    \item $F_{ess}$ --- топологически плоский $A$-модуль и $F/F_{ess}$ является
    $\mathscr{L}_1$-пространством.
\end{enumerate}
\end{proposition}
\begin{proof} Будем рассматривать $A$ как подалгебру унитальной банаховой
алгебры $B:=A_+$. Тогда $F$ --- унитальный левый $B$-модуль. Из
предложения~\ref{DualBanModDecomp} мы имеем
$F^*\isom{\mathbf{mod}-A}F_{ess}^*\bigoplus_\infty {(F/F_{ess})}^*$. Далее
предложения~\ref{MetTopFlatCharac} и~\ref{MetTopInjModProd} дают, что $A$-модуль
$F$ топологически плоский тогда и только тогда, когда таковы $F_{ess}$ и
$F/F_{ess}$. Осталось заметить, что по
предложению~\ref{MetTopFlatAnnihModCharac} аннуляторный $A$-модуль $F/F_{ess}$
является топологически плоским тогда и только тогда, когда он является
$\mathscr{L}_1$-пространством.
\end{proof}

Теперь нам необходимо более детально ознакомится с понятием аменабельной
банаховой алгебры. Рассмотрим морфизм $A$-бимодулей $\Pi_A:A\projtens A\to
A:a\projtens b\mapsto ab$. Банахова алгебра $A$ называется относительно
$C$-аменабельной если $\Pi_{A_+}^*$ является $C$-коретракцией $A$-бимодулей.
Будем говорить, что банахова алгебра $A$ относительно аменабельна, если она
относительно $C$-аменабельна для некоторого $C\geq 1$.  С небольшими изменениями
в [\cite{HelBanLocConvAlg}, предложение 7.1.72] можно показать, что $A$
относительно $C$-аменабельна тогда и только тогда, когда существует
направленность ${(d_\nu)}_{\nu\in N}\subset A\projtens A$ по норме не
превосходящая $C$, причем для всех $a\in A$ выполнено $\lim_\nu(a\cdot
d_\nu-d_\nu\cdot a)=0$ и $\lim_\nu a\Pi_A(d_\nu)=a$. Такая направленность
называется аппроксимативной диагональю. С гомологической точки зрения, основное
преимущество относительно аменабельной алгебры в том, что любой левый и правый
банахов модуль над ней является относительно плоским [\cite{HelBanLocConvAlg},
теорема 7.1.60].

\begin{proposition}\label{MetTopEssL1FlatModAoverAmenBanAlg} Пусть $A$
---относительно $\langle$~$1$-аменабельная / $c$-аменабельная~$\rangle$ банахова
алгебра и $F$ --- существенный банахов $A$-модуль, являющийся, как банахово
пространство, $\langle$~$L_1$-пространством /
$\mathscr{L}_{1,C}$-пространством~$\rangle$. Тогда $F$ --- $\langle$~метрически
/ $c^2C$-топологически~$\rangle$ плоский $A$-модуль.
\end{proposition}
\begin{proof} 
Мы можем считать, что $A$ относительно $c$-аменабельна с $\langle$~$c=1$ /
$c\geq 1$~$\rangle$. Пусть ${(d_\nu)}_{\nu\in N}$ --- аппроксимативная диагональ
для $A$ по норме не превосходящая $c$. Напомним, что ${(\Pi_A(d_\nu))}_{\nu\in N}$
есть двусторонняя $\langle$~сжимающая / ограниченная~$\rangle$ аппроксимативная
единица в $A$. Так как $F$ --- существенный $A$-модуль, то
$\lim_{\nu}\Pi_A(d_\nu)\cdot x=x$ для всех $x\in F$ [\cite{HelHomolBanTopAlg},
предложение 0.3.15]. Как следствие, множество
$c\pi_F(B_{A\projtens\ell_1(B_F)})$ плотно в $B_F$. Тогда для любого $f\in F^*$
выполнено
$$
\Vert\pi_F^*(f)\Vert
=\sup \{|f(\pi_F(u))|:u\in B_{A\projtens\ell_1(B_F)} \}
=\sup \{|f(x)|:x\in \operatorname{cl}_F(\pi_F(B_{A\projtens\ell_1(B_F)})) \}
$$
$$
\geq\sup \{c^{-1}|f(x)|:x\in B_F \}=c^{-1}\Vert f\Vert
$$
Это означает, что $\pi_F^*$ --- $c$-топологически инъективный оператор. По
предположению $F$ является $\langle$~$L_1$-пространством /
$\mathscr{L}_{1,C}$-пространством~$\rangle$, тогда из
$\langle$~[\cite{GrothMetrProjFlatBanSp}, теорема 1] /
[\cite{StegRethNucOpL1LInfSp}, теорема VI.6]~$\rangle$ следует, что банахово
пространство $F^*$ будет $\langle$~метрически / $C$-топологически~$\rangle$
инъективным. Так как оператор $\pi_F^*$ $\langle$~изометричен /
$c$-топологически инъективен~$\rangle$, то существует линейный оператор
$R:{(A\projtens\ell_1(B_F))}^*\to F^*$ нормы $\langle$~не более $1$ / не более
$cC$~$\rangle$ такой, что $R\pi_F^*=1_{F^*}$.

Пусть $h\in {(A\projtens\ell_1(B_F))}^*$ и $x\in F$. Рассмотрим билинейный
функционал $M_{h,x}:A\times A\to\mathbb{C}:(a,b)\mapsto R(h\cdot a)(b\cdot x)$.
Очевидно, $\Vert M_{h,x}\Vert\leq\Vert R\Vert\Vert h\Vert\Vert x\Vert$. По
свойству универсальности проективного тензорного произведения мы получаем
ограниченный линейный функционал $m_{h,x}:A\projtens A\to\mathbb{C}:a\projtens
b\mapsto R(h\cdot a)(b\cdot x)$. Отметим, что $m_{h,x}$ линеен по $h$ и $x$.
Более того, для любых $u\in A\projtens A$, $a\in A$ и $f\in F^*$ выполнено
$m_{\pi_F^*(f),x}(u)=f(\Pi_A(u)\cdot x)$, $m_{h\cdot a,x}(u)=m_{h,x}(a\cdot u)$,
$m_{h,a\cdot x}(u)=m_{h,x}(u\cdot a)$. Это легко проверить на элементарных
тензорах. Далее остается заметить, что их линейная оболочка плотна в $A\projtens
A$.

Пусть $\mathfrak{F}$ --- фильтр сечений на $N$ и пусть $\mathfrak{U}$ ---
ультрафильтр содержащий $\mathfrak{F}$. Для всех 
$h\in {(A\projtens\ell_1(B_F))}^*$ и $x\in F$ мы имеем 
$|m_{h,x}(d_\nu)|\leq c\Vert R\Vert\Vert h\Vert\Vert x\Vert$, 
то есть ${(m_{h,x}(d_\nu))}_{\nu\in N}$ ---
ограниченная направленность комплексных чисел. Следовательно корректно определен
предел $\lim_{\mathfrak{U}}m_{h,x}(d_\nu)$ вдоль ультрафильтра $\mathfrak{U}$.
Рассмотрим линейный оператор 
$\tau:{(A\projtens\ell_1(B_F))}^*\to F^*
:h\mapsto(x\mapsto\lim_{\mathfrak{U}}m_{h,x}(d_\nu))$. Из оценок на норму
$m_{h,x}$ следует, что $\tau$ --- ограниченный линейный оператор, причем
$\Vert\tau\Vert\leq c\Vert R\Vert$. Для всех $a\in A$, $x\in F$ и 
$h\in {(A\projtens\ell_1(B_F))}^*$ выполнено
$$
\tau(h\cdot a)(x)-(\tau(h)\cdot a)(x)
=\tau(h\cdot a)(x)-\tau(h)(a\cdot x)
=\lim_{\mathfrak{U}}m_{h\cdot a,x}(d_\nu)-\lim_{\mathfrak{U}}m_{h,a\cdot x}(d_\nu).
$$
$$
=\lim_{\mathfrak{U}}m_{h,x}(a\cdot d_\nu)-m_{h,x}(d_\nu\cdot a)
=m_{h,x}\left(\lim_{\mathfrak{U}}(a\cdot d_\nu-d_\nu\cdot a)\right)
$$
$$
=m_{h,x}\left(\lim_{\nu}(a\cdot d_\nu-d_\nu\cdot a)\right)
=m_{h,x}(0)
=0.
$$
Следовательно, $\tau$ --- морфизм правых $A$-модулей. Далее, для всех $f\in F^*$
и $x\in F$ мы имеем
$$
(\tau(\pi_F^*)(f))(x)
=\lim_{\mathfrak{U}}m_{\pi_F^*(f),x}(d_\nu)
=\lim_{\mathfrak{U}}f(\Pi_A(d_\nu)\cdot x)
=\lim_{\nu}f(\Pi_A(d_\nu)\cdot x)
$$
$$
=f\left(\lim_{\nu}\Pi_A(d_\nu)\cdot x\right)
=f(x).
$$
Поэтому $\tau\pi_F^*=1_{F^*}$. Это значит, что $F^*$ --- $\langle$~$1$-ретракт /
 $c^2 C$-ретракт~$\rangle$ модуля ${(A\projtens\ell_1(B_F))}^*$ в
 $\langle$~$\mathbf{mod}_1-A$ / $\mathbf{mod}-A$~$\rangle$. Последний модуль
 $\langle$~метрически / $1$-топологически~$\rangle$ инъективен, потому что
 ${(A_+\projtens\ell_1(B_F))}^*
 \isom{\mathbf{mod}_1-A}\mathcal{B}(A_+,\ell_\infty(B_F))$.
 По предложению $\langle$~\ref{RetrMetTopInjIsMetTopInj}
 /~\ref{RetrCTopInjIsCTopInj}~$\rangle$ модуль $F^*$ также будет
 $\langle$~метрически / $c^2 C$-топологически~$\rangle$ инъективным.
\end{proof}

\begin{theorem}\label{TopL1FlatModAoverAmenBanAlg} Пусть $A$ --- относительно
аменабельная банахова алгебра и $F$ --- левый банахов $A$-модуль, являющийся,
как банахово пространство, $\mathscr{L}_1$-пространством. Тогда $F$ ---
топологически плоский $A$-модуль.
\end{theorem}
\begin{proof} Так как $A$ аменабельна, то она обладает двусторонней ограниченной
аппроксимативной единицей. По предложению~\ref{DualBanModDecomp} существенный
$A$-модуль $F_{ess}$ и аннуляторный $A$-модуль $F/F_{ess}$ являются
$\mathscr{L}_1$-пространствами. Тогда из
предложений~\ref{MetTopEssL1FlatModAoverAmenBanAlg},~\ref{MetTopFlatAnnihModCharac}
мы получаем, что $F_{ess}$ и $F/F_{ess}$ --- топологически плоские $A$-модули.
Снова из предложению~\ref{DualBanModDecomp} мы имеем
$F^*\isom{\mathbf{mod}-A}F_{ess}^*\bigoplus_\infty {(F/F_{ess})}^*$. Учитывая
вышесказанное, из предложений~\ref{MetTopFlatCharac} и~\ref{MetTopInjModProd} мы
заключаем, что $F$ --- топологически плоский $A$-модуль.
\end{proof}

В топологической банаховой гомологии, в отличие от относительной, для некоторых
алгебр можно дать полное описание плоских модулей.

\begin{corollary}\label{TopFlatModAoverAmenL1BanAlgCharac} Пусть $A$ ---
относительно аменабельная банахова алгебра, являющаяся, как банахово
пространство, $\mathscr{L}_1$-пространством. Тогда для банахова $A$-модуля $F$
следующие условия эквивалентны:
\begin{enumerate}[label = (\roman*)]
    \item $F$ --- топологически плоский $A$-модуль; 

    \item $F$ является $\mathscr{L}_1$-пространством.
\end{enumerate}
\end{corollary}
\begin{proof} Эквивалентность следует из
предложения~\ref{TopProjInjFlatModOverMthscrL1SpCharac} и
теоремы~\ref{TopL1FlatModAoverAmenBanAlg}.
\end{proof}

Теперь мы можем дать пример относительно плоского, но не топологически плоского
идеала в банаховой алгебре. Рассмотрим $A=L_1(\mathbb{T})$. Известно, что $A$
имеет трансляционно инвариантное бесконечномерное замкнутое подпространство $I$
изоморфное гильбертову пространству [\cite{RosProjTransInvSbspLpG}, стр. 52].
Так как $I$ трансляционно инвариантно, то из [\cite{KaniBanAlg}, предложение
1.4.7] мы знаем, что $I$ является двусторонним идеалом в $A$. Тогда из
[\cite{DefFloTensNorOpId}, следствие 23.3(4)] этот идеал не может быть
$\mathscr{L}_1$-пространством. Тогда по
следствию~\ref{TopFlatModAoverAmenL1BanAlgCharac} идеал $I$ не является
топологически плоским $A$-модулем. Мы утверждаем, что он все же относительно
плоский. Так как $\mathbb{T}$ --- компактная группа, то она аменабельна
[\cite{PierAmenLCA}, предложение 3.12.1]. Тогда $A$ --- относительно
аменабельная банахова алгебра [\cite{HelBanLocConvAlg}, предложение VII.1.86],
поэтому все ее левые идеалы, в частности $I$, являются относительно плоскими
$A$-модулями [\cite{HelBanLocConvAlg}, предложение VII.1.60(I)].

Перейдем к обсуждению инъективных идеалов банаховых алгебр. Такие идеалы
встретятся нам при изучении метрической и топологической инъективности
$C^*$-алгебр.

\begin{proposition}\label{MetTopInjOfId} Пусть $I$ --- правый идеал банаховой
алгебры $A$. Допустим, $I$ $\langle$~метрически / топологически~$\rangle$
инъективен как $A$-модуль. Тогда $I$ имеет $\langle$~левую единицу нормы $1$ /
левую единицу~$\rangle$ и является ретрактом $A$ в $\langle$~$\mathbf{mod}_1-A$
/ $\mathbf{mod}-A$~$\rangle$.
\end{proposition}
\begin{proof} Рассмотрим изометрическое вложение $\rho^+:I\to A_+$. Ясно, что
это $A$-морфизм. Поскольку $I$ $\langle$~метрически / топологически~$\rangle$
инъективен, то $\rho^+$ имеет $\langle$~сжимающий / ограниченный~$\rangle$ левый
обратный $A$-морфизм $\tau^+:A_+\to I$. Теперь для всех $x\in I$ мы имеем
$x=\tau^+(\rho^+(x))=\tau^+(e_{A_+}\rho^+(x))
=\tau^+(e_{A_+})\rho^+(x)=\tau^+(e_{A_+})x$.
Другими словами $p=\tau^+(e_{A_+})\in I$ есть левая единица для $I$. Очевидно,
$\Vert p\Vert\leq\Vert\tau^+\Vert\Vert e_{A_+}\Vert\leq\Vert\tau^+\Vert$,
поэтому $\langle$~$\Vert p\Vert\leq 1$ / $\Vert p\Vert<\infty$~$\rangle$.
Рассмотрим $A$-модульные операторы $ \rho:I\to A:x\mapsto x$ и $\tau:A\to
I:x\mapsto p x$. Очевидно, они $\langle$~сжимающие / ограниченные~$\rangle$
морфизмы правых $A$-модулей и $\tau\rho=1_I$. Значит, $I$ есть ретракт $A$ в
$\langle$~$\mathbf{mod}_1-A$ / $\mathbf{mod}-A$~$\rangle$.
\end{proof}

\begin{proposition}\label{ReduceInjIdToInjAlg} Пусть $I$ --- двусторонний идеал
банаховой алгебры $A$, являющийся правым верным $I$-модулем. Тогда $I$
$\langle$~метрически / топологически~$\rangle$ инъективен как $A$-модуль тогда и
только тогда, когда он $\langle$~метрически / топологически~$\rangle$ инъективен
как $I$-модуль.
\end{proposition}
\begin{proof} Допустим, $I$ $\langle$~метрически / топологически~$\rangle$
инъективен как $A$-модуль, тогда по предложению~\ref{MetTopInjOfId} он является
ретрактом $A$ в $\langle$~$\mathbf{mod}_1-A$ / $\mathbf{mod}-A$~$\rangle$. Из
пункта $(ii)$ предложения~\ref{MetTopInjUnderChangeOfAlg} мы получаем, что
$I$-модуль $I$ $\langle$~метрически / топологически~$\rangle$ инъективен. 

Обратная импликация непосредственно следует из пункта $(i)$
предложения~\ref{MetTopInjUnderChangeOfAlg}.
\end{proof}
 
% chktex-file 19 chktex-file 35 Chapter Template Main chapter title Change X to
% a consecutive number; for referencing this chapter elsewhere,
% use~\ref{ChapterX}
\chapter{Приложения к алгебрам
    анализа}\label{ChapterApplicationsToAlgebrasOfAnalysis} 

Грубо говоря, среди конкретных банаховых модулей существует три типа: модули с
поточечным умножением, модули с композицией операторов в роли внешнего умножения
и модули со сверткой. Мы исследуем главные примеры модулей этих типов. Следуя
подходу Дэйлса и Полякова из~\cite{DalPolHomolPropGrAlg}, мы систематизируем все
результаты о классических модулях анализа, но в этот раз для метрической и
топологической теории. Мы рассмотрим модули над $C^*$-алгебрами, модули над
алгебрами последовательностей и, наконец, классические модули гармонического
анализа.

%-------------------------------------------------------------------------------
%	Applications to operator algebras
%-------------------------------------------------------------------------------


\section{Приложения к модулям над 
    \texorpdfstring{$C^*$}{C*}-алгебрами}\label{
        SectionApplicationsToCStarAlgebras}

%-------------------------------------------------------------------------------
%	Spatial modules
%-------------------------------------------------------------------------------

\subsection{Пространственные модули}\label{SubSectionSpatialModules}

Мы начнем с простейшего примера модулей над операторными алгебрами. По теореме
Гельфанда-Наймарка [\cite{HelBanLocConvAlg}, теорема 4.7.57] для любой
$C^*$-алгебры $A$ существует гильбертово пространство $H$ и изометрический
${}^*$-гомоморфизм $\varrho:A\to\mathcal{B}(H)$. Для гильбертовых пространств
$H$, для которых существует такой гомоморфизм, мы можем рассмотреть  левый
$A$-модуль $H_\varrho$ с внешним умножением определенным равенством $a\cdot
x=\varrho(a)(x)$. Автоматически мы получаем структуру правого $A$-модуля на
пространстве $H^*$, которое по теореме Рисса изометрически изоморфно $H^{cc}$.
Этот изоморфизм позволяет определить структуру правого $A$-модуля на $H^{cc}$ с
помощью равенства $\overline{x}\cdot a=\overline{\varrho(a^*)(x)}$. Такие модули
мы будем называть пространственными. Их подробное исследование можно найти в
работах
Головина~\cite{GolovibHomolPropHilbModOverNestOpAlg},
~\cite{GolovinSpatProjPropInClOfCSLAlg}.
В дальнейшем, для фиксированных $x_1,x_2\in H$ через $x_1\bigcirc x_2$ мы будем
обозначать одномерный оператор $x_1\bigcirc x_2:H\to H:x\mapsto \langle x,
x_2\rangle x_1$. 

\begin{proposition}\label{SpatModOverCStarAlgProp} Пусть $A$ --- $C^*$-алгебра и
пусть $\varrho:A\to\mathcal{B}(H)$ --- изометрический ${}^*$-гомоморфизм такой,
что его образ содержит подпространство одномерных операторов вида 
$ \{x\bigcirc x_0:x\in H \}$ для некоторого ненулевого вектора $x_0\in H$. 
Тогда левый $A$-модуль $H_\varrho$ --- метрически проективный и плоский, 
а правый $A$-модуль $H_\varrho^{cc}$ --- метрически инъективный.
\end{proposition}
\begin{proof} Не теряя общности, мы можем считать, что $\Vert x_0\Vert=1$.
Рассмотрим линейные операторы 
$\pi:A_+\to H_\varrho:a\oplus_1 z\mapsto \varrho(a)(x_0)+zx_0$ и 
$\sigma:H_\varrho\to A_+:x\mapsto \varrho^{-1}(x\bigcirc x_0)$. 
Прямая проверка показывает, что $\pi$ и $\sigma$ --- сжимающие
$A$-морфизмы причем $\pi\sigma=1_{H_\varrho}$. Следовательно, $H_\varrho$ есть
ретракт $A_+$ в $A-\mathbf{mod}_1$. Из предложений~\ref{UnitalAlgIsMetTopProj}
и~\ref{RetrMetTopProjIsMetTopProj} следует, что $H_\varrho$ --- метрически
проективный $A$-модуль. По предложению~\ref{MetTopProjIsMetTopFlat} он также
метрически плоский. Так как $H_\varrho^{cc}\isom{\mathbf{mod}_1-A}H_\varrho^*$,
предложение~\ref{DualMetTopProjIsMetrInj} дает метрическую инъективность
$H_\varrho^{cc}$.
\end{proof}

В дальнейшем нам понадобится следующее простое следствие предыдущего
предложения.

\begin{proposition}\label{FinDimNHModTopProjFlat} Пусть $H$ --- конечномерное
гильбертово пространство. Тогда $\mathcal{N}(H)$ топологически проективный и
плоский $\mathcal{B}(H)$-модуль.
\end{proposition}
\begin{proof} Из теоремы Шаттена-фон Нойманна [\cite{HelBanLocConvAlg},
предложение 0.3.38] мы знаем, что 
$\mathcal{N}(H)\isom{\mathbf{Ban}_1}H\projtens H^*$. 
Пусть $\varrho=1_{\mathcal{B}(H)}$, тогда можно утверждать чуть больше:
$\mathcal{N}(H)\isom{\mathcal{B}(H)-\mathbf{mod}_1} H_\varrho\projtens H^*$. Так
как $H^*$ конечномерно, то $H^*\isom{\mathbf{Ban}}\ell_1(\mathbb{N}_n)$ для
$n=\dim(H)$ и, как следствие, $\mathcal{N}(H)\isom{\mathcal{B}(H)-\mathbf{mod}}
H_\varrho\projtens\ell_1(\mathbb{N}_n)$. По
предложению~\ref{SpatModOverCStarAlgProp} модуль $H_\varrho$ топологически
проективен, поэтому из следствия~\ref{MetTopProjTensProdWithl1} мы получаем, что
$\mathcal{N}(H)$ топологически проективен как $\mathcal{B}(H)$-модуль.
Утверждение о топологической плоскости следует из
предложения~\ref{MetTopProjIsMetTopFlat}.
\end{proof}

%-------------------------------------------------------------------------------
%	Projective ideals of C^*-algebras
%-------------------------------------------------------------------------------

\subsection{Проективные идеалы 
    \texorpdfstring{$C^*$}{C*}-алгебр}\label{
        SubSectionProjectiveIdealsOfCStarAlgebras}

Изучение гомологически тривиальных идеалов $C^*$-алгебр мы начнем с
проективности, но перед тем, как сформулировать главный результат, нам нужна
подготовительная лемма.

\begin{lemma}\label{ContFuncCalcOnIdealOfCStarAlg} Пусть $I$ --- левый идеал
унитальной $C^*$-алгебры $A$. Пусть $a\in I$ --- самосопряженный элемент, и
пусть $E$ --- действительное подпространство исчезающих в нуле
действительнозначных функций из $C(\operatorname{sp}_A(a))$. Тогда существует
изометрический гомоморфизм $\operatorname{RCont}_a^0:E\to I$ корректно
определенный равенством $\operatorname{RCont}_a^0(f)=a$, где
$f:\operatorname{sp}_A(a)\to\mathbb{C}:t\mapsto t$.
\end{lemma}
\begin{proof} Через $\mathbb{R}_0[t]$ мы обозначим действительное линейное
подпространство в $E$, состоящее из многочленов исчезающих в нуле. Так как $I$
--- левый идеал в $A$ и многочлен $p\in\mathbb{R}_0[t]$ не имеет свободного
члена, то $p(a)\in I$. Следовательно, корректно определен $\mathbb{R}$-линейный
гомоморфизм алгебр $\operatorname{RPol}_a^0:\mathbb{R}_0[t]\to I:p\mapsto p(a)$.
Из непрерывного функционального исчисления для любого многочлена $p$ выполнено
$\Vert p(a)\Vert=\Vert p|_{\operatorname{sp}_A(a)}\Vert_\infty$, поэтому
$\Vert\operatorname{RPol}_a^0(p)\Vert=\Vert
p|_{\operatorname{sp}_A(a)}\Vert_\infty$. Значит, $\operatorname{RPol}_a^0$
изометричен. Так как $\mathbb{R}_0[t]$ плотно в $E$ и $I$ полно, то
$\operatorname{RPol}_a^0$ имеет изометрическое продолжение
$\operatorname{RCont}_a^0:E\to I$, которое является $\mathbb{R}$-линейным
гомоморфизмом. 
\end{proof}

Следующее доказательство основано на идеях Блечера и Каниа. В
[\cite{BleKanFinGenCStarAlgHilbMod}, лемма 2.1] они доказали, что любой
алгебраически конечно порожденный левый идеал $C^*$-алгебры является главным.  

\begin{theorem}\label{LeftIdealOfCStarAlgMetTopProjCharac} Пусть $I$ --- левый
идеал $C^*$-алгебры $A$. Тогда следующие условия эквивалентны:
\begin{enumerate}[label = (\roman*)]
    \item $I=Ap$ для некоторого самосопряженного идемпотента $p\in I$;

    \item $I$ --- метрически проективный $A$-модуль;

    \item $I$ --- топологически проективный $A$-модуль.
\end{enumerate}
\end{theorem}
\begin{proof} $(i) \implies (ii)$ Так как $p$ --- самосопряженный идемпотент,
то $\Vert p\Vert=1$, поэтому из пункта $(i)$
предложения~\ref{UnIdeallIsMetTopProj} следует, что идеал $I$ метрически
проективен как $A$-модуль.

$(ii) \implies (iii)$ Импликация следует из
предложения~\ref{MetProjIsTopProjAndTopProjIsRelProj}.

$(iii) \implies (i)$ По теореме 4.7.79 из~\cite{HelBanLocConvAlg} мы знаем,
что $I$ обладает некоторой правой сжимающей аппроксимативной единицей
${(e_\nu)}_{\nu\in N}$. Так как идеал $I$ имеет правую аппроксимативную единицу,
то он является существенным левым $I$-модулем, и тем более существенным левым
$A$-модулем. По предложению~\ref{NonDegenMetTopProjCharac} морфизм $\pi_I$ имеет
правый обратный $A$-морфизм $\sigma:I\to A\projtens \ell_1(B_I)$. Для каждого
$d\in B_I$ рассмотрим $A$-морфизмы $p_d:A\projtens \ell_1(B_I)\to A:a\projtens
\delta_x\mapsto \delta_x(d)a$ и $\sigma_d=p_d\sigma$. Тогда
$\sigma(x)=\sum_{d\in B_I}\sigma_d(x)\projtens \delta_d$ для всех $x\in I$. Из
отождествления 
$A\projtens\ell_1(B_I)\isom{\mathbf{Ban}_1}\bigoplus_1 \{ A:d\in B_I \}$, 
для всех $x\in I$ мы имеем $\Vert\sigma(x)\Vert=\sum_{d\in B_I}
\Vert\sigma_d(x)\Vert$. Так как $\sigma$ есть правый обратный морфизм к $\pi_I$,
то $x=\pi_I(\sigma(x))=\sum_{d\in B_I}\sigma_d(x)d$ для всех $x\in I$. 

Для всех $x\in I$ мы имеем 
$\Vert\sigma_d(x)\Vert=\Vert\sigma_d(\lim_\nu xe_\nu)\Vert
=\lim_\nu\Vert x\sigma_d(e_\nu)\Vert 
\leq\Vert x\Vert\liminf_\nu\Vert\sigma_d(e_\nu)\Vert$, поэтому 
$\Vert\sigma_d\Vert\leq \liminf_\nu\Vert\sigma_d(e_\nu)\Vert$. 
Тогда для любого множества
$S\in\mathcal{P}_0(B_I)$ выполнено
$$
\sum_{d\in S}\Vert \sigma_d\Vert
\leq \sum_{d\in S}\liminf_\nu\Vert \sigma_d(e_\nu)\Vert
\leq \liminf_\nu\sum_{d\in S}\Vert \sigma_d(e_\nu)\Vert
\leq \liminf_\nu\sum_{d\in B_I}\Vert \sigma_d(e_\nu) \Vert
$$
$$
=\liminf_{\nu}\Vert\sigma(e_\nu)\Vert
\leq \Vert\sigma\Vert\liminf_{\nu}\Vert e_\nu\Vert
\leq \Vert\sigma\Vert.
$$
Так как $S\in \mathcal{P}_0(B_I)$ произвольно, то сумма $\sum_{d\in
B_I}\Vert\sigma_d\Vert$ конечна. Как следствие, сумма $\sum_{d\in
B_I}\Vert\sigma_d\Vert^2$ тоже конечна. 

Теперь будем рассматривать алгебру $A$ как идеал в своей унитизации $A_\#$.
Тогда $I$ также идеал в $A_\#$. Зафиксируем натуральное число $m\in\mathbb{N}$ и
действительное число $\epsilon>0$. Тогда существует множество
$S\in\mathcal{P}_0(B_I)$ такое, что $\sum_{d\in B_I\setminus
S}\Vert\sigma_d\Vert<\epsilon$. Обозначим мощность этого множества через $N$.
Рассмотрим положительный элемент 
$b=\sum_{d\in B_I}\Vert\sigma_d\Vert^2 d^*d\in I$. 
Из леммы~\ref{ContFuncCalcOnIdealOfCStarAlg} мы знаем, что $b^{1/m}\in I$,
поэтому $b^{1/m}=\sum_{d\in B_I}\sigma_d(b^{1/m})d$. Из непрерывного
функционального исчисления следует, что $\Vert
b^{1/m}\Vert=\sup_{t\in\operatorname{sp}_{A_\#}(b)} t^{1/m}\leq\Vert
b\Vert^{1/m}$, тогда $\limsup_{m\to\infty}\Vert b^{1/m}\Vert\leq 1$.
Следовательно, $\Vert b^{1/m}\Vert\leq 2$ для достаточно больших $m$. Положим
$\varsigma_d:=\sigma_d(b^{1/m})$, $u:=\sum_{d\in S}\varsigma_d d$ и
$v:=\sum_{d\in B_I\setminus S}\varsigma_d d$. Тогда 
$$
b^{2/m}={(b^{1/m})}^*b^{1/m}=u^*u+u^*v+v^*u+v^*v.
$$
Ясно, что 
$\varsigma_d^*\varsigma_d\leq \Vert \varsigma_d\Vert^2 e_{A_\#}
\leq \Vert \sigma_d\Vert^2\Vert b^{1/m}\Vert^2 e_{A_\#}
\leq 4\Vert \sigma_d\Vert^2 e_{A_\#}$. Для любых $x,y \in A$ всегда выполнено
$x^*x+y^*y-(x^*y+y^*x)={(x-y)}^*(x-y)\geq 0$, и поэтому 
$$
d^*\varsigma_d^* \varsigma_c c+c^*\varsigma_c^* \varsigma_d d
\leq d^*\varsigma_d^*\varsigma_d d + c^*\varsigma_c^*\varsigma_c c
\leq 4\Vert \sigma_d\Vert^2 d^*d+4\Vert \sigma_c\Vert^2 c^*c
$$
для всех $c,d\in B_I$. Просуммируем эти неравенства по $c\in S$ и $d\in S$,
тогда
$$
\begin{aligned}
\sum_{c\in S}\sum_{d\in S}c^*\varsigma_c^* \varsigma_d d
&=\frac{1}{2}\left(\sum_{c\in S}\sum_{d\in S}d^*\varsigma_d^* \varsigma_c c
+\sum_{c\in S}\sum_{d\in S}c^*\varsigma_c^* \varsigma_d d\right)\\
&\leq\frac{1}{2}\left(4 N\sum_{d\in S} \Vert \sigma_d\Vert^2 d^*d+
4 N\sum_{c\in S} \Vert \sigma_c\Vert^2 c^*c\right)\\
&=4 N\sum_{d\in S} \Vert \sigma_d\Vert^2 d^*d.
\end{aligned}
$$
Следовательно,
$$
u^*u
={\left(\sum_{c\in S}\varsigma_c c\right)}^*
\left(\sum_{d\in S}\varsigma_d d\right)
=\sum_{c\in S}\sum_{d\in S}c^*\varsigma_c^* \varsigma_d d
\leq N\sum_{d\in S} 4\Vert \sigma_d\Vert^2 d^*d
\leq 4N b.
$$
Заметим, что
$$
\Vert u\Vert
\leq \sum_{d\in S}\Vert\varsigma_d\Vert\Vert d\Vert
\leq \sum_{d\in S}2\Vert\sigma_d\Vert
\leq 2\Vert\sigma\Vert,
\qquad
\Vert v\Vert
\leq \sum_{d\in B_I\setminus S}\Vert\varsigma_d\Vert\Vert d\Vert
\leq \sum_{d\in B_I\setminus S}2\Vert\sigma_d\Vert
\leq 2\epsilon;
$$
поэтому $\Vert u^*v+v^*u\Vert\leq 8\Vert\sigma\Vert\epsilon$ и $\Vert
v^*v\Vert\leq 4\epsilon^2$. Так как $u^*v+v^*u$ и $v^*v$ ---  самосопряженные
элементы, то $u^*v+v^*u\leq 8\Vert\sigma\Vert\epsilon e_{A_\#}$ и $v^*v\leq
4\epsilon^2 e_{A_\#}$ Таким образом, для любого $\epsilon>0$ и достаточно
большого $m$ выполнено 
$$
b^{2/m}
=u^*u+u^*v+v^*u+v^*v
\leq 4Nb+\epsilon(8\Vert\sigma\Vert+4\epsilon)e_{A_\#}.
$$

Другими словами, $g_m(b)\geq 0$ для непрерывной функции
$g_m:\mathbb{R}_+\to\mathbb{R}:t\mapsto
4Nt+\epsilon(8\Vert\sigma\Vert+4\epsilon)-t^{2/m}$. Теперь выберем $\epsilon>0$
так, чтобы $M:=\epsilon(8\Vert\sigma\Vert+4\epsilon)<1$. По теореме об
отображении спектра [\cite{HelLectAndExOnFuncAn}, теорема 6.4.2] мы получаем
$g_m(\operatorname{sp}_{A_\#}(b))
=\operatorname{sp}_{A_\#}(g_m(b))\subset\mathbb{R}_+$.
Легко проверить, что $g_m$ имеет только одну точку экстремума
$t_{0,m}={(2Nm)}^{\frac{m}{2-m}}$, где она достигает минимума. Так как
$\lim_{m\to\infty} g_m(t_{0,m})=M-1<0$, $g_m(0)=M>0$ и $\lim_{t\to\infty}
g_m(t)=+\infty$, то для достаточно больших $m$ функция $g_m$ имеет ровно два
корня: $t_{1,m}\in(0,t_{0,m})$ и $t_{2,m}\in(t_{0,m},+\infty)$. Следовательно,
решением неравенства $g_m(t)\geq 0$ будет
$t\in[0,t_{1,m}]\cup[t_{2,m},+\infty)$. Значит,  % chktex 9
$\operatorname{sp}_{A_\#}(b)\subset[0,t_{1,m}]\cup[t_{2,m},+\infty)$  % chktex 9
для всех достаточно больших $m$. Так как $\lim_m t_{0,m}=0$, то так же 
$\lim_m t_{1,m}=0$. Заметим, что $g_m(1)=4N+M-1>0$ для достаточно больших 
$m$, и поэтому $t_{2,m}\leq 1$. Следовательно, 
$\operatorname{sp}_{A_\#}(b)\subset \{0 \}\cup[1,+\infty)$.  % chktex 9

Рассмотрим непрерывную функцию $h:\mathbb{R}_+\to\mathbb{R}:t\mapsto\min(1, t)$.
Тогда по лемме~\ref{ContFuncCalcOnIdealOfCStarAlg} мы получаем идемпотент
$p=h(b)=\operatorname{RCont}_b^0(h)\in I$, такой, что $\Vert
p\Vert=\sup_{t\in\operatorname{sp}_{A_\#}(b)}|h(t)|\leq 1$. Следовательно, $p$
--- самосопряженный идемпотент. Так как $h(t)t=th(t)=t$ для всех $t\in
\operatorname{sp}_{A_\#}(b)$, то $bp=pb=b$. Последнее равенство влечет
$$
0=(e_{A_\#}-p)b(e_{A_\#}-p)
=\sum_{d\in B_I}
{(\Vert\sigma_d\Vert d(e_{A_\#}-p))}^*\Vert\sigma_d\Vert d(e_{A_\#}-p).
$$
Так как правая часть этого равенства неотрицательна, то $d=dp$ для всех 
$d\in B_I$, для которых $\sigma_d\neq 0$. Наконец, для всех $x\in I$ мы получаем
$xp=\sum_{d\in B_I}\sigma_d(x)dp=\sum_{d\in B_I}\sigma_d(x)d=x$, то есть $I=Ap$
для некоторого самосопряженного идемпотента $p\in I$.
\end{proof}

Следует отметить, что в относительной теории нет аналогичного описания
относительной проективности левых идеалов $C^*$-алгебр. Правда, известно, что
для случая сепарабельных $C^*$-алгебр (то есть для $C^*$-алгебр сепарабельных
как банахово пространство) все левые идеалы относительно проективны. В
[\cite{LykProjOfBanAndCStarAlgsOfContFld}, параграф 6] можно найти хороший обзор
последних результатов на эту тему.

\begin{corollary}\label{BiIdealOfCStarAlgMetTopProjCharac} Пусть $I$ ---
двусторонний идеал $C^*$-алгебры $A$. Тогда следующие условия эквивалентны:
\begin{enumerate}[label = (\roman*)]
    \item $I$ унитален;

    \item $I$ метрически проективен как $A$-модуль;

    \item $I$ топологически проективен как $A$-модуль.
\end{enumerate}
\end{corollary}
\begin{proof} Идеал $I$ имеет сжимающую аппроксимативную единицу
[\cite{HelBanLocConvAlg}, теорема 4.7.79]. Следовательно, $I$ имеет правую
единицу тогда и только тогда, когда он унитален. Теперь все эквивалентности
следуют из теоремы~\ref{LeftIdealOfCStarAlgMetTopProjCharac}. 
\end{proof}

\begin{corollary}\label{IdealofCommCStarAlgMetTopProjCharac} Пусть $L$ ---
хаусдорфово локально компактное пространство, и пусть $I$ --- идеал в $C_0(L)$.
Тогда следующие условия эквивалентны:
\begin{enumerate}[label = (\roman*)]
    \item $\operatorname{Spec}(I)$ компактен;

    \item $I$ метрически проективный $C_0(L)$-модуль;

    \item $I$ топологически проективный $C_0(L)$-модуль.
\end{enumerate} 
\end{corollary}
\begin{proof} По теореме Гельфанда-Наймарка
$I\isom{\mathbf{Ban}_1}C_0(\operatorname{Spec}(I))$; следовательно, идеал $I$
полупрост. Отсюда, в силу теоремы Шилова об идемпотентах, идеал $I$ унитален
тогда и только тогда, когда $\operatorname{Spec}(I)$ компактен. Осталось
применить следствие~\ref{BiIdealOfCStarAlgMetTopProjCharac}. 
\end{proof}

Отметим, что класс относительно проективных идеалов в $C_0(L)$ намного шире.
Известно, что идеал $I$ в алгебре $C_0(L)$ относительно проективен тогда и
только тогда, когда $\operatorname{Spec}(I)$ паракомпактен
[\cite{HelHomolBanTopAlg}, глава IV,\S\S 2--3].

%-------------------------------------------------------------------------------
%	Injective ideals of C^*-algebras
%-------------------------------------------------------------------------------

\subsection{Инъективные идеалы 
    \texorpdfstring{$C^*$}{C*}-алгебр}\label{
        SubSectionInjectiveIdealsOfCStarAlgebras}

Перейдем к обсуждению инъективности двусторонних идеалов $C^*$-алгебр. К
сожалению, мы не получим полного их описания, но приведем много примеров и
некоторые необходимые условия. 

Заметим, что двусторонний идеал $I$ в $C^*$-алгебре $A$ сам является
$C^*$-алгеброй с сжимающей аппроксимативной единицей [\cite{HelBanLocConvAlg},
теорема 4.7.79]. Следовательно, $I$ верен как $I$-модуль. Теперь из
предложения~\ref{ReduceInjIdToInjAlg} мы получаем, что $I$ топологически
инъективен как $A$-модуль тогда и только тогда, когда $I$ топологически
инъективен как $I$-модуль. Следовательно, при рассмотрении идеалов мы можем
ограничиться рассмотрением $C^*$-алгебр $\langle$~метрически /
топологически~$\rangle$ инъективных над собой как правые модули.

Нам необходимо напомнить несколько фактов об $AW^*$-алгебрах, так как в этом
параграфе они играют ключевую роль. В попытках дать чисто алгебраическое
определение для $W^*$-алгебр в~\cite{KaplProjInBanAlg} Капланский определил этот
класс $C^*$-алгебр. Алгебра $A$ называется $AW^*$-алгеброй, если это
$C^*$-алгебра, такая, что для любого подмножества $S\subset A$ правый
алгебраический аннулятор $\operatorname{r.ann}_A(S)= \{y\in A: Sy= \{0 \} \}$
имеет вид $pA$ для некоторого самосопряженного идемпотента $p\in A$. Этот класс
содержит все $W^*$-алгебры, но он строго больше. Заметим, что в случае
коммутативных $C^*$-алгебр свойство быть $AW^*$-алгеброй эквивалентно тому, что
$\operatorname{Spec}(A)$ является стоуновым пространством
[\cite{BerbBaerStarRings}, теорема 1.7.1]. Главные результаты об $AW^*$-алгебрах
и более общих бэровских ${}^*$-кольцах можно найти в~\cite{BerbBaerStarRings}. 

Следующее предложение есть комбинация результатов Хаманы и Такесаки.

\begin{proposition}[Хамана, Такесаки]\label{MetInjCStarAlgCharac} $C^*$-алгебра
метрически инъективна как правый модуль над собой тогда и только тогда, когда
она является коммутативной $AW^*$-алгеброй.
\end{proposition}
\begin{proof} Если алгебра $A$ метрически инъективна как $A$-модуль, то по
предложению~\ref{MetTopInjOfId} она имеет левую единицу нормы $1$. Так как $A$
также обладает сжимающей аппроксимативной единицей  [\cite{HelBanLocConvAlg},
теорема 4.7.79], то $A$ унитальна. Теперь из результата Хаманы
[\cite{HamInjEnvBanMod}, предложение 2] алгебра $A$ есть коммутативная
$AW^*$-алгебра. Хотя Хамана доказал этот факт для левых модулей, его
докзательство легко модифицировать и для случая правых модулей.

Обратную импликацию доказал Такесаки в [\cite{TakHanBanThAndJordDecomOfModMap},
теорема 2]. Хотя в работе рассматривались двусторонние модули, рассуждение для
правых модулей точно такое же.
\end{proof}

Осталось рассмотреть топологическую инъективность. Как показывает следующее
предложение, топологически инъективные $C^*$-алгебры, говоря нестрого, не так уж
далеки от коммутативных. Это предложение использует банахово-геометрическое
свойство l.u.st. Его определение можно найти в
параграфе~\ref{
    SubSectionHomologicallyTrivialModulesOverBanachAlgebrasWithSpecificGeometry
}.

\begin{proposition}\label{TopInjIdHaveLUST} Пусть $A$ --- $C^*$-алгебра,
топологически инъективная как $A$-модуль. Тогда $A$ обладает свойством l.u.st и,
как следствие, не может содержать $\mathcal{B}(\ell_2(\mathbb{N}_n))$ как
${}^*$-подалгебру для достаточно большого $n\in\mathbb{N}$.
\end{proposition}
\begin{proof} По теореме Гельфанда-Наймарка [\cite{HelBanLocConvAlg}, теорема
4.7.57] существует гильбертово пространство $H$ и изометрический
${}^*$-гомоморфизм $\varrho:A\to\mathcal{B}(H)$. Обозначим
$\Lambda:=B_{H_\varrho^{cc}}$. Легко проверить, что оператор 
$$
\rho:A\to\bigoplus\nolimits_\infty \{H_\varrho^{cc}:\overline{x}\in \Lambda \}
:a\mapsto \bigoplus\nolimits_\infty \{
    \overline{x}\cdot a:\overline{x}\in \Lambda \}
$$
является изометрическим морфизмом правых $A$-модулей. Так как $A$ топологически
инъективна как $A$-модуль, то $\rho$ имеет правый обратный $A$-морфизм $\tau$.
Следовательно, $A$ дополняемо в $E:=\bigoplus_\infty
\{H_\varrho^{cc}:\overline{x}\in \Lambda \}$ посредством проектора $\rho\tau$.
Заметим, что $H_{\varrho}^{cc}$, как любое гильбертово пространство, является
банаховой решеткой, поэтому $E$ тоже банахова решетка. Как любая банахова
решетка $E$ имеет свойство l.u.st [\cite{DiestAbsSumOps}, теорема 17.1], и,
следовательно, это свойство имеет $A$, так как свойство l.u.st наследуется
дополняемыми подпространствами.

Допустим, $A$ содержит $\mathcal{B}(\ell_2(\mathbb{N}_n))$ как ${}^*$-подалгебру
для произвольного $n\in\mathbb{N}$. На самом деле, такая копия алгебры
$\mathcal{B}(\ell_2(\mathbb{N}_n))$ необходимо $1$-дополняема в $A$
[\cite{LauLoyWillisAmnblOfBanAndCStarAlgsOfLCG}, лемма 2.1]. Следовательно, для
локальных безусловных констант выполнено неравенство
$\kappa_u(\mathcal{B}(\ell_2(\mathbb{N}_n)))\leq \kappa_u(A)$. По теореме 5.1
из~\cite{GorLewAbsSmOpAndLocUncondStrct} мы знаем, что 
$\lim_n \kappa_u(\mathcal{B}(\ell_2(\mathbb{N}_n)))=+\infty$, поэтому
$\kappa_u(A)=+\infty$. Это противоречит тому, что $A$ обладает свойством l.u.st.
Значит, $A$ не может содержать $\mathcal{B}(\ell_2(\mathbb{N}_n))$ как
${}^*$-подалгебру для достаточно большого $n\in\mathbb{N}$.
\end{proof}

Напомним важную в теории $C^*$-алгебр конструкцию матричной алгебры. Для
 заданной $C^*$-алгебры $A$ через $M_n(A)$ мы обозначим линейное пространство
 матриц размера $n\times n$ со значениями в $A$. Это линейное пространство можно
 наделить структурой ${}^*$-алгебры с инволюцией и умножением с помощью равенств
$$
{(ab)}_{i,j}=\sum_{k=1}^n a_{i,k}b_{k,j},
\qquad\qquad
{(a^*)}_{i,j}=(a_{j,i}^*)
$$ 
для всех $a,b\in M_n(A)$ и $i,j\in\mathbb{N}_n$. Существует единственная норма
на $M_n(A)$, которая превращает ее в $C^*$-алгебру
[\cite{MurphyCStarAlgsAndOpTh}, теорема 3.4.2]. Очевидно, $M_n(\mathbb{C})$
изометрически изоморфна как ${}^*$-алгебра алгебре
$\mathcal{B}(\ell_2(\mathbb{N}_n))$. Из [\cite{MurphyCStarAlgsAndOpTh},
замечание 3.4.1] следует, что естественные вложения $i_{k,l}:A\to
M_n(A):a\mapsto{(a\delta_{i,k}\delta_{j,l})}_{i,j\in\mathbb{N}_n}$ и проекции
$\pi_{k,l}:M_n(A)\to A:a\mapsto a_{k,l}$ непрерывны. Следовательно, для
заданного непрерывного оператора $\phi:A\to B$ между $C^*$-алгебрами $A$ и $B$
линейный оператор 
$$
M_n(\phi):M_n(A)\to M_n(B):a\mapsto {(\phi(a_{i,j}))}_{i,j\in\mathbb{N}_n}
$$ 
также непрерывен. Более того, если $\phi$ --- $A$-морфизм, то $M_n(\phi)$ есть
$M_n(A)$-морфизм. Наконец, упомянем два изометрических изоморфизма связанных с
матричными алгебрами:
$$
M_n\left(\bigoplus\nolimits_\infty \{A_\lambda:\lambda\in\Lambda \}\right)
\isom{\mathbf{Ban}_1}
\bigoplus\nolimits_\infty \{M_n\left(A_\lambda\right):\lambda\in\Lambda \},
\qquad
M_n(C(K))\isom{\mathbf{Ban}_1}C(K,M_n(\mathbb{C})).
$$

Как показывает предложение~\ref{TopInjIdHaveLUST}, топологически инъективные над
собой $C^*$-алгебры не могут содержать $\mathcal{B}(\ell_2(\mathbb{N}_n))$ как
${}^*$-подалгебру для достаточно большого $n\in\mathbb{N}$. $C^*$-алгебры не
содержащие $\mathcal{B}(\ell_2(\mathbb{N}_n))$ как ${}^*$-подалгебру для
достаточно большого $n\in\mathbb{N}$ называются субоднородными. Они являются
замкнутыми ${}^*$-подалгебрами [\cite{BlackadarOpAlg}, предложение IV.1.4.3]
алгебр $M_n(C(K))$ для некоторого компактного хаусдорфова пространства $K$ и
некоторого натурального числа $n$. Больше подробностей о субоднородных
$C^*$-алгебрах можно найти в [\cite{BlackadarOpAlg}, параграф IV.1.4]. 

Приведем два важных примера некоммутативных $C^*$-алгебр топологически
инъективных над собой.

\begin{proposition}\label{FinDimBHModTopInj} Пусть $H$ --- конечномерное
гильбертово пространство. Тогда пространство $\mathcal{B}(H)$ --- топологически
инъективный $\mathcal{B}(H)$-модуль. 
\end{proposition}
\begin{proof} Напомним, что
$\mathcal{B}(H)\isom{\mathbf{mod}_1-\mathcal{B}(H)}{\mathcal{N}(H)}^*$. Теперь
результат немедленно следует из предложений~\ref{FinDimNHModTopProjFlat}
и~\ref{DualMetTopProjIsMetrInj}.
\end{proof}

\begin{proposition}\label{CKMatrixModTopInj} Пусть $K$ --- стоуново пространство
и $n\in\mathbb{N}$, тогда пространство $M_n(C(K))$ --- топологически инъективный
$M_n(C(K))$-модуль.
\end{proposition}
\begin{proof} Для заданного $s\in K$ через $\mathbb{C}_s$ мы будем обозначать
правый $C(K)$-модуль $\mathbb{C}$ с внешним умножением определенным равенством
$z\cdot a=a(s)z$ для всех $a\in C(K)$ и $z\in\mathbb{C}_s$. Аналогично, через
$M_n(\mathbb{C}_s)$ мы обозначим правый банахов $M_n(C(K))$-модуль
$M_n(\mathbb{C})$ с внешним умножением определенным равенством
$$
{(x\cdot a)}_{i,j}=\sum_{k=1}^n x_{i,k}a_{k,j}
$$
для каждого $a\in M_n(C(K))$ и $x\in M_n(\mathbb{C}_s)$. Известно, что
$C^*$-алгебра $M_n(C(K))$ ядерна [\cite{BroOzaCStarAlgFinDimApprox}, следствие
2.4.4], тогда из [\cite{HaaNucCStarAlgAmen}, теорема 3.1] следует, что она
относительно аменабельна и даже $1$-аменабельна [\cite{RundeAmenConstFour},
пример 2]. Так как пространство $M_n(\mathbb{C}_s)$ конечномерно, то оно
является $\mathscr{L}_{1,C}$-пространством для некоторой константы $C\geq 1$ не
зависящей от $s$. Тогда по предложению~\ref{MetTopEssL1FlatModAoverAmenBanAlg}
банахов $M_n(C(K))$-модуль ${M_n(\mathbb{C}_s)}^*$ $C$-топологически плоский. Так
как этот модуль существенный, то по предложению~\ref{CTopFlatCharac} правый
$M_n(C(K))$-модуль ${M_n(\mathbb{C}_s)}^{**}$ $C$-топологически инъективен. Так
как ${M_n(\mathbb{C}_s)}^{**}$ изометрически изоморфен $M_n(\mathbb{C}_s)$ как
правый $M_n(C(K))$-модуль, то из предложения~\ref{MetTopInjModProd} мы получаем,
что $\bigoplus_\infty \{M_n(\mathbb{C}_s):s\in K \}$ --- топологически
инъективный $M_n(C(K))$-модуль.

Заметим, что по предложению~\ref{MetInjCStarAlgCharac} банахов $C(K)$-модуль
$C(K)$ метрически инъективен, поэтому изометрический $C(K)$-морфизм
$\widetilde{\rho}:C(K)\to\bigoplus_\infty \{ \mathbb{C}_s:s\in K \}
:x\mapsto \bigoplus_\infty \{x(s):s\in K \}$ 
имеет левый обратный сжимающий $C(K)$-морфизм
$\widetilde{\tau}:\bigoplus_\infty \{ \mathbb{C}_s:s\in K \} \to C(K)$. Теперь
легко проверить, что линеные операторы
$$
\rho:M_n(C(K))\to\bigoplus\nolimits_\infty \{M_n(\mathbb{C}_s):s\in K \}
:x\mapsto \bigoplus\nolimits_\infty \{
    {(x_{i,j}(s))}_{i,j\in\mathbb{N}_n}:s\in K 
 \}
$$
$$
\tau:\bigoplus\nolimits_\infty \{M_n(\mathbb{C}_s):s\in K \}\to M_n(C(K))
:y\mapsto {\left(
    \widetilde{\tau}\left(\bigoplus\nolimits_\infty \{y_{s,i,j}:s\in K \}\right)
    \right)}_{i,j\in\mathbb{N}_n}
$$
являются $M_n(C(K))$-морфизмами, причем $\tau \rho=1_{M_n(C(K))}$.
Следовательно, $M_n(C(K))$ является ретрактом топологически инъективного
$M_n(C(K))$-модуля $\bigoplus_\infty \{M_n(\mathbb{C}_s):s\in K \}$ в
$\mathbf{mod}_1-M_n(C(K))$. Наконец, из
предложения~\ref{RetrMetTopInjIsMetTopInj} мы получаем, что $M_n(C(K))$
топологически инъективен как $M_n(C(K))$-модуль.
\end{proof}

\begin{theorem}\label{TopInjAWStarAlgCharac} Пусть $A$ --- $C^*$-алгебра. Тогда
следующие условия эквивалентны:

\begin{enumerate}[label = (\roman*)]
    \item $A$ --- топологически инъективная как $A$-модуль $AW^*$-алгебра;

    \item $A$ изоморфна как $C^*$-алгебра алгебре $\bigoplus_\infty
    \{M_{n_\lambda}(C(K_\lambda)):\lambda\in\Lambda \}$ для некоторого конечного
    набора стоуновых пространств ${(K_\lambda)}_{\lambda\in\Lambda}$ 
    и натуральных чисел ${(n_\lambda)}_{\lambda\in\Lambda}$.
\end{enumerate}
\end{theorem}
\begin{proof}$(i) \implies (ii)$ Из предложения 6.6
в~\cite{SmithDecompPropCStarAlg} мы знаем, что $AW^*$-алгебра либо изоморфна как
$C^*$-алгебра алгебре $\bigoplus_\infty
\{M_{n_\lambda}(C(K_\lambda)):\lambda\in\Lambda \}$ для некоторого конечного
набора натуральных чисел ${(n_\lambda)}_{\lambda\in\Lambda}$ и стоуновых
пространств ${(K_\lambda)}_{\lambda\in\Lambda}$, либо содержит 
$\bigoplus \infty \{ \mathcal{B}(\ell_2(\mathbb{N}_n)):n\in\mathbb{N} \}$ 
как ${}^*$-подалгебру. Последняя возможность исключается 
предложением~\ref{TopInjIdHaveLUST}.

$(ii) \implies (i)$ Для каждого $\lambda\in\Lambda$ алгебра
$M_{n_\lambda}(C(K_\lambda))$ унитальна, так как $K_\lambda$ компактно.
Следовательно, $M_{n_\lambda}(C(K_\lambda))$ --- верный
$M_{n_\lambda}(C(K_\lambda))$-модуль. По предложению~\ref{CKMatrixModTopInj} это
еще и топологически инъективный $M_{n_\lambda}(C(K_\lambda))$-модуль. Теперь
топологическая инъективность $A$ как $A$-модуля следует из пункта $(ii)$
предложения~\ref{MetTopProjInjFlatUnderSumOfAlg}. Достаточно положить $p=\infty$
и $X_\lambda=A_\lambda=M_{n_\lambda}(C(K_\lambda))$ для всех
$\lambda\in\Lambda$. 

Для всех $\lambda\in\Lambda$ алгебра $C(K_\lambda)$ является $AW^*$-алгеброй
потому, что $K_\lambda$ --- стоуново пространство [\cite{BerbBaerStarRings},
теорема 1.7.1]. Следовательно, $M_{n_\lambda}(C(K_\lambda))$ также является
$AW^*$-алгеброй [\cite{BerbBaerStarRings}, следствие 9.62.1]. Наконец, $A$ есть
$AW^*$-алгебра как $\bigoplus_\infty$-сумма $AW^*$-алгебр
[\cite{BerbBaerStarRings}, предложение 1.10.1].
\end{proof}

Хотелось бы доказать, что топологически инъективные над собой $C^*$-алгебры
являются $AW^*$-алгебрами, но, похоже, это очень сложная задача даже в
коммутативном случае.

%-------------------------------------------------------------------------------
%	Flat ideals of C^*-algebras
%-------------------------------------------------------------------------------

\subsection{Плоские идеалы 
    \texorpdfstring{$C^*$}{C*}-алгебр}\label{
        SubSectionFlatIdealsOfCStarAlgebras}

Рассмотрением метрической и топологической плоскости мы завершим это длинное
изучение идеалов $C^*$-алгебр.

\begin{proposition}\label{IdealofCstarAlgisMetTopFlat} Пусть $I$ --- левый идеал
$C^*$-алгебры $A$. Тогда $I$ --- метрически и топологически плоский $A$-модуль.
\end{proposition}
\begin{proof} По предложению 4.7.78 из~\cite{HelBanLocConvAlg} идеал $I$ имеет
сжимающую аппроксимативную единицу. Остается применить
предложение~\ref{MetTopFlatIdealsInUnitalAlg}.
\end{proof}

Для доказательства следующего предложения нам понадобится определение слабо
секвенциально полного банахова пространства. Будем говорить, что банахово
пространство $E$ слабо секвенциально полно, если для любой последовательности
${(x_n)}_{n\in\mathbb{N}}\subset E$ такой, что последовательность
${(f(x_n))}_{n\in\mathbb{N}}\subset\mathbb{C}$ фундаментальна для 
всех $f\in E^*$, существует вектор $x\in E$ такой, 
что $\lim_n f(x_n)=f(x)$ для всех $f\in E^*$.
Другими словами: всякая последовательность, фундаментальная в слабой топологии,
сходится в слабой топологии. Стандартный пример слабо секвенциально полного
банахова пространства --- это любое $L_1$-пространство
[\cite{WojBanSpForAnalysts}, следствие III.C.14]. Это свойство наследуется
замкнутыми подпространствами. Стандартный пример банахова пространства, не
являющегося слабо секвенциально полным, --- это $c_0(\mathbb{N})$. Чтобы в этом
убедиться, достаточно рассмотреть последовательность 
${(\sum_{k=1}^n\delta_k)}_{n\in\mathbb{N}}$. 

\begin{proposition}\label{CStarAlgIsTopFlatOverItsIdeal} Пусть $I$ ---
собственный двусторонний идеал $C^*$-алгебры $A$. Тогда следующие условия
эквивалентны: 

\begin{enumerate}[label = (\roman*)]
    \item $A$ является $\langle$~метрически / топологически~$\rangle$ плоским
    $I$-модулем;

    \item $\langle$~$\operatorname{dim}(A)=1$, $I= \{0 \}$ / факторалгебра $A/I$
    конечномерна~$\rangle$.
\end{enumerate}
\end{proposition}
\begin{proof} Мы можем рассматривать $I$ как идеал в унитизации $A_\#$ алгебры
$A$. Так как $I$ --- двусторонний идеал, то он имеет сжимающую двустороннюю
аппроксимативную единицу ${(e_\nu)}_{\nu\in N}$ такую, что $0\leq e_\nu\leq
e_{A_\#}$ [\cite{HelBanLocConvAlg}, предложение 4.7.79]. Как следствие,
$\sup_{\nu\in N}\Vert e_{A_\#}-e_\nu\Vert\leq 1$. Снова из-за наличия
аппроксимативной единицы в $I$ мы имеем $A_{ess}:=\operatorname{cl}_A(IA)=I$.

Для начала рассмотрим случай топологической плоскости. По
предложению~\ref{TopFlatModCharac} банахов $I$-модуль $A$ будет топологически
плоским тогда и только тогда, когда $A_{ess}=I$ и $A/A_{ess}=A/I$ будут
топологически плоскими $I$-модулями. По
предложению~\ref{IdealofCstarAlgisMetTopFlat} идеал $I$ топологически плоский
как $I$-модуль. По предложению~\ref{MetTopFlatAnnihModCharac} аннуляторный
$I$-модуль $A/I$ топологически плоский тогда и только тогда, когда он является
$\mathscr{L}_1$-пространством. Мы утверждаем, что модуль $A/I$ есть
$\mathscr{L}_1$-пространство тогда и только тогда, когда он конечномерен.
Допустим, модуль $A/I$ является $\mathscr{L}_1$-пространством, тогда он слабо
секвенциально полон [\cite{BourgNewClOfLpSp}, предложение 1.29]. Так как $I$ ---
двусторонний идеал, то $A/I$ есть $C^*$-алгебра [\cite{HelBanLocConvAlg},
теорема 4.7.81]. По предложению 2 из~\cite{SakWeakCompOpOnOpAlg} каждая слабо
секвенциально полная $C^*$-алгебра конечномерна, поэтому $A/I$ конечномерно.
Обратно, если алгебра $A/I$ конечномерна, то она является
$\mathscr{L}_1$-пространством, как и любое конечномерное банахово пространство.

Перейдем к рассмотрению метрической плоскости. Допустим, $A$ --- метрически
плоский $I$-модуль. Из предложения~\ref{MetFlatIsTopFlatAndTopFlatIsRelFlat}
следует, что $A$ --- топологически плоский $I$-модуль, поэтому из рассуждений
предыдущего абзаца мы знаем, что $A/I$ --- конечномерная $C^*$-алгебра. Как мы
сказали ранее, $\sup_{\nu\in N}\Vert e_{A_\#}-e_\nu\Vert\leq 1$, поэтому из
пункта $(iii)$ предложения~\ref{DualBanModDecomp} следует, что
${(A/A_{ess})}^*={(A/I)}^*$ есть ретракт $A^*$ в $\mathbf{mod}_1-I$. Теперь из
предложений~\ref{MetTopFlatCharac} и~\ref{RetrMetTopInjIsMetTopInj} следует, что
$A/I$ есть метрически плоский $I$-модуль. Так как это аннуляторный $I$-модуль,
то из предложения~\ref{MetTopFlatAnnihModCharac} следует, что $I= \{0 \}$ и
$A/I$ есть $L_1$-пространство. Как мы показали, ранее пространство $A/I$
конечномерно, поэтому $A/I\isom{\mathbf{Ban}_1}\ell_1(\mathbb{N}_n)$ для
$n=\operatorname{dim}(A/I)$. С другой стороны, $A/I$ --- конечномерная
$C^*$-алгебра, поэтому она изометрически изоморфна пространству
$\bigoplus_\infty \{ \mathcal{B}(\ell_2(\mathbb{N}_{n_k})):k\in\mathbb{N}_m \}$
для некоторых натуральных чисел ${(n_k)}_{k\in\mathbb{N}_m}$
[\cite{DavCSatrAlgByExmpl}, теорема III.1.1]. Допустим, что
$\operatorname{dim}(A/I)>1$, тогда $A$ содержит изометрическую копию
$\ell_\infty(\mathbb{N}_2)$. Следовательно, имеется изометрическое вложение
$\ell_\infty(\mathbb{N}_2)$ в $\ell_1(\mathbb{N}_n)$. Это невозможно по теореме
$1$ из~\cite{LyubIsomEmdbFinDimLp}. Следовательно, $\operatorname{dim}(A/I)=1$.
Так как $I= \{0 \}$, то $\operatorname{dim}(A)=1$. Обратно, если $I= \{0 \}$ и
$\operatorname{dim}(A)=1$, то мы имеем аннуляторный $I$-модуль $A$ который
изометрически изоморфен $\ell_1(\mathbb{N}_1)$. По
предложению~\ref{MetTopFlatAnnihModCharac} он является метрически плоским. 
\end{proof}

%-------------------------------------------------------------------------------
%	\mathcal{K}(H)- and \mathcal{B}(H)-modules
%-------------------------------------------------------------------------------

\subsection{\texorpdfstring{$\mathcal{K}(H)$}{K (H)}- и
    \texorpdfstring{$\mathcal{B}(H)$}{B (H)}-модули}\label{
        SubSectionKHAndBHModules}

В этом параграфе мы применим результаты об идеалах $C^*$-алгебр к изучению
классических модулей над алгеброй компактных и алгеброй ограниченных операторов
на гильбертовом пространстве. Для произвольного гильбертова пространства $H$
рассмотрим $\mathcal{B}(H)$, $\mathcal{K}(H)$ и $\mathcal{N}(H)$ как левые и
правые банаховы модули над $\mathcal{B}(H)$ и $\mathcal{K}(H)$. Для всех этих
модулей внешнее умножение --- это композиция операторов. Нам также пригодятся
изоморфизмы Шаттена-фон Нойманна
$\mathcal{N}(H)\isom{\mathbf{Ban}_1}{\mathcal{K}(H)}^*$,
$\mathcal{B}(H)\isom{\mathbf{Ban}_1}{\mathcal{N}(H)}^*$ [\cite{TakThOpAlgVol1},
теоремы II.1.6, II.1.8]. На самом деле, это изоморфизмы левых и правых
$\mathcal{B}(H)$- и, тем более, $\mathcal{K}(H)$-модулей.

\begin{proposition}\label{KHAndBHModBH} Пусть $H$ --- гильбертово пространство.
Тогда:

\begin{enumerate}[label = (\roman*)]
    \item $\mathcal{B}(H)$ метрически и топологически проективный и плоский как
    $\mathcal{B}(H)$-модуль;

    \item $\mathcal{B}(H)$ метрически или топологически проективный 
    или плоский как $\mathcal{K}(H)$-модуль тогда и 
    только тогда, когда $H$ конечномерно;

    \item $\mathcal{B}(H)$ топологически инъективен как $\mathcal{B}(H)$- или
    $\mathcal{K}(H)$-модуль тогда и только тогда, когда $H$ конечномерно;

    \item $\mathcal{B}(H)$ метрически инъективен как $\mathcal{B}(H)$- или
    $\mathcal{K}(H)$-модуль тогда и только тогда, когда $\dim(H)\leq 1$.
\end{enumerate}
\end{proposition}
\begin{proof} $(i)$ Так как $\mathcal{B}(H)$ унитальная алгебра, то по
предложению~\ref{UnitalAlgIsMetTopProj} она метрически и топологически
проективна как $\mathcal{B}(H)$-модуль. Оба результата о плоскости следуют из
предложения~\ref{MetTopProjIsMetTopFlat}.

$(ii)$ Для бесконечномерного $H$ банахово пространство
$\mathcal{B}(H)/\mathcal{K}(H)$ бесконечномерно, поэтому из предложения
~\ref{CStarAlgIsTopFlatOverItsIdeal} следует, что модуль $\mathcal{B}(H)$ не
является ни метрически ни топологически плоским как $\mathcal{K}(H)$-модуль. Оба
утверждения о проективности следуют из предложения~\ref{MetTopProjIsMetTopFlat}.
Если $H$ конечномерно, то $\mathcal{K}(H)=\mathcal{B}(H)$, поэтому утверждение
следует из пункта $(i)$.

$(iii)$ Если $H$ бесконечномерно, то $\mathcal{B}(H)$ содержит
$\mathcal{B}(\ell_2(\mathbb{N}_n))$ как ${}^*$-подалгебру для всех
$n\in\mathbb{N}$. Тогда из предложения~\ref{TopInjIdHaveLUST} следует, что
$\mathcal{B}(H)$ не является топологически инъективным $\mathcal{B}(H)$-модулем.
Все остальное следует из пункта $(i)$
предложения~\ref{MetTopInjUnderChangeOfAlg}. Если $H$ конечномерно, то
$\mathcal{K}(H)=\mathcal{B}(H)$, поэтому утверждение следует из
предложения~\ref{FinDimBHModTopInj}.

$(iv)$ Если $\dim(H)>1$, то $\mathcal{B}(H)$ некоммутативная $C^*$-алгебра. По
предложению~\ref{MetInjCStarAlgCharac} она не будет метрически инъективной как
$\mathcal{B}(H)$-модуль. Теперь из пункта $(i)$
предложения~\ref{MetTopInjUnderChangeOfAlg} мы получаем, что $\mathcal{B}(H)$ не
является метрически инъективным как $\mathcal{K}(H)$-модуль. Если 
$\dim(H)\leq 1$, оба утверждения, очевидно, следуют из
предложения~\ref{MetInjCStarAlgCharac}.
\end{proof}

\begin{proposition}\label{KHAndBHModKH} Пусть $H$ --- гильбертово пространство.
Тогда:

\begin{enumerate}[label = (\roman*)]
    \item $\mathcal{K}(H)$ метрически и топологически плоский 
    как $\mathcal{B}(H)$- или $\mathcal{K}(H)$-модуль;

    \item $\mathcal{K}(H)$ метрически или топологически проективный как
    $\mathcal{B}(H)$- или $\mathcal{K}(H)$-модуль тогда и только 
    тогда, когда $H$ конечномерно;

    \item $\mathcal{K}(H)$ топологически инъективный как $\mathcal{B}(H)$- или
    $\mathcal{K}(H)$-модуль тогда и только тогда, когда $H$ конечномерно;

    \item $\mathcal{K}(H)$ метрически инъективный как $\mathcal{B}(H)$- или
    $\mathcal{K}(H)$-модуль тогда и только тогда, когда $\dim(H)\leq 1$.
\end{enumerate}
\end{proposition}
\begin{proof} Через $A$ мы обозначим одну из алгебр $\mathcal{B}(H)$ или
$\mathcal{K}(H)$. Отметим, что $\mathcal{K}(H)$ --- это двусторонний идеал в
$A$. 

$(i)$ Напомним, что $\mathcal{K}(H)$ имеет сжимающую аппроксимативную единицу
состоящую из конечномерных проекторов на все конечномерные подпространства в
$H$. Так как $\mathcal{K}(H)$ --- это двусторонний идеал в $A$, то утверждение
следует из предложения~\ref{IdealofCstarAlgisMetTopFlat}.

$(ii)$, $(iii)$, $(iv)$ Если $H$ бесконечномерно, то $\mathcal{K}(H)$ 
не является унитальной банаховой алгеброй. Из
следствия~\ref{BiIdealOfCStarAlgMetTopProjCharac} и
предложения~\ref{MetTopInjOfId} банахов $A$-модуль $\mathcal{K}(H)$ не является
ни метрически ни топологически проективным или инъективным. Если $H$
конечномерно, то $\mathcal{K}(H)=\mathcal{B}(H)$, поэтому оба результата следуют
из пунктов $(i)$, $(iii)$ и $(iv)$ предложения~\ref{KHAndBHModBH}.
\end{proof}

\begin{proposition}\label{KHAndBHModNH} Пусть $H$ --- гильбертово пространство.
Тогда:

\begin{enumerate}[label = (\roman*)]
    \item $\mathcal{N}(H)$ метрически и топологически инъективен как
    $\mathcal{B}(H)$- или $\mathcal{K}(H)$-модуль;

    \item $\mathcal{N}(H)$ топологически проективный или плоский 
    $\mathcal{B}(H)$- или $\mathcal{K}(H)$-модуль тогда и 
    только тогда, когда $H$ конечномерно;

    \item $\mathcal{N}(H)$ метрически проективный или плоский 
    как $\mathcal{B}(H)$- или $\mathcal{K}(H)$-модуль тогда и 
    только тогда, когда $\dim(H)\leq 1$.
\end{enumerate}
\end{proposition}
\begin{proof} Через $A$ мы обозначим одну из алгебр $\mathcal{B}(H)$ или
$\mathcal{K}(H)$.

$(i)$ Заметим, что $\mathcal{N}(H)\isom{\mathbf{mod}_1-A}{\mathcal{K}(H)}^*$,
поэтому утверждение следует из предложения~\ref{MetTopFlatCharac} и пункта $(i)$
предложения~\ref{KHAndBHModKH}.

$(ii)$ Допустим, $H$ бесконечномерно. Так как
$\mathcal{B}(H)\isom{\mathbf{mod}_1-A}{\mathcal{N}(H)}^*$, то из
предложения~\ref{DualMetTopProjIsMetrInj} и пункта $(iii)$
предложения~\ref{KHAndBHModBH} мы получаем, что $\mathcal{N}(H)$ не является
топологически проективным как $A$-модуль. Оба результата о плоскости следуют из
предложения~\ref{MetTopProjIsMetTopFlat}. Если $H$ конечномерно, то результат
следует из предложения~\ref{FinDimNHModTopProjFlat}.

$(iii)$ Допустим, что $\dim(H)>1$, тогда из пункта $(iv)$
предложения~\ref{KHAndBHModBH} банахов $A$-модуль $\mathcal{B}(H)$ не является
метрически инъективным. Так как
$\mathcal{B}(H)\isom{\mathbf{mod}_1-A}{\mathcal{N}(H)}^*$, то из
предложения~\ref{MetTopFlatCharac} мы получаем, что $\mathcal{N}(H)$ не является
метрическим плоским $A$-модулем. По предложению~\ref{MetTopProjIsMetTopFlat}, он
не будет метрически проективным $A$-модулем. Если $\dim(H)\leq 1$, тогда
$\mathcal{N}(H)=\mathcal{K}(H)=\mathcal{B}(H)$, поэтому утверждение следует из
пункта $(i)$ предложения~\ref{KHAndBHModBH}.
\end{proof}

\begin{proposition}\label{KHAndBHModsRelTh} Пусть $H$ --- гильбертово
пространство. Тогда:

\begin{enumerate}[label = (\roman*)]
    \item как $\mathcal{K}(H)$-модуль $\mathcal{N}(H)$ является относительно
    проективным, инъективным и плоским, $\mathcal{K}(H)$ является относительно
    проективным и плоским, но относительно инъективным только для конечномерного
    $H$, $\mathcal{B}(H)$ является относительно инъективным и плоским, но
    относительно проективным только для конечномерного $H$;

    \item как $\mathcal{B}(H)$-модуль $\mathcal{N}(H)$ является относительно
    проективным, инъективным и плоским, $\mathcal{K}(H)$ является относительно
    проективным и плоским, $\mathcal{B}(H)$ является относительно проективным,
    инъективным и плоским.
\end{enumerate}
\end{proposition}
\begin{proof} $(i)$ Заметим, что $H$ есть относительно проективный
$\mathcal{K}(H)$-модуль [\cite{HelBanLocConvAlg}, теорема 7.1.27], поэтому из
предложения 7.1.13 в~\cite{HelBanLocConvAlg} мы получаем, что
$\mathcal{N}(H)\isom{\mathcal{K}(H)-\mathbf{mod}_1}H\projtens H^*$ относительно
проективен как $\mathcal{K}(H)$-модуль. По теореме IV.2.16
из~\cite{HelHomolBanTopAlg} банахов $\mathcal{K}(H)$-модуль $\mathcal{K}(H)$
относительно проективен. Тем более $\mathcal{N}(H)$ и $\mathcal{K}(H)$
относительно плоские $\mathcal{K}(H)$-модули [\cite{HelBanLocConvAlg},
предложение 7.1.40], поэтому
$\mathcal{N}(H)\isom{\mathbf{mod}_1-\mathcal{K}(H)}{\mathcal{K}(H)}^*$ и
$\mathcal{B}(H)\isom{\mathbf{mod}_1-\mathcal{K}(H)}{\mathcal{N}(H)}^*$ есть
относительно инъективные $\mathcal{K}(H)$-модули. Из
[\cite{RamsHomPropSemgroupAlg}, предложение 2.2.8  (i)] мы знаем, что банахова
алгебра, относительно инъективная над собой, как правый модуль, обязана иметь
левую единицу. Следовательно, $\mathcal{K}(H)$ не является относительно
инъективным $\mathcal{K}(H)$-модулем для бесконечномерного $H$. Если $H$
конечномерно, то $\mathcal{K}(H)$-модуль $\mathcal{K}(H)$ относительно
инъективен потому, что $\mathcal{K}(H)=\mathcal{B}(H)$, и как мы показали ранее,
$\mathcal{B}(H)$ относительно инъективен как $\mathcal{K}(H)$-модуль. По
следствию 5.5.64 из~\cite{DalBanAlgAutCont} алгебра $\mathcal{K}(H)$
относительно аменабельна, поэтому все ее левые модули относительно плоские
[\cite{HelBanLocConvAlg}, теорема 7.1.60]. В частности, $\mathcal{B}(H)$
относительно плоский $\mathcal{K}(H)$-модуль. Из [\cite{HelHomolBanTopAlg},
упражнение V.2.20] мы знаем, что $\mathcal{B}(H)$ не является относительно
проективным $\mathcal{K}(H)$-модулем для бесконечномерного $H$. Если $H$
конечномерно, то $\mathcal{B}(H)$ относительно проективен как
$\mathcal{K}(H)$-модуль потому, что $\mathcal{B}(H)=\mathcal{K}(H)$, и как мы
показали ранее, $\mathcal{K}(H)$ относительно проективный
$\mathcal{K}(H)$-модуль.

$(ii)$ Из пункта $(i)$ предложения~\ref{KHAndBHModBH} и
предложения~\ref{MetProjIsTopProjAndTopProjIsRelProj} следует, что
$\mathcal{B}(H)$ --- относительно проективный $\mathcal{B}(H)$-модуль. Из
[\cite{RamsHomPropSemgroupAlg}, предложения 2.3.3, 2.3.4] мы знаем, что
$\langle$~существенный относительно проективный / верный относительно
инъективный~$\rangle$ модуль над идеалом банаховой алгебры будет
$\langle$~относительно проективным / относительно инъективным~$\rangle$ над
самой алгеброй. Так как $\mathcal{K}(H)$ и $\mathcal{N}(H)$ --- существенные и
верные $\mathcal{K}(H)$-модули, то из результатов предыдущего пункта мы
получаем, что $\mathcal{N}(H)$ является относительно проективным и плоским, а
$\mathcal{K}(H)$ является относительно проективным  $\mathcal{B}(H)$-модулем.
Теперь, из [\cite{HelBanLocConvAlg}, предложение 7.1.40] следует, что все
вышеупомянутые модули относительно плоские как $\mathcal{B}(H)$-модули. В
частности, 
$\mathcal{B}(H)\isom{\mathbf{mod}_1-\mathcal{B}(H)}{\mathcal{N}(H)}^*$
относительно инъективен как $\mathcal{B}(H)$-модуль.
\end{proof}

Результаты этого параграфа собраны в следующих трех таблицах. Каждая ячейка
таблицы содержит условие, при котором соответствующий модуль имеет
соответствующее свойство, и предложение, в котором это доказано. Мы используем
символ ??? для случаев где ответ нам не известен.  % chktex 26
Из таблиц видно, что для
некоммутативных алгебр гомологическая тривиальность встречается в метрической и
топологической теории очень редко. Проще указать случаи, в которых метрические и
топологические свойства совпадают с относительными: плоскость $\mathcal{K}(H)$
как $\mathcal{B}(H)$- или $\mathcal{K}(H)$-модуля, инъективность
$\mathcal{N}(H)$ как $\mathcal{B}(H)$- или $\mathcal{K}(H)$-модуля,
проективность и плоскость $\mathcal{B}(H)$-модуля $\mathcal{B}(H)$. В остальных
случаях $H$ должно быть хотя бы конечномерно, чтобы эти свойства оказались
эквивалентны в метрической, топологической и относительной теории.

\begin{scriptsize}
    \begin{longtable}{|c|c|c|c|c|c|c|} 
        \multicolumn{7}{c}{
            \mbox{
                Гомологически тривиальные $\mathcal{K}(H)$- и 
                $\mathcal{B}(H)$-модули в метрической теории
            }
        } \\
        \hline & 
            \multicolumn{3}{c|}{
                $\mathcal{K}(H)$-модули
            } & 
            \multicolumn{3}{c|}{
                $\mathcal{B}(H)$-модули
            } \\
            \hline & 
            Проективность & 
            Инъективность & 
            Плоскость & 
            \mbox{Проективность} & 
            Инъективность & 
            Плоскость \\ 
        \hline
            $\mathcal{N}(H)$ & 
            \begin{tabular}{@{}c@{}}
                $\dim(H)\leq 1$ \\
                {\ref{KHAndBHModNH}}
            \end{tabular} & 
            \begin{tabular}{@{}c@{}}
                $H$ любое \\
                {\ref{KHAndBHModNH}}
            \end{tabular} &
            \begin{tabular}{@{}c@{}}
                $\dim(H)\leq 1$ \\
                {\ref{KHAndBHModNH}}
            \end{tabular} & 
            \begin{tabular}{@{}c@{}}
                $\dim(H)\leq 1$ \\
                {\ref{KHAndBHModNH}}
            \end{tabular} & 
            \begin{tabular}{@{}c@{}
                }$H$\mbox{ любое } \\
                {\ref{KHAndBHModNH}}
            \end{tabular} & 
            \begin{tabular}{@{}c@{}}
                $\dim(H)\leq 1$ \\
                {\ref{KHAndBHModNH}}
            \end{tabular} \\
        \hline
            $\mathcal{B}(H)$ & 
            \begin{tabular}{@{}c@{}}
                $\dim(H)<\aleph_0$ \\
                {\ref{KHAndBHModBH}}
            \end{tabular} &
            \begin{tabular}{@{}c@{}}
                $\dim(H)\leq 1$ \\
                {\ref{KHAndBHModBH}}
            \end{tabular} & 
            \begin{tabular}{@{}c@{}}
                $\dim(H)<\aleph_0$ \\
                {\ref{KHAndBHModBH}}
            \end{tabular} & 
            \begin{tabular}{@{}c@{}
                }$H$\mbox{ любое } \\
                {\ref{KHAndBHModBH}}
            \end{tabular} & 
            \begin{tabular}{@{}c@{}}
                $\dim(H)\leq 1$ \\
                {\ref{KHAndBHModBH}}
            \end{tabular} & 
            \begin{tabular}{@{}c@{}}
                $H$\mbox{ любое } \\
                {\ref{KHAndBHModBH}}
            \end{tabular} \\ 
        \hline
            $\mathcal{K}(H)$ &
            \begin{tabular}{@{}c@{}}
                $\dim(H)<\aleph_0$ \\
                {\ref{KHAndBHModKH}}
            \end{tabular} &
            \begin{tabular}{@{}c@{}}
                $\dim(H)\leq 1$ \\
                {\ref{KHAndBHModKH}}
            \end{tabular} & 
            \begin{tabular}{@{}c@{}}
                $H$\mbox{ любое } \\
                {\ref{KHAndBHModKH}}
            \end{tabular} & 
            \begin{tabular}{@{}c@{}}
                $\dim(H)<\aleph_0$ \\
                {\ref{KHAndBHModKH}}
            \end{tabular} &
            \begin{tabular}{@{}c@{}}
                $\dim(H)\leq 1$ \\
                {\ref{KHAndBHModKH}}
            \end{tabular} &
            \begin{tabular}{@{}c@{}}
                $H$\mbox{ любое } \\
                {\ref{KHAndBHModKH}}
            \end{tabular} \\ 
        \hline
            \multicolumn{7}{c}{
                \mbox{
                    Гомологически тривиальные $\mathcal{K}(H)$- и
                    $\mathcal{B}(H)$-модули в топологической теории
                }
            } \\
        \hline & 
            \multicolumn{3}{c|}{
                $\mathcal{K}(H)$-модули
            } & 
            \multicolumn{3}{c|}{
                $\mathcal{B}(H)$-модули
            } \\
        \hline & 
            \mbox{Проективность} & 
            Инъективность & 
            Плоскость & 
            \mbox{Проективность} & 
            Инъективность & 
            Плоскость \\ 
        \hline
            $\mathcal{N}(H)$ & 
            \begin{tabular}{@{}c@{}}
                $\dim(H)<\aleph_0$ \\
                {\ref{KHAndBHModNH}}
            \end{tabular} &
            \begin{tabular}{@{}c@{}}
                $H$ любое \\
                {\ref{KHAndBHModNH}}
            \end{tabular} &
            \begin{tabular}{@{}c@{}}
                $\dim(H)<\aleph_0$ \\
                {\ref{KHAndBHModNH}}
            \end{tabular} &
            \begin{tabular}{@{}c@{}}
                $\dim(H)<\aleph_0$ \\
                {\ref{KHAndBHModNH}}
            \end{tabular} &
            \begin{tabular}{@{}c@{}}
                $H$ любое \\
                {\ref{KHAndBHModNH}}
            \end{tabular} & 
            \begin{tabular}{@{}c@{}}
                $\dim(H)<\aleph_0$ \\
                {\ref{KHAndBHModNH}}
            \end{tabular} \\
        \hline
            $\mathcal{B}(H)$ &
            \begin{tabular}{@{}c@{}}
                $\dim(H)<\aleph_0$ \\
                {\ref{KHAndBHModBH}}
            \end{tabular} &
            \begin{tabular}{@{}c@{}}
                $\dim(H)<\aleph_0$ \\
                {\ref{KHAndBHModBH}}
            \end{tabular} & 
            \begin{tabular}{@{}c@{}}
                $\dim(H)<\aleph_0$ \\
                {\ref{KHAndBHModBH}}
            \end{tabular} & 
            \begin{tabular}{@{}c@{}}
                $H$ любое  \\
                {\ref{KHAndBHModBH}}
            \end{tabular} & 
            \begin{tabular}{@{}c@{}}
                $\dim(H)<\aleph_0$ \\
                {\ref{KHAndBHModBH}}
            \end{tabular} & 
            \begin{tabular}{@{}c@{}}
                $H$ любое  \\
                {\ref{KHAndBHModBH}}
            \end{tabular} \\ 
        \hline
            $\mathcal{K}(H)$ & 
            \begin{tabular}{@{}c@{}}
                $\dim(H)<\aleph_0$ \\
                {\ref{KHAndBHModKH}}
            \end{tabular} &
            \begin{tabular}{@{}c@{}}
                $\dim(H)<\aleph_0$ \\
                {\ref{KHAndBHModKH}}
            \end{tabular} & 
            \begin{tabular}{@{}c@{}}
                $H$ любое \\
                {\ref{KHAndBHModKH}}
            \end{tabular} & 
            \begin{tabular}{@{}c@{}}
                $\dim(H)<\aleph_0$ \\
                {\ref{KHAndBHModKH}}
            \end{tabular} & 
            \begin{tabular}{@{}c@{}}
                $\dim(H)<\aleph_0$ \\
                {\ref{KHAndBHModKH}}
            \end{tabular} &
            \begin{tabular}{@{}c@{}}
                $H$ любое  \\
                {\ref{KHAndBHModKH}}
            \end{tabular} \\ 
        \hline
            \multicolumn{7}{c}{
                \mbox{
                    Гомологически тривиальные $\mathcal{K}(H)$- и
                    $\mathcal{B}(H)$-модули в относительной теории}
            } \\
        \hline & 
            \multicolumn{3}{c|}{
                $\mathcal{K}(H)$-модули
            } &
            \multicolumn{3}{c|}{
                $\mathcal{B}(H)$-модули
            } \\
        \hline & 
            \mbox{Проективность} & 
            Инъективность &
            Плоскость & 
            \mbox{Проективность} & 
            Инъективность & 
            Плоскость \\ 
        \hline
            $\mathcal{N}(H)$ & 
            \begin{tabular}{@{}c@{}}
                $H$ любое \\
                {\ref{KHAndBHModsRelTh}}, (i)
            \end{tabular} & 
            \begin{tabular}{@{}c@{}}
                $H$ любое \\
                {\ref{KHAndBHModsRelTh}}, (i)
            \end{tabular} & 
            \begin{tabular}{@{}c@{}}
                $H$ любое \\
                {\ref{KHAndBHModsRelTh}}, (i)
            \end{tabular} & 
            \begin{tabular}{@{}c@{}}
                $H$ любое \\
                {\ref{KHAndBHModsRelTh}}, (ii)
            \end{tabular} & 
            \begin{tabular}{@{}c@{}}
                $H$ любое \\
                {\ref{KHAndBHModsRelTh}}, (ii)
            \end{tabular} & 
            \begin{tabular}{@{}c@{}}
                $H$ любое \\
                {\ref{KHAndBHModsRelTh}}, (ii)
            \end{tabular} \\
        \hline
            $\mathcal{B}(H)$ & 
            \begin{tabular}{@{}c@{}}
                $\dim(H)<\aleph_0$ \\
                {\ref{KHAndBHModsRelTh}}, (i)
            \end{tabular} & 
            \begin{tabular}{@{}c@{}}
                $H$ любое \\
                {\ref{KHAndBHModsRelTh}}, (i)
            \end{tabular} & 
            \begin{tabular}{@{}c@{}}
                $H$ любое \\
                {\ref{KHAndBHModsRelTh}}, (i)
            \end{tabular} & 
            \begin{tabular}{@{}c@{}}
                $H$ любое \\
                {\ref{KHAndBHModsRelTh}}, (ii)
            \end{tabular} &
            \begin{tabular}{@{}c@{}}
                $H$ любое \\
                {\ref{KHAndBHModsRelTh}}, (ii)
            \end{tabular} & 
            \begin{tabular}{@{}c@{}}
                $H$ любое \\
                {\ref{KHAndBHModsRelTh}}, (ii)
            \end{tabular} \\
        \hline
            $\mathcal{K}(H)$ &
            \begin{tabular}{@{}c@{}}
                $H$ любое \\
                {\ref{KHAndBHModsRelTh}}, (i)
            \end{tabular} &
            \begin{tabular}{@{}c@{}}
                $\dim(H)<\aleph_0$ \\
                {\ref{KHAndBHModsRelTh}}, (i)
            \end{tabular} & 
            \begin{tabular}{@{}c@{}}
                $H$ любое \\
                {\ref{KHAndBHModsRelTh}}, (i)
            \end{tabular} & 
            \begin{tabular}{@{}c@{}}
                $H$ любое \\
                {\ref{KHAndBHModsRelTh}}, (ii)
            \end{tabular} & 
            \begin{tabular}{@{}c@{}} 
                ???  % chktex 26
            \end{tabular} &
            \begin{tabular}{@{}c@{}}
                $H$ любое \\
                {\ref{KHAndBHModsRelTh}}, (ii)
            \end{tabular} \\
        \hline
    \end{longtable}
\end{scriptsize}




%-------------------------------------------------------------------------------
%	c_0(\Lambda)- and l_infty(\Lambda)-modules
%-------------------------------------------------------------------------------

\subsection{\texorpdfstring{$c_0(\Lambda)$}{c0(Lambda)}- и
    \texorpdfstring{$\ell_\infty(\Lambda)$}{lInfty (Lambda)}-модули}\label{
        SubSectionc0AndlInftyModules}

Мы продолжим наше изучение модулей над $C^*$-алгебрами и перейдем к
коммутативным примерам. Для заданного индексного множества $\Lambda$ мы
рассмотрим пространства $c_0(\Lambda)$ и $\ell_p(\Lambda)$ при 
$1\leq p\leq+\infty$ как левые и правые модули над алгебрами $c_0(\Lambda)$ и
$\ell_\infty(\Lambda)$. Для всех этих модулей внешнее умножение --- это
поточечное умножение. Хорошо известно, что
${c_0(\Lambda)}^*\isom{\mathbf{Ban}_1}\ell_1(\Lambda)$ и
${\ell_p(\Lambda)}^*\isom{\mathbf{Ban}_1}\ell_{p^*}(\Lambda)$ для 
$1\leq p<+\infty$. На самом деле, эти изоморфизмы являются изоморфизмами
$\ell_\infty(\Lambda)$- и $c_0(\Lambda)$-модулей. 

Для заданного $\lambda\in\Lambda$ мы определим $\mathbb{C}_\lambda$ как левый
или правый $\ell_\infty(\Lambda)$- или $c_0(\Lambda)$-модуль $\mathbb{C}$ с
внешним умножением определенным равенствами
$$
a\cdot_\lambda z=a(\lambda)z,\qquad z\cdot_\lambda a=a(\lambda) z.
$$

\begin{proposition}\label{OneDimlInftyc0ModMetTopProjIngFlat} Пусть $\Lambda$
--- произвольное множество и $\lambda\in\Lambda$. Тогда $\mathbb{C}_\lambda$
метрически и топологически проективный, инъективный и плоский
$\ell_\infty(\Lambda)$- или $c_0(\Lambda)$-модуль.
\end{proposition}
\begin{proof} Пусть $A$ обозначает одну из алгебр $\ell_\infty(\Lambda)$ или
$c_0(\Lambda)$. Легко проверить, что отображения
$\pi:A_+\to\mathbb{C}_\lambda:a\oplus_1 z\mapsto a(\lambda)+z$ и
$\sigma:\mathbb{C}_\lambda\to A_+:z\mapsto z\delta_\lambda\oplus_1 0$ являются
сжимающими $A$-морфизмами левых $A$-модулей. Так как
$\pi\sigma=1_{\mathbb{C}_\lambda}$, то $\mathbb{C}_\lambda$ есть ретракт $A_+$ в
$A-\mathbf{mod}_1$. Из предложений~\ref{UnitalAlgIsMetTopProj}
и~\ref{RetrMetTopProjIsMetTopProj} следует, что $\mathbb{C}_\lambda$ метрически
и топологически проективен как $A$-модуль и, тем более, метрически и
топологически плоский по предложению~\ref{MetTopProjIsMetTopFlat}. Из
предложения~\ref{DualMetTopProjIsMetrInj} мы знаем, что $\mathbb{C}_\lambda^*$
метрически и топологически инъективен как $A$-модуль. Теперь метрическая и
топологическая инъективность $\mathbb{C}_\lambda$ следует из изоморфизма
$\mathbb{C}_\lambda\isom{\mathbf{mod}_1-A}\mathbb{C}_\lambda^*$.
\end{proof}

\begin{proposition}\label{c0AndlInftyModlIfty} Пусть $\Lambda$ --- произвольное
множество. Тогда:

\begin{enumerate}[label = (\roman*)]
    \item $\ell_\infty(\Lambda)$ метрически и топологически плоский
    $\ell_\infty(\Lambda)$-модуль;

    \item $\ell_\infty(\Lambda)$ метрически или топологически проективный или
    плоский как $c_0(\Lambda)$-модуль тогда и только тогда, 
    когда $\Lambda$ конечно;

    \item $\ell_\infty(\Lambda)$ метрически и топологически инъективен как
    $\ell_\infty(\Lambda)$- и $c_0(\Lambda)$-модуль.
\end{enumerate}
\end{proposition}
\begin{proof} $(i)$ Так как $\ell_\infty(\Lambda)$ --- унитальная алгебра, то она
метрически и топологически проективна как $\ell_\infty(\Lambda)$-модуль по
предложению~\ref{UnitalAlgIsMetTopProj}. Оба результата о плоскости следуют из
предложения~\ref{MetTopProjIsMetTopFlat}.

$(ii)$ Для бесконечного $\Lambda$ банахово пространство
$\ell_\infty(\Lambda)/c_0(\Lambda)$ бесконечномерно, поэтому по
предложению~\ref{CStarAlgIsTopFlatOverItsIdeal} модуль $\ell_\infty(\Lambda)$ не
является ни метрически, ни топологически плоским как $c_0(\Lambda)$-модуль. Оба
утверждения о проективности следуют из предложения~\ref{MetTopProjIsMetTopFlat}.
Если $\Lambda$ конечно, то $c_0(\Lambda)=\ell_\infty(\Lambda)$, поэтому
результат следует из пункта $(i)$.

$(iii)$ Пусть $A$ обозначает одну из алгебр $\ell_\infty(\Lambda)$ или
$c_0(\Lambda)$. Заметим, что
$\ell_\infty(\Lambda)\isom{A-\mathbf{mod}_1}\bigoplus_\infty
\{\mathbb{C}_\lambda:\lambda\in\Lambda \}$, поэтому из
предложений~\ref{OneDimlInftyc0ModMetTopProjIngFlat} и~\ref{MetTopInjModProd}
следует, что $\ell_\infty(\Lambda)$ метрически инъективный $A$-модуль.
Утверждение о топологической инъективности следует из
предложения~\ref{MetInjIsTopInjAndTopInjIsRelInj}.
\end{proof}

\begin{proposition}\label{c0AndlInftyModc0} Пусть $\Lambda$ --- произвольное
множество. Тогда:

\begin{enumerate}[label = (\roman*)]
    \item $c_0(\Lambda)$ метрически и топологически плоский 
    $\ell_\infty(\Lambda)$- и $c_0(\Lambda)$-модуль;

    \item $c_0(\Lambda)$ метрически или топологически проективный
    $\ell_\infty(\Lambda)$- или $c_0(\Lambda)$-модуль тогда и только тогда, 
    когда $\Lambda$ конечно;

    \item $c_0(\Lambda)$ метрически или топологически инъективный
    $\ell_\infty(\Lambda)$- или $c_0(\Lambda)$-модуль тогда и только тогда, 
    когда $\Lambda$ конечно.
\end{enumerate}
\end{proposition}
\begin{proof} Пусть $A$ обозначает одну из алгебр $\ell_\infty(\Lambda)$ или
$c_0(\Lambda)$. Отметим, что $c_0(\Lambda)$ --- двусторонний идеал в $A$. 

$(i)$ Напомним, что $c_0(\Lambda)$ имеет сжимающую аппроксимативную единицу вида
${(\sum_{\lambda\in S}\delta_\lambda)}_{S\in\mathcal{P}_0(\Lambda)}$. Так как
$c_0(\Lambda)$ есть двусторонний идеал в $A$, то результат следует из
предложения~\ref{IdealofCstarAlgisMetTopFlat}.

$(ii)$, $(iii)$ Если $\Lambda$ бесконечно, то $c_0(\Lambda)$ не унитальная
банахова алгебра. Из следствия~\ref{BiIdealOfCStarAlgMetTopProjCharac} и
предложения~\ref{MetTopInjOfId} банахов $A$-модуль $c_0(\Lambda)$ не является ни
метрически ни топологически проективным или инъективным. Если $\Lambda$ конечно,
то $c_0(\Lambda)=\ell_\infty(\Lambda)$, поэтому оба результата следуют из
пунктов $(i)$ и $(iii)$ предложения~\ref{c0AndlInftyModlIfty}.
\end{proof}

\begin{proposition}\label{c0AndlInftyModl1} Пусть $\Lambda$ --- произвольное
множество. Тогда:

\begin{enumerate}[label = (\roman*)]
    \item $\ell_1(\Lambda)$ метрически и топологически инъективный
    $\ell_\infty(\Lambda)$- или $c_0(\Lambda)$-модуль;

    \item$\ell_1(\Lambda)$ метрически и топологически проективный и плоский
    $\ell_\infty(\Lambda)$- или $c_0(\Lambda)$-модуль;
\end{enumerate}
\end{proposition}
\begin{proof} Пусть $A$ обозначает одну из алгебр $\ell_\infty(\Lambda)$ или
$c_0(\Lambda)$.

$(i)$ Заметим, что $\ell_1(\Lambda)\isom{\mathbf{mod}_1-A}{c_0(\Lambda)}^*$,
поэтому утверждение следует из предложения~\ref{MetTopFlatCharac} и пункта $(i)$
предложения~\ref{c0AndlInftyModc0}.

$(ii)$ Так как $\ell_1(\Lambda)\isom{A-\mathbf{mod}_1}\bigoplus_1
\{\mathbb{C}_\lambda:\lambda\in\Lambda \}$, то из
предложений~\ref{OneDimlInftyc0ModMetTopProjIngFlat} и~\ref{MetTopProjModCoprod}
следует, что $\ell_1(\Lambda)$ метрически проективен как $A$-модуль.
Топологическая проективность следует из
предложения~\ref{MetProjIsTopProjAndTopProjIsRelProj}. Утверждение о метрической
и топологической плоскости теперь следуют из
предложения~\ref{MetTopProjIsMetTopFlat}.
\end{proof}

\begin{proposition}\label{c0AndlInftyModlp} Пусть $\Lambda$ --- произвольное
множество и $1<p<+\infty$. Тогда:

\begin{enumerate}[label = (\roman*)]
    \item $\ell_p(\Lambda)$ топологически проективный, инъективный и плоский
    $\ell_\infty(\Lambda)$- или $c_0(\Lambda)$-модуль тогда и только тогда, 
    когда $\Lambda$ конечно;

    \item если $\ell_p(\Lambda)$ метрически проективный, инъективный или плоский
    $\ell_\infty(\Lambda)$- или $c_0(\Lambda)$-модуль, то $\Lambda$ конечно.
\end{enumerate}
\end{proposition}
\begin{proof} Пусть $A$ обозначает одну из алгебр $\ell_\infty(\Lambda)$ или
$c_0(\Lambda)$, тогда $A$ есть $\mathscr{L}_\infty$-пространство. 

$(i)$, $(ii)$ Так как банахово пространство $\ell_p(\Lambda)$ рефлексивно при
$1<p<+\infty$, то из
следствия~\ref{NoInfDimRefMetTopProjInjFlatModOverMthscrL1OrLInfty} следует, что
пространство $\ell_p(\Lambda)$ необходимо конечномерно если оно является
метрически или топологически проективным, инъективным или плоским
$\ell_\infty(\Lambda)$- или $c_0(\Lambda)$-модулем. Это эквивалентно тому, что
$\Lambda$ конечно. Если $\Lambda$ конечно, то
$\ell_p(\Lambda)\isom{A-\mathbf{mod}}\ell_1(\Lambda)$ и
$\ell_p(\Lambda)\isom{\mathbf{mod}-A}\ell_1(\Lambda)$, поэтому топологическая
проективность, инъективность и плоскость следуют из
предложения~\ref{c0AndlInftyModl1}.
\end{proof}

\begin{proposition}\label{c0AndlInftyModsRelTh} Пусть $\Lambda$ --- произвольное
множество. Тогда:

\begin{enumerate}[label = (\roman*)]
    \item как $c_0(\Lambda)$-модули $\ell_p(\Lambda)$ для $1\leq p<+\infty$ и
    $\mathbb{C}_\lambda$ для $\lambda\in\Lambda$ являются относительно
    проективными, инъективными и плоскими, $c_0(\Lambda)$ является относительно
    проективным и плоским, но относительно инъективным только для конечного
    $\Lambda$, $\ell_\infty(\Lambda)$ является относительно инъективным и
    плоским, но относительно проективным только для конечного $\Lambda$;

    \item как $\ell_\infty(\Lambda)$-модули $\ell_p(\Lambda)$ для $1\leq
    p\leq+\infty$ и $\mathbb{C}_\lambda$ для $\lambda\in\Lambda$ являются
    относительно проективными, инъективными и плоскими, $c_0(\Lambda)$ является
    относительно проективным и плоским.
\end{enumerate}
\end{proposition}
\begin{proof} $(i)$ Алгебра $c_0(\Lambda)$ относительно бипроективна
[\cite{HelHomolBanTopAlg}, теорема IV.5.26] и имеет сжимающую аппроксимативную
единицу, поэтому по теореме 7.1.60 из~\cite{HelBanLocConvAlg} все существенные
$c_0(\Lambda)$-модули проективны. Тогда $c_0(\Lambda)$ и $\ell_p(\Lambda)$ при
$1\leq p<+\infty$ являются относительно проективными $c_0(\Lambda)$-модулями.
Тем более, они являются относительно плоскими $c_0(\Lambda)$-модулями
[\cite{HelBanLocConvAlg}, предложение 7.1.40]. Из того же предложения
$c_0(\Lambda)$-модули
$\ell_1(\Lambda)\isom{\mathbf{mod}_1-c_0(\Lambda)}{c_0(\Lambda)}^*$ и
$\ell_{p^*}(\Lambda)\isom{\mathbf{mod}_1-c_0(\Lambda)}{\ell_p(\Lambda)}^*$ для
$1\leq p<+\infty$ относительно инъективны. Из [\cite{RamsHomPropSemgroupAlg},
предложение 2.2.8 (i)] мы знаем, что банахова алгебра относительно инъективная
над собой как правый модуль, обязана иметь левую единицу. Следовательно,
$c_0(\Lambda)$ не является относительно инъективным $c_0(\Lambda)$-модулем для
бесконечного $\Lambda$. Если $\Lambda$ конечно, то $c_0(\Lambda)$-модуль
$c_0(\Lambda)$  относительно инъективен потому, что
$c_0(\Lambda)=\ell_\infty(\Lambda)$, и, как было доказано выше,
$\ell_\infty(\Lambda)$ есть относительно инъективный $c_0(\Lambda)$-модуль. Из
[\cite{HelHomolBanTopAlg}, следствие V.2.16(II)] мы знаем, что
$\ell_\infty(\Lambda)$ не может быть относительно проективным
$c_0(\Lambda)$-модулем для бесконечного $\Lambda$. Если же $\Lambda$ конечно, то
$\ell_\infty(\Lambda)$ относительно проективен как $c_0(\Lambda)$-модуль потому,
что  $\ell_\infty(\Lambda)=c_0(\Lambda)$ и как было показано выше $c_0(\Lambda)$
есть относительно проективный $c_0(\Lambda)$-модуль. Из
предложений~\ref{OneDimlInftyc0ModMetTopProjIngFlat},
~\ref{MetProjIsTopProjAndTopProjIsRelProj},
~\ref{MetInjIsTopInjAndTopInjIsRelInj}
и~\ref{MetFlatIsTopFlatAndTopFlatIsRelFlat} следуют утверждения о модулях
$\mathbb{C}_\lambda$, где $\lambda\in\Lambda$ произвольно.

$(ii)$ Пункт $(i)$ предложения~\ref{c0AndlInftyModlIfty} и
предложение~\ref{MetProjIsTopProjAndTopProjIsRelProj} показывают, что
$\ell_\infty(\Lambda)$ есть относительно проективный
$\ell_\infty(\Lambda)$-модуль. Из [\cite{RamsHomPropSemgroupAlg}, предложения
2.3.3, 2.3.4] мы знаем, что $\langle$~существенный относительно проективный /
верный относительно инъективный~$\rangle$ модуль над идеалом банаховой алгебры
$\langle$~относительно проективен / относительно инъективен~$\rangle$ над самой
алгеброй. Так как $c_0(\Lambda)$ и $\ell_p(\Lambda)$ для $1\leq p<+\infty$ есть
существенные и верные $c_0(\Lambda)$-модули, то из результатов предыдущего
пункта мы получаем, что $\ell_p(\Lambda)$ для $1\leq p<+\infty$ относительно
проективны и инъективны как $\ell_\infty(\Lambda)$-модули. Также мы получаем,
что $c_0(\Lambda)$ относительно проективен как $\ell_\infty(\Lambda)$-модуль.
Таким образом, все эти $\ell_\infty(\Lambda)$-модули также будут относительно
плоскими [\cite{HelBanLocConvAlg}, предложение 7.1.40]. Как следствие,
$\ell_\infty(\Lambda)
\isom{\mathbf{mod}_1-\ell_\infty(\Lambda)}{\ell_1(\Lambda)}^*$
является относительно инъективным $\ell_\infty(\Lambda)$-модулем. Из
предложений~\ref{OneDimlInftyc0ModMetTopProjIngFlat},
~\ref{MetProjIsTopProjAndTopProjIsRelProj},
~\ref{MetInjIsTopInjAndTopInjIsRelInj}
и~\ref{MetFlatIsTopFlatAndTopFlatIsRelFlat} следуют утверждения о модулях
$\mathbb{C}_\lambda$,  где $\lambda\in\Lambda$ произвольно.
\end{proof}

Результаты этого параграфа собраны в следующих трех таблицах. Каждая ячейка
таблицы содержит условие, при котором соответствующий модуль имеет
соответствующее свойство, и предложение, в котором это доказано. Для случая
$\ell_\infty(\Lambda)$- и $c_0(\Lambda)$-модулей $\ell_p(\Lambda)$ при
$1<p<+\infty$ мы не имеем критерия гомологической тривиальности в метрической
теории, только необходимое условие. Чтобы это подчеркнуть, мы используем символ
$\implies$. Сомнительно, что $\ell_p(\Lambda)$ при $1<p<+\infty$ будет
метрически проективным инъективным или плоским при
$\operatorname{Card}(\Lambda)>1$. К сожалению, мы также не знаем является ли
$\ell_\infty(\Lambda)$-модуль $c_0(\Lambda)$ относительно инъективным, поэтому
мы пишем ??? в соответствующей ячейке.  % chktex 26 

\begin{scriptsize}
    \begin{longtable}{|c|c|c|c|c|c|c|} 
        \multicolumn{7}{c}{
            \mbox{
                Гомологическая тривиальность 
                $c_0(\Lambda)$- и $\ell_\infty(\Lambda)$-модулей 
                в метрической теории
            }
        } \\
		\hline & 
            \multicolumn{3}{c|}{
                $c_0(\Lambda)$-модули
            } & 
            \multicolumn{3}{c|}{
                $\ell_\infty(\Lambda)$-модули
            } \\
        \hline & 
            Проективность & 
            Инъективность & 
            Плоскость & 
            Проективность & 
            Инъективность & 
            Плоскость \\ 
        \hline
            $\ell_1(\Lambda)$ & 
            \begin{tabular}{@{}c@{}}
                $\Lambda$ любое \\
                {\ref{c0AndlInftyModl1}}
            \end{tabular} &
            \begin{tabular}{@{}c@{}}
                $\Lambda$ любое \\
                {\ref{c0AndlInftyModl1}}
            \end{tabular} &
            \begin{tabular}{@{}c@{}}
                $\Lambda$ любое \\
                {\ref{c0AndlInftyModl1}}
            \end{tabular} &
            \begin{tabular}{@{}c@{}}
                $\Lambda$ любое \\
                {\ref{c0AndlInftyModl1}}
            \end{tabular} &
            \begin{tabular}{@{}c@{}}
                $\Lambda$ любое \\
                {\ref{c0AndlInftyModl1}}
            \end{tabular} &
            \begin{tabular}{@{}c@{}}
                $\Lambda$ любое \\
                {\ref{c0AndlInftyModl1}}
            \end{tabular} \\
        \hline
            $\ell_p(\Lambda)$ &
            \begin{tabular}{@{}c@{}}
                $\operatorname{Card}(\Lambda)<\aleph_0$ \\
                ($\implies$){\ref{c0AndlInftyModlp}}
            \end{tabular} &
            \begin{tabular}{@{}c@{}}
                $\operatorname{Card}(\Lambda)<\aleph_0$ \\
                ($\implies$){\ref{c0AndlInftyModlp}}
            \end{tabular} &
            \begin{tabular}{@{}c@{}}
                $\operatorname{Card}(\Lambda)<\aleph_0$ \\
                ($\implies$){\ref{c0AndlInftyModlp}}
            \end{tabular} &
            \begin{tabular}{@{}c@{}}
                $\operatorname{Card}(\Lambda)<\aleph_0$ \\
                ($\implies$){\ref{c0AndlInftyModlp}}
            \end{tabular} &
            \begin{tabular}{@{}c@{}}
                $\operatorname{Card}(\Lambda)<\aleph_0$ \\
                ($\implies$){\ref{c0AndlInftyModlp}}
            \end{tabular} &
            \begin{tabular}{@{}c@{}}
                $\operatorname{Card}(\Lambda)<\aleph_0$ \\
                ($\implies$){\ref{c0AndlInftyModlp}}
            \end{tabular} \\
        \hline
            $\ell_\infty(\Lambda)$ &
            \begin{tabular}{@{}c@{}}
                $\operatorname{Card}(\Lambda)<\aleph_0$ \\
                {\ref{c0AndlInftyModlIfty}}
            \end{tabular} &
            \begin{tabular}{@{}c@{}}
                $\Lambda$ любое \\
                {\ref{c0AndlInftyModlIfty}}
            \end{tabular} &
            \begin{tabular}{@{}c@{}}
                $\operatorname{Card}(\Lambda)<\aleph_0$ \\
                {\ref{c0AndlInftyModlIfty}}
            \end{tabular} &
            \begin{tabular}{@{}c@{}}
                $\Lambda$ любое \\
                {\ref{c0AndlInftyModlIfty}}
            \end{tabular} &
            \begin{tabular}{@{}c@{}}
                $\Lambda$ любое \\
                {\ref{c0AndlInftyModlIfty}}
            \end{tabular} &
            \begin{tabular}{@{}c@{}}
                $\Lambda$ любое \\
                {\ref{c0AndlInftyModlIfty}}
            \end{tabular} \\ 
        \hline
            $c_0(\Lambda)$ &
            \begin{tabular}{@{}c@{}}
                $\operatorname{Card}(\Lambda)<\aleph_0$ \\
                {\ref{c0AndlInftyModc0}}
            \end{tabular} &
            \begin{tabular}{@{}c@{}}
                $\operatorname{Card}(\Lambda)< \aleph_0$ \\
                {\ref{c0AndlInftyModc0}}
            \end{tabular} &
            \begin{tabular}{@{}c@{}}
                $\Lambda$ любое \\
                {\ref{c0AndlInftyModc0}}
            \end{tabular} & 
            \begin{tabular}{@{}c@{}}
                $\operatorname{Card}(\Lambda)<\aleph_0$ \\
                {\ref{c0AndlInftyModc0}}
            \end{tabular} &
            \begin{tabular}{@{}c@{}}
                $\operatorname{Card}(\Lambda)< \aleph_0$ \\
                {\ref{c0AndlInftyModc0}}
            \end{tabular} &
            \begin{tabular}{@{}c@{}}
                $\Lambda$ любое\\
                {\ref{c0AndlInftyModc0}}
            \end{tabular} \\ 
        \hline
            $\mathbb{C}_\lambda$ &
            \begin{tabular}{@{}c@{}}
                $\lambda$ любое \\
                {\ref{OneDimlInftyc0ModMetTopProjIngFlat}}
            \end{tabular} &
            \begin{tabular}{@{}c@{}}
                $\lambda$ любое \\
                {\ref{OneDimlInftyc0ModMetTopProjIngFlat}}
            \end{tabular} &
            \begin{tabular}{@{}c@{}}
                $\lambda$ любое \\
                {\ref{OneDimlInftyc0ModMetTopProjIngFlat}}
            \end{tabular} &
            \begin{tabular}{@{}c@{}}
                $\lambda$ любое \\
                {\ref{OneDimlInftyc0ModMetTopProjIngFlat}}
            \end{tabular} &
            \begin{tabular}{@{}c@{}}
                $\lambda$ любое \\
                {\ref{OneDimlInftyc0ModMetTopProjIngFlat}}
            \end{tabular} &
            \begin{tabular}{@{}c@{}}
                $\lambda$ любое \\
                {\ref{OneDimlInftyc0ModMetTopProjIngFlat}}
            \end{tabular} \\
        \hline
            \multicolumn{7}{c}{
                \mbox{
                    Гомологическая тривиальность $c_0(\Lambda)$- и
                    $\ell_\infty(\Lambda)$-модулей в топологической теории
                }
            } \\
        \hline & 
            \multicolumn{3}{c|}{
                $c_0(\Lambda)$-модули
            } & 
            \multicolumn{3}{c|}{
                $\ell_\infty(\Lambda)$-модули
            } \\
        \hline & 
            Проективность & 
            Инъективность &
            Плоскость & 
            Проективность & 
            Инъективность & 
            Плоскость \\ 
        \hline
            $\ell_1(\Lambda)$ & 
            \begin{tabular}{@{}c@{}}
                $\Lambda$ любое \\
                {\ref{c0AndlInftyModl1}}
            \end{tabular} &
            \begin{tabular}{@{}c@{}}
                $\Lambda$ любое  \\
                {\ref{c0AndlInftyModl1}}
            \end{tabular} & 
            \begin{tabular}{@{}c@{}}
                $\Lambda$ любое \\
                {\ref{c0AndlInftyModl1}}
            \end{tabular} &
            \begin{tabular}{@{}c@{}}
                $\Lambda$ любое \\
                {\ref{c0AndlInftyModl1}}
            \end{tabular} & 
            \begin{tabular}{@{}c@{}}
                $\Lambda$ любое \\
                {\ref{c0AndlInftyModl1}}
            \end{tabular} &
            \begin{tabular}{@{}c@{}}
                $\Lambda$ любое \\
                {\ref{c0AndlInftyModl1}}
            \end{tabular} \\
        \hline
            $\ell_p(\Lambda)$ &
            \begin{tabular}{@{}c@{}}
                $\operatorname{Card}(\Lambda)<\aleph_0$ \\
                {\ref{c0AndlInftyModlp}}
            \end{tabular} &
            \begin{tabular}{@{}c@{}}
                $\operatorname{Card}(\Lambda)<\aleph_0$ \\
                {\ref{c0AndlInftyModlp}}
            \end{tabular} &
            \begin{tabular}{@{}c@{}}
                $\operatorname{Card}(\Lambda)<\aleph_0$ \\
                {\ref{c0AndlInftyModlp}}
            \end{tabular} &
            \begin{tabular}{@{}c@{}}
                $\operatorname{Card}(\Lambda)<\aleph_0$ \\
                {\ref{c0AndlInftyModlp}}
            \end{tabular} &
            \begin{tabular}{@{}c@{}}
                $\operatorname{Card}(\Lambda)<\aleph_0$ \\
                {\ref{c0AndlInftyModlp}}
            \end{tabular} &
            \begin{tabular}{@{}c@{}}
                $\operatorname{Card}(\Lambda)<\aleph_0$ \\
                {\ref{c0AndlInftyModlp}}
            \end{tabular} \\
        \hline
            $\ell_\infty(\Lambda)$ &
            \begin{tabular}{@{}c@{}}
                $\operatorname{Card}(\Lambda)<\aleph_0$ \\
                {\ref{c0AndlInftyModlIfty}}
            \end{tabular} &
            \begin{tabular}{@{}c@{}}
                $\Lambda$ любое \\
                {\ref{c0AndlInftyModlIfty}}
            \end{tabular} &
            \begin{tabular}{@{}c@{}}
                $\operatorname{Card}(\Lambda)<\aleph_0$ \\
                {\ref{c0AndlInftyModlIfty}}
            \end{tabular} &
            \begin{tabular}{@{}c@{}}
                $\Lambda$ любое \\
                {\ref{c0AndlInftyModlIfty}}
            \end{tabular} &
            \begin{tabular}{@{}c@{}}
                $\Lambda$ любое \\
                {\ref{c0AndlInftyModlIfty}}
            \end{tabular} &
            \begin{tabular}{@{}c@{}}
                $\Lambda$ любое \\
                {\ref{c0AndlInftyModlIfty}}
            \end{tabular} \\ 
        \hline
            $c_0(\Lambda)$ &
            \begin{tabular}{@{}c@{}}
                $\operatorname{Card}(\Lambda)<\aleph_0$ \\
                {\ref{c0AndlInftyModc0}}
            \end{tabular} &
            \begin{tabular}{@{}c@{}}
                $\operatorname{Card}(\Lambda)<\aleph_0$ \\
                {\ref{c0AndlInftyModc0}}
            \end{tabular} &
            \begin{tabular}{@{}c@{}}
                $\Lambda$ любое  \\
                {\ref{c0AndlInftyModc0}}
            \end{tabular} &
            \begin{tabular}{@{}c@{}}
                $\operatorname{Card}(\Lambda)<\aleph_0$ \\
                {\ref{c0AndlInftyModc0}}
            \end{tabular} &
            \begin{tabular}{@{}c@{}}
                $\operatorname{Card}(\Lambda)<\aleph_0$ \\
                {\ref{c0AndlInftyModc0}}
            \end{tabular} &
            \begin{tabular}{@{}c@{}}
                $\Lambda$ любое \\
                {\ref{c0AndlInftyModc0}}
            \end{tabular} \\ 
        \hline
            $\mathbb{C}_\lambda$ &
            \begin{tabular}{@{}c@{}}
                $\lambda$ любое \\
                {\ref{OneDimlInftyc0ModMetTopProjIngFlat}}
            \end{tabular} &
            \begin{tabular}{@{}c@{}}
                $\lambda$ любое \\
                {\ref{OneDimlInftyc0ModMetTopProjIngFlat}}
            \end{tabular} &
            \begin{tabular}{@{}c@{}}
                $\lambda$ любое \\
                {\ref{OneDimlInftyc0ModMetTopProjIngFlat}}
            \end{tabular} &
            \begin{tabular}{@{}c@{}}
                $\lambda$ любое \\
                {\ref{OneDimlInftyc0ModMetTopProjIngFlat}}
            \end{tabular} &
            \begin{tabular}{@{}c@{}}
                $\lambda$ любое \\
                {\ref{OneDimlInftyc0ModMetTopProjIngFlat}}
            \end{tabular} &
            \begin{tabular}{@{}c@{}}
                $\lambda$ любое \\
                {\ref{OneDimlInftyc0ModMetTopProjIngFlat}}
            \end{tabular} \\
        \hline
    \multicolumn{7}{c}{
        \mbox{Гомологическая тривиальность $c_0(\Lambda)$- и 
            $\ell_\infty(\Lambda)$-модулей в относительной теории
        }
    } \\
        \hline & 
            \multicolumn{3}{c|}{
                $c_0(\Lambda)$-модули
            } & 
            \multicolumn{3}{c|}{
                $\ell_\infty(\Lambda)$-модули
            } \\
        \hline & 
            Проективность & 
            Инъективность & 
            Плоскость & 
            Проективность & 
            Инъективность & 
            Плоскость \\ 
        \hline
            $\ell_1(\Lambda)$ & 
            \begin{tabular}{@{}c@{}}
                $\Lambda$ любое \\
                {\ref{c0AndlInftyModsRelTh}}, (i)
            \end{tabular} &
            \begin{tabular}{@{}c@{}}
                $\Lambda$ любое \\
                {\ref{c0AndlInftyModsRelTh}}, (i)
            \end{tabular} & 
            \begin{tabular}{@{}c@{}}
                $\Lambda$ любое \\
                {\ref{c0AndlInftyModsRelTh}}, (i)
            \end{tabular} &
            \begin{tabular}{@{}c@{}}
                $\Lambda$ любое \\
                {\ref{c0AndlInftyModsRelTh}}, (ii)
            \end{tabular} & 
            \begin{tabular}{@{}c@{}}
                $\Lambda$ любое \\
                {\ref{c0AndlInftyModsRelTh}}, (ii)
            \end{tabular} &
            \begin{tabular}{@{}c@{}}
                $\Lambda$ любое \\
                {\ref{c0AndlInftyModlIfty}}, (ii)
            \end{tabular} \\
        \hline
            $\ell_p(\Lambda)$ & 
            \begin{tabular}{@{}c@{}}
                $\Lambda$ любое \\
                {\ref{c0AndlInftyModsRelTh}}, (i)
            \end{tabular} &
            \begin{tabular}{@{}c@{}}
                $\Lambda$ любое \\
                {\ref{c0AndlInftyModsRelTh}}, (i)
            \end{tabular} & 
            \begin{tabular}{@{}c@{}}
                $\Lambda$ любое \\
                {\ref{c0AndlInftyModsRelTh}}, (i)
            \end{tabular} &
            \begin{tabular}{@{}c@{}}
                $\Lambda$ любое \\
                {\ref{c0AndlInftyModsRelTh}}, (ii)
            \end{tabular} & 
            \begin{tabular}{@{}c@{}}
                $\Lambda$ любое \\
                {\ref{c0AndlInftyModsRelTh}}, (ii)
            \end{tabular} &
            \begin{tabular}{@{}c@{}}
                $\Lambda$ любое \\
                {\ref{c0AndlInftyModlIfty}}, (ii)
            \end{tabular} \\
        \hline
            $\ell_\infty(\Lambda)$ &
            \begin{tabular}{@{}c@{}}
                $\operatorname{Card}(\Lambda)<\aleph_0$ \\
                {\ref{c0AndlInftyModsRelTh}}, (i)
            \end{tabular} &
            \begin{tabular}{@{}c@{}}
                $\Lambda$ любое \\
                {\ref{c0AndlInftyModsRelTh}}, (i)
            \end{tabular} & 
            \begin{tabular}{@{}c@{}}
                $\Lambda$ любое \\
                {\ref{c0AndlInftyModsRelTh}}, (i)
            \end{tabular} &
            \begin{tabular}{@{}c@{}}
                $\Lambda$ любое \\
                {\ref{c0AndlInftyModsRelTh}}, (ii)
            \end{tabular} & 
            \begin{tabular}{@{}c@{}}
                $\Lambda$ любое \\
                {\ref{c0AndlInftyModsRelTh}}, (ii)
            \end{tabular} &
            \begin{tabular}{@{}c@{}}
                $\Lambda$ любое \\
                {\ref{c0AndlInftyModlIfty}}, (ii)
            \end{tabular} \\
        \hline
            $c_0(\Lambda)$ & 
            \begin{tabular}{@{}c@{}}
                $\Lambda$ любое \\
                {\ref{c0AndlInftyModsRelTh}}, (i)
            \end{tabular} &
            \begin{tabular}{@{}c@{}}
                $\operatorname{Card}(\Lambda)<\aleph_0$ \\
                {\ref{c0AndlInftyModsRelTh}}, (i)
            \end{tabular} &
            \begin{tabular}{@{}c@{}}
                $\Lambda$ любое \\
                {\ref{c0AndlInftyModsRelTh}}, (i)
            \end{tabular} & 
            \begin{tabular}{@{}c@{}}
                $\Lambda$ любое \\
                {\ref{c0AndlInftyModsRelTh}}, (ii)
            \end{tabular} &
            \begin{tabular}{@{}c@{}}
                \mbox{ ??? }  % chktex 26
            \end{tabular}  & 
            \begin{tabular}{@{}c@{}}
                $\Lambda$ любое \\
                {\ref{c0AndlInftyModlIfty}}, (ii)
            \end{tabular} \\
        \hline
            $\mathbb{C}_\lambda$ & 
            \begin{tabular}{@{}c@{}}
                $\lambda$ любое \\
                {\ref{c0AndlInftyModsRelTh}}, (i)
            \end{tabular} &
            \begin{tabular}{@{}c@{}}
                $\lambda$ любое \\
                {\ref{c0AndlInftyModsRelTh}}, (i)
            \end{tabular} & 
            \begin{tabular}{@{}c@{}}
                $\lambda$ любое \\
                {\ref{c0AndlInftyModsRelTh}}, (i)
            \end{tabular} &
            \begin{tabular}{@{}c@{}}
                $\lambda$ любое \\
                {\ref{c0AndlInftyModsRelTh}}, (ii)
            \end{tabular} & 
            \begin{tabular}{@{}c@{}}
                $\lambda$ любое \\
                {\ref{c0AndlInftyModsRelTh}}, (ii)
            \end{tabular} &
            \begin{tabular}{@{}c@{}}
                $\lambda$ любое \\
                {\ref{c0AndlInftyModlIfty}}, (ii)
            \end{tabular} \\
        \hline
    \end{longtable}
\end{scriptsize}

Из этих таблиц легко видеть, что для модулей над коммутативными $C^*$-алгебрами,
есть много общего между относительной, метрической и топологической теориями.
Например, $\ell_1(\Lambda)$ --- проективный, инъективный и плоский
$\ell_\infty(\Lambda)$- или $c_0(\Lambda)$-модуль во всех трех теориях. 


%-------------------------------------------------------------------------------
%	Applications to harmonic analysis
%-------------------------------------------------------------------------------

\section{Приложения к гармоническому 
    анализу}\label{SectionApplicationsToHarmonicAnalysis}


%-------------------------------------------------------------------------------
%	Preliminaries on harmonic analysis
%-------------------------------------------------------------------------------

\subsection{Предварительные сведения по гармоническому 
    анализу}\label{SubSectionPreliminariesOnHarmonicAnalysis}

Пусть $G$ --- локально компактная группа. Ее единицу мы будем обозначать через
$e_G$. По хорошо известной теореме Хаара [\cite{HewRossAbstrHarmAnalVol1},
параграф 15.8] существует единственная с точностью до положительной константы
регулярная по Борелю мера $m_G$, которая конечна на всех компактных множествах,
положительна на всех открытых множествах и инвариантна относительно левых
сдвигов, то есть $m_G(sE)=m_G(E)$ для всех $s\in G$ и $E\in Bor(G)$. Здесь,
через $Bor(G)$ мы обозначаем борелевскую $\sigma$-алгебру открытых множеств в
$G$. Мера обладающая вышеперечисленными свойствами, называется левой мерой Хаара
$G$. Если $G$ компактно, то мы предполагаем, что $m_G(G)=1$. Если $G$ бесконечно
и дискретно, то в роли $m_G$ мы берем считающую меру. Для каждого $s\in G$
отображение $m:Bor(G)\to[0,+\infty]:E\mapsto m_G(Es)$ также является левой мерой
Хаара, поэтому из единственности мы получаем, что $m(E)=\Delta(s)m_G(E)$ для
некоторого $\Delta(s)>0$. Функция $\Delta:G\to(0,+\infty)$ называется модулярной
функцией группы $G$. Ясно, что $\Delta(st)=\Delta(s)\Delta(t)$ для всех 
$s,t\in G$. Если модулярная функция тождественно равна единице, 
то соответствующая группа называется модулярной. 
В частности, модулярными являются компактные,
коммутативные и дискретные группы. В дальнейшем мы будем использовать
обозначение $L_p(G)$ для $L_p(G,m_G)$ при всех $1\leq p\leq+\infty$. Для
фиксированного $s\in G$ мы определим оператор левого сдвига $L_s:L_1(G)\to
L_1(G):f\mapsto(t\mapsto f(s^{-1}t))$ и оператор правого сдвига $R_s:L_1(G)\to
L_1(G):f\mapsto (t\mapsto f(ts))$. 

Структура группы на $G$ позволяет задать на $L_1(G)$ структуру банаховой
алгебры. Для заданных $f,g\in L_1(G)$ мы определим их свертку по формуле
$$
(f\convol g)(s)=\int_G f(t)g(t^{-1}s)dm_G(t)
=\int_G f(st)g(t^{-1})dm_G(t)
=\int_G f(st^{-1})g(t)\Delta(t^{-1})dm_G(t)
$$
для почти всех $s\in G$. В этом случае $L_1(G)$ с операцией свертки в качестве
умножения становится банаховой алгеброй. Банахова алгебра $L_1(G)$ имеет
сжимающую двустороннюю аппроксимативную единицу состоящую из положительных
непрерывных функций с компактным носителем. Банахова алгебра $L_1(G)$ унитальна
тогда и только тогда, когда $G$ дискретно, и в этом случае $\delta_{e_G}$ есть
единица в $L_1(G)$. Структура группы на $G$ также позволяет сделать банахово
пространство комплексных конечных борелевских мер $M(G)$ банаховой алгеброй.
Свертку двух мер $\mu,\nu\in M(G)$ мы определим по формуле
$$
(\mu\convol \nu)(E)=\int_G\nu(s^{-1}E)d\mu(s)=\int_G\mu(Es^{-1})d\nu(s)
$$
для всех $E\in Bor(G)$. Банахово пространство $M(G)$ вместе с такой сверткой
есть унитальная банахова алгебра. Роль единицы играет  мера Дирака
$\delta_{e_G}$ сосредоточенная в $e_G$. На самом деле, $M(G)$ есть
копроизведение в $L_1(G)-\mathbf{mod}_1$ (но не в $M(G)-\mathbf{mod}_1$) своего
двустороннего идеала $M_a(G)$ мер, абсолютно непрерывных по отношению к $m_G$, и
подалгебры $M_s(G)$, состоящей из мер, сингулярных по отношению к $m_G$.
Напомним, что $M_a(G)\isom{M(G)-\mathbf{mod}_1}L_1(G)$ и $M_s(G)$ есть
аннуляторный $L_1(G)$-модуль. Наконец, $M(G)=M_a(G)$ тогда и только тогда, когда
$G$ дискретна. 

Перейдем к обсуждению классических левых и правых модулей над алгебрами $L_1(G)$
и $M(G)$. Так как $L_1(G)$ можно рассматривать как двусторонний идеал в $M(G)$ с
помощью изометрического $M(G)$-морфизма $i:L_1(G)\to M(G):f\mapsto f m_G$, то
нам будет достаточно задавать структуру $M(G)$-модуля. Для $1\leq p<+\infty$ и
любых $f\in L_p(G)$, $\mu\in M(G)$ положим по определению
$$
(\mu\convol_p f)(s)=\int_G f(t^{-1}s)d\mu(t),
\qquad\qquad
(f\convol_p \mu)(s)=\int_G f(st^{-1}){\Delta(t^{-1})}^{1/p}d\mu(t).
$$
Эти внешние умножения превращают банаховы пространства $L_p(G)$ для 
$1\leq p<+\infty$ в левые и правые $M(G)$-модули. Заметим, что при $p=1$ и 
$\mu\in M_a(G)$ мы получаем обычное определение свертки. 
Для $1<p\leq +\infty$ и любых
$f\in L_p(G)$, $\mu\in M(G)$ положим по определению,
$$
(\mu\cdot_p f)(s)=\int_G {\Delta(t)}^{1/p}f(st)d\mu(t),
\qquad\qquad
(f\cdot_p \mu)(s)=\int_G f(ts)d\mu(t).
$$
Эти внешние умножения также превращают банаховы пространства $L_p(G)$ для
$1<p\leq+\infty$ в левые и правые $M(G)$-модули. Этот специальный выбор внешних
умножений хорошо взаимодействует с двойственностью. Действительно,
${(L_p(G),\convol_p)}^*\isom{\mathbf{mod}_1-M(G)}(L_{p^*}(G),\cdot_{p^*})$ для
всех $1\leq p<+\infty$. Более того, банахово пространство $C_0(G)$ также можно
наделить структурой левого и правого $M(G)$-модуля с помощью внешнего умножения
$\cdot_\infty$. Более того,
${(C_0(G),\cdot_\infty)}^*\isom{M(G)-\mathbf{mod}_1}(M(G),\convol)$ и также
$C_0(G)$ есть замкнутый левый и правый $M(G)$-подмодуль в $L_\infty(G)$.

Характером локально компактно группы $G$ называется непрерывный гомоморфизм из
$G$ в $\mathbb{T}$. Множество характеров группы $G$ есть группа. Следуя
Понтрягину, мы будем ее обозначать $\widehat{G}$. Она становится локально
компактной группой, если на ней рассмотреть компактно-открытую топологию. Для
любого характера $\gamma\in\widehat{G}$ можно определить непрерывный характер
$\varkappa_\gamma^L:L_1(G)\to\mathbb{C}:f\mapsto \int_G
f(s)\overline{\gamma(s)}dm_G(s)$ на $L_1(G)$. Все характеры на $L_1(G)$ устроены
таким образом. Этот результат доказан Гельфандом [\cite{KaniBanAlg}, теоремы
2.7.2, 2.7.5]. Аналогично, для каждого $\gamma\in\widehat{G}$ 
можно таким же образом 
$\varkappa_\gamma^M:M(G)\to\mathbb{C}
:\mu\mapsto\int_{G} \overline{\gamma(s)}d\mu(s)$ определить характер $M(G)$. 
Через $\mathbb{C}_\gamma$ мы будем обозначать так называемый 
аугментационный левый и правый $L_1(G)$- или $M(G)$-модуль. 
Его внешние умножения определяются по формуле
$$
f\cdot_{\gamma}z=z\cdot_{\gamma}f=\varkappa_\gamma^L(f)z,
\qquad\qquad
\mu\cdot_{\gamma}z=z\cdot_{\gamma}\mu=\varkappa_\gamma^M(\mu)z
$$
для всех $f\in L_1(G)$, $\mu\in M(G)$ и $z\in\mathbb{C}$. 

Одно из многочисленных эквивалентных определений аменабельной группы говорит,
что локально компактная группа $G$ аменабельна, если существует $L_1(G)$-морфизм
правых модулей $M:L_\infty(G)\to\mathbb{C}_{e_{\widehat{G}}}$ такой, что
$M(\chi_G)=1$ [\cite{HelBanLocConvAlg}, параграф VII.2.5]. Можно даже
предполагать, что $M$ --- сжимающий [\cite{HelBanLocConvAlg}, замечание 7.1.54].

Большинство результатов перечисленных в этом параграфе, но не снабженных
ссылкой, можно найти в [\cite{DalBanAlgAutCont}, параграф 3.3].

%-------------------------------------------------------------------------------
%	L_1(G)-modules
%-------------------------------------------------------------------------------

\subsection{
    \texorpdfstring{$L_1(G)$}{L1(G)}-модули}\label{SubSectionL1GModules}

В метрической теории гомологически тривиальные $L_1(G)$-модули гармонического
анализа были изучены в~\cite{GravInjProjBanMod}. Мы используем идеи этой работы,
чтобы объединить подходы к изучению гомологически тривиальных модулей в
метрической и топологической теории. 

\begin{proposition}\label{LInfIsL1MetrInj} Пусть $G$ --- локально компактная
группа. Тогда $L_1(G)$ метрически и топологически плоский $L_1(G)$-модуль, то
есть $L_1(G)$-модуль $L_\infty(G)$ метрически и топологически инъективен.
\end{proposition} 
\begin{proof} Так как $L_1(G)$ имеет сжимающую аппроксимативную единицу, то
$L_1(G)$ метрически и топологически плоский $L_1(G)$-модуль по
предложению~\ref{MetTopFlatIdealsInUnitalAlg}. Так как
$L_\infty(G)\isom{\mathbf{mod}_1-L_1(G)}{L_1(G)}^*$, то по
предложению~\ref{MetTopFlatCharac} этот $L_1(G)$-модуль метрически и
топологически инъективен.
\end{proof}

\begin{proposition}\label{OneDimL1ModMetTopProjCharac} Пусть $G$ --- локально
компактная группа, и $\gamma\in\widehat{G}$. Тогда следующие условия
эквивалентны: 

\begin{enumerate}[label = (\roman*)]
    \item $G$ компактна;

    \item $\mathbb{C}_\gamma$ метрически проективный $L_1(G)$-модуль;

    \item $\mathbb{C}_\gamma$ топологически проективный $L_1(G)$-модуль.
\end{enumerate}
\end{proposition}
\begin{proof} $(i) \implies (ii)$ Рассмотрим $L_1(G)$-морфизмы
$\sigma^+:\mathbb{C}_\gamma\to {L_1(G)}_+:z\mapsto z\gamma \oplus_1 0$ и
$\pi^+:{L_1(G)}_+\to\mathbb{C}_\gamma: f\oplus_1 w\to f\cdot_{\gamma}1+w$. Легко
проверить, что $\Vert\pi^+\Vert=\Vert\sigma^+\Vert=1$ и
$\pi^+\sigma^+=1_{\mathbb{C}_\gamma}$. Следовательно, $\mathbb{C}_\gamma$ есть
ретракт ${L_1(G)}_+$ в $L_1(G)-\mathbf{mod}_1$. Из
предложений~\ref{UnitalAlgIsMetTopProj} и~\ref{RetrMetTopProjIsMetTopProj}
следует, что $\mathbb{C}_\gamma$ метрически проективен.

$(ii) \implies (iii)$ Импликация следует из
предложения~\ref{MetProjIsTopProjAndTopProjIsRelProj}.

$(iii) \implies (i)$ Рассмотрим $L_1(G)$-морфизм
$\pi:L_1(G)\to\mathbb{C}_\gamma:f\mapsto f\cdot_{\gamma} 1$. Легко видеть, что
$\pi$ строго коизометричен. Так как $\mathbb{C}_\gamma$ топологически
проективен, то существует $L_1(G)$-морфизм $\sigma:\mathbb{C}_\gamma\to L_1(G)$
такой, что $\pi\sigma=1_{\mathbb{C}_\gamma}$. Пусть $f=\sigma(1)\in L_1(G)$ и
${(e_\nu)}_{\nu\in N}$ --- стандартная аппроксимативная единица $L_1(G)$. Так как
$\sigma$ является $L_1(G)$-морфизмом, то для всех $s,t\in G$ выполнено 
$$
f(s^{-1}t)
=L_s(f)(t)
=\lim_\nu L_s(e_\nu\convol \sigma(1))(t)
=\lim_\nu((\delta_s\convol e_\nu)\convol \sigma(1))(t)
=\lim_\nu\sigma((\delta_s\convol e_\nu)\cdot_{\gamma} 1)(t)
$$
$$
=\lim_\nu\sigma(\varkappa_\gamma^L(\delta_s\convol e_\nu))(t)
=\lim_\nu\varkappa_\gamma^L(\delta_s\convol e_\nu)\sigma(1)(t)
=\lim_\nu(e_\nu\convol\gamma)(s^{-1})f(t)
=\gamma(s^{-1})f(t).
$$
Значит, для функции $g(t):=\gamma(t^{-1})f(t)$ из $L_1(G)$ выполнено
$g(st)=g(t)$ для всех $s,t\in G$. Тогда $g$ является константной функцией в
$L_1(G)$. Последнее возможно тогда и только тогда, когда $G$ компактна.
\end{proof}

\begin{proposition}\label{OneDimL1ModMetTopInjFlatCharac} Пусть $G$ --- локально
компактная группа, и $\gamma\in\widehat{G}$. Тогда следующие условия
эквивалентны: 

\begin{enumerate}[label = (\roman*)]
    \item $G$ аменабельна;

    \item $\mathbb{C}_\gamma$ метрически инъективный $L_1(G)$-модуль;

    \item $\mathbb{C}_\gamma$ топологически инъективный 
    и плоский $L_1(G)$-модуль.

    \item $\mathbb{C}_\gamma$ метрически плоский $L_1(G)$-модуль;

    \item $\mathbb{C}_\gamma$ топологически плоский $L_1(G)$-модуль;
\end{enumerate}

\end{proposition}
\begin{proof} 
$(i) \implies (ii)$ Так как $G$ аменабельна, то существует сжимающий
$L_1(G)$-морфизм $M:L_\infty(G)\to\mathbb{C}_{e_{\widehat{G}}}$ со свойством
$M(\chi_G)=1$. Рассмотрим линейные операторы 
$\rho:\mathbb{C}_\gamma\to L_\infty(G):z\mapsto z\overline{\gamma}$ и
$\tau:L_\infty(G)\to\mathbb{C}_\gamma:f\mapsto M(f\gamma)$. Это
$L_1(G)$-морфизмы  правых $L_1(G)$-модулей. Проверим это для оператора 
$\tau$: для всех $f\in L_\infty(G)$ и $g\in L_1(G)$ выполнено
$$
\tau(f\cdot_\infty g)
=M((f\cdot_\infty g)\gamma)
=M(f\gamma\cdot_\infty g\overline{\gamma})
=M(f\gamma)\cdot_{e_{\widehat{G}}} g\overline{\gamma}
=M(f\gamma)\varkappa_\gamma^L(g)
=\tau(f)\cdot_{\gamma} g.
$$  
Легко проверить, что $\rho$ и $\tau$ сжимающие и
$\tau\rho=1_{\mathbb{C}_\gamma}$. Следовательно, $\mathbb{C}_\gamma$ --- ретракт
$L_\infty(G)$ в $\mathbf{mod}_1-L_1(G)$. Из предложений~\ref{LInfIsL1MetrInj}
и~\ref{RetrMetTopInjIsMetTopInj} следует, что $\mathbb{C}_\gamma$ метрически
инъективен как $L_1(G)$-модуль.

$(ii) \implies (iii)$ Импликация следует из
предложения~\ref{MetInjIsTopInjAndTopInjIsRelInj}.

$(iii) \implies (i)$ Так как $\rho$ есть изометрический $L_1(G)$-морфизм
правых $L_1(G)$-модулей, и $\mathbb{C}_\gamma$ топологически инъективен как
$L_1(G)$-модуль, то $\rho$ является коретракцией в $\mathbf{mod}-L_1(G)$.
Обозначим его левый обратный морфизм через $\pi$, тогда
$\pi(\overline{\gamma})=\pi(\rho(1))=1$. Рассмотрим ограниченный линейный
функционал $M:L_\infty(G)\to\mathbb{C}_\gamma:f\mapsto \pi(f\overline{\gamma})$.
Для всех $f\in L_\infty(G)$ и $g\in L_1(G)$ выполнено
$$
M(f\cdot_\infty g)
=\pi((f\cdot_\infty g)\overline{\gamma})
=\pi(f\overline{\gamma}\cdot_\infty g\gamma)
=\pi(f\overline{\gamma})\cdot_{\gamma} g\gamma
=M(f)\varkappa_\gamma^L(g\gamma)
=M(f)\cdot_{e_{\widehat{G}}}g.
$$
Следовательно, $M$ является $L_1(G)$-морфизмом, причем
$M(\chi_G)=\pi(\overline{\gamma})=1$. Значит, $G$ аменабельна.

$(ii) \Longleftrightarrow (iv)$, $(iii) \Longleftrightarrow (v)$ Заметим, что
$\mathbb{C}_\gamma^*\isom{\mathbf{mod}_1-L_1(G)}\mathbb{C}_\gamma$, поэтому все
эквивалентности следуют из трех предыдущих пунктов и
предложения~\ref{MetTopFlatCharac}.
\end{proof}

В следующем предложении мы рассмотрим некоторый специфический тип идеалов
алгебры $L_1(G)$. Они имеют вид $L_1(G)\convol \mu$ для некоторой идемпотентной
меры $\mu$. На самом деле, этот тип идеалов, в случае коммутативной компактной
группы $G$, совпадает с классом левых идеалов в $L_1(G)$, имеющих правую
ограниченную аппроксимативную единицу [\cite{KaniBanAlg}, следствие 5.6.2].

\begin{proposition}\label{CommIdealByIdemMeasL1MetTopProjCharac} Пусть $G$ ---
локально компактная группа, и $\mu\in M(G)$ --- идемпотентная мера, то есть
$\mu\convol\mu=\mu$. Если левый идеал $I=L_1(G)\convol\mu$ банаховой алгебры
$L_1(G)$ топологически проективен как $L_1(G)$-модуль, то $\mu=p m_G$, для
некоторого $p\in I$.
\end{proposition}
\begin{proof} Пусть $\phi:I\to L_1(G)$ --- произвольный морфизм левых
$L_1(G)$-модулей. Рассмотрим $L_1(G)$-морфизм $\phi':L_1(G)\to
L_1(G):x\mapsto\phi(x\convol\mu)$. По теореме Венделя [\cite{WendLeftCentrzrs},
теорема 1], существует мера $\nu\in M(G)$ такая, что $\phi'(x)=x\convol\nu$ для
всех $x\in L_1(G)$. В частности,
$\phi(x)=\phi(x\convol\mu)=\phi'(x)=x\convol\nu$ для всех $x\in I$. Теперь ясно,
что оператор $\psi:I\to I:x\mapsto\nu\convol x$ является морфизмом правых
$I$-модулей, причем $\phi(x)y=x\psi(y)$ для всех $x,y\in I$. Теперь мы видим,
что выполнено условие $(**)$ леммы~\ref{GoodIdealMetTopProjIsUnital}, поэтому
$I$ имеет правую единицу, назовем ее $e\in I$. Тогда
$x\convol\mu=x\convol\mu\convol e$ для всех $x\in L_1(G)$. Две меры равны, если
их свертки со всеми функциями из $L_1(G)$ совпадают [\cite{DalBanAlgAutCont},
следствие 3.3.24], поэтому $\mu=\mu\convol e m_G$. Так как 
$e\in I\subset L_1(G)$, то $\mu=\mu\convol e m_G\in M_a(G)$. 
Положим $p=\mu\convol e\in I$, тогда $\mu=p m_G$.
\end{proof}

Мы предполагаем, что левый идеал $L_1(G)\convol \mu$ для идемпотентной меры
$\mu$ метрически проективен как $L_1(G)$-модуль тогда и только тогда, когда
$\mu=p m_G$ для $p\in I$ нормы $1$.

\begin{theorem}\label{L1ModL1MetTopProjCharac} Пусть $G$ --- локально компактная
группа. Тогда следующие условия эквивалентны:

\begin{enumerate}[label = (\roman*)]
    \item $G$ дискретно;

    \item $L_1(G)$ метрически проективный $L_1(G)$-модуль;

    \item $L_1(G)$ топологически проективный $L_1(G)$-модуль.
\end{enumerate}
\end{theorem}
\begin{proof} $(i) \implies (ii)$ Если $G$ дискретно, то $L_1(G)$ ---
унитальная банахова алгебра с единицей нормы $1$. Из
предложения~\ref{UnIdeallIsMetTopProj} следует, что $L_1(G)$ метрически
проективен как $L_1(G)$-модуль.

$(ii) \implies (iii)$ Импликация следует из предложения
~\ref{MetProjIsTopProjAndTopProjIsRelProj}.

$(iii) \implies (i)$ Очевидно, что $\delta_{e_G}$ --- идемпотентная мера. Так
как идеал $L_1(G)=L_1(G)\convol \delta_{e_G}$ топологически проективен как
$L_1(G)$-модуль, то из предложения~\ref{CommIdealByIdemMeasL1MetTopProjCharac}
мы получаем, что $\delta_{e_G}=f m_G$ для некоторой функции $f\in L_1(G)$. Это
возможно только если $G$ дискретна.
\end{proof}

Стоит отметить, что $L_1(G)$-модуль $L_1(G)$ относительно проективен для любой
локально компактной группы $G$ [\cite{HelBanLocConvAlg}, упражнение 7.1.17].

\begin{proposition}\label{L1MetTopProjAndMetrFlatOfMeasAlg} Пусть $G$ ---
локально компактная группа. Тогда следующие свойства эквивалентны:

\begin{enumerate}[label = (\roman*)]
    \item $G$ дискретна;

    \item $M(G)$ метрически проективный $L_1(G)$-модуль;

    \item $M(G)$ топологически проективный $L_1(G)$-модуль;

    \item $M(G)$ метрически плоский $L_1(G)$-модуль.
\end{enumerate}
\end{proposition}
\begin{proof} 
$(i) \implies (ii)$ Так как $M(G)\isom{L_1(G)-\mathbf{mod}_1}L_1(G)$ для
дискретной группы $G$, то утверждение следует из
теоремы~\ref{L1ModL1MetTopProjCharac}. 

$(ii) \implies (iii)$ Импликация следует из
предложения~\ref{MetProjIsTopProjAndTopProjIsRelProj}.

$(ii) \implies (iv)$ Импликация следует из
предложения~\ref{MetTopProjIsMetTopFlat}.

$(iii) \implies (i)$ Напомним, что $M(G)\isom{L_1(G)-\mathbf{mod}_1}
L_1(G)\bigoplus_1 M_s(G)$, поэтому по предложению~\ref{MetTopProjModCoprod}
банахов $L_1(G)$-модуль $M_s(G)$ топологически проективен. Так как $M_s(G)$ есть
аннуляторный $L_1(G)$-модуль, то из предложения~\ref{MetTopProjOfAnnihModCharac}
мы получаем, что $L_1(G)$ имеет правую единицу. Так как $L_1(G)$ также имеет
двустороннюю ограниченную аппроксимативную единицу, то $L_1(G)$ унитальна.
Последнее возможно только если $G$ дискретно.

$(iv) \implies (i)$ Так как 
$M(G)\isom{L_1(G)-\mathbf{mod}_1} L_1(G)\bigoplus_1 M_s(G)$, 
то из предложения~\ref{MetTopFlatModCoProd} банахов $L_1(G)$-модуль
$M_s(G)$ является метрически плоским. Так как $M_s(G)$ есть аннуляторный
$L_1(G)$-модуль, то из предложения~\ref{MetTopFlatAnnihModCharac} следует, что
$M_s(G)= \{0 \}$. Последнее означает, что $G$ дискретно.
\end{proof}

\begin{proposition}\label{MeasAlgIsL1TopFlat} Пусть $G$ --- локально компактна
группа. Тогда $M(G)$ топологически плоский $L_1(G)$-модуль.
\end{proposition}
\begin{proof} Так как $M(G)$ есть $L_1$-пространство, то, тем более, это
$\mathscr{L}_1$-пространство. Так как $M_s(G)$ дополняемо в $M(G)$, то $M_s(G)$
есть $\mathscr{L}_1$-пространство [\cite{BourgNewClOfLpSp}, предложение 1.28].
Так как $M_s(G)$ --- аннуляторный $L_1(G)$-модуль, то из
предложения~\ref{MetTopFlatAnnihModCharac} мы имеем, что $L_1(G)$-модуль
$M_s(G)$ топологически плоский. Напомним, что по
предложению~\ref{LInfIsL1MetrInj} банахов $L_1(G)$-модуль $L_1(G)$ является
топологически плоским. Так как
$M(G)\isom{L_1(G)-\mathbf{mod}_1}L_1(G)\bigoplus_1 M_s(G)$, то из
предложения~\ref{MetTopFlatModCoProd}, мы получаем, что $L_1(G)$-модуль $M(G)$
топологически плоский.
\end{proof}

%-------------------------------------------------------------------------------
%	M(G)-modules
%-------------------------------------------------------------------------------

\subsection{
    \texorpdfstring{$M(G)$}{M (G)}-модули}\label{SubSectionMGModules}

Мы переходим к обсуждению классических $M(G)$-модулей в гармоническом анализе.
Как мы сейчас увидим, большинство результатов можно вывести из результатов об
$L_1(G)$-модулях.

\begin{proposition}\label{MGMetTopProjInjFlatRedToL1} Пусть $G$ --- локально
компактная группа, и пусть $X$ --- $\langle$~существенный / верный /
существенный~$\rangle$ $L_1(G)$-модуль. Тогда:

\begin{enumerate}[label = (\roman*)]
    \item $X$ --- метрически 
    $\langle$~проективный / инъективный / плоский~$\rangle$
    $M(G)$-модуль тогда и только тогда, когда он метрически 
    $\langle$~проективный / инъективный / плоский~$\rangle$ $L_1(G)$-модуль;

    \item $X$ --- топологически 
    $\langle$~проективный / инъективный / плоский~$\rangle$ 
    $M(G)$-модуль тогда и только тогда, когда он топологически
    $\langle$~проективный / инъективный / плоский~$\rangle$ $L_1(G)$-модуль.
\end{enumerate}
\end{proposition}
\begin{proof} Напомним, что $L_1(G)\isom{L_1(G)-\mathbf{mod}_1}M_a(G)$ является
двусторонним $1$-дополняемым идеалом алгебры $M(G)$. Теперь пункты $(i)$ и 
$(ii)$ следуют из предложения $\langle$~\ref{MetTopProjUnderChangeOfAlg}
/~\ref{MetTopInjUnderChangeOfAlg}  /~\ref{MetTopFlatUnderChangeOfAlg}~$\rangle$.
\end{proof} 

Следует напомнить, что $L_1(G)$-модули $C_0(G)$, $L_p(G)$ для $1\leq p<\infty$ и
$\mathbb{C}_\gamma$ для $\gamma\in\widehat{G}$ существенны, и $L_1(G)$-модули
$C_0(G)$, $M(G)$, $L_p(G)$ для $1\leq p\leq \infty$ и $\mathbb{C}_\gamma$ для
$\gamma\in\widehat{G}$ верны. 

\begin{proposition}\label{MGModMGMetTopProjFlatCharac} Пусть $G$ --- локально
компактная группа. Тогда $M(G)$ метрически и топологически проективный
$M(G)$-модуль. Как следствие, он метрически и топологически плоский
$M(G)$-модуль.
\end{proposition} 
\begin{proof} Так как $M(G)$ --- унитальная алгебра, то метрическая и
топологическая проективность следуют из предложения~\ref{UnitalAlgIsMetTopProj}.
Теперь остается применить предложение~\ref{MetTopProjIsMetTopFlat}.
\end{proof}

%-------------------------------------------------------------------------------
%	Banach geometric restrictions
%-------------------------------------------------------------------------------

\subsection{Банахово-геометрические 
    ограничения}\label{SubSectionBanachGeometricRestriction}

В этом параграфе мы покажем, что многие модули гармонического анализа не
являются ни метрически, ни топологически проективными, инъективными или плоскими
по причинам ``плохой'' банаховой геометрии. 

\begin{proposition}\label{StdModAreNotRetrOfL1LInf} Пусть $G$ --- бесконечная
локально компактная группа. Тогда:

\begin{enumerate}[label = (\roman*)]
    \item $L_1(G)$, $C_0(G)$, $M(G)$, ${L_\infty(G)}^*$ не являются 
    топологически инъективными банаховыми пространствами;

    \item $C_0(G)$, $L_\infty(G)$ не дополняемы ни в одном $L_1$-пространстве.
\end{enumerate}
\end{proposition}
\begin{proof}
Так как $G$ бесконечно, то все рассматриваемые модули бесконечномерны.

$(i)$ Если бесконечномерное банахово пространство топологически инъективно,  то
оно содержит копию $\ell_\infty(\mathbb{N})$ [\cite{RosOnRelDisjFamOfMeas},
следствие 1.1.4], и, следовательно, копию $c_0(\mathbb{N})$. Банахово
пространство $L_1(G)$ слабо секвенциально полно [\cite{WojBanSpForAnalysts},
следствие III.C.14], поэтому по следствию 5.2.11 из~\cite{KalAlbTopicsBanSpTh}
оно не может содержать копию $c_0(\mathbb{N})$. Значит, $L_1(G)$ не может быть
топологически инъективным банаховым пространством. Если $M(G)$ топологически
инъективно, то таково же и его дополняемое пространство
$M_a(G)\isom{\mathbf{Ban}_1}L_1(G)$. Из рассуждений выше мы знаем, что это
невозможно, значит $M(G)$ не может быть топологически инъективным банаховым
пространством. Из следствия 3 в~\cite{LauMingComplSubspInLInfOfG} пространство
$C_0(G)$ не дополняемо в $L_\infty(G)$. Значит, и $C_0(G)$ не может быть
топологически инъективным банаховым пространством. Далее, банахово пространство
$L_1(G)$ дополняемо в ${L_\infty(G)}^*\isom{\mathbf{Ban}_1}{L_1(G)}^{**}$
[\cite{DefFloTensNorOpId}, предложение B10]. Значит, если бы банахово
пространство ${L_\infty(G)}^*$ было бы топологически инъективно, 
то таково же было бы и $L_1(G)$. Как мы показали ранее, это невозможно, 
значит, ${L_\infty(G)}^*$ не является топологически инъективным 
банаховым пространством. 

$(ii)$ Если $C_0(G)$ --- ретракт $L_1$-пространства, то
$M(G)\isom{\mathbf{Ban}_1}{C_0(G)}^*$ будет ретрактом $L_\infty$-пространства, 
и, как следствие, будет топологически инъективным банаховым пространством. Это
противоречит пункту $(i)$, поэтому $C_0(G)$ не может быть ретрактом
$L_1$-пространства. Заметим, что $\ell_\infty(\mathbb{N})$ вкладывается в
$L_\infty(G)$, и, как следствие, мы имеем вложение $c_0(\mathbb{N})$ в
$L_\infty(G)$. Если бы $L_\infty(G)$ было бы ретрактом $L_1$-пространства, то
нашлось бы $L_1$-пространство содержащее копию $c_0(\mathbb{N})$. Как мы
показали в пункте $(i)$ это невозможно.
\end{proof}

Начиная с этого момента и до конца параграфа, через $A$ мы будем обозначать одну
из алгебр $L_1(G)$ или $M(G)$. Напомним, что $L_1(G)$ и $M(G)$ являются
$L_1$-пространствами.

\begin{proposition}\label{StdModAreNotL1MGMetTopProj} Пусть $G$ --- бесконечная
локально компактная группа. Тогда:

\begin{enumerate}[label = (\roman*)]
    \item $C_0(G)$, $L_\infty(G)$ не являются ни метрически, ни топологически
    проективными $A$-модулями;

    \item $L_1(G)$, $C_0(G)$, $M(G)$, ${L_\infty(G)}^*$ не являются 
    ни метрически, ни топологически проективными $A$-модулями;

    \item $L_\infty(G)$, $C_0(G)$ не являются ни метрически, ни топологически
    плоскими $A$-модулями.
\end{enumerate}
\end{proposition}
\begin{proof} $(i)$ Утверждение следует из пункта $(i)$
предложения~\ref{TopProjInjFlatModOverL1Charac} и предложения
~\ref{StdModAreNotRetrOfL1LInf}.

$(ii)$ Утверждение следует из пункта $(ii)$
предложения~\ref{TopProjInjFlatModOverL1Charac} и
предложения~\ref{StdModAreNotRetrOfL1LInf}.

$(iii)$ Напомним, что ${C_0(G)}^*\isom{\mathbf{mod}_1-A}M(G)$. 
Значит утверждение следует из пункта $(i)$ и предложения~\ref{MetTopFlatCharac}.
\end{proof}

Осталось рассмотреть гомологическую тривиальность $A$-модулей в метрической и
топологической теории для конечной группы $G$.

\begin{proposition}\label{LpFinGrL1MGMetrInjProjCharac} Пусть $G$ ---
нетривиальная конечная группа и пусть $1\leq p\leq \infty$. Тогда $A$-модуль
$L_p(G)$ метрически $\langle$~проективен / инъективен~$\rangle$ тогда и только
тогда, когда $\langle$~$p=1$ / $p=\infty$~$\rangle$
\end{proposition}
\begin{proof} Допустим, $L_p(G)$ метрически $\langle$~проективен /
инъективен~$\rangle$ как $A$-модуль. Так как $L_p(G)$ конечномерно, то из
пунктов $(i)$ и $(ii)$ предложения~\ref{TopProjInjFlatModOverL1Charac} мы
получаем, что 
$\langle$~$L_p(G)\isom{\mathbf{Ban}_1}\ell_1(\mathbb{N}_n)$ /
$L_p(G)\isom{\mathbf{Ban}_1}C(\mathbb{N}_n)
\isom{\mathbf{Ban}_1}\ell_\infty(\mathbb{N}_n)$~$\rangle$, 
для $n=\operatorname{Card}(G)>1$. Теперь мы воспользуемся теоремой
$1$ из~\cite{LyubIsomEmdbFinDimLp} для банаховых пространств над полем
$\mathbb{C}$. Она гласит, что если для $2\leq m\leq k$ и $1\leq r,s\leq \infty$
существует изометрическое вложение из $\ell_r(\mathbb{N}_m)$ в
$\ell_s(\mathbb{N}_k)$, то либо $r=2$, $s\in 2\mathbb{N}$ либо $r=s$.
Следовательно, $\langle$~$p=1$ / $p=\infty$~$\rangle$. Обратная импликация легко
следует из $\langle$~теоремы~\ref{L1ModL1MetTopProjCharac} /
предложения~\ref{LInfIsL1MetrInj}~$\rangle$
\end{proof}

\begin{proposition}\label{StdModFinGrL1MGMetrInjProjFlatCharac} Пусть $G$ ---
конечная группа. Тогда:

\begin{enumerate}[label = (\roman*)]
    \item $C_0(G)$, $L_\infty(G)$ метрически инъективны как $A$-модули;

    \item $C_0(G)$, $L_p(G)$ для $1<p\leq\infty$ метрически проективны как
    $A$-модули тогда и только тогда, когда $G$ тривиальна;

    \item $M(G)$, $L_p(G)$ для $1\leq p<\infty$ метрически инъективны как
    $A$-модули тогда и только тогда, когда $G$ тривиальна;

    \item $C_0(G)$, $L_p(G)$ для $1<p\leq\infty$ метрически 
    плоские как $A$-модули тогда и только тогда, когда $G$ тривиальна.
\end{enumerate}
\end{proposition}
\begin{proof}
$(i)$ Так как $G$ конечна,то $C_0(G)=L_\infty(G)$. Теперь утверждение следует из
предложения~\ref{LInfIsL1MetrInj}.

$(ii)$ Если $G$ тривиальна, то есть $G= \{e_G \}$, то $L_p(G)=C_0(G)=L_1(G)$ и
утверждение следует из пункта $(i)$. Если $G$ нетривиальна, то заметим, что
$C_0(G)=L_\infty(G)$ и воспользуемся
предложением~\ref{LpFinGrL1MGMetrInjProjCharac}.

$(iii)$ Если $G= \{e_G \}$, то $M(G)=L_p(G)=L_\infty(G)$ и тогда утверждение
следует из пункта $(i)$. Если $G$ нетривиальна, то заметим, что $M(G)=L_1(G)$ и
воспользуемся предложением~\ref{LpFinGrL1MGMetrInjProjCharac}.

$(iv)$ Из пункта $(iii)$ следует, что $L_p(G)$ для $1\leq p<\infty$ является
метрически инъективным $A$-модулей тогда и только тогда, когда $G$ тривиальна.
Теперь утверждение следует из предложения~\ref{MetTopFlatCharac} и изоморфизмов
${C_0(G)}^*\isom{\mathbf{mod}_1-L_1(G)}M(G)\isom{\mathbf{mod}_1-L_1(G)}L_1(G)$,
${L_p(G)}^*\isom{\mathbf{mod}_1-L_1(G)}L_{p^*}(G)$ при $1\leq p^*<\infty$.
\end{proof}

Здесь следует сказать, что если бы мы рассматривали банаховы пространства над
полем действительных чисел, то $L_\infty(G)$ и $L_1(G)$ были бы, соответственно,
метрически проективны и инъективны, еще в одном случае, когда $G$ состоит из
двух элементов. Причина этого эффекта в том, что
$L_\infty(\mathbb{Z}_2)\isom{L_1(\mathbb{Z}_2)-\mathbf{mod}_1}
\mathbb{R}_{\gamma_0}\bigoplus\nolimits_1\mathbb{R}_{\gamma_1}
$ и
$L_1(\mathbb{Z}_2)\isom{L_1(\mathbb{Z}_2)-\mathbf{mod}_1}
\mathbb{R}_{\gamma_0}\bigoplus\nolimits_\infty\mathbb{R}_{\gamma_1}$
для $\gamma_0,\gamma_1\in\widehat{\mathbb{Z}_2}$ определенных равенствами
$\gamma_0(0)=\gamma_0(1)=\gamma_1(0)=-\gamma_1(1)=1$.

\begin{proposition}\label{StdModFinGrL1MGTopInjProjFlatCharac} Пусть $G$ ---
конечная группа. Тогда $A$-модули $C_0(G)$, $M(G)$, $L_p(G)$ для 
$1\leq p\leq \infty$ являются топологически проективными, 
инъективными и плоскими.
\end{proposition} 
\begin{proof}
Так как $G$ конечна, то $M(G)=L_1(G)$ и $C_0(G)=L_\infty(G)$ и, как следствие,
эти модули не требуют отдельного рассмотрения. Так как $M(G)=L_1(G)$, мы
ограничимся случаем $A=L_1(G)$. Тождественное отображение $i:L_1(G)\to
L_p(G):f\mapsto f$ является топологическим изоморфизмом банаховых пространств
потому, что $L_1(G)$ и $L_p(G)$ для $1\leq p<\infty$ имеют одинаковую конечную
размерность. Так как $G$ конечна, то она унимодулярна. Следовательно, внешние
умножения в модулях $(L_1(G),\convol)$ и $(L_p(G),\convol_p)$ совпадают при
$1\leq p<\infty$. Значит, $i$ --- изоморфизм в $L_1(G)-\mathbf{mod}$ и
$\mathbf{mod}-L_1(G)$. Аналогично можно показать, что
$(L_\infty(G),\cdot_\infty)$ и $(L_p(G),\cdot_p)$ при $1<p\leq\infty$ изоморфны
в $L_1(G)-\mathbf{mod}$ и $\mathbf{mod}-L_1(G)$. Наконец, легко проверить, что
$(L_1(G),\convol)$ и $(L_\infty(G),\cdot_\infty)$ изоморфны в
$L_1(G)-\mathbf{mod}$ и $\mathbf{mod}-L_1(G)$ посредством морфизма 
$j:L_1(G)\to L_\infty(G):f\mapsto(s\mapsto f(s^{-1}))$. 
Таким образом, все рассматриваемые
модули изоморфны. Осталось заметить, что по предложению~\ref{LInfIsL1MetrInj}
банахов $L_1(G)$-модуль $L_1(G)$ топологически проективный и плоский по
теореме~\ref{L1ModL1MetTopProjCharac} и предложению~\ref{LInfIsL1MetrInj}, а
$L_\infty(G)$ топологически инъективный $A$-модуль по
предложению~\ref{LInfIsL1MetrInj}.
\end{proof}

Результаты этого параграфа собраны в первых двух таблицах. В третьей таблице мы
приводим известные результаты из относительной теории. Каждая ячейка таблицы
содержит условие, при котором соответствующий модуль имеет соответствующее
свойство, и предложения в которых это доказано. Формулировки и доказательства
теорем, описывающих гомологически тривиальные модули $\mathbb{C}_\gamma$ в
относительной теории такие же, как и в
предложениях~\ref{OneDimL1ModMetTopProjCharac},
~\ref{OneDimL1ModMetTopInjFlatCharac}
и~\ref{OneDimL1ModMetTopInjFlatCharac}. Как обычно, стрелка $\implies$
обозначает, что известно только необходимое условие. 

\begin{scriptsize}
    \begin{longtable}{|c|c|c|c|c|c|c|} 
    \multicolumn{7}{c}{
        \mbox{
            Гомологически тривиальные $L_1(G)$- 
            и $M(G)$-модули в метрической теории
        }
    } \\
    \hline & 
        \multicolumn{3}{c|}{
            $L_1(G)$-модули
        } & 
        \multicolumn{3}{c|}{
            $M(G)$-модули
        } \\
    \hline & 
        Проективность & 
        Инъективность & 
        Плоскость & 
        Проективность &
        Инъективность & 
        Плоскость \\ 
    \hline
        $L_1(G)$ & 
        \begin{tabular}{@{}c@{}}
            $G$ дискретна \\
            {\ref{L1ModL1MetTopProjCharac}}
        \end{tabular} & 
        \begin{tabular}{@{}c@{}}
            $G= \{e_G \}$ \\
            {\ref{StdModAreNotL1MGMetTopProj}},
            {\ref{StdModFinGrL1MGMetrInjProjFlatCharac}}
        \end{tabular} & 
        \begin{tabular}{@{}c@{}}
            $G$ любая \\
            {\ref{LInfIsL1MetrInj}}
        \end{tabular} &
        \begin{tabular}{@{}c@{}}
            $G$ дискретна \\
            {\ref{L1ModL1MetTopProjCharac}},
            {\ref{MGMetTopProjInjFlatRedToL1}}
        \end{tabular} & 
        \begin{tabular}{@{}c@{}}
            $G= \{e_G \}$ \\
            {\ref{StdModAreNotL1MGMetTopProj}},
            {\ref{StdModFinGrL1MGMetrInjProjFlatCharac}}
        \end{tabular} & 
        \begin{tabular}{@{}c@{}}
            $G$ любая \\
            {\ref{LInfIsL1MetrInj}},
            {\ref{MGMetTopProjInjFlatRedToL1}}
        \end{tabular} \\ 
    \hline 
        $L_p(G)$ & 
        \begin{tabular}{@{}c@{}}
            $G= \{e_G \}$ \\
            {\ref{NoInfDimRefMetTopProjInjFlatModOverMthscrL1OrLInfty}},
            {\ref{LpFinGrL1MGMetrInjProjCharac}}
        \end{tabular} & 
        \begin{tabular}{@{}c@{}}$G= \{e_G \}$ \\
            {\ref{NoInfDimRefMetTopProjInjFlatModOverMthscrL1OrLInfty}},
            {\ref{LpFinGrL1MGMetrInjProjCharac}}
        \end{tabular} & 
        \begin{tabular}{@{}c@{}}
            $G= \{e_G \}$ \\
            {\ref{NoInfDimRefMetTopProjInjFlatModOverMthscrL1OrLInfty}},
            {\ref{StdModFinGrL1MGMetrInjProjFlatCharac}}
        \end{tabular} & 
        \begin{tabular}{@{}c@{}}
            $G= \{e_G \}$ \\
            {\ref{NoInfDimRefMetTopProjInjFlatModOverMthscrL1OrLInfty}},
            {\ref{LpFinGrL1MGMetrInjProjCharac}}
        \end{tabular} & 
        \begin{tabular}{@{}c@{}}
            $G= \{e_G \}$ \\
            {\ref{NoInfDimRefMetTopProjInjFlatModOverMthscrL1OrLInfty}},
            {\ref{LpFinGrL1MGMetrInjProjCharac}}
        \end{tabular} & 
        \begin{tabular}{@{}c@{}}
            $G= \{e_G \}$ \\
            {\ref{NoInfDimRefMetTopProjInjFlatModOverMthscrL1OrLInfty}},
            {\ref{StdModFinGrL1MGMetrInjProjFlatCharac}}
        \end{tabular} \\
    \hline
        $L_\infty(G)$ & 
        \begin{tabular}{@{}c@{}}
            $G= \{e_G \}$ \\
            {\ref{StdModAreNotL1MGMetTopProj}},
            {\ref{LpFinGrL1MGMetrInjProjCharac}}
        \end{tabular} & 
        \begin{tabular}{@{}c@{}}
            $G$ любая \\
            {\ref{LInfIsL1MetrInj}}
        \end{tabular} & 
        \begin{tabular}{@{}c@{}}
            $G= \{e_G \}$ \\
            {\ref{StdModAreNotL1MGMetTopProj}},
            {\ref{StdModFinGrL1MGMetrInjProjFlatCharac}}
        \end{tabular} & 
        \begin{tabular}{@{}c@{}}
            $G= \{e_G \}$ \\
            {\ref{StdModAreNotL1MGMetTopProj}},
            {\ref{LpFinGrL1MGMetrInjProjCharac}}
        \end{tabular} & 
        \begin{tabular}{@{}c@{}}
            $G$ любая \\
            {\ref{LInfIsL1MetrInj}},
            {\ref{MGMetTopProjInjFlatRedToL1}}
        \end{tabular} & 
        \begin{tabular}{@{}c@{}}
            $G= \{e_G \}$ \\
            {\ref{StdModAreNotL1MGMetTopProj}},
            {\ref{StdModFinGrL1MGMetrInjProjFlatCharac}}
        \end{tabular} \\ 
    \hline
        $M(G)$ & 
        \begin{tabular}{@{}c@{}}
            $G$ дискретна \\
            {\ref{L1MetTopProjAndMetrFlatOfMeasAlg}}
        \end{tabular} & 
        \begin{tabular}{@{}c@{}}
            $G= \{e_G \}$ \\
            {\ref{StdModAreNotL1MGMetTopProj}},
            {\ref{StdModFinGrL1MGMetrInjProjFlatCharac}}
        \end{tabular} & 
        \begin{tabular}{@{}c@{}}
            $G$ дискретна \\
            {\ref{MeasAlgIsL1TopFlat}}
        \end{tabular} & 
        \begin{tabular}{@{}c@{}}
            $G$ любая \\
            {\ref{MGModMGMetTopProjFlatCharac}}
        \end{tabular} & 
        \begin{tabular}{@{}c@{}}
            $G= \{e_G \}$ \\
            {\ref{StdModAreNotL1MGMetTopProj}},
            {\ref{StdModFinGrL1MGMetrInjProjFlatCharac}}
        \end{tabular} & 
        \begin{tabular}{@{}c@{}}
            $G$ любая \\
            {\ref{MGModMGMetTopProjFlatCharac}}
        \end{tabular} \\ 
    \hline
        $C_0(G)$ & 
        \begin{tabular}{@{}c@{}}
            $G= \{e_G \}$ \\
            {\ref{StdModAreNotL1MGMetTopProj}},
            {\ref{StdModFinGrL1MGMetrInjProjFlatCharac}}
        \end{tabular} & 
        \begin{tabular}{@{}c@{}}
            $G$ конечна \\
            {\ref{StdModAreNotL1MGMetTopProj}},
            {\ref{StdModFinGrL1MGMetrInjProjFlatCharac}}
        \end{tabular} & 
        \begin{tabular}{@{}c@{}}
            $G= \{e_G \}$ \\
            {\ref{StdModAreNotL1MGMetTopProj}},
            {\ref{StdModFinGrL1MGMetrInjProjFlatCharac}}
        \end{tabular} & 
        \begin{tabular}{@{}c@{}}
            $G= \{e_G \}$ \\
            {\ref{StdModAreNotL1MGMetTopProj}},
            {\ref{StdModFinGrL1MGMetrInjProjFlatCharac}}
        \end{tabular} & 
        \begin{tabular}{@{}c@{}}
            $G$ конечна \\
            {\ref{StdModAreNotL1MGMetTopProj}},
            {\ref{StdModFinGrL1MGMetrInjProjFlatCharac}}
        \end{tabular} & 
        \begin{tabular}{@{}c@{}}
            $G= \{e_G \}$ \\
            {\ref{StdModAreNotL1MGMetTopProj}},
            {\ref{StdModFinGrL1MGMetrInjProjFlatCharac}}
        \end{tabular} \\ 
    \hline
        $\mathbb{C}_\gamma$ & 
        \begin{tabular}{@{}c@{}}
            $G$ компактна \\
            {\ref{OneDimL1ModMetTopProjCharac}}
        \end{tabular} & 
        \begin{tabular}{@{}c@{}}
            $G$ аменабельна \\
            {\ref{OneDimL1ModMetTopInjFlatCharac}}
        \end{tabular} & 
        \begin{tabular}{@{}c@{}}
            $G$ аменабельна \\
            {\ref{OneDimL1ModMetTopInjFlatCharac}}
        \end{tabular} & 
        \begin{tabular}{@{}c@{}}
            $G$ компактна \\
            {\ref{OneDimL1ModMetTopProjCharac}},
            {\ref{MGMetTopProjInjFlatRedToL1}}
        \end{tabular} & 
        \begin{tabular}{@{}c@{}}
            $G$ аменабельна \\
            {\ref{OneDimL1ModMetTopInjFlatCharac}},
            {\ref{MGMetTopProjInjFlatRedToL1}}
        \end{tabular} & 
        \begin{tabular}{@{}c@{}}
            $G$ аменабельна \\
            {\ref{OneDimL1ModMetTopInjFlatCharac}},
            {\ref{MGMetTopProjInjFlatRedToL1}}
        \end{tabular} \\ 
    \hline
        \multicolumn{7}{c}{
            \mbox{Гомологически тривиальные $L_1(G)$- 
                и $M(G)$-модули в топологической теории
            }
        } \\
    \hline & 
        \multicolumn{3}{c|}{
            $L_1(G)$-модули
        } & 
        \multicolumn{3}{c|}{
            $M(G)$-модули
        } \\
    \hline & 
        Проективность & 
        Инъективность & 
        Плоскость & 
        Проективность & 
        Инъективность & 
        Плоскость \\ 
    \hline
        $L_1(G)$ & 
        \begin{tabular}{@{}c@{}}
            $G$ дискретна \\
            {\ref{L1ModL1MetTopProjCharac}}
        \end{tabular} & 
        \begin{tabular}{@{}c@{}}
            $G$ конечна \\
            {\ref{StdModAreNotL1MGMetTopProj}},
            {\ref{StdModFinGrL1MGTopInjProjFlatCharac}}
        \end{tabular} & 
        \begin{tabular}{@{}c@{}}
            $G$ любая \\
            {\ref{LInfIsL1MetrInj}}
        \end{tabular} & 
        \begin{tabular}{@{}c@{}}
            $G$ дискретна \\
            {\ref{L1ModL1MetTopProjCharac}},
            {\ref{MGMetTopProjInjFlatRedToL1}}
        \end{tabular} & 
        \begin{tabular}{@{}c@{}}
            $G$ конечна \\
            {\ref{StdModAreNotL1MGMetTopProj}},
            {\ref{StdModFinGrL1MGTopInjProjFlatCharac}}
        \end{tabular} & 
        \begin{tabular}{@{}c@{}}
            $G$ любая \\
            {\ref{LInfIsL1MetrInj}},
            {\ref{MGMetTopProjInjFlatRedToL1}}
        \end{tabular} \\ 
    \hline
        $L_p(G)$ & 
        \begin{tabular}{@{}c@{}}
            $G$ конечна \\
            {\ref{NoInfDimRefMetTopProjInjFlatModOverMthscrL1OrLInfty}},
            {\ref{StdModFinGrL1MGTopInjProjFlatCharac}}
        \end{tabular} & 
        \begin{tabular}{@{}c@{}}
            $G$ конечна \\
            {\ref{NoInfDimRefMetTopProjInjFlatModOverMthscrL1OrLInfty}},
            {\ref{StdModFinGrL1MGTopInjProjFlatCharac}}
        \end{tabular} & 
        \begin{tabular}{@{}c@{}}
            $G$ конечна \\
            {\ref{NoInfDimRefMetTopProjInjFlatModOverMthscrL1OrLInfty}},
            {\ref{StdModFinGrL1MGTopInjProjFlatCharac}}
        \end{tabular} & 
        \begin{tabular}{@{}c@{}}
            $G$ конечна \\
            {\ref{NoInfDimRefMetTopProjInjFlatModOverMthscrL1OrLInfty}},
            {\ref{StdModFinGrL1MGTopInjProjFlatCharac}}
        \end{tabular} & 
        \begin{tabular}{@{}c@{}}
            $G$ конечна \\
            {\ref{NoInfDimRefMetTopProjInjFlatModOverMthscrL1OrLInfty}},
            {\ref{StdModFinGrL1MGTopInjProjFlatCharac}}
        \end{tabular} & 
        \begin{tabular}{@{}c@{}}
            $G$ конечна \\
            {\ref{NoInfDimRefMetTopProjInjFlatModOverMthscrL1OrLInfty}},
            {\ref{StdModFinGrL1MGTopInjProjFlatCharac}}
        \end{tabular} \\ 
    \hline
        $L_\infty(G)$ &
        \begin{tabular}{@{}c@{}}
            $G$ конечна \\
            {\ref{StdModAreNotL1MGMetTopProj}},
            {\ref{StdModFinGrL1MGTopInjProjFlatCharac}}
        \end{tabular} & 
        \begin{tabular}{@{}c@{}}
            $G$ любая \\
            {\ref{LInfIsL1MetrInj}}
        \end{tabular} & 
        \begin{tabular}{@{}c@{}}
            $G$ конечна \\
            {\ref{StdModAreNotL1MGMetTopProj}},
            {\ref{StdModFinGrL1MGTopInjProjFlatCharac}}
        \end{tabular} & 
        \begin{tabular}{@{}c@{}}
            $G$ конечна \\
            {\ref{StdModAreNotL1MGMetTopProj}},
            {\ref{StdModFinGrL1MGTopInjProjFlatCharac}}
        \end{tabular} & 
        \begin{tabular}{@{}c@{}}
            $G$ любая \\
            {\ref{LInfIsL1MetrInj}},
            {\ref{MGMetTopProjInjFlatRedToL1}}
        \end{tabular} & 
        \begin{tabular}{@{}c@{}}
            $G$ конечна \\
            {\ref{StdModAreNotL1MGMetTopProj}},
            {\ref{StdModFinGrL1MGTopInjProjFlatCharac}}
        \end{tabular} \\ 
    \hline
        $M(G)$ & 
        \begin{tabular}{@{}c@{}}
            $G$ дискретна \\
            {\ref{L1MetTopProjAndMetrFlatOfMeasAlg}}
        \end{tabular} & 
        \begin{tabular}{@{}c@{}}
            $G$ конечна \\
            {\ref{NoInfDimRefMetTopProjInjFlatModOverMthscrL1OrLInfty}},
            {\ref{StdModFinGrL1MGTopInjProjFlatCharac}}
        \end{tabular} & 
        \begin{tabular}{@{}c@{}}
            $G$ любая \\
            {\ref{MeasAlgIsL1TopFlat}}
        \end{tabular} & 
        \begin{tabular}{@{}c@{}}
            $G$ любая \\
            {\ref{MGModMGMetTopProjFlatCharac}}
        \end{tabular} & 
        \begin{tabular}{@{}c@{}}
            $G$ конечна \\
            {\ref{NoInfDimRefMetTopProjInjFlatModOverMthscrL1OrLInfty}},
            {\ref{StdModFinGrL1MGTopInjProjFlatCharac}}
        \end{tabular} & 
        \begin{tabular}{@{}c@{}}
            $G$ любая \\
            {\ref{MGModMGMetTopProjFlatCharac}}
        \end{tabular} \\ 
    \hline 
        $C_0(G)$ & 
        \begin{tabular}{@{}c@{}}
            $G$ конечна \\
            {\ref{StdModAreNotL1MGMetTopProj}},
            {\ref{StdModFinGrL1MGTopInjProjFlatCharac}}
        \end{tabular} & 
        \begin{tabular}{@{}c@{}}
            $G$ конечна \\
            {\ref{StdModAreNotL1MGMetTopProj}},
            {\ref{StdModFinGrL1MGTopInjProjFlatCharac}}
        \end{tabular} & 
        \begin{tabular}{@{}c@{}}
            $G$ конечна \\
            {\ref{StdModAreNotL1MGMetTopProj}},
            {\ref{StdModFinGrL1MGTopInjProjFlatCharac}}
        \end{tabular} & 
        \begin{tabular}{@{}c@{}}
            $G$ конечна \\
            {\ref{StdModAreNotL1MGMetTopProj}},
            {\ref{StdModFinGrL1MGTopInjProjFlatCharac}}
        \end{tabular} & 
        \begin{tabular}{@{}c@{}}
            $G$ конечна \\
            {\ref{StdModAreNotL1MGMetTopProj}},
            {\ref{StdModFinGrL1MGTopInjProjFlatCharac}}
        \end{tabular} & 
        \begin{tabular}{@{}c@{}}
            $G$ конечна \\
            {\ref{StdModAreNotL1MGMetTopProj}},
            {\ref{StdModFinGrL1MGTopInjProjFlatCharac}}
        \end{tabular} \\ 
    \hline  
        $\mathbb{C}_\gamma$ & 
        \begin{tabular}{@{}c@{}}
            $G$ компактна \\
            {\ref{OneDimL1ModMetTopProjCharac}}
        \end{tabular} & 
        \begin{tabular}{@{}c@{}}
            $G$ аменабельна \\
            {\ref{OneDimL1ModMetTopInjFlatCharac}}
        \end{tabular} & 
        \begin{tabular}{@{}c@{}}
            $G$ аменабельна \\
            {\ref{OneDimL1ModMetTopInjFlatCharac}}
        \end{tabular} & 
        \begin{tabular}{@{}c@{}}
            $G$ компактна \\
            {\ref{OneDimL1ModMetTopProjCharac}},
            {\ref{MGMetTopProjInjFlatRedToL1}}
        \end{tabular} & 
        \begin{tabular}{@{}c@{}}
            $G$ аменабельна \\
            {\ref{OneDimL1ModMetTopInjFlatCharac}},
            {\ref{MGMetTopProjInjFlatRedToL1}}
        \end{tabular} & 
        \begin{tabular}{@{}c@{}}
            $G$ аменабельна \\
            {\ref{OneDimL1ModMetTopInjFlatCharac}},
            {\ref{MGMetTopProjInjFlatRedToL1}}
        \end{tabular} \\ 
    \hline
        \multicolumn{7}{c}{
            \mbox{
                Гомологически тривиальные $L_1(G)$- 
                и $M(G)$-модули в относительной теории
            }
        } \\

    \hline & 
        \multicolumn{3}{c|}{
            $L_1(G)$-модули
        } & 
        \multicolumn{3}{c|}{
            $M(G)$-модули
        } \\
    \hline & 
        Проективность & 
        Инъективность & 
        Плоскость & 
        Проективность & 
        Инъективность & 
        Плоскость \\
    \hline  
        $L_1(G)$ & 
        \begin{tabular}{@{}c@{}}
            $G$ любая \\
            \mbox{
                {\cite{DalPolHomolPropGrAlg}}, \S 6
            }
        \end{tabular} & 
        \begin{tabular}{@{}c@{}}
            $G$ аменабельна \\ 
            и дискретна \\
            \mbox{
                {\cite{DalPolHomolPropGrAlg}}, \S 6
            }
        \end{tabular} & 
        \begin{tabular}{@{}c@{}}
            $G$ любая \\ 
            \mbox{
                {\cite{DalPolHomolPropGrAlg}}, \S 6
            }
        \end{tabular}                                                                &
        \begin{tabular}{@{}c@{}}
            $G$ любая \\ 
            \mbox{
                {\cite{RamsHomPropSemgroupAlg}}, \S 3.5
            }
        \end{tabular} &
        \begin{tabular}{@{}c@{}}
            $G$ аменабельна \\ 
            и дискретна \\
            \mbox{
                {\cite{RamsHomPropSemgroupAlg}}, \S 3.5
            }
        \end{tabular} & 
        \begin{tabular}{@{}c@{}}
            $G$ любая \\
            \mbox{
                {\cite{RamsHomPropSemgroupAlg}}, \S 3.5
            }
        \end{tabular} \\ 
    \hline 
        $L_p(G)$ & 
        \begin{tabular}{@{}c@{}}
            $G$ компактна \\
            \mbox{
                {\cite{DalPolHomolPropGrAlg}}, \S 6
            }
        \end{tabular} & 
        \begin{tabular}{@{}c@{}}
            $G$ аменабельна \\
            {\cite{RachInjModAndAmenGr}}
        \end{tabular} & 
        \begin{tabular}{@{}c@{}}
            $G$ аменабельна \\
            {\cite{RachInjModAndAmenGr}}
        \end{tabular} & 
        \begin{tabular}{@{}c@{}}
            $G$ компактна \\
            \mbox{
                {\cite{RamsHomPropSemgroupAlg}}, \S 3.5
            }
        \end{tabular} & 
        \begin{tabular}{@{}c@{}}
            $G$ аменабельна \\
            \mbox{
                {\cite{RamsHomPropSemgroupAlg}}, \S 3.5
            },{\cite{RachInjModAndAmenGr}}
        \end{tabular} &
        \begin{tabular}{@{}c@{}}
            $G$ аменабельна \\
            \mbox{
                {\cite{RamsHomPropSemgroupAlg}}, \S 3.5
            }
        \end{tabular} \\
    \hline
        $L_\infty(G)$ & 
        \begin{tabular}{@{}c@{}}
            $G$ конечна \\
            \mbox{
                {\cite{DalPolHomolPropGrAlg}}, \S 6
            }
        \end{tabular} & 
        \begin{tabular}{@{}c@{}}
            $G$ любая \\ 
            \mbox{
                {\cite{DalPolHomolPropGrAlg}}, \S 6
            }
        \end{tabular} &
        \begin{tabular}{@{}c@{}}
            $G$ аменабельна \\
            \mbox{
                {\cite{DalPolHomolPropGrAlg}}, \S 6
            }
        \end{tabular} &
        \begin{tabular}{@{}c@{}}
            $G$ конечна \\
            \mbox{
                {\cite{RamsHomPropSemgroupAlg}}, \S 3.5
            }
        \end{tabular} &
        \begin{tabular}{@{}c@{}}
            $G$ любая \\
            \mbox{
                {\cite{RamsHomPropSemgroupAlg}}, \S 3.5
            }
        \end{tabular} &
        \begin{tabular}{@{}c@{}}
            $G$ аменабельна \\
            ($\implies$)\mbox{
                {\cite{RamsHomPropSemgroupAlg}}, \S 3.5
            }
        \end{tabular} \\ 
    \hline
        $M(G)$ & 
        \begin{tabular}{@{}c@{}}
            $G$ дискретна \\
            \mbox{
                {\cite{DalPolHomolPropGrAlg}}, \S 6
            }
        \end{tabular} & 
        \begin{tabular}{@{}c@{}}
            $G$ аменабельна \\
            \mbox{
                {\cite{DalPolHomolPropGrAlg}}, \S 6
            }
        \end{tabular} &
        \begin{tabular}{@{}c@{}}
            $G$ любая \\
            \mbox{
                {\cite{RamsHomPropSemgroupAlg}}, \S 3.5
            }
        \end{tabular} &
        \begin{tabular}{@{}c@{}}
            $G$ любая \\
            \mbox{
                {\cite{RamsHomPropSemgroupAlg}}, \S 3.5
            }
        \end{tabular} &
        \begin{tabular}{@{}c@{}}
            $G$ аменабельна \\
            \mbox{
                {\cite{RamsHomPropSemgroupAlg}}, \S 3.5
            }
        \end{tabular} &
        \begin{tabular}{@{}c@{}}
            $G$ любая \\
            \mbox{
                {\cite{RamsHomPropSemgroupAlg}}, \S 3.5
            }
        \end{tabular} \\
    \hline
        $C_0(G)$ & 
        \begin{tabular}{@{}c@{}}
            $G$ компактна \\
            \mbox{
                {\cite{DalPolHomolPropGrAlg}}, \S 6
            }
        \end{tabular} & 
        \begin{tabular}{@{}c@{}}
            $G$ конечна \\ 
            \mbox{
                {\cite{DalPolHomolPropGrAlg}}, \S 6
            }
        \end{tabular} &
        \begin{tabular}{@{}c@{}}
            $G$ аменабельна \\
            \mbox{
                {\cite{DalPolHomolPropGrAlg}}, \S 6
            }
        \end{tabular} &
        \begin{tabular}{@{}c@{}}
            $G$ компактна \\
            \mbox{
                {\cite{RamsHomPropSemgroupAlg}}, \S 3.5
            }
        \end{tabular} &
        \begin{tabular}{@{}c@{}}
            $G$ конечна \\
            \mbox{
                {\cite{RamsHomPropSemgroupAlg}}, \S 3.5
            }
        \end{tabular} &
        \begin{tabular}{@{}c@{}}
            $G$ аменабельна \\
            \mbox{
                {\cite{RamsHomPropSemgroupAlg}}, \S 3.5
            }
        \end{tabular} \\
    \hline
        $\mathbb{C}_\gamma$ & 
        \begin{tabular}{@{}c@{}}
            $G$ компактна \\
            {\ref{OneDimL1ModMetTopProjCharac}}
        \end{tabular} & 
        \begin{tabular}{@{}c@{}}
            $G$ аменабельна \\
            {\ref{OneDimL1ModMetTopInjFlatCharac}}
        \end{tabular} & 
        \begin{tabular}{@{}c@{}}
            $G$ аменабельна \\
            {\ref{OneDimL1ModMetTopInjFlatCharac}}
        \end{tabular} & 
        \begin{tabular}{@{}c@{}}
            $G$ компактна \\
            {\ref{OneDimL1ModMetTopProjCharac}},
            {\ref{MGMetTopProjInjFlatRedToL1}}
        \end{tabular} & 
        \begin{tabular}{@{}c@{}}
            $G$ аменабельна \\
            {\ref{OneDimL1ModMetTopInjFlatCharac}},
            {\ref{MGMetTopProjInjFlatRedToL1}}
        \end{tabular} & 
        \begin{tabular}{@{}c@{}}
            $G$ аменабельна \\
            {\ref{OneDimL1ModMetTopInjFlatCharac}},
            {\ref{MGMetTopProjInjFlatRedToL1}}
        \end{tabular} \\
    \hline
    \end{longtable}
\end{scriptsize}

Следует сказать, что результаты о модулях $L_p(G)$ верны для обоих видов
внешнего умножения $\convol_p$ и $\cdot_p$. Наиболее интересный результат
параграфа состоит в том, что $L_1(G)$-модуль $L_1(G)$ метрически или
топологически проективен только для дискретной группы $G$. Для метрического
случая это было доказано в [\cite{GravInjProjBanMod}, теорема 4.14(ii)]. 

%-------------------------------------------------------------------------------
%	An example of small category.
%-------------------------------------------------------------------------------

\section{Пример ``маленькой'' 
    категории}\label{SectionAnExampleOfSmallCategory}

Как мы видели по многим примерам, большинство стандартных модулей анализа
оказываются гомологически нетривиальными по отношению к большим категориям,
таким, например, как категория всех банаховых модулей. Ситуация может
кардинально измениться для сравнительно ``маленьких'' категорий. Данный параграф
посвящен построению содержательного примера такого рода.

%-------------------------------------------------------------------------------
%	Preliminaries on measure theory and L_p-spaces
%-------------------------------------------------------------------------------

\subsection{Предварительные сведения по теории 
    меры и \texorpdfstring{$L_p$}{Lp}-пространствам}\label{
        SubSectionPreliminariesOnMeasureTheoryAndLpSpaces}

Прежде чем переходить к основной теме, мы напомним некоторые основные факты из
теории меры и договоримся об обозначениях. Все эти предварительные сведения
можно найти в первых двух томах~\cite{FremMeasTh}. Пусть $(\Omega, \Sigma, \mu)$
--- пространство с мерой. Будем говорить, что измеримое множество $E$ является
атомом, если $\mu(E)>0$ и для любого измеримого подмножества $F$ в $E$ либо $F$
либо $E\setminus F$ имеют меру $0$. Непосредственно из определения следует, что
атомы в строго локализуемых пространствах с мерой имеют конечную меру. Вообще
говоря, атом может и не быть одноточечным множеством.

Пространство с мерой называется неатомическим, если его мера не имеет ни одного
атома. Пространство с мерой называется атомическим, если каждое измеримое
множество положительной меры содержит атом. С помощью леммы Цорна легко
показать, что атомическое пространство с мерой можно представить в виде
объединения непересекающихся атомов.  Это семейство счетно, если пространство с
мерой $\sigma$-конечно. Эти факты позволяют сказать, что структура атомических
пространств с мерой полностью изучена. 

Структура строго локализуемых неатомических пространств с мерой описана Дороти
Махарам [\cite{FremMeasTh}, теорема 332B], но нам понадобится лишь следующее
свойство таких пространств с мерой [\cite{FremMeasTh}, предложение 215D]: если
$E$ --- измеримое множество положительной меры в неатомическом пространстве с
мерой, то для всех $0\leq t\leq \mu(E)$ существует измеримое подмножество $F$ в
$E$ такое, что $\mu(F)=t$.

Допустим, $(\Omega,\Sigma,\mu)$ является $\sigma$-конечным пространством с
мерой, тогда существуют атомическая мера $\mu_1:\Sigma\to[0,+\infty]$ и
неатомическая мера $\mu_2:\Sigma\to[0,+\infty]$ такие, что $\mu=\mu_1+\mu_2$.
Более того, существуют измеримые множества $\Omega_a^{\mu}$ и
$\Omega_{na}^{\mu}=\Omega\setminus \Omega_a^{\mu}$ такие, что
$\mu_1(\Omega_{na}^{\mu})=\mu_2(\Omega_a^{\mu})=0$. Множества $\Omega_a^{\mu}$ и
$\Omega_{na}^{\mu}$ называются соответственно атомической и неатомической частью
пространства с мерой $(\Omega,\Sigma,\mu)$.

Вообще говоря, две меры могут быть ни абсолютно непрерывны, ни сингулярны по
отношению друг к другу. На этот случай есть теорема Лебега о разложении мер. Для
двух заданных $\sigma$-конечных мер $\mu$ и $\nu$, заданных на измеримом
пространстве $(\Omega,\Sigma)$, существует измеримая функция
$\rho_{\nu,\mu}:\Omega\to\mathbb{R}$, $\sigma$-конечная мера
$\nu_s:\Sigma\to[0,+\infty]$ и два измеримых множества $\Omega_s^{\nu,\mu}$,
$\Omega_c^{\nu,\mu}=\Omega\setminus\Omega_s^{\nu,\mu}$ таких, что
$\nu=\rho_{\nu,\mu}\mu+\nu_s$ и
$\mu(\Omega_s^{\nu,\mu})=\nu_s(\Omega_c^{\nu,\mu})=0$, то есть $\mu\perp\nu_s$.

Перейдем к обсуждению $L_p$-пространств. Пусть $1\leq p\leq+\infty$ и
$(\Omega,\Sigma,\mu)$ --- пространство с мерой. Если все пространство $\Omega$
является атомом, то его $L_p$-пространство одномерно и существует изометрический
изоморфизм
$$
J_p
:L_p(\Omega,\mu)\to \ell_p(\mathbb{N}_1)
:f\mapsto{\left(
    1\mapsto {\mu(\Omega)}^{1/p-1}\int_{\Omega} f(\omega)d\mu(\omega)
\right)}.
$$
Если $\Omega=\bigcup_{\lambda\in\Lambda}\Omega_\lambda$ есть представление
$\Omega$ как объединения непересекающихся измеримых множеств, то для всех $1\leq
p\leq+\infty$ мы имеем изометрический изоморфизм
$$
I_p:L_p(\Omega,\mu)\to \bigoplus\nolimits_p \{ 
    L_p(\Omega_\lambda,\mu|_{\Omega_\lambda}):\lambda\in\Lambda 
 \}
: f\mapsto (\lambda\mapsto f|_{\Omega_\lambda}).
$$
Если каждое множество $\Omega_\lambda$ является атомом, то $\Omega$ ---
атомическое пространство с мерой, и мы получаем еще один изометрический
изоморфизм
$$
\widetilde{I}_p:L_p(\Omega,\mu)\to \ell_p(\Lambda)
:f\mapsto\left (\lambda\mapsto {\mu(\Omega_\lambda)}^{1/p-1}
\int_{\Omega_\lambda} f(\omega)d\mu(\omega)\right).
$$
Замечание: работая с индексами $p$, мы будем полагать по определению, что
$1/0=\infty$ и $1/\infty=0$. Еще один полезный прием в изучении
$L_p$-пространств --- это так называемая замена плотности: если
$\rho:\Omega\to(0,+\infty)$ --- измеримая функция, то оператор
$$
\bar{I}_p:L_p(\Omega,\mu)\to L_p(\Omega,\rho\mu): f\mapsto\rho^{-1/p} f
$$
является изометрическим изоморфизмом. Для различных значений $p$
бесконечномерные $L_p$-пространства не топологически изоморфны. Это можно
доказать с использованием понятий котипа и типа [\cite{KalAlbTopicsBanSpTh},
теорема 6.2.14]. Очевидно, в конечномерном случае изоморфизм существует только
для пространств одинаковой размерности. Более точно: если $\Lambda$ --- конечное
множество и $1\leq p,q\leq +\infty$, то существует константа $C_{p,q}>0$ такая,
что $\Vert x\Vert_{\ell_p(\Lambda)}\leq C_{p,q}\Vert x\Vert_{\ell_q(\Lambda)}$
для всех $x\in\mathbb{C}^\Lambda$.

%-------------------------------------------------------------------------------
%	The category of B(\Omega,\Sigma)-module L_p
%-------------------------------------------------------------------------------

\subsection{Категория 
    \texorpdfstring{$B(\Omega,\Sigma)$}{B (Omega,Sigma)}-модулей
    \texorpdfstring{$L_p$}{Lp}}\label{
        SubSectionTheCategoryOfBOmegaSigmaModulesLp}

``Маленькая'' категория, которую мы будем изучать --- это категория
$B(\Omega,\Sigma)$-модулей вида $L_p(\Omega,\mu)$ на некотором измеримом
пространстве $(\Omega,\Sigma)$ для различных $\sigma$-конечных положительных мер
$\mu$ и различных $1\leq p\leq +\infty$. Мы обозначим эту категорию как
$B(\Omega,\Sigma)-\mathbf{mod(L)}$. Мы покажем, что все ее модули являются
метрически и топологически проективными, инъективными и плоскими. По пути мы
дадим полное описание топологически сюръективных, топологически инъективных,
коизометрических и изометрических операторов умножения между
$L_p$-пространствами. В~\cite{HelTensProdAndMultModLp} Хелемский полностью
описал морфизмы $B(\Omega,\Sigma)-\mathbf{mod(L)}$, но только для случая
локально компактного пространства $\Omega$ с борелевской $\sigma$-алгеброй.
Внимательное изучение его доказательства показывает, что это описание верно для
всех $\sigma$-конечных пространств с мерой. Чтобы правильно сформулировать
результат, нам понадобится несколько обозначений. Через $L_0(\Omega,\Sigma)$ мы
обозначим линейное пространство измеримых функций на $\Omega$. Для $1\leq
p,q\leq +\infty$ и положительных $\sigma$-конечных мер $\mu,\nu$ на измеримом
пространстве $(\Omega,\Sigma)$ мы обозначим $\Omega_+:=
\{\omega\in\Omega_c^{\nu,\mu}:\rho_{\nu,\mu}(\omega)>0 \}$ и
$$
L_{p,q,\mu,\nu}(\Omega):=
\begin{cases}
 \{g\in L_0(\Omega,\Sigma)
    :g\in L_{pq/(p-q)}(\Omega,\rho_{\nu,\mu}^{p/(p-q)}\mu),\quad 
    g|_{\Omega\setminus\Omega_+}=0 \}&\text{ если }\quad p>q\\
 \{g\in L_0(\Omega,\Sigma)
    :g\rho_{\nu,\mu}^{1/p}\in L_{\infty}(\Omega,\mu),\quad 
    g|_{\Omega\setminus\Omega_+}=0 \}&\text{ если }\quad p=q\\
 \{g\in L_0(\Omega,\Sigma)
    :g\rho_{\nu,\mu}^{1/p}\mu^{pq/(p-q)}\in L_{\infty}(\Omega,\mu),\quad 
    g|_{\Omega\setminus\Omega_a^{\mu}}=0 \}&\text{ если }\quad p<q,\\
\end{cases}
$$
$$
\Vert g\Vert_{L_{p,q,\mu,\nu}(\Omega)}:=
\begin{cases}
    \Vert g\Vert_{L_{pq/(p-q)}(\Omega,\rho_{\nu,\mu}^{p/(p-q)}\mu)}         &
    \text{ если }\quad p>q                                                  \\
    \Vert g\rho_{\nu,\mu}^{1/p}\Vert_{L_{\infty}(\Omega,\mu)}               &
    \text{ если }\quad p=q                                                  \\
    \Vert g\rho_{\nu,\mu}^{1/p}\mu^{pq/(p-q)}\Vert_{L_{\infty}(\Omega,\mu)} &
    \text{ если }\quad p<q.                                                 \\
\end{cases}
$$

\begin{theorem}[\cite{HelTensProdAndMultModLp}, теорема
4.1]\label{LpModMorphCharac} Пусть $(\Omega,\Sigma)$ --- измеримое пространство,
$1\leq p,q\leq +\infty$ и $\mu,\nu$ --- две $\sigma$-конечные меры на 
$(\Omega, \Sigma)$. Тогда существует изометрический изоморфизм
$$
\mathcal{I}_{p,q,\mu,\nu}:L_{p,q,\mu,\nu}(\Omega)
    \to
\operatorname{Hom}_{
    B(\Omega,\Sigma)-\mathbf{mod(L)}
}(L_p(\Omega,\mu),L_q(\Omega,\nu)):
g\mapsto (f\mapsto g f).
$$
\end{theorem}

Проще говоря, все морфизмы в $B(\Omega,\Sigma)-\mathbf{mod(L)}$ являются
операторами умножения.

%-------------------------------------------------------------------------------
%	Multiplication operators
%-------------------------------------------------------------------------------

\subsection{Операторы умножения}\label{SubSectionMultiplicationOperators}

Пусть $(\Omega,\Sigma,\mu)$ и $(\Omega,\Sigma,\nu)$ --- два пространства с мерой
с одной и той же $\sigma$-алгеброй измеримых множеств. Для заданной функции
$g\in L_0(\Omega,\Sigma)$ и чисел $1\leq p,q\leq +\infty$ мы определим оператор
умножения
$$
M_g:L_p(\Omega,\mu)\to L_q(\Omega,\nu): f\mapsto g f.
$$ 
Конечно, требуются некоторые ограничения на $g$, $\mu$ и $\nu$ чтобы оператор
$M_g$ был корректно определен. Для заданного множества $E\in\Sigma$ через
$M_g^E$ мы обозначим линейный оператор
$$
M_g^E:L_p(E,\mu|_E)\to L_q(E,\nu|_E):f\mapsto g|_E f.
$$
Он корректно определен, так как равенство $f|_{\Omega\setminus E}=0$ влечет
$M_g(f)|_{\Omega\setminus E}=0$. 

\begin{proposition}\label{MultpOpSurjInjDesc} Пусть $(\Omega,\Sigma,\mu)$ ---
пространство с мерой и $g\in L_0(\Omega,\Sigma)$. Обозначим 
$Z_g:=g^{-1}( \{0 \})$. Тогда для оператора 
$M_g:L_p(\Omega,\mu)\to L_q(\Omega,\mu)$ выполнено:

\begin{enumerate}[label = (\roman*)]
    \item $\operatorname{Ker}(M_g)= \{
        f\in L_p(\Omega,\mu):f|_{\Omega\setminus {Z_g}}=0 
    \}$, то есть оператор $M_g$ инъективен тогда и только тогда, когда
    $\mu(Z_g)=0$;

    \item $\operatorname{Im}(M_g)\subset \{h\in L_q(\Omega,\mu): h|_{Z_g}=0 \}$,
    поэтому если $M_g$ сюръективен, то $\mu(Z_g)=0$.
\end{enumerate}
\end{proposition}
\begin{proof}
$(i)$ Желаемое равенство следует из цепочки эквивалентностей:
$$
f\in\operatorname{Ker}(M_g)
\Longleftrightarrow g f=0
\Longleftrightarrow f|_{\Omega\setminus Z_g}=0.
$$
$(ii)$ Так как $g|_{Z_g}=0$, то для всех функций $f\in L_p(\Omega,\mu)$ в
ыполнено $M_g(f)|_{Z_g}=(g f)|_{Z_g}=0$ и мы получаем нужное включение. 
Если оператор $M_g$ сюръективен, то, очевидно, $\mu(Z_g)=0$.
\end{proof} 

Для заданного измеримого множества $E\in\Sigma$ и функции $f\in
L_0(E,\Sigma|_{E})$ через $\widetilde{f}$ мы будем обозначать функцию из
$L_0(\Omega, \Sigma)$ такую, что $\widetilde{f}|_E=f$ и
$\widetilde{f}|_{\Omega\setminus E}=0$.

\begin{proposition}\label{MultOpDecompDecomp} Пусть $(\Omega,\Sigma,\mu)$,
$(\Omega,\Sigma,\nu)$ --- два пространства с мерой и $1\leq p,q\leq +\infty$.
Допустим, имеется представление
$\Omega=\bigcup_{\lambda\in\Lambda}\Omega_\lambda$ множества $\Omega$ в виде
конечного объединения непересекающихся измеримых множеств. Тогда 

\begin{enumerate}[label = (\roman*)]
    \item оператор $M_g$ топологически инъективен тогда и только тогда, когда
    операторы $M_g^{\Omega_\lambda}$ топологически инъективны для всех
    $\lambda\in\Lambda$;

    \item оператор $M_g$ топологически сюръективен тогда и только тогда, когда
    операторы $M_g^{\Omega_\lambda}$ топологически сюръективны для всех
    $\lambda\in\Lambda$;

    \item если оператор $M_g$ изометричен, то таковы и операторы
    $M_g^{\Omega_\lambda}$ для всех $\lambda\in\Lambda$;

    \item если оператор $M_g$ коизометричен, то таковы и операторы
    $M_g^{\Omega_\lambda}$ для всех $\lambda\in\Lambda$.
\end{enumerate}
\end{proposition}
\begin{proof}
$(i)$ Пусть оператор $M_g$ $c$-топологически инъективен. Зафиксируем индекс
$\lambda\in\Lambda$ и функцию $f\in L_p(\Omega_\lambda,\mu|_{\Omega_\lambda})$.
Тогда 
$$
\Vert M_g^{\Omega_\lambda}(f)\Vert_{L_q(\Omega_\lambda,\nu|_{\Omega_\lambda})}
=\Vert g \widetilde{f}\Vert_{L_q(\Omega,\nu)}
\geq c^{-1}\Vert\widetilde{f}\Vert_{L_p(\Omega,\mu)}
=c^{-1}\Vert f\Vert_{L_p(\Omega_\lambda,\mu|_{\Omega_\lambda})}.
$$
Следовательно, операторы $M_g^{\Omega_\lambda}$ $c$-топологически инъективны для
всех $\lambda\in\Lambda$. 

Обратно, допустим, что операторы $M_g^{\Omega_\lambda}$ $c'$-топологически
инъективны для всех $\lambda\in\Lambda$. Тогда для любой функции $f\in
L_p(\Omega,\mu)$ мы имеем
$$
\Vert M_g(f)\Vert_{L_q(\Omega,\nu)}
=\left\Vert\left(
    \Vert 
        M_g^{\Omega_\lambda}(f|_{\Omega_\lambda})
    \Vert_{L_q(\Omega_\lambda,\nu|_{\Omega_\lambda})}
    :\lambda\in\Lambda
\right)\right\Vert_{\ell_q(\Lambda)}
$$
$$
\geq {(c')}^{-1}\left\Vert\left(
    \Vert 
        f|_{\Omega_\lambda}
    \Vert_{L_p(\Omega_\lambda,\mu|_{\Omega_\lambda})}
    :\lambda\in\Lambda
\right)\right\Vert_{\ell_q(\Lambda)}
$$
$$
\geq {(c')}^{-1} C_{p,q}^{-1}\left\Vert\left(
    \Vert 
        f|_{\Omega_\lambda}
    \Vert_{L_p(\Omega_\lambda,\mu|_{\Omega_\lambda})}
    :\lambda\in\Lambda
\right)\right\Vert_{\ell_p(\Lambda)}
={(c')}^{-1}C_{p,q}^{-1}\Vert f\Vert_{L_p(\Omega,\mu)}.
$$
Следовательно, оператор $M_g$ $c$-топологически инъективен для $c=c'C_{p,q}>0$.

$(ii)$ Допустим, оператор $M_g$ $c$-топологически сюръективен. 
Зафиксируем индекс $\lambda\in\Lambda$ 
и функцию $h\in L_q(\Omega_\lambda,\nu|_{\Omega_\lambda})$.
Тогда существует функция $f\in L_p(\Omega,\mu)$ такая, что
$M_g(f)=\widetilde{h}$ и $\Vert f\Vert_{L_p(\Omega,\mu)}\leq c\Vert
\widetilde{h}\Vert_{L_q(\Omega,\nu)}$. Следовательно,
$M_g^{\Omega_\lambda}(f|_{\Omega_\lambda})=\widetilde{h}|_{\Omega_\lambda}=h$ и
$\Vert f|_{\Omega_\lambda}\Vert_{L_p(\Omega_\lambda,\mu|_{\Omega_\lambda})}
\leq\Vert f\Vert_{L_p(\Omega,\mu)}
\leq c\Vert\widetilde{h}\Vert_{L_q(\Omega,\nu)}
=c\Vert h\Vert_{L_q(\Omega_\lambda,\nu|_{\Omega_\lambda})}$. Так как функция $h$
произвольна, то операторы $M_g^{\Omega_\lambda}$ $c$-топологически сюръективны.

Обратно, допустим операторы $M_g^{\Omega_\lambda}$ $c'$-топологически
сюръективны для всех $\lambda\in\Lambda$. Пусть $h$ --- произвольная функция из
$L_q(\Omega,\nu)$. Из предположения мы получаем, что для каждого
$\lambda\in\Lambda$ существует функция $f_\lambda\in
L_p(\Omega_\lambda,\mu|_{\Omega_\lambda})$ такая, что
$M_g^{\Omega_\lambda}(f_\lambda)=h|_{\Omega_\lambda}$ и 
$\Vert f_\lambda\Vert_{L_p(\Omega_\lambda,\mu|_{\Omega_\lambda})}
\leq c'\Vert h|_{\Omega_\lambda}\Vert_{
    L_q(\Omega_\lambda,\nu|_{\Omega_\lambda})
}$. Определим функцию $f\in L_0(\Omega,\Sigma)$ так, 
что $f|_{\Omega_\lambda}=f_\lambda$. Тогда
$$
\Vert f\Vert_{L_p(\Omega,\mu)}
=\left\Vert\left(
    \Vert f_\lambda\Vert_{L_p(\Omega_\lambda,\mu|_{\Omega_\lambda})}
    :\lambda\in\Lambda
\right)\right\Vert_{\ell_p(\Lambda)}
\leq c'\left\Vert\left(
    \Vert h|_{\Omega_\lambda}\Vert_{L_q(\Omega_\lambda,\nu|_{\Omega_\lambda})}
    :\lambda\in\Lambda
\right)\right\Vert_{\ell_p(\Lambda)}
$$
$$
\leq c'C_{p,q}\left\Vert\left(
    \Vert h|_{\Omega_\lambda}\Vert_{L_q(\Omega_\lambda,\nu|_{\Omega_\lambda})}
    :\lambda\in\Lambda
    \right)\right\Vert_{\ell_q(\Lambda)}
=c'C_{p,q}\Vert h\Vert_{L_q(\Omega,\nu)}.
$$
Очевидно, $M_g(f)=h$. Так как функция $h$ произвольна, то оператор $M_g$
$c$-топологически сюръективен для $c=c'C_{p,q}>0$.

$(iii)$ Зафиксируем индекс $\lambda\in\Lambda$ и функцию $f\in
L_p(\Omega_\lambda,\mu|_{\Omega_\lambda})$. Тогда 
$$
\Vert M_g^{\Omega_\lambda}(f)\Vert_{L_q(\Omega_\lambda,\nu|_{\Omega_\lambda})}
=\Vert g \widetilde{f}\Vert_{L_q(\Omega,\nu)}
=\Vert\widetilde{f}\Vert_{L_p(\Omega,\mu)}
=\Vert f\Vert_{L_p(\Omega_\lambda,\mu|_{\Omega_\lambda})}.
$$
Значит, операторы $M_g^{\Omega_\lambda}$ изометричны для всех
$\lambda\in\Lambda$.

$(iv)$ Зафиксируем индекс $\lambda\in\Lambda$. Так как оператор $M_g$
коизометричен, то он сжимающий и $1$-топологически сюръективный. Из
доказательства пункта $(ii)$ мы получаем, что оператор $M_g^{\Omega_\lambda}$
$1$-топологически сюръективен. Пусть $f$ --- произвольная функция из $
L_p(\Omega_\lambda,\mu|_{\Omega_\lambda})$. Так как $M_g$ сжимающий, то
$$
\Vert M_g^{\Omega_\lambda}(f)\Vert_{L_q(\Omega_\lambda,\nu|_{\Omega_\lambda})}
=\Vert M_g(\widetilde{f})\chi_{\Omega_\lambda}\Vert_{L_q(\Omega,\nu)}
=\Vert M_g(\widetilde{f}\chi_{\Omega_\lambda})\Vert_{L_q(\Omega,\nu)}
\leq \Vert\widetilde{f}\chi_{\Omega_\lambda}\Vert_{L_p(\Omega,\mu)}
=\Vert f\Vert_{L_p(\Omega_{\lambda},\mu|_{\Omega_\lambda})}.
$$
Следовательно, оператор $M_g^{\Omega_\lambda}$ сжимающий и $1$-топологически
сюръективный, то есть коизометрический.
\end{proof}

\begin{proposition}\label{MultOpCharacBtwnTwoSingMeasSp} Пусть
$(\Omega,\Sigma,\mu)$ и $(\Omega,\Sigma,\nu)$ --- два $\sigma$-конечных
пространства с мерой. Пусть $1\leq p,q\leq +\infty$ и $g\in L_0(\Omega,\Sigma)$.
Если $\mu\perp\nu$, то оператор $M_g:L_p(\Omega,\mu)\to L_q(\Omega,\nu)$
нулевой.
\end{proposition}
\begin{proof} В силу того, что $\mu\perp\nu$ существует множество
$\Omega_s^{\nu,\mu}\in\Sigma$ такое, что
$\mu(\Omega_s^{\nu,\mu})=\nu(\Omega_c^{\nu,\mu})=0$, где
$\Omega_c^{\nu,\mu}=\Omega\setminus\Omega_s^{\nu,\mu}$. Так как
$\mu(\Omega_s^{\nu,\mu})=0$, то $\chi_{\Omega_c^{\nu,\mu}}=\chi_{\Omega}$ в
$L_p(\Omega,\mu)$ и $\chi_{\Omega_c^{\nu,\mu}}=0$ в $L_q(\Omega,\nu)$. Теперь
для всех функций $f\in L_p(\Omega,\mu)$ мы имеем 
$M_g(f)=M_g(f \chi_{\Omega})
=M_g(f \chi_{\Omega_c^{\nu,\mu}})
=g f\chi_{\Omega_c^{\nu,\mu}}=0$.
Следовательно, $M_g=0$.
\end{proof}

С этого момента начинается главная часть нашего исследования операторов
умножения. Мы покажем, что $\langle$~изометрические / топологически
инъективные~$\rangle$ морфизмы в $B(\Omega,\Sigma)-\mathbf{mod(L)}$ являются
коретракциями. Аналогично, $\langle$~строго коизометрические / топологически
сюръективные~$\rangle$ морфизмы в $B(\Omega,\Sigma)-\mathbf{mod(L)}$ являются
ретракциями. Позже, используя эти описания, мы легко опишем, все метрически и
топологически проективные, инъективные и плоские модули категории
$B(\Omega,\Sigma)-\mathbf{mod(L)}$.

\begin{proposition}\label{MultpOpPropIfPeqqualsQ} Пусть $(\Omega,\Sigma,\mu)$
--- пространство с мерой и $g\in L_0(\Omega,\Sigma)$. Тогда 

\begin{enumerate}[label = (\roman*)]
    \item оператор $M_g:L_p(\Omega,\mu)\to L_p(\Omega,\mu)$ ограничен 
    тогда и только тогда, когда $g\in L_\infty(\Omega,\mu)$;

    \item оператор $M_g$ --- топологический изоморфизм тогда и только 
    тогда, когда $c\leq |g|\leq C$ для некоторых $C,c>0$.
\end{enumerate}
\end{proposition}
\begin{proof}
$(i)$ Допустим существует множество $E\in\Sigma$ положительной меры такое, что
$|g|_E|>\Vert M_g\Vert$. Тогда
$$
\Vert M_g(\chi_E)\Vert_{L_p(\Omega,\mu)}
=\Vert g\chi_E\Vert_{L_p(\Omega,\mu)}
>\Vert M_g\Vert\Vert\chi_E\Vert_{L_p(\Omega,\mu)}.
$$
Противоречие, значит, для всех множеств $E\in\Sigma$ положительной меры
выполнено $|g|_E|\leq \Vert M_g\Vert$, то есть $|g|\leq \Vert M_g\Vert$ и 
$g\in L_\infty(\Omega,\mu)$. Обратно, пусть $g\in L_\infty(\Omega,\mu)$, тогда
$|g|\leq C$ для некоторого $C>0$. Для произвольной функции 
$f\in L_p(\Omega,\mu)$ мы имеем
$$
\Vert M_g(f)\Vert_{L_p(\Omega,\mu)}
=\Vert g  f\Vert_{L_p(\Omega,\mu)}
\leq C\Vert f\Vert_{L_p(\Omega,\mu)}.
$$
Значит, $M_g\in\mathcal{B}(L_p(\Omega,\mu))$.

$(ii)$ Заметим, что $M_g^{-1}=M_{1/g}$ при условии, что функция $1/g$ корректно
определена. Оператор $M_g$ является топологическим изоморфизмом тогда и только
тогда, когда $M_g$ и $M_g^{-1}$ суть ограниченные операторы. Из предыдущего
пункта и равенства $M_g^{-1}=M_{1/g}$ мы видим, что это эквивалентно
ограниченности $g$ и $1/g$. Значит, $c\leq|g|\leq C$ для некоторых $C,c>0$.
\end{proof}

\begin{proposition}\label{EquivMultOp} Пусть $(\Omega,\Sigma,\mu)$ ---
$\sigma$-конечное атомическое пространство с мерой, $1\leq p,q\leq +\infty$ и
$g\in L_0(\Omega,\Sigma)$. Тогда оператор
$\widetilde{M}_{\widetilde{g}}:=\widetilde{I}_q
M_g\widetilde{I}_p^{-1}\in\mathcal{B}(\ell_p(\Lambda),\ell_q(\Lambda))$ является
оператором умножения на функцию
$\widetilde{g}:\Lambda\to\mathbb{C}:\lambda\mapsto
{\mu(\Omega_\lambda)}^{1/q-1/p-1}\int_{\Omega_\lambda}f(\omega)d\mu(\omega)$, 
где $ \{\Omega_\lambda:\lambda\in\Lambda \}$ --- не более чем счетное разложение
$\Omega$ на семейство непересекающихся атомов.
\end{proposition}
\begin{proof} Пусть $1\leq p,q\leq +\infty$. Для любого $x\in\ell_p(\Lambda)$ мы
имеем
$$
\widetilde{M}_{\widetilde{g}}(x)(\lambda)
=(\widetilde{I}_q((M_g\widetilde{I}_p^{-1})(x))(\lambda)
=J_q(M_g(\widetilde{I}_p^{-1}(x))|_{\Omega_\lambda})(1)
$$
$$
=J_q((g \widetilde{I}_p^{-1}(x))|_{\Omega_\lambda})(1)
={\mu(\Omega_\lambda)}^{1/q-1}\int_{\Omega_\lambda}(g|_{\Omega_\lambda} 
\widetilde{I}_p^{-1}(x)|_{\Omega_\lambda})(\omega)d\mu(\omega)
$$
$$
={\mu(\Omega_\lambda)}^{1/q-1}
\int_{\Omega_\lambda}(
    g {\mu(\Omega)}^{-1/p}x(\lambda)\chi_{\Omega_{\lambda}}
)(\omega)d\mu(\omega)
=x(\lambda){\mu(\Omega_\lambda)}^{1/q-1/p-1}
\int_{\Omega_\lambda} g(\omega)d\mu(\omega).
$$
Значит, $\widetilde{M}_{\widetilde{g}}$ --- оператор умножения, причем
$\widetilde{g}(\lambda)
={\mu(\Omega_\lambda)}^{1/q-1/p-1}\int_{\Omega_\lambda} g(\omega)d\mu(\omega)$.
\end{proof}

Так как $\widetilde{I}_p$ и $\widetilde{I}_q$ --- изометрические изоморфизмы, то
оператор $M_g$ топологически инъективен тогда и только тогда, когда оператор
$\widetilde{M}_{\widetilde{g}}$ топологически инъективен. 

\begin{proposition}\label{TopInjMultOpCharacOnPureAtomMeasSp} Пусть
$(\Omega,\Sigma,\mu)$ --- $\sigma$-конечное атомическое пространство с мерой,
$1\leq p,q\leq +\infty$ и $g\in L_0(\Omega,\Sigma)$. Тогда следующие условия
эквивалентны:

\begin{enumerate}[label = (\roman*)]
    \item $M_g\in\mathcal{B}(L_p(\Omega,\mu),L_q(\Omega,\mu))$ --- топологически
    инъективный оператор;

    \item $|g|\geq c$ для некоторого $c>0$, при этом если $p\neq q$, 
    то пространство $(\Omega,\Sigma,\mu)$ состоит из конечного числа атомов.
\end{enumerate}
\end{proposition}
\begin{proof}
$(i) \implies (ii)$ По предположению, оператор $M_g$ топологически инъективен,
тогда таков же и $\widetilde{M}_{\widetilde{g}}$, то есть
$\Vert\widetilde{M}_{\widetilde{g}}(x)\Vert_{\ell_q(\Lambda)}\geq c'\Vert
x\Vert_{\ell_p(\Lambda)}$ для некоторого $c'>0$ и всех $x\in\ell_p(\Lambda)$.
Через $ \{\Omega_\lambda:\lambda\in\Lambda \}$ мы обозначим не более, чем
счетное разбиение $\Omega$ на семейство непересекающихся атомов. Мы рассмотрим
два случая.

Пусть $p\neq q$. Допустим, что множество $\Lambda$ счетно. Если $p,q<+\infty$,
то мы приходим к противоречию так как по теореме Питта
[\cite{KalAlbTopicsBanSpTh}, предложение 2.1.6] не существует вложения
$\ell_p(\Lambda)$ в $\ell_q(\Lambda)$ для счетного $\Lambda$ и 
$1\leq p,q< +\infty$, $p\neq q$. 

Если $1\leq p<+\infty$ и $q=+\infty$, то рассмотрим произвольное конечное
множество $F\in\mathcal{P}_0(\Lambda)$. Тогда 
$$
\sup_{\lambda\in\Lambda}|\widetilde{g}(\lambda)|
\geq\max_{\lambda\in F}|\widetilde{g}(\lambda)|
=\left\Vert\widetilde{M}_{\widetilde{g}}\left(
    \sum_{\lambda\in F}\delta_\lambda
\right)\right\Vert_{\ell_\infty(\Lambda)}
\geq c'\left\Vert\sum_{\lambda\in F}\delta_\lambda\right\Vert_{\ell_p(\Lambda)}
=c'{\operatorname{Card}(F)}^{1/p}.
$$
Так как множество $\Lambda$ счетно, то
$\sup_{\lambda\in\Lambda}|\widetilde{g}(\lambda)|\geq
c'\sup_{F\in\mathcal{P}_0(\Lambda)}{\operatorname{Card}(F)}^{1/p}=+\infty$. С
другой стороны, поскольку $\widetilde{M}_{\widetilde{g}}$ --- ограниченный
оператор, мы получаем, что 
$$
\sup_{\lambda\in\Lambda}|\widetilde{g}(\lambda)|
=\sup_{\lambda\in\Lambda}\Vert
    \widetilde{M}_{\widetilde{g}}(\delta_\lambda)
\Vert_{\ell_\infty(\Lambda)}
\leq\Vert
    \widetilde{M}_{\widetilde{g}}\Vert\Vert \delta_\lambda
\Vert_{\ell_p(\Lambda)}
=\Vert\widetilde{M}_{\widetilde{g}}\Vert<+\infty.
$$
Противоречие.

Если $1\leq q<+\infty$ и $p=+\infty$ то мы снова приходим к противоречию.
Действительно, так как множество $\Lambda$ счетно, то пространство
$\ell_\infty(\Lambda)$ несепарабельно, а $\ell_q(\Lambda)$ сепарабельно.
Поскольку оператор $\widetilde{M}_{\widetilde{g}}$ топологически инъективен, то
$\operatorname{Im}(\widetilde{M}_{\widetilde{g}})$ --- несепарабельное
подпространство в $\ell_q(\Lambda)$. Противоречие.

Во всех случаях мы пришли к противоречию, значит, множество $\Lambda$ конечно и
пространство $(\Omega,\Sigma,\mu)$ состоит из конечного числа атомов. Очевидно,
функция $g$ однозначно определяется своими значениями $k_\lambda\in\mathbb{C}$
на атомах $ \{\Omega_\lambda:\lambda\in\Lambda \}$. По
предложению~\ref{MultpOpSurjInjDesc} функция $g$ равна нулю только на множествах
меры $0$, поэтому $k_\lambda\neq 0$ для всех $\lambda\in\Lambda$. Поскольку
множество $\Lambda$ конечно, то $|g|\geq c$ для
$c=\min_{\lambda\in\Lambda}|k_\lambda|>0$. 

Пусть $p=q$. Для любого $\lambda\in\Lambda$ мы имеем
$$
|\widetilde{g}(\lambda)|
=\Vert \widetilde{g} \delta_\lambda\Vert_{\ell_q(\Lambda)}
=\Vert \widetilde{M}_{\widetilde{g}}(\delta_\lambda)\Vert_{\ell_q(\Lambda)}
\geq c'\Vert \delta_\lambda\Vert_{\ell_p(\Lambda)}
=c'.
$$
Поэтому для всех $\omega\in\Omega_\lambda$ выполнено
$$
|g(\omega)|
=\left|{\mu(\Omega_\lambda)}^{-1}
\int_{\Omega_\lambda}g(\omega)d\mu(\omega)\right|
=\left|{\mu(\Omega_\lambda)}^{-1}{\mu(\Omega_\lambda)}^{1+1/p-1/p}
\widetilde{g}(\lambda)\right|
=|\widetilde{g}(\lambda)|\geq c'.
$$
Так как $\Omega=\bigcup_{\lambda\in\Lambda}\Omega_\lambda$, то в итоге мы
получаем, что $|g|\geq c'$.

$(ii) \implies i)$ Допустим $|g|\geq c$ для $c>0$. Тогда из
предложения~\ref{EquivMultOp} следует, что $|\widetilde{g}|\geq c$. Если 
$p\neq q$, то мы дополнительно предполагаем, 
что пространство $(\Omega,\Sigma,\mu)$ состоит из конечного числа атомов. 
В этом случае пространство $L_p(\Omega,\mu)$
конечномерно. По предположению функция $g$ не принимает нулевых значений,
поэтому оператор $M_g$ топологически инъективен. Если же $p=q$, то для всех
$x\in\ell_p(\Lambda)$ мы имеем
$$
\Vert \widetilde{M}_{\widetilde{g}}(x)\Vert_{\ell_p(\Lambda)}
=\Vert g x\Vert_{\ell_p(\Lambda)}\geq c\Vert x\Vert_{\ell_p(\Lambda)},
$$
поэтому оператор $\widetilde{M}_{\widetilde{g}}$ топологически инъективен, а
вместе с ним и $M_g$.
\end{proof}

\begin{proposition}\label{TopInjMultOpCharacOnNonAtomMeasSp} Пусть
$(\Omega,\Sigma,\mu)$ --- неатомическое пространство с мерой, 
$1\leq p,q\leq +\infty$ и $g\in L_0(\Omega,\Sigma)$. 
Тогда следующие условия эквивалентны:

\begin{enumerate}[label = (\roman*)]
    \item $M_g\in\mathcal{B}(L_p(\Omega,\mu),L_q(\Omega,\mu))$ --- топологически
    инъективный оператор;

    \item $|g|\geq c$ для некоторого $c>0$, и $p=q$.
\end{enumerate}
\end{proposition}
\begin{proof}
$(i) \implies (ii)$ Согласно условию 
$\Vert M_g(f)\Vert_{L_q(\Omega,\mu)}\geq c\Vert f\Vert_{L_p(\Omega,\mu)}$ 
для некоторого $c>0$ и всех функций $f\in L_p(\Omega,\mu)$. 
Мы рассмотрим три случая.

Пусть $p>q$. Существуют константа $C>0$ и множество $E\in\Sigma$ положительной
меры такие, что $|g|_E|\leq C$, иначе $M_g$ не определен корректно. Так как
$(\Omega,\Sigma,\mu)$ --- неатомическое пространство с мерой, то существует
последовательность $ \{E_n:n\in\mathbb{N} \}\subset\Sigma$ подмножеств $E$
такая, что $\mu(E_n)=2^{-n}$. Поскольку $p>q$, мы получаем
$$
c
\leq\frac{
    \Vert M_g(\chi_{E_n})\Vert_{L_q(\Omega,\mu)}
}{
    \Vert \chi_{E_n}\Vert_{L_p(\Omega,\mu)}
}
\leq\frac{
    C\Vert\chi_{E_n}\Vert_{L_q(\Omega,\mu)}
}{
    \Vert \chi_{E_n}\Vert_{L_p(\Omega,\mu)}
}
\leq C{\mu(E_n)}^{1/q-1/p},
$$
$$
c
\leq\inf_{n\in\mathbb{N}}C{\mu(E_n)}^{1/q-1/p}
=C\inf_{n\in\mathbb{N}} 2^{n(1/p-1/q)}=0.
$$
Противоречие.

Пусть $p=q$. Зафиксируем число $c'<c$. Допустим, найдется множество 
$E\in\Sigma$ положительной меры такое, что $|g|_{E}|<c'$, тогда
$$
\Vert M_g(\chi_{E})\Vert_{L_p(\Omega,\mu)}
=\Vert g \chi_{E}\Vert_{L_p(\Omega,\mu)}
\leq c' \Vert \chi_{E}\Vert_{L_p(\Omega,\mu)}
<c\Vert \chi_{E}\Vert_{L_p(\Omega,\mu)}.
$$
Противоречие. Так как $c'<c$ произвольно, то мы заключаем, что $|g|_E|\geq c$
для всех множеств $E\in\Sigma$ положительной меры. Следовательно, $|g|\geq c$.

Теперь пусть $p<q$. Допустим, нашлись число $c'>0$ и множество $E\in\Sigma$
положительной меры такие, что $|g|_E|>c'$. Снова рассмотрим последовательность 
$\{E_n:n\in\mathbb{N} \}\subset\Sigma$ подмножеств в $E$ такую, что
$\mu(E_n)=2^{-n}$. Из неравенства $p<q$ следует, что
$$
\Vert M_g\Vert
\geq\frac{
    \Vert M_g(\chi_{E_n})\Vert_{L_q(\Omega,\mu)}
}{
    \Vert \chi_{E_n}\Vert_{L_p(\Omega,\mu)}
}
\geq\frac{
    c'\Vert\chi_{E_n}\Vert_{L_q(\Omega,\mu)}
}{
    \Vert \chi_{E_n}\Vert_{L_p(\Omega,\mu)}
}
\geq c'{\mu(E_n)}^{1/q-1/p}
$$
$$
\Vert M_g\Vert
\geq\sup_{n\in\mathbb{N}}c'{\mu(E_n)}^{1/q-1/p}
\geq c'\sup_{n\in\mathbb{N}}2^{n(1/p-1/q)}
=+\infty.
$$
Противоречие, значит, $g=0$. В этом случае по
предложению~\ref{MultpOpSurjInjDesc} оператор $M_g$ не топологически инъективен.

$(ii) \implies (i)$ Обратно, допустим $|g|\geq c>0$ и пусть $p=q$. Тогда для
любой функции $f\in L_p(\Omega,\mu)$ выполнено
$$
\Vert M_g(f)\Vert_{L_p(\Omega,\mu)}
=\Vert g f\Vert_{L_p(\Omega,\mu)}
\geq c\Vert f\Vert_{L_p(\Omega,\mu)}.
$$
Значит, оператор $M_g$ топологически инъективен.
\end{proof}

\begin{proposition}\label{TopInjMultOpCharacOnMeasSp} Пусть
$(\Omega,\Sigma,\mu)$ --- $\sigma$-конечное пространство с мерой, 
$1\leq p,q\leq +\infty$ и $g\in L_0(\Omega,\Sigma)$. 
Тогда следующие условия эквивалентны:

\begin{enumerate}[label = (\roman*)]
    \item $M_g\in\mathcal{B}(L_p(\Omega,\mu),L_q(\Omega,\mu))$ --- топологически
    инъективный оператор;

    \item $M_g$ --- топологический изоморфизм;

    \item $|g|\geq c$ для некоторого $c>0$, при этом если $p\neq q$ 
    то пространство $(\Omega,\Sigma,\mu)$ состоит из конечного числа атомов.
\end{enumerate}
\end{proposition}
\begin{proof} $(i) \Longleftrightarrow (iii)$ Рассмотрим разложение
$\Omega=\Omega_a^{\mu}\cup\Omega_{na}^{\mu}$, где
$(\Omega_{na}^{\mu},\Sigma|_{\Omega_{na}^{\mu}},\mu|_{\Omega_{na}^{\mu}})$ ---
неатомическое пространство с мерой и
$(\Omega_a^{\mu},\Sigma|_{\Omega_a^{\mu}},\mu|_{\Omega_a^{\mu}})$ ---
атомическое пространство с мерой. По предложению~\ref{MultOpDecompDecomp}
оператор $M_g$ топологически инъективен тогда и только тогда, когда таковы
$M_g^{\Omega_a^{\mu}}$ и $M_g^{\Omega_{na}^{\mu}}$.
Предложения~\ref{TopInjMultOpCharacOnPureAtomMeasSp},
~\ref{TopInjMultOpCharacOnNonAtomMeasSp}
дают для этого необходимые и достаточные условия.

$(i) \implies (ii)$ Допустим, оператор $M_g$ топологически инъективен. Если
$p=q$, то из рассуждений выше следует, что $|g|\geq c$ для некоторого $c>0$.
Поскольку оператор $M_g$ ограничен, то из
предложения~\ref{MultpOpPropIfPeqqualsQ} мы также получаем, что $C\geq |g|$ для
некоторого $C>0$. Теперь из того же самого предложения мы заключаем, что $M_g$
есть топологический изоморфизм. Если $p\neq q$, то из предыдущего пункта
известно, что пространство $(\Omega,\Sigma,\mu)$ состоит из конечного числа
атомов и функция $g$ не равна нулю. Следовательно,
$\operatorname{dim}(L_p(\Omega,\Sigma,\mu))
=\operatorname{dim}(\ell_p(\Lambda))=\operatorname{Card}(\Lambda)<+\infty$.
Аналогично,
$\operatorname{dim}(L_q(\Omega,\Sigma,\mu))
=\operatorname{Card}(\Lambda)<+\infty$.
Так как функция $g$ не равна нулю, то по предложению~\ref{MultpOpSurjInjDesc}
оператор $M_g$ инъективен. Итак, $M_g$ --- инъективный оператор между
конечномерными пространствами равной размерности. Следовательно, он является
топологическим изоморфизмом.

$(ii) \implies (i)$ Обратно, если $M_g$ --- топологический изоморфизм, то,
очевидно, он топологически инъективен.
\end{proof}

\begin{proposition}\label{TopInjMultOpCharacBtwnTwoContMeasSp} Пусть
$(\Omega,\Sigma,\mu)$ --- $\sigma$-конечное пространство с мерой, $1\leq
p,q\leq+\infty$. Допустим, $g,\rho\in L_0(\Omega,\Sigma)$ и функция $\rho$
неотрицательна. Тогда следующие условия эквивалентны:

\begin{enumerate}[label = (\roman*)]
    \item $M_g\in\mathcal{B}(L_p(\Omega,\mu),L_q(\Omega,\rho\mu))$ --- топологически
    инъективный оператор;

    \item $M_g$ --- топологический изоморфизм;

    \item функция $\rho$ положительна, $|g \rho^{1/q}|\geq c$ для некоторого 
    $c>0$, при этом если $p\neq q$, 
    то пространство $(\Omega,\Sigma,\mu)$ состоит из конечного числа атомов.
\end{enumerate}
\end{proposition}
\begin{proof} $(i) \implies (iii)$ Рассмотрим множество $E=\rho^{-1}( \{0 \})$.
Предположим, что $\mu(E)>0$, тогда $\chi_E\neq 0$ в $L_p(\Omega,\mu)$. С другой
стороны, $(\rho\mu)(E)=\int_E\rho(\omega)d\mu(\omega)=0$, поэтому $\chi_E=0$ в
$L_q(\Omega,\rho\mu)$ и $M_g(\chi_E)=g\chi_E=0$ в $L_q(\Omega,\rho \mu)$. Таким
образом, мы получили, что оператор $M_g$ не инъективен и, как следствие, не
топологически инъективен. Противоречие, поэтому $\mu(E)=0$ и функция $\rho$
положительна. Следовательно, корректно определен изометрический изоморфизм
$\bar{I}_q:L_q(\Omega,\mu)\to L_q(\Omega,\rho\mu):f\mapsto \rho^{-1/q} f$.
Очевидно, $M_{g\rho^{1/q}}=\bar{I}_q^{-1}
M_g\in\mathcal{B}(L_p(\Omega,\mu),L_q(\Omega,\mu))$. Поскольку $\bar{I}_q$ ---
изометрический изоморфизм и оператор $M_g$ топологически инъективен, то 
$M_{g \rho^{1/q}}$ также топологически инъективен. Из
предложения~\ref{TopInjMultOpCharacOnMeasSp} мы получаем, что 
$|g\rho^{1/q}|\geq c$ для некоторого $c>0$ и если $p\neq q$, то пространство 
$(\Omega,\Sigma,\mu)$ состоит из конечного числа атомов.

$(iii) \implies (i)$ Из предложения~\ref{TopInjMultOpCharacOnMeasSp} следует,
что оператор $M_{g \rho^{1/q}}$ топологически инъективен. Так как функция $\rho$
положительна, то мы имеем изометрический изоморфизм $\bar{I}_q$. Теперь из
равенства $M_g=\bar{I}_q M_{g \rho^{1/q}}$ следует, что оператор $M_g$ также
топологически инъективен.

$(i) \implies (ii)$ Как мы доказали ранее оператор $M_{g \rho^{1/q}}$
топологически инъективен и $\bar{I}_q$ --- изометрический изоморфизм. По
предложению~\ref{TopInjMultOpCharacOnMeasSp} оператор $M_{g \rho^{1/q}}$
является топологическим изоморфизмом. Так как $M_g=\bar{I}_q M_{g \rho^{1/q}}$ и
$\bar{I}_q$ есть изометрический изоморфизм, то $M_g$  является топологическим
изоморфизмом.

$(ii) \implies (i)$ Если $M_g$ --- топологический изоморфизм, тогда, очевидно,
он топологически инъективен.
\end{proof}
\begin{proposition}\label{TopInjMultOpCharacBtwnTwoMeasSp} Пусть
$(\Omega,\Sigma,\mu)$, $(\Omega,\Sigma,\nu)$ --- два $\sigma$-конечных
пространства с мерой, $1\leq p,q\leq +\infty$ и $g\in L_0(\Omega,\Sigma)$. Тогда
следующие условия эквивалентны:

\begin{enumerate}[label = (\roman*)]
    \item $M_g\in\mathcal{B}(L_p(\Omega,\mu), L_q(\Omega,\nu))$ --- топологически
    инъективный оператор;

    \item $M_g^{\Omega_c^{\nu,\mu}}$ --- топологический изоморфизм;

    \item функция $\rho_{\nu,\mu}|_{\Omega_c^{\nu,\mu}}$ положительна, $|g
    \rho_{\nu,\mu}^{1/q}|_{\Omega_c^{\nu,\mu}}|\geq c$ для некоторого $c>0$, 
    при этом если $p\neq q$, то пространство $(\Omega,\Sigma,\mu)$ 
    состоит из конечного числа атомов.
\end{enumerate}
\end{proposition}
\begin{proof}
По предложению~\ref{MultOpDecompDecomp} оператор $M_g$ топологически инъективен
тогда и только тогда, когда операторы
$M_g^{\Omega_c^{\nu,\mu}}:L_p(\Omega_c^{\nu,\mu},\mu|_{\Omega_c^{\nu,\mu}})\to
L_q(\Omega_c^{\nu,\mu},\rho_{\nu,\mu} \mu|_{\Omega_c^{\nu,\mu}})$ и
$M_g^{\Omega_s^{\nu,\mu}}:L_p(\Omega_s^{\nu,\mu},\mu|_{\Omega_s^{\nu,\mu}})\to
L_q(\Omega_s^{\nu,\mu},\nu_s|_{\Omega_s^{\nu,\mu}})$  топологически инъективны.
По предложению~\ref{MultOpCharacBtwnTwoSingMeasSp} оператор
$M_g^{\Omega_s^{\nu,\mu}}$ нулевой. Так как $\mu(\Omega_s^{\nu,\mu})=0$, то
$L_p(\Omega_s^{\nu,\mu},\mu|_{\Omega_s^{\nu,\mu}})= \{0 \}$. Отсюда мы
заключаем, что оператор $M_g^{\Omega_s^{\nu,\mu}}$ топологически инъективен.
Значит, топологическая инъективность $M_g$ эквивалентна топологической
инъективности  $M_g^{\Omega_c^{\nu,\mu}}$. Остается применить
предложение~\ref{TopInjMultOpCharacBtwnTwoContMeasSp}.
\end{proof}

\begin{proposition}\label{TopInjMultOpDescBtwnTwoMeasSp} Пусть
$(\Omega,\Sigma,\mu)$, $(\Omega,\Sigma,\nu)$ --- два $\sigma$-конечных
пространства с мерой, $1\leq p,q\leq +\infty$ и $g\in L_0(\Omega,\Sigma)$. Тогда
следующие условия эквивалентны:

\begin{enumerate}[label = (\roman*)]
    \item $M_g\in\mathcal{B}(L_p(\Omega,\mu),L_q(\Omega,\nu))$ --- топологически
    инъективный оператор;

    \item $M_{\chi_{\Omega_c^{\nu,\mu}}/g}\in\mathcal{B}(L_q(\Omega,\nu),
    L_p(\Omega,\mu))$ --- топологически сюръективный левый обратный оператор к
    $M_g$.
\end{enumerate}
\end{proposition}
\begin{proof}
$(i) \implies (ii)$ По предложению~\ref{MultOpDecompDecomp} оператор
$M_g^{\Omega_c^{\nu,\mu}}$ топологически инъективен. По
предложению~\ref{TopInjMultOpCharacBtwnTwoContMeasSp} оператор
$M_g^{\Omega_c^{\nu,\mu}}$ обратим и
${(M_g^{\Omega_c^{\nu,\mu}})}^{-1}=M_{1/g}^{\Omega_c^{\nu,\mu}}$. Тогда для любой
функции $h\in L_q(\Omega,\nu)$ выполнено
$$
\Vert M_{\chi_{\Omega_c^{\nu,\mu}}/g}(h)\Vert_{L_p(\Omega,\mu)}=
\Vert M_{1/g}(h)\chi_{\Omega_c^{\nu,\mu}}\Vert_{L_p(\Omega,\mu)}=
\Vert 
    M_{1/g}^{\Omega_c^{\nu,\mu}}(h|_{\Omega_c^{\nu,\mu}})
\Vert_{L_p(\Omega_c^{\nu,\mu},\mu|_{\Omega_c^{\nu,\mu}})}
$$
$$
\leq\Vert 
    M_{1/g}^{\Omega_c^{\nu,\mu}}
\Vert
\Vert 
    h|_{\Omega_c^{\nu,\mu}}
\Vert_{L_q(\Omega_c^{\nu,\mu},\nu|_{\Omega_c^{\nu,\mu}})}
\leq\Vert M_{1/g}^{\Omega_c^{\nu,\mu}}\Vert\Vert h\Vert_{L_q(\Omega,\nu)}.
$$ 
Поэтому оператор $M_{\chi_{\Omega_c^{\nu,\mu}}/g}$ ограничен. Теперь заметим,
что для любой функции $f\in L_p(\Omega,\mu)$ выполнено
$$
M_{\chi_{\Omega_c^{\nu,\mu}}/g}(M_g(f))
=M_{\chi_{\Omega_c^{\nu,\mu}}/g}(g  f)
=(\chi_{\Omega_c^{\nu,\mu}}/g)  g  f
=f \chi_{\Omega_c^{\nu,\mu}}.
$$
Так как $\mu(\Omega\setminus\Omega_c^{\nu,\mu})=0$, то
$\chi_{\Omega_c^{\nu,\mu}}=\chi_{\Omega}$, поэтому
$M_{\chi_{\Omega_c^{\nu,\mu}}/g}(M_g(f))
=f \chi_{\Omega_c^{\nu,\mu}}
=f\chi_{\Omega}=f$. Это означает, 
что $M_{\chi_{\Omega_c^{\nu,\mu}}/g}$ --- левый обратный оператор умножения 
к $M_g$. Рассмотрим произвольную функцию $f\in L_p(\Omega,\mu)$, 
тогда для $h=M_g(f)$ выполнено $M_{\chi_{\Omega_c^{\nu,\mu}}/g}(h)=f$ и 
$\Vert h\Vert_{L_q(\Omega,\nu)}
\leq\Vert M_g\Vert\Vert f\Vert_{L_p(\Omega,\mu)}$. 
Так как функция $f$ произвольна, то оператор $M_{\chi_{\Omega_c^{\nu,\mu}}/g}$
топологически сюръективен.

Обратно, если $M_g$ имеет левый обратный оператор
$M_{\chi_{\Omega_c^{\nu,\mu}}/g}$, то для всех $f\in L_p(\Omega,\mu)$ выполнено
$$
\Vert M_g(f)\Vert_{L_q(\Omega,\nu)}
\geq\Vert 
    M_{\chi_{\Omega_c^{\nu,\mu}}/g}
\Vert^{-1}
\Vert 
    M_{\chi_{\Omega_c^{\nu,\mu}}/g}(M_g(f))
\Vert_{L_p(\Omega,\mu)}
\geq\Vert 
    M_{\chi_{\Omega_c^{\nu,\mu}}/g}
\Vert^{-1}
\Vert f\Vert_{L_p(\Omega,\mu)}.
$$
Следовательно, оператор $M_g$ топологически инъективен.
\end{proof}

\begin{proposition}\label{IsomMultOpCharacOnMeasSp} Пусть $(\Omega,\Sigma,\mu)$
--- $\sigma$-конечное пространство с мерой, $1\leq p,q\leq +\infty$ и 
$g\in L_0(\Omega,\Sigma)$. Тогда следующие условия эквивалентны:

\begin{enumerate}[label = (\roman*)]
    \item $M_g\in\mathcal{B}(L_p(\Omega,\mu),L_q(\Omega,\mu))$ --- 
    изометрический оператор;

    \item $|g|={\mu(\Omega)}^{1/p-1/q}$, при этом если $p\neq q$, 
    то пространство $(\Omega,\Sigma,\mu)$ состоит из одного атома.
\end{enumerate}
\end{proposition}
\begin{proof} $(i) \implies (ii)$ Пусть $p=q$. Допустим существует множество
$E\in\Sigma$ положительной меры такое, что $|g|_E|<1$, тогда
$$
\Vert M_g(\chi_E)\Vert_{L_p(\Omega,\mu)}
=\Vert g \chi_E\Vert_{L_p(\Omega,\mu)}
<\Vert\chi_E\Vert_{L_p(\Omega,\mu)}
=\Vert M_g(\chi_E)\Vert_{L_p(\Omega,\mu)}.
$$
Противоречие, значит для любого множества  $E\in\Sigma$ положительной меры
выполнено $|g|_E|\geq 1$, то есть $|g|\geq 1$. Аналогично доказывается, что
$|g|\leq 1$. Следовательно, $|g|=1={\mu(\Omega)}^{1/p-1/q}$. Пусть $p\neq q$,
тогда так как оператор $M_g$ --- изометрия, то он топологически инъективен. По
предложению~\ref{TopInjMultOpCharacOnMeasSp} пространство $(\Omega,\Sigma,\mu)$
состоит из конечного числа атомов. Допустим, что имеется хотя бы два атома,
назовем их $\Omega_1$ и $\Omega_2$. Они оба конечной меры, поэтому мы можем
рассмотреть функции
$h_k=\Vert\chi_{\Omega_k}\Vert_{L_p(\Omega,\mu)}^{-1}\chi_{\Omega_k}$ для
$k\in\mathbb{N}_2$. Так как эти атомы не пересекаются, то $h_1h_2=0$ и как
результат $M_g(h_1)M_g(h_2)=0$. Заметим, что для любого $1\leq r\leq +\infty$ и
любых функций $f_1,f_2\in L_r(\Omega,\mu)$ со свойством $f_1f_2=0$ верно
$$
\Vert f_1+f_2\Vert_{L_r(\Omega,\mu)}
=\left\Vert\left(\Vert f_\lambda\Vert_{L_r(\Omega,\mu)}
:\lambda\in\mathbb{N}_2\right)\right\Vert_{\ell_r(\mathbb{N}_2)}.
$$
Значит,
$$
\Vert M_g(h_1+h_2)\Vert_{L_q(\Omega,\mu)}
=\Vert h_1+h_2\Vert_{L_p(\Omega,\mu)}
=\left\Vert
    \left( 1 :\lambda\in\mathbb{N}_2\right)
\right\Vert_{\ell_p(\mathbb{N}_2)}
=2^{1/p}.
$$
С другой стороны
$$
\Vert M_g(h_1+h_2)\Vert_{L_q(\Omega,\mu)}
=\Vert M_g(h_1)+M_g(h_2)\Vert_{L_q(\Omega,\mu)}
$$
$$
=\left\Vert
    \left(
        \Vert M_g(h_\lambda)\Vert_{L_q(\Omega,\mu)}:\lambda\in\mathbb{N}_2
    \right)
\right\Vert_{\ell_q(\mathbb{N}_2)}
=\left\Vert
    \left(
        \Vert h_\lambda\Vert_{L_p(\Omega,\mu)}:\lambda\in\mathbb{N}_2
    \right)
\right\Vert_{\ell_q(\mathbb{N}_2)}
=2^{1/q}.
$$
Поэтому $2^{1/p}=2^{1/q}$. Противоречие, значит, $(\Omega,\Sigma,\mu)$ состоит
из одного атома. В этом случае для любой функции $f\in L_p(\Omega,\mu)$
выполнено
$$
\Vert M_g(f)\Vert_{L_q(\Omega,\mu)}
=\Vert J_q(M_g(f))\Vert_{\ell_q(\mathbb{N}_1)}
=\Vert J_q(g  f)\Vert_{\ell_q(\mathbb{N}_1)}
={\mu(\Omega)}^{1/q-1}\left|\int_\Omega g(\omega) f(\omega)d\mu(\omega)\right|
$$
$$
\Vert f\Vert_{L_p(\Omega,\mu)}
=\Vert J_p(f)\Vert_{\ell_p(\mathbb{N}_1)}
={\mu(\Omega)}^{1/p-1}\left|\int_\Omega f(\omega)d\mu(\omega)\right|.
$$
Через $c$ мы обозначим константное значение функции $g$, тогда
$$
\Vert M_g(f)\Vert_{L_q(\Omega,\mu)}
={\mu(\Omega)}^{1/q-1}\left|\int_\Omega g(\omega) f(\omega)d\mu(\omega)\right|
={\mu(\Omega)}^{1/q-1}|c|\left|\int_\Omega f(\omega)d\mu(\omega)\right|.
$$
Из этого равенства следует, что оператор $M_g$ будет изометрией, если
$|g|=|c|={\mu(\Omega)}^{1/p-1/q}$.

$(ii) \implies (i)$. Пусть $p=q$, тогда $|g|=1$. Поэтому для произвольной
функции $f\in L_p(\Omega,\mu)$ выполнено
$$
\Vert M_g(f)\Vert_{L_p(\Omega,\mu)}
=\Vert g  f\Vert_{L_p(\Omega,\mu)}
=\Vert |g|  f\Vert_{L_p(\Omega,\mu)}
=\Vert f\Vert_{L_p(\Omega,\mu)},
$$
значит, оператор $M_g$ --- изометрия. Пусть $p\neq q$, тогда по предположению
$(\Omega,\Sigma,\mu)$ состоит из одного атома, и тогда
$$
\Vert M_g(f)\Vert_{L_q(\Omega,\mu)}
={\mu(\Omega)}^{1/q-1}\left|\int_\Omega g(\omega) f(\omega)d\mu(\omega)\right|
={\mu(\Omega)}^{1/q-1}|c|\left|\int_\Omega f(\omega)d\mu(\omega)\right|
$$
$$
={\mu(\Omega)}^{1/p-1}\left|\int_\Omega f(\omega)d\mu(\omega)\right|
=\Vert f\Vert_{L_p(\Omega,\mu)}.
$$
Следовательно, оператор $M_g$ изометричен.
\end{proof}

\begin{proposition}\label{IsomMultOpCharacBtwnTwoContMeasSp} Пусть
$(\Omega,\Sigma,\mu)$ --- $\sigma$-конечное пространство с мерой и $1\leq
p,q\leq +\infty$. Допустим, $g,\rho\in L_0(\Omega,\Sigma)$ и функция $\rho$
неотрицательна. Тогда следующие условия эквивалентны:

\begin{enumerate}[label = (\roman*)]
    \item оператор $M_g\in\mathcal{B}(L_p(\Omega,\mu), L_q(\Omega,\rho \mu))$
    изометричен;

    \item $M_g$ --- изометрический изоморфизм;

    \item функция $\rho$ положительна, 
    $|g  \rho^{1/q}|={\mu(\Omega)}^{1/p-1/q}$, при этом если $p\neq q$, 
    то пространство $(\Omega,\Sigma,\mu)$ состоит из одного атома.
\end{enumerate}
\end{proposition}
\begin{proof} $(i) \implies (iii)$ Так как оператор $M_g$ изометричен, то он
топологически инъективен и из предложения~\ref{TopInjMultOpCharacBtwnTwoMeasSp}
следует, что функция $\rho$ положительна. Таким образом, мы имеем изометрический
изоморфизм $\bar{I}_q:L_q(\Omega,\mu)\to L_q(\Omega,\rho \mu):f\mapsto
\rho^{-1/q}  f$. Очевидно, $M_{g \rho^{1/q}}=\bar{I}_q^{-1}
M_g\in\mathcal{B}(L_p(\Omega,\mu),L_q(\Omega,\mu))$. Так как $\bar{I}_q$ ---
изометрический изоморфизм и оператор $M_g$ изометричен, то таков же и 
$M_{g \rho^{1/q}}$. Осталось применить 
предложение~\ref{IsomMultOpCharacOnMeasSp}.

$(iii) \implies (i)$ По предложению~\ref{IsomMultOpCharacOnMeasSp} оператор
$M_{g \rho^{1/q}}$ изометричен. Так как функция $\rho$ положительна, то
корректно определен изометрически изоморфизм $\bar{I}_q$. Тогда из равенства
$M_g=\bar{I}_q M_{g \rho^{1/q}}$ следует, что оператор $M_g$ также изометричен.

$(i) \implies (ii)$ Поскольку оператор $M_g$ изометричен, он топологически
инъективен, и по предложению~\ref{TopInjMultOpCharacBtwnTwoContMeasSp} он
является изоморфизмом, причем, по предположению, изометрическим.

$(ii) \implies (i)$ Очевидно.
\end{proof}

\begin{proposition}\label{IsomMultOpCharacBtwnTwoMeasSp} Пусть
$(\Omega,\Sigma,\mu)$, $(\Omega,\Sigma,\nu)$ --- два $\sigma$-конечных
пространства с мерой, $1\leq p,q\leq +\infty$ и $g\in L_0(\Omega,\Sigma)$. Тогда
следующие условия эквивалентны:

\begin{enumerate}[label = (\roman*)]
    \item $M_g$ --- изометрический оператор;

    \item $M_g^{\Omega_c^{\nu,\mu}}$ --- изометрический оператор;

    \item функция $\rho_{\nu,\mu}|_{\Omega_c^{\nu,\mu}}$ положительна, 
    $|g \rho_{\nu,\mu}^{1/q}|_{\Omega_c^{\nu,\mu}}|
    ={\mu(\Omega_c^{\nu,\mu})}^{1/p-1/q}$, при этом если $p\neq q$ 
    то пространство $(\Omega,\Sigma,\mu)$ состоит из одного атома.
\end{enumerate}
\end{proposition}
\begin{proof} $(i) \implies (ii) \implies (iii)$ Так как оператор $M_g$
изометричен, то по предложению~\ref{MultOpDecompDecomp} оператор
$M_g^{\Omega_c^{\nu,\mu}}$ также изометричен. Осталось применить
предложение~\ref{IsomMultOpCharacBtwnTwoContMeasSp}.

$(iii) \implies (i)$ По предложению~\ref{IsomMultOpCharacBtwnTwoContMeasSp}
оператор $M_g^{\Omega_c^{\nu,\mu}}$ изометричен. Теперь рассмотрим произвольную
функцию $f\in L_p(\Omega,\mu)$. Так как
$\mu(\Omega\setminus\Omega_c^{\nu,\mu})=0$, то
$\chi_{\Omega_c^{\nu,\mu}}=\chi_{\Omega}$ в $L_p(\Omega,\mu)$. Как следствие,
$f=f\chi_{\Omega}=f\chi_{\Omega_c^{\nu,\mu}}
=f\chi_{\Omega_c^{\nu,\mu}}\chi_{\Omega_c^{\nu,\mu}}$
в $L_p(\Omega,\mu)$ и
$M_g(f)=M_g(f\chi_{\Omega_c^{\nu,\mu}})\chi_{\Omega_c^{\nu,\mu}}$. Учитывая, что
оператор $M_g^{\Omega_c^{\nu,\mu}}$ изометричен, мы получаем
$$
\Vert M_g(f)\Vert_{L_q(\Omega,\nu)}
=\Vert 
    M_g(f\chi_{\Omega_c^{\nu,\mu}})\chi_{\Omega_c^{\nu,\mu}}
\Vert_{L_q(\Omega,\nu)}
=\Vert 
    M_g(f\chi_{\Omega_c^{\nu,\mu}})
\Vert_{L_q(\Omega_c^{\nu,\mu},\nu|_{\Omega_c^{\nu,\mu}})}
$$
$$
=\Vert 
    M_g^{\Omega_c^{\nu,\mu}}(f|_{\Omega_c^{\nu,\mu}})
\Vert_{L_q(\Omega_c^{\nu,\mu},\nu|_{\Omega_c^{\nu,\mu}})}
=\Vert 
    f|_{\Omega_c^{\nu,\mu}}
\Vert_{L_p(\Omega_c^{\nu,\mu},\mu|_{\Omega_c^{\nu,\mu}})}.
$$
Поскольку $\mu(\Omega\setminus\Omega_c^{\nu,\mu})=0$, то 
$\Vert 
    f|_{\Omega_c^{\nu,\mu}}
\Vert_{L_p(\Omega_c^{\nu,\mu},\mu|_{\Omega_c^{\nu,\mu}})}
=\Vert f\Vert_{L_p(\Omega,\mu)}$, и поэтому 
$\Vert M_g(f)\Vert_{L_q(\Omega,\nu)}=\Vert f\Vert_{L_p(\Omega,\mu)}$. 
Значит, оператор $M_g$ изометричен.
\end{proof}

\begin{proposition}\label{IsomMultOpDescBtwnTwoMeasSp} Пусть
$(\Omega,\Sigma,\mu)$, $(\Omega,\Sigma,\nu)$ --- два $\sigma$-конечных
пространства с мерой, $1\leq p,q\leq +\infty$ и $g\in L_0(\Omega,\Sigma)$. Тогда
следующие условия эквивалентны:

\begin{enumerate}[label = (\roman*)]
    \item $M_g\in\mathcal{B}(L_p(\Omega,\mu),L_q(\Omega,\nu))$ --- 
    изометрический оператор;

    \item $M_{\chi_{\Omega_c^{\nu,\mu}}/g}\in\mathcal{B}(L_q(\Omega,\nu),
    L_p(\Omega,\mu))$ --- строго коизометрический левый 
    обратный оператор к $M_g$.
\end{enumerate}
\end{proposition}
\begin{proof} $(i) \implies (ii)$ По предложению~\ref{MultOpDecompDecomp}
оператор $M_g^{\Omega_c^{\nu,\mu}}$ изометричен, и по
предложению~\ref{IsomMultOpCharacBtwnTwoContMeasSp} он обратим, причем,
очевидно, что ${(M_g^{\Omega_c^{\nu,\mu}})}^{-1}=M_{1/g}^{\Omega_c^{\nu,\mu}}$.
Так как оператор $M_g^{\Omega_c^{\nu,\mu}}$ изометричен, то таков же и его левый
обратный. Тогда для любой функции $h\in L_q(\Omega,\nu)$ выполнено
$$
\Vert M_{\chi_{\Omega_c^{\nu,\mu}}/g}(h)\Vert_{L_p(\Omega,\mu)}
=\Vert 
    M_{1/g}(h|_{\Omega_c^{\nu,\mu}})
\Vert_{L_p(\Omega_c^{\nu,\mu},\mu|_{\Omega_c^{\nu,\mu}})}
=\Vert 
    M_{1/g}^{\Omega_c^{\nu,\mu}}(h|_{\Omega_c^{\nu,\mu}})
\Vert_{L_p(\Omega_c^{\nu,\mu},\mu|_{\Omega_c^{\nu,\mu}})}
$$
$$
=\Vert 
    h|_{\Omega_c^{\nu,\mu}}
\Vert_{L_q(\Omega_c^{\nu,\mu},\nu|_{\Omega_c^{\nu,\mu}})}
\leq \Vert h \Vert_{L_q(\Omega,\nu)}.
$$
Поэтому оператор $M_{\chi_{\Omega_c^{\nu,\mu}}/g}$ сжимающий. Далее для всех
функций $f\in L_p(\Omega,\mu)$ мы имеем
$$
M_{\chi_{\Omega_c^{\nu,\mu}}/g}(M_g(f))
=M_{\chi_{\Omega_c^{\nu,\mu}}/g}(g  f)
=(\chi_{\Omega_c^{\nu,\mu}}/g)  g  f
=f \chi_{\Omega_c^{\nu,\mu}}.
$$
Так как $\mu(\Omega\setminus\Omega_c^{\nu,\mu})=0$, то
$\chi_{\Omega_c^{\nu,\mu}}=\chi_{\Omega}$, и поэтому
$M_{\chi_{\Omega_c^{\nu,\mu}}/g}(M_g(f))
=f \chi_{\Omega_c^{\nu,\mu}}
=f\chi_{\Omega}=f$. Последнее означает, что оператор
$M_{\chi_{\Omega_c^{\nu,\mu}}/g}$ является левым обратным оператором умножения к
$M_g$. Рассмотрим произвольную функцию $f\in L_p(\Omega,\mu)$, тогда для
$h=M_g(f)$ выполнено $M_{\chi_{\Omega_c^{\nu,\mu}}/g}(h)=f$ и 
$\Vert h\Vert_{L_q(\Omega,\nu)}\leq\Vert f\Vert_{L_p(\Omega,\mu)}$. 
Следовательно, $M_{\chi_{\Omega_c^{\nu,\mu}}}/g$ есть 
строго $1$-топологически сюръективный оператор, но он еще и сжимающий, 
а, значит, строго коизометрический.

$(ii) \implies (i)$ Рассмотрим произвольную функцию $f\in L_p(\Omega,\mu)$,
тогда существует функция $h\in L_q(\Omega,\nu)$ такая, что
$M_{\chi_{\Omega_c^{\nu,\mu}}/g}(h)=f$ и 
$\Vert h\Vert_{L_q(\Omega,\nu)}\leq \Vert f\Vert_{L_p(\Omega,\mu)}$. 
Следовательно,
$$
\Vert M_g(f)\Vert_{L_q(\Omega,\nu)}
=\Vert M_g(M_{\chi_{\Omega_c^{\nu,\mu}}/g}(h))\Vert_{L_q(\Omega,\nu)}
=\Vert \chi_{\Omega_c^{\nu,\mu}}h\Vert_{L_q(\Omega,\nu)}
\leq\Vert h\Vert_{L_q(\Omega,\nu|)}
\leq\Vert f\Vert_{L_p(\Omega,\mu)}.
$$
Поскольку оператор $M_{\chi_{\Omega_c^{\nu,\mu}}/g}$ сжимающий и левый обратный
к $M_g$, то
$$
\Vert f\Vert_{L_p(\Omega,\mu)}
=\Vert M_{\chi_{\Omega_c^{\nu,\mu}}/g}(M_g(f))\Vert_{L_p(\Omega,\mu)}
\leq\Vert M_g(f)\Vert_{L_q(\Omega,\nu)}
$$
и, значит, $\Vert M_g(f)\Vert_{L_q(\Omega,\nu)}=\Vert f\Vert_{L_p(\Omega,\mu)}$.
Так как функция $f$ произвольна, то оператор $M_g$ изометричен.
\end{proof}

Теперь мы обсудим топологически сюръективные и коизометрические операторы
умножения. Их описание получить проще и большинство доказательств схожи (но не
идентичны) с рассуждениями для топологически инъективных и изометрических
операторов.

\begin{proposition}\label{TopSurMultOpCharacOnMeasSp} Пусть
$(\Omega,\Sigma,\mu)$ --- $\sigma$-конечное пространство с мерой, 
$1\leq p,q\leq +\infty$ и $g\in L_0(\Omega,\Sigma)$. 
Тогда следующие условия эквивалентны:

\begin{enumerate}[label = (\roman*)]
    \item $M_g\in\mathcal{B}(L_p(\Omega,\mu),L_q(\Omega,\mu))$ --- топологически
    сюръективный оператор;

    \item $M_g$ --- топологический изоморфизм;

    \item $|g|\geq c$ для некоторого $c>0$, при этом если $p\neq q$, то
    пространство $(\Omega,\Sigma,\mu)$ состоит из конечного числа атомов.
\end{enumerate}
\end{proposition}
\begin{proof} $(i) \implies (ii)$ Так как оператор $M_g$ топологически
сюръективен, то он сюръективен и по предложению~\ref{MultpOpSurjInjDesc} он
инъективен. Следовательно, оператор $M_g$ биективен. Так как $L_p$ пространства
полны, то из теоремы об открытом отображении мы получаем, что $M_g$ ---
топологический изоморфизм. 

$(ii) \implies (i)$ Если $M_g$ --- топологический изоморфизм, то он, очевидно,
топологически сюръективен.

$(ii) \Longleftrightarrow (iii)$ Эквивалентность следует из
предложения~\ref{TopInjMultOpCharacOnMeasSp}.
\end{proof}
 
\begin{proposition}\label{TopSurMultOpCharacBtwnTwoContMeasSp} Пусть
$(\Omega,\Sigma,\nu)$ --- $\sigma$-конечное пространство с мерой, 
$1\leq p,q\leq +\infty$ и $g,\rho\in L_0(\Omega,\Sigma)$ причем функция 
$\rho$  неотрицательна. Тогда следующие условия эквивалентны:

\begin{enumerate}[label = (\roman*)]
    \item $M_g\in\mathcal{B}(L_p(\Omega,\rho \nu),L_q(\Omega,\nu))$ --- 
    топологически сюръективный оператор;

    \item $M_g$ --- топологический изоморфизм;

    \item функция $\rho$ положительна, $|g  \rho^{-1/p}|\geq c$ для некоторого
    $c>0$, при этом если $p\neq q$, то пространство $(\Omega,\Sigma,\mu)$ 
    состоит из конечного числа атомов.
\end{enumerate}
\end{proposition}
\begin{proof} $(i) \implies (iii)$ Рассмотрим множество $E=\rho^{-1}( \{0 \})$.
Допустим, $\nu(E)>0$, тогда $\chi_E\neq 0$ в $L_q(\Omega,\nu)$. С другой
стороны, $(\rho \nu)(E)=\int_E\rho(\omega)d\nu(\omega)=0$, поэтому $\chi_E=0$ в
$L_p(\Omega,\rho \nu)$. Тогда для всех функций $f\in L_p(\Omega,\rho \nu)$
выполнено $M_g(f)\chi_E=M_g(f \chi_E)=M_g(0)=0$ в $L_q(\Omega,\nu)$. Последнее
равенство означает, что 
$\operatorname{Im}(M_g)\subset \{h\in L_q(\Omega,\nu): h|_E=0 \}$. 
Поскольку $\nu(E)\neq 0$, оператор $M_g$ не сюръективен и, как
следствие, не является топологически сюръективным. Противоречие, значит,
$\nu(E)=0$ и $\rho$ --- положительная функция. Значит, корректно определен
изометрический изоморфизм 
$\bar{I}_p:L_p(\Omega,\nu)\to L_p(\Omega,\rho \nu):f\mapsto \rho^{-1/p}  f$. 
Очевидно, 
$M_{g \rho^{-1/p}}
=M_g \bar{I}_p\in\mathcal{B}(L_p(\Omega,\nu),L_q(\Omega,\nu))$. 
Так как $\bar{I}_p$ --- изометрический изоморфизм и оператор $M_g$ 
топологически сюръективен, то таков же и $M_{g  \rho^{-1/p}}$. 
Осталось применить предложение~\ref{TopSurMultOpCharacOnMeasSp}.

$(iii) \implies (i)$ По предложению~\ref{TopSurMultOpCharacOnMeasSp} оператор
$M_{g \rho^{-1/p}}$ топологически сюръективен. Так как функция $\rho$
положительна, то корректно определен изометрический изоморфизм $\bar{I}_p$. Из
равенства $M_g= M_{g \rho^{-1/p}}\bar{I}_p^{-1}$ следует, что оператор $M_g$
также топологически сюръективен.

$(i) \implies (ii)$ Как мы доказали выше, оператор $M_{g \rho^{1/q}}$
топологически инъективен и $\bar{I}_q$ --- изометрический изоморфизм. Тогда по
предложению~\ref{TopSurMultOpCharacOnMeasSp} оператор $M_{g \rho^{1/q}}$
является топологическим изоморфизмом. Так как $M_g=\bar{I}_q M_{g \rho^{1/q}}$
то оператор $M_g$ тоже является топологическим изоморфизмом.

$(ii) \implies (i)$. Если $M_g$ --- топологический изоморфизм, то он, очевидно,
топологически сюръективен.
\end{proof}

\begin{proposition}\label{TopSurMultOpCharacBtwnTwoMeasSp} Пусть
$(\Omega,\Sigma,\mu)$, $(\Omega,\Sigma,\nu)$ --- два $\sigma$-конечных
пространства с мерой, $1\leq p,q\leq +\infty$ и $g\in L_0(\Omega,\Sigma)$. Тогда
следующие условия эквивалентны:

\begin{enumerate}[label = (\roman*)]
    \item $M_g\in\mathcal{B}(L_p(\Omega,\mu), L_q(\Omega,\nu))$ --- 
    топологически сюръективный оператор;

    \item $M_g^{\Omega_c^{\mu,\nu}}$ --- топологически сюръективный оператор;

    \item функция $\rho_{\mu,\nu}|_{\Omega_c^{\mu,\nu}}$ положительна, $|g
    \rho_{\mu,\nu}^{-1/p}|_{\Omega_c^{\mu,\nu}}|\geq c$ для некоторого $c>0$, 
    при этом если $p\neq q$, то пространство $(\Omega,\Sigma,\mu)$ состоит 
    из конечного числа атомов.
\end{enumerate}
\end{proposition}
\begin{proof} По предложению~\ref{MultOpDecompDecomp} оператор $M_g$
топологически сюръективен тогда и только тогда, когда таковы же операторы
$M_g^{\Omega_c^{\mu,\nu}}:L_p(\Omega_c^{\mu,\nu},\rho_{\mu,\nu}
\nu|_{\Omega_c^{\mu,\nu}})\to L_q(\Omega_c^{\mu,\nu},\nu|_{\Omega_c^{\mu,\nu}})$
и
$M_g^{\Omega_s^{\mu,\nu}}
:L_p(\Omega_s^{\mu,\nu},\mu_s|_{\Omega_s^{\mu,\nu}})
    \to 
L_q(\Omega_s^{\mu,\nu},\nu|_{\Omega_s^{\mu,\nu}})$. По
предложению~\ref{MultOpCharacBtwnTwoSingMeasSp} оператор
$M_g^{\Omega_s^{\mu,\nu}}$ нулевой. Так как $\nu(\Omega_s^{\mu,\nu})=0$, то
$L_p(\Omega_s^{\mu,\nu},\nu|_{\Omega_s^{\mu,\nu}})= \{0 \}$. Отсюда мы
заключаем, что $M_g^{\Omega_s^{\mu,\nu}}$ топологически сюръективен. Значит,
топологическая сюръективность оператора $M_g$ эквивалентна топологической
сюръективности оператора $M_g^{\Omega_c^{\mu,\nu}}$. Осталось применить
предложение~\ref{TopSurMultOpCharacBtwnTwoContMeasSp}.
\end{proof}

\begin{proposition}\label{TopSurMultOpDescBtwnTwoMeasSp} Пусть
$(\Omega,\Sigma,\mu)$, $(\Omega,\Sigma,\nu)$ --- два $\sigma$-конечных
пространства с мерой, $1\leq p,q\leq +\infty$ и $g\in L_0(\Omega,\Sigma)$. Тогда
следующие условия эквивалентны:

\begin{enumerate}[label = (\roman*)]
    \item $M_g\in\mathcal{B}(L_p(\Omega,\mu),L_q(\Omega,\nu))$ --- топологически
    сюръективный оператор;

    \item $M_{\chi_{\Omega_c^{\mu,\nu}}/g}\in\mathcal{B}(L_q(\Omega,\nu),
    L_p(\Omega,\mu))$ --- топологически инъективный правый обратный оператор к
    $M_g$.
\end{enumerate}
\end{proposition}
\begin{proof}
$(i) \implies (ii)$ По предложению~\ref{MultOpDecompDecomp} оператор
$M_g^{\Omega_c^{\mu,\nu}}$ топологически сюръективен. По
предложению~\ref{TopSurMultOpCharacBtwnTwoContMeasSp} он обратим, причем,
очевидно, ${(M_g^{\Omega_c^{\mu,\nu}})}^{-1}=M_{1/g}^{\Omega_c^{\mu,\nu}}$. 
Тогда для любой функции $h\in L_q(\Omega,\nu)$ выполнено
$$
\Vert M_{\chi_{\Omega_c^{\mu,\nu}}/g}(h)\Vert_{L_p(\Omega,\mu)}=
\Vert 
    M_{1/g}(h|_{\Omega_c^{\mu,\nu}})
\Vert_{L_p(\Omega_c^{\mu,\nu},\mu|_{\Omega_c^{\mu,\nu}})}
=\Vert 
    M_{1/g}^{\Omega_c^{\mu,\nu}}(h|_{\Omega_c^{\mu,\nu}})
\Vert_{L_p(\Omega_c^{\mu,\nu},\mu|_{\Omega_c^{\mu,\nu}})}
$$
$$
\leq\Vert 
    M_{1/g}^{\Omega_c^{\mu,\nu}}\Vert\Vert h|_{\Omega_c^{\mu,\nu}}
\Vert_{L_q(\Omega_c^{\mu,\nu},\nu|_{\Omega_c^{\mu,\nu}})}
\leq\Vert M_{1/g}^{\Omega_c^{\mu,\nu}}\Vert\Vert h\Vert_{L_q(\Omega,\nu)}.
$$ 
Следовательно, оператор $M_{\chi_{\Omega_c^{\mu,\nu}}/g}$ ограничен. Теперь
заметим, что для любой функции $h\in L_q(\Omega,\nu)$ также выполнено
$$
M_g(M_{\chi_{\Omega_c^{\mu,\nu}}/g}(h))
=M_g(\chi_{\Omega_c^{\mu,\nu}}/g  h)
=g (\chi_{\Omega_c^{\mu,\nu}}/g)   h
=h \chi_{\Omega_c^{\mu,\nu}}.
$$
Так как $\nu(\Omega\setminus\Omega_c^{\mu,\nu})=0$, то
$\chi_{\Omega_c^{\mu,\nu}}=\chi_{\Omega}$ и поэтому
$M_g(M_{\chi_{\Omega_c^{\mu,\nu}}/g}(h))=h \chi_{\Omega_c^{\mu,\nu}}=h
\chi_{\Omega}=h$. Последнее означает, что $M_{\chi_{\Omega_c^{\mu,\nu}}/g}$
является правым обратным оператором умножения к $M_g$. Наконец, заметим, что
$$
\Vert M_{\chi_{\Omega_c^{\mu,\nu}}/g}(h)\Vert_{L_p(\Omega,\mu)}
\geq\Vert M_g\Vert
\Vert M_g(M_{\chi_{\Omega_c^{\mu,\nu}}/g}(h))\Vert_{L_q(\Omega,\nu)}
\geq\Vert M_g\Vert\Vert h\Vert_{L_q(\Omega,\nu)}.
$$
для всех функций $h\in L_q(\Omega,\nu)$. Значит, оператор
$M_{\chi_{\Omega_c^{\mu,\nu}}/g}$ топологически инъективен.

$(ii) \implies (i)$ Для произвольной функции $h\in L_q(\Omega,\nu)$ рассмотрим
функцию $f=M_{\chi_{\Omega_c^{\mu,\nu}}/g}(h)$. Тогда
$M_g(f)=M_g(M_{\chi_{\Omega_c^{\mu,\nu}}/g}(h))=h$ и 
$\Vert f\Vert_{L_p(\Omega,\mu)}
\leq\Vert M_{\chi_{\Omega_c^{\mu,\nu}}/g}\Vert\Vert h\Vert_{L_q(\Omega,\nu)}$. 
Так как $h$ произвольна, то оператор $M_g$ топологически сюръективен.
\end{proof}

\begin{proposition}\label{CoisomMultOpCharacOnMeasSp} Пусть
$(\Omega,\Sigma,\mu)$ --- $\sigma$-конечное пространство с мерой, 
$1\leq p,q\leq +\infty$ и $g\in L_0(\Omega,\Sigma)$. 
Тогда следующие условия эквивалентны:

\begin{enumerate}[label = (\roman*)]
    \item $M_g\in\mathcal{B}(L_p(\Omega,\mu),L_q(\Omega,\mu))$ --- 
    коизометричеcкий оператор;

    \item $M_g$ --- изометрический изоморфизм;

    \item $|g|={\mu(\Omega)}^{1/q-1/p}$, при этом если $p\neq q$, 
    то пространство $(\Omega,\Sigma,\mu)$ состоит из одного атома.
\end{enumerate}
\end{proposition}
\begin{proof} Так как оператор $M_g$ коизометричен, то он топологически
сюръективен и поэтому из предложения~\ref{TopSurMultOpCharacOnMeasSp} мы
получаем, что он топологический изоморфизм. Как следствие, он инъективен, но
инъективный коизометрический оператор есть изометрический изоморфизм. Осталось
заметить, что всякий изометрический изоморфизм является строгой коизометрией.
Таким образом, оператор $M_g$ коизометричен тогда и только тогда, когда он
строго коизометричен тогда и только тогда, когда он изометрический изоморфизм.
Осталось применить предложение~\ref{IsomMultOpCharacOnMeasSp}.
\end{proof}

\begin{proposition}\label{CoisomMultOpCharacBtwnTwoContMeasSp} Пусть
$(\Omega,\Sigma,\nu)$ --- $\sigma$-конечное пространство с мерой, 
$1\leq p,q\leq +\infty$ и $g,\rho\in L_0(\Omega,\Sigma)$ 
причем функция $\rho$ неотрицательна. Тогда следующие условия эквивалентны:

\begin{enumerate}[label = (\roman*)]
    \item $M_g\in\mathcal{B}(L_p(\Omega,\rho \nu),L_q(\Omega,\nu))$ ---
    коизометрический оператор; 

    \item $M_g$ --- изометрический изоморфизм;

    \item функция $\rho$ положительна, 
    $|g  \rho^{-1/p}|={\mu(\Omega)}^{1/p-1/q}$, при этом если $p\neq q$, 
    то пространство $(\Omega,\Sigma,\mu)$ состоит из одного атома.
\end{enumerate}
\end{proposition}
\begin{proof} $(i) \implies (ii)$ Допустим, оператор $M_g$ коизометричен, тогда
он топологически сюръективен. По
предложению~\ref{TopSurMultOpCharacBtwnTwoContMeasSp} оператор $M_g$ является
топологический изоморфизмом, и, как следствие, он биективен. Осталось заметить,
что биективная коизометрия есть изометрический изоморфизм.

$(ii) \implies (i)$ Если оператор $M_g$ --- изометрический изоморфизм, то он,
конечно, коизометрия и даже строгая коизометрия.

$(ii) \Longleftrightarrow (iii)$ Эквивалентность следует из
предложения~\ref{IsomMultOpCharacBtwnTwoContMeasSp}.
\end{proof}

\begin{proposition}\label{CoisomMultOpCharacBtwnTwoMeasSp} Пусть
$(\Omega,\Sigma,\mu)$, $(\Omega,\Sigma,\nu)$ --- два $\sigma$-конечных
пространства с мерой, $1\leq p,q\leq +\infty$ и $g\in L_0(\Omega,\Sigma)$. Тогда
следующие условия эквивалентны: 

\begin{enumerate}[label = (\roman*)]
    \item $M_g\in\mathcal{B}(L_p(\Omega,\mu), L_q(\Omega,\nu))$ --- 
    коизометрический оператор;

    \item $M_g^{\Omega_c^{\mu,\nu}}$ --- изометрический изоморфизм;

    \item функция $\rho_{\mu,\nu}|_{\Omega_c^{\mu,\nu}}$ положительна, 
    $|g \rho_{\mu,\nu}^{-1/p}|_{\Omega_c^{\mu,\nu}}|
    ={\mu(\Omega_c^{\mu,\nu})}^{1/p-1/q}$, при этом если $p\neq q$, 
    то пространство $(\Omega,\Sigma,\mu)$ состоит из одного атома.
\end{enumerate}
\end{proposition}
\begin{proof} $(i) \implies (ii)$ Так как оператор $M_g$ коизометричен, то из
предложения~\ref{MultOpDecompDecomp} мы знаем, что оператор
$M_g^{\Omega_c^{\mu,\nu}}$ также коизометричен. Из
предложения~\ref{CoisomMultOpCharacBtwnTwoContMeasSp} мы получаем, что он также
является изометрическим изоморфизмом. 

$(ii) \implies (i)$ Рассмотрим произвольную функцию $h\in L_q(\Omega,\nu)$,
тогда существует функция $f\in
L_p(\Omega_c^{\mu,\nu},\mu|_{\Omega_c^{\mu,\nu}})$ такая, что
$M_g^{\Omega_c^{\mu,\nu}}(f)=h|_{\Omega_c^{\mu,\nu}}$. По
предложению~\ref{MultOpCharacBtwnTwoSingMeasSp} оператор
$M_g^{\Omega_s^{\mu,\nu}}$ нулевой, поэтому
$$
M_g(\widetilde{f})
=\widetilde{M_g^{\Omega_c^{\mu,\nu}}(\widetilde{f}|_{\Omega_c^{\mu,\nu}})}
+\widetilde{M_g^{\Omega_s^{\mu,\nu}}(\widetilde{f}|_{\Omega_s^{\mu,\nu}})}
=\widetilde{h|_{\Omega_c^{\mu,\nu}}}.
$$
Так как $\nu(\Omega_s^{\mu,\nu})=0$, то 
$\Vert h-\widetilde{h|_{\Omega_c^{\mu,\nu}}}\Vert_{L_q(\Omega,\nu)}
=\Vert h\chi_{\Omega_s^{\mu,\nu}}\Vert_{L_q(\Omega,\nu)}=0$, и мы заключаем, что
$h=\widetilde{h|_{\Omega_c^{\mu,\nu}}}$. Итак, мы построили функцию
$\widetilde{f}\in L_p(\Omega,\mu)$ такую, что $M_g(\widetilde{f})=h$ и 
$\Vert \widetilde{f}\Vert_{L_p(\Omega,\mu)}
=\Vert f\Vert_{L_p(\Omega_c^{\mu,\nu},\mu|_{\Omega_c^{\mu,\nu}})}
=\Vert 
    h|_{\Omega_c^{\mu,\nu}}
\Vert_{L_q(\Omega_c^{\mu,\nu},\nu|_{\Omega_c^{\mu,\nu}})}
\leq\Vert h\Vert_{L_q(\Omega,\nu)}$. Так как функция $h$ произвольна, 
то оператор $M_g$ $1$-топологически сюръективен. Заметим, что
$$
\Vert M_g(f)\Vert_{L_q(\Omega,\nu)}
=\Vert
    \widetilde{M_g^{\Omega_c^{\mu,\nu}}(f|_{\Omega_c^{\mu,\nu}})}
    +\widetilde{M_g^{\Omega_s^{\mu,\nu}}(f|_{\Omega_s^{\mu,\nu}})}
\Vert_{L_q(\Omega,\nu)}
=\Vert
    \widetilde{M_g^{\Omega_c^{\mu,\nu}}(f|_{\Omega_c^{\mu,\nu}})}
\Vert_{L_q(\Omega,\nu)}
$$
$$
=\Vert 
    M_g^{\Omega_c^{\mu,\nu}}(f|_{\Omega_c^{\mu,\nu}})
\Vert_{L_q(\Omega_c^{\mu,\nu},\nu|_{\Omega_c^{\mu,\nu}})}
=\Vert 
    f|_{\Omega_c^{\mu,\nu}}
\Vert_{L_p(\Omega_c^{\mu,\nu},\mu|_{\Omega_c^{\mu,\nu}})}
\leq\Vert f \Vert_{L_p(\Omega,\mu)}.
$$
для всех функций $f\in L_p(\Omega,\mu)$, значит оператор $M_g$ сжимающий, но от
также $1$-топологически сюръективный. Следовательно, оператор $M_g$
коизометрический.

$(ii) \Longleftrightarrow (iii)$ Эквивалентность следует из
предложения~\ref{CoisomMultOpCharacBtwnTwoContMeasSp}.
\end{proof}

\begin{proposition}\label{CoisomMultOpDescBtwnTwoMeasSp} Пусть
$(\Omega,\Sigma,\mu)$, $(\Omega,\Sigma,\nu)$ --- два $\sigma$-конечных
пространства с мерой, $1\leq p,q\leq +\infty$ и $g\in L_0(\Omega,\Sigma)$. Тогда
следующие условия эквивалентны: 

\begin{enumerate}[label = (\roman*)]
    \item $M_g\in\mathcal{B}(L_p(\Omega,\mu),L_q(\Omega,\nu))$ --- 
    коизометрический оператор;

    \item $M_{\chi_{\Omega_c^{\mu,\nu}}/g}\in\mathcal{B}(L_q(\Omega,\nu),
    L_p(\Omega,\mu))$ изометрический  правый обратный оператор к $M_g$.
\end{enumerate}
\end{proposition}
\begin{proof}
$(i) \implies (ii)$ По предложению~\ref{MultOpDecompDecomp} оператор
$M_g^{\Omega_c^{\mu,\nu}}$ коизометричен и по
предложению~\ref{CoisomMultOpCharacBtwnTwoContMeasSp} он изометричен и обратим.
При этом, очевидно,
${(M_g^{\Omega_c^{\mu,\nu}})}^{-1}=M_{1/g}^{\Omega_c^{\mu,\nu}}$. 
Тогда для любой функции $h\in L_q(\Omega,\nu)$ мы имеем
$$
\Vert M_{\chi_{\Omega_c^{\mu,\nu}}/g}(h)\Vert_{L_p(\Omega,\mu)}=
\Vert M_{1/g}(h)\chi_{\Omega_c^{\mu,\nu}}\Vert_{L_p(\Omega,\mu)}=
\Vert 
    M_{1/g}(h|_{\Omega_c^{\mu,\nu}})
\Vert_{L_p(\Omega_c^{\mu,\nu},\mu|_{\Omega_c^{\mu,\nu}})}
$$
$$
=\Vert 
    M_{1/g}^{\Omega_c^{\mu,\nu}}(h|_{\Omega_c^{\mu,\nu}})
\Vert_{L_p(\Omega_c^{\mu,\nu},\mu|_{\Omega_c^{\mu,\nu}})}
=\Vert 
    h|_{\Omega_c^{\mu,\nu}}
\Vert_{L_q(\Omega_c^{\mu,\nu},\nu|_{\Omega_c^{\mu,\nu}})}
\leq\Vert h\Vert_{L_q(\Omega,\nu)}.
$$ 
Значит, $M_{\chi_{\Omega_c^{\mu,\nu}}/g}$ --- сжимающий оператор. Теперь
заметим, что для всех функций $h\in L_q(\Omega,\nu)$ выполнено
$$
M_g(M_{\chi_{\Omega_c^{\mu,\nu}}/g}(h))
=M_g(\chi_{\Omega_c^{\mu,\nu}}/g  h)
=g (\chi_{\Omega_c^{\mu,\nu}}/g) h
=h \chi_{\Omega_c^{\mu,\nu}}.
$$
Так как $\nu(\Omega\setminus\Omega_c^{\mu,\nu})=0$, то
$\chi_{\Omega_c^{\mu,\nu}}=\chi_{\Omega}$ и поэтому
$M_g(M_{\chi_{\Omega_c^{\mu,\nu}}/g}(h))
=h \chi_{\Omega_c^{\mu,\nu}}
=h\chi_{\Omega}=h$. Последнее означает, что $M_{\chi_{\Omega_c^{\mu,\nu}}/g}$
правый обратный оператор умножения к $M_g$. Наконец, для произвольной функции
$h\in L_q(\Omega,\nu)$ мы имеем
$$
\Vert M_{\chi_{\Omega_c^{\mu,\nu}}/g}(h)\Vert_{L_p(\Omega,\mu)}
\geq\Vert 
    M_g\Vert\Vert M_g(M_{\chi_{\Omega_c^{\mu,\nu}}/g}(h))
\Vert_{L_q(\Omega,\nu)}
\geq\Vert h\Vert_{L_q(\Omega,\nu)}.
$$
Значит, оператор $M_{\chi_{\Omega_c^{\mu,\nu}}/g}$ $1$-топологически инъективен,
но также сжимающий. Следовательно, оператор $M_{\chi_{\Omega_c^{\mu,\nu}}/g}$
изометрический.

$(ii) \implies (i)$ Рассмотрим произвольную функцию $h\in L_q(\Omega,\nu)$ и
функцию $f=M_{\chi_{\Omega_c^{\mu,\nu}}/g}(h)$. Тогда
$M_g(f)=M_g(M_{\chi_{\Omega_c^{\mu,\nu}}/g}(h))=h$ и $\Vert
f\Vert_{L_p(\Omega,\mu)}\leq\Vert h\Vert_{L_q(\Omega,\nu)}$. Так как $h$
произвольна, то оператор $M_g$ строго $1$-топологически сюръективен. Пусть 
$f\in L_p(\Omega,\mu)$. По предположению оператор 
$M_{\chi_{\Omega_c^{\mu,\nu}}/g}$ изометричен, поэтому 
$$
\Vert M_g(f)\Vert_{L_q(\Omega,\nu)}
=\Vert M_{\chi_{\Omega_c^{\mu,\nu}}/g}(M_g(f))\Vert_{L_p(\Omega,\mu)}
=\Vert f\chi_{\Omega_c^{\mu,\nu}}\Vert_{L_p(\Omega,\mu)}
\leq\Vert f\Vert_{L_p(\Omega,\mu)}.
$$
Так как $f$ произвольна, то оператор $M_g$ сжимающий, но он еще и
$1$-топологически сюръективен, а значит строго коизометричен.
\end{proof}

Это доказательство также показывает, что каждый коизометрический оператор
умножения строго коизометричен.



%-------------------------------------------------------------------------------
%	Homological triviality of the category B(Omega,Sigma)-modules L_p
%-------------------------------------------------------------------------------

\subsection{Гомологическая тривиальность категории
    \texorpdfstring{$B(\Omega,\Sigma)$-модулей $L_p$}{B (Omega)-модулей Lp}
}\label{
    SubSectionHomologicalTrivialityOfTheCategoryBOmegaSigmaModulesLp}

Теперь мы готовы доказать, что категория рассмотренная в начале параграфа
гомологически тривиальна, то есть все ее модули являются проективными,
инъективными и плоскими.

\begin{proposition}\label{HomTrivlOfLpCat} Пусть $(\Omega,\Sigma)$ измеримое
пространство и $\mu$ --- $\sigma$-конечная мера на $\Omega$. Тогда
$B(\Omega,\Sigma)$-модуль $L_p(\Omega,\mu)$ $\langle$~метрически /
топологически~$\rangle$ проективный, инъективный и плоский по отношению к
категории $\langle$~$B(\Omega,\Sigma)-\mathbf{mod(L)}_1$ /
$B(\Omega,\Sigma)-\mathbf{mod(L)}$~$\rangle$.
\end{proposition}
\begin{proof} Обозначим $X:=L_p(\Omega,\mu)$ и
$\langle$~$\mathbf{C}:=B(\Omega,\Sigma)-\mathbf{mod(L)}_1$ /
$\mathbf{C}:=B(\Omega,\Sigma)-\mathbf{mod(L)}$~$\rangle$. 

Рассмотрим ковариантный функтор
$\langle$~$F_{proj}
    :=\operatorname{Hom}_{\mathbf{C}}(X,-):\mathbf{C}\to\mathbf{Ban}_1$
/
$F_{proj}
    :=\operatorname{Hom}_{\mathbf{C}}(X,-):\mathbf{C}\to\mathbf{Ban}$~$\rangle$.
По предложению $\langle$~\ref{CoisomMultOpDescBtwnTwoMeasSp}
/~\ref{TopSurMultOpCharacBtwnTwoMeasSp}~$\rangle$ любой
$\langle$~коизометрический / топологически сюръективный~$\rangle$ морфизм $\xi$
в $\mathbf{C}$ есть ретракция, значит, оператор $F_{proj}(\xi)$ --- ретракция в
$\langle$~$\mathbf{Ban}_1$ / $\mathbf{Ban}$~$\rangle$, и, как следствие,
$\langle$~строго коизометричен / сюръективен~$\rangle$. Так как морфизм $\xi$
произволен, то модуль $X$ $\langle$~метрически / топологически~$\rangle$
проективен.

Рассмотрим контравариантный функтор
$\langle$~$F_{inj}
    :=\operatorname{Hom}_{\mathbf{C}}(-,X):\mathbf{C}\to\mathbf{Ban}_1$
/
$F_{inj}
    :=\operatorname{Hom}_{\mathbf{C}}(-,X):\mathbf{C}\to\mathbf{Ban}$~$\rangle$.
Из предложения $\langle$~\ref{IsomMultOpDescBtwnTwoMeasSp}
/~\ref{TopInjMultOpDescBtwnTwoMeasSp}~$\rangle$ любой $\langle$~изометрический /
топологически инъективный~$\rangle$ морфизм $\xi$ из $\mathbf{C}$ является
коретракцией, значит, оператор $F_{inj}(\xi)$ --- ретракция в
$\langle$~$\mathbf{Ban}_1$ / $\mathbf{Ban}$~$\rangle$, и, как следствие,
$\langle$~строго коизометричен / сюръективен~$\rangle$. Так как морфизм $\xi$
произволен, то модуль $X$ $\langle$~метрически / топологически~$\rangle$
инъективен.

Рассмотрим ковариантный функтор
$\langle$~$F_{flat}
    :=-\projmodtens{B(\Omega,\Sigma)}X:\mathbf{C}\to\mathbf{Ban}_1$
/
$F_{flat}
    :=-\projmodtens{B(\Omega,\Sigma)}X:\mathbf{C}\to\mathbf{Ban}$~$\rangle$.
Снова, по предложению $\langle$~\ref{IsomMultOpDescBtwnTwoMeasSp}
/~\ref{TopInjMultOpDescBtwnTwoMeasSp}~$\rangle$ любой $\langle$~изометрический /
топологически инъективный~$\rangle$ морфизм $\xi$ в $\mathbf{C}$ является
коретракцией, значит, оператор $F_{flat}(\xi)$ --- коретракция в
$\langle$~$\mathbf{Ban}_1$ / $\mathbf{Ban}$~$\rangle$, и, как следствие, он
$\langle$~изометричен / топологически инъективен~$\rangle$. Так как морфизм
$\xi$ произволен, то модуль $X$ $\langle$~метрически / топологически~$\rangle$
плоский.
\end{proof}  % chktex 17

%----------------------------------------------------------------------------------------
%	THESIS CONTENT - APPENDICES
%----------------------------------------------------------------------------------------

\addtocontents{toc}{\vspace{2em}} % Add a gap in the Contents, for aesthetics

\appendix % Cue to tell LaTeX that the following 'chapters' are Appendices

% Include the appendices of the thesis as separate files from the Appendices folder
% Uncomment the lines as you write the Appendices

%\input{./Appendices/AppendixA}


\addtocontents{toc}{\vspace{2em}} % Add a gap in the Contents, for aesthetics

%----------------------------------------------------------------------------------------
%	SYMBOLS
%----------------------------------------------------------------------------------------

%\clearpage % Start a new page

%\lhead{\emph{Symbols}} % Set the left side page header to "Symbols"

%\listofnomenclature{ll} % Include a list of Symbols (a three column table)
%{
%$a$ & distance \\
%$P$ & power  \\
% Symbol & Description \\
%}


%----------------------------------------------------------------------------------------
%	BIBLIOGRAPHY
%----------------------------------------------------------------------------------------

\backmatter

\label{Bibliography}

\lhead{\emph{Bibliography}} % Change the page header to say "Bibliography"

\bibliographystyle{unsrtnat} % Use the "unsrtnat" BibTeX style for formatting the Bibliography

\bibliography{Bibliography} % The references (bibliography) information are stored in the file named "Bibliography.bib"

\end{document}