% Chapter Template

\chapter{Applications to algebras of analysis} % Main chapter title

\label{ChapterApplicationsToAlgebrasOfAnalysis} % Change X to a consecutive number; for referencing this chapter elsewhere, use \ref{ChapterX}

\lhead{Chapter 3. \emph{Applications to algebras of analysis}} % Change X to a consecutive number; this is for the header on each page - perhaps a shortened title

Vaguely speaking there are three types of Banach modules depending on the type of module action: modules with pointwise multiplication, modules with composition of operators in the role of module action and modules with convolution. We shall investigate main examples of these types. Following the style of Dales and Polyakov from \cite{DalPolHomolPropGrAlg} we shall systematize all results on classical modules of analysis, but this time for metric and topological theory. We shall consider modules over operator algebras, sequence algebras, algebras of continuous functions and, finally, classical modules of harmonic analysis.


%----------------------------------------------------------------------------------------
%	Applications to operator algebras
%----------------------------------------------------------------------------------------


\section{Applications to modules over \texorpdfstring{$C^*$}{C*}-algebras}
\label{SectionApplicationsToModulesOverCStarAlgebras}

%----------------------------------------------------------------------------------------
%	Spatial modules
%----------------------------------------------------------------------------------------

\subsection{Spatial modules}
\label{SubSectionSpatialModules}

We start from the simplest  examples of modules over operator algebras --- the spatial modules. By Gelfand-Naimark's theorem (see e.g. [\cite{HelBanLocConvAlg}, theorem 4.7.57]) for any $C^*$-algebra $A$ there exists a Hilbert space $H$ and an isometric ${}^*$-homomorphism $\varrho:A\to\mathcal{B}(H)$. For Hilbert spaces that admit such homomorphism we may consider the left $A$-module $H_\varrho$ with module action defined as $a\cdot x=\varrho(a)(x)$. Automatically we get the structure of right $A$-module on $H^*$ which is by Riesz's theorem is isometrically isomorphic to $H^{cc}$. This isomorphism allows one to define the right $A$-module structure on $H^{cc}$ by $\overline{x}\cdot a=\overline{\varrho(a^*)(x)}$. For a given $x_1,x_2\in H$ we define the rank one operator $x_1\bigcirc x_2:H\to H:x\mapsto \langle x, x_2\rangle x_1$. 

\begin{proposition}\label{SpatModOverCStarAlgProp} Let $A$ be a $C^*$-algebra and $\varrho:A\to\mathcal{B}(H)$ be an isometric ${}^*$-homomorphism, such that its image contains a subspace of rank one operators of the form $\{x\bigcirc x_0:x\in H\}$ for some non zero $x_0\in H$. Then the left $A$-module $H_\varrho$ is metrically projective and flat, while the  right $A$-module $H_\varrho^{cc}$ is metrically injective.
\end{proposition}
\begin{proof} Without loss of generality we may assume that $\Vert x_0\Vert=1$. Consider linear operators $\pi:A_+\to H_\varrho:a\oplus_1 z\mapsto \varrho(a)(x_0)+zx_0$ and $\sigma:H_\varrho\to A_+:x\mapsto \varrho^{-1}(x\bigcirc x_0)$. It is straightforward to check that $\pi$ and $\sigma$ are contractive $A$-morphisms such that $\pi\sigma=1_{H_\varrho}$. Therefore $H_\varrho$ is a retract of $A_+$ in $A-\mathbf{mod}_1$. From propositions \ref{UnitalAlgIsMetTopProj} and \ref{RetrMetTopProjIsMetTopProj} it follows that $H_\varrho$ is metrically projective $A$-module. From proposition \ref{MetTopProjIsMetTopFlat} it follows that $H_\varrho$ is metrically flat too. Since $H_\varrho^{cc}\isom{\mathbf{mod}_1-A}H_\varrho^*$, proposition \ref{DualMetTopProjIsMetrInj} gives that $H_\varrho^{cc}$ is metrically injective.
\end{proof}

In what follows we shall use the following simple application of the above result.

\begin{proposition}\label{FinDimNHModTopProjFlat} Let $H$ be a finite dimensional Hilbert space. Then $\mathcal{N}(H)$ is topologically projective and hence flat as $\mathcal{B}(H)$-module.
\end{proposition}
\begin{proof} From [\cite{HelBanLocConvAlg}, proposition 0.3.38] we know that $\mathcal{N}(H)\isom{\mathbf{Ban}_1}H\projtens H^*$. Let $\varrho=1_{\mathcal{B}(H)}$, then we can claim a little bit more: $\mathcal{N}(H)\isom{\mathcal{B}(H)-\mathbf{mod}_1} H_\varrho\projtens H^*$. Since $H^*$ is finite dimensional, then  $H^*\isom{\mathbf{Ban}}\ell_1(\mathbb{N}_n)$ for $n=\dim(H)$ and as the result $\mathcal{N}(H)\isom{\mathcal{B}(H)-\mathbf{mod}} H_\varrho\projtens\ell_1(\mathbb{N}_n)$. By proposition \ref{SpatModOverCStarAlgProp} the module $H_\varrho$ it topologically projective, so from corollary \ref{MetTopProjTensProdWithl1} we get that $\mathcal{N}(H)$ is topologically projective as $\mathcal{B}(H)$-module. The last claim of theorem follows from proposition \ref{MetTopProjIsMetTopFlat}.
\end{proof}

%----------------------------------------------------------------------------------------
%	Projective ideals of C^*-algebras
%----------------------------------------------------------------------------------------

\subsection{Projective ideals of \texorpdfstring{$C^*$}{C*}-algebras}
\label{SubSectionProjectiveIdealsOfCStarAlgebras}

The study of homologically trivial ideals of $C^*$-algebras we start from projectivity, but before stating the main result we need a preparatory lemma.

\begin{lemma}\label{ContFuncCalcOnIdealOfCStarAlg} Let $I$ be a left ideal of a unital $C^*$-algebra $A$. Assume $a\in I$ is a self-adjoint element and let $E$ be the real subspace of real valued functions in $C(\operatorname{sp}_A(a))$ vanishing at zero. Then there is an isometric homomorphism $\operatorname{RCont}_a^0:E\to I$ well defined by $\operatorname{RCont}_a^0(f)=a$, where $f:\operatorname{sp}_A(a)\to\mathbb{C}:t\mapsto t$.
\end{lemma}
\begin{proof} By $\mathbb{R}_0[t]$ we denote the real linear subspace of $E$ consisting of polynomials vanishing at zero. Since $I$ is an ideal of $A$ and and $p\in\mathbb{R}_0[t]$ has no constant term then $p(a)\in I$.  Hence we have well defined $\mathbb{R}$-linear homomorphism of algebras $\operatorname{RPol}_a^0:\mathbb{R}_0[t]\to I:p\mapsto p(a)$. By continuous functional calculus for any polynomial $p$ holds $\Vert p(a)\Vert=\Vert p|_{\operatorname{sp}_A(a)}\Vert_\infty$, so $\Vert\operatorname{RPol}_a^0(p)\Vert=\Vert p|_{\operatorname{sp}_A(a)}\Vert_\infty$. Thus $\operatorname{RPol}_a^0$ is isometric. As $\mathbb{R}_0[t]$ is dense in $E$ and $I$ is complete, then $\operatorname{RPol}_a^0$ has an isometric extension $\operatorname{RCont}_a^0:E\to I$ which is also an $\mathbb{R}$-linear homomorphism. 
\end{proof}

The following proof is inspired by ideas of D. P. Blecher and T. Kania. In [\cite{BleKanFinGenCStarAlgHilbMod}, lemma 2.1] they proved that any algebraically finitely generated left ideal of $C^*$-algebras is principal.  

\begin{theorem}\label{LeftIdealOfCStarAlgMetTopProjCharac} Let $I$ be a left ideal of a $C^*$-algebra $A$. Then the following are equivalent:

$i)$ $I=Ap$ for some self-adjoint idempotent $p\in I$;

$ii)$ $I$ is metrically projective $A$-module;

$iii)$ $I$ is topologically projective $A$-module.
\end{theorem}
\begin{proof} $i)$ $\implies$ $ii)$ Since $p$ is a self-adjoint idempotent, then $\Vert p\Vert=1$, so by proposition \ref{UnIdeallIsMetTopProj} paragraph $i)$ the ideal $I$ is metrically projective as $A$-module.

$ii)$ $\implies$ $iii)$ See proposition \ref{MetProjIsTopProjAndTopProjIsRelProj}.

$iii)$ $\implies$ $i)$ Let $(e_\nu)_{\nu\in N}$ be a right contractive approximate identity of ideal $I$ [\cite{HelBanLocConvAlg}, theorem 4.7.79]. Since $I$ admits a right approximate identity, then it is an essential left $I$-module, and a fortiori an essential $A$-module. By proposition \ref{NonDegenMetTopProjCharac} we have a right inverse $A$-morphism $\sigma:I\to A\projtens \ell_1(B_I)$ of $\pi_I$ in $A-\mathbf{mod}$. For each $d\in B_I$ consider $A$-morphisms $p_d:A\projtens \ell_1(B_I)\to A:a\projtens \delta_x\mapsto \delta_x(d)a$ and $\sigma_d=p_d\sigma$. Then $\sigma(x)=\sum_{d\in B_I}\sigma_d(x)\projtens \delta_d$ for all $x\in I$. From identification $A\projtens\ell_1(B_I)\isom{\mathbf{Ban}_1}\bigoplus_1\{ A:d\in B_I\}$, for all $x\in I$ we have $\Vert\sigma(x)\Vert=\sum_{d\in B_I} \Vert\sigma_d(x)\Vert$. Since $\sigma$ is a right inverse morphism of $\pi_I$ we have $x=\pi_I(\sigma(x))=\sum_{d\in B_I}\sigma_d(x)d$ for all $x\in I$. 

For all $x\in I$ we have
$\Vert\sigma_d(x)\Vert=\Vert\sigma_d(\lim_\nu xe_\nu)\Vert=\lim_\nu\Vert x\sigma_d(e_\nu)\Vert \leq\Vert x\Vert\liminf_\nu\Vert\sigma_d(e_\nu)\Vert$, so $\Vert\sigma_d\Vert\leq \liminf_\nu\Vert\sigma_d(e_\nu)\Vert$. Then for all $S\in\mathcal{P}_0(B_I)$ holds
$$
\sum_{d\in S}\Vert \sigma_d\Vert
\leq \sum_{d\in S}\liminf_\nu\Vert \sigma_d(e_\nu)\Vert
\leq \liminf_\nu\sum_{d\in S}\Vert \sigma_d(e_\nu)\Vert
\leq \liminf_\nu\sum_{d\in B_I}\Vert \sigma_d(e_\nu) \Vert
$$
$$
=\liminf_{\nu}\Vert\sigma(e_\nu)\Vert
\leq \Vert\sigma\Vert\liminf_{\nu}\Vert e_\nu\Vert
\leq \Vert\sigma\Vert
$$
Since $S\in \mathcal{P}_0(B_I)$ is arbitrary, then the sum $\sum_{d\in B_I}\Vert\sigma_d\Vert$ is finite. As the consequence, the sum $\sum_{d\in B_I}\Vert\sigma_d\Vert^2$ is finite too. 

Now we regard $A$ as an ideal in its unitization $A_\#$, then $I$ is an ideal of $A_\#$ too. Fix a natural number $m\in\mathbb{N}$ and a real number $\epsilon>0$. Then there exists a set $S\in\mathcal{P}_0(B_I)$ such that $\sum_{d\in B_I\setminus S}\Vert\sigma_d\Vert<\epsilon$. Denote its cardinality by $N$. Consider positive element $b=\sum_{d\in B_I}\Vert\sigma_d\Vert^2 d^*d\in I$. Now we perform a ``power trick'' by considering different powers $b^{1/m}$ of positive element $b$, where $m\in\mathbb{N}$. By lemma \ref{ContFuncCalcOnIdealOfCStarAlg} we have that $b^{1/m}\in I$, so $b^{1/m}=\sum_{d\in B_I}\sigma_d(b^{1/m})d$. By continuous functional calculus we have $\Vert b^{1/m}\Vert=\sup_{t\in\operatorname{sp}_{A_\#}(b)} t^{1/m}\leq\Vert b\Vert^{1/m}$, then $\limsup_{m\to\infty}\Vert b^{1/m}\Vert\leq 1$. Therefore $\Vert b^{1/m}\Vert\leq 2$ for sufficiently big $m$. Denote $\varsigma_d=\sigma_d(b^{1/m})$, $u=\sum_{d\in S}\varsigma_dd$ and $v=\sum_{d\in B_I\setminus S}\varsigma_d d$, so 
$$
b^{2/m}=(b^{1/m})^*b^{1/m}=u^*u+u^*v+v^*u+v^*v
$$
Clearly, $\varsigma_d^*\varsigma_d\leq \Vert \varsigma_d\Vert^2 e_{A_\#}\leq \Vert \sigma_d\Vert^2\Vert b^{1/m}\Vert^2 e_{A_\#}\leq 4\Vert \sigma_d\Vert^2 e_{A_\#}$. For any $x,y\in A$ we have $x^*x+y^*y-(x^*y+y^*x)=(x-y)^*(x-y)\geq 0$, therefore 
$$
d^*\varsigma_d^* \varsigma_c c+c^*\varsigma_c^* \varsigma_d d
\leq d^*\varsigma_d^*\varsigma_d d + c^*\varsigma_c^*\varsigma_c c
\leq 4\Vert \sigma_d\Vert^2 d^*d+4\Vert \sigma_c\Vert^2 c^*c
$$
for all $c,d\in B_I$. We sum up these inequalities over $c\in S$ and $d\in S$, then 
$$
\begin{aligned}
\sum_{c\in S}\sum_{d\in S}c^*\varsigma_c^* \varsigma_d d
&=\frac{1}{2}\left(\sum_{c\in S}\sum_{d\in S}d^*\varsigma_d^* \varsigma_c c+\sum_{c\in S}\sum_{d\in S}c^*\varsigma_c^* \varsigma_d d\right)\\
&\leq\frac{1}{2}\left(4 N\sum_{d\in S} \Vert \sigma_d\Vert^2 d^*d+
4 N\sum_{c\in S} \Vert \sigma_c\Vert^2 c^*c\right)\\
&=4 N\sum_{d\in S} \Vert \sigma_d\Vert^2 d^*d.
\end{aligned}
$$
Therefore
$$
u^*u
=\left(\sum_{c\in S}\varsigma_c c\right)^*\left(\sum_{d\in S}\varsigma_d d\right)
=\sum_{c\in S}\sum_{d\in S}c^*\varsigma_c^* \varsigma_d d
\leq 4N\sum_{d\in S} \Vert \sigma_d\Vert^2 d^*d
= 4N b
$$
Note that
$$
\Vert u\Vert
\leq \sum_{d\in S}\Vert\varsigma_d\Vert\Vert d\Vert
\leq \sum_{d\in S}2\Vert\sigma_d\Vert
\leq 2\Vert\sigma\Vert,
\qquad
\Vert v\Vert
\leq \sum_{d\in B_I\setminus S}\Vert\varsigma_d\Vert\Vert d\Vert
\leq \sum_{d\in B_I\setminus S}2\Vert\sigma_d\Vert
\leq 2\epsilon,
$$
so $\Vert u^*v+v^*u\Vert\leq 8\Vert\sigma\Vert\epsilon$ and $\Vert v^*v\Vert\leq 4\epsilon^2$. Since $u^*v+v^*u$ and $v^*v$ are self adjoint, then $u^*v+v^*u\leq 8\Vert\sigma\Vert\epsilon e_{A_\#}$ and $v^*v\leq 4\epsilon^2 e_{A_\#}$
Therefore for any $\epsilon>0$ and sufficiently big $m$ we have 
$$
b^{2/m}
=u^*u+u^*v+v^*u+v^*v
\leq 4Nb+\epsilon(8\Vert\sigma\Vert+4\epsilon)e_{A_\#}.
$$


In other words $g_m(b)\geq 0$ for continuous function $g_m:\mathbb{R}_+\to\mathbb{R}:t\mapsto 4Nt+\epsilon(8\Vert\sigma\Vert+4\epsilon)-t^{2/m}$. Now choose $\epsilon>0$ such that $M:=\epsilon(8\Vert\sigma\Vert+4\epsilon)<1$.  By spectral mapping theorem [\cite{HelLectAndExOnFuncAn}, theorem 6.4.2] we get $g_m(\operatorname{sp}_{A_\#}(b))=\operatorname{sp}_{A_\#}(g_m(b))\subset\mathbb{R}_+$. It is routine to check that $g_m$ has the only one extreme point $t_{0,m}=(2Nm)^{\frac{m}{2-m}}$ where the minimum of $g_m$ is attained. Since $\lim_{m\to\infty} g_m(t_{0,m})=M-1<0$, $g_m(0)=M>0$ and $\lim_{t\to\infty} g_m(t)=+\infty$, then for sufficiently big $m$ the function $g_m$ has exactly two roots: $t_{1,m}\in(0,t_{0,m})$ and $t_{2,m}\in(t_{0,m},+\infty)$. Therefore solution of the inequality $g_m(t)\geq 0$ is $t\in[0,t_{1,m}]\cup[t_{2,m},+\infty)$. Hence $\operatorname{sp}_{A_\#}(b)\subset[0,t_{1,m}]\cup[t_{2,m},+\infty)$ for all sufficiently large $m$. Since $\lim_{m\to\infty} t_{0,m}=0$ then $\lim_{m\to\infty} t_{1,m}=0$ also. Note that $g_m(1)=4N+M-1>0$, so for sufficiently large $m$ we also have $t_{2,m}\leq 1$. Consequently, $\operatorname{sp}_{A_\#}(b)\subset\{0\}\cup[1,+\infty)$.

Consider a continuous function $h:\mathbb{R}_+\to\mathbb{R}:t\mapsto\min(1, t)$, then from lemma \ref{ContFuncCalcOnIdealOfCStarAlg} we get an idempotent $p=h(b)=\operatorname{RCont}_b^0(h)\in I$ such that $\Vert p\Vert=\sup_{t\in\operatorname{sp}_{A_\#}(b)}|h(t)|\leq 1$. Therefore $p$ is a self-adjoint idempotent. Since $h(t)t=th(t)=t$ for all $t\in \operatorname{sp}_{A_\#}(b)$ we have $bp=pb=b$. The last equality implies
$$
0=(e_{A_\#}-p)b(e_{A_\#}-p)=\sum_{d\in B_I}(\Vert\sigma_d\Vert d(e_{A_\#}-p))^*\Vert\sigma_d\Vert d(e_{A_\#}-p).
$$
Since the right hand side of this equality is non negative, then $d=dp$ for all $d\in B_I$ with $\sigma_d\neq 0$. Finally, for any $x\in I$ we obtain $xp=\sum_{d\in B_I}\sigma_d(x)dp=\sum_{d\in B_I}\sigma_d(x)d=x$, i.e. $I=Ap$, for self-adjoint idempotent $p\in I$.
\end{proof}

It is worth to point out here that in relative theory there no such description for relative projectivity of left ideals of $C^*$-algebras. Though for the case of separable $C^*$-algebras (that is $C^*$-algebras that are separable as Banach spaces) all left ideals are relatively projective. See [\cite{LykProjOfBanAndCStarAlgsOfContFld}, section 6] for a nice overview of the current state of the problem.

\begin{corollary}\label{BiIdealOfCStarAlgMetTopProjCharac} Let $I$ be a two-sided ideal of a $C^*$-algebra $A$. Then the following are equivalent:

$i)$ $I$ is unital;

$ii)$ $I$ is metrically projective $A$-module;

$iii)$ $I$ is topologically projective $A$-module.
\end{corollary}
\begin{proof} The ideal $I$ has a contractive approximate identity [\cite{HelBanLocConvAlg}, theorem 4.7.79]. Therefore $I$ has a right identity iff $I$ is unital. Now all equivalences follow from theorem \ref{LeftIdealOfCStarAlgMetTopProjCharac}. 
\end{proof}

\begin{corollary}\label{IdealofCommCStarAlgMetTopProjCharac} Let $S$ be a locally compact Hausdorff space, and $I$ be an ideal of $C_0(S)$. Then the following are equivalent:

$i)$ $\operatorname{Spec}(I)$ is compact;

$ii)$ $I$ is metrically projective $C_0(S)$-module;

$iii)$ $I$ is topologically projective $C_0(S)$-module.
\end{corollary}
\begin{proof} By Gelfand-Naimark's theorem $I\isom{\mathbf{Ban}_1}C_0(\operatorname{Spec}(I))$, therefore $I$ is semisimple. Now by Shilov's idempotent theorem $I$ is unital iff $\operatorname{Spec}(I)$ is compact. It remains to apply corollary \ref{BiIdealOfCStarAlgMetTopProjCharac}. 
\end{proof}

It is worth to mention that the class of relatively projective ideals of $C_0(S)$ is much larger. In fact a closed ideal $I$ of $C_0(S)$ is relatively projective iff $\operatorname{Spec}(I)$ is paracompact [\cite{HelHomolBanTopAlg}, chapter IV,\S\S 2-3].

%----------------------------------------------------------------------------------------
%	Injective ideals of C^*-algebras
%----------------------------------------------------------------------------------------

\subsection{Injective ideals of \texorpdfstring{$C^*$}{C*}-algebras}
\label{SubSectionInjectiveIdealsOfCStarAlgebras}

Now we proceed to injectivity of two-sided ideals of $C^*$-algebras. Unfortunately we don't have a complete characterization at hand, but some necessary conditions and several examples. The following proposition shows that we may restrict investigation of injective ideals to the case of $C^*$-algebras that $\langle$~metrically / topologically~$\rangle$ injective over themselves as right modules.

\begin{proposition}\label{MetTopInjOvrAlgIffOvrItslf} Let $I$ be a two-sided ideal of a $C^*$-algebra $A$, then $I$ is $\langle$~metrically / topologically~$\rangle$ injective as $A$-module iff $I$ is $\langle$~metrically / topologically~$\rangle$ injective as $I$-module.
\end{proposition}
\begin{proof} Note that any two-sided ideal $I$ of a $C^*$-algebra $A$ is again a $C^*$-algebra with contractive approximate identity [\cite{HelBanLocConvAlg}, theorem 4.7.79]. Therefore $I$ is faithful as right $I$-module. Now proposition \ref{ReduceInjIdToInjAlg} gives the desired equivalence.
\end{proof}

We shall say a few words on so called $AW^*$-algebras, since they are key players here. In attempts to find a purely algebraic definition of $W^*$-algebras Kaplanski introduced this class of $C^*$-algebras in \cite{KaplProjInBanAlg}. A $C^*$-algebra $A$ is called an $AW^*$-algebra if for any subset $S\subset A$ the right algebraic annihilator $\operatorname{r.ann}_A(S)=\{y\in A: Sy=\{0\}\}$ is of the form $pA$ for some self-adjoint idempotent $p\in A$. This class contains all $W^*$-algebras, but strictly larger. Note that for the case of commutative $C^*$-algebras the property of being an $AW^*$-algebra is equivalent to  $\operatorname{Spec}(A)$ being a Stonean space [\cite{BerbBaerStarRings}, theorem 1.7.1]. The main reference to $AW^*$-algebras and more general Baer ${}^*$-rings is \cite{BerbBaerStarRings}. 

The following proposition is a combination of results by M. Hamana and M. Takesaki.

\begin{proposition}[Hamana, Takesaki]\label{MetInjCStarAlgCharac} Let $A$ be a $C^*$-algebra, then it is metrically injective right $A$-module iff it is a commutative $AW^*$-algebra, that is $\operatorname{Spec}(A)$ is a Stonean space.
\end{proposition}
\begin{proof} 

If $A$ is metrically injective as $A$-module, then it has norm one left identity by proposition \ref{MetTopInjOfId}. But $A$ also has a contractive approximate identity  [\cite{HelBanLocConvAlg}, theorem 4.7.79], therefore $A$ is unital. Now by result of Hamana  [\cite{HamInjEnvBanMod}, proposition 2] the $C^*$-algebra $A$ is a commutative $AW^*$-algebra. Hamana's argument was for left modules, but one can easily modify his proof to get the result for right modules.

The converse proved by Takesaki in [\cite{TakHanBanThAndJordDecomOfModMap}, theorem 2]. Although only two-sided Banach modules were treated there, the reasoning is essentially the same for right modules.
\end{proof}

It remains to consider topological injectivity. As the following proposition shows topologically injective $C^*$-algebras are not so far from commutative ones. This proposition exploits the l.u.st. property, for its definition see section  \ref{SubSectionHomologicallyTrivialModulesOverBanachAlgebrasWithSpecificGeometry}.

\begin{proposition}\label{TopInjIdHaveLUST} Let $A$ be a $C^*$-algebra which is topologically injective as an $A$-module. Then $A$ has the l.u.st. property and as the consequence it can't contain  $\mathcal{B}(\ell_2(\mathbb{N}_n))$ as ${}^*$-subalgebra for arbitrary $n\in\mathbb{N}$.
\end{proposition}
\begin{proof} By Gelfand-Naimark's theorem [\cite{HelBanLocConvAlg}, theorem 4.7.57] there exists a Hilbert space $H$ and an isometric ${}^*$-homomorphism $\varrho:A\to\mathcal{B}(H)$. Denote $\Lambda:=B_{H_\varrho^{cc}}$. It is easy to check that 
$$
\rho:A\to\bigoplus\nolimits_\infty\{H_\varrho^{cc}:\overline{x}\in \Lambda\}:a\mapsto \bigoplus\nolimits_\infty\{\overline{x}\cdot a:\overline{x}\in \Lambda\}
$$
is an isometric $A$-morphism of right $A$-modules. Since $A$ is topologically injective $A$-module, then $\rho$ has a left inverse $A$-morphism $\tau$. Therefore $A$ is complemented in $E:=\bigoplus_\infty\{H_\varrho^{cc}:\overline{x}\in \Lambda\}$ via projection $\rho\tau$. Note that $H_{\varrho}^{cc}$ is a Banach lattice as any Hilbert space, then so does $E$. As any Banach lattice $E$ has the l.u.st. property [\cite{DiestAbsSumOps}, theorem 17.1], then so does $A$, because the l.u.st. property is inherited by complemented subspaces.

Assume $A$ contains $\mathcal{B}(\ell_2(\mathbb{N}_n))$ as ${}^*$-subalgebra for arbitrarily large $n\in\mathbb{N}$. In fact this copy of $\mathcal{B}(\ell_2(\mathbb{N}_n))$ is $1$-complemented in $A$ [\cite{LauLoyWillisAmnblOfBanAndCStarAlgsOfLCG}, lemma 2.1]. Therefore we have an inequality for local unconditional constants $\kappa_u(\mathcal{B}(\ell_2(\mathbb{N}_n)))\leq \kappa_u(A)$. By theorem 5.1 in \cite{GorLewAbsSmOpAndLocUncondStrct} we know that $\lim_n \kappa_u(\mathcal{B}(\ell_2(\mathbb{N}_n)))=+\infty$, so $\kappa_u(A)=+\infty$. This contradicts the l.u.st. property of $A$. Hence $A$ can't contain $\mathcal{B}(\ell_2(\mathbb{N}_n))$ as ${}^*$-subalgebra for arbitrary $n\in\mathbb{N}$.
\end{proof}

As the proposition \ref{TopInjIdHaveLUST} shows $C^*$-algebras that are topologically injective over themselves can't contain $\mathcal{B}(\ell_2(\mathbb{N}_n))$ as ${}^*$-subalgebra for arbitrary $n\in\mathbb{N}$. Such $C^*$-algebras are called subhomogeneous, and in fact [\cite{BlackadarOpAlg}, proposition IV.1.4.3] they can be treated as closed ${}^*$-subalgebras of $M_n(C(K))$ for some compact Hausdorff space $K$ and some natural number $n$. For more on subhomogeneous $C^*$-algebras see [\cite{BlackadarOpAlg}, section IV.1.4]. 

We shall give two important examples of non commutative $C^*$-algebras that are topologically injective over themselves.

\begin{proposition}\label{FinDimBHModTopInj} Let $H$ be a finite dimensional Hilbert space. Then $\mathcal{B}(H)$ is topologically injective as $\mathcal{B}(H)$-module. 
\end{proposition}
\begin{proof} Note that $\mathcal{B}(H)\isom{\mathbf{mod}_1-\mathcal{B}(H)}\mathcal{N}(H)^*$, and the the result immediately follows from propositions \ref{FinDimNHModTopProjFlat} and \ref{DualMetTopProjIsMetrInj}.
\end{proof}

\begin{proposition}\label{CKMatrixModTopInj} Let $K$ be a Stonean space and $n\in\mathbb{N}$, then $M_n(C(K))$ is topologically injective $M_n(C(K))$-module.
\end{proposition}
\begin{proof} For a fixed $s\in K$ by $\mathbb{C}_s$ we denote the right $C(K)$-module $\mathbb{C}$ with outer action defined by $z\cdot a=a(s)z$ for all $a\in C(K)$ and $z\in\mathbb{C}$. By $M_n(\mathbb{C}_s)$ we denote the right Banach $M_n(C(K))$-module $M_n(\mathbb{C})$ with outer action defined by $(x\cdot a)_{i,j}=\sum_{k=1}^n x_{i,k}a_{k,j}(s)$ for $a\in M_n(C(K))$ and $x\in M_n(\mathbb{C}_s)$. The $C^*$-algebra $M_n(C(K))$ is nuclear [\cite{BroOzaCStarAlgFinDimApprox}, corollary 2.4.4], then by [\cite{HaaNucCStarAlgAmen}, theorem 3.1] this algebra is relatively amenable and even $1$-amenable [\cite{RundeAmenConstFour}, example 2]. Since $M_n(\mathbb{C}_s)$ is finite dimensional, it is an $\mathscr{L}_{1, C}^g$-space for some constant $C\geq 1$ independent of $s$. Thus, by proposition \ref{MetTopEssL1FlatModAoverAmenBanAlg} the $M_n(C(K))$-module $M_n(\mathbb{C}_s)^*$ is $C$-topologically flat. Since the latter module is essential, by proposition \ref{CTopFlatCharac} the right $M_n(C(K))$-module $M_n(\mathbb{C}_s)^{**}$ is $C$-topologically injective. Note that $M_n(\mathbb{C}_s)^{**}$ is isometrically isomorphic to $M_n(\mathbb{C}_s)$ as right $M_n(C(K))$-module, so from proposition \ref{MetTopInjModProd} we get that $\bigoplus_\infty\{M_n(\mathbb{C}_s):s\in K\}$ is topologically injective as $M_n(C(K))$-module.

Note that by proposition \ref{MetInjCStarAlgCharac} the $C(K)$-module $C(K)$ is metrically injective, therefore an isometric $C(K)$-morphism $\widetilde{\rho}:C(K)\to\bigoplus_\infty\{ \mathbb{C}_s:s\in K\}:x\mapsto \bigoplus_\infty\{x(s):s\in K\}$ admits a left inverse contractive $C(K)$-morphism $\widetilde{\tau}:\bigoplus_\infty\{ \mathbb{C}_s:s\in K\} \to C(K)$. It is routine to check now that linear operators
$$
\rho:M_n(C(K))\to\bigoplus\nolimits_\infty\{M_n(\mathbb{C}_s):s\in K\}:x\mapsto \bigoplus\nolimits_\infty\{(x_{i,j}(s))_{i,j\in\mathbb{N}_n}:s\in K\}
$$
$$
\tau:\bigoplus\nolimits_\infty\{M_n(\mathbb{C}_s):s\in K\}\to M_n(C(K)):y\mapsto \left(\widetilde{\tau}\left(\bigoplus\nolimits_\infty\{y_{s,i,j}:s\in K\}\right)\right)_{i,j\in\mathbb{N}_n}
$$
are bounded $M_n(C(K))$-morphisms such that $\tau \rho=1_{M_n(C(K))}$. That is $M_n(C(K))$ is a retract of topologically injective $M_n(C(K))$-module $\bigoplus_\infty\{M_n(\mathbb{C}_s):s\in K\}$ in $\mathbf{mod}_1-M_n(C(K))$. Finally, from proposition \ref{RetrMetTopInjIsMetTopInj} we conclude that $M_n(C(K))$ is topologically injective $M_n(C(K))$-module.
\end{proof}

\begin{theorem}\label{TopInjAWStarAlgCharac} Let $A$ be a $C^*$-algebra. Then the following are equivalent:

$i)$ $A$ is an $AW^*$-algebra which is topologically injective as $A$-module;

$ii)$ $A=\bigoplus_\infty\{M_{n_\lambda}(C(K_\lambda)):\lambda\in\Lambda\}$ for some finite families of natural numbers $(n_\lambda)_{\lambda\in\Lambda}$ and Stonean spaces $(K_\lambda)_{\lambda\in\Lambda}$.
\end{theorem}
\begin{proof}$i)\implies ii)$ From proposition 6.6 in \cite{SmithDecompPropCStarAlg} we know that an $AW^*$-algebra is either isomorphic as $C^*$-algebra to $\bigoplus_\infty\{M_{n_\lambda}(C(K_\lambda)):\lambda\in\Lambda\}$ for some finite families of natural numbers $(n_\lambda)_{\lambda\in\Lambda}$ and Stonean spaces $(K_\lambda)_{\lambda\in\Lambda}$ or contains a ${}^*$-subalgebra $\bigoplus_\infty\{ \mathcal{B}(\ell_2(\mathbb{N}_n)):n\in\mathbb{N}\}$. The second option is canceled out by proposition \ref{TopInjIdHaveLUST}.

$ii)\implies i)$ For each $\lambda\in\Lambda$ the algebra $M_{n_\lambda}(C(K_\lambda))$ is unital because $K_\lambda$ is compact. Therefore $M_{n_\lambda}(C(K_\lambda))$ is faithful as $M_{n_\lambda}(C(K_\lambda))$-module. It is also topologically injective as $M_{n_\lambda}(C(K_\lambda))$-module by proposition  \ref{CKMatrixModTopInj}. Now the topological injectivity of $A$ as $A$-module follows from paragraph $ii)$ of proposition \ref{MetTopProjInjFlatUnderSumOfAlg} with $p=\infty$ and $X_\lambda=A_\lambda=M_{n_\lambda}(C(K_\lambda))$ for all $\lambda\in\Lambda$. 

For all $\lambda\in\Lambda$ the algebra $C(K_\lambda)$ is an $AW^*$-algebra, because $K_\lambda$ is a Stonean space [\cite{BerbBaerStarRings}, theorem 1.7.1]. Therefore $M_{n_\lambda}(C(K_\lambda))$ is an $AW^*$-algebra too [\cite{BerbBaerStarRings}, corollary 9.62.1]. Finally $A$ is an $AW^*$-algebra as $\bigoplus_\infty$-sum of such algebras [\cite{BerbBaerStarRings}, proposition 1.10.1].
\end{proof}

It is desirable to prove that any topologically injective over itself $C^*$-algebra is an $AW^*$-algebra, but it seems to be a challenge even in the commutative case.

%----------------------------------------------------------------------------------------
%	Flat ideals of C^*-algebras
%----------------------------------------------------------------------------------------

\subsection{Flat ideals of \texorpdfstring{$C^*$}{C*}-algebras}
\label{SubSectionFlatIdealsOfCStarAlgebras}

By considering flatness we finalize this lengthy investigations of ideals of $C^*$-algebras.

\begin{proposition}\label{IdealofCstarAlgisMetTopFlat} Let $I$ be a left ideal of a $C^*$-algebra $A$. Then $I$ is metrically and topologically flat as $A$-module.
\end{proposition}
\begin{proof} From [\cite{HelBanLocConvAlg}, proposition 4.7.78] it follows that $I$ has a right contractive identity. It remains to apply proposition \ref{MetTopFlatIdealsInUnitalAlg}.
\end{proof}

\begin{proposition}\label{CStarAlgIsL1IfFinDim} Let $A$ be a $C^*$-algebra, then $A$ is an $\langle$~$L_1$-space / $\mathscr{L}_1^g$-space~$\rangle$ iff $\langle$~$\operatorname{dim}(A)\leq 1$ / $A$ is finite dimensional~$\rangle$.
\end{proposition}
\begin{proof} Assume $A$ is an $\mathscr{L}_1^g$-space, then $A^{**}$ is complemented in some $L_1$-space [\cite{DefFloTensNorOpId}, corollary 23.2.1(2)]. Since $A$ isometrically embeds in its second dual via $\iota_{A}$ we may regard $A$ as closed subspace of some $L_1$-space. The latter space is weakly sequentially complete [\cite{WojBanSpForAnalysts}, corollary III.C.14]. The property of being weakly sequentially complete is preserved by closed subspaces, therefore $A$ is weakly sequentially complete too. By proposition 2 in \cite{SakWeakCompOpOnOpAlg} every weakly sequentially complete $C^*$-algebra is finite dimensional, hence $A$ is finite dimensional. Conversely, if $A$ is finite dimensional it is an $\mathscr{L}_1^g$-space as any finite dimensional Banach space.

Assume $A$ is an $L_1$-space and, a fortiori, an $\mathscr{L}_1^g$-space. As was noted above $A$ is a finite dimensional, so $A\isom{\mathbf{Ban}_1}\ell_1(\mathbb{N}_n)$ for $n=\operatorname{dim}(A)$. On the other hand, $A$ is a finite dimensional $C^*$-algebra, so it is isometrically isomorphic to $\bigoplus_\infty\{ \mathcal{B}(\ell_2(\mathbb{N}_{n_k})):k\in\mathbb{N}_m\}$ for some natural numbers $(n_k)_{k\in\mathbb{N}_m}$ [\cite{DavCSatrAlgByExmpl}, theorem III.1.1]. Assume $\operatorname{dim}(A)>1$, then $A$ contains an isometric copy of $\ell_\infty(\mathbb{N}_2)$. Therefore we have an isometric embedding of $\ell_\infty(\mathbb{N}_2)$ into $\ell_1(\mathbb{N}_n)$. This is impossible by theorem 1 from \cite{LyubIsomEmdbFinDimLp}. Therefore $\operatorname{dim}(A)\leq 1$. 
\end{proof}

\begin{proposition}\label{CStarAlgIsTopFlatOverItsIdeal} Let $I$ be a proper two-sided ideal of a  $C^*$-algebra $A$. Then the following are equivalent:

$i)$ $A$ is $\langle$~metrically / topologically~$\rangle$ flat $I$-module;

$ii)$ $\langle$~$\operatorname{dim}(A)=1$, $I=\{0\}$ / $A/I$ is finite dimensional ~$\rangle$.

\end{proposition}
\begin{proof} We may regard $I$ as an  ideal of unitazation $A_\#$ of $A$. Since $I$ is a two-sided ideal, then it has a contractive approximate identity $(e_\nu)_{\nu\in N}$ such that $0\leq e_\nu\leq e_{A_\#}$ [\cite{HelBanLocConvAlg}, proposition 4.7.79]. As a corollary $\sup_{\nu\in N}\Vert e_{A_\#}-e_\nu\Vert\leq 1$. Since $I$ has an approximate identity we also have $A_{ess}:=\operatorname{cl}_A(IA)=I$. Since $I$ is a two sided ideal then $A/I$ is a $C^*$-algebra [\cite{HelBanLocConvAlg}, theorem 4.7.81].

Assume, $A$ is a metrically flat $I$-module. Since $\sup_{\nu\in N}\Vert e_{A_\#}-e_\nu\Vert\leq 1$, then paragraph $ii)$ of proposition \ref{DualBanModDecomp} tells us that $(A/A_{ess})^*=(A/I)^*$ is a retract of $A^*$ in $\mathbf{mod}_1-I$. Now from propositions \ref{MetTopFlatCharac} and \ref{RetrMetTopInjIsMetTopInj} it follows that $A/I$ is metrically flat $I$-module. Since this is an annihilator module, then from proposition \ref{MetTopFlatAnnihModCharac} it follows that $I=\{0\}$ and $A/I$ is an $L_1$-space. Now from proposition \ref{CStarAlgIsL1IfFinDim} we get that $\operatorname{dim}(A/I)\leq 1$. Since $A$ contains a proper ideal $I=\{0\}$, then $\operatorname{dim}(A)=1$. Conversely, if $I=\{0\}$ and $\operatorname{dim}(A)=1$, then we have an annihilator $I$-module $A$ which is isometrically isomorphic to $\ell_1(\mathbb{N}_1)$. By proposition \ref{MetTopFlatAnnihModCharac} it is metrically flat. 

By proposition \ref{TopFlatModCharac} the $I$-module $A$ is topologically flat iff $A_{ess}=I$ and $A/A_{ess}=A/I$ are topologically flat $I$-modules. By proposition \ref{IdealofCstarAlgisMetTopFlat} the ideal $I$ is topologically flat $I$-module. By proposition \ref{MetTopFlatAnnihModCharac} the annihilator $I$-module $A/I$ is topologically flat iff it is an $\mathscr{L}_1^g$-space. By proposition \ref{CStarAlgIsL1IfFinDim} this is equivalent to $A/I$ being finite dimensional.
\end{proof}

%----------------------------------------------------------------------------------------
%	\mathcal{K}(H)- and \mathcal{B}(H)-modules
%----------------------------------------------------------------------------------------

\subsection{\texorpdfstring{$\mathcal{K}(H)$}{K(H)}- and \texorpdfstring{$\mathcal{B}(H)$}{B(H)}-modules}
\label{SubSectionKHAndBHModules}

In this section we apply general results on ideals obtained above to classical modules over $C^*$-algebras.  For a given Hilbert space $H$ we consider $\mathcal{B}(H)$, $\mathcal{K}(H)$ and $\mathcal{N}(H)$ as left and right Banach modules over $\mathcal{B}(H)$ and $\mathcal{K}(H)$. For all modules the module action is just the composition of operators. The Schatten-von Neumann isomorphisms $\mathcal{N}(H)\isom{\mathbf{Ban}_1}\mathcal{K}(H)^*$, $\mathcal{B}(H)\isom{\mathbf{Ban}_1}\mathcal{N}(H)^*$ (see [\cite{TakThOpAlgVol1}, theorems II.1.6, II.1.8]) will be of use here. They are in fact isomorphisms of left and right $\mathcal{B}(H)$-modules and a fortiori of $\mathcal{K}(H)$-modules.

\begin{proposition}\label{KHAndBHModBH} Let $H$ be a Hilbert space. Then

$i)$ $\mathcal{B}(H)$ is metrically and topologically projective and flat as $\mathcal{B}(H)$-module;

$ii)$ $\mathcal{B}(H)$ is metrically or topologically projective or flat as $\mathcal{K}(H)$-module iff $H$ is finite dimensional;

$iii)$ $\mathcal{B}(H)$ is topologically injective as $\mathcal{B}(H)$- or $\mathcal{K}(H)$-module iff $H$ is finite dimensional;

$iv)$ $\mathcal{B}(H)$ is metrically injective as $\mathcal{B}(H)$- or $\mathcal{K}(H)$-module iff $\dim(H)\leq 1$.
\end{proposition}
\begin{proof} $i)$ Since $\mathcal{B}(H)$ is a unital algebra it is metrically and topologically projective as $\mathcal{B}(H)$-module by proposition \ref{UnitalAlgIsMetTopProj}. Both results regarding flatness follow from proposition \ref{MetTopProjIsMetTopFlat}.

$ii)$ For infinite dimensional $H$ the Banach space $\mathcal{B}(H)/\mathcal{K}(H)$ is of infinite dimension, so by proposition \ref{CStarAlgIsTopFlatOverItsIdeal} the module $\mathcal{B}(H)$ neither topologically nor metrically flat as $\mathcal{K}(H)$-module. Both claims regarding projectivity follow from proposition \ref{MetTopProjIsMetTopFlat}. If $H$ is finite dimensional, then $\mathcal{K}(H)=\mathcal{B}(H)$, so the result follows from paragraph $i)$.

$iii)$ If $H$ is infinite dimensional, then $\mathcal{B}(H)$ contains $\mathcal{B}(\ell_2(\mathbb{N}_n))$ as ${}^*$-subalgebra for all $n\in\mathbb{N}$. By proposition \ref{TopInjIdHaveLUST} we get that $\mathcal{B}(H)$ is not topologically injective as $\mathcal{B}(H)$-module. The rest follows from paragraph $i)$ of proposition \ref{MetTopInjUnderChangeOfAlg}. If $H$ is finite dimensional, then $\mathcal{K}(H)=\mathcal{B}(H)$, so the result follows from proposition \ref{FinDimBHModTopInj}.

$iv)$ If $\dim(H)>1$, then $C^*$-algebra $\mathcal{B}(H)$ is not commutative. By  proposition \ref{MetInjCStarAlgCharac} we get that it is not metrically injective as $\mathcal{B}(H)$-module. Now from paragraph $i)$ of \ref{MetTopInjUnderChangeOfAlg} we get that $\mathcal{B}(H)$ is not metrically injective as $\mathcal{K}(H)$-module. If $\dim(H)\leq 1$ both claims obviously follow from \ref{MetInjCStarAlgCharac}.
\end{proof}

\begin{proposition}\label{KHAndBHModKH} Let $H$ be a Hilbert space. Then 

$i)$ $\mathcal{K}(H)$ is metrically and topologically flat as $\mathcal{B}(H)$- or $\mathcal{K}(H)$-module;

$ii)$ $\mathcal{K}(H)$ is metrically or topologically projective as $\mathcal{B}(H)$- or $\mathcal{K}(H)$-module iff $H$ is finite dimensional;

$iii)$ $\mathcal{K}(H)$ is topologically injective as $\mathcal{B}(H)$- or $\mathcal{K}(H)$-module iff $H$ is finite dimensional;

$iv)$ $\mathcal{K}(H)$ is metrically injective as $\mathcal{B}(H)$- or $\mathcal{K}(H)$-module iff $\dim(H)\leq 1$.
\end{proposition}
\begin{proof} Let $A$ be either $\mathcal{B}(H)$ or $\mathcal{K}(H)$. Note that $\mathcal{K}(H)$ is a two-sided ideal of $A$. 

$i)$ Recall that $\mathcal{K}(H)$ has a contractive approximate identity consisting of orthogonal projections onto all finite-dimensional subspaces of $H$. Since $\mathcal{K}(H)$ is a two-sided ideal of $A$, then the result follows from proposition \ref{IdealofCstarAlgisMetTopFlat}.

$ii)$, $iii)$, $iv)$ If $H$ is infinite dimensional, then $\mathcal{K}(H)$ is not unital as Banach algebra. From corollary \ref{BiIdealOfCStarAlgMetTopProjCharac} and proposition \ref{MetTopInjOfId} the $A$-module $\mathcal{K}(H)$ is neither metrically nor topologically projective or injective. If $H$ is finite dimensional, then $\mathcal{K}(H)=\mathcal{B}(H)$, so both results follow from paragraphs $i)$, $iii)$ and $iv)$ of proposition \ref{KHAndBHModBH}.
\end{proof}

\begin{proposition}\label{KHAndBHModNH} Let $H$ be a Hilbert space. Then

$i)$ $\mathcal{N}(H)$ is metrically and topologically injective as $\mathcal{B}(H)$- or $\mathcal{K}(H)$-module;

$ii)$ $\mathcal{N}(H)$ is topologically projective or flat as $\mathcal{B}(H)$- or $\mathcal{K}(H)$-module iff $H$ is finite dimensional;

$iii)$ $\mathcal{N}(H)$ is metrically projective or flat as $\mathcal{B}(H)$- or $\mathcal{K}(H)$-module iff $\dim(H)\leq 1$.
\end{proposition}
\begin{proof} Let $A$ be either $\mathcal{B}(H)$ or $\mathcal{K}(H)$.

$i)$ Note that $\mathcal{N}(H)\isom{\mathbf{mod}_1-A}\mathcal{K}(H)^*$, so the result follows from proposition \ref{MetTopFlatCharac} and paragraph $i)$ of proposition \ref{KHAndBHModKH}.

$ii)$ Assume $H$ is infinite dimensional. Note that $\mathcal{B}(H)\isom{\mathbf{mod}_1-A}\mathcal{N}(H)^*$, so from proposition \ref{DualMetTopProjIsMetrInj} and paragraph $iii)$ of proposition \ref{KHAndBHModBH} we get that $\mathcal{N}(H)$ is not topologically projective as $A$-module. Both results regarding flatness follow from proposition \ref{MetTopProjIsMetTopFlat}. If $H$ is finite dimensional we use proposition \ref{FinDimNHModTopProjFlat}.

$iii)$ Assume $\dim(H)>1$, then by paragraph $iv)$ of proposition \ref{KHAndBHModBH} the $A$-module $\mathcal{B}(H)$ is not metrically injective. Since $\mathcal{B}(H)\isom{\mathbf{mod}_1-A}\mathcal{N}(H)^*$, then from proposition \ref{MetTopFlatCharac} we get that $\mathcal{N}(H)$ is not metrically flat as $A$-module. By proposition \ref{MetTopProjIsMetTopFlat}, it is not metrically projective as $A$-module. If $\dim(H)\leq 1$, then $\mathcal{N}(H)=\mathcal{K}(H)=\mathcal{B}(H)$, so both results follow from paragraph $i)$ of proposition \ref{KHAndBHModBH}.
\end{proof}

\begin{proposition}\label{KHAndBHModsRelTh} Let $H$ be a Hilbert space. Then

$i)$ as $\mathcal{K}(H)$-modules $\mathcal{N}(H)$ is relatively projective injective and flat, $\mathcal{K}(H)$ is relatively projective and flat, but relatively injective only for finite dimensional $H$, $\mathcal{B}(H)$ is relatively injective and flat, but relatively projective only for finite dimensional $H$;

$ii)$ as $\mathcal{B}(H)$-modules $\mathcal{N}(H)$ is relatively projective injective and flat, $\mathcal{K}(H)$ is relatively projective and flat, $\mathcal{B}(H)$ is relatively projective, injective and flat.

\end{proposition}
\begin{proof} $i)$ Note that $H$ is relatively projective as $\mathcal{K}(H)$-module [\cite{HelBanLocConvAlg}, theorem 7.1.27], so from proposition 7.1.13 in \cite{HelBanLocConvAlg} we get that $\mathcal{N}(H)\isom{\mathcal{K}(H)-\mathbf{mod}_1}H\projtens H^*$ is also relatively projective as $\mathcal{K}(H)$-module. By theorem IV.2.16 in \cite{HelHomolBanTopAlg} the $\mathcal{K}(H)$-module $\mathcal{K}(H)$ is relatively projective. A fortiori $\mathcal{N}(H)$ and $\mathcal{K}(H)$ are relatively flat $\mathcal{K}(H)$-modules [\cite{HelBanLocConvAlg}, proposition 7.1.40], so $\mathcal{N}(H)\isom{\mathbf{mod}_1-\mathcal{K}(H)}\mathcal{K}(H)^*$ and  $\mathcal{B}(H)\isom{\mathbf{mod}_1-\mathcal{K}(H)}\mathcal{N}(H)^*$ are relatively injective $\mathcal{K}(H)$-modules. From [\cite{RamsHomPropSemgroupAlg}, proposition 2.2.8  (i)] we know that a Banach algebra relatively injective over itself as a right module, necessarily has a left identity. Therefore $\mathcal{K}(H)$ is not relatively injective $\mathcal{K}(H)$-module for infinite dimensional $H$. If $H$ is finite dimensional, then $\mathcal{K}(H)$-module $\mathcal{K}(H)$ is relatively injective because $\mathcal{K}(H)=\mathcal{B}(H)$ and $\mathcal{B}(H)$ is relatively injective $\mathcal{K}(H)$-module as was shown above. By corollary 5.5.64 from \cite{DalBanAlgAutCont} the algebra $\mathcal{K}(H)$ is relatively amenable, so all its left modules are relatively flat [\cite{HelBanLocConvAlg}, theorem 7.1.60]. In particular $\mathcal{B}(H)$ is relatively flat $\mathcal{K}(H)$-module. From [\cite{HelHomolBanTopAlg}, exercise V.2.20] we know that $\mathcal{B}(H)$ is not relatively projective as $\mathcal{K}(H)$-module when $H$ is infinite dimensional. If $H$ is finite dimensional then $\mathcal{B}(H)$ is relatively projective $\mathcal{K}(H)$-module because $\mathcal{B}(H)=\mathcal{K}(H)$ and $\mathcal{K}(H)$ is relatively projective $\mathcal{K}(H)$-module as was shown above.

$ii)$ From proposition \ref{KHAndBHModBH} paragraph $i)$ and proposition \ref{MetProjIsTopProjAndTopProjIsRelProj} it follows that $\mathcal{B}(H)$ is relatively projective $\mathcal{B}(H)$-module. From [\cite{RamsHomPropSemgroupAlg}, propositions 2.3.3, 2.3.4] we know that $\langle$~an essential relatively projective / a faithful relatively injective~$\rangle$ module over ideal of Banach algebra is $\langle$~relatively projective / relative injective~$\rangle$ over algebra itself. Since $\mathcal{K}(H)$ and $\mathcal{N}(H)$ are essential and faithful $\mathcal{K}(H)$-modules, then from results of previous paragraph $\mathcal{N}(H)$ is relatively projective and injective, while $\mathcal{K}(H)$ is relatively projective as $\mathcal{B}(H)$-modules. Now, by [\cite{HelBanLocConvAlg}, proposition 7.1.40] all aforementioned modules are relatively flat $\mathcal{B}(H)$-modules. In particular $\mathcal{B}(H)\isom{\mathbf{mod}_1-\mathcal{B}(H)}\mathcal{N}(H)^*$ is relatively injective $\mathcal{B}(H)$-module.
\end{proof}

Results of this section are summarized in the following three tables. Each cell contains a condition under which the respective module has the respective property and propositions where this is proved. We use ??? symbol to indicate open problems. These tables confirm that the property of being homologically trivial in metric and topological theory is too restrictive. It is easier to mention cases where metric and topological properties coincide with relative ones: flatness of $\mathcal{K}(H)$ as $\mathcal{B}(H)$- or $\mathcal{K}(H)$-module, injectivity of $\mathcal{N}(H)$ as $\mathcal{B}(H)$- or $\mathcal{K}(H)$-module, projectivity and flatness of $\mathcal{B}(H)$-module $\mathcal{B}(H)$. In the remaining cases $H$ needs to be at least finite dimensional in order to these properties be equivalent in metric, topological and relative theory.


\begin{scriptsize}
\begin{longtable}{|c|c|c|c|c|c|c|} 
\multicolumn{7}{c}{\mbox{Homologically trivial $\mathcal{K}(H)$- and $\mathcal{B}(H)$-modules in metric theory}}                                                                                                                                                                                                                                                                                                                                                                                                                                                    \\
				 
\hline          & \multicolumn{3}{c|}{$\mathcal{K}(H)$-modules}                                                                                                                                                                                                                     & \multicolumn{3}{c|}{$\mathcal{B}(H)$-modules}                                                                                                                                                                                                                       \\
\hline
                & \mbox{Projectivity}                                                                   & \mbox{Injectivity}                                                                    & \mbox{Flatness}                                                                        & \mbox{Projectivity}                                                                    & \mbox{Injectivity}                                                                     & \mbox{Flatness}                                                                        \\ 
\hline
$\mathcal{N}(H)$  & \begin{tabular}{@{}c@{}}$\dim(H)\leq 1$ \\ \ref{KHAndBHModNH}\end{tabular}            & \begin{tabular}{@{}c@{}}$H$\mbox{ is any }  \\ \ref{KHAndBHModNH}\end{tabular}        & \begin{tabular}{@{}c@{}}$\dim(H)\leq 1$ \\ \ref{KHAndBHModNH}\end{tabular}             & \begin{tabular}{@{}c@{}}$\dim(H)\leq 1$ \\ \ref{KHAndBHModNH}\end{tabular}             & \begin{tabular}{@{}c@{}}$H$\mbox{ is any }  \\ \ref{KHAndBHModNH}\end{tabular}         & \begin{tabular}{@{}c@{}}$\dim(H)\leq 1$ \\ \ref{KHAndBHModNH}\end{tabular}             \\
\hline
$\mathcal{B}(H)$  & \begin{tabular}{@{}c@{}}$\dim(H)<\aleph_0$ \\ \ref{KHAndBHModBH}\end{tabular}         & \begin{tabular}{@{}c@{}}$\dim(H)\leq 1$ \\ \ref{KHAndBHModBH}\end{tabular}            & \begin{tabular}{@{}c@{}}$\dim(H)<\aleph_0$ \\ \ref{KHAndBHModBH}\end{tabular}          & \begin{tabular}{@{}c@{}}$H$\mbox{ is any } \\ \ref{KHAndBHModBH}\end{tabular}          & \begin{tabular}{@{}c@{}}$\dim(H)\leq 1$ \\ \ref{KHAndBHModBH}\end{tabular}             & \begin{tabular}{@{}c@{}}$H$\mbox{ is any } \\ \ref{KHAndBHModBH}\end{tabular}          \\ 
\hline
$\mathcal{K}(H)$  & \begin{tabular}{@{}c@{}}$\dim(H)<\aleph_0$ \\ \ref{KHAndBHModKH}\end{tabular}         & \begin{tabular}{@{}c@{}}$\dim(H)\leq 1$ \\ \ref{KHAndBHModKH}\end{tabular}            & \begin{tabular}{@{}c@{}}$H$\mbox{ is any } \\ \ref{KHAndBHModKH}\end{tabular}          & \begin{tabular}{@{}c@{}}$\dim(H)<\aleph_0$ \\ \ref{KHAndBHModKH}\end{tabular}          & \begin{tabular}{@{}c@{}}$\dim(H)\leq 1$ \\ \ref{KHAndBHModKH}\end{tabular}             & \begin{tabular}{@{}c@{}}$H$\mbox{ is any } \\ \ref{KHAndBHModKH}\end{tabular}          \\ 
\hline

\multicolumn{7}{c}{\mbox{Homologically trivial $\mathcal{K}(H)$- and $\mathcal{B}(H)$-modules in topological theory}}                                                                                                                                                                                                                                                                                                                                                                                                                                               \\
					 
\hline          & \multicolumn{3}{c|}{$\mathcal{K}(H)$-modules}                                                                                                                                                                                                                     & \multicolumn{3}{c|}{$\mathcal{B}(H)$-modules}                                                                                                                                                                                                                       \\
\hline
                & \mbox{Projectivity}                                                                   & \mbox{Injectivity}                                                                    & \mbox{Flatness}                                                                        & \mbox{Projectivity}                                                                    & \mbox{Injectivity}                                                                     & \mbox{Flatness}                                                                        \\ 
\hline
$\mathcal{N}(H)$  & \begin{tabular}{@{}c@{}}$\dim(H)<\aleph_0$ \\ \ref{KHAndBHModNH}\end{tabular}         & \begin{tabular}{@{}c@{}}$H$\mbox{ is any } \\ \ref{KHAndBHModNH}\end{tabular}         & \begin{tabular}{@{}c@{}}$\dim(H)<\aleph_0$ \\ \ref{KHAndBHModNH}\end{tabular}          & \begin{tabular}{@{}c@{}}$\dim(H)<\aleph_0$ \\ \ref{KHAndBHModNH}\end{tabular}          & \begin{tabular}{@{}c@{}}$H$\mbox{ is any } \\ \ref{KHAndBHModNH}\end{tabular}          & \begin{tabular}{@{}c@{}}$\dim(H)<\aleph_0$ \\ \ref{KHAndBHModNH}\end{tabular}          \\
\hline
$\mathcal{B}(H)$  & \begin{tabular}{@{}c@{}}$\dim(H)<\aleph_0$ \\ \ref{KHAndBHModBH}\end{tabular}         & \begin{tabular}{@{}c@{}}$\dim(H)<\aleph_0$ \\ \ref{KHAndBHModBH}\end{tabular}         & \begin{tabular}{@{}c@{}}$\dim(H)<\aleph_0$ \\ \ref{KHAndBHModBH}\end{tabular}          & \begin{tabular}{@{}c@{}}$H$\mbox{ is any } \\ \ref{KHAndBHModBH}\end{tabular}          & \begin{tabular}{@{}c@{}}$\dim(H)<\aleph_0$ \\ \ref{KHAndBHModBH}\end{tabular}          & \begin{tabular}{@{}c@{}}$H$\mbox{ is any } \\ \ref{KHAndBHModBH}\end{tabular}          \\ 
\hline
$\mathcal{K}(H)$  & \begin{tabular}{@{}c@{}}$\dim(H)<\aleph_0$ \\ \ref{KHAndBHModKH}\end{tabular}         & \begin{tabular}{@{}c@{}}$\dim(H)<\aleph_0$ \\ \ref{KHAndBHModKH}\end{tabular}         & \begin{tabular}{@{}c@{}}$H$\mbox{ is any } \\ \ref{KHAndBHModKH}\end{tabular}          & \begin{tabular}{@{}c@{}}$\dim(H)<\aleph_0$ \\ \ref{KHAndBHModKH}\end{tabular}          & \begin{tabular}{@{}c@{}}$\dim(H)<\aleph_0$ \\ \ref{KHAndBHModKH}\end{tabular}          & \begin{tabular}{@{}c@{}}$H$\mbox{ is any } \\ \ref{KHAndBHModKH}\end{tabular}          \\ 
\hline

\multicolumn{7}{c}{\mbox{Homologically trivial $\mathcal{K}(H)$- and $\mathcal{B}(H)$-modules in relative theory}}                                                                                                                                                                                                                                                                                                                                                                                                                                                  \\

\hline          & \multicolumn{3}{c|}{$\mathcal{K}(H)$-modules}                                                                                                                                                                                                                     & \multicolumn{3}{c|}{$\mathcal{B}(H)$-modules}                                                                                                                                                                                                                       \\
\hline
                & \mbox{Projectivity}                                                                   & \mbox{Injectivity}                                                                    & \mbox{Flatness}                                                                        & \mbox{Projectivity}                                                                    & \mbox{Injectivity}                                                                     & \mbox{Flatness}                                                                        \\ 
\hline
$\mathcal{N}(H)$  & \begin{tabular}{@{}c@{}}$H$\mbox{ is any } \\ \ref{KHAndBHModsRelTh}, i)\end{tabular} & \begin{tabular}{@{}c@{}}$H$\mbox{ is any } \\ \ref{KHAndBHModsRelTh}, i)\end{tabular} & \begin{tabular}{@{}c@{}}$H$\mbox{ is any } \\ \ref{KHAndBHModsRelTh}, i)\end{tabular}  & \begin{tabular}{@{}c@{}}$H$\mbox{ is any } \\ \ref{KHAndBHModsRelTh}, ii)\end{tabular} & \begin{tabular}{@{}c@{}}$H$\mbox{ is any } \\ \ref{KHAndBHModsRelTh}, ii)\end{tabular} & \begin{tabular}{@{}c@{}}$H$\mbox{ is any } \\ \ref{KHAndBHModsRelTh}, ii)\end{tabular} \\
\hline
$\mathcal{B}(H)$  & \begin{tabular}{@{}c@{}}$\dim(H)<\aleph_0$ \\ \ref{KHAndBHModsRelTh}, i)\end{tabular} & \begin{tabular}{@{}c@{}}$H$\mbox{ is any } \\ \ref{KHAndBHModsRelTh}, i)\end{tabular} & \begin{tabular}{@{}c@{}}$H$\mbox{ is any } \\ \ref{KHAndBHModsRelTh}, i)\end{tabular}  & \begin{tabular}{@{}c@{}}$H$\mbox{ is any } \\ \ref{KHAndBHModsRelTh}, ii)\end{tabular} & \begin{tabular}{@{}c@{}}$H$\mbox{ is any } \\ \ref{KHAndBHModsRelTh}, ii)\end{tabular} & \begin{tabular}{@{}c@{}}$H$\mbox{ is any } \\ \ref{KHAndBHModsRelTh}, ii)\end{tabular} \\
\hline
$\mathcal{K}(H)$  & \begin{tabular}{@{}c@{}}$H$\mbox{ is any } \\ \ref{KHAndBHModsRelTh}, i)\end{tabular} & \begin{tabular}{@{}c@{}}$\dim(H)<\aleph_0$ \\ \ref{KHAndBHModsRelTh}, i)\end{tabular} & \begin{tabular}{@{}c@{}}$H$\mbox{ is any } \\ \ref{KHAndBHModsRelTh}, i)\end{tabular}  & \begin{tabular}{@{}c@{}}$H$\mbox{ is any } \\ \ref{KHAndBHModsRelTh}, ii)\end{tabular} & \begin{tabular}{@{}c@{}} ??? \end{tabular}                                             & \begin{tabular}{@{}c@{}}$H$\mbox{ is any } \\ \ref{KHAndBHModsRelTh}, ii)\end{tabular} \\
\hline
\end{longtable}
\end{scriptsize}

%----------------------------------------------------------------------------------------
%	c_0(\Lambda)- and l_infty(\Lambda)-modules
%----------------------------------------------------------------------------------------

\subsection{\texorpdfstring{$c_0(\Lambda)$}{c0(Lambda)}- and \texorpdfstring{$\ell_\infty(\Lambda)$}{lInfty(Lambda)}-modules}
\label{SubSectionc0AndlInftyModules}

We continue our study of modules over $C^*$-algebras and move to commutative examples. For a given index set $\Lambda$ we consider spaces $c_0(\Lambda)$ and $\ell_p(\Lambda)$ for $1\leq p\leq+\infty$ as left and right modules over algebras $c_0(\Lambda)$ and $\ell_\infty(\Lambda)$. For all these modules the module action is just the pointwise multiplication. It is well known that $c_0(\Lambda)^*\isom{\mathbf{Ban}_1}\ell_1(\Lambda)$ and  $\ell_p(\Lambda)^*\isom{\mathbf{Ban}_1}\ell_{p^*}(\Lambda)$ for $1\leq p<+\infty$. In fact these isomorphisms are isomorphisms of $\ell_\infty(\Lambda)$- and $c_0(\Lambda)$-modules. 

For a given $\lambda\in\Lambda$ we define $\mathbb{C}_\lambda$ as left or right $\ell_\infty(\Lambda)$- or $c_0(\Lambda)$-module $\mathbb{C}$ with module action defined by
$$
a\cdot_\lambda z=a(\lambda)z,\qquad z\cdot_\lambda a=a(\lambda) z.
$$
for $a\in \ell_\infty(\Lambda)$ and $z\in\mathbb{C}_s$. 

\begin{proposition}\label{OneDimlInftyc0ModMetTopProjIngFlat} Let $\Lambda$ be a set and $\lambda\in\Lambda$. Then $\mathbb{C}_\lambda$ is metrically and topologically projective injective and flat $\ell_\infty(\Lambda)$- or $c_0(\Lambda)$-module.
\end{proposition}
\begin{proof} Let $A$ be either $\ell_\infty(\Lambda)$ or $c_0(\Lambda)$. One can easily check that the maps $\pi:A_+\to\mathbb{C}_\lambda:a\oplus_1 z\mapsto a(\lambda)+z$ and $\sigma:\mathbb{C}_\lambda\to A_+:z\mapsto z\delta_\lambda\oplus_1 0$ are contractive $A$-morphisms of left $A$-modules. Since $\pi\sigma=1_{\mathbb{C}_\lambda}$, then $\mathbb{C}_\lambda$ is retract of $A_+$ in $A-\mathbf{mod}_1$. From proposition \ref{UnitalAlgIsMetTopProj} and \ref{RetrMetTopProjIsMetTopProj} it follows that $\mathbb{C}_\lambda$ is metrically and topologically projective left $A$-module and a fortiori metrically and topologically flat by proposition \ref{MetTopProjIsMetTopFlat}. By proposition \ref{DualMetTopProjIsMetrInj} we have that $\mathbb{C}_\lambda^*$ is metrically and topologically injective as $A$-module. Now metric and topological injectivity of $\mathbb{C}_\lambda$ follow from isomorphism $\mathbb{C}_\lambda\isom{\mathbf{mod}_1-A}\mathbb{C}_\lambda^*$.
\end{proof}

\begin{proposition}\label{c0AndlInftyModlIfty} Let $\Lambda$ be a set. Then

$i)$ $\ell_\infty(\Lambda)$ is metrically and topologically projective and flat as $\ell_\infty(\Lambda)$-module;

$ii)$ $\ell_\infty(\Lambda)$ is metrically or topologically projective or flat as $c_0(\Lambda)$-module iff $\Lambda$ is finite;

$iii)$ $\ell_\infty(\Lambda)$ is metrically and topologically injective as $\ell_\infty(\Lambda)$- and $c_0(\Lambda)$-module.
\end{proposition}
\begin{proof} $i)$ Since $\ell_\infty(\Lambda)$ is a unital algebra, then it is metrically and topologically projective as $\ell_\infty(\Lambda)$-module by proposition \ref{UnitalAlgIsMetTopProj}. Results on flatness follow from proposition \ref{MetTopProjIsMetTopFlat}.

$ii)$ For infinite $\Lambda$ the Banach space $\ell_\infty(\Lambda)/c_0(\Lambda)$ is of infinite dimension, so by proposition \ref{CStarAlgIsTopFlatOverItsIdeal} the module $\ell_\infty(\Lambda)$ neither topologically nor metrically flat as $c_0(\Lambda)$-module. Both claims regarding projectivity follow from proposition \ref{MetTopProjIsMetTopFlat}. If $\Lambda$ is finite, then $c_0(\Lambda)=\ell_\infty(\Lambda)$, so the result follows from paragraph $i)$.

$iii)$ Let $A$ be either $\ell_\infty(\Lambda)$ or $c_0(\Lambda)$. Note that $\ell_\infty(\Lambda)\isom{A-\mathbf{mod}_1}\bigoplus_\infty\{\mathbb{C}_\lambda:\lambda\in\Lambda\}$, then from propositions \ref{OneDimlInftyc0ModMetTopProjIngFlat} and \ref{MetTopInjModProd} it follows that $\ell_\infty(\Lambda)$ is metrically injective as $A$-module. Topological injectivity follows from proposition \ref{MetInjIsTopInjAndTopInjIsRelInj}.
\end{proof}

\begin{proposition}\label{c0AndlInftyModc0} Let $\Lambda$ be a set. Then 

$i)$ $c_0(\Lambda)$ is metrically and topologically flat as $\ell_\infty(\Lambda)$- or $c_0(\Lambda)$-module;

$ii)$ $c_0(\Lambda)$ is metrically or topologically projective as $\ell_\infty(\Lambda)$- or $c_0(\Lambda)$-module iff $\Lambda$ is finite;

$iii)$ $c_0(\Lambda)$ is metrically or topologically injective as $\ell_\infty(\Lambda)$- or $c_0(\Lambda)$-module iff $\Lambda$ is finite.
\end{proposition}
\begin{proof} Let $A$ be either $\ell_\infty(\Lambda)$ or $c_0(\Lambda)$. Note that $c_0(\Lambda)$ is a two-sided ideal of $A$. 

$i)$ Recall that $c_0(\Lambda)$ has a contractive approximate identity of the form $(\sum_{\lambda\in S}\delta_\lambda)_{S\in\mathcal{P}_0(\Lambda)}$. Since $c_0(\Lambda)$ is a two-sided ideal of $A$, then the result follows from proposition \ref{IdealofCstarAlgisMetTopFlat}.

$ii)$, $iii)$ If $\Lambda$ is infinite, then $c_0(\Lambda)$ is not unital as Banach algebra. From corollary \ref{BiIdealOfCStarAlgMetTopProjCharac} and proposition \ref{MetTopInjOfId} the $A$-module $c_0(\Lambda)$ is neither metrically nor topologically projective or injective. If $\Lambda$ is finite, then $c_0(\Lambda)=\ell_\infty(\Lambda)$, so both results follow from paragraphs $i)$ and $iii)$ of proposition \ref{c0AndlInftyModlIfty}.
\end{proof}

\begin{proposition}\label{c0AndlInftyModl1} Let $\Lambda$ be a set. Then

$i)$ $\ell_1(\Lambda)$ is metrically and topologically injective as $\ell_\infty(\Lambda)$- or $c_0(\Lambda)$-module;

$ii)$ $\ell_1(\Lambda)$ is metrically and topologically projective and flat as $\ell_\infty(\Lambda)$- or $c_0(\Lambda)$-module;
\end{proposition}
\begin{proof} Let $A$ be either $\ell_\infty(\Lambda)$ or $c_0(\Lambda)$.

$i)$ Note that $\ell_1(\Lambda)\isom{\mathbf{mod}_1-A}c_0(\Lambda)^*$, so the result follows from proposition \ref{MetTopFlatCharac} and paragraph $i)$ of proposition \ref{c0AndlInftyModc0}.

$ii)$ Note that $\ell_1(\Lambda)\isom{A-\mathbf{mod}_1}\bigoplus_1\{\mathbb{C}_\lambda:\lambda\in\Lambda\}$, then from propositions \ref{OneDimlInftyc0ModMetTopProjIngFlat} and \ref{MetTopProjModCoprod} it follows that $\ell_1(\Lambda)$ is metrically projective as $A$-module. Topological projectivity follows from proposition \ref{MetProjIsTopProjAndTopProjIsRelProj}. Metric and topological flatness follow from proposition \ref{MetTopProjIsMetTopFlat}.
\end{proof}

\begin{proposition}\label{c0AndlInftyModlp} Let $\Lambda$ be a set and $1<p<+\infty$. Then $\ell_p(\Lambda)$ is metrically or topologically projective, injective or flat as $\ell_\infty(\Lambda)$- or $c_0(\Lambda)$-module iff $\Lambda$ is finite.
\end{proposition}
\begin{proof} Let $A$ be either $\ell_\infty(\Lambda)$ or $c_0(\Lambda)$, then $A$ is an $\mathscr{L}_\infty^g$-space. Since $\ell_p(\Lambda)$ is reflexive for $1<p<+\infty$, then from corollary  \ref{NoInfDimRefMetTopProjInjFlatModOverMthscrL1OrLInfty} it follows that $\ell_p(\Lambda)$ is necessarily finite dimensional if it is metrically or topologically projective injective or flat. This is equivalent to $\Lambda$ being finite. If $\Lambda$ is finite then $\ell_p(\Lambda)\isom{A-\mathbf{mod}}\ell_1(\Lambda)$ and $\ell_p(\Lambda)\isom{\mathbf{mod}-A}\ell_1(\Lambda)$, so topological projectivity injectivity and flatness follow from proposition \ref{c0AndlInftyModl1}.
\end{proof}

\begin{proposition}\label{c0AndlInftyModsRelTh} Let $\Lambda$ be a set. Then

$i)$ as $c_0(\Lambda)$-modules $\ell_p(\Lambda)$ for $1\leq p<+\infty$ and $\mathbb{C}_\lambda$ for $\lambda\in\Lambda$ are relatively projective, injective and flat, $c_0(\Lambda)$ relatively projective and flat, but relatively injective only for finite $\Lambda$, $\ell_\infty(\Lambda)$ is relatively injective and flat, but relatively projective only for finite $\Lambda$;

$ii)$ as $\ell_\infty(\Lambda)$-modules $\ell_p(\Lambda)$ for $1\leq p\leq+\infty$ and $\mathbb{C}_\lambda$ for $\lambda\in\Lambda$ are relatively projective, injective and flat, $c_0(\Lambda)$ is relatively projective and flat.
\end{proposition}
\begin{proof} $i)$ The algebra $c_0(\Lambda)$ is relatively biprojective [\cite{HelHomolBanTopAlg}, theorem IV.5.26] and admits a contractive approximate identity, so by [\cite{HelBanLocConvAlg}, theorem 7.1.60] all essential $c_0(\Lambda)$-modules are projective. Thus $c_0(\Lambda)$ and $\ell_p(\Lambda)$ for $1\leq p<+\infty$ are relatively projective $c_0(\Lambda)$-modules. A fortiori they are relatively flat as $c_0(\Lambda)$-modules [\cite{HelBanLocConvAlg}, proposition 7.1.40]. By the same proposition $\ell_1(\Lambda)\isom{\mathbf{mod}_1-c_0(\Lambda)}c_0(\Lambda)^*$ and $\ell_{p^*}(\Lambda)\isom{\mathbf{mod}_1-c_0(\Lambda)}\ell_p(\Lambda)^*$ for $1\leq p<+\infty$ are relatively injective $c_0(\Lambda)$-modules. From [\cite{RamsHomPropSemgroupAlg}, proposition 2.2.8 (i)] we know that a Banach algebra relatively injective over itself as right module, necessarily has a left identity. Therefore $c_0(\Lambda)$ is not relatively injective $c_0(\Lambda)$-module for infinite $\Lambda$. If $\Lambda$ is finite, then $c_0(\Lambda)$-module $c_0(\Lambda)$ is relatively injective because $c_0(\Lambda)=\ell_\infty(\Lambda)$ and $\ell_\infty(\Lambda)$ is relatively injective $c_0(\Lambda)$-module as was shown above. From [\cite{HelHomolBanTopAlg}, corollary V.2.16(II)] we know that $\ell_\infty(\Lambda)$ is not relatively projective as $c_0(\Lambda)$-module provided $\Lambda$ is infinite. If $\Lambda$ is finite then $\ell_\infty(\Lambda)$ is relatively projective $c_0(\Lambda)$-module because $\ell_\infty(\Lambda)=c_0(\Lambda)$ and $c_0(\Lambda)$ is relatively projective $c_0(\Lambda)$-module as was shown above. Propositions \ref{OneDimlInftyc0ModMetTopProjIngFlat}, \ref{MetProjIsTopProjAndTopProjIsRelProj}, \ref{MetInjIsTopInjAndTopInjIsRelInj} and \ref{MetFlatIsTopFlatAndTopFlatIsRelFlat} give the result for modules $\mathbb{C}_\lambda$, where $\lambda\in\Lambda$.

$ii)$ From proposition \ref{c0AndlInftyModlIfty} paragraph $i)$ and proposition \ref{MetProjIsTopProjAndTopProjIsRelProj} it follows that $\ell_\infty(\Lambda)$ is relatively projective $\ell_\infty(\Lambda)$-module. From [\cite{RamsHomPropSemgroupAlg}, propositions 2.3.3, 2.3.4] we know that $\langle$~an essential relatively projective / a  faithful relatively injective~$\rangle$ module over ideal of Banach algebra is $\langle$~relatively projective / relatively injective~$\rangle$ over algebra itself. Since $c_0(\Lambda)$ and $\ell_p(\Lambda)$ for $1\leq p<+\infty$ are essential and faithful $c_0(\Lambda)$-modules then from results of previous paragraph $\ell_p(\Lambda)$ for $1\leq p<+\infty$ are relatively projective and injective as $\ell_\infty(\Lambda)$-modules. Also we get that $c_0(\Lambda)$ is relatively projective $\ell_\infty(\Lambda)$-module. Therefore all these $\ell_\infty(\Lambda)$-modules are relatively flat [\cite{HelBanLocConvAlg}, proposition 7.1.40]. As the consequence $\ell_\infty(\Lambda)\isom{\mathbf{mod}_1-\ell_\infty(\Lambda)}\ell_1(\Lambda)^*$ is relatively injective $\ell_\infty(\Lambda)$-module.  Propositions \ref{OneDimlInftyc0ModMetTopProjIngFlat}, \ref{MetProjIsTopProjAndTopProjIsRelProj}, \ref{MetInjIsTopInjAndTopInjIsRelInj} and \ref{MetFlatIsTopFlatAndTopFlatIsRelFlat} give the result for modules $\mathbb{C}_\lambda$, where $\lambda\in\Lambda$.
\end{proof}

Results of this section are summarized in the following three tables. Each cell contains a condition under which the respective module has the respective property and propositions where this is proved. We use ??? symbol to indicate open problems. For the case of $\ell_\infty(\Lambda)$- and $c_0(\Lambda)$-modules $\ell_p(\Lambda)$ for $1<p<+\infty$ we don't have a criterion of homological triviality in metric theory, just a necessary condition. We indicate this fact via symbol $\implies$. From these table one can easily see that for modules over commutative $C^*$-algebras, there is much more in common between relative, metric and topological  theory. For example $\ell_1(\Lambda)$ is projective injective and flat as $\ell_\infty(\Lambda)$- or $c_0(\Lambda)$-module in all three theories.


\begin{scriptsize}
\begin{longtable}{|c|c|c|c|c|c|c|} 
\multicolumn{7}{c}{\mbox{Homologically trivial $c_0(\Lambda)$- and $\ell_\infty(\Lambda)$-modules in metric theory}}                                                                                                                                                                                                                                                                                                                                                                                                                                                                                                                                                                                                                                     \\
				 
\hline                 & \multicolumn{3}{c|}{$c_0(\Lambda)$-modules}                                                                                                                                                                                                                                                                                                                     & \multicolumn{3}{c|}{$\ell_\infty(\Lambda)$-modules}                                                                                                                                                                                                                                                                                                        \\
\hline
                       & \mbox{Projectivity}                                                                                                 & \mbox{Injectivity}                                                                                                  & \mbox{Flatness}                                                                                                     & \mbox{Projectivity}                                                                                                 & \mbox{Injectivity}                                                                                                  & \mbox{Flatness}                                                                                                     \\ 
\hline
$\ell_1(\Lambda)$      & \begin{tabular}{@{}c@{}}$\Lambda$\mbox{ is any } \\ \ref{c0AndlInftyModl1}\end{tabular}                             & \begin{tabular}{@{}c@{}}$\Lambda$\mbox{ is any }  \\ \ref{c0AndlInftyModl1}\end{tabular}                            & \begin{tabular}{@{}c@{}}$\Lambda$\mbox{ is any } \\ \ref{c0AndlInftyModl1}\end{tabular}                             & \begin{tabular}{@{}c@{}}$\Lambda$\mbox{ is any } \\ \ref{c0AndlInftyModl1}\end{tabular}                             & \begin{tabular}{@{}c@{}}$\Lambda$\mbox{ is any }  \\ \ref{c0AndlInftyModl1}\end{tabular}                            & \begin{tabular}{@{}c@{}}$\Lambda$\mbox{ is any } \\ \ref{c0AndlInftyModl1}\end{tabular}                             \\
\hline
$\ell_p(\Lambda)$      & \begin{tabular}{@{}c@{}}$\operatorname{Card}(\Lambda)<\aleph_0$ \\ ($\implies$) \ref{c0AndlInftyModlp}\end{tabular} & \begin{tabular}{@{}c@{}}$\operatorname{Card}(\Lambda)<\aleph_0$ \\ ($\implies$) \ref{c0AndlInftyModlp}\end{tabular} & \begin{tabular}{@{}c@{}}$\operatorname{Card}(\Lambda)<\aleph_0$ \\ ($\implies$) \ref{c0AndlInftyModlp}\end{tabular} & \begin{tabular}{@{}c@{}}$\operatorname{Card}(\Lambda)<\aleph_0$ \\ ($\implies$) \ref{c0AndlInftyModlp}\end{tabular} & \begin{tabular}{@{}c@{}}$\operatorname{Card}(\Lambda)<\aleph_0$ \\ ($\implies$) \ref{c0AndlInftyModlp}\end{tabular} & \begin{tabular}{@{}c@{}}$\operatorname{Card}(\Lambda)<\aleph_0$ \\ ($\implies$) \ref{c0AndlInftyModlp}\end{tabular} \\
\hline
$\ell_\infty(\Lambda)$ & \begin{tabular}{@{}c@{}}$\operatorname{Card}(\Lambda)<\aleph_0$ \\ \ref{c0AndlInftyModlIfty}\end{tabular}           & \begin{tabular}{@{}c@{}}$\Lambda$\mbox{ is any } \\ \ref{c0AndlInftyModlIfty}\end{tabular}                          & \begin{tabular}{@{}c@{}}$\operatorname{Card}(\Lambda)<\aleph_0$ \\ \ref{c0AndlInftyModlIfty}\end{tabular}           & \begin{tabular}{@{}c@{}}$\Lambda$\mbox{ is any } \\ \ref{c0AndlInftyModlIfty}\end{tabular}                          & \begin{tabular}{@{}c@{}}$\Lambda$\mbox{ is any } \\ \ref{c0AndlInftyModlIfty}\end{tabular}                          & \begin{tabular}{@{}c@{}}$\Lambda$\mbox{ is any } \\ \ref{c0AndlInftyModlIfty}\end{tabular}                          \\ 
\hline
$c_0(\Lambda)$         & \begin{tabular}{@{}c@{}}$\operatorname{Card}(\Lambda)<\aleph_0$ \\ \ref{c0AndlInftyModc0}\end{tabular}              & \begin{tabular}{@{}c@{}}$\operatorname{Card}(\Lambda)< \aleph_0$ \\ \ref{c0AndlInftyModc0}\end{tabular}             & \begin{tabular}{@{}c@{}}$\Lambda$\mbox{ is any } \\ \ref{c0AndlInftyModc0}\end{tabular}                             & \begin{tabular}{@{}c@{}}$\operatorname{Card}(\Lambda)<\aleph_0$ \\ \ref{c0AndlInftyModc0}\end{tabular}              & \begin{tabular}{@{}c@{}}$\operatorname{Card}(\Lambda)< \aleph_0$ \\ \ref{c0AndlInftyModc0}\end{tabular}             & \begin{tabular}{@{}c@{}}$\Lambda$\mbox{ is any } \\ \ref{c0AndlInftyModc0}\end{tabular}                             \\ 
\hline
$\mathbb{C}_\lambda$   & \begin{tabular}{@{}c@{}}$\lambda$\mbox{ is any } \\ \ref{OneDimlInftyc0ModMetTopProjIngFlat}\end{tabular}           & \begin{tabular}{@{}c@{}}$\lambda$\mbox{ is any } \\ \ref{OneDimlInftyc0ModMetTopProjIngFlat}\end{tabular}           & \begin{tabular}{@{}c@{}}$\lambda$\mbox{ is any } \\ \ref{OneDimlInftyc0ModMetTopProjIngFlat}\end{tabular}           & \begin{tabular}{@{}c@{}}$\lambda$\mbox{ is any } \\ \ref{OneDimlInftyc0ModMetTopProjIngFlat}\end{tabular}           & \begin{tabular}{@{}c@{}}$\lambda$\mbox{ is any } \\ \ref{OneDimlInftyc0ModMetTopProjIngFlat}\end{tabular}           & \begin{tabular}{@{}c@{}}$\lambda$\mbox{ is any } \\ \ref{OneDimlInftyc0ModMetTopProjIngFlat}\end{tabular}           \\
\hline

\multicolumn{7}{c}{\mbox{Homologically trivial $c_0(\Lambda)$- and $\ell_\infty(\Lambda)$-modules in topological theory}}                                                                                                                                                                                                                                                                                                                                                                                                                                                                                                                                                                                                                                \\
					 
\hline                 & \multicolumn{3}{c|}{$c_0(\Lambda)$-modules}                                                                                                                                                                                                                                                                                                                     & \multicolumn{3}{c|}{$\ell_\infty(\Lambda)$-modules}                                                                                                                                                                                                                                                                                                        \\
\hline
                       & \mbox{Projectivity}                                                                                                 & \mbox{Injectivity}                                                                                                  & \mbox{Flatness}                                                                                                     & \mbox{Projectivity}                                                                                                 & \mbox{Injectivity}                                                                                                  & \mbox{Flatness}                                                                                                     \\ 
\hline
$\ell_1(\Lambda)$      & \begin{tabular}{@{}c@{}}$\Lambda$\mbox{ is any }  \\ \ref{c0AndlInftyModl1}\end{tabular}                            & \begin{tabular}{@{}c@{}}$\Lambda$\mbox{ is any } \\ \ref{c0AndlInftyModl1}\end{tabular}                             & \begin{tabular}{@{}c@{}}$\Lambda$\mbox{ is any }  \\ \ref{c0AndlInftyModl1}\end{tabular}                            & \begin{tabular}{@{}c@{}}$\Lambda$\mbox{ is any }  \\ \ref{c0AndlInftyModl1}\end{tabular}                            & \begin{tabular}{@{}c@{}}$\Lambda$\mbox{ is any } \\ \ref{c0AndlInftyModl1}\end{tabular}                             & \begin{tabular}{@{}c@{}}$\Lambda$\mbox{ is any }  \\ \ref{c0AndlInftyModl1}\end{tabular}                            \\
\hline
$\ell_p(\Lambda)$      & \begin{tabular}{@{}c@{}}$\operatorname{Card}(\Lambda)<\aleph_0$ \\ \ref{c0AndlInftyModlp}\end{tabular}              & \begin{tabular}{@{}c@{}}$\operatorname{Card}(\Lambda)<\aleph_0$ \\ \ref{c0AndlInftyModlp}\end{tabular}              & \begin{tabular}{@{}c@{}}$\operatorname{Card}(\Lambda)<\aleph_0$ \\ \ref{c0AndlInftyModlp}\end{tabular}              & \begin{tabular}{@{}c@{}}$\operatorname{Card}(\Lambda)<\aleph_0$ \\ \ref{c0AndlInftyModlp}\end{tabular}              & \begin{tabular}{@{}c@{}}$\operatorname{Card}(\Lambda)<\aleph_0$ \\ \ref{c0AndlInftyModlp}\end{tabular}              & \begin{tabular}{@{}c@{}}$\operatorname{Card}(\Lambda)<\aleph_0$ \\ \ref{c0AndlInftyModlp}\end{tabular}              \\
\hline
$\ell_\infty(\Lambda)$ & \begin{tabular}{@{}c@{}}$\operatorname{Card}(\Lambda)<\aleph_0$ \\ \ref{c0AndlInftyModlIfty}\end{tabular}           & \begin{tabular}{@{}c@{}}$\Lambda$\mbox{ is any } \\ \ref{c0AndlInftyModlIfty}\end{tabular}                          & \begin{tabular}{@{}c@{}}$\operatorname{Card}(\Lambda)<\aleph_0$ \\ \ref{c0AndlInftyModlIfty}\end{tabular}           & \begin{tabular}{@{}c@{}}$\Lambda$\mbox{ is any } \\ \ref{c0AndlInftyModlIfty}\end{tabular}                          & \begin{tabular}{@{}c@{}}$\Lambda$\mbox{ is any } \\ \ref{c0AndlInftyModlIfty}\end{tabular}                          & \begin{tabular}{@{}c@{}}$\Lambda$\mbox{ is any } \\ \ref{c0AndlInftyModlIfty}\end{tabular}                          \\ 
\hline
$c_0(\Lambda)$         & \begin{tabular}{@{}c@{}}$\operatorname{Card}(\Lambda)<\aleph_0$ \\ \ref{c0AndlInftyModc0}\end{tabular}              & \begin{tabular}{@{}c@{}}$\operatorname{Card}(\Lambda)<\aleph_0$ \\ \ref{c0AndlInftyModc0}\end{tabular}              & \begin{tabular}{@{}c@{}}$\Lambda$\mbox{ is any } \\ \ref{c0AndlInftyModc0}\end{tabular}                             & \begin{tabular}{@{}c@{}}$\operatorname{Card}(\Lambda)<\aleph_0$ \\ \ref{c0AndlInftyModc0}\end{tabular}              & \begin{tabular}{@{}c@{}}$\operatorname{Card}(\Lambda)<\aleph_0$ \\ \ref{c0AndlInftyModc0}\end{tabular}              & \begin{tabular}{@{}c@{}}$\Lambda$\mbox{ is any } \\ \ref{c0AndlInftyModc0}\end{tabular}                             \\ 
\hline
$\mathbb{C}_\lambda$   & \begin{tabular}{@{}c@{}}$\lambda$\mbox{ is any } \\ \ref{OneDimlInftyc0ModMetTopProjIngFlat}\end{tabular}           & \begin{tabular}{@{}c@{}}$\lambda$\mbox{ is any } \\ \ref{OneDimlInftyc0ModMetTopProjIngFlat}\end{tabular}           & \begin{tabular}{@{}c@{}}$\lambda$\mbox{ is any } \\ \ref{OneDimlInftyc0ModMetTopProjIngFlat}\end{tabular}           & \begin{tabular}{@{}c@{}}$\lambda$\mbox{ is any } \\ \ref{OneDimlInftyc0ModMetTopProjIngFlat}\end{tabular}           & \begin{tabular}{@{}c@{}}$\lambda$\mbox{ is any } \\ \ref{OneDimlInftyc0ModMetTopProjIngFlat}\end{tabular}           & \begin{tabular}{@{}c@{}}$\lambda$\mbox{ is any } \\ \ref{OneDimlInftyc0ModMetTopProjIngFlat}\end{tabular}           \\
\hline

\multicolumn{7}{c}{\mbox{Homologically trivial $c_0(\Lambda)$- and $\ell_\infty(\Lambda)$-modules in relative theory}}                                                                                                                                                                                                                                                                                                                                                                                                                                                                                                                                                                                                                                   \\

\hline                 & \multicolumn{3}{c|}{$c_0(\Lambda)$-modules}                                                                                                                                                                                                                                                                                                                     & \multicolumn{3}{c|}{$\ell_\infty(\Lambda)$-modules}                                                                                                                                                                                                                                                                                                        \\
\hline
                       & \mbox{Projectivity}                                                                                                 & \mbox{Injectivity}                                                                                                  & \mbox{Flatness}                                                                                                     & \mbox{Projectivity}                                                                                                 & \mbox{Injectivity}                                                                                                  & \mbox{Flatness}                                                                                                     \\ 
\hline
$\ell_1(\Lambda)$      & \begin{tabular}{@{}c@{}}$\Lambda$\mbox{ is any } \\ \ref{c0AndlInftyModsRelTh}, i)\end{tabular}                     & \begin{tabular}{@{}c@{}}$\Lambda$\mbox{ is any }  \\ \ref{c0AndlInftyModsRelTh}, i)\end{tabular}                    & \begin{tabular}{@{}c@{}}$\Lambda$\mbox{ is any } \\ \ref{c0AndlInftyModsRelTh}, i)\end{tabular}                     & \begin{tabular}{@{}c@{}}$\Lambda$\mbox{ is any }  \\ \ref{c0AndlInftyModsRelTh}, ii)\end{tabular}                   & \begin{tabular}{@{}c@{}}$\Lambda$\mbox{ is any } \\ \ref{c0AndlInftyModsRelTh}, ii)\end{tabular}                    & \begin{tabular}{@{}c@{}}$\Lambda$\mbox{ is any }  \\ \ref{c0AndlInftyModlIfty}, ii)\end{tabular}                    \\
\hline
$\ell_p(\Lambda)$      & \begin{tabular}{@{}c@{}}$\Lambda$\mbox{ is any } \\ \ref{c0AndlInftyModsRelTh}, i)\end{tabular}                     & \begin{tabular}{@{}c@{}}$\Lambda$\mbox{ is any }  \\ \ref{c0AndlInftyModsRelTh}, i)\end{tabular}                    & \begin{tabular}{@{}c@{}}$\Lambda$\mbox{ is any } \\ \ref{c0AndlInftyModsRelTh}, i)\end{tabular}                     & \begin{tabular}{@{}c@{}}$\Lambda$\mbox{ is any }  \\ \ref{c0AndlInftyModsRelTh}, ii)\end{tabular}                   & \begin{tabular}{@{}c@{}}$\Lambda$\mbox{ is any } \\ \ref{c0AndlInftyModsRelTh}, ii)\end{tabular}                    & \begin{tabular}{@{}c@{}}$\Lambda$\mbox{ is any }  \\ \ref{c0AndlInftyModlIfty}, ii)\end{tabular}                    \\
\hline
$\ell_\infty(\Lambda)$ & \begin{tabular}{@{}c@{}}$\operatorname{Card}(\Lambda)<\aleph_0$ \\ \ref{c0AndlInftyModsRelTh}, i)\end{tabular}      & \begin{tabular}{@{}c@{}}$\Lambda$\mbox{ is any }  \\ \ref{c0AndlInftyModsRelTh}, i)\end{tabular}                    & \begin{tabular}{@{}c@{}}$\Lambda$\mbox{ is any } \\ \ref{c0AndlInftyModsRelTh}, i)\end{tabular}                     & \begin{tabular}{@{}c@{}}$\Lambda$\mbox{ is any }  \\ \ref{c0AndlInftyModsRelTh}, ii)\end{tabular}                   & \begin{tabular}{@{}c@{}}$\Lambda$\mbox{ is any } \\ \ref{c0AndlInftyModsRelTh}, ii)\end{tabular}                    & \begin{tabular}{@{}c@{}}$\Lambda$\mbox{ is any }  \\ \ref{c0AndlInftyModlIfty}, ii)\end{tabular}                    \\
\hline
$c_0(\Lambda)$         & \begin{tabular}{@{}c@{}}$\Lambda$\mbox{ is any } \\ \ref{c0AndlInftyModsRelTh}, i)\end{tabular}                     & \begin{tabular}{@{}c@{}}$\operatorname{Card}(\Lambda)<\aleph_0$  \\ \ref{c0AndlInftyModsRelTh}, i) \end{tabular}    & \begin{tabular}{@{}c@{}}$\Lambda$\mbox{ is any } \\ \ref{c0AndlInftyModsRelTh}, i)\end{tabular}                     & \begin{tabular}{@{}c@{}}$\Lambda$\mbox{ is any }  \\ \ref{c0AndlInftyModsRelTh}, ii)\end{tabular}                   & \begin{tabular}{@{}c@{}}\mbox{ ??? } \end{tabular}                                                                  & \begin{tabular}{@{}c@{}}$\Lambda$\mbox{ is any }  \\ \ref{c0AndlInftyModlIfty}, ii)\end{tabular}                    \\
\hline
$\mathbb{C}_\lambda$   & \begin{tabular}{@{}c@{}}$\lambda$\mbox{ is any } \\ \ref{c0AndlInftyModsRelTh}, i)\end{tabular}                     & \begin{tabular}{@{}c@{}}$\lambda$\mbox{ is any }  \\ \ref{c0AndlInftyModsRelTh}, i)\end{tabular}                    & \begin{tabular}{@{}c@{}}$\lambda$\mbox{ is any } \\ \ref{c0AndlInftyModsRelTh}, i)\end{tabular}                     & \begin{tabular}{@{}c@{}}$\lambda$\mbox{ is any }  \\ \ref{c0AndlInftyModsRelTh}, ii)\end{tabular}                   & \begin{tabular}{@{}c@{}}$\lambda$\mbox{ is any } \\ \ref{c0AndlInftyModsRelTh}, ii)\end{tabular}                    & \begin{tabular}{@{}c@{}}$\lambda$\mbox{ is any }  \\ \ref{c0AndlInftyModlIfty}, ii)\end{tabular}                    \\
\hline
\end{longtable}
\end{scriptsize}

%----------------------------------------------------------------------------------------
%	C_0(S)-modules
%----------------------------------------------------------------------------------------

\subsection{\texorpdfstring{$C_0(S)$}{C0(S)}-modules}
\label{SubSectionC0SModules}

This section is devoted to study of homological triviality of classical modules over algebra $C_0(S)$, where $S$ is a locally compact Hausdorff space. By classical we mean modules $C_0(S)$, $M(S)$ and $L_p(S,\mu)$ for positive measure $\mu\in M(S)$. The pointwise multiplication plays the role of outer action for these modules.

Further we give short preliminaries on these modules. Recall that $C_0(S)^*\isom{\mathbf{Ban}_1}M(S)$ and $L_p(S,\mu)^*\isom{\mathbf{Ban}_1}L_{p^*}(S,\mu)$ for $1\leq p<+\infty$. In fact these identifications are isomorphisms of left and right $C_0(S)$-modules. For a given positive measure $\mu\in M(S)$ by $M_s(S,\mu)$ we shall denote the closed $C_0(S)$-submodule of $M(S)$ consisting of measures strictly singular with respect to $\mu$. Then the well known Lebesgue decomposition theorem can be stated as $M(S)\isom{\mathbf{Ban}_1}L_1(S,\mu)\bigoplus_1 M_s(S,\mu)$. Even more, this identification is an isomorphism of left and right $C_0(S)$-modules. 

We obliged to emphasize here that we consider only finite Borel regular positive measures. This shall simplify many considerations. For example, any atom of regular measure on a locally compact Hausdorff spaces is a point [\cite{BourbElemMathIntegLivVI}, chapter 5, \S 5, exercise 7]. Since we consider only finite measures, one can not say that this section simply generalizes results of the previous one. Strictly speaking these sections are different, though their methods have much in common.

For a fixed point $s\in S$ by $\mathbb{C}_s$ we denote left or right Banach $C_0(S)$-module $\mathbb{C}$ with outer action defined by
$$
a\cdot_s z=a(s)z,\qquad z\cdot_s a=a(s)z
$$
\begin{proposition}\label{OneDimC0SModMetTopRelProjIngFlat} Let $S$ be a locally compact Hausdorff space and let $s\in S$. Then 

$i)$ $\mathbb{C}_s$ is metrically, topologically or relatively projective as $C_0(S)$-module iff $s$ is an isolated point of $S$;

$ii)$ $\mathbb{C}_s$ is metrically, topologically and relatively flat as $C_0(S)$-module.

$iii)$ $\mathbb{C}_s$ is metrically, topologically and relatively injective as $C_0(S)$-module;

\end{proposition}
\begin{proof} $i)$ If $\mathbb{C}_s$ is metrically or topologically or relatively projective, then by proposition \ref{MetProjIsTopProjAndTopProjIsRelProj} it is at least relatively projective. Now from [\cite{HelBanLocConvAlg}, proposition 7.1.31] we know that the latter forces $s$ to be an isolated point of $S$. Conversely, assume $s$ is an isolated point of $S$. One can easily check that the maps $\pi:C_0(S)_+\to\mathbb{C}_s:a\oplus_1 z\mapsto a(s)+z$ and $\sigma:\mathbb{C}_s\to C_0(S)_+:z\mapsto z\delta_s\oplus_1 0$ are contractive $C_0(S)$-morphisms. Since $\pi\sigma=1_{\mathbb{C}_s}$, then $\mathbb{C}_s$ is a retract of $C_0(S)_+$ in $C_0(S)-\mathbf{mod}_1$. From propositions \ref{RetrMetTopProjIsMetTopProj} and \ref{UnitalAlgIsMetTopProj} it follows that $\mathbb{C}_s$ is metrically and topologically projective left $C_0(S)$-module. From \ref{MetProjIsTopProjAndTopProjIsRelProj} we conclude that $\mathbb{C}_s$ is also relatively projective $C_0(S)$-module.

$ii)$ By [\cite{HelBanLocConvAlg}, theorem 7.1.87] the algebra $C_0(S)$ is relatively amenable. Since this algebra is a $C^*$-algebra it is $1$-amenable [\cite{RundeAmenConstFour}, example 3]. Clearly, $\mathbb{C}_s$ is a $1$-dimensional $L_1$-space and an essential $C_0(S)$-module. Therefore, by proposition \ref{MetTopEssL1FlatModAoverAmenBanAlg} this module is metrically flat. Now the result follows from proposition \ref{MetFlatIsTopFlatAndTopFlatIsRelFlat}.

$iii)$ From paragraph $ii)$ and proposition \ref{MetTopFlatCharac} it follows that $\mathbb{C}_s^*$ is metrically injective. By proposition \ref{MetFlatIsTopFlatAndTopFlatIsRelFlat} it is topologically and relatively injective too. It remains to note that $\mathbb{C}_s\isom{\mathbf{mod}_1-A}\mathbb{C}_s^*$. 
\end{proof}

\begin{proposition}\label{C0SC0SModMetTopRelProjIngFlat} Let $S$ be a locally compact Hausdorff space and let $s\in S$. Then

$i)$ $C_0(S)$ is $\langle$~metrically / topologically / relatively~$\rangle$ projective as $C_0(S)$-module iff $S$ is $\langle$~compact / compact / paracompact~$\rangle$;

$ii)$ $C_0(S)$ is metrically injective as $C_0(S)$-module iff $S$ is a Stonean space;

$iii)$ $C_0(S)$ is metrically, topologically and relatively flat as $C_0(S)$-module.
\end{proposition}
\begin{proof} We regard $C_0(S)$ as a two-sided ideal of $C_0(S)$. Recall that $\operatorname{Spec}(C_0(S))$ is homeomorphic to $S$ [\cite{HelHomolBanTopAlg}, corollary 3.1.6].

$i)$ It is enough to note that by $\langle$~proposition \ref{IdealofCommCStarAlgMetTopProjCharac} / proposition \ref{IdealofCommCStarAlgMetTopProjCharac} / [\cite{HelHomolBanTopAlg}, chapter IV,\S\S 2-3]~$\rangle$ the spectrum of $C_0(S)$ is $\langle$~compact / compact / paracompact~$\rangle$. 

$ii)$ This result is a weakened version of proposition \ref{MetInjCStarAlgCharac}.

$iii)$ From proposition \ref{IdealofCstarAlgisMetTopFlat} it immediately follows that $C_0(S)$-module $C_0(S)$ is metrically and topologically flat. By proposition \ref{MetFlatIsTopFlatAndTopFlatIsRelFlat} it is also relatively flat.
\end{proof}

\begin{proposition}\label{AtomsOfRelProjLpMod} Let $S$ be a locally compact Hausdorff space, $\mu$ be a finite Borel regular positive measure on $S$. Assume $1\leq p\leq+\infty$ and $C_0(S)$-module $L_p(S,\mu)$ is relatively projective. Then any atom of $\mu$ is an isolated point in $S$.
\end{proposition} 
\begin{proof} Assume $\mu$ has at least one atom, otherwise there is nothing to prove. From [\cite{BourbElemMathIntegLivVI}, chapter 5, \S 5, exercise 7] we know that any atom of $\mu$ is a point. Call it $s$. Consider well defined linear maps $\pi:L_p(\Omega,\mu)\to\mathbb{C}_s:f\mapsto f(s)$ and $\sigma:\mathbb{C}_s\to L_p(\Omega,\mu):z\mapsto z\delta_s$. One can easily check that these maps are $C_0(S)$-morphisms and $\pi\sigma=1_{\mathbb{C}_s}$. Therefore, $\mathbb{C}_s$ is a retract of $L_p(S,\mu)$ in $C_0(S)-\mathbf{mod}$. By assumption, the latter module is relatively projective, so by [\cite{HelBanLocConvAlg}, proposition 7.1.6] the $C_0(S)$-module $\mathbb{C}_s$ is relatively projective. By paragraph $i)$ of proposition \ref{C0SC0SModMetTopRelProjIngFlat} we see that $s$ is an isolated point of $S$.
\end{proof}

\begin{proposition}\label{SuppsOfSomeFuncInC0SAndLp} Let $S$ be a locally compact Hausdorff space, $\mu$ be a finite Borel regular positive measure on $S$. Let $1\leq p<+\infty$ and $V$ be an open subset of $S$, then

$i)$ there exists a net $(e_\nu^V)_{\nu\in N}$ of continuous functions that pointwise converge to $\chi_V$ and $0\leq e_\nu^V\leq 1$, $e_\nu^V|_{S\setminus V}=0$ for all $\nu\in N$. Even more $(e_\nu^V)_{\nu\in N}$ converges to $\chi_V$ in $L_p(S,\mu)$; 

$ii)$ for any $C_0(S)$-morphism $\phi:L_p(S,\mu)\to C_0(S)$ holds $\phi(\chi_V)|_{S\setminus V}=0$.
\end{proposition}
\begin{proof} $i)$ Let $(K_\nu)_{\nu\in N}$ be the net of all compact subsets of $V$. By Urysohn's lemma for each $\nu\in N$ we can construct a continuous function $e_\nu^V:S\to\mathbb{C}$ such that $0\leq e_\nu^V\leq 1$, $e_\nu^V|_{K_\nu}=1$ and $e_\nu^V|_{S\setminus V}=0$. Since $V=\bigcup_{\nu\in N} K_\nu$, then $(e_\nu^V)_{\nu\in N}$ pointwise converges to $\chi_V$. Recall that $\mu$ is a regular Borel measure, so $\lim_\nu\mu(V\setminus K_\nu)=0$. As the consequence $\lim_\nu e_\nu^V=\chi_V$ in $L_p(S,\mu)$.

$ii)$ Consider the net of functions $(e_\nu^V)_{\nu\in N}$ constructed in the paragraph $i)$. Clearly $e_\nu^V\chi_V=e_\nu^V$ for all $\nu\in N$, so $\phi(\chi_V)=\phi(\lim_\nu e_\nu^V\chi_V)=\lim_\nu e_\nu^V\phi(\chi_V)$. Since $e_\nu^V$ is zero outside $V$, then so does $\lim_\nu e_\nu^V\phi(\chi_V)$. Therefore $\phi(\chi_V)|_{S\setminus V}=0$.
\end{proof}

For the next proposition we need a short reminder on approximation property. Recall that $C(K)$-spaces have the approximation property [\cite{DefFloTensNorOpId}, section 5.2(3)], and this property is inherited by complemented subspaces [\cite{DefFloTensNorOpId}, exercise 5.5]. Therefore any space of the form $C_0(S)$ for some locally compact Hausdorff space has the approximation property because it is complemented in $C(\alpha S)$.

\begin{proposition}\label{RelProjLpModImplsPureAtomMeas} Let $S$ be a locally compact Hausdorff space, $\mu$ be a finite Borel regular positive measure on $S$. Assume $1\leq p<+\infty$ and $C_0(S)$-module $L_p(S,\mu)$ is relatively projective. Then $\mu$ is purely atomic and all its atoms are isolated points.
\end{proposition}
\begin{proof} By proposition \ref{AtomsOfRelProjLpMod} the set $S_a^{\mu}$ of atoms of measure $\mu$ consist of isolated points. Therefore $S\setminus S_a^{\mu}$ is a locally compact space and $\mu|_{S\setminus S_a^{\mu}}$ is a finite non atomic Borel regular measure. One more fact worth noting:  since $S_a^{\mu}$ is discrete, then a set $V$ is open in $S\setminus S_a^{\mu}$ iff it is open in $S$. Now assume there is an open set $V\subset S\setminus S_a^{\mu}$ with $\mu(V)>0$. Then $\chi_V\neq 0$ in $L_p(S,\mu)$. Recall that $C_0(S)$-module $L_p(S,\mu)$ is essential and $C_0(S)$ has the approximation property. Bearing all this in mind we can apply lemma 1.4 from \cite{SelivBiproBanAlg}. It guarantees existence of $C_0(S)$-morphism $\phi:L_p(S,\mu)\to C_0(S)$ such that $\phi(\chi_V)\neq 0$. 

Denote $f=\phi(\chi_V)$. By paragraph $ii)$ of proposition \ref{SuppsOfSomeFuncInC0SAndLp} we have $f|_{S\setminus V}=0$. Since $f\neq 0$, then there necessarily exists an open set $U\subset V$ such that $|f|_U|> 0$. Consider net $(e_\nu^U)_{\nu\in N}$ constructed from set $U$ in paragraph $i)$ of proposition \ref{SuppsOfSomeFuncInC0SAndLp}. Note that $e_\nu^U\chi_V=e_\nu^U$, so for any $t\in U$ we have
$$
|\phi(\chi_U)(t)|
=|\phi(\lim_\nu e_\nu^U)(t)|
=\lim_\nu|\phi(e_\nu^U\chi_V)(t)|
=\lim_\nu |e_\nu^U(t)f(t)|=|f(t)|
>0.
$$ 
Thus $\phi(\chi_U)\neq 0$. Hence $\chi_U\neq 0$ in $L_p(S,\mu)$ which is equivalent to $\mu(U)>0$. The latter implies existence of some point $r\in U\cap\operatorname{supp}(\mu)$. For the future note that $|f(r)|>0$.

Since $r\in U\cap\operatorname{supp}(\mu)\subset S\setminus S_a^{\mu}$ and $\mu|_{S\setminus S_a^{\mu}}$ is non atomic, then for the net $(W_\nu)_{\nu\in M}$ of all open neighborhoods of $r$ we have $\lim_\nu\mu(W_\nu)=0$. By Urysohn's lemma for each $\nu\in M$ we have a continuous function $h_\nu:S\to\mathbb{C}$ such that $0\leq h_\nu\leq 1$, $h_\nu|_{S\setminus (W_\nu\cap V)}=0$ and $h_\nu(r)=1$. By construction $h_\nu\chi_V=h_\nu$ for all $\nu\in N$. Now we have a long chain of inequalities
$$
0<|f(r)|
=|(h_\nu f)(r)|
=|(h_\nu \phi(\chi_V))(r)|
=|\phi(h_\nu\chi_V)(r)|
=|\phi(h_\nu)(r)|
\leq\Vert\phi(h_\nu)\Vert
$$
$$
\leq\limsup_\nu\Vert\phi(h_\nu)\Vert
\leq\limsup_\nu\Vert\phi\Vert\Vert h_\nu\Vert
\leq\Vert\phi\Vert\limsup_\nu\mu(W_\nu\cap V)^{1/p}
$$
$$
\leq\Vert\phi\Vert\limsup_\nu\mu(W_\nu)^{1/p}
=\Vert\phi\Vert\lim_\nu\mu(W_\nu)^{1/p}
=0.
$$
Contradiction, hence $\mu(V)=0$ for all open subsets $V$ of $S\setminus S_a^{\mu}$. Therefore by [\cite{FremMeasTh}, proposition 414L] the measure $\mu|_{S\setminus S_a^{\mu}}$ is zero. In other words $\mu$ is purely atomic.
\end{proof}

\begin{proposition}\label{L1C0SModMetTopRelProjInjFlat}
Let $S$ be a locally compact Hausdorff space and $\mu$ be a finite Borel regular positive measure on $S$. Then 

$i)$ $L_1(S,\mu)$ is metrically or topologically or relatively projective as $C_0(S)$-module iff $\mu$ is purely atomic and all its atoms are isolated points in $S$;

$ii)$ $L_1(S,\mu)$ is metrically, topologically and relatively injective as $C_0(S)$-module;

$iii)$ $L_1(S,\mu)$ is metrically, topologically and relatively flat as $C_0(S)$-module.
\end{proposition}
\begin{proof} $i)$ If $L_1(S,\mu)$ is metrically or topologically or relatively projective, then by proposition \ref{MetProjIsTopProjAndTopProjIsRelProj} it is at least relatively projective. Now from proposition \ref{RelProjLpModImplsPureAtomMeas} the measure $\mu$ is purely atomic and all atoms are isolated  points. Conversely, assume that $\mu$ is purely atomic and all atoms are isolated points. By $S_a^{\mu}$ we denote the set of these atoms. Now one can easily show that the linear map $i:L_1(S,\mu)\to\bigoplus_1\{\mathbb{C}_s:s\in S_a^{\mu}\}:f\mapsto \bigoplus_1\{\mu(\{s\})f(s):s\in S_a^{\mu}\}$ is an isometric isomorphism of $C_0(S)$-modules. By paragraphs $i)$ of propositions \ref{MetTopProjModCoprod} and \ref{OneDimC0SModMetTopRelProjIngFlat} the $C_0(S)$-module $\bigoplus_1\{\mathbb{C}_s:s\in S_a^{\mu}\}$ is metrically projective. Therefore so does $L_1(S,\mu)$. By proposition \ref{MetProjIsTopProjAndTopProjIsRelProj} it is also topologically and relatively projective.

$ii)$ By paragraph $iii)$ of proposition \ref{C0SC0SModMetTopRelProjIngFlat} the $C_0(S)$-module $C_0(S)$ is metrically flat. From proposition \ref{DualMetTopProjIsMetrInj} we get that $M(S)\isom{\mathbf{mod}_1-C_0(S)}C_0(S)^*$ is metrically injective. Since $M(S)\isom{\mathbf{mod}_1-C_0(S)}L_1(S,\mu)\bigoplus_1 M_s(S,\mu)$, then $L_1(S,\mu)$ is a retract of $M(S)$ in $\mathbf{mod}_1-C_0(S)$. So by proposition \ref{RetrMetTopInjIsMetTopInj} the $C_0(S)$-module $L_1(S,\mu)$ is metrically injective. Relative and topological injectivity of $L_1(S,\mu)$ follows from proposition \ref{MetInjIsTopInjAndTopInjIsRelInj}.

$iii)$ By [\cite{HelBanLocConvAlg}, theorem 7.1.87] the algebra $C_0(S)$ is relatively amenable. Since this algebra is a $C^*$-algebra it is $1$-amenable [\cite{RundeAmenConstFour}, example 3]. Since $L_1(S,\mu)$ is an essential $C_0(S)$-module which tautologically an $L_1$-space, then by proposition \ref{MetTopEssL1FlatModAoverAmenBanAlg} this module is metrically flat. From proposition \ref{MetFlatIsTopFlatAndTopFlatIsRelFlat} the $C_0(S)$-module $L_1(S,\mu)$ is also topologically and relatively flat.
\end{proof}

\begin{proposition}\label{LpC0SModMetTopRelProjIngFlat}
Let $S$ be a locally compact Hausdorff space and $\mu$ be a finite Borel regular positive measure on $S$. Assume $1<p<+\infty$, then 

$i)$ $L_p(S,\mu)$ is relatively injective and flat, but relatively projective iff $\mu$ is purely atomic and all atoms are isolated points;

$ii)$ $L_p(S,\mu)$ is topologically projective or injective or flat iff $\mu$ is purely atomic with finitely many atoms;

$iii)$ if $L_p(S,\mu)$ is metrically projective or injective or flat, then $\mu$ is purely atomic with finitely many atoms.
\end{proposition}
\begin{proof} $i)$ By [\cite{HelBanLocConvAlg}, theorem 7.1.87] the algebra $C_0(S)$ is relatively amenable. Now from [\cite{HelBanLocConvAlg}, theorem 7.1.60] it follows that $L_p(S,\mu)$ is relatively flat for all $1<p<+\infty$. Note that $L_p(S,\mu)\isom{\mathbf{mod}_1-C_0(S)}L_{p^*}(S,\mu)^*$. Then from [\cite{HelBanLocConvAlg}, proposition 7.1.42] we get that $L_p(S,\mu)$ is relatively injective for any $1<p<+\infty$. Now assume that $L_p(S,\mu)$ is relatively projective, then by proposition \ref{RelProjLpModImplsPureAtomMeas} the measure $\mu$ is purely atomic and all its atoms are isolated points. Conversely, let $\mu$ be purely atomic with all atoms isolated. Denote by $S_a^{\mu}$ the set of these atoms. Since $S_a^{\mu}$ is discrete, then $C_0(S_a^{\mu})$ is relatively biprojective [\cite{HelHomolBanTopAlg}, theorem 4.5.26]. Then $L_p(S,\mu)$ is relatively projective $C_0(S_a^{\mu})$-module because it is essential module over relatively biprojective algebra with two-sided bounded approximate identity. Clearly $C_0(S_a^{\mu})$ is a two-sided ideal of $C_0(S)$, so from [\cite{RamsHomPropSemgroupAlg}, proposition 2.3.2(i)] we get that $L_p(S,\mu)$ is relatively projective as $C_0(S)$-module.

$ii), iii)$ Assume that $L_p(S,\mu)$ is metrically or topologically projective or injective or flat $C_0(S)$-module. Since $L_p(S,\mu)$ is reflexive and $C_0(S)$ is an $\mathscr{L}_\infty^g$-space, then $L_p(S,\mu)$ is finite dimensional by corollary \ref{NoInfDimRefMetTopProjInjFlatModOverMthscrL1OrLInfty}. The latter is equivalent to measure $\mu$ being purely atomic with finitely many atoms. On the other hand, if $\mu$ is purely atomic with finitely many atoms, then $L_p(S,\mu)$ is topologically isomorphic to $L_1(S,\mu)$ as left or right $C_0(S)$-module. The latter module is topologically projective, injective and flat for our measure by proposition \ref{L1C0SModMetTopRelProjInjFlat}. Hence so does $L_p(S,\mu)$.
\end{proof}

\begin{proposition}\label{LinftyC0SModMetTopRelProjIngFlat}
Let $S$ be a locally compact Hausdorff space and $\mu$ be a finite Borel regular positive measure on $S$. Then

$i)$ $L_\infty(S,\mu)$ is metrically, topologically and relatively injective as $C_0(S)$-module;

$ii)$ $L_\infty(S,\mu)$ is relatively flat $C_0(S)$-module.
\end{proposition}
\begin{proof} $i)$ Since $L_\infty(S,\mu)\isom{\mathbf{mod}_1-C_0(S)}L_1(S,\mu)^*$, then the result immediately follows from proposition \ref{DualMetTopProjIsMetrInj} and paragraph $iii)$ of proposition \ref{L1C0SModMetTopRelProjInjFlat}.

$ii)$ By [\cite{HelBanLocConvAlg}, theorem 7.1.87] the algebra $C_0(S)$ is relatively amenable. Any left Banach module over relatively amenable Banach algebra is relatively flat [\cite{HelBanLocConvAlg}, theorem 7.1.60]. In particular $L_\infty(S,\mu)$ is relatively flat $C_0(S)$-module.
\end{proof}

\begin{proposition}\label{MSC0SModMetTopRelProjIngFlat}
Let $S$ be a locally compact Hausdorff space and $\mu$ be a finite Borel regular positive measure on $S$. Then

$i)$ $M(S)$ is metrically or topologically or relatively projective as $C_0(S)$-module iff $S$ is discrete; 

$ii)$ $M(S)$ is metrically, topologically and relatively injective as $C_0(S)$-module; 

$iii)$ $M(S)$ is metrically, topologically and relatively flat as $C_0(S)$-module.
\end{proposition}
\begin{proof} $i)$ If $M(S)$ is metrically or topologically or relatively projective, then by proposition \ref{MetProjIsTopProjAndTopProjIsRelProj} it is at least relatively projective. For arbitrary $s\in S$ consider measure $\mu=\delta_s$ and recall the decomposition $M(S)\isom{C_0(S)-\mathbf{mod}_1}L_1(S,\mu)\bigoplus_1 M_s(S,\mu)$. Then $L_1(S,\mu)$ is a retract of $M(S)$ in $C_0(S)-\mathbf{mod}_1$. So from [\cite{HelBanLocConvAlg}, proposition 7.1.6] we get that $L_1(S,\mu)$ is relatively projective $C_0(S)$-module. Since $s$ is the only atom of $\mu$, then from proposition \ref{L1C0SModMetTopRelProjInjFlat} it follows that $s$ is an isolated point in $S$. Since $s\in S$ is arbitrary, then $S$ is discrete. Conversely, assume $S$ is discrete. Then $C_0(S)=c_0(S)$, and $M(S)\isom{C_0(S)-\mathbf{mod}_1}C_0(S)^*\isom{C_0(S)-\mathbf{mod}_1}\ell_1(S)\isom{C_0(S)-\mathbf{mod}_1}\bigoplus_1\{\mathbb{C}_s:s\in S\}$. The latter $C_0(S)$-module is metrically projective by paragraphs $i)$ of propositions \ref{MetTopProjModCoprod} and \ref{OneDimC0SModMetTopRelProjIngFlat}. Therefore $M(S)$ is metrically projective $C_0(S)$-module too. By proposition \ref{MetProjIsTopProjAndTopProjIsRelProj} it is also topologically and relatively projective.

$ii)$ Since $M(S)\isom{\mathbf{mod}_1-C_0(S)}C_0(S)^*$, then the result immediately follows from proposition \ref{DualMetTopProjIsMetrInj} and paragraph $iii)$ of proposition \ref{C0SC0SModMetTopRelProjIngFlat}.

$iii)$ By [\cite{HelBanLocConvAlg}, theorem 7.1.87] the algebra $C_0(S)$ is relatively amenable. Since this algebra is a $C^*$-algebra it is $1$-amenable [\cite{RundeAmenConstFour}, example 2]. Note that $M(S)$ is an essential $C_0(S)$-module which as Banach space is an $L_1$-space [\cite{DalLauSecondDualOfMeasAlg}, discussion after proposition 2.14]. Then by proposition \ref{MetTopEssL1FlatModAoverAmenBanAlg} this module is metrically flat. From proposition \ref{MetFlatIsTopFlatAndTopFlatIsRelFlat} it is also topologically and relatively flat.
\end{proof}

Results of this section are summarized in the following three tables. Each cell contains a condition under which the respective module has the respective property and propositions where it is proved. We use ??? symbol to indicate open problems. Open problems of this section are divided into three parts: injectivity of $C_0(S)$, projectivity of $L_\infty(S,\mu)$ and flatness of $L_\infty(S,\mu)$. Complete description of relatively and topologically injective $C_0(S)$-modules $C_0(S)$ seems quite a challenge for one simple reason --- still there is no standard category of functional analysis where even topologically injective objects were fully understood. The question of relative projectivity of $C_0(S)$-module $L_\infty(S,\mu)$ is rather old. It seems that even relative projectivity of $L_\infty(S,\mu)$ is a rare property. Our conjecture that $\mu$ must be purely atomic with finitely many atoms. Finally we presume that a necessary condition for metric and topological flatness of $C_0(S)$-module $L_\infty(S,\mu)$ is compactness of $S$.

As for partial results, we don't have a criterion of homological triviality of $C_0(S)$-modules $L_p(S,\mu)$ in metric theory for $1<p<+\infty$. We indicate this fact via symbol $\implies$. Using advanced Banach geometric techniques on factorization constants through finite dimensional Hilbert spaces one may show that atoms count for metrically projective modules $L_p(S,\mu)$ doesn't exceed some universal constant. It seems that $L_p(S,\mu)$ is homologically trivial $C_0(S)$-module in metric theory only for purely atomic measures with unique atom. 

\begin{scriptsize}
\begin{longtable}{|c|c|c|c|} 
\multicolumn{4}{c}{\mbox{Homologically trivial $C_0(S)$-modules in metric theory}}                                                                                                                                                                                                                                                                                                                                                                                                                               \\
				 
\hline
                       & \mbox{Projectivity}                                                                                                                                         & \mbox{Injectivity}                                                                                                                                          & \mbox{Flatness}                                                                                                                                             \\
\hline
$L_1(S,\mu)$           & \begin{tabular}{@{}c@{}}$\mu$\mbox{ is purely atomic, all } \\ \mbox{ atoms are isolated points } \\ \ref{L1C0SModMetTopRelProjInjFlat}\end{tabular}        & \begin{tabular}{@{}c@{}}$\mu$\mbox{ is any }  \\ \ref{L1C0SModMetTopRelProjInjFlat}\end{tabular}                                                            & \begin{tabular}{@{}c@{}}$\mu$\mbox{ is any }  \\ \ref{L1C0SModMetTopRelProjInjFlat}\end{tabular}                                                            \\
\hline
$L_p(S,\mu)$           & \begin{tabular}{@{}c@{}}$\mu$\mbox{ is purely atomic } \\ \mbox{ with finitely many atoms } \\ ($\implies$) \ref{LpC0SModMetTopRelProjIngFlat}\end{tabular} & \begin{tabular}{@{}c@{}}$\mu$\mbox{ is purely atomic } \\ \mbox{ with finitely many atoms } \\ ($\implies$) \ref{LpC0SModMetTopRelProjIngFlat}\end{tabular} & \begin{tabular}{@{}c@{}}$\mu$\mbox{ is purely atomic } \\ \mbox{ with finitely many atoms } \\ ($\implies$) \ref{LpC0SModMetTopRelProjIngFlat}\end{tabular} \\
\hline
$L_\infty(S,\mu)$      & \begin{tabular}{@{}c@{}} ??? \end{tabular}                                                                                                                  & \begin{tabular}{@{}c@{}}$\mu$\mbox{ is any } \\ \ref{LinftyC0SModMetTopRelProjIngFlat}\end{tabular}                                                         & \begin{tabular}{@{}c@{}} ??? \end{tabular}                                                                                                                  \\
\hline
$M(S)$                 & \begin{tabular}{@{}c@{}}$S$\mbox{ is discrete } \\ \ref{MSC0SModMetTopRelProjIngFlat}\end{tabular}                                                          & \begin{tabular}{@{}c@{}}$S$\mbox{ is any } \\ \ref{MSC0SModMetTopRelProjIngFlat}\end{tabular}                                                             & \begin{tabular}{@{}c@{}}$S$\mbox{ is any } \\ \ref{MSC0SModMetTopRelProjIngFlat}\end{tabular}                                                               \\
\hline
$C_0(S)$               & \begin{tabular}{@{}c@{}}$S$\mbox{ is compact } \\ \ref{C0SC0SModMetTopRelProjIngFlat}\end{tabular}                                                          & \begin{tabular}{@{}c@{}}$S$\mbox{ is Stonean } \\ \ref{C0SC0SModMetTopRelProjIngFlat} \end{tabular}                                                          & \begin{tabular}{@{}c@{}}$S$\mbox{ is any } \\ \ref{C0SC0SModMetTopRelProjIngFlat}\end{tabular}                                                              \\
\hline
$\mathbb{C}_s$         & \begin{tabular}{@{}c@{}}$s$\mbox{ is an isolated point } \\ \ref{OneDimC0SModMetTopRelProjIngFlat}\end{tabular}                                             & \begin{tabular}{@{}c@{}}$s$\mbox{ is any } \\ \ref{OneDimC0SModMetTopRelProjIngFlat}\end{tabular}                                                           & \begin{tabular}{@{}c@{}}$s$\mbox{ is any } \\ \ref{OneDimC0SModMetTopRelProjIngFlat}\end{tabular}                                                           \\
\hline

\multicolumn{4}{c}{\mbox{Homologically trivial $C_0(S)$-modules in topological theory}}                                                                                                                                                                                                                                                                                                                                                                                                                          \\
					 
\hline
                       & \mbox{Projectivity}                                                                                                                                         & \mbox{Injectivity}                                                                                                                                          & \mbox{Flatness}                                                                                                                                             \\
\hline
$L_1(S,\mu)$           & \begin{tabular}{@{}c@{}}$\mu$\mbox{ is purely atomic, all } \\ \mbox{ atoms are isolated points } \\ \ref{L1C0SModMetTopRelProjInjFlat}\end{tabular}        & \begin{tabular}{@{}c@{}}$\mu$\mbox{ is any }  \\ \ref{L1C0SModMetTopRelProjInjFlat}\end{tabular}                                                            & \begin{tabular}{@{}c@{}}$\mu$\mbox{ is any }  \\ \ref{L1C0SModMetTopRelProjInjFlat}\end{tabular}                                                            \\
\hline
$L_p(S,\mu)$           & \begin{tabular}{@{}c@{}}$\mu$\mbox{ is purely atomic } \\ \mbox{ with finitely many atoms } \\ \ref{LpC0SModMetTopRelProjIngFlat}\end{tabular}              & \begin{tabular}{@{}c@{}}$\mu$\mbox{ is purely atomic } \\ \mbox{ with finitely many atoms } \\ \ref{LpC0SModMetTopRelProjIngFlat}\end{tabular}              & \begin{tabular}{@{}c@{}}$\mu$\mbox{ is purely atomic } \\ \mbox{ with finitely many atoms } \\ \ref{LpC0SModMetTopRelProjIngFlat}\end{tabular}              \\
\hline
$L_\infty(S,\mu)$      & \begin{tabular}{@{}c@{}} ??? \end{tabular}                                                                                                                  & \begin{tabular}{@{}c@{}}$\mu$\mbox{ is any } \\ \ref{LinftyC0SModMetTopRelProjIngFlat}\end{tabular}                                                         & \begin{tabular}{@{}c@{}} ??? \end{tabular}                                                                                                                  \\
\hline
$M(S)$                 & \begin{tabular}{@{}c@{}}$S$\mbox{ is discrete } \\ \ref{MSC0SModMetTopRelProjIngFlat}\end{tabular}                                                          & \begin{tabular}{@{}c@{}}$S$\mbox{ is any } \\ \ref{MSC0SModMetTopRelProjIngFlat}\end{tabular}                                                             & \begin{tabular}{@{}c@{}}$S$\mbox{ is any } \\ \ref{MSC0SModMetTopRelProjIngFlat}\end{tabular}                                                               \\
\hline
$C_0(S)$               & \begin{tabular}{@{}c@{}}$S$\mbox{ is compact } \\ \ref{C0SC0SModMetTopRelProjIngFlat}\end{tabular}                                                          & \begin{tabular}{@{}c@{}} ??? \end{tabular}                                                                                                                  & \begin{tabular}{@{}c@{}}$S$\mbox{ is any } \\ \ref{C0SC0SModMetTopRelProjIngFlat}\end{tabular}                                                              \\
\hline
$\mathbb{C}_s$         & \begin{tabular}{@{}c@{}}$s$\mbox{ is an isolated point } \\ \ref{OneDimC0SModMetTopRelProjIngFlat}\end{tabular}                                             & \begin{tabular}{@{}c@{}}$s$\mbox{ is any } \\ \ref{OneDimC0SModMetTopRelProjIngFlat}\end{tabular}                                                           & \begin{tabular}{@{}c@{}}$s$\mbox{ is any } \\ \ref{OneDimC0SModMetTopRelProjIngFlat}\end{tabular}                                                           \\
\hline

\multicolumn{4}{c}{\mbox{Homologically trivial $C_0(S)$-modules in relative theory}}                                                                                                                                                                                                                                                                                                                                                                                                                             \\

\hline
                       & \mbox{Projectivity}                                                                                                                                         & \mbox{Injectivity}                                                                                                                                          & \mbox{Flatness}                                                                                                                                             \\
\hline
$L_1(S,\mu)$           & \begin{tabular}{@{}c@{}}$\mu$\mbox{ is purely atomic, all } \\ \mbox{ atoms are isolated points } \\ \ref{L1C0SModMetTopRelProjInjFlat}\end{tabular}         & \begin{tabular}{@{}c@{}}$\mu$\mbox{ is any }  \\ \ref{L1C0SModMetTopRelProjInjFlat}\end{tabular}                                                            & \begin{tabular}{@{}c@{}}$\mu$\mbox{ is any } \\ \ref{L1C0SModMetTopRelProjInjFlat}, i)\end{tabular}                                                         \\
\hline
$L_p(S,\mu)$           & \begin{tabular}{@{}c@{}}$\mu$\mbox{ is purely atomic, all } \\ \mbox{ atoms are isolated points } \\ \ref{LpC0SModMetTopRelProjIngFlat}\end{tabular}         & \begin{tabular}{@{}c@{}}$\mu$\mbox{ is any } \\ \ref{LpC0SModMetTopRelProjIngFlat}\end{tabular}                                                             & \begin{tabular}{@{}c@{}}$\mu$\mbox{ is any } \\ \ref{LpC0SModMetTopRelProjIngFlat}, i)\end{tabular}                                                         \\
\hline
$L_\infty(S,\mu)$      & \begin{tabular}{@{}c@{}} ??? \end{tabular}                                                                                                                  & \begin{tabular}{@{}c@{}}$\mu$\mbox{ is any } \\ \ref{LinftyC0SModMetTopRelProjIngFlat}\end{tabular}                                                         & \begin{tabular}{@{}c@{}}$\mu$\mbox{ is any } \\ \ref{LinftyC0SModMetTopRelProjIngFlat}, i)\end{tabular}                                                     \\
\hline
$M(S)$                 & \begin{tabular}{@{}c@{}}$S$\mbox{ is discrete } \\ \ref{MSC0SModMetTopRelProjIngFlat}\end{tabular}                                                          & \begin{tabular}{@{}c@{}}$S$\mbox{ is any } \\ \ref{MSC0SModMetTopRelProjIngFlat}\end{tabular}                                                             & \begin{tabular}{@{}c@{}}$S$\mbox{ is any } \\ \ref{MSC0SModMetTopRelProjIngFlat}\end{tabular}                                                               \\
\hline
$C_0(S)$               & \begin{tabular}{@{}c@{}}$S$\mbox{ is paracompact } \\ \ref{C0SC0SModMetTopRelProjIngFlat}\end{tabular}                                                      & \begin{tabular}{@{}c@{}} ???  \end{tabular}                                                                                                                 & \begin{tabular}{@{}c@{}}$S$\mbox{ is any } \\ \ref{C0SC0SModMetTopRelProjIngFlat}, i)\end{tabular}                                                          \\
\hline
$\mathbb{C}_s$         & \begin{tabular}{@{}c@{}}$s$\mbox{ is an isolated point } \\ \ref{OneDimC0SModMetTopRelProjIngFlat}\end{tabular}                                             & \begin{tabular}{@{}c@{}}$s$\mbox{ is any } \\ \ref{OneDimC0SModMetTopRelProjIngFlat}\end{tabular}                                                           & \begin{tabular}{@{}c@{}}$s$\mbox{ is any } \\ \ref{OneDimC0SModMetTopRelProjIngFlat}\end{tabular}                                                           \\
\hline
\end{longtable}
\end{scriptsize}


%----------------------------------------------------------------------------------------
%	Applications to harmonic analysis
%----------------------------------------------------------------------------------------

\section{Applications to modules of harmonic analysis}
\label{SectionApplicationsToModulesOfHarmonicAnalysis}

%----------------------------------------------------------------------------------------
%	Preliminaries on harmonic analysis
%----------------------------------------------------------------------------------------

\subsection{Preliminaries on harmonic analysis}
\label{SectionPreliminariesOnHarmonicAnalysis} 

Let $G$ be a locally compact group. Its identity we shall denote by $e_G$. By well known Haar's theorem [\cite{HewRossAbstrHarmAnalVol1},section 15.8] there exists a unique up to positive constant Borel regular measure $m_G$ which is finite on all compact sets, positive on all open sets and left translation invariant, that is $m_G(sE)=m_G(E)$ for all $s\in G$ and $E\in Bor(G)$. It is called the left Haar measure of group $G$. If $G$ is compact we assume $m_G(G)=1$. If $G$ is infinite and discrete we choose $m_G$ as counting measure. For each $s\in G$ the map $m:Bor(G)\to[0,+\infty]:E\mapsto m_G(Es)$ is also a left Haar measure, so from uniqueness we infer that $m(E)=\Delta_G(s)m_G(E)$ for some $\Delta_G(s)>0$. The function $\Delta_G:G\to(0,+\infty)$ is called the modular function of the group $G$. It is clear that $\Delta_G(st)=\Delta_G(s)\Delta_G(t)$ for all $s,t\in G$. Groups with modular function equal to one are called unimodular. Examples of groups with unimodular function include compact groups, commutative groups and discrete groups. In what follows we use the notation $L_p(G)$ instead of $L_p(G,m_G)$ for $1\leq p\leq+\infty$. For a fixed $s\in G$ we define the left shift operator $L_s:L_1(G)\to L_1(G):f\mapsto(t\mapsto f(s^{-1}t))$ and the right shift operator $R_s:L_1(G)\to L_1(G):f\mapsto (t\mapsto f(ts))$. 

Group structure of $G$ allows us to introduce the Banach algebra structure on $L_1(G)$. For a given $f,g\in L_1(G)$ we define their convolution as
$$
(f\convol g)(s)=\int_G f(t)g(t^{-1}s)dm_G(t)=\int_G f(st)g(t^{-1})dm_G(t)
$$
$$=\int_G f(st^{-1})g(t)\Delta_G(t^{-1})dm_G(t)
$$
for almost all $s\in G$. In this case $L_1(G)$ endowed with convolution product becomes a Banach algebra. The Banach algebra $L_1(G)$ has a contractive two-sided approximate identity consisting of positive compactly supported continuous functions. The algebra $L_1(G)$ is unital iff $G$ is discrete, and in this case $\delta_{e_G}$ is the identity of $L_1(G)$. The group structure of $G$ allows us to turn the Banach space of complex finite Borel regular measures $M(G)$ into the Banach algebra too. We define convolution of two measures $\mu,\nu\in M(G)$ as
$$
(\mu\convol \nu)(E)=\int_G\nu(s^{-1}E)d\mu(s)=\int_G\mu(Es^{-1})d\nu(s)
$$
for all $E\in Bor(G)$. The Banach space $M(G)$ along with this convolution is a unital Banach algebra. The role of identity is played by Dirac delta measure $\delta_{e_G}$ supported on $e_G$. In fact $M(G)$ is a coproduct in $L_1(G)-\mathbf{mod}_1$ (but not in $M(G)-\mathbf{mod}_1$) of two-sided ideal $M_a(G)$ of measures absolutely continuous with respect to $m_G$ and subalgebra $M_s(G)$ of measures singular with respect to $m_G$. Note that $M_a(G)\isom{M(G)-\mathbf{mod}_1}L_1(G)$ and $M_s(G)$ is an annihilator $L_1(G)$-module. Finally, $M(G)=M_a(G)$ iff $G$ is discrete. 

Now we proceed to the discussion of standard left and right modules over $L_1(G)$ and $M(G)$. Since $L_1(G)$ can be regarded as two-sided ideal of $M(G)$ because of isometric left and right $M(G)$-morphism $i:L_1(G)\to M(G):f\mapsto f m_G$ it is enough to define module structure over $M(G)$. For $1\leq p<+\infty$ and any $f\in L_p(G)$, $\mu\in M(G)$ we define
$$
(\mu\convol_p f)(s)=\int_G f(t^{-1}s)d\mu(t),
\qquad\qquad
(f\convol_p \mu)(s)=\int_G f(st^{-1})\Delta_G(t^{-1})^{1/p}d\mu(t)
$$
These module actions turn any Banach space $L_p(G)$ for $1\leq p<+\infty$ into the left and right $M(G)$-module. Note that for $p=1$ and $\mu\in M_a(G)$ we get the usual definition of convolution. For $1<p\leq +\infty$ and any $f\in L_p(G)$, $\mu\in M(G)$ we define module actions
$$
(\mu\cdot_p f)(s)=\int_G \Delta_G(t)^{1/p}f(st)d\mu(t),
\qquad\qquad
(f\cdot_p \mu)(s)=\int_G f(ts)d\mu(t)
$$
These module actions turn any Banach space $L_p(G)$ for $1<p\leq+\infty$ into the left and right $M(G)$-module too. This special choice of module structure nicely interacts with duality. Indeed we have and $(L_p(G),\convol_p)^*\isom{\mathbf{mod}_1-M(G)}(L_{p^*}(G),\cdot_{p^*})$ for all $1\leq p<+\infty$. Finally, the Banach space $C_0(G)$ also becomes left and right $M(G)$-module when endowed with $\cdot_\infty$ in the role of module action. Even more, $C_0(G)$ is a closed left and right $M(G)$-submodule of $L_\infty(G)$ and $(C_0(G),\cdot_\infty)^*\isom{M(G)-\mathbf{mod}_1}(M(G),\convol)$.

A character on a locally compact group $G$ is by definition a continuous homomorphism from $G$ to $\mathbb{T}$. The set of characters on $G$ forms a group denoted by $\widehat{G}$. It becomes a locally compact group when considered with compact open topology. Any character $\gamma\in\widehat{G}$ gives rise to the continuous character $\varkappa_\gamma^L:L_1(G)\to\mathbb{C}:f\mapsto \int_G f(s)\overline{\gamma(s)}dm_G(s)$ on $L_1(G)$. In fact all characters of $L_1(G)$ arise this way. This result is due to Gelfand [\cite{KaniBanAlg}, theorems 2.7.2, 2.7.5]. Similarly, for each $\gamma\in\widehat{G}$ we have a character on $M(G)$ defined by $\varkappa_\gamma^M:M(G)\to\mathbb{C}:\mu\mapsto\int_{G} \overline{\gamma(s)}d\mu(s)$. By $\mathbb{C}_\gamma$ we denote the respective augmentation left and right $L_1(G)$- or $M(G)$-module. Their module actions are defined by
$$
f\cdot_{\gamma}z=z\cdot_{\gamma}f=\varkappa_\gamma^L(f)z
\qquad\qquad
\mu\cdot_{\gamma}z=z\cdot_{\gamma}\mu=\varkappa_\gamma^M(\mu)z
$$
for all $f\in L_1(G)$, $\mu\in M(G)$ and $z\in\mathbb{C}$. 

One of the numerous definitions of amenable group says, that a locally compact group $G$ is amenable if there exists an $L_1(G)$-morphism of right modules $M:L_\infty(G)\to\mathbb{C}_{e_{\widehat{G}}}$ such that $M(\chi_G)=1$ [\cite{HelBanLocConvAlg}, section VII.2.5]. We can even assume that $M$ is contractive [\cite{HelBanLocConvAlg}, remark 7.1.54].

Most of results of this section that not supported with references are presented in a full detail in [\cite{DalBanAlgAutCont}, section 3.3].

%----------------------------------------------------------------------------------------
%	L_1(G)-modules
%----------------------------------------------------------------------------------------

\subsection{\texorpdfstring{$L_1(G)$}{L1(G)}-modules}
\label{SubSectionL1GModules}

Metric homological properties of most of the standard $L_1(G)$-modules of harmonic analysis are studied in \cite{GravInjProjBanMod}. We borrow these ideas to unify approaches to metrical and topological homological properties of modules over group algebras.

\begin{proposition}\label{LInfIsL1MetrInj} Let $G$ be a locally compact group. Then $L_1(G)$ is metrically and topologically flat $L_1(G)$-module, i.e. $L_1(G)$-module $L_\infty(G)$ is metrically and topologically injective.
\end{proposition} 
\begin{proof} Since $L_1(G)$ has contractive approximate identity, then $L_1(G)$ is metrically and topologically flat $L_1(G)$-module by proposition \ref{MetTopFlatIdealsInUnitalAlg}. Since $L_\infty(G)\isom{\mathbf{mod}_1-L_1(G)}L_1(G)^*$, then by proposition \ref{MetTopFlatCharac} it is metrically and topologically injective.
\end{proof}

\begin{proposition}\label{OneDimL1ModMetTopProjCharac} Let $G$ be a locally compact group, and $\gamma\in\widehat{G}$. Then the following are equivalent:

$i)$ $G$ is compact;

$ii)$ $\mathbb{C}_\gamma$ is metrically projective $L_1(G)$-module;

$iii)$ $\mathbb{C}_\gamma$ is topologically projective $L_1(G)$-module.
\end{proposition}
\begin{proof} $i)$$\implies$$ ii)$ Consider $L_1(G)$-morphisms $\sigma^+:\mathbb{C}_\gamma\to L_1(G)_+:z\mapsto z\gamma \oplus_1 0$ and $\pi^+:L_1(G)_+\to\mathbb{C}_\gamma: f\oplus_1 w\to f\cdot_{\gamma}1+w$. One can easily check that $\Vert\pi^+\Vert=\Vert\sigma^+\Vert=1$ and $\pi^+\sigma^+=1_{\mathbb{C}_\gamma}$. Therefore $\mathbb{C}_\gamma$ is a retract of $L_1(G)_+$ in $L_1(G)-\mathbf{mod}_1$. From propositions \ref{UnitalAlgIsMetTopProj} and \ref{RetrMetTopProjIsMetTopProj} it follows that $\mathbb{C}_\gamma$ is metrically projective.

$ii)$$\implies$$ iii)$ See proposition \ref{MetProjIsTopProjAndTopProjIsRelProj}.

$iii)$$\implies$$i)$ Consider $L_1(G)$-morphism $\pi:L_1(G)\to\mathbb{C}_\gamma:f\mapsto f\cdot_{\gamma} 1$. It is easy to see that $\pi$ is strictly coisometric. Since $\mathbb{C}_\gamma$ is topologically projective, then there exists an $L_1(G)$-morphism $\sigma:\mathbb{C}_\gamma\to L_1(G)$ such that $\pi\sigma=1_{\mathbb{C}_\gamma}$. Let $f=\sigma(1)\in L_1(G)$ and $(e_\nu)_{\nu\in N}$ be a standard approximate identity of $L_1(G)$. Since $\sigma$ is an $L_1(G)$-morphism, then for all $s,t\in G$ we have 
$$
f(s^{-1}t)
=L_s(f)(t)
=\lim_\nu L_s(e_\nu\convol \sigma(1))(t)
=\lim_\nu((\delta_s\convol e_\nu)\convol \sigma(1))(t)
=\lim_\nu\sigma((\delta_s\convol e_\nu)\cdot_{\gamma} 1)(t)
$$
$$
=\lim_\nu\sigma(\varkappa_\gamma^L(\delta_s\convol e_\nu))(t)
=\lim_\nu\varkappa_\gamma^L(\delta_s\convol e_\nu)\sigma(1)(t)
=\lim_\nu(e_\nu\convol\gamma)(s^{-1})f(t)
=\gamma(s^{-1})f(t).
$$
Therefore, for the function $g(t):=\gamma(t^{-1})f(t)$ in $L_1(G)$ we have $g(st)=g(t)$ for all $s,t\in G$. Thus $g$ is a constant function in $L_1(G)$, which is possible only for compact group $G$.
\end{proof}

\begin{proposition}\label{OneDimL1ModMetTopInjFlatCharac} Let $G$ be a locally compact group, and $\gamma\in\widehat{G}$. Then the following are equivalent:

$i)$ $G$ is amenable;

$ii)$ $\mathbb{C}_\gamma$ is metrically injective $L_1(G)$-module;

$iii)$ $\mathbb{C}_\gamma$ is topologically injective $L_1(G)$-module.

$iv)$ $\mathbb{C}_\gamma$ is metrically flat $L_1(G)$-module;

$v)$ $\mathbb{C}_\gamma$ is topologically flat $L_1(G)$-module.
\end{proposition}
\begin{proof} $i)$$\implies$$ ii)$ Since $G$ is amenable, then we have contractive $L_1(G)$-morphism $M:L_\infty(G)\to\mathbb{C}_{e_{\widehat{G}}}$ with $M(\chi_G)=1$. Consider linear operators $\rho:\mathbb{C}_\gamma\to L_\infty(G):z\mapsto z\overline{\gamma}$ and $\tau:L_\infty(G)\to\mathbb{C}_\gamma:f\mapsto M(f\gamma)$. These are $L_1(G)$-morphisms of right $L_1(G)$-modules. We shall check this for operator $\tau$: for all $f\in L_\infty(G)$ and $g\in L_1(G)$ we have
$$
\tau(f\cdot_\infty g)
=M((f\cdot_\infty g)\gamma)
=M(f\gamma\cdot_\infty g\overline{\gamma})
=M(f\gamma)\cdot_{e_{\widehat{G}}} g\overline{\gamma}
=M(f\gamma)\varkappa_\gamma^L(g)
=\tau(f)\cdot_{\gamma} g.
$$  
It is easy to check that $\rho$ and $\tau$ are contractive and $\tau\rho=1_{\mathbb{C}_\gamma}$. Therefore $\mathbb{C}_\gamma$ is a retract of $L_\infty(G)$ in $\mathbf{mod}_1-L_1(G)$. From propositions \ref{LInfIsL1MetrInj} and \ref{RetrMetTopInjIsMetTopInj} it follows that $\mathbb{C}_\gamma$ is metrically injective as $L_1(G)$-module.

$ii)$$\implies$$ iii)$ See proposition \ref{MetInjIsTopInjAndTopInjIsRelInj}.

$iii)$$ \implies$$ i)$ Since $\rho$ is an isometric $L_1(G)$-morphism of right $L_1(G)$-modules and $\mathbb{C}_\gamma$ is topologically injective as $L_1(G)$-module, then $\rho$ is a coretraction in $\mathbf{mod}-L_1(G)$. Denote its left inverse morphism by $\pi$, then $\pi(\overline{\gamma})=\pi(\rho(1))=1$. Consider bounded linear functional $M:L_\infty(G)\to\mathbb{C}_\gamma:f\mapsto \pi(f\overline{\gamma})$. For all $f\in L_\infty(G)$ and $g\in L_1(G)$ we have
$$
M(f\cdot_\infty g)
=\pi((f\cdot_\infty g)\overline{\gamma})
=\pi(f\overline{\gamma}\cdot_\infty g\gamma)
=\pi(f\overline{\gamma})\cdot_{\gamma} g\gamma
=M(f)\varkappa_\gamma^L(g\gamma)
=M(f)\cdot_{e_{\widehat{G}}}g.
$$
Therefore $M$ is an $L_1(G)$-morphism, but we also have $M(\chi_G)=\pi(\overline{\gamma})=1$. Therefore $G$ is amenable.

$ii)$ $\Longleftrightarrow$ $iv)$, $iii)$ $\Longleftrightarrow$ $v)$ Note that $\mathbb{C}_\gamma^*\isom{\mathbf{mod}_1-L_1(G)}\mathbb{C}_\gamma$, so all equivalences  follow from three previous paragraphs and proposition \ref{MetTopFlatCharac}.
\end{proof}

In the next proposition we shall study specific ideals of Banach algebra $L_1(G)$. They are of the form $L_1(G)\convol\mu$ for some idempotent measure $\mu$. In fact, this class of ideals in case of commutative compact groups $G$ coincides with those left ideals of $L_1(G)$ that admit a right bounded approximate identity.

\begin{proposition}\label{CommIdealByIdemMeasL1MetTopProjCharac} Let $G$ be a locally compact group and  $\mu\in M(G)$ be an idempotent measure, that is $\mu\convol\mu=\mu$. If the left ideal $I=L_1(G)\convol\mu$ of Banach algebra $L_1(G)$ is topologically projective $L_1(G)$-module, then $\mu=p m_G$, for some $p\in I$.
\end{proposition}
\begin{proof} Let $\phi:I\to L_1(G)$ be arbitrary morphism of left $L_1(G)$-modules. Consider $L_1(G)$-morphism $\phi':L_1(G)\to L_1(G):x\mapsto\phi(x\convol\mu)$. By Wendel's theorem [\cite{WendLeftCentrzrs}, theorem 1], there exists a measure $\nu\in M(G)$ such that $\phi'(x)=x\convol\nu$ for all $x\in L_1(G)$. In particular, $\phi(x)=\phi(x\convol\mu)=\phi'(x)=x\convol\nu$ for all $x\in I$. It is now clear that $\psi:I\to I:x\mapsto\nu\convol x$ is a morphism of right $I$-modules satisfying $\phi(x)y=x\psi(y)$ for all $x,y\in I$. By paragraph $ii)$ of lemma \ref{GoodIdealMetTopProjIsUnital} the ideal $I$ has a right identity, say $e\in I$. Then $x\convol\mu=x\convol\mu\convol e$ for all $x\in L_1(G)$. Two measures are equal if their convolutions with all functions of $L_1(G)$ coincide [\cite{DalBanAlgAutCont}, corollary 3.3.24], so $\mu=\mu\convol e m_G$. Since $e\in I\subset L_1(G)$, then $\mu=\mu\convol e m_G\in M_a(G)$. Set $p=\mu\convol e\in I$, then $\mu=p m_G$.
\end{proof}

We conjecture that the left ideal $L_1(G)\convol \mu$ for idempotent measure $\mu$ is metrically projective $L_1(G)$-module iff $\mu=p m_G$ where $p\in I$ and $\Vert p\Vert=1$.

\begin{theorem}\label{L1ModL1MetTopProjCharac} Let $G$ be a locally compact group. Then the following are equivalent:

$i)$ $G$ is discrete;

$ii)$ $L_1(G)$ is metrically projective $L_1(G)$-module;

$iii)$ $L_1(G)$ is topologically projective $L_1(G)$-module.
\end{theorem}
\begin{proof} $i)$$\implies$$ ii)$ If $G$ is discrete, then $L_1(G)$ is unital with unit of norm $1$. By  proposition \ref{UnIdeallIsMetTopProj} we see that $L_1(G)$ is metrically projective as $L_1(G)$-module.

$ii)$$\implies$$ iii)$ See proposition \ref{MetProjIsTopProjAndTopProjIsRelProj}.

$iii)$$ \implies$$ i)$ Clearly, $\delta_{e_G}$ is an idempotent measure. Since $L_1(G)=L_1(G)\convol \delta_{e_G}$ is topologically projective, then by proposition \ref{CommIdealByIdemMeasL1MetTopProjCharac} we have $\delta_{e_G}=f m_G$ for some $f\in L_1(G)$. This is possible only if $G$ is discrete.
\end{proof}

Note that $L_1(G)$-module $L_1(G)$ is relatively projective for any locally compact group $G$ [\cite{HelBanLocConvAlg}, exercise 7.1.17].

\begin{proposition}\label{L1MetTopProjAndMetrFlatOfMeasAlg} Let $G$ be a locally compact group. Then the following are equivalent:

$i)$ $G$ is discrete;

$ii)$ $M(G)$ is metrically projective $L_1(G)$-module;

$iii)$ $M(G)$ is topologically projective $L_1(G)$-module;

$iv)$ $M(G)$ is metrically flat $L_1(G)$-module.
\end{proposition}
\begin{proof} 
$i)$$\implies$$ ii)$ We have $M(G)\isom{L_1(G)-\mathbf{mod}_1}L_1(G)$ for discrete $G$, so the result follows from theorem \ref{L1ModL1MetTopProjCharac}. 

$ii)$$\implies$$ iii)$ See proposition \ref{MetProjIsTopProjAndTopProjIsRelProj}.

$ii)$$\implies$$ iv)$ See proposition \ref{MetTopProjIsMetTopFlat}.

$iii)$$\implies$$ i)$ Note that $M(G)\isom{L_1(G)-\mathbf{mod}_1} L_1(G)\bigoplus_1 M_s(G)$, so $M_s(G)$ is topologically projective by proposition \ref{MetTopProjModCoprod}. Note that $M_s(G)$ is an annihilator $L_1(G)$-module, then by proposition \ref{MetTopProjOfAnnihModCharac} the algebra $L_1(G)$ has a right identity. Recall that $L_1(G)$ also has a two-sided bounded approximate identity, so $L_1(G)$ is unital. The last is equivalent to $G$ being discrete.

$iv)$$\implies$$ i)$ Note that $M(G)\isom{L_1(G)-\mathbf{mod}_1} L_1(G)\bigoplus_1 M_s(G)$, so $M_s(G)$ is metrically flat by proposition \ref{MetTopFlatModCoProd}. Note that $M_s(G)$ is an annihilator $L_1(G)$-module, then by proposition \ref{MetTopFlatAnnihModCharac} it is equal to zero. The last is equivalent to $G$ being discrete.
\end{proof}

\begin{proposition}\label{MeasAlgIsL1TopFlat} Let $G$ be a locally compact group. Then $M(G)$ is topologically flat $L_1(G)$-module.
\end{proposition}
\begin{proof} Since $M(G)$ is an $L_1$-space it is a fortiori an $\mathscr{L}_1^g$-space. Since $M_s(G)$ is complemented in $M(G)$, then $M_s(G)$ is an $\mathscr{L}_1^g$-space too [\cite{DefFloTensNorOpId}, corollary 23.2.1(2)]. Since $M_s(G)$ is an annihilator $L_1(G)$-module, then from proposition \ref{MetTopFlatAnnihModCharac} we have that $M_s(G)$ is topologically flat $L_1(G)$-module. The $L_1(G)$-module $L_1(G)$ is also topologically flat by proposition \ref{LInfIsL1MetrInj} . Since $M(G)\isom{L_1(G)-\mathbf{mod}_1}L_1(G)\bigoplus_1 M_s(G)$, then $M(G)$ is topologically flat $L_1(G)$-module by proposition \ref{MetTopFlatModCoProd}.
\end{proof}

%----------------------------------------------------------------------------------------
%	M(G)-modules
%----------------------------------------------------------------------------------------

\subsection{\texorpdfstring{$M(G)$}{M(G)}-modules}
\label{SubSectionMGModules}

We turn to the study of standard $M(G)$-modules of harmonic analysis. As we shall see most of results can be derived from results on $L_1(G)$-modules.

\begin{proposition}\label{MGMetTopProjInjFlatRedToL1} Let $G$ be a locally compact group, and $X$ be $\langle$~essential / faithful / essential~$\rangle$ $L_1(G)$-module. Then

$i)$ $X$ is metrically $\langle$~projective / injective / flat~$\rangle$ $M(G)$-module iff it is metrically $\langle$~projective / injective / flat~$\rangle$ $L_1(G)$-module;

$ii)$ $X$ is topologically $\langle$~projective / injective / flat~$\rangle$ $M(G)$-module iff it is topologically $\langle$~projective / injective / flat~$\rangle$ $L_1(G)$-module.
\end{proposition}
\begin{proof} Recall that $L_1(G)\isom{L_1(G)-\mathbf{mod}_1}M_a(G)$ is a two-sided complemented in $\mathbf{Ban}_1$ ideal of $M(G)$. Now $i)$ and $ii)$ follow from proposition $\langle$~\ref{MetTopProjUnderChangeOfAlg} / \ref{MetTopInjUnderChangeOfAlg}  / \ref{MetTopFlatUnderChangeOfAlg}~$\rangle$.
\end{proof} 

It is worth to mention here that the $L_1(G)$-modules $C_0(G)$, $L_p(G)$ for $1\leq p<\infty$ and $\mathbb{C}_\gamma$ for $\gamma\in\widehat{G}$ are essential and $L_1(G)$-modules $C_0(G)$, $M(G)$, $L_p(G)$ for $1\leq p\leq \infty$ and $\mathbb{C}_\gamma$ for $\gamma\in\widehat{G}$ are faithful. 

\begin{proposition}\label{MGModMGMetTopProjFlatCharac} Let $G$ be a locally compact group. Then $M(G)$ is metrically and topologically projective $M(G)$-module. As the consequence it is metrically and topologically flat $M(G)$-module.
\end{proposition} 
\begin{proof} Since $M(G)$ is a unital algebra, the metric and topological projectivity of $M(G)$ follow from proposition \ref{UnitalAlgIsMetTopProj}. It remains to apply proposition \ref{MetTopProjIsMetTopFlat}.
\end{proof}

%----------------------------------------------------------------------------------------
%	Banach geometric restrictions
%----------------------------------------------------------------------------------------

\subsection{Banach geometric restriction}
\label{SubSectionBanachGeometricRestriction}

In this section we shall show that many modules of harmonic analysis are fail to be metrically or topologically projective, injective or flat for purely Banach geometric reasons. 

\begin{proposition}\label{StdModAreNotRetrOfL1LInf} Let $G$ be an infinite locally compact group. Then

$i)$ $L_1(G)$, $C_0(G)$, $M(G)$, $L_\infty(G)^*$ are not topologically injective Banach spaces;

$ii)$ $C_0(G)$, $L_\infty(G)$ are not complemented in any $L_1$-space.
\end{proposition}
\begin{proof}
Since $G$ is infinite all modules in question are infinite dimensional.

$i)$ If an infinite dimensional Banach space is topologically injective, then it contains a copy of $\ell_\infty(\mathbb{N})$ [\cite{RosOnRelDisjFamOfMeas}, corollary 1.1.4], and consequently a copy of $c_0(\mathbb{N})$. The Banach space $L_1(G)$ is weakly sequentially complete [\cite{WojBanSpForAnalysts}, corollary III.C.14], so by corollary 5.2.11 in \cite{KalAlbTopicsBanSpTh} it can't contain a copy of $c_0(\mathbb{N})$. Therefore, $L_1(G)$ is not topologically injective Banach space.  If $M(G)$ is topologically injective Banach space, then so does its complemented subspace $M_a(G)\isom{\mathbf{Ban}_1}L_1(G)$. By previous argument this is impossible. So $M(G)$ is not topologically injective as Banach space. By corollary 3 of \cite{LauMingComplSubspInLInfOfG} the space $C_0(G)$ is not complemented in $L_\infty(G)$. Then $C_0(G)$ can't be topologically injective either. The Banach space $L_1(G)$ is complemented in $L_\infty(G)^*\isom{\mathbf{Ban}_1}L_1(G)^{**}$ [\cite{DefFloTensNorOpId}, proposition  B10]. Therefore if $L_\infty(G)^*$ is topologically injective as Banach space, then so does its retract $L_1(G)$. By previous arguments this is impossible, so $L_\infty(G)^*$ is not topologically injective Banach space.

$ii)$ If $C_0(G)$ is a retract of $L_1$-space, then $M(G)\isom{\mathbf{Ban}_1}C_0(G)^*$ is a retract of $L_\infty$-space, so it must be a topologically injective Banach space. This contradicts paragraph $i)$, so $C_0(G)$ is not a retract of $L_1$-space. Note that $\ell_\infty(\mathbb{N})$ embeds in $L_\infty(G)$, then so does $c_0(\mathbb{N})$. So if $L_\infty(G)$ is a retract of $L_1$-space, then there would exist an $L_1$-space containing a copy of $c_0(\mathbb{N})$. This is impossible as already showed in paragraph $i)$.
\end{proof}

From now on by $A$ we denote either $L_1(G)$ or $M(G)$. Recall that $L_1(G)$ and $M(G)$ are both $L_1$-spaces.

\begin{proposition}\label{StdModAreNotL1MGMetTopProjInjFlat} Let $G$ be an infinite locally compact group. Then

$i)$ $C_0(G)$, $L_\infty(G)$ are neither topologically nor metrically projective $A$-modules;

$ii)$ $L_1(G)$, $C_0(G)$, $M(G)$, $L_\infty(G)^*$ are neither topologically nor metrically injective $A$-modules;

$iii)$ $L_\infty(G)$, $C_0(G)$ are neither topologically nor metrically flat $A$-modules.
\end{proposition}

$iv)$ $L_p(G)$ for $1<p<\infty$ are neither topologically nor metrically projective, injective or flat $A$-flat.

\begin{proof} $i)$ The result follows from propositions \ref{TopProjInjFlatModOverL1Charac} paragraph $i)$ and  \ref{StdModAreNotRetrOfL1LInf} paragraph $ii)$.

$ii)$ The result follows from propositions \ref{TopProjInjFlatModOverL1Charac} paragraph $ii)$ and \ref{StdModAreNotRetrOfL1LInf}.

$iii)$ Note that $C_0(G)^*\isom{\mathbf{mod}_1-A}M(G)$. Now the result follows from paragraph $i)$ and proposition \ref{MetTopFlatCharac}.

$iv)$ Since $L_p(G)$ is reflexive for $1<p<\infty$ the result follows from \ref{NoInfDimRefMetTopProjInjFlatModOverMthscrL1OrLInfty}.
\end{proof}

It remains to consider metric and topological homological properties of $A$-modules when $G$ is finite.

\begin{proposition}\label{LpFinGrL1MGMetrInjProjCharac} Let $G$ be a non trivial finite group and $1\leq p\leq \infty$. Then the $A$-module $L_p(G)$ is metrically $\langle$~projective / injective~$\rangle$ iff $\langle$~$p=1$ / $p=\infty$~$\rangle$
\end{proposition}
\begin{proof} 
Assume $L_p(G)$ is metrically $\langle$~projective / injective~$\rangle$ as $A$-module. Since $L_p(G)$ is finite dimensional, then by paragraphs $i)$ and $ii)$ of proposition \ref{TopProjInjFlatModOverL1Charac} we have identifications $\langle$~$L_p(G)\isom{\mathbf{Ban}_1}\ell_1(\mathbb{N}_n)$ / $L_p(G)\isom{\mathbf{Ban}_1}C(\mathbb{N}_n)\isom{\mathbf{Ban}_1}\ell_\infty(\mathbb{N}_n)$ ~$\rangle$, where $n=\operatorname{Card}(G)>1$. Now we use the result of theorem 1 from \cite{LyubIsomEmdbFinDimLp} for Banach spaces over field $\mathbb{C}$: if for $2\leq m\leq k$ and $1\leq r,s\leq \infty$, there exists an isometric embedding from $\ell_r(\mathbb{N}_m)$ into $\ell_s(\mathbb{N}_k)$, then either $r=2$, $s\in 2\mathbb{N}$ or $r=s$. Therefore $\langle$~$p=1$ / $p=\infty$~$\rangle$. The converse easily follows from $\langle$~theorem \ref{L1ModL1MetTopProjCharac} / proposition \ref{LInfIsL1MetrInj}~$\rangle$
\end{proof}

\begin{proposition}\label{StdModFinGrL1MGMetrInjProjFlatCharac} Let $G$ be a finite group. Then

$i)$ $C_0(G)$, $L_\infty(G)$ are metrically injective $A$-modules;

$ii)$ $C_0(G)$, $L_p(G)$ for $1<p\leq\infty$ are metrically projective $A$-modules iff $G$ is trivial;

$iii)$ $M(G)$, $L_p(G)$ for $1\leq p<\infty$ are metrically injective  $A$-modules iff $G$ is trivial;

$iv)$ $C_0(G)$, $L_p(G)$ for $1<p\leq\infty$ are metrically flat $A$-modules iff $G$ is trivial.
\end{proposition}
\begin{proof}
$i)$ Since $G$ is finite then $C_0(G)=L_\infty(G)$. The result follows from proposition \ref{LInfIsL1MetrInj}.

$ii)$ If $G$ is trivial, that is $G=\{e_G\}$, then $L_p(G)=C_0(G)=L_1(G)$ and the result follows from paragraph $i)$. If $G$ is non trivial, then we recall that $C_0(G)=L_\infty(G)$ and use proposition \ref{LpFinGrL1MGMetrInjProjCharac}.

$iii)$ If $G=\{e_G\}$, then $M(G)=L_p(G)=L_\infty(G)$ and the result follows from paragraph $i)$. If $G$ is non trivial, then we note that $M(G)=L_1(G)$ and use proposition \ref{LpFinGrL1MGMetrInjProjCharac}.

$iv)$ From paragraph $iii)$ it follows that $L_p(G)$ for $1\leq p<\infty$ is metrically injective $A$-module iff $G$ is trivial. Now the result follows from proposition \ref{MetTopFlatCharac} and the facts that $C_0(G)^*\isom{\mathbf{mod}_1-L_1(G)}M(G)\isom{\mathbf{mod}_1-L_1(G)}L_1(G)$, $L_p(G)^*\isom{\mathbf{mod}_1-L_1(G)}L_{p^*}(G)$ for $1\leq p^*<\infty$.
\end{proof}

It is worth to mention here that if we would consider all Banach spaces over the field of real numbers, then $L_\infty(G)$ and $L_1(G)$ would be metrically projective and injective respectively,  additionally for $G$ consisting of two elements, because
$$
L_\infty(\mathbb{Z}_2)\isom{L_1(\mathbb{Z}_2)-\mathbf{mod}_1}\mathbb{R}_{\gamma_0}\bigoplus\nolimits_1\mathbb{R}_{\gamma_1},
\qquad
L_1(\mathbb{Z}_2)\isom{L_1(\mathbb{Z}_2)-\mathbf{mod}_1}\mathbb{R}_{\gamma_0}\bigoplus\nolimits_\infty\mathbb{R}_{\gamma_1}
$$
for $\gamma_0,\gamma_1\in\widehat{\mathbb{Z}_2}$ defined by $\gamma_0(0)=\gamma_0(1)=\gamma_1(0)=-\gamma_1(1)=1$. Here $\mathbb{Z}_2$ denotes the unique group of two elements.

\begin{proposition}\label{StdModFinGrL1MGTopInjProjFlatCharac} Let $G$ be a finite group. Then the $A$-modules $C_0(G)$, $M(G)$, $L_p(G)$ for $1\leq p\leq \infty$ are both topologically projective, injective and flat.
\end{proposition} 
\begin{proof}
For finite group $G$ we have $M(G)=L_1(G)$ and $C_0(G)=L_\infty(G)$, so these modules do not require special considerations. Since $M(G)=L_1(G)$, we can restrict our considerations to the case $A=L_1(G)$. The identity map $i:L_1(G)\to L_p(G):f\mapsto f$ is a topological isomorphism of Banach spaces, because $L_1(G)$ and $L_p(G)$ for $1\leq p<+\infty$ are of equal finite dimension. Since $G$ is finite, it is unimodular. Therefore, the module actions in $(L_1(G),\convol)$ and $(L_p(G),\convol_p)$ coincide for $1\leq p<+\infty$ and $i$ is an isomorphism in $L_1(G)-\mathbf{mod}$ and $\mathbf{mod}-L_1(G)$. Similarly one can show that $(L_\infty(G),\cdot_\infty)$ and $(L_p(G),\cdot_p)$ for $1<p\leq+\infty$ are isomorphic in $L_1(G)-\mathbf{mod}$ and $\mathbf{mod}-L_1(G)$. Finally, one can easily check that $(L_1(G),\convol)$ and $(L_\infty(G),\cdot_\infty)$ are isomorphic in $L_1(G)-\mathbf{mod}$ and $\mathbf{mod}-L_1(G)$ via the map $j:L_1(G)\to L_\infty(G):f\mapsto(s\mapsto f(s^{-1}))$. Therefore all the discussed modules are isomorphic. It remains to recall that $L_1(G)$ is topologically projective and flat by theorem \ref{L1ModL1MetTopProjCharac} and proposition \ref{LInfIsL1MetrInj}, while $L_\infty(G)$ is topologically injective by proposition \ref{LInfIsL1MetrInj}.
\end{proof}

Now we can summarize results on homological properties of modules of harmonic analysis into three tables. Each cell of the table contains a condition under which the respective module has respective property and propositions where this is proved. We shall mention that results for modules $L_p(G)$ are valid for both module actions $\convol_p$ and $\cdot_p$. Characterization and proofs for homologically trivial modules $\mathbb{C}_\gamma$ in case of relative theory is the same as in propositions \ref{OneDimL1ModMetTopProjCharac}, \ref{OneDimL1ModMetTopInjFlatCharac} and \ref{OneDimL1ModMetTopInjFlatCharac}. As usually, the arrow $\implies$ indicates that only a necessary conditions is known. As we showed above even topological theory is too restrictive for $L_1(G)$ to be projective as $L_1(G)$-module. Similarly a Banach space is topologically projective iff it is an $L_1$-space, and the underlying measure space is atomic. This analogy confirms important role of Banach geometry in metric and topological Banach homology.

\bigskip

\begin{scriptsize}
\begin{longtable}{|c|c|c|c|c|c|c|} 
\multicolumn{7}{c}{\mbox{Homologically trivial $L_1(G)$- and $M(G)$-modules in metric theory}}                                                                                                                                                                                                                                                                                                                                                                                                                                                                                                                                                                                                                                                                                                                                                                                                                                                                                                                                             \\
				 
\hline            & \multicolumn{3}{c|}{$L_1(G)$-modules}                                                                                                                                                                                                                                                                                                                                                                                                                                                                      & \multicolumn{3}{c|}{$M(G)$-modules}                                                                                                                                                                                                                                                                                                                                                                                                                                                                  \\
\hline
                  & \mbox{Projectivity}                                                                                                                                               & \mbox{Injectivity}                                                                                                                                                & \mbox{Flatness}                                                                                                                                                    & \mbox{Projectivity}                                                                                                                                               & \mbox{Injectivity}                                                                                                                                                & \mbox{Flatness}                                                                                                                                                   \\ 
\hline
 $L_1(G)$           & \begin{tabular}{@{}c@{}}$G$\mbox{ is discrete } \\ \ref{L1ModL1MetTopProjCharac}\end{tabular}                                                                     & \begin{tabular}{@{}c@{}}$G=\{e_G\}$ \\ \ref{StdModAreNotL1MGMetTopProjInjFlat}, \ref{StdModFinGrL1MGMetrInjProjFlatCharac}\end{tabular}                                  & \begin{tabular}{@{}c@{}}$G$\mbox{ is any } \\ \ref{LInfIsL1MetrInj}\end{tabular}                                                                                   & \begin{tabular}{@{}c@{}}$G$\mbox{ is discrete } \\ \ref{L1ModL1MetTopProjCharac},\ref{MGMetTopProjInjFlatRedToL1}\end{tabular}                                   & \begin{tabular}{@{}c@{}}$G=\{e_G\}$ \\ \ref{StdModAreNotL1MGMetTopProjInjFlat}, \ref{StdModFinGrL1MGMetrInjProjFlatCharac}\end{tabular}                                   & \begin{tabular}{@{}c@{}}$G$\mbox{ is any } \\ \ref{LInfIsL1MetrInj},\ref{MGMetTopProjInjFlatRedToL1}\end{tabular}                                                 \\ 
\hline
 $L_p(G)$           & \begin{tabular}{@{}c@{}}$G=\{e_G\}$ \\ \ref{StdModAreNotL1MGMetTopProjInjFlat},\ref{LpFinGrL1MGMetrInjProjCharac}\end{tabular}                  & \begin{tabular}{@{}c@{}}$G=\{e_G\}$ \\ \ref{StdModAreNotL1MGMetTopProjInjFlat},\ref{LpFinGrL1MGMetrInjProjCharac}\end{tabular}                  & \begin{tabular}{@{}c@{}}$G=\{e_G\}$ \\ \ref{StdModAreNotL1MGMetTopProjInjFlat},\ref{StdModFinGrL1MGMetrInjProjFlatCharac}\end{tabular}           & \begin{tabular}{@{}c@{}}$G=\{e_G\}$ \\ \ref{StdModAreNotL1MGMetTopProjInjFlat},\ref{LpFinGrL1MGMetrInjProjCharac}\end{tabular}                 & \begin{tabular}{@{}c@{}}$G=\{e_G\}$ \\ \ref{StdModAreNotL1MGMetTopProjInjFlat},\ref{LpFinGrL1MGMetrInjProjCharac}\end{tabular}                   & \begin{tabular}{@{}c@{}}$G=\{e_G\}$ \\ \ref{StdModAreNotL1MGMetTopProjInjFlat},\ref{StdModFinGrL1MGMetrInjProjFlatCharac}\end{tabular}          \\
\hline
 $L_\infty(G)$      & \begin{tabular}{@{}c@{}}$G=\{e_G\}$ \\ \ref{StdModAreNotL1MGMetTopProjInjFlat},\ref{LpFinGrL1MGMetrInjProjCharac}\end{tabular}                                           & \begin{tabular}{@{}c@{}}$G$\mbox{ is any } \\ \ref{LInfIsL1MetrInj}\end{tabular}                                                                                  & \begin{tabular}{@{}c@{}}$G=\{e_G\}$ \\ \ref{StdModAreNotL1MGMetTopProjInjFlat},\ref{StdModFinGrL1MGMetrInjProjFlatCharac}\end{tabular}                                    & \begin{tabular}{@{}c@{}}$G=\{e_G\}$ \\ \ref{StdModAreNotL1MGMetTopProjInjFlat},\ref{LpFinGrL1MGMetrInjProjCharac}\end{tabular}                                          & \begin{tabular}{@{}c@{}}$G$\mbox{ is any } \\ \ref{LInfIsL1MetrInj},\ref{MGMetTopProjInjFlatRedToL1}\end{tabular}                                                  & \begin{tabular}{@{}c@{}}$G=\{e_G\}$ \\ \ref{StdModAreNotL1MGMetTopProjInjFlat},\ref{StdModFinGrL1MGMetrInjProjFlatCharac}\end{tabular}                                   \\ 
\hline
$M(G)$              & \begin{tabular}{@{}c@{}}$G$\mbox{ is discrete } \\ \ref{L1MetTopProjAndMetrFlatOfMeasAlg}\end{tabular}                                                            & \begin{tabular}{@{}c@{}}$G=\{e_G\}$ \\ \ref{StdModAreNotL1MGMetTopProjInjFlat},\ref{StdModFinGrL1MGMetrInjProjFlatCharac}\end{tabular}                                   & \begin{tabular}{@{}c@{}}$G$\mbox{ is discrete } \\ \ref{MeasAlgIsL1TopFlat}\end{tabular}                                                                           & \begin{tabular}{@{}c@{}}$G$\mbox{ is any } \\ \ref{MGModMGMetTopProjFlatCharac}\end{tabular}                                                                     & \begin{tabular}{@{}c@{}}$G=\{e_G\}$ \\ \ref{StdModAreNotL1MGMetTopProjInjFlat},\ref{StdModFinGrL1MGMetrInjProjFlatCharac}\end{tabular}                                    & \begin{tabular}{@{}c@{}}$G$\mbox{ is any } \\ \ref{MGModMGMetTopProjFlatCharac}\end{tabular}                                                                      \\ 
\hline
$C_0(G)$            & \begin{tabular}{@{}c@{}}$G=\{e_G\}$ \\ \ref{StdModAreNotL1MGMetTopProjInjFlat},\ref{StdModFinGrL1MGMetrInjProjFlatCharac}\end{tabular}                                   & \begin{tabular}{@{}c@{}}$G$\mbox{ is finite } \\ \ref{StdModAreNotL1MGMetTopProjInjFlat},\ref{StdModFinGrL1MGMetrInjProjFlatCharac}\end{tabular}                         & \begin{tabular}{@{}c@{}}$G=\{e_G\}$ \\ \ref{StdModAreNotL1MGMetTopProjInjFlat},\ref{StdModFinGrL1MGMetrInjProjFlatCharac}\end{tabular}                                    & \begin{tabular}{@{}c@{}}$G=\{e_G\}$ \\ \ref{StdModAreNotL1MGMetTopProjInjFlat},\ref{StdModFinGrL1MGMetrInjProjFlatCharac}\end{tabular}                                  & \begin{tabular}{@{}c@{}}$G$\mbox{ is finite } \\ \ref{StdModAreNotL1MGMetTopProjInjFlat},\ref{StdModFinGrL1MGMetrInjProjFlatCharac}\end{tabular}                          & \begin{tabular}{@{}c@{}}$G=\{e_G\}$ \\ \ref{StdModAreNotL1MGMetTopProjInjFlat},\ref{StdModFinGrL1MGMetrInjProjFlatCharac}\end{tabular}                                   \\ 
\hline          
$\mathbb{C}_\gamma$ & \begin{tabular}{@{}c@{}}$G$\mbox{ is compact } \\ \ref{OneDimL1ModMetTopProjCharac}\end{tabular}                                                                  & \begin{tabular}{@{}c@{}}$G$\mbox{ is amenable } \\ \ref{OneDimL1ModMetTopInjFlatCharac}\end{tabular}                                                              & \begin{tabular}{@{}c@{}}$G$\mbox{ is amenable } \\ \ref{OneDimL1ModMetTopInjFlatCharac}\end{tabular}                                                               & \begin{tabular}{@{}c@{}}$G$\mbox{ is compact } \\ \ref{OneDimL1ModMetTopProjCharac},\ref{MGMetTopProjInjFlatRedToL1}\end{tabular}                                & \begin{tabular}{@{}c@{}}$G$\mbox{ is amenable } \\ \ref{OneDimL1ModMetTopInjFlatCharac},\ref{MGMetTopProjInjFlatRedToL1}\end{tabular}                              & \begin{tabular}{@{}c@{}}$G$\mbox{ is amenable } \\ \ref{OneDimL1ModMetTopInjFlatCharac},\ref{MGMetTopProjInjFlatRedToL1}\end{tabular}                             \\ 
\hline
\multicolumn{7}{c}{\mbox{Homologically trivial $L_1(G)$- and $M(G)$-modules in topological theory}}                                                                                                                                                                                                                                                                                                                                                                                                                                                                                                                                                                                                                                                                                                                                                                                                                                                                                                                                        \\
					 
\hline            & \multicolumn{3}{c|}{$L_1(G)$-modules}                                                                                                                                                                                                                                                                                                                                                                                                                                                                     & \multicolumn{3}{c|}{$M(G)$-modules}                                                                                                                                                                                                                                                                                                                                                                                                                                                                  \\
\hline
                  & \mbox{Projectivity}                                                                                                                                               & \mbox{Injectivity}                                                                                                                                                & \mbox{Flatness}                                                                                                                                                    & \mbox{Projectivity}                                                                                                                                               & \mbox{Injectivity}                                                                                                                                                & \mbox{Flatness}                                                                                                                                                   \\ 
\hline
$L_1(G)$            & \begin{tabular}{@{}c@{}}$G$\mbox{ is discrete } \\ \ref{L1ModL1MetTopProjCharac}\end{tabular}                                                                     & \begin{tabular}{@{}c@{}}$G$\mbox{ is finite } \\ \ref{StdModAreNotL1MGMetTopProjInjFlat}, \ref{StdModFinGrL1MGTopInjProjFlatCharac}\end{tabular}                         & \begin{tabular}{@{}c@{}}$G$\mbox{ is any } \\ \ref{LInfIsL1MetrInj}\end{tabular}                                                                                   & \begin{tabular}{@{}c@{}}$G$\mbox{ is discrete } \\ \ref{L1ModL1MetTopProjCharac},\ref{MGMetTopProjInjFlatRedToL1}\end{tabular}                                    & \begin{tabular}{@{}c@{}}$G$\mbox{ is finite } \\ \ref{StdModAreNotL1MGMetTopProjInjFlat}, \ref{StdModFinGrL1MGTopInjProjFlatCharac}\end{tabular}                         & \begin{tabular}{@{}c@{}}$G$\mbox{ is any } \\ \ref{LInfIsL1MetrInj},\ref{MGMetTopProjInjFlatRedToL1}\end{tabular}                                                 \\ 
\hline
 $L_p(G)$           & \begin{tabular}{@{}c@{}}$G$\mbox{ is finite } \\ \ref{StdModAreNotL1MGMetTopProjInjFlat},\ref{StdModFinGrL1MGTopInjProjFlatCharac}\end{tabular} & \begin{tabular}{@{}c@{}}$G$\mbox{ is finite } \\ \ref{StdModAreNotL1MGMetTopProjInjFlat},\ref{StdModFinGrL1MGTopInjProjFlatCharac}\end{tabular} & \begin{tabular}{@{}c@{}}$G$\mbox{ is finite } \\ \ref{StdModAreNotL1MGMetTopProjInjFlat},\ref{StdModFinGrL1MGTopInjProjFlatCharac}\end{tabular}  & \begin{tabular}{@{}c@{}}$G$\mbox{ is finite } \\ \ref{StdModAreNotL1MGMetTopProjInjFlat},\ref{StdModFinGrL1MGTopInjProjFlatCharac}\end{tabular} & \begin{tabular}{@{}c@{}}$G$\mbox{ is finite } \\ \ref{StdModAreNotL1MGMetTopProjInjFlat},\ref{StdModFinGrL1MGTopInjProjFlatCharac}\end{tabular} & \begin{tabular}{@{}c@{}}$G$\mbox{ is finite } \\ \ref{StdModAreNotL1MGMetTopProjInjFlat},\ref{StdModFinGrL1MGTopInjProjFlatCharac}\end{tabular} \\ 
\hline
 $L_\infty(G)$      & \begin{tabular}{@{}c@{}}$G$\mbox{ is finite } \\ \ref{StdModAreNotL1MGMetTopProjInjFlat},\ref{StdModFinGrL1MGTopInjProjFlatCharac}\end{tabular}                          & \begin{tabular}{@{}c@{}}$G$\mbox{ is any } \\ \ref{LInfIsL1MetrInj}\end{tabular}                                                                                  & \begin{tabular}{@{}c@{}}$G$\mbox{ is finite } \\ \ref{StdModAreNotL1MGMetTopProjInjFlat},\ref{StdModFinGrL1MGTopInjProjFlatCharac}\end{tabular}                           & \begin{tabular}{@{}c@{}}$G$\mbox{ is finite } \\ \ref{StdModAreNotL1MGMetTopProjInjFlat},\ref{StdModFinGrL1MGTopInjProjFlatCharac}\end{tabular}                          & \begin{tabular}{@{}c@{}}$G$\mbox{ is any } \\ \ref{LInfIsL1MetrInj},\ref{MGMetTopProjInjFlatRedToL1}\end{tabular}                                                 & \begin{tabular}{@{}c@{}}$G$\mbox{ is finite } \\ \ref{StdModAreNotL1MGMetTopProjInjFlat},\ref{StdModFinGrL1MGTopInjProjFlatCharac}\end{tabular}                          \\ 
\hline
$M(G)$              & \begin{tabular}{@{}c@{}}$G$\mbox{ is discrete } \\ \ref{L1MetTopProjAndMetrFlatOfMeasAlg}\end{tabular}                                                            & \begin{tabular}{@{}c@{}}$G$\mbox{ is finite } \\ \ref{StdModAreNotL1MGMetTopProjInjFlat},\ref{StdModFinGrL1MGTopInjProjFlatCharac}\end{tabular} & \begin{tabular}{@{}c@{}}$G$\mbox{ is any } \\ \ref{MeasAlgIsL1TopFlat}\end{tabular}                                                                                & \begin{tabular}{@{}c@{}}$G$\mbox{ is any } \\ \ref{MGModMGMetTopProjFlatCharac}\end{tabular}                                                                      & \begin{tabular}{@{}c@{}}$G$\mbox{ is finite } \\ \ref{StdModAreNotL1MGMetTopProjInjFlat},\ref{StdModFinGrL1MGTopInjProjFlatCharac}\end{tabular} & \begin{tabular}{@{}c@{}}$G$\mbox{ is any } \\ \ref{MGModMGMetTopProjFlatCharac}\end{tabular}                                                                      \\ 
\hline
$C_0(G)$            & \begin{tabular}{@{}c@{}}$G$\mbox{ is finite } \\ \ref{StdModAreNotL1MGMetTopProjInjFlat},\ref{StdModFinGrL1MGTopInjProjFlatCharac}\end{tabular}                          & \begin{tabular}{@{}c@{}}$G$\mbox{ is finite } \\ \ref{StdModAreNotL1MGMetTopProjInjFlat},\ref{StdModFinGrL1MGTopInjProjFlatCharac}\end{tabular}                          & \begin{tabular}{@{}c@{}}$G$\mbox{ is finite } \\ \ref{StdModAreNotL1MGMetTopProjInjFlat},\ref{StdModFinGrL1MGTopInjProjFlatCharac}\end{tabular}                           & \begin{tabular}{@{}c@{}}$G$\mbox{ is finite } \\ \ref{StdModAreNotL1MGMetTopProjInjFlat},\ref{StdModFinGrL1MGTopInjProjFlatCharac}\end{tabular}                          & \begin{tabular}{@{}c@{}}$G$\mbox{ is finite } \\ \ref{StdModAreNotL1MGMetTopProjInjFlat},\ref{StdModFinGrL1MGTopInjProjFlatCharac}\end{tabular}                          & \begin{tabular}{@{}c@{}}$G$\mbox{ is finite } \\ \ref{StdModAreNotL1MGMetTopProjInjFlat},\ref{StdModFinGrL1MGTopInjProjFlatCharac}\end{tabular}                          \\ 
\hline          
$\mathbb{C}_\gamma$ & \begin{tabular}{@{}c@{}}$G$\mbox{ is compact } \\ \ref{OneDimL1ModMetTopProjCharac}\end{tabular}                                                                  & \begin{tabular}{@{}c@{}}$G$\mbox{ is amenable } \\ \ref{OneDimL1ModMetTopInjFlatCharac}\end{tabular}                                                              & \begin{tabular}{@{}c@{}}$G$\mbox{ is amenable } \\ \ref{OneDimL1ModMetTopInjFlatCharac}\end{tabular}                                                               & \begin{tabular}{@{}c@{}}$G$\mbox{ is compact } \\ \ref{OneDimL1ModMetTopProjCharac},\ref{MGMetTopProjInjFlatRedToL1}\end{tabular}                                 & \begin{tabular}{@{}c@{}}$G$\mbox{ is amenable } \\ \ref{OneDimL1ModMetTopInjFlatCharac},\ref{MGMetTopProjInjFlatRedToL1}\end{tabular}                             & \begin{tabular}{@{}c@{}}$G$\mbox{ is amenable } \\ \ref{OneDimL1ModMetTopInjFlatCharac},\ref{MGMetTopProjInjFlatRedToL1}\end{tabular}                             \\ 
\hline
\multicolumn{7}{c}{\mbox{Homologically trivial $L_1(G)$- and $M(G)$-modules in relative theory}}                                                                                                                                                                                                                                                                                                                                                                                                                                                                                                                                                                                                                                                                                                                                                                                                                                                                                                                                           \\

\hline            & \multicolumn{3}{c|}{$L_1(G)$-modules}                                                                                                                                                                                                                                                                                                                                                                                                                                                                      & \multicolumn{3}{c|}{$M(G)$-modules}                                                                                                                                                                                                                                                                                                                                                                                                                                                                  \\
\hline
                  & \mbox{Projectivity}                                                                                                                                               & \mbox{Injectivity}                                                                                                                                                & \mbox{Flatness}                                                                                                                                                    & \mbox{Projectivity}                                                                                                                                               & \mbox{Injectivity}                                                                                                                                                & \mbox{Flatness}                                                                                                                                                   \\ 
\hline
$L_1(G)$            & \begin{tabular}{@{}c@{}}$G$\mbox{ is any } \\ \mbox{\cite{DalPolHomolPropGrAlg}, \S 6}\end{tabular}                                                               & \begin{tabular}{@{}c@{}}$G$\mbox{ is amenable } \\ \mbox{ and discrete } \\ \mbox{\cite{DalPolHomolPropGrAlg}, \S 6}\end{tabular}                                 & \begin{tabular}{@{}c@{}}$G$\mbox{ is any } \\ \mbox{\cite{DalPolHomolPropGrAlg}, \S 6}\end{tabular}                                                                & \begin{tabular}{@{}c@{}}$G$\mbox{ is any } \\ \mbox{\cite{RamsHomPropSemgroupAlg}, \S 3.5}\end{tabular}                                                           & \begin{tabular}{@{}c@{}}$G$\mbox{ is amenable } \\ \mbox{ and discrete } \\ \mbox{\cite{RamsHomPropSemgroupAlg}, \S 3.5}\end{tabular}                             & \begin{tabular}{@{}c@{}}$G$\mbox{ is any } \\ \mbox{\cite{RamsHomPropSemgroupAlg}, \S 3.5}\end{tabular}                                                           \\ 
\hline
 $L_p(G)$           & \begin{tabular}{@{}c@{}}$G$\mbox{ is compact } \\ \mbox{\cite{DalPolHomolPropGrAlg}, \S 6}\end{tabular}                                                           & \begin{tabular}{@{}c@{}}$G$\mbox{ is amenable } \\ \cite{RachInjModAndAmenGr}\end{tabular}                                                                        & \begin{tabular}{@{}c@{}}$G$\mbox{ is amenable } \\ \cite{RachInjModAndAmenGr}\end{tabular}                                                                         & \begin{tabular}{@{}c@{}}$G$\mbox{ is compact } \\ \mbox{\cite{RamsHomPropSemgroupAlg}, \S 3.5}\end{tabular}                                                       & \begin{tabular}{@{}c@{}}$G$\mbox{ is amenable } \\ \mbox{\cite{RamsHomPropSemgroupAlg}, \S 3.5}, \cite{RachInjModAndAmenGr}\end{tabular}                          & \begin{tabular}{@{}c@{}}$G$\mbox{ is amenable } \\ \mbox{\cite{RamsHomPropSemgroupAlg}, \S 3.5}\end{tabular}                                                      \\
\hline
 $L_\infty(G)$      & \begin{tabular}{@{}c@{}}$G$\mbox{ is finite } \\ \mbox{\cite{DalPolHomolPropGrAlg}, \S 6}\end{tabular}                                                            & \begin{tabular}{@{}c@{}}$G$\mbox{ is any } \\ \mbox{\cite{DalPolHomolPropGrAlg}, \S 6}\end{tabular}                                                               & \begin{tabular}{@{}c@{}}$G$\mbox{ is amenable } \\ \mbox{\cite{DalPolHomolPropGrAlg}, \S 6}\end{tabular}                                                           & \begin{tabular}{@{}c@{}}$G$\mbox{ is finite } \\ \mbox{\cite{RamsHomPropSemgroupAlg}, \S 3.5}\end{tabular}                                                        & \begin{tabular}{@{}c@{}}$G$\mbox{ is any } \\ \mbox{\cite{RamsHomPropSemgroupAlg}, \S 3.5}\end{tabular}                                                           & \begin{tabular}{@{}c@{}}$G$\mbox{ is amenable } \\ ($\implies$)\mbox{\cite{RamsHomPropSemgroupAlg}, \S 3.5}\end{tabular}                                          \\ 
\hline
$M(G)$              & \begin{tabular}{@{}c@{}}$G$\mbox{ is discrete } \\ \mbox{\cite{DalPolHomolPropGrAlg}, \S 6}\end{tabular}                                                          & \begin{tabular}{@{}c@{}}$G$\mbox{ is amenable }\\ \mbox{\cite{DalPolHomolPropGrAlg}, \S 6}\end{tabular}                                                           & \begin{tabular}{@{}c@{}}$G$\mbox{ is any } \\ \mbox{\cite{RamsHomPropSemgroupAlg}, \S 3.5}\end{tabular}                                                            & \begin{tabular}{@{}c@{}}$G$\mbox{ is any } \\ \mbox{\cite{RamsHomPropSemgroupAlg}, \S 3.5}\end{tabular}                                                           & \begin{tabular}{@{}c@{}}$G$\mbox{ is amenable } \\ \mbox{\cite{RamsHomPropSemgroupAlg}, \S 3.5}\end{tabular}                                                      & \begin{tabular}{@{}c@{}}$G$\mbox{ is any } \\ \mbox{\cite{RamsHomPropSemgroupAlg}, \S 3.5}\end{tabular}                                                           \\ 
\hline
$C_0(G)$            & \begin{tabular}{@{}c@{}}$G$\mbox{ is compact } \\ \mbox{\cite{DalPolHomolPropGrAlg}, \S 6}\end{tabular}                                                           & \begin{tabular}{@{}c@{}}$G$\mbox{ is finite } \\ \mbox{\cite{DalPolHomolPropGrAlg}, \S 6}\end{tabular}                                                            & \begin{tabular}{@{}c@{}}$G$\mbox{ is amenable } \\ \mbox{\cite{DalPolHomolPropGrAlg}, \S 6}\end{tabular}                                                           & \begin{tabular}{@{}c@{}}$G$\mbox{ is compact } \\ \mbox{\cite{RamsHomPropSemgroupAlg}, \S 3.5}\end{tabular}                                                       & \begin{tabular}{@{}c@{}}$G$\mbox{ is finite } \\ \mbox{\cite{RamsHomPropSemgroupAlg}, \S 3.5}\end{tabular}                                                        & \begin{tabular}{@{}c@{}}$G$\mbox{ is amenable } \\ \mbox{\cite{RamsHomPropSemgroupAlg}, \S 3.5}\end{tabular}                                                      \\ 
\hline          
$\mathbb{C}_\gamma$ & \begin{tabular}{@{}c@{}}$G$\mbox{ is compact } \\ \ref{OneDimL1ModMetTopProjCharac}\end{tabular}                                                                  & \begin{tabular}{@{}c@{}}$G$\mbox{ is amenable } \\ \ref{OneDimL1ModMetTopInjFlatCharac}\end{tabular}                                                              & \begin{tabular}{@{}c@{}}$G$\mbox{ is amenable } \\ \ref{OneDimL1ModMetTopInjFlatCharac}\end{tabular}                                                               & \begin{tabular}{@{}c@{}}$G$\mbox{ is compact } \\ \ref{OneDimL1ModMetTopProjCharac},\ref{MGMetTopProjInjFlatRedToL1}\end{tabular}                                 & \begin{tabular}{@{}c@{}}$G$\mbox{ is amenable } \\ \ref{OneDimL1ModMetTopInjFlatCharac},\ref{MGMetTopProjInjFlatRedToL1}\end{tabular}                             & \begin{tabular}{@{}c@{}}$G$\mbox{ is amenable } \\ \ref{OneDimL1ModMetTopInjFlatCharac},\ref{MGMetTopProjInjFlatRedToL1}\end{tabular}                             \\                   
\hline
\end{longtable}
\end{scriptsize}

% :1067,1124s/NoInfDimRefMetTopProjInjFlatModOverMthscrL1OrLInfty/StdModAreNotL1MGMetTopProjInjFlat/gc
%----------------------------------------------------------------------------------------
%	An example of small category.
%----------------------------------------------------------------------------------------

\section{An example of ``small'' category}
\label{SectionAnExampleOfSmallCategory}

As we have seen by a lot of examples above most of natural modules of algebras of analysis turn out to be homologically non trivial when considered with respect to big categories. The situation may change dramatically for comparatively ``small'' categories. This section is devoted to construction of meaningful example of this kind.

%----------------------------------------------------------------------------------------
%	The category of B(\Omega,\Sigma)-module L_p
%----------------------------------------------------------------------------------------

\subsection{The category of \texorpdfstring{$B(\Omega,\Sigma)$}{B(Omega,Sigma)}-modules \texorpdfstring{$L_p$}{Lp}}
\label{SubSectionTheCategoryOfBOmegaSigmaModulesLp}

An example of ``small'' category we shall construct is the category of $B(\Omega,\Sigma)$-modules of the form $L_p(\Omega,\mu)$ on some measurable space $(\Omega,\Sigma)$ with different $\sigma$-finite positive measures $\mu$. We denote it by $B(\Omega,\Sigma)-\mathbf{mod(L)}$. We shall show that all its modules are metrically and topologically projective, injective and flat. Along the way we shall give a complete characterization of topologically surjective, topologically injective, coisometric and isometric multiplication operators between $L_p$-spaces. In \cite{HelTensProdAndMultModLp} Helemskii described morphisms of $B(\Omega,\Sigma)-\mathbf{mod(L)}$, but only for a locally compact space $\Omega$, with Borel $\sigma$-algebra. Careful inspection of his proof shows that his characterization is valid for all $\sigma$-finite measure spaces. To properly state the result we need to introduce the notation. By $L_0(\Omega,\Sigma)$ we denote the linear space of measurable functions on $\Omega$. For a given $1\leq p,q\leq +\infty$ and positive $\sigma$-finite measures $\mu,\nu$ on a measurable space $(\Omega,\Sigma)$ denote $\Omega_+:=\{\omega\in\Omega_c^{\nu,\mu}:\rho_{\nu,\mu}(\omega)>0\}$ and
$$
L_{p,q,\mu,\nu}(\Omega):=
\begin{cases}
\{g\in L_0(\Omega,\Sigma):g\in L_{pq/(p-q)}(\Omega,\rho_{\nu,\mu}^{p/(p-q)}\mu),\quad g|_{\Omega\setminus\Omega_+}=0\}&\text{if}\quad p>q\\
\{g\in L_0(\Omega,\Sigma):g\rho_{\nu,\mu}^{1/p}\in L_{\infty}(\Omega,\mu),\quad g|_{\Omega\setminus\Omega_+}=0\}&\text{if}\quad p=q\\
\{g\in L_0(\Omega,\Sigma):g\rho_{\nu,\mu}^{1/p}\mu^{pq/(p-q)}\in L_{\infty}(\Omega,\mu),\quad g|_{\Omega\setminus\Omega_a^{\mu}}=0\}&\text{if}\quad p<q\\
\end{cases}
$$
$$
\Vert g\Vert_{L_{p,q,\mu,\nu}(\Omega)}:=
\begin{cases}
\Vert g\Vert_{L_{pq/(p-q)}(\Omega,\rho_{\nu,\mu}^{p/(p-q)}\mu)}&\text{if}\quad p>q\\
\Vert g\rho_{\nu,\mu}^{1/p}\Vert_{L_{\infty}(\Omega,\mu)}&\text{if}\quad p=q\\
\Vert g\rho_{\nu,\mu}^{1/p}\mu^{pq/(p-q)}\Vert_{L_{\infty}(\Omega,\mu)}&\text{if}\quad p<q\\
\end{cases}
$$


\begin{theorem}[\cite{HelTensProdAndMultModLp}, theorem 4.1]\label{LpModMorphCharac}
Let $(\Omega,\Sigma)$ be a measurable space, $1\leq p,q\leq +\infty$ and $\mu,\nu$ be two $\sigma$-finite measures on $(\Omega, \Sigma)$. Then there exists an isometric isomorphism
$$
\mathcal{I}_{p,q,\mu,\nu}:L_{p,q,\mu,\nu}(\Omega)\to\operatorname{Hom}_{B(\Omega,\Sigma)-\mathbf{mod(L)}}(L_p(\Omega,\mu),L_q(\Omega,\nu)):g\mapsto (f\mapsto g f)
$$
\end{theorem}

Simply speaking all morphisms in $B(\Omega,\Sigma)-\mathbf{mod(L)}$ are multiplication operators.

%----------------------------------------------------------------------------------------
%	Decomposition of Lp-spaces
%----------------------------------------------------------------------------------------

\subsection{Decomposition of \texorpdfstring{$L_p$}{Lp}-spaces}
\label{SubSectionDecompositionOfLpSpaces}

For further investigations we need to recall some obvious facts on decomposition of $L_p$-spaces induced by decomposition of their underlying measure spaces. Let $1\leq p\leq+\infty$ and $(\Omega,\Sigma,\mu)$ be a measure space. If the whole space $\Omega$ is a single atom, then its $L_p$-space is one dimensional and we have an isometric isomorphism
$$
J_p:L_p(\Omega,\mu|_{\Omega})\to \ell_p(\mathbb{N}_1):f\mapsto\left(1\mapsto \mu(\Omega)^{1/p-1}\int_{\Omega} f(\omega)d\mu(\omega)\right)
$$
If $\Omega=\bigcup_{\lambda\in\Lambda}\Omega_\lambda$ is a representation of $\Omega$ as disjoint union of measurable sets, then for all $1\leq p\leq+\infty$ we always have an isometric isomorphism
$$
I_p:L_p(\Omega,\mu)\to \bigoplus\nolimits_p\{ L_p(\Omega_\lambda,\mu|_{\Omega_\lambda}):\lambda\in\Lambda\}: f\mapsto (\lambda\mapsto f|_{\Omega_\lambda})
$$
If each set $\Omega_\lambda$ is an atom, then $\Omega$ is a purely atomic measure space and we have one more isometric isomorphism
$$
\widetilde{I}_p:L_p(\Omega,\mu)\to \ell_p(\Lambda):f\mapsto\left (\lambda\mapsto \mu(\Omega_\lambda)^{1/p-1}\int_{\Omega_\lambda} f(\omega)d\mu(\omega)\right)
$$
Note: when dealing with $p$ indexes we take by definition that $1/0=\infty$ and $1/\infty=0$. Another useful technique in the study of $L_p$-spaces is a so called change of density: if $\rho:\Omega\to(0,+\infty)$ is a measurable function, then
$$
\bar{I}_p:L_p(\Omega,\mu)\to L_p(\Omega,\rho\mu): f\mapsto\rho^{-1/p} f
$$
is an isometric isomorphism. For different values of $p$ infinite dimensional $L_p$-spaces are not topologically isomorphic. This can be proved via type and cotype techniques [\cite{KalAlbTopicsBanSpTh}, theorem 6.2.14]. Obviously, in finite dimensional setting we have an isomorphism only for spaces of equal dimension. More precisely: if $\Lambda$ is a finite set and $1\leq p,q\leq +\infty$, then there exists a $c_{p,q}>0$ such that $\Vert x\Vert_{\ell_p(\Lambda)}\leq c_{p,q}\Vert x\Vert_{\ell_q(\Lambda)}$ for all $x\in\mathbb{C}^\Lambda$.


%----------------------------------------------------------------------------------------
%	Multiplication operators
%----------------------------------------------------------------------------------------

\subsection{Multiplication operators}
\label{SubSectionMultiplicationOperators}

Let $(\Omega,\Sigma,\mu)$ and $(\Omega,\Sigma,\nu)$ be two measure spaces with the same $\sigma$-algebra of measurable sets. For a given $g\in L_0(\Omega,\Sigma)$ and $1\leq p,q\leq +\infty$ we define the multiplication operator
$$
M_g:L_p(\Omega,\mu)\to L_q(\Omega,\nu): f\mapsto g f
$$ 
Of course certain restrictions on $g$, $\mu$ and $\nu$ are required for $M_g$ to be well defined. For a given $E\in\Sigma$ by $M_g^E$ we denote the linear operator
$$
M_g^E:L_p(E,\mu|_E)\to L_q(E,\nu|_E):f\mapsto g|_E f
$$
It is well defined because $f|_{\Omega\setminus E}=0$ implies $M_g(f)|_{\Omega\setminus E}=0$. 

\begin{proposition}\label{MultpOpSurjInjDesc} Let $(\Omega,\Sigma,\mu)$ be a measure space and $g\in L_0(\Omega,\Sigma)$ and denote $Z_g:=g^{-1}(\{0\})$. Then for the operator $M_g:L_p(\Omega,\mu)\to L_q(\Omega,\mu)$ we have

$i)$ $\operatorname{Ker}(M_g)=\{f\in L_p(\Omega,\mu):f|_{\Omega\setminus {Z_g}}=0\}$, so $M_g$ is injective iff $\mu(Z_g)=0$

$ii)$ $\operatorname{Im}(M_g)\subset\{h\in L_q(\Omega,\mu): h|_{Z_g}=0\}$, so if $M_g$ is surjective then $\mu(Z_g)=0$.

\end{proposition}
\begin{proof}
$i)$ The desired equality follows from the chain of equivalences
$f\in\operatorname{Ker}(M_g)
\Longleftrightarrow g f=0
\Longleftrightarrow f|_{\Omega\setminus Z_g}=0$


$ii)$ Since $g|_{Z_g}=0$ then for all $f\in L_p(\Omega,\mu)$ we have $M_g(f)|_{Z_g}=(g f)|_{Z_g}=0$, thus we get the inclusion. If $M_g$ is surjective then, clearly,  $\mu(Z_g)=0$.
\end{proof}

For a given $E\in \Sigma$ and $f\in L_0(E,\Sigma|_{E})$ by $\widetilde{f}$ we will denote the function in $L_0(\Omega, \Sigma)$ such that $\widetilde{f}(\omega)=f(\omega)$ if $\omega\in E$ and $\widetilde{f}(\omega)=0$ otherwise.

\begin{proposition}\label{MultOpDecompDecomp} Let $(\Omega,\Sigma,\mu)$, $(\Omega,\Sigma,\nu)$ be measure spaces and $1\leq p,q\leq +\infty$. Assume we have a representation $\Omega=\bigcup_{\lambda\in\Lambda}\Omega_\lambda$ of $\Omega$ as finite disjoint union of measurable sets. Then 

$i)$ operator $M_g$ is $c$-topologically injective for some $c>0$ iff operators $M_g^{\Omega_\lambda}$ are $c'$-topologically injective for all $\lambda\in\Lambda$ and some $c'>0$

$ii)$ operator $M_g$ is $c$-topologically surjective for some $c>0$ iff operators $M_g^{\Omega_\lambda}$ are $c'$-topologically surjective for all $\lambda\in\Lambda$ and some $c'>0$

$iii)$ if operator $M_g$ is isometric then so does $M_g^{\Omega_\lambda}$ for all $\lambda\in\Lambda$

$iv)$ if operator $M_g$ is coisometric then so does $M_g^{\Omega_\lambda}$ for all $\lambda\in\Lambda$

\end{proposition}
\begin{proof}
$i)$ Let $M_g$ be $c$-topologically injective. Fix $\lambda\in\Lambda$ and $f\in L_p(\Omega_\lambda,\mu|_{\Omega_\lambda})$, then 
$$
\Vert M_g^{\Omega_\lambda}(f)\Vert_{L_q(\Omega_\lambda,\nu|_{\Omega_\lambda})}
=\Vert g \widetilde{f}\Vert_{L_q(\Omega,\nu)}
\geq c^{-1}\Vert\widetilde{f}\Vert_{L_p(\Omega,\mu)}
=c^{-1}\Vert f\Vert_{L_p(\Omega_\lambda,\mu|_{\Omega_\lambda})}
$$
So $M_g^{\Omega_\lambda}$ is $c$-topologically injective for all $\lambda\in\Lambda$. 

Conversely, assume that operators $\{M_g^{\Omega_\lambda}:\lambda\in\Lambda\}$ are $c'$-topologically injective. Let $f\in L_p(\Omega,\mu)$, then 
$$
\Vert M_g(f)\Vert_{L_q(\Omega,\nu)}
=\left\Vert\left(\Vert M_g^{\Omega_\lambda}(f|_{\Omega_\lambda})\Vert_{L_q(\Omega_\lambda,\nu|_{\Omega_\lambda})}:\lambda\in\Lambda\right)\right\Vert_{\ell_q(\Lambda)}
$$
$$
\geq (c')^{-1}\left\Vert\left(\Vert f|_{\Omega_\lambda}\Vert_{L_p(\Omega_\lambda,\mu|_{\Omega_\lambda})}:\lambda\in\Lambda\right)\right\Vert_{\ell_q(\Lambda)}
$$
$$
\geq (c')^{-1} c_{p,q}^{-1}\left\Vert\left(\Vert f|_{\Omega_\lambda}\Vert_{L_p(\Omega_\lambda,\mu|_{\Omega_\lambda})}:\lambda\in\Lambda\right)\right\Vert_{\ell_p(\Lambda)}
=(c')^{-1}c_{p,q}^{-1}\Vert f\Vert_{L_p(\Omega,\mu)}
$$
Since $f$ is arbitrary, then $M_g$ is $c$-topologically injective for $c=c'c_{p,q}>0$.

$ii)$ Let $M_g$ be $c$-topologically surjective. Fix $\lambda\in\Lambda$ and $h\in L_q(\Omega_\lambda,\nu|_{\Omega_\lambda})$. Then there exists $f\in L_p(\Omega,\mu)$ such that $M_g(f)=\widetilde{h}$ and $\Vert f\Vert_{L_p(\Omega,\mu)}< c\Vert \widetilde{h}\Vert_{L_q(\Omega,\nu)}$. Then $M_g^{\Omega_\lambda}(f|_{\Omega_\lambda})=\widetilde{h}|_{\Omega_\lambda}=h$ and $\Vert f|_{\Omega_\lambda}\Vert_{L_p(\Omega_\lambda,\mu|_{\Omega_\lambda})}\leq \Vert f\Vert_{L_p(\Omega,\mu)}< c\Vert\widetilde{h}\Vert_{L_q(\Omega,\nu)}=c\Vert h\Vert_{L_q(\Omega_\lambda,\nu|_{\Omega_\lambda})}$. Since $h$ is arbitrary, then $M_g^{\Omega_\lambda}$ is $c$-topologically surjective for all $\lambda\in\Lambda$.

Conversely, assume operators $\{M_g^{\Omega_\lambda}:\lambda\in\Lambda\}$ are $c'$-topologically surjective. Let $h\in L_q(\Omega,\nu)$. From assumption for each $\lambda\in\Lambda$ we have $f_\lambda\in L_p(\Omega_\lambda,\mu|_{\Omega_\lambda})$ such that $M_g^{\Omega_\lambda}(f_\lambda)=h|_{\Omega_\lambda}$ and $\Vert f_\lambda\Vert_{L_p(\Omega_\lambda,\mu|_{\Omega_\lambda})}< c'\Vert h|_{\Omega_\lambda}\Vert_{L_q(\Omega_\lambda,\nu|_{\Omega_\lambda})}$. Define $f\in L_0(\Omega,\Sigma)$ such that $f(\omega)=f_\lambda(\omega)$ if $\omega\in\Omega_\lambda$, then
$$
\Vert f\Vert_{L_p(\Omega,\mu)}
=\left\Vert\left(\Vert f_\lambda\Vert_{L_p(\Omega_\lambda,\mu|_{\Omega_\lambda})}:\lambda\in\Lambda\right)\right\Vert_{\ell_p(\Lambda)}
< c'\left\Vert\left(\Vert h|_{\Omega_\lambda}\Vert_{L_q(\Omega_\lambda,\nu|_{\Omega_\lambda})}:\lambda\in\Lambda\right)\right\Vert_{\ell_p(\Lambda)}
$$
$$
\leq c'c_{p,q}\left\Vert\left(\Vert h|_{\Omega_\lambda}\Vert_{L_q(\Omega_\lambda,\nu|_{\Omega_\lambda})}:\lambda\in\Lambda\right)\right\Vert_{\ell_q(\Lambda)}
=c'c_{p,q}\Vert h\Vert_{L_q(\Omega,\nu)}
$$
Obviously, $M_g(f)=h$. Since $h$ is arbitrary, then we get that $M_g$ is $c$-topologically surjective for $c=c'c_{p,q}>0$.

$iii)$ Fix $\lambda\in\Lambda$ and $f\in L_p(\Omega_\lambda,\mu|_{\Omega_\lambda})$, then 
$$
\Vert M_g^{\Omega_\lambda}(f)\Vert_{L_q(\Omega_\lambda,\nu|_{\Omega_\lambda})}
=\Vert g \widetilde{f}\Vert_{L_q(\Omega,\nu)}
=\Vert\widetilde{f}\Vert_{L_p(\Omega,\mu)}
=\Vert f\Vert_{L_p(\Omega_\lambda,\mu|_{\Omega_\lambda})}
$$
So $M_g^{\Omega_\lambda}$ is isometric for all $\lambda\in\Lambda$

$iv)$ Fix $\lambda\in\Lambda$. Since $M_g$ is coisometric, then it is $1$-topologically surjective and contractive. So from paragraph $ii)$ we see that $M_g^{\Omega_\lambda}$ is $1$-topologically surjective. Let $f\in L_p(\Omega_\lambda,\mu|_{\Omega_\lambda})$. Since $M_g$ is contractive we get
$$
\Vert M_g^{\Omega_\lambda}(f)\Vert_{L_q(\Omega_\lambda,\nu|_{\Omega_\lambda})}
=\Vert M_g(\widetilde{f})\chi_{\Omega_\lambda}\Vert_{L_q(\Omega,\nu)}
=\Vert M_g(\widetilde{f}\chi_{\Omega_\lambda})\Vert_{L_q(\Omega,\nu)}
$$
$$
\leq \Vert\widetilde{f}\chi_{\Omega_\lambda}\Vert_{L_p(\Omega,\mu)}
=\Vert f\Vert_{L_p(\Omega_{\lambda},\mu|_{\Omega_\lambda})}
$$
Since $M_g^{\Omega_\lambda}$ is contractive and $1$-topologically surjective it is coisometric.
\end{proof}


\begin{proposition}\label{MultOpCharacBtwnTwoSingMeasSp} Let $(\Omega,\Sigma,\mu)$ and $(\Omega,\Sigma,\nu)$ be two $\sigma$-finite measure spaces. Let $1\leq p,q\leq +\infty$ and $g\in L_0(\Omega,\Sigma)$. If $\mu\perp\nu$, then $M_g:L_p(\Omega,\mu)\to L_q(\Omega,\nu)$ is the zero operator.
\end{proposition}
\begin{proof} Since $\mu\perp\nu$, then there exists an $\Omega_s^{\nu,\mu}\in\Sigma$ such that $\mu(\Omega_s^{\nu,\mu})=\nu(\Omega_c^{\nu,\mu})=0$, where $\Omega_c^{\nu,\mu}=\Omega\setminus\Omega_s^{\nu,\mu}$. Since $\mu(\Omega_s^{\nu,\mu})=0$, then $\chi_{\Omega_c^{\nu,\mu}}=\chi_{\Omega}$ in $L_p(\Omega,\mu)$ and $\chi_{\Omega_c^{\nu,\mu}}=0$ in $L_q(\Omega,\nu)$. Now for all $f\in L_p(\Omega,\mu)$ we have $M_g(f)=M_g(f \chi_{\Omega})=M_g(f \chi_{\Omega_c^{\nu,\mu}})=g f\chi_{\Omega_c^{\nu,\mu}}=0$. Since $f$ is arbitrary, then $M_g=0$.
\end{proof}

From this point we start our main study of multiplication operators. We shall show that $\langle$~isometric / topologically injective~$\rangle$ morphisms in $B(\Omega,\Sigma)-\mathbf{mod(L)}$ are coretractions in $\langle$~$\mathbf{Ban}_1$ / $\mathbf{Ban}$~$\rangle$, while $\langle$~strictly coisometric / topologically surjective~$\rangle$ morphisms in $B(\Omega,\Sigma)-\mathbf{mod(L)}$ are retractions in $\langle$~$\mathbf{Ban}_1$ / $\mathbf{Ban}$~$\rangle$. Using this characterizations we shall easily describe metrically and topologically projective, injective and flat modules of the category $B(\Omega,\Sigma)-\mathbf{mod(L)}$.

\begin{proposition}\label{MultpOpPropIfPeqqualsQ} Let $(\Omega,\Sigma,\mu)$ be a measure space and $g\in L_0(\Omega,\Sigma)$. Then 

$i)$ $M_g\in\mathcal{B}(L_p(\Omega,\mu))$ iff $g\in L_\infty(\Omega,\mu)$;

$ii)$ $M_g$ is a topological isomorphism iff $c\leq |g|\leq c'$ for some $c,c'>0$.
\end{proposition}
\begin{proof}
$i)$ Consider $M_g\in\mathcal{B}(L_p(\Omega,\mu))$. Assume there exists an $E\in\Sigma$ with $\mu(E)>0$ such that $|g|_E|>\Vert M_g\Vert$, then
$$
\Vert M_g(\chi_E)\Vert_{L_p(\Omega,\mu)}
=\Vert g\chi_E\Vert_{L_p(\Omega,\mu)}
>\Vert M_g\Vert\Vert\chi_E\Vert_{L_p(\Omega,\mu)}
$$
Contradiction, hence for all $E\in\Sigma$ with $\mu(E)>0$ we have $|g|_E|\leq \Vert M_g\Vert$ i.e.  $|g|\leq \Vert M_g\Vert$. Thus $g\in L_\infty(\Omega,\mu)$. Conversely, let $g\in L_\infty(\Omega,\mu)$. Now for any $1\leq p\leq +\infty$ and $f\in L_p(\Omega,\mu)$ we have
$$
\Vert M_g(f)\Vert_{L_p(\Omega,\mu)}
=\Vert g  f\Vert_{L_p(\Omega,\mu)}
\leq \Vert g\Vert_{L_\infty(\Omega,\mu)}\Vert f\Vert_{L_p(\Omega,\mu)}
$$
Hence $M_g\in\mathcal{B}(L_p(\Omega,\mu))$.

$ii)$ Note that $M_g^{-1}=M_{1/g}$ as linear maps provided $1/g$ is well defined. Now $M_g$ is a topological isomorphism iff $M_g$ and $M_g^{-1}$ are bounded operators. From previous paragraph and equality $M_g^{-1}=M_{1/g}$ we see that it is equivalent to boundedness of $g$ and $1/g$. This is equivalent to $c\leq|g|\leq c'$ for some $c,c'>0$.
\end{proof}

\begin{proposition}\label{EquivMultOp} Let $(\Omega,\Sigma,\mu)$ be a $\sigma$-finite purely atomic measure space, $1\leq p,q\leq +\infty$ and $g\in L_0(\Omega,\Sigma)$. Then the operator $\widetilde{M}_{\widetilde{g}}:=\widetilde{I}_q M_g\widetilde{I}_p^{-1}\in\mathcal{B}(\ell_p(\Lambda),\ell_q(\Lambda))$ is a multiplication operator by the function $\widetilde{g}:\Lambda\to\mathbb{C}:\lambda\mapsto \mu(\Omega_\lambda)^{1/q-1/p-1}\int_{\Omega_\lambda}f(\omega)d\mu(\omega)$ where $\{\Omega_\lambda:\lambda\in\Lambda\}$ is at most countable decomposition of $\Omega$ into disjoint family of atoms.
\end{proposition}
\begin{proof} Let $1\leq p,q\leq +\infty$. For any $x\in\ell_p(\Lambda)$ we have
$$
\widetilde{M}_{\widetilde{g}}(x)(\lambda)
=(\widetilde{I}_q((M_g\widetilde{I}_p^{-1})(x))(\lambda)
=J_q(M_g(\widetilde{I}_p^{-1}(x))|_{\Omega_\lambda})(1)
$$
$$
=J_q((g \widetilde{I}_p^{-1}(x))|_{\Omega_\lambda})(1)
=\mu(\Omega_\lambda)^{1/q-1}\int_{\Omega_\lambda}(g|_{\Omega_\lambda} \widetilde{I}_p^{-1}(x)|_{\Omega_\lambda})(\omega)d\mu(\omega)
$$
$$
=\mu(\Omega_\lambda)^{1/q-1}\int_{\Omega_\lambda}(g \mu(\Omega)^{-1/p}x(\lambda)\chi_{\Omega_{\lambda}})(\omega)d\mu(\omega)
=x(\lambda)\mu(\Omega_\lambda)^{1/q-1/p-1}\int_{\Omega_\lambda} g(\omega)d\mu(\omega)
$$
Thus $\widetilde{M}_{\widetilde{g}}$ is a multiplication operator where $\widetilde{g}(\lambda)=\mu(\Omega_\lambda)^{1/q-1/p-1}\int_{\Omega_\lambda} g(\omega)d\mu(\omega)$.
\end{proof}

Since $\widetilde{I}_p$ and $\widetilde{I}_q$ are isometric isomorphisms then $M_g$ is topologically injective iff $\widetilde{M}_{\widetilde{g}}$ is topologically injective. 

\begin{proposition}\label{TopInjMultOpCharacOnPureAtomMeasSp} Let $(\Omega,\Sigma,\mu)$ be $\sigma$-finite purely atomic measure space, $1\leq p,q\leq +\infty$ and $g\in L_0(\Omega,\Sigma)$. Then the following are equivalent:

$i)$ $M_g\in\mathcal{B}(L_p(\Omega,\mu),L_q(\Omega,\mu))$ is topologically injective;

$ii)$ $|g|\geq c$ for some $c>0$ and if $p\neq q$ the space $(\Omega,\Sigma,\mu)$ consist of finitely many atoms.
\end{proposition}
\begin{proof}
$i)$$\implies$$ ii)$ Assume $M_g$ is topologically injective, then so does $\widetilde{M}_{\widetilde{g}}$, i.e. $\Vert\widetilde{M}_{\widetilde{g}}(x)\Vert_{\ell_q(\Lambda)}\geq c'\Vert x\Vert_{\ell_p(\Lambda)}$ for some $c'>0$ and all $x\in\ell_p(\Lambda)$. By $\{\Omega_\lambda:\lambda\in\Lambda\}$ we denote at most countable decomposition  of $\Omega$ into disjoint family atoms. We shall consider two cases.

1) Let $p\neq q$. Assume $\Lambda$ is countable. 

1.1) Consider subcase $p,q<+\infty$. Since $\Lambda$ is countable, then we get a  contradiction, because by Pitt's theorem [\cite{KalAlbTopicsBanSpTh}, proposition 2.1.6] there is no embedding of $\ell_p(\Lambda)$ into $\ell_q(\Lambda)$ for countable $\Lambda$ and $1\leq p,q< +\infty$, $p\neq q$. 

1.2) Consider subcase $1\leq p <+\infty$ and $q=+\infty$. Take any $F\in\mathcal{P}_0(\Lambda)$, then 
$$
\sup_{\lambda\in\Lambda}|\widetilde{g}(\lambda)|
\geq\max_{\lambda\in F}|\widetilde{g}(\lambda)|
=\left\Vert\widetilde{M}_{\widetilde{g}}\left(\sum_{\lambda\in F}\delta_\lambda\right)\right\Vert_{\ell_\infty(\Lambda)}
\geq c'\left\Vert\sum_{\lambda\in F}\delta_\lambda\right\Vert_{\ell_p(\Lambda)}
=c'\operatorname{Card}(F)^{1/p}
$$
Since $\Lambda$ is countable, then $\sup_{\lambda\in\Lambda}|\widetilde{g}(\lambda)|\geq c'\sup_{F\in\mathcal{P}_0(\Lambda)}\operatorname{Card}(F)^{1/p}=+\infty$. On the other hand, since $\widetilde{M}_{\widetilde{g}}$ is bounded we have 
$$\sup_{\lambda\in\Lambda}|\widetilde{g}(\lambda)|
=\sup_{\lambda\in\Lambda}\Vert\widetilde{M}_{\widetilde{g}}(\delta_\lambda)\Vert_{\ell_\infty(\Lambda)}
\leq\Vert\widetilde{M}_{\widetilde{g}}\Vert\Vert \delta_\lambda\Vert_{\ell_p(\Lambda)}
=\Vert\widetilde{M}_{\widetilde{g}}\Vert<+\infty
$$
Contradiction.

1.3) Consider subcase $1\leq q<+\infty$ and $p=+\infty$. Since $\Lambda$ is countable, then $\ell_\infty(\Lambda)$ is non separable and $\ell_q(\Lambda)$ is separable. As $\widetilde{M}_{\widetilde{g}}$ is topologically injective, then $\operatorname{Im}(\widetilde{M}_{\widetilde{g}})$ is a non separable subspace of $\ell_q(\Lambda)$. Contradiction.

In all subcases we got a contradiction. Hence $\Lambda$ is finite i.e. $(\Omega,\Sigma,\mu)$ consist of finitely many atoms. Obviously,
$g$ is completely determined by its values $k_\lambda\in\mathbb{C}$ on atoms $\{\Omega_\lambda:\lambda\in\Lambda\}$. By proposition \ref{MultpOpSurjInjDesc} the function $g$ is zero only on sets of measure zero, so $k_\lambda\neq 0$ for all $\lambda\in\Lambda$. Since $\Lambda$ is finite we conclude $|g|\geq c$ for $c=\min_{\lambda\in\Lambda}|k_\lambda|$. 


2) Let $p=q$. Fix $\lambda\in\Lambda$, then
$$
|\widetilde{g}(\lambda)|
=\Vert \widetilde{g} \delta_\lambda\Vert_{\ell_q(\Lambda)}
=\Vert \widetilde{M}_{\widetilde{g}}(\delta_\lambda)\Vert_{\ell_q(\Lambda)}
\geq c'\Vert \delta_\lambda\Vert_{\ell_p(\Lambda)}
=c'
$$
For $\mu$-almost all $\omega\in\Omega_\lambda$ we have
$$
|g(\omega)|
=\left|\mu(\Omega_\lambda)^{-1}\int_{\Omega_\lambda}g(\omega)d\mu(\omega)\right|
=\left|\mu(\Omega_\lambda)^{-1}\mu(\Omega_\lambda)^{1+1/p-1/p}\widetilde{g}(\lambda)\right|
=|\widetilde{g}(\lambda)|\geq c'
$$
Since $\lambda\in\Lambda$ is arbitrary and $\Omega=\bigcup_{\lambda\in\Lambda}\Omega_\lambda$, then $|g|\geq c'$.



$ii)$$\implies$$ i)$ Assume $|g|\geq c$ for $c>0$. Then from proposition \ref{EquivMultOp} we see that $|\widetilde{g}|\geq c$.

1) Let $p\neq q$. By assumption $(\Omega,\Sigma,\mu)$ consist of finitely many atoms, then $L_p(\Omega,\mu)$ is finite dimensional. From assumption on $g$ we see that it has no zero values, hence operator $M_g$ is topologically injective. 

2) Let $p=q$, then for all $x\in\ell_p(\Lambda)$ we have
$$
\Vert \widetilde{M}_{\widetilde{g}}(x)\Vert_{\ell_p(\Lambda)}=\Vert g x\Vert_{\ell_p(\Lambda)}\geq c\Vert x\Vert_{\ell_p(\Lambda)}
$$
so $\widetilde{M}_{\widetilde{g}}$ is topologically injective and so does $M_g$.
\end{proof}

\begin{proposition}\label{TopInjMultOpCharacOnNonAtomMeasSp} Let $(\Omega,\Sigma,\mu)$ be a non atomic measure space, $1\leq p,q\leq +\infty$ and $g\in L_0(\Omega,\Sigma)$. Then the following are equivalent:

$i)$ $M_g\in\mathcal{B}(L_p(\Omega,\mu),L_q(\Omega,\mu))$ is topologically injective;

$ii)$ $|g|\geq c$ for some $c>0$ and $p=q$.
\end{proposition}
\begin{proof}
$i)$$\implies$$ ii)$ Assume $M_g$ is topologically injective i.e. $\Vert M_g(f)\Vert_{L_q(\Omega,\mu)}\geq c\Vert f\Vert_{L_p(\Omega,\mu)}$ for some $c>0$ and all $f\in L_p(\Omega,\mu)$. We shall consider three cases.

1)  Let $p>q$. There exist $c'>0$ and $E\in\Sigma$ with $\mu(E)>0$ such that $|g|_E|\leq c'$, otherwise $M_g$ is not well defined. Since $(\Omega,\Sigma,\mu)$ is a non atomic measure space, then we have a sequence $\{E_n:n\in\mathbb{N}\}\subset\Sigma$ of subsets of $E$ such that $\mu(E_n)=2^{-n}$. Then since $p>q$ we get
$$
c
\leq\frac{\Vert M_g(\chi_{E_n})\Vert_{L_q(\Omega,\mu)}}{\Vert \chi_{E_n}\Vert_{L_p(\Omega,\mu)}}
\leq\frac{c'\Vert\chi_{E_n}\Vert_{L_q(\Omega,\mu)}}{\Vert \chi_{E_n}\Vert_{L_p(\Omega,\mu)}}
\leq c'\mu(E_n)^{1/q-1/p},
$$
$$
c
\leq\inf_{n\in\mathbb{N}}c'\mu(E_n)^{1/q-1/p}
=c'\inf_{n\in\mathbb{N}} 2^{n(1/p-1/q)}=0
$$
Contradiction, so in this case $M_g$ can not be topologically injective.

2) Let $p=q$. Fix $\epsilon > 0$. Assume there exist $E\in\Sigma$ with $\mu(E)>0$ and $|g|_{E}|<c-\epsilon$, then
$$
\Vert M_g(\chi_{E})\Vert_{L_p(\Omega,\mu)}
=\Vert g \chi_{E}\Vert_{L_p(\Omega,\mu)}
\leq (c-\epsilon) \Vert \chi_{E}\Vert_{L_p(\Omega,\mu)}
<c\Vert \chi_{E}\Vert_{L_p(\Omega,\mu)}
$$
Contradiction. Therefore $|g|_E|\geq c$ for any $E\in\Sigma$ with $\mu(E)>0$. Thus $|g|\geq c$.

3) Let $p<q$. Assume we have some $c'>0$ and $E\in\Sigma$ such that $\mu(E)>0$, $|g|_E|>c'$. Again we have a sequence  $\{E_n:n\in\mathbb{N}\}\subset\Sigma$ of subsets of $E$ such that $\mu(E_n)=2^{-n}$, because $(\Omega,\Sigma,\mu)$ is a non atomic measure space. Then from inequality $p<q$ we get
$$
\Vert M_g\Vert
\geq\frac{\Vert M_g(\chi_{E_n})\Vert_{L_q(\Omega,\mu)}}{\Vert \chi_{E_n}\Vert_{L_p(\Omega,\mu)}}
\geq\frac{c'\Vert\chi_{E_n}\Vert_{L_q(\Omega,\mu)}}{\Vert \chi_{E_n}\Vert_{L_p(\Omega,\mu)}}
\geq c'\mu(E_n)^{1/q-1/p}
$$
$$
\Vert M_g\Vert
\geq\sup_{n\in\mathbb{N}}c'\mu(E_n)^{1/q-1/p}
\geq c'\sup_{n\in\mathbb{N}}2^{n(1/p-1/q)}
=+\infty
$$
Contradiction, hence $g=0$. In this case by proposition \ref{MultpOpSurjInjDesc} operator $M_g$ is not topologically injective.

$ii)$$\implies$$ i)$ Conversely, assume $|g|\geq c$ for $c>0$ and $p=q$. Then for all $f\in L_p(\Omega,\mu)$ we have
$$
\Vert M_g(f)\Vert_{L_p(\Omega,\mu)}
=\Vert g f\Vert_{L_p(\Omega,\mu)}
\geq c\Vert f\Vert_{L_p(\Omega,\mu)}
$$
So $M_g$ is topologically injective.
\end{proof}

\begin{proposition}\label{TopInjMultOpCharacOnMeasSp} Let $(\Omega,\Sigma,\mu)$ be a $\sigma$-finite measure space, $1\leq p,q\leq +\infty$ and $g\in L_0(\Omega,\Sigma)$. Then the following are equivalent:

$i)$ $M_g\in\mathcal{B}(L_p(\Omega,\mu),L_q(\Omega,\mu))$ is topologically injective;

$ii)$ $M_g$ is a topological isomorphism;

$iii)$ $|g|\geq c$ for some $c>0$, if $p\neq q$ the space $(\Omega,\Sigma,\mu)$ consist of finitely many atoms.
\end{proposition}
\begin{proof} $i)\Longleftrightarrow iii)$ Consider decomposition
$\Omega=\Omega_a^{\mu}\cup\Omega_{na}^{\mu}$, where $(\Omega_{na}^{\mu},\Sigma|_{\Omega_{na}^{\mu}},\mu|_{\Omega_{na}^{\mu}})$ is a non atomic measure space and $(\Omega_a^{\mu},\Sigma|_{\Omega_a^{\mu}},\mu|_{\Omega_a^{\mu}})$ is a purely atomic measure space. By proposition \ref{MultOpDecompDecomp} operator $M_g$ is topologically injective iff so does $M_g^{\Omega_a^{\mu}}$ and $M_g^{\Omega_{na}^{\mu}}$. Propositions \ref{TopInjMultOpCharacOnPureAtomMeasSp}, \ref{TopInjMultOpCharacOnNonAtomMeasSp} give necessary and sufficient conditions for this to happen. 

$i)$$\implies$$ ii)$ Assume $M_g$ is topologically injective. If $p=q$ from considerations above it follows that $|g|\geq c$ for some $c>0$. Since operator $M_g$ is bounded, then from proposition \ref{MultpOpPropIfPeqqualsQ} we also have $c'\geq |g|$ for some $c'>0$. Now from the same proposition we conclude that $M_g$ is a topological isomorphism because $c'\geq|g|\geq c$. Assume $p\neq q$, then from previous paragraph the space $(\Omega,\Sigma,\mu)$ consist of finite amount of atoms and $g$ is non zero. Hence $\operatorname{dim}(L_p(\Omega,\Sigma,\mu))=\operatorname{dim}(\ell_p(\Lambda))=\operatorname{Card}(\Lambda)<+\infty$. Similarly, $\operatorname{dim}(L_q(\Omega,\Sigma,\mu))=\operatorname{Card}(\Lambda)<+\infty$. Since $g$ is non zero, then by proposition \ref{MultpOpSurjInjDesc} operator $M_g$ is injective. Thus $M_g$ is an injective operator between finite dimensional spaces of equal dimension. Hence it is a topological isomorphism.

$ii)$$\implies$$ i)$ Conversely, if $M_g$ is a topological isomorphism, then, clearly, it is topologically injective.
\end{proof}

\begin{proposition}\label{TopInjMultOpCharacBtwnTwoContMeasSp} Let $(\Omega,\Sigma,\mu)$ be a $\sigma$-finite measure space, $1\leq p,q\leq+\infty$. Assume $g,\rho\in L_0(\Omega,\Sigma)$ and $\rho$ is non negative. Then the following are equivalent:

$i)$ $M_g\in\mathcal{B}(L_p(\Omega,\mu),L_q(\Omega,\rho\mu))$ is topologically injective;

$ii)$ $M_g$ is a topological isomorphism;

$iii)$ $\rho$ is  positive, $|g \rho^{1/q}|\geq c$ for some $c>0$, if $p\neq q$ the space $(\Omega,\Sigma,\mu)$ consist of finitely many atoms.
\end{proposition}
\begin{proof} $i)$$\implies$$ iii)$ Consider set $E=\rho^{-1}(\{0\})$. Assume $\mu(E)>0$ then $\chi_E\neq 0$ in $L_p(\Omega,\mu)$. On the other hand $(\rho\mu)(E)=\int_E\rho(\omega)d\mu(\omega)=0$, so $\chi_E=0$ in $L_q(\Omega,\rho\mu)$ and $M_g(\chi_E)=g\chi_E=0$ in $L_q(\Omega,\rho \mu)$. Thus we see that $M_g$ is not injective and as the consequence it is not topologically injective. Contradiction, so $\mu(E)=0$ and $\rho$ is  positive. Hence we have an isometric isomorphism $\bar{I}_q:L_q(\Omega,\mu)\to L_q(\Omega,\rho\mu):f\mapsto \rho^{-1/q} f$. Obviously $M_{g\rho^{1/q}}=\bar{I}_q^{-1} M_g\in\mathcal{B}(L_p(\Omega,\mu),L_q(\Omega,\mu))$. Since $\bar{I}_q$ is an isometric isomorphism and $M_g$ is topologically injective, then $M_{g \rho^{1/q}}$ is topologically injective too. From proposition \ref{TopInjMultOpCharacOnMeasSp} we get that $|g\rho^{1/q}|\geq c$ for some $c>0$ and if $p\neq q$ the space is $(\Omega,\Sigma,\mu)$ consist of finite amount of atoms.

$iii)$$\implies$$ i)$ By proposition \ref{TopInjMultOpCharacOnMeasSp} operator $M_{g \rho^{1/q}}$ is topologically injective. Since $\rho$ is positive, then  
we have an isometric isomorphism $\bar{I}_q$. Then from equality $M_g=\bar{I}_q M_{g \rho^{1/q}}$ it follows that $M_g$ is also topologically injective.

$i)$$\implies$$ ii)$ As we proved above assumption implies that $M_{g \rho^{1/q}}$ is topologically injective and $\bar{I}_q$ is an isometric isomorphism. By proposition \ref{TopInjMultOpCharacOnMeasSp} $M_{g \rho^{1/q}}$ is a topological isomorphism. Since $M_g=\bar{I}_q M_{g \rho^{1/q}}$ and $\bar{I}_q$ is an isometric isomorphism, then $M_g$ is a topological isomorphism.

$ii)$$\implies$$ i)$ If $M_g$ is a topological isomorphism, then, obviously, it is topologically injective.
\end{proof}
\begin{proposition}\label{TopInjMultOpCharacBtwnTwoMeasSp} Let $(\Omega,\Sigma,\mu)$, $(\Omega,\Sigma,\nu)$ be two $\sigma$-finite measure spaces, $1\leq p,q\leq +\infty$ and $g\in L_0(\Omega,\Sigma)$. Then the following are equivalent:

$i)$ $M_g\in\mathcal{B}(L_p(\Omega,\mu), L_q(\Omega,\nu))$ is topologically injective;

$ii)$ $M_g^{\Omega_c^{\nu,\mu}}$ is a topological isomorphism;

$iii)$ $\rho_{\nu,\mu}|_{\Omega_c^{\nu,\mu}}$ is positive, $|g \rho_{\nu,\mu}^{1/q}|_{\Omega_c^{\nu,\mu}}|\geq c$ for some $c>0$, if $p\neq q$ the space $(\Omega,\Sigma,\mu)$ consist of finitely many atoms.
\end{proposition}
\begin{proof}
By proposition \ref{MultOpDecompDecomp} operator $M_g$ is topologically injective iff operators $M_g^{\Omega_c^{\nu,\mu}}:L_p(\Omega_c^{\nu,\mu},\mu|_{\Omega_c^{\nu,\mu}})\to L_q(\Omega_c^{\nu,\mu},\rho_{\nu,\mu} \mu|_{\Omega_c^{\nu,\mu}})$ and $M_g^{\Omega_s^{\nu,\mu}}:L_p(\Omega_s^{\nu,\mu},\mu|_{\Omega_s^{\nu,\mu}})\to L_q(\Omega_s^{\nu,\mu},\nu_s|_{\Omega_s^{\nu,\mu}})$ are topologically injective. By proposition \ref{MultOpCharacBtwnTwoSingMeasSp} operator $M_g^{\Omega_s^{\nu,\mu}}$ is zero. Since $\mu(\Omega_s^{\nu,\mu})=0$, then $L_p(\Omega_s^{\nu,\mu},\mu|_{\Omega_s^{\nu,\mu}})=\{0\}$. From these two facts we conclude that $M_g^{\Omega_s^{\nu,\mu}}$ is topologically injective. Thus topological injectivity of $M_g$ is equivalent to topological injectivity of  $M_g^{\Omega_c^{\nu,\mu}}$. It remains to apply proposition \ref{TopInjMultOpCharacBtwnTwoContMeasSp}.
\end{proof}

\begin{proposition}\label{TopInjMultOpDescBtwnTwoMeasSp} Let $(\Omega,\Sigma,\mu)$, $(\Omega,\Sigma,\nu)$ be two $\sigma$-finite measure spaces, $1\leq p,q\leq +\infty$ and $g\in L_0(\Omega,\Sigma)$. Then the following are equivalent:

$i)$ $M_g\in\mathcal{B}(L_p(\Omega,\mu),L_q(\Omega,\nu))$ is topologically injective;

$ii)$ $M_{\chi_{\Omega_c^{\nu,\mu}}/g}\in\mathcal{B}(L_q(\Omega,\nu), L_p(\Omega,\mu))$ is topologically surjective and a left inverse to $M_g$.
\end{proposition}
\begin{proof}
$i)$$\implies$$ ii)$ By proposition \ref{MultOpDecompDecomp}  $M_g^{\Omega_c^{\nu,\mu}}$ is topologically injective. By proposition \ref{TopInjMultOpCharacBtwnTwoContMeasSp} operator $M_g^{\Omega_c^{\nu,\mu}}$ is invertible and $(M_g^{\Omega_c^{\nu,\mu}})^{-1}=M_{1/g}^{\Omega_c^{\nu,\mu}}$. Then for all $h\in L_q(\Omega,\nu)$ we have
$$
\Vert M_{\chi_{\Omega_c^{\nu,\mu}}/g}(h)\Vert_{L_p(\Omega,\mu)}=
\Vert M_{1/g}(h)\chi_{\Omega_c^{\nu,\mu}}\Vert_{L_p(\Omega,\mu)}=
\Vert M_{1/g}^{\Omega_c^{\nu,\mu}}(h|_{\Omega_c^{\nu,\mu}})\Vert_{L_p(\Omega_c^{\nu,\mu},\mu|_{\Omega_c^{\nu,\mu}})}
$$
$$
\leq\Vert M_{1/g}^{\Omega_c^{\nu,\mu}}\Vert\Vert h|_{\Omega_c^{\nu,\mu}}\Vert_{L_q(\Omega_c^{\nu,\mu},\nu|_{\Omega_c^{\nu,\mu}})}
\leq\Vert M_{1/g}^{\Omega_c^{\nu,\mu}}\Vert\Vert h\Vert_{L_q(\Omega,\nu)}
$$ 
So $M_{\chi_{\Omega_c^{\nu,\mu}}/g}$ is bounded. Now note that for all $f\in L_p(\Omega,\mu)$ we have 
$$
M_{\chi_{\Omega_c^{\nu,\mu}}/g}(M_g(f))
=M_{\chi_{\Omega_c^{\nu,\mu}}/g}(g  f)
=(\chi_{\Omega_c^{\nu,\mu}}/g)  g  f
=f \chi_{\Omega_c^{\nu,\mu}}
$$
Since $\mu(\Omega\setminus\Omega_c^{\nu,\mu})=0$, then $\chi_{\Omega_c^{\nu,\mu}}=\chi_{\Omega}$, so $M_{\chi_{\Omega_c^{\nu,\mu}}/g}(M_g(f))=f \chi_{\Omega_c^{\nu,\mu}}=f \chi_{\Omega}=f$. This means that $M_{\chi_{\Omega_c^{\nu,\mu}}/g}$ is a left inverse multiplication operator to $M_g$. Take any $f\in L_p(\Omega,\mu)$, then for $h=M_g(f)$ we have $M_{\chi_{\Omega_c^{\nu,\mu}}/g}(h)=f$ and $\Vert h\Vert_{L_q(\Omega,\nu)}\leq\Vert M_g\Vert\Vert f\Vert_{L_p(\Omega,\mu)}$. Since $f$ is arbitrary, then $M_{\chi_{\Omega_c^{\nu,\mu}}/g}$ is topologically surjective.

Conversely, if $M_g$ has a left inverse operator $M_{\chi_{\Omega_c^{\nu,\mu}}/g}$ then for all $f\in L_p(\Omega,\mu)$ we have 
$$
\Vert M_g(f)\Vert_{L_q(\Omega,\nu)}
\geq\Vert M_{\chi_{\Omega_c^{\nu,\mu}}/g}\Vert^{-1}\Vert M_{\chi_{\Omega_c^{\nu,\mu}}/g}(M_g(f))\vert_{L_p(\Omega,\mu)}
\geq\Vert M_{\chi_{\Omega_c^{\nu,\mu}}/g}\Vert^{-1}\Vert f\Vert_{L_p(\Omega,\mu)}
$$
So $M_g$ is topologically injective.
\end{proof}


\begin{proposition}\label{IsomMultOpCharacOnMeasSp} Let $(\Omega,\Sigma,\mu)$ be a $\sigma$-finite measure space, $1\leq p,q\leq +\infty$ and $g\in L_0(\Omega,\Sigma)$. Then the following are equivalent:

$i)$ $M_g\in\mathcal{B}(L_p(\Omega,\mu),L_q(\Omega,\mu))$ is an isometry;

$ii)$ $|g|=\mu(\Omega)^{1/p-1/q}$, if $p\neq q$, then $(\Omega,\Sigma,\mu)$ consist of single atom.

\end{proposition}
\begin{proof} $i)$$\implies$$ ii)$ Let $p=q$. Assume there exist $E\in\Sigma$ with $\mu(E)>0$ such that $|g|_E|<1$, then
$$
\Vert M_g(\chi_E)\Vert_{L_p(\Omega,\mu)}
=\Vert g \chi_E\Vert_{L_p(\Omega,\mu)}
<\Vert\chi_E\Vert_{L_p(\Omega,\mu)}
=\Vert M_g(\chi_E)\Vert_{L_p(\Omega,\mu)}
$$
Contradiction, hence for all $E\in\Sigma$ with $\mu(E)>0$ we have $|g|_E|\geq 1$ i.e.  $|g|\geq 1$. Assume there exist $E\in\Sigma$ with $\mu(E)>0$ such that $|g|_E|>1$, then
$$
\Vert M_g(\chi_E)\Vert_{L_p(\Omega,\mu)}
=\Vert g \chi_E\Vert_{L_p(\Omega,\mu)}
>\Vert\chi_E\Vert_{L_p(\Omega,\mu)}
=\Vert M_g(\chi_E)\Vert_{L_p(\Omega,\mu)}
$$
Contradiction, hence for all $E\in\Sigma$ with $\mu(E)>0$ we have $|g|_E|\leq 1$ i.e.  $|g|\leq 1$. From both inequalities we get $|g|=1=\mu(\Omega)^{1/p-1/q}$. Let $p\neq q$, then since $M_g$ is an isometry it is topologically injective. By proposition \ref{TopInjMultOpCharacOnMeasSp} the space $(\Omega,\Sigma,\mu)$ consist of finitely many atoms. Assume there are at least two disjoint atoms, say $\Omega_1$ and $\Omega_2$. They are of finite measure, so we can consider respective normalized functions $h_k=\Vert\chi_{\Omega_k}\Vert_{L_p(\Omega,\mu)}^{-1}\chi_{\Omega_k}$ where $k\in\mathbb{N}_2$. Since these atoms are disjoint, then $h_1h_2=0$ and as the result $M_g(h_1)M_g(h_2)=0$. Note that for any $1\leq r\leq +\infty$ and all $f_1,f_2\in L_r(\Omega,\mu)$ such that $f_1f_2=0$ we have 
$$
\Vert f_1+f_2\Vert_{L_r(\Omega,\mu)}
=\left\Vert\left(\Vert f_\lambda\Vert_{L_r(\Omega,\mu)}:\lambda\in\mathbb{N}_2\right)\right\Vert_{\ell_r(\mathbb{N}_2)}
$$
Hence
$$
\Vert M_g(h_1+h_2)\Vert_{L_q(\Omega,\mu)}
=\Vert h_1+h_2\Vert_{L_p(\Omega,\mu)}
=\left\Vert\left( 1 :\lambda\in\mathbb{N}_2\right)\right\Vert_{\ell_p(\mathbb{N}_2)}
=2^{1/p}
$$
But on the other hand
$$
\Vert M_g(h_1+h_2)\Vert_{L_q(\Omega,\mu)}
=\Vert M_g(h_1)+M_g(h_2)\Vert_{L_q(\Omega,\mu)}
$$
$$
=\left\Vert\left(\Vert M_g(h_\lambda)\Vert_{L_q(\Omega,\mu)}:\lambda\in\mathbb{N}_2\right)\right\Vert_{\ell_q(\mathbb{N}_2)}
=\left\Vert\left(\Vert h_\lambda\Vert_{L_p(\Omega,\mu)}:\lambda\in\mathbb{N}_2\right)\right\Vert_{\ell_q(\mathbb{N}_2)}
=2^{1/q}
$$
Thus $2^{1/p}=2^{1/q}$. Contradiction, so $(\Omega,\Sigma,\mu)$ consist of single atom. In this case for all $f\in L_p(\Omega,\mu)$ we have
$$
\Vert M_g(f)\Vert_{L_q(\Omega,\mu)}
=\Vert J_q(M_g(f))\Vert_{\ell_q(\mathbb{N}_1)}
=\Vert J_q(g  f)\Vert_{\ell_q(\mathbb{N}_1)}
=\mu(\Omega)^{1/q-1}\left|\int_\Omega g(\omega) f(\omega)d\mu(\omega)\right|
$$
$$
\Vert f\Vert_{L_p(\Omega,\mu)}
=\Vert J_p(f)\Vert_{\ell_p(\mathbb{N}_1)}
=\mu(\Omega)^{1/p-1}\left|\int_\Omega f(\omega)d\mu(\omega)\right|
$$
By $c$ we denote the constant value of $g$, then
$$
\Vert M_g(f)\Vert_{L_q(\Omega,\mu)}
=\mu(\Omega)^{1/q-1}\left|\int_\Omega g(\omega) f(\omega)d\mu(\omega)\right|
=\mu(\Omega)^{1/q-1}|c|\left|\int_\Omega f(\omega)d\mu(\omega)\right|
$$
From this equality we conclude that in this case $M_g$ is an isometry if
$$
|g|=|c|=\mu(\Omega)^{1/p-1/q}
$$ 

$ii)$$\implies$$ i)$. Let $p=q$, then $|g|=1$. So for all $f\in L_p(\Omega,\mu)$ we have
$$
\Vert M_g(f)\Vert_{L_p(\Omega,\mu)}
=\Vert g  f\Vert_{L_p(\Omega,\mu)}
=\Vert |g|  f\Vert_{L_p(\Omega,\mu)}
=\Vert f\Vert_{L_p(\Omega,\mu)}
$$
hence $M_g$ is an isometry. Let $p\neq q$, then $(\Omega,\Sigma,\mu)$ consist of single atom and we conclude
$$
\Vert M_g(f)\Vert_{L_q(\Omega,\mu)}
=\mu(\Omega)^{1/q-1}\left|\int_\Omega g(\omega) f(\omega)d\mu(\omega)\right|
=\mu(\Omega)^{1/q-1}|c|\left|\int_\Omega f(\omega)d\mu(\omega)\right|
$$
$$
=\mu(\Omega)^{1/p-1}\left|\int_\Omega f(\omega)d\mu(\omega)\right|
=\Vert f\Vert_{L_p(\Omega,\mu)}
$$
Hence $M_g$ is isometric.
\end{proof}

\begin{proposition}\label{IsomMultOpCharacBtwnTwoContMeasSp} Let $(\Omega,\Sigma,\mu)$ be a $\sigma$-finite measure space and $1\leq p,q\leq +\infty$. Assume $g,\rho\in L_0(\Omega,\Sigma)$ and $\rho$ is non negative. Then the following are equivalent:

$i)$ $M_g\in\mathcal{B}(L_p(\Omega,\mu), L_q(\Omega,\rho \mu))$ is isometric;

$ii)$ $M_g$ is an isometric isomorphism;

$iii)$ $\rho$ is positive, $|g  \rho^{1/q}|=\mu(\Omega)^{1/p-1/q}$ and if $p\neq q$ the space $(\Omega,\Sigma,\mu)$ consist of single atom.
\end{proposition}
\begin{proof} $i)$$\implies$$ iii)$ Since $M_g$ is isometric, then it is topologically injective and by proposition \ref{TopInjMultOpCharacBtwnTwoMeasSp} we see that $\rho$ is positive. Hence we have an isometric isomorphism $\bar{I}_q:L_q(\Omega,\mu)\to L_q(\Omega,\rho \mu):f\mapsto \rho^{-1/q}  f$. Obviously $M_{g \rho^{1/q}}=\bar{I}_q^{-1} M_g\in\mathcal{B}(L_p(\Omega,\mu),L_q(\Omega,\mu))$. Since $\bar{I}_q$ is an isometric isomorphism and $M_g$ is isometric, then $M_{g  \rho^{1/q}}$ is isometric too. It remains to apply proposition \ref{IsomMultOpCharacOnMeasSp}.

$iii)$$\implies$$ i)$ By proposition \ref{IsomMultOpCharacOnMeasSp} operator $M_{g \rho^{1/q}}$ is isometric. Since $\rho$ is positive, then we have an isometric isomorphism $\bar{I}_q$. Then from equality $M_g=\bar{I}_q M_{g \rho^{1/q}}$ it follows that $M_g$ is also isometric.

$i)$$\implies$$ ii)$ Since $M_g$ is isometric, then it is topologically injective and by proposition \ref{TopInjMultOpCharacBtwnTwoContMeasSp} it is a topological isomorphism, which is isometric by assumption.

$ii)$$\implies$$ i)$ Since $M_g$ is an isometric isomorphism, trivially, it is isometric.
\end{proof}

\begin{proposition}\label{IsomMultOpCharacBtwnTwoMeasSp} Let $(\Omega,\Sigma,\mu)$, $(\Omega,\Sigma,\nu)$ be two $\sigma$-finite measure spaces, $1\leq p,q\leq +\infty$ and $g\in L_0(\Omega,\Sigma)$. Then the following are equivalent:

$i)$ $M_g$ is isometric;

$ii)$ $M_g^{\Omega_c^{\nu,\mu}}$ is isometric;

$iii)$ $\rho_{\nu,\mu}|_{\Omega_c^{\nu,\mu}}$ is positive, $|g  \rho_{\nu,\mu}^{1/q}|_{\Omega_c^{\nu,\mu}}|=\mu(\Omega_c^{\nu,\mu})^{1/p-1/q}$ and if  $p\neq q$ the space $(\Omega,\Sigma,\mu)$ consist of single atom.
\end{proposition}
\begin{proof}
$i)$$\implies$$ ii)$$\implies$$ iii)$ Since $M_g$ is isometric, then by proposition \ref{MultOpDecompDecomp} operator $M_g^{\Omega_c^{\nu,\mu}}$, is isometric. It remains to apply proposition \ref{IsomMultOpCharacBtwnTwoContMeasSp}.

$iii)$$\implies$$ i)$ By proposition \ref{IsomMultOpCharacBtwnTwoContMeasSp} operator $M_g^{\Omega_c^{\nu,\mu}}$ is isometric. Now take arbitrary $f\in L_p(\Omega,\mu)$. Since $\mu(\Omega\setminus\Omega_c^{\nu,\mu})=0$, then $\chi_{\Omega_c^{\nu,\mu}}=\chi_{\Omega}$ in $L_p(\Omega,\mu)$. As the result $f=f\chi_{\Omega}=f\chi_{\Omega_c^{\nu,\mu}}=f\chi_{\Omega_c^{\nu,\mu}}\chi_{\Omega_c^{\nu,\mu}}$ in $L_p(\Omega,\mu)$ and $M_g(f)=M_g(f\chi_{\Omega_c^{\nu,\mu}})\chi_{\Omega_c^{\nu,\mu}}$. Thus using that $M_g^{\Omega_c^{\nu,\mu}}$ is isometric we get
$$
\Vert M_g(f)\Vert_{L_q(\Omega,\nu)}
=\Vert M_g(f\chi_{\Omega_c^{\nu,\mu}})\chi_{\Omega_c^{\nu,\mu}}\Vert_{L_q(\Omega,\nu)}
=\Vert M_g(f\chi_{\Omega_c^{\nu,\mu}})\Vert_{L_q(\Omega_c^{\nu,\mu},\nu|_{\Omega_c^{\nu,\mu}})}
$$
$$
=\Vert M_g^{\Omega_c^{\nu,\mu}}(f|_{\Omega_c^{\nu,\mu}})\Vert_{L_q(\Omega_c^{\nu,\mu},\nu|_{\Omega_c^{\nu,\mu}})}
=\Vert f|_{\Omega_c^{\nu,\mu}}\Vert_{L_p(\Omega_c^{\nu,\mu},\mu|_{\Omega_c^{\nu,\mu}})}
$$
Since $\mu(\Omega\setminus\Omega_c^{\nu,\mu})=0$ we have $\Vert f|_{\Omega_c^{\nu,\mu}}\Vert_{L_p(\Omega_c^{\nu,\mu},\mu|_{\Omega_c^{\nu,\mu}})}=\Vert f\Vert_{L_p(\Omega,\mu)}$ so $\Vert M_g(f)\Vert_{L_q(\Omega,\nu)}=\Vert f\Vert_{L_p(\Omega,\mu)}$. Therefore $M_g$ is isometric.
\end{proof}

\begin{proposition}\label{IsomMultOpDescBtwnTwoMeasSp} Let $(\Omega,\Sigma,\mu)$, $(\Omega,\Sigma,\nu)$ be two $\sigma$-finite measure spaces, $1\leq p,q\leq +\infty$ and $g\in L_0(\Omega,\Sigma)$. Then the following are equivalent:

$i)$ $M_g\in\mathcal{B}(L_p(\Omega,\mu),L_q(\Omega,\nu))$ is isometric;

$ii)$ $M_{\chi_{\Omega_c^{\nu,\mu}}/g}\in\mathcal{B}(L_q(\Omega,\nu), L_p(\Omega,\mu))$ is strictly coisometric and a left inverse to $M_g$.
\end{proposition}
\begin{proof}
$i)$$\implies$$ ii)$ By proposition \ref{MultOpDecompDecomp} operator $M_g^{\Omega_c^{\nu,\mu}}$ is isometric and by proposition \ref{IsomMultOpCharacBtwnTwoContMeasSp} it is invertible with $(M_g^{\Omega_c^{\nu,\mu}})^{-1}=M_{1/g}^{\Omega_c^{\nu,\mu}}$. Since $M_g^{\Omega_c^{\nu,\mu}}$ is isometric then so does its inverse. Then for all $h\in L_q(\Omega,\nu)$ we have
$$
\Vert M_{\chi_{\Omega_c^{\nu,\mu}}/g}(h)\Vert_{L_p(\Omega,\mu)}=
\Vert M_{1/g}(h|_{\Omega_c^{\nu,\mu}})\Vert_{L_p(\Omega_c^{\nu,\mu},\mu|_{\Omega_c^{\nu,\mu}})}=
\Vert M_{1/g}^{\Omega_c^{\nu,\mu}}(h|_{\Omega_c^{\nu,\mu}})\Vert_{L_p(\Omega_c^{\nu,\mu},\mu|_{\Omega_c^{\nu,\mu}})}
$$
$$
=\Vert h|_{\Omega_c^{\nu,\mu}}\Vert_{L_q(\Omega_c^{\nu,\mu},\nu|_{\Omega_c^{\nu,\mu}})}
\leq \Vert h \Vert_{L_q(\Omega,\nu)}
$$
So $M_{\chi_{\Omega_c^{\nu,\mu}}/g}$ is contractive. Now note that for all $f\in L_p(\Omega,\mu)$ we have 
$$
M_{\chi_{\Omega_c^{\nu,\mu}}/g}(M_g(f))
=M_{\chi_{\Omega_c^{\nu,\mu}}/g}(g  f)
=(\chi_{\Omega_c^{\nu,\mu}}/g)  g  f
=f \chi_{\Omega_c^{\nu,\mu}}
$$
Since $\mu(\Omega\setminus\Omega_c^{\nu,\mu})=0$, then $\chi_{\Omega_c^{\nu,\mu}}=\chi_{\Omega}$, so $M_{\chi_{\Omega_c^{\nu,\mu}}/g}(M_g(f))=f \chi_{\Omega_c^{\nu,\mu}}=f \chi_{\Omega}=f$. This means that $M_{\chi_{\Omega_c^{\nu,\mu}}/g}$ is a left inverse multiplication operator to $M_g$. Take any $f\in L_p(\Omega,\mu)$, then for $h=M_g(f)$ we have $M_{\chi_{\Omega_c^{\nu,\mu}}/g}(h)=f$ and $\Vert h\Vert_{L_q(\Omega,\nu)}\leq\Vert f\Vert_{L_p(\Omega,\mu)}$ i.e.  $M_{\chi_{\Omega_c^{\nu,\mu}}}/g$ is strictly $1$-topologically surjective. Since $M_{\chi_{\Omega_c^{\nu,\mu}}/g}$ is also contractive, then it is strictly coisometric.

$ii)$$\implies$$ i)$ Take any $f\in L_p(\Omega,\mu)$, then there exist $h\in L_q(\Omega,\nu)$ such that $M_{\chi_{\Omega_c^{\nu,\mu}}/g}(h)=f$ and $\Vert h\Vert_{L_q(\Omega,\nu)}\leq \Vert f\Vert_{L_p(\Omega,\mu)}$. Hence
$$
\Vert M_g(f)\Vert_{L_q(\Omega,\nu)}
=\Vert M_g(M_{\chi_{\Omega_c^{\nu,\mu}}/g}(h))\Vert_{L_q(\Omega,\nu)}
=\Vert \chi_{\Omega_c^{\nu,\mu}}h\Vert_{L_q(\Omega,\nu)}
\leq\Vert h\Vert_{L_q(\Omega,\nu|)}
\leq\Vert f\Vert_{L_p(\Omega,\mu)}
$$
Since $M_{\chi_{\Omega_c^{\nu,\mu}}/g}$ is contractive and left inverse to $M_g$ then
$$
\Vert f\Vert_{L_p(\Omega,\mu)}
=\Vert M_{\chi_{\Omega_c^{\nu,\mu}}/g}(M_g(f))\Vert_{L_p(\Omega,\mu)}
\leq\Vert M_g(f)\Vert_{L_q(\Omega,\nu)}
$$
so $\Vert M_g(f)\Vert_{L_q(\Omega,\nu)}=\Vert f\Vert_{L_p(\Omega,\mu)}$. Since $f$ is arbitrary $M_g$ is isometric.
\end{proof}

Now we shall discuss topologically surjective and coisometric multiplication operators. Their description is easier to achieve and most of the proofs are similar (but not identical) to the case of topologically injective and isometric operators.

\begin{proposition}\label{TopSurMultOpCharacOnMeasSp} Let $(\Omega,\Sigma,\mu)$ be a $\sigma$-finite measure space, $1\leq p,q\leq +\infty$ and $g\in L_0(\Omega,\Sigma)$. Then the following are equivalent:

$i)$ $M_g\in\mathcal{B}(L_p(\Omega,\mu),L_q(\Omega,\mu))$ is topologically surjective;

$ii)$ $M_g$ is a topological isomorphism;

$iii)$ $|g|\geq c$ for some $c>0$, if $p\neq q$ the space $(\Omega,\Sigma,\mu)$ consist of finitely many atoms.
\end{proposition}
\begin{proof} $i)$$\implies$$ ii)$ Since $M_g$ is topologically surjective, then it is surjective and by proposition \ref{MultpOpSurjInjDesc} it is also injective. Thus $M_g$ is bijective. Since $L_p$ spaces are complete, from open mapping theorem we see that $M_g$ is a topological isomorphism. 

$ii)$$\implies$$ i)$ If $M_g$ is a topological isomorphism, obviously, it is topologically surjective.

$ii)\Longleftrightarrow iii)$ Follows from proposition \ref{TopInjMultOpCharacOnMeasSp}
\end{proof}
 
\begin{proposition}\label{TopSurMultOpCharacBtwnTwoContMeasSp} Let $(\Omega,\Sigma,\nu)$ be a $\sigma$-finite measure space, $1\leq p,q\leq +\infty$ and $g,\rho\in L_0(\Omega,\Sigma)$ and $\rho$ is non negative, then the following are equivalent:

$i)$ $M_g\in\mathcal{B}(L_p(\Omega,\rho \nu),L_q(\Omega,\nu))$ is topologically surjective;

$ii)$ $M_g$ is a topological isomorphism;

$iii)$ $\rho$ is positive, $|g  \rho^{-1/p}|\geq c$ for some $c>0$, if $p\neq q$ the space $(\Omega,\Sigma,\mu)$ consist of finitely many atoms.
\end{proposition}
\begin{proof} $i)$$\implies$$ iii)$ Consider set $E=\rho^{-1}(\{0\})$. Assume $\nu(E)>0$ then $\chi_E\neq 0$ in $L_q(\Omega,\nu)$. On the other hand $(\rho \nu)(E)=\int_E\rho(\omega)d\nu(\omega)=0$, so $\chi_E=0$ in $L_p(\Omega,\rho \nu)$. Then for all $f\in L_p(\Omega,\rho \nu)$ holds $M_g(f)\chi_E=M_g(f \chi_E)=M_g(0)=0$ in $L_q(\Omega,\nu)$. The last equality means that $\operatorname{Im}(M_g)\subset\{h\in L_q(\Omega,\nu): h|_E=0\}$. Since $\nu(E)\neq 0$ we see that $M_g$ is not surjective and as the consequence it is not topologically surjective. Contradiction, so $\nu(E)=0$ and $\rho$ is positive. Hence we have an isometric isomorphism $\bar{I}_p:L_p(\Omega,\nu)\to L_p(\Omega,\rho \nu):f\mapsto \rho^{-1/p}  f$. Obviously $M_{g \rho^{-1/p}}=M_g \bar{I}_p\in\mathcal{B}(L_p(\Omega,\nu),L_q(\Omega,\nu))$. Since $\bar{I}_p$ is an isometric isomorphism and $M_g$ is topologically surjective, then $M_{g  \rho^{-1/p}}$ is topologically surjective too. It remains to apply proposition \ref{TopSurMultOpCharacOnMeasSp}.

$iii)$$\implies$$ i)$ By proposition \ref{TopSurMultOpCharacOnMeasSp} operator $M_{g \rho^{-1/p}}$ is topologically surjective. Since $\rho$ is positive, then we have an isometric isomorphism $\bar{I}_p$. Then from equality $M_g= M_{g \rho^{-1/p}}\bar{I}_p^{-1}$ it follows that $M_g$ is also topologically surjective.

$i)$$\implies$$ ii)$ As we proved above the operator $M_{g \rho^{1/q}}$ is topologically injective and $\bar{I}_q$ is an isometric isomorphism. By proposition \ref{TopSurMultOpCharacOnMeasSp} $M_{g \rho^{1/q}}$ is a topological isomorphism. Since $M_g=\bar{I}_q M_{g \rho^{1/q}}$ we see that $M_g$ is also a topological isomorphism, as composition of such.

$ii)$$\implies$$ i)$. If $M_g$ is a topological isomorphism, obviously, it is topologically surjective.
\end{proof}

\begin{proposition}\label{TopSurMultOpCharacBtwnTwoMeasSp} Let $(\Omega,\Sigma,\mu)$, $(\Omega,\Sigma,\nu)$ be two $\sigma$-finite measure spaces, $1\leq p,q\leq +\infty$ and $g\in L_0(\Omega,\Sigma)$. Then the following are equivalent:

$i)$ $M_g\in\mathcal{B}(L_p(\Omega,\mu), L_q(\Omega,\nu))$ is topologically surjective;

$ii)$ $M_g^{\Omega_c^{\mu,\nu}}$ is topologically surjective;

$iii)$ $\rho_{\mu,\nu}|_{\Omega_c^{\mu,\nu}}$ is positive, $|g \rho_{\mu,\nu}^{-1/p}|_{\Omega_c^{\mu,\nu}}|\geq c$ for some $c>0$, if $p\neq q$ the space $(\Omega,\Sigma,\mu)$ consist of finitely many atoms.
\end{proposition}
\begin{proof}
By proposition \ref{MultOpDecompDecomp} operator $M_g$ is topologically surjective iff operators $M_g^{\Omega_c^{\mu,\nu}}:L_p(\Omega_c^{\mu,\nu},\rho_{\mu,\nu} \nu|_{\Omega_c^{\mu,\nu}})\to L_q(\Omega_c^{\mu,\nu},\nu|_{\Omega_c^{\mu,\nu}})$ and $M_g^{\Omega_s^{\mu,\nu}}:L_p(\Omega_s^{\mu,\nu},\mu_s|_{\Omega_s^{\mu,\nu}})\to L_q(\Omega_s^{\mu,\nu},\nu|_{\Omega_s^{\mu,\nu}})$ are topologically surjective. By proposition \ref{MultOpCharacBtwnTwoSingMeasSp} operator $M_g^{\Omega_s^{\mu,\nu}}$ is zero. Since $\nu(\Omega_s^{\mu,\nu})=0$, then $L_p(\Omega_s^{\mu,\nu},\nu|_{\Omega_s^{\mu,\nu}})=\{0\}$. From these two facts we conclude that $M_g^{\Omega_s^{\mu,\nu}}$ is topologically surjective. Thus topological surjectivity of $M_g$ is equivalent to topological surjectivity of  $M_g^{\Omega_c^{\mu,\nu}}$. It remains to apply proposition \ref{TopSurMultOpCharacBtwnTwoContMeasSp}.
\end{proof}

\begin{proposition}\label{TopSurMultOpDescBtwnTwoMeasSp} Let $(\Omega,\Sigma,\mu)$, $(\Omega,\Sigma,\nu)$ be two $\sigma$-finite measure spaces, $1\leq p,q\leq +\infty$ and $g\in L_0(\Omega,\Sigma)$. Then the following are equivalent:

$i)$ $M_g\in\mathcal{B}(L_p(\Omega,\mu),L_q(\Omega,\nu))$ is topologically surjective;

$ii)$ $M_{\chi_{\Omega_c^{\mu,\nu}}/g}\in\mathcal{B}(L_q(\Omega,\nu), L_p(\Omega,\mu))$ is topologically injective operator and a right inverse to $M_g$.
\end{proposition}
\begin{proof}
$i)$$\implies$$ ii)$ By proposition \ref{MultOpDecompDecomp} operator $M_g^{\Omega_c^{\mu,\nu}}$ is topologically surjective. By proposition \ref{TopSurMultOpCharacBtwnTwoContMeasSp} it is invertible and $(M_g^{\Omega_c^{\mu,\nu}})^{-1}=M_{1/g}^{\Omega_c^{\mu,\nu}}$. Then for all $h\in L_q(\Omega,\nu)$ we have
$$
\Vert M_{\chi_{\Omega_c^{\mu,\nu}}/g}(h)\Vert_{L_p(\Omega,\mu)}=
\Vert M_{1/g}(h|_{\Omega_c^{\mu,\nu}})\Vert_{L_p(\Omega_c^{\mu,\nu},\mu|_{\Omega_c^{\mu,\nu}})}=
\Vert M_{1/g}^{\Omega_c^{\mu,\nu}}(h|_{\Omega_c^{\mu,\nu}})\Vert_{L_p(\Omega_c^{\mu,\nu},\mu|_{\Omega_c^{\mu,\nu}})}
$$
$$
\leq\Vert M_{1/g}^{\Omega_c^{\mu,\nu}}\Vert\Vert h|_{\Omega_c^{\mu,\nu}}\Vert_{L_q(\Omega_c^{\mu,\nu},\nu|_{\Omega_c^{\mu,\nu}})}
\leq\Vert M_{1/g}^{\Omega_c^{\mu,\nu}}\Vert\Vert h\Vert_{L_q(\Omega,\nu)}
$$ 
So $M_{\chi_{\Omega_c^{\mu,\nu}}/g}$ is bounded. Now note that for all $h\in L_q(\Omega,\nu)$ we have 
$$
M_g(M_{\chi_{\Omega_c^{\mu,\nu}}/g}(h))
=M_g(\chi_{\Omega_c^{\mu,\nu}}/g  h)
=g (\chi_{\Omega_c^{\mu,\nu}}/g)   h
=h \chi_{\Omega_c^{\mu,\nu}}
$$
Since $\nu(\Omega\setminus\Omega_c^{\mu,\nu})=0$, then $\chi_{\Omega_c^{\mu,\nu}}=\chi_{\Omega}$, so $M_g(M_{\chi_{\Omega_c^{\mu,\nu}}/g}(h))=h \chi_{\Omega_c^{\mu,\nu}}=h \chi_{\Omega}=h$. This means that $M_{\chi_{\Omega_c^{\mu,\nu}}/g}$ is a right inverse multiplication operator to $M_g$. Take any $h\in L_q(\Omega,\nu)$, then
$$
\Vert M_{\chi_{\Omega_c^{\mu,\nu}}/g}(h)\Vert_{L_p(\Omega,\mu)}
\geq\Vert M_g\Vert\Vert M_g(M_{\chi_{\Omega_c^{\mu,\nu}}/g}(h))\Vert_{L_q(\Omega,\nu)}
\geq\Vert M_g\Vert\Vert h\Vert_{L_q(\Omega,\nu)}
$$
Since $h$ is arbitrary we get that $M_{\chi_{\Omega_c^{\mu,\nu}}/g}$ is topologically injective.

$ii)$$\implies$$ i)$ Take arbitrary $h\in L_q(\Omega,\nu)$ and consider $f=M_{\chi_{\Omega_c^{\mu,\nu}}/g}(h)$. Then $M_g(f)=M_g(M_{\chi_{\Omega_c^{\mu,\nu}}/g}(h))=h$ and $\Vert f\Vert_{L_p(\Omega,\mu)}\leq\Vert M_{\chi_{\Omega_c^{\mu,\nu}}/g}\Vert\Vert h\Vert_{L_q(\Omega,\nu)}$. Since $h$ is arbitrary we get that $M_g$ is topologically surjective.
\end{proof}


\begin{proposition}\label{CoisomMultOpCharacOnMeasSp} Let $(\Omega,\Sigma,\mu)$ be a $\sigma$-finite measure space, $1\leq p,q\leq +\infty$ and $g\in L_0(\Omega,\Sigma)$. Then the following are equivalent:

$i)$ $M_g\in\mathcal{B}(L_p(\Omega,\mu),L_q(\Omega,\mu))$ is coisometric;

$ii)$ $M_g$ is an isometric isomorphism;

$iii)$ $|g|=\mu(\Omega)^{1/q-1/p}$, if $p\neq q$ the space $(\Omega,\Sigma,\mu)$ consist of single atom.
\end{proposition}
\begin{proof} Since $M_g$ is coisometric it is topologically injective, so from proposition \ref{TopSurMultOpCharacOnMeasSp} we get that $M_g$ is in fact a topological isomorphism. As the consequence it is injective, but injective coisometric operator is an isometric isomorphisms. It remains to note that every isometric isomorphism is a strict coisometry. Thus we conclude that $M_g$ is coisometric iff it is strictly coisometric iff it is an isometric isomorphism. Now we apply proposition \ref{IsomMultOpCharacOnMeasSp}.
\end{proof}

\begin{proposition}\label{CoisomMultOpCharacBtwnTwoContMeasSp} Let $(\Omega,\Sigma,\nu)$ be a $\sigma$-finite measure space, $1\leq p,q\leq +\infty$. Assume $g,\rho\in L_0(\Omega,\Sigma)$ and $\rho$ is non negative. Then the following are equivalent:

$i)$ $M_g\in\mathcal{B}(L_p(\Omega,\rho \nu),L_q(\Omega,\nu))$ is coisometric;

$ii)$ $M_g$ is an isometric isomorphism;

$iii)$ $\rho$ is positive, $|g  \rho^{-1/p}|=\mu(\Omega)^{1/p-1/q}$, if $p\neq q$ the space $(\Omega,\Sigma,\mu)$ consist single atom.
\end{proposition}
\begin{proof} $i)$$\implies$$ ii)$ Assume $M_g$ is coisometric, then it is topologically surjective. By proposition \ref{TopSurMultOpCharacBtwnTwoContMeasSp} operator $M_g$ is a topological isomorphism, hence bijective. It remains to note that bijective coisometry is an isometric isomorphism.

$ii)$$\implies$$ i)$ If $M_g$ is an isometric isomorphism, of course, it is coisometry and even more a strict coisometry.

$ii)\Longleftrightarrow iii)$ Follows from proposition \ref{IsomMultOpCharacBtwnTwoContMeasSp}.
\end{proof}

\begin{proposition}\label{CoisomMultOpCharacBtwnTwoMeasSp} Let $(\Omega,\Sigma,\mu)$, $(\Omega,\Sigma,\nu)$ be two $\sigma$-finite measure spaces, $1\leq p,q\leq +\infty$ and $g\in L_0(\Omega,\Sigma)$. Then the following are equivalent: 

$i)$ $M_g\in\mathcal{B}(L_p(\Omega,\mu), L_q(\Omega,\nu))$ is coisometric;

$ii)$ $M_g^{\Omega_c^{\mu,\nu}}$ is an isometric isomorphism;

$iii)$ $\rho_{\mu,\nu}|_{\Omega_c^{\mu,\nu}}$ is positive, $|g \rho_{\mu,\nu}^{-1/p}|_{\Omega_c^{\mu,\nu}}|=\mu(\Omega_c^{\mu,\nu})^{1/p-1/q}$, if $p\neq q$ the space $(\Omega,\Sigma,\mu)$ consist of single atom.
\end{proposition}
\begin{proof} $i)$$\implies$$ ii)$ Since $M_g$ is coisometric, then from proposition \ref{MultOpDecompDecomp} we know that $M_g^{\Omega_c^{\mu,\nu}}$ is also coisometric. From proposition \ref{CoisomMultOpCharacBtwnTwoContMeasSp} we get that $M_g^{\Omega_c^{\mu,\nu}}$ is an isometric isomorphism. 

$ii)$$\implies$$ i)$ Take arbitrary $h\in L_q(\Omega,\nu)$, then there exists $f\in L_p(\Omega_c^{\mu,\nu},\mu|_{\Omega_c^{\mu,\nu}})$ such that $M_g^{\Omega_c^{\mu,\nu}}(f)=h|_{\Omega_c^{\mu,\nu}}$. By proposition \ref{MultOpCharacBtwnTwoSingMeasSp} operator $M_g^{\Omega_s^{\mu,\nu}}=0$, so
$$
M_g(\widetilde{f})
=\widetilde{M_g^{\Omega_c^{\mu,\nu}}(\widetilde{f}|_{\Omega_c^{\mu,\nu}})}+\widetilde{M_g^{\Omega_s^{\mu,\nu}}(\widetilde{f}|_{\Omega_s^{\mu,\nu}})}
=\widetilde{h|_{\Omega_c^{\mu,\nu}}}
$$
Since $\nu(\Omega_s^{\mu,\nu})=0$, then $\Vert h-\widetilde{h|_{\Omega_c^{\mu,\nu}}}\Vert_{L_q(\Omega,\nu)}=\Vert h\chi_{\Omega_s^{\mu,\nu}}\Vert_{L_q(\Omega,\nu)}=0$ and we conclude $h=\widetilde{h|_{\Omega_c^{\mu,\nu}}}$. So we found $\widetilde{f}\in L_p(\Omega,\mu)$ such that $M_g(\widetilde{f})=h$ and $\Vert \widetilde{f}\Vert_{L_p(\Omega,\mu)}=\Vert f\Vert_{L_p(\Omega_c^{\mu,\nu},\mu|_{\Omega_c^{\mu,\nu}})}=\Vert h|_{\Omega_c^{\mu,\nu}}\Vert_{L_q(\Omega_c^{\mu,\nu},\nu|_{\Omega_c^{\mu,\nu}})}\leq\Vert h\Vert_{L_q(\Omega,\nu)}$. Since $h$ is arbitrary, then $M_g$ is $1$-topologically surjective. For all $f\in L_p(\Omega,\mu)$ we have
$$
\Vert M_g(f)\Vert_{L_q(\Omega,\nu)}
=\Vert\widetilde{M_g^{\Omega_c^{\mu,\nu}}(f|_{\Omega_c^{\mu,\nu}})}+\widetilde{M_g^{\Omega_s^{\mu,\nu}}(f|_{\Omega_s^{\mu,\nu}})}\Vert_{L_q(\Omega,\nu)}
=\Vert\widetilde{M_g^{\Omega_c^{\mu,\nu}}(f|_{\Omega_c^{\mu,\nu}})}\Vert_{L_q(\Omega,\nu)}
$$
$$
=\Vert M_g^{\Omega_c^{\mu,\nu}}(f|_{\Omega_c^{\mu,\nu}})\Vert_{L_q(\Omega_c^{\mu,\nu},\nu|_{\Omega_c^{\mu,\nu}})}
=\Vert f|_{\Omega_c^{\mu,\nu}}\Vert_{L_p(\Omega_c^{\mu,\nu},\mu|_{\Omega_c^{\mu,\nu}})}
\leq\Vert f \Vert_{L_p(\Omega,\mu)}
$$
Since $f$ is arbitrary, then $M_g$ is contractive, but it is also $1$-topologically surjective. Thus $M_g$ is coisometric.

$ii)\Longleftrightarrow iii)$ Follows from proposition \ref{CoisomMultOpCharacBtwnTwoContMeasSp}.
\end{proof}

\begin{proposition}\label{CoisomMultOpDescBtwnTwoMeasSp} Let $(\Omega,\Sigma,\mu)$, $(\Omega,\Sigma,\nu)$ be two $\sigma$-finite measure spaces, $1\leq p,q\leq +\infty$ and $g\in L_0(\Omega,\Sigma)$. Then the following are equivalent: 

$i)$ $M_g\in\mathcal{B}(L_p(\Omega,\mu),L_q(\Omega,\nu))$ is coisometric;

$ii)$ $M_{\chi_{\Omega_c^{\mu,\nu}}/g}\in\mathcal{B}(L_q(\Omega,\nu), L_p(\Omega,\mu))$ is isometric and a right inverse to $M_g$.
\end{proposition}
\begin{proof}
$i)$$\implies$$ ii)$ By proposition \ref{MultOpDecompDecomp} operator $M_g^{\Omega_c^{\mu,\nu}}$ is coisometric and by proposition \ref{CoisomMultOpCharacBtwnTwoContMeasSp} it is isometric, invertible and $(M_g^{\Omega_c^{\mu,\nu}})^{-1}=M_{1/g}^{\Omega_c^{\mu,\nu}}$. Then for all $h\in L_q(\Omega,\nu)$ we have
$$
\Vert M_{\chi_{\Omega_c^{\mu,\nu}}/g}(h)\Vert_{L_p(\Omega,\mu)}=
\Vert M_{1/g}(h)\chi_{\Omega_c^{\mu,\nu}}\Vert_{L_p(\Omega,\mu)}=
\Vert M_{1/g}(h|_{\Omega_c^{\mu,\nu}})\Vert_{L_p(\Omega_c^{\mu,\nu},\mu|_{\Omega_c^{\mu,\nu}})}$$
$$
=
\Vert M_{1/g}^{\Omega_c^{\mu,\nu}}(h|_{\Omega_c^{\mu,\nu}})\Vert_{L_p(\Omega_c^{\mu,\nu},\mu|_{\Omega_c^{\mu,\nu}})}
=\Vert h|_{\Omega_c^{\mu,\nu}}\Vert_{L_q(\Omega_c^{\mu,\nu},\nu|_{\Omega_c^{\mu,\nu}})}
\leq\Vert h\Vert_{L_q(\Omega,\nu)}
$$ 
So $M_{\chi_{\Omega_c^{\mu,\nu}}/g}$ is contractive. Now note that for all $h\in L_q(\Omega,\nu)$ we have 
$$
M_g(M_{\chi_{\Omega_c^{\mu,\nu}}/g}(h))
=M_g(\chi_{\Omega_c^{\mu,\nu}}/g  h)
=g (\chi_{\Omega_c^{\mu,\nu}}/g)   h
=h \chi_{\Omega_c^{\mu,\nu}}
$$
Since $\nu(\Omega\setminus\Omega_c^{\mu,\nu})=0$, then $\chi_{\Omega_c^{\mu,\nu}}=\chi_{\Omega}$, so $M_g(M_{\chi_{\Omega_c^{\mu,\nu}}/g}(h))=h \chi_{\Omega_c^{\mu,\nu}}=h \chi_{\Omega}=h$. This means that $M_{\chi_{\Omega_c^{\mu,\nu}}/g}$ is a right inverse multiplication operator to $M_g$. Take any $h\in L_q(\Omega,\nu)$, then
$$
\Vert M_{\chi_{\Omega_c^{\mu,\nu}}/g}(h)\Vert_{L_p(\Omega,\mu)}
\geq\Vert M_g\Vert\Vert M_g(M_{\chi_{\Omega_c^{\mu,\nu}}/g}(h))\Vert_{L_q(\Omega,\nu)}
\geq\Vert h\Vert_{L_q(\Omega,\nu)}
$$
Since $h$ is arbitrary $M_{\chi_{\Omega_c^{\mu,\nu}}/g}$ is $1$-topologically injective, but it is contractive. Thus $M_{\chi_{\Omega_c^{\mu,\nu}}/g}$ is isometric.

$ii)$$\implies$$ i)$ Take arbitrary $h\in L_q(\Omega,\nu)$ and consider $f=M_{\chi_{\Omega_c^{\mu,\nu}}/g}(h)$. Then $M_g(f)=M_g(M_{\chi_{\Omega_c^{\mu,\nu}}/g}(h))=h$ and $\Vert f\Vert_{L_p(\Omega,\mu)}\leq\Vert h\Vert_{L_q(\Omega,\nu)}$. Since $h$ is arbitrary $M_g$ is strictly $1$-topologically surjective. Let $f\in L_p(\Omega,\mu)$. By assumption $M_{\chi_{\Omega_c^{\mu,\nu}}/g}$ is isometric, so
$$
\Vert M_g(f)\Vert_{L_q(\Omega,\nu)}
=\Vert M_{\chi_{\Omega_c^{\mu,\nu}}/g}(M_g(f))\Vert_{L_p(\Omega,\mu)}
=\Vert f\chi_{\Omega_c^{\mu,\nu}}\Vert_{L_p(\Omega,\mu)}
\leq\Vert f\Vert_{L_p(\Omega,\mu)}
$$
Since $f$ is arbitrary $M_g$ is contractive, but it is also strictly $1$-topologically surjective, hence strictly coisometric.
\end{proof}

Note that this proof shows that every coisometric multiplication operator is strictly coisometric.



%----------------------------------------------------------------------------------------
%	Homological triviality of the category B(Omega,Sigma)-modules L_p
%----------------------------------------------------------------------------------------

\subsection{Homological triviality of the category \texorpdfstring{$B(\Omega,\Sigma)$-modules $L_p$}{B(Omega)-modules Lp}}
\label{SubSectionHomologicalTrivialityOfTheCategoryBOmegaSigmaModulesLp}

We are ready to prove simultaneously funny and disappointing result.

\begin{proposition}\label{HomTrivlOfLpCat} Let $(\Omega,\Sigma)$ be a measurable space and $\mu$ be a $\sigma$-finite measure on $\Omega$. Then
$L_p(\Omega,\mu)$ is $\langle$~metrically / topologically~$\rangle$ projective, injective and flat in $\langle$~$B(\Omega,\Sigma)-\mathbf{mod(L)}_1$ / $B(\Omega,\Sigma)-\mathbf{mod(L)}$~$\rangle$.
\end{proposition}
\begin{proof} Denote $X:=L_p(\Omega,\mu)$ and $\langle$~$\mathbf{C}:=B(\Omega,\Sigma)-\mathbf{mod(L)}_1$ / $\mathbf{C}:=B(\Omega,\Sigma)-\mathbf{mod(L)}$~$\rangle$. 

Consider covariant functor $\langle$~$F_{proj}:=\operatorname{Hom}_{\mathbf{C}}(X,-):\mathbf{C}\to\mathbf{Ban}_1$ / $F_{proj}:=\operatorname{Hom}_{\mathbf{C}}(X,-):\mathbf{C}\to\mathbf{Ban}$~$\rangle$. By proposition $\langle$~\ref{CoisomMultOpDescBtwnTwoMeasSp} / \ref{TopSurMultOpCharacBtwnTwoMeasSp}~$\rangle$ any $\langle$~coisometric / topologically surjective~$\rangle$ morphism $\xi$ in $\mathbf{C}$ is a retraction, hence $F_{proj}(\xi)$ is a retraction in $\langle$~$\mathbf{Ban}_1$ / $\mathbf{Ban}$~$\rangle$, and therefore $\langle$~strictly coisometric / surjective~$\rangle$. Since $\xi$ is arbitrary, then $X$ is $\langle$~metrically / topologically~$\rangle$ projective.

Consider contravariant functor $\langle$~$F_{inj}=\operatorname{Hom}_{\mathbf{C}}(-,X):\mathbf{C}\to\mathbf{Ban}_1$ / $F_{inj}=\operatorname{Hom}_{\mathbf{C}}(-,X):\mathbf{C}\to\mathbf{Ban}$~$\rangle$. By proposition $\langle$~\ref{IsomMultOpDescBtwnTwoMeasSp} / \ref{TopInjMultOpDescBtwnTwoMeasSp}~$\rangle$ any $\langle$~isometric / topologically injective~$\rangle$ morphism $\xi$ in $\mathbf{C}$ is a coretraction, hence $F_{inj}(\xi)$ is a retraction in $\langle$~$\mathbf{Ban}_1$ / $\mathbf{Ban}$~$\rangle$, and therefore $\langle$~strictly coisometric / surjective~$\rangle$. Since $\xi$ is arbitrary, then $X$ is $\langle$~metrically / topologically~$\rangle$ injective.

Consider covariant functor $\langle$~$F_{flat}=-\projmodtens{B(\Omega,\Sigma)}X:\mathbf{C}\to\mathbf{Ban}_1$ / $F_{flat}=-\projmodtens{B(\Omega,\Sigma)}X:\mathbf{C}\to\mathbf{Ban}$~$\rangle$. Again, by proposition $\langle$~\ref{IsomMultOpDescBtwnTwoMeasSp} / \ref{TopInjMultOpDescBtwnTwoMeasSp}~$\rangle$ any $\langle$~isometric / topologically injective~$\rangle$ morphism $\xi$ in $\mathbf{C}$ is a coretraction, hence  $F_{flat}(\xi)$ is a coretraction in $\langle$~$\mathbf{Ban}_1$ / $\mathbf{Ban}$~$\rangle$, in particular it is $\langle$~isometric / topologically injective~$\rangle$. Since $\xi$ is arbitrary, then $X$ is $\langle$~metrically / topologically~$\rangle$ flat.
\end{proof}